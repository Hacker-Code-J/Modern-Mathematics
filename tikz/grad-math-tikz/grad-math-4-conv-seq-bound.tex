\begin{tikzpicture}[scale=.6]
	% Define the coordinates for L and the bounding lines
	\def\L{3} % Set L at 3 for visualization
	\def\epsilon{1} % Epsilon value
	\def\N{12} % Example N value
	\def\xMax{25}
	\def\yMax{6} % Max value for y-axis
	
	% Draw axes
	\draw[thick,-Stealth] (0,0) -- (\xMax+1,0) node[anchor=west] {$n$};
	\draw[thick,-Stealth] (0,0) -- (0,\yMax) node[anchor=south] {$a_n$};
	
	% Draw L and epsilon lines
	\draw[dashed, magenta, opacity=.3] (0,\L) -- (\xMax+1,\L) node[right] {$L$};
	\draw[dashed, opacity=.3] (0,\L+\epsilon) -- (\xMax+1,\L+\epsilon) node[right] {$L+1$};
	\draw[dashed, opacity=.3] (0,\L-\epsilon) -- (\xMax+1,\L-\epsilon) node[right] {$L-1$};
	
	% Mark the N line
	\draw[dashed, blue] (\N,0) -- (\N,\yMax) node[at start, below] {$N$};
	
	% Points of a_n for n < N and n >= N
	\foreach \x in {1,2,3,5,6,...,11} {
		\pgfmathsetmacro{\y}{rand*1.25+\L}
		\filldraw[red] (\x,\y) circle (3pt);
	}
	\filldraw[red] (4,4.5) circle (3pt);
	\foreach \x in {\N,...,\xMax} {
		\pgfmathsetmacro{\y}{rand*0.25+\L}
		\filldraw[red] (\x,\y) circle (3pt);
	}
	
	% Draw M line
	\def\M{4.5} % example M value
	\draw[dashed, green!50!black] (0,\M) -- (\xMax+1,\M) node[right] {$M$};
	
	% Labeling M
	\draw[<-, green!50!black] (20.5,4.75) -- (20.5,5.25) node[above] {Maximum of $\{\lvert a_1 \rvert, \dots, \lvert a_{N-1} \rvert, \abs{L}+1\}$};
\end{tikzpicture}