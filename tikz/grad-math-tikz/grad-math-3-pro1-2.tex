\begin{tikzpicture}[>=Stealth, scale=1.5]
	% Define colors for clarity
	\definecolor{infcolor}{RGB}{244, 67, 54}       % Red for infimum
	\definecolor{epsiloncolor}{RGB}{76, 175, 80}   % Green for epsilon interval
	\definecolor{pointcolor}{RGB}{33, 150, 243}    % Blue for point in T
	
	% Draw the number line
	\draw[->, thick] (0, 0) -- (8, 0) node[right] {$\R$};
	
	% Mark the infimum point gamma
	\node[fill=infcolor, circle, inner sep=1.5pt, label=below:{\(\gamma\)}] (inf) at (2, 0) {};
	
	% Draw the interval (gamma, gamma + epsilon) with arrow to mark epsilon distance
	\node[fill=epsiloncolor!50, circle, inner sep=1.5pt, label=below:{\(\gamma + \varepsilon\)}] (epsilon) at (5, 0) {};
	\draw[decorate, decoration={brace, amplitude=10pt}, thick, color=epsiloncolor] (2, 0.1) -- (5, 0.1)
	node[midway, above=10pt, color=epsiloncolor!80!black] {\( \varepsilon \)};
		
	% Draw a point within the interval and label as x_epsilon
	\node[fill=pointcolor, circle, inner sep=1.5pt] (point) at (4, 0) {};
	
	% Arrow pointing to the existence of a point x_epsilon
	\draw[->, color=pointcolor, thick] (point) -- ++(0, -0.5) node[anchor=north] {\( \exists x_\varepsilon\in T \)};
	
	\draw[<-, color=infcolor, thick] (inf) -- ++(0, 0.5) node[anchor=south] {\(\inf T\)};
	\draw[<-, color=epsiloncolor, thick] (epsilon) -- ++(0, 0.5) node[anchor=south] {\(\mu\)};
\end{tikzpicture}