% Group Homomorphism
\section{Exponentiation Function}

Consider the following groups:
\begin{itemize}
	\item The \textbf{additive group on integers} \((\mathbb{Z}, +)\):
	\begin{itemize}
		\item Set: \(\mathbb{Z}\)
		\item Operation: Addition (+)
		\item Identity Element: 0
		\item Inverses: For each \(a \in \mathbb{Z}\), the inverse is \(-a\).
	\end{itemize}
	
	\item The \textbf{multiplicative group on nonzero rational numbers} \((\mathbb{Q}^*, \cdot)\):
	\begin{itemize}
		\item Set: \(\mathbb{Q}^*\)
		\item Operation: Multiplication (\(\cdot\))
		\item Identity Element: 1
		\item Inverses: For each \(q \in \mathbb{Q}^*\), the inverse is \(q^{-1} = \frac{1}{q}\).
	\end{itemize}
\end{itemize}

We define the exponential function \( \exp : \mathbb{Z} \to \mathbb{Q}^* \) by:
\[
\exp(n) = 2^n \quad \text{for all} \ n \in \mathbb{Z}.
\]

\subsection*{Verification}

\paragraph{Homomorphism Property:}
\[
\exp(a + b) = 2^{a + b} = 2^a \cdot 2^b = \exp(a) \cdot \exp(b).
\]

\paragraph{Identity Element:}
\begin{itemize}
	\item In \((\mathbb{Z}, +)\), the identity element is 0.
	\item In \((\mathbb{Q}^*, \cdot)\), the identity element is 1.
\end{itemize}
\[
\exp(0) = 2^0 = 1.
\]

\paragraph{Inverses:}
\begin{itemize}
	\item For each \( n \in \mathbb{Z} \), the inverse of \( n \) in \(\mathbb{Z}\) is \(-n\).
	\item The inverse of \( \exp(n) = 2^n \) in \(\mathbb{Q}^*\) should be \( \exp(-n) = 2^{-n} \).
\end{itemize}
\[
\exp(-n) = 2^{-n} = \frac{1}{2^n} = (\exp(n))^{-1}.
\]

Thus, the exponential function \( \exp(n) = 2^n \) preserves the group structure between the additive group on integers \((\mathbb{Z}, +)\) and the multiplicative group on nonzero rational numbers \((\mathbb{Q}^*, \cdot)\).