% !TeX program = pdflatex
\documentclass[11pt,a4paper]{article}
\usepackage[utf8]{inputenc}
\usepackage[T1]{fontenc}
\usepackage{amsmath,amssymb,amsthm}
\usepackage{geometry}
\geometry{left=30mm,right=30mm,top=30mm,bottom=30mm}
\usepackage{hyperref}
\usepackage{booktabs}
\usepackage{xcolor}
\usepackage{adjustbox}

% Theorem environments
\theoremstyle{definition}
\newtheorem{definition}{Definition}[section]
\newtheorem{example}{Example}[section]
\theoremstyle{remark}
\newtheorem*{remark}{Remark}

\title{Big-Number Arithmetic: HW1}
\author{Ji, Yonghyeon}
\date{\today}
\renewcommand{\emph}[1]{\textbf{#1}}
\begin{document}
\maketitle
\tableofcontents
%	\begin{abstract}
%		This article presents a fully formal symbolic framework for hexadecimal
%		addition and subtraction.  We introduce precise definitions, detail the
%		carry and borrow mechanisms via modular arithmetic, and provide
%		penetrating worked examples.  A set of rigorous exercises concludes
%		the treatment, suitable for graduate-level study and publication.
%	\end{abstract}
%	\section{Introduction}
%	In computational number theory and computer science, arithmetic in bases
%	other than ten is ubiquitous.  In particular, base--16, or \emph{hexadecimal},
%	offers a compact representation of binary data.  This article develops
%	a rigorous, symbolic foundation for hexadecimal addition and
%	subtraction, suitable for inclusion in a graduate-level mathematics text.
\newpage
\section{Preliminaries}
\begin{definition}[Hexadecimal Expansion]
	A \emph{hexadecimal integer} of length $n$ is any element
	\[
	A \;=\;\sum_{i=0}^{n-1} a_i\,16^i,\quad a_i\in\{\texttt{0},\texttt{1},\texttt{2},\dots,\texttt{9},\texttt{A},\dots,\texttt{F}\}.
	\]
	We denote $A$ in radix--16 by $(a_{n-1}\dots a_1 a_0)_{16}$.
\end{definition}

\section{Addition}
\subsection{Algorithmic Formulation}
\begin{definition}[Hexadecimal Addition]
	Let \[
	A=\sum_{i=0}^{n-1} a_i\,16^i,\quad B=\sum_{i=0}^{n-1} b_i\,16^i,
	\] with $a_i,b_i\in\{\texttt{0},\dots,\texttt{F}\}$.  Define the recurrence: \[
	r_i= a_i + b_i + c_{i-1},\quad c_{-1}=0,
	\] and set \[
	s_i = r_i\bmod 16,\qquad c_i = \left\lfloor\frac{r_i}{16}\right\rfloor,
	\] for $i=0,1,\dots,n-1$.  Then \[
	A+B = \sum_{i=0}^{n-1} s_i\,16^i + c_n\,16^{n},
	\]
	expressed as the $(n+1)$--digit hexadecimal
	$(c_n\, s_{n-1}\dots s_1 s_0)_{16}$.
\end{definition}
\vfill
\subsection{Examples}
\begin{example}[Addition of $\texttt{F4C3}_{16}$ and $\texttt{2B7D}_{16}$]
	Let \[
	A=\texttt{F4C3}_{16},\quad B=\texttt{2B7D}_{16},\quad n=3.
	\] Compute: 
	\begin{center}
	\begin{minipage}{.45\textwidth} 
	\adjustbox{scale=1, center}{
		\begin{tabular}{|l|l|l|c|c|c|}
			\multicolumn{1}{l}{\ } & \multicolumn{1}{c}{\texttt{1}} & \multicolumn{1}{c}{\texttt{1}} & \multicolumn{1}{c}{\texttt{1}} & \multicolumn{1}{c}{\texttt{1}} & \multicolumn{1}{c}{\color{red}0} \\ \cline{3-6}
			\multicolumn{2}{l}{\ } & \multicolumn{1}{|c|}{\texttt{F}} & \texttt{4} & \texttt{C} & \texttt{3} \\ \cline{3-6}
			\multicolumn{2}{l}{$+$} & \multicolumn{1}{|c|}{\texttt{2}} & \texttt{B} & \texttt{7} & \texttt{D} \\
			\specialrule{1pt}{2pt}{1pt}
			\hline
			\multicolumn{1}{l}{\ } & \multicolumn{1}{|c|}{\texttt{1}} & \multicolumn{1}{|c|}{\texttt{2}} & \texttt{0} & \texttt{4} & \texttt{0} \\ \cline{2-6}
			%		\hspace{.5cm} & \hspace{.5cm} & \hspace{.5cm} & \hspace{.5cm} & \multicolumn{2}{c|}{$a_0b_0$} \\ \hline
			%		& & & \multicolumn{2}{c|}{$a_1b_0$} & \\ \hline
			%		& & \multicolumn{2}{c|}{$a_2b_0$} & \hspace{.5cm} & \hspace{.5cm} \\ \hline\hline
			%		& & & \multicolumn{2}{c|}{$a_0b_1$} & \\ \hline
			%		& & \multicolumn{2}{c|}{$a_1b_1$} & & \\ \hline
			%		& \multicolumn{2}{c|}{$a_2b_1$} & & & \\ \hline\hline
			%		& & \multicolumn{2}{c|}{$a_0b_2$} & & \\ \hline
			%		& \multicolumn{2}{c|}{$a_1b_2$} & & & \\ \hline
			%		\multicolumn{2}{|c|}{$a_2b_2$} & & & & \\ \hline
	\end{tabular}}	
	\end{minipage}
	\begin{minipage}{.45\textwidth} \[
	\begin{array}{c|cccc}
		i & 0 & 1 & 2 & 3 \\ \hline
		a_i & 3 & \mathrm{C} & 4 & \mathrm{F} \\
		b_i & \mathrm{D} & 7 & \mathrm{B} & 2 \\
		c_{i-1} & 0 & 1 & 1 & 1 \\
		r_i=a_i+b_i+c_{i-1} & 16 & 20 & 16 & 18 \\
		s_i=r_i\bmod16 & 0 & 4 & 0 & 2 \\
		c_i=\lfloor r_i/16\rfloor & 1 & 1 & 1 & 1
	\end{array}\]
	\end{minipage}
	\end{center}
	Since $c_3=1$, one obtains \[
	\texttt{F4C3}_{16}+\texttt{2B7D}_{16} = \texttt{12040}_{16}.
	\]
\end{example}

\newpage
\begin{example}[Addition of $\texttt{C0FFEE}_{16}$ and $\texttt{1BADF00D}_{16}$]
	Pad to $n=7$ by writing \[
	A = \texttt{00\,C0\,FF\,EE}_{16},\quad B = \texttt{1B\,AD\,F0\,0D}_{16}.
	\]
	Compute for $i=0,\dots,7$: 
	\begin{center}
	\begin{minipage}{.475\textwidth}\centering
	\adjustbox{scale=1, center}{
		\begin{tabular}{|l|l|l|l|l|l|c|c|c|}
			\multicolumn{1}{l}{\ } & \multicolumn{1}{c}{\texttt{0}} & \multicolumn{1}{c}{\texttt{1}} & \multicolumn{1}{c}{\texttt{1}} & \multicolumn{1}{c}{\texttt{1}} & \multicolumn{1}{c}{\texttt{0}} & \multicolumn{1}{c}{\texttt{0}} & \multicolumn{1}{c}{\texttt{1}} & \multicolumn{1}{c}{\color{red}0} \\ \cline{4-9}
			\multicolumn{3}{l}{\ } & \multicolumn{1}{|l}{\texttt{C}} & \multicolumn{1}{|c|}{\texttt{0}} & \multicolumn{1}{|c|}{\texttt{F}} & \texttt{F} & \texttt{E} & \texttt{E} \\ \cline{2-9}
			\multicolumn{1}{l}{$+$} & \multicolumn{1}{|c|}{\texttt{1}} & \multicolumn{1}{|c|}{\texttt{B}} & \texttt{A} & \texttt{D} & \texttt{F} & \texttt{0} & \texttt{0} & \texttt{D} \\
			\specialrule{1pt}{2pt}{1pt}
			\hline
			\multicolumn{1}{l}{\ } & \multicolumn{1}{|c|}{\texttt{1}} & \multicolumn{1}{|c|}{\texttt{C}} & \texttt{6} & \texttt{E} & \texttt{E} & \texttt{F} & \texttt{F} & \texttt{B} \\ \cline{2-9}
			%		\hspace{.5cm} & \hspace{.5cm} & \hspace{.5cm} & \hspace{.5cm} & \multicolumn{2}{c|}{$a_0b_0$} \\ \hline
			%		& & & \multicolumn{2}{c|}{$a_1b_0$} & \\ \hline
			%		& & \multicolumn{2}{c|}{$a_2b_0$} & \hspace{.5cm} & \hspace{.5cm} \\ \hline\hline
			%		& & & \multicolumn{2}{c|}{$a_0b_1$} & \\ \hline
			%		& & \multicolumn{2}{c|}{$a_1b_1$} & & \\ \hline
			%		& \multicolumn{2}{c|}{$a_2b_1$} & & & \\ \hline\hline
			%		& & \multicolumn{2}{c|}{$a_0b_2$} & & \\ \hline
			%		& \multicolumn{2}{c|}{$a_1b_2$} & & & \\ \hline
			%		\multicolumn{2}{|c|}{$a_2b_2$} & & & & \\ \hline
	\end{tabular}}
	\end{minipage}\hfill
	\begin{minipage}{.475\textwidth} \[
	\begin{array}{c|cccccccc}
		i & 0 & 1 & 2 & 3 & 4 & 5 & 6 & 7 \\ \hline
		a_i & \mathrm{E} & \mathrm{E} & \mathrm{F} & \mathrm{F} & 0 & \mathrm{C} & 0 & 0 \\
		b_i & \mathrm{D} & 0 & 0 & \mathrm{F} & \mathrm{D} & \mathrm{A} & \mathrm{B} & 1 \\
		c_{i-1} & 0 & 1 & 0 & 0 & 1 & 0 & 1 & 0 \\
		r_i & 27 & 15 & 15 & 30 & 14 & 22 & 12 & 1 \\
		s_i = r_i \bmod 16 & \mathrm{B} & \mathrm{F} & \mathrm{F} & \mathrm{E} & \mathrm{E} & 6 & \mathrm{C} & 1 \\
		c_i = \lfloor r_i/16\rfloor & 1 & 0 & 0 & 1 & 0 & 1 & 0 & 0
		\end{array}\]
	\end{minipage}
	\end{center}
	Thus \[
	\texttt{C0FFEE}_{16} + \texttt{1BADF00D}_{16} = \texttt{1C6EEFFB}_{16}.
	\]
\end{example}

\section{Subtraction}
\subsection{Algorithmic Formulation}
\begin{definition}[Hexadecimal Subtraction]
	Let \[
	A=\sum_{i=0}^{n-1} a_i\,16^i,\quad B=\sum_{i=0}^{n-1} b_i\,16^i,
	\] with $a_i,b_i\in\{\texttt{0},\dots,\texttt{F}\}$.  Define the recurrence: \[
	r_i = a_i - b_i - d_{i-1},\quad d_{-1}=0,
	\] and set \[
	t_i = r _i \bmod 16,\qquad
	d_i = 
	\begin{cases}
		1, & r_i < 0,\\
		0, & r_i \ge 0.
	\end{cases}\]
	Then \[
	A - B = \sum_{i=0}^{n-1} t_i\,16^i \;-\; d_n\,16^{\,n+1}.
	\]
\end{definition}

\subsection{Examples}
\begin{example}[Subtraction of $\texttt{A5B2}_{16}$ and $\texttt{3C7F}_{16}$]
	Let
	\[
	A = \texttt{A5B2}_{16},\quad B = \texttt{3C7F}_{16},\quad n=3.
	\]
	Compute:
\begin{center}
	\begin{minipage}{.475\textwidth}\centering
	\adjustbox{scale=1, center}{
		\begin{tabular}{|l|l|c|c|c|}
			\multicolumn{1}{l}{\ } & \multicolumn{1}{c}{1} & \multicolumn{1}{c}{\texttt{0}} & \multicolumn{1}{c}{\texttt{1}} & \multicolumn{1}{c}{\color{red}0} \\ \cline{2-5}
			\multicolumn{1}{l}{\ } & \multicolumn{1}{|c|}{\texttt{A}} & \texttt{5} & \texttt{B} & \texttt{2} \\ \cline{2-5}
			\multicolumn{1}{l}{$-$}  & \multicolumn{1}{|c|}{\texttt{3}} & \texttt{C} & \texttt{7} & \texttt{F} \\
			\specialrule{1pt}{2pt}{1pt}
			\hline
			\multicolumn{1}{l}{$+$} & \multicolumn{1}{|c|}{\texttt{6}} & \texttt{9} & \texttt{3} & \texttt{3} \\ \cline{2-5}
	\end{tabular}}
	\end{minipage}\hfill
	\begin{minipage}{.475\textwidth} 	\[
		\begin{array}{c|cccc}
			i & 0 & 1 & 2 & 3 \\ \hline
			a_i & 2 & B & 5 & A \\
			b_i & F & 7 & C & 3 \\
			d_{i-1} & 0 & 1 & 0 & 1 \\
			r_i = a_i - b_i - d_{i-1} & -13 & 3 & -7 & 6 \\
			t_i = r_i\bmod16 & 3 & 3 & 9 & 6 \\
			d_i = \mathbf{1}_{r_i<0} & 1 & 0 & 1 & 0
		\end{array}
		\]
	\end{minipage}
\end{center}
	Hence
	\[
	\texttt{A5B2}_{16} - \texttt{3C7F}_{16} = \texttt{6933}_{16}.
	\]
\end{example}
\newpage
\begin{example}[Subtraction of $\texttt{DEAD}_{16}$ and $\texttt{BEEF}_{16}$]
	Let \[
	A = \texttt{DEAD}_{16},\quad B = \texttt{BEEF}_{16},\quad n=3.
	\]
	Compute: 
\begin{center}
	\begin{minipage}{.475\textwidth}\centering
		\adjustbox{scale=1, center}{
			\begin{tabular}{|l|l|c|c|c|}
				\multicolumn{1}{l}{\ } & \multicolumn{1}{c}{1} & \multicolumn{1}{c}{\texttt{1}} & \multicolumn{1}{c}{\texttt{1}} & \multicolumn{1}{c}{\color{red}0} \\ \cline{2-5}
				\multicolumn{1}{l}{\ } & \multicolumn{1}{|c|}{\texttt{D}} & \texttt{E} & \texttt{A} & \texttt{D} \\ \cline{2-5}
				\multicolumn{1}{l}{$-$}  & \multicolumn{1}{|c|}{\texttt{B}} & \texttt{E} & \texttt{E} & \texttt{F} \\
				\specialrule{1pt}{2pt}{1pt}
				\hline
				\multicolumn{1}{l}{$+$} & \multicolumn{1}{|c|}{\texttt{1}} & \texttt{F} & \texttt{B} & \texttt{E} \\ \cline{2-5}
		\end{tabular}}
	\end{minipage}\hfill
	\begin{minipage}{.475\textwidth} 	\[
		\begin{array}{c|cccc}
			i & 0 & 1 & 2 & 3 \\ \hline
			a_i & D & A & E & D \\
			b_i & F & E & E & B \\
			d_{i-1} & 0 & 1 & 1 & 1 \\
			r_i = a_i - b_i - d_{i-1} & -2 & -5 & -1 & 1 \\
			t_i = r_i\bmod16 & E & B & F & 1 \\
			d_i = \mathbf{1}_{r_i<0} & 1 & 1 & 1 & 0
		\end{array}
		\]
	\end{minipage}
\end{center}
	Hence \[
	\texttt{DEAD}_{16} - \texttt{BEEF}_{16} = \texttt{1FBE}_{16}.
	\]
\end{example}

%\begin{example}[Subtraction of $\texttt{1BADF00D}_{16}$ and $\texttt{C0FFEE}_{16}$]
%	Let
%	\[
%	A = \texttt{1BADF00D}_{16},\quad B = \texttt{C0FFEE}_{16},\quad n=7.
%	\]
%	Compute for $i=0,\dots,7$:
%\begin{center}
%	\begin{minipage}{.475\textwidth}\centering
%		\adjustbox{scale=1, center}{
%			\begin{tabular}{|l|l|c|c|c|}
%				\multicolumn{1}{l}{\ } & \multicolumn{1}{c}{1} & \multicolumn{1}{c}{\texttt{1}} & \multicolumn{1}{c}{\texttt{1}} & \multicolumn{1}{c}{\color{red}0} \\ \cline{2-5}
%				\multicolumn{1}{l}{\ } & \multicolumn{1}{|c|}{\texttt{D}} & \texttt{E} & \texttt{A} & \texttt{D} \\ \cline{2-5}
%				\multicolumn{1}{l}{$-$}  & \multicolumn{1}{|c|}{\texttt{B}} & \texttt{E} & \texttt{E} & \texttt{F} \\
%				\specialrule{1pt}{2pt}{1pt}
%				\hline
%				\multicolumn{1}{l}{$+$} & \multicolumn{1}{|c|}{\texttt{1}} & \texttt{F} & \texttt{B} & \texttt{E} \\ \cline{2-5}
%		\end{tabular}}
%	\end{minipage}\hfill
%	\begin{minipage}{.475\textwidth} 	\[
%		\begin{array}{c|cccccccc}
%			i & 0 & 1 & 2 & 3 & 4 & 5 & 6 & 7 \\ \hline
%			a_i & D & 0 & 0 & F & F & D & A & 1 \\
%			b_i & E & E & F & F & 0 & C & 0 & 0 \\
%			d_{i-1} & 0 & 1 & 1 & 1 & 1 & 1 & 1 & 1 \\
%			r_i & -1 & -2 & -2 & -1 & 15 & 1 & 10 & 0 \\
%			t_i = r_i\bmod16 & F & E & E & F & F & 1 & A & 0 \\
%			d_i & 1 & 1 & 1 & 1 & 0 & 0 & 0 & 0
%		\end{array}
%		\]
%	\end{minipage}
%\end{center}
%	Thus
%	\[
%	\texttt{1BADF00D}_{16} - \texttt{C0FFEE}_{16} = \texttt{1AECF01F}_{16}.
%	\]
%\end{example}

\section{Exercises}
%	Compute each of the following in radix--16, presenting full detail of
%	carry or borrow steps:
\begin{enumerate}
	\item $\texttt{A5B2}_{16} + \texttt{C3F9}_{16}$
	\item $\texttt{7D3E}_{16} + \texttt{1A4C}_{16}$
	\item $\texttt{F4C3}_{16} + \texttt{2B7D}_{16}$
	\item $\texttt{9AFE}_{16} + \texttt{65B1}_{16}$
	\item $\texttt{BEEF}_{16} + \texttt{DEAD}_{16}$
	\item $\texttt{C0FFEE}_{16} + \texttt{1BADF00D}_{16}$
	\item $\texttt{F4C3}_{16} - \texttt{2A9D}_{16}$
	\item $\texttt{A5B2}_{16} - \texttt{3C7F}_{16}$
	\item $\texttt{DEAD}_{16} - \texttt{BEEF}_{16}$
	\item $\texttt{1BADF00D}_{16} - \texttt{C0FFEE}_{16}$
	\item $\texttt{7D3E}_{16} - \texttt{1A4C}_{16}$
	\item $\texttt{9AFE}_{16} - \texttt{65B1}_{16}$
\end{enumerate}

\end{document}
