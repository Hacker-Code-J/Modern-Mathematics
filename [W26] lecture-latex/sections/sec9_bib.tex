\subsection{Bibliography and Citations}

Citations are a core part of academic and technical writing.

\begin{itemize}
	\item give credit to original authors (avoids plagiarism),
	\item support claims with reliable evidence,
	\item help readers locate the sources for verification or further study.
\end{itemize}
A good citation system should be:
\begin{itemize}
	\item consistent (same style everywhere),
	\item complete (enough details to find the source),
	\item easy to maintain (especially in large documents).
\end{itemize}

\subsubsection{Manual Bibliography with \texttt{thebibliography}}
This is the simplest approach and works well for:
\begin{itemize}
	\item short assignments,
	\item one-file handouts,
	\item documents with only a few references.
\end{itemize}

The reference slides show the standard pattern: \verb|\begin{thebibliography}{...}| and one \verb|\bibitem{key}| per entry.
	
\begin{example}
\;\medskip

\begin{lstlisting}
\begin{thebibliography}{9}
	
	\bibitem{knuth}
	D. E. Knuth,
	\emph{The \TeX book},
	Addison-Wesley, 1984.
	
	\bibitem{lamport}
	L. Lamport,
	\emph{\LaTeX: A Document Preparation System},
	2nd ed., Addison-Wesley, 1994.
	
\end{thebibliography}
\end{lstlisting}
In \verb|\begin{thebibliography}{9}|, the \texttt{9} helps LaTeX reserve space for label widths.
For small bibliographies (under 10 items), \texttt{9} is common. For up to 99 items, use \texttt{99}.
\end{example}

\begin{example}[Citing a Manual Entry]
To cite a reference inside our text, use \verb|\cite{key}|:
\begin{lstlisting}
As shown in \cite{knuth}, the TeX system is powerful.
\end{lstlisting}
\end{example}

\newpage
	
\subsubsection{Using BibTeX (Recommended for Larger Projects)}
For large projects, BibTeX separates bibliography data into a \texttt{.bib} file. Benefits:

\begin{itemize}
	\item clean separation between writing and reference data,
	\item easy reuse of a bibliography database across multiple documents,
	\item automatic formatting via bibliography styles.
\end{itemize}
		
\begin{example}
Write BibTeX entries in a file such as \texttt{references.bib}.
(Shown here as plain text you place in the \texttt{.bib} file.)

\begin{lstlisting}
@book{knuth,
	author    = {Donald E. Knuth},
	title     = {The TeXbook},
	year      = {1984},
	publisher = {Addison-Wesley}
}

@article{einstein,
	author  = {Albert Einstein},
	title   = {Does the Inertia of a Body Depend Upon Its Energy Content?},
	journal = {Annalen der Physik},
	year    = {1905}
}
\end{lstlisting}
Near the end of your document body (where you want the bibliography to appear):
\begin{lstlisting}
\bibliographystyle{plain}
\bibliography{references}
\end{lstlisting}
Cite inside the text as usual:
\begin{lstlisting}
Einstein explained this in \cite{einstein}.
\end{lstlisting}
\end{example}

\begin{remark}
A typical compile sequence is:
\begin{itemize}
	\item run \texttt{pdflatex} (creates citation requests),
	\item run \texttt{bibtex} (builds bibliography from \texttt{.bib}),
	\item run \texttt{pdflatex} twice (resolves references and formatting).
\end{itemize}
\end{remark}
		
%		If citations show as “??” or the bibliography is empty, it usually means the BibTeX step was skipped or the sequence was incomplete.
%		
%		\section{Modern Approach: \texttt{biblatex} + \texttt{biber}}
%		
%		For modern documents, \texttt{biblatex} is often preferred because it provides:
%		\begin{itemize}
%			\item more flexible citation commands,
%			\item better style customization,
%			\item cleaner integration of citations and bibliography formatting.
%		\end{itemize}
%		
%		\subsection{Loading \texttt{biblatex}}
%		
%		In your preamble:
%		
%		\begin{example}
%			\begin{verbatim}
%				\usepackage[backend=biber, style=alphabetic]{biblatex}
%				\addbibresource{references.bib}
%			\end{verbatim}
%		\end{example}
%		
%		Then, where you want the bibliography printed:
%		
%		\begin{example}
%			\begin{verbatim}
%				\printbibliography
%			\end{verbatim}
%		\end{example}
%		
%		\subsection{Compilation Workflow (Biber)}
%		
%		A common compile sequence is:
%		
%		\begin{itemize}
%			\item \texttt{pdflatex} $\rightarrow$ \texttt{biber} $\rightarrow$ \texttt{pdflatex} $\rightarrow$ \texttt{pdflatex}.
%		\end{itemize}
%		
%		\subsection{Cite Commands}
%		
%		With \texttt{biblatex}, you gain multiple citation commands:
%		
%		\begin{itemize}
%			\item \verb|\cite{key}| -- basic citation
%			\item \verb|\parencite{key}| -- citation in parentheses
%			\item \verb|\textcite{key}| -- citation integrated into the sentence
%		\end{itemize}
%		
%		\section{Citation Styles}
%		
%		Citation style controls how citations appear in text and how bibliography entries are formatted.
%		
%		\subsection{Common Style Choices}
%		
%		\begin{itemize}
%			\item \texttt{numeric} -- citations like [1], [2]
%			\item \texttt{alphabetic} -- citations like [Knu84]
%			\item \texttt{authoryear} -- citations like (Knuth, 1984)
%			\item \texttt{apa} -- APA-like formatting (requires appropriate style support)
%		\end{itemize}
%		
%		\subsection{Changing the Style in \texttt{biblatex}}
%		
%		\begin{example}
%			\begin{verbatim}
%				\usepackage[backend=biber, style=authoryear]{biblatex}
%			\end{verbatim}
%		\end{example}
%		
%		\section{Best Practices}
%		
%		\begin{itemize}
%			\item Use consistent citation keys (e.g., \texttt{knuth1984}, \texttt{einstein1905}).
%			\item Keep your \texttt{.bib} file clean and reusable across projects.
%			\item Prefer \texttt{biblatex} for large documents unless your institution requires BibTeX.
%			\item Compile with the correct sequence (BibTeX or Biber) so references resolve properly.
%		\end{itemize}
%		
%		\section{Troubleshooting}
%		
%		\subsection{Citations show as “??”}
%		
%		Common causes:
%		\begin{itemize}
%			\item you did not run BibTeX/Biber,
%			\item you did not compile enough times after BibTeX/Biber,
%			\item the citation key in \verb|\cite{...}| does not match the \texttt{.bib} entry.
%		\end{itemize}
%		
%		\subsection{Bibliography does not appear}
%		
%		Check:
%		\begin{itemize}
%			\item For BibTeX: did you include both \verb|\bibliographystyle| and \verb|\bibliography|?
%			\item For biblatex: did you include \verb|\printbibliography|?
%			\item Are you pointing to the correct \texttt{.bib} filename?
%		\end{itemize}
%		
%		\section{Exercises}
%		
%		\begin{exercise}
%			Create a document that:
%			\begin{itemize}
%				\item uses either \texttt{thebibliography} or \texttt{biblatex},
%				\item adds at least two references (one book and one article),
%				\item cites both references in the document body using \verb|\cite|,
%				\item uses a citation style other than the default.
%			\end{itemize}
%			For manual bibliographies, follow the standard \texttt{thebibliography} pattern shown in the reference slides:contentReference[oaicite:2]{index=2}.
%		\end{exercise}
%		
%		\begin{exercise}
%			Create a small \texttt{references.bib} file with three entries:
%			\begin{itemize}
%				\item one \texttt{@book},
%				\item one \texttt{@article},
%				\item one \texttt{@misc} (for a website or online resource).
%			\end{itemize}
%			Then write a short paragraph that cites all three using \verb|\parencite| and \verb|\textcite| (if using \texttt{biblatex}).
%		\end{exercise}
