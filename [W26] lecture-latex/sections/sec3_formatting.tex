\subsection{Text Formatting and Layout}

\subsubsection{Basic Text Commands}

\LaTeX{} offers two main approaches to text styling:
\begin{itemize}
	\item \textbf{Direct styling commands} such as \verb|\textbf| and \verb|\textit|.
	\item \textbf{Logical emphasis} using \verb|\emph|, which adapts to context.
\end{itemize}
The most common beginner commands are:
\begin{itemize}
	\item \verb|\textbf{bold}| $\rightarrow$ \textbf{bold}
	\item \verb|\textit{italic}| $\rightarrow$ \textit{italic}
	\item \verb|\underline{underline}| $\rightarrow$ \underline{underline}
	\item \verb|\emph{emphasized}| $\rightarrow$ \emph{emphasized}
\end{itemize}
A useful rule:
\begin{itemize}
	\item Use \verb|\emph| when you mean “emphasize this concept.”
	\item Use \verb|\textit| or \verb|\textbf| when you want a specific visual style.
\end{itemize}
\subsubsection{Paragraphs and Line Breaks}
In \LaTeX{}, a new paragraph is created by leaving a blank line in your source.
A manual line break can be forced using \verb|\\|.

\begin{example}
\;\medskip
	
\begin{lstlisting}
This is the first line.\\
This is the second line.
\end{lstlisting}
\end{example}
\
\subsubsection{Quotes and Quotations}

\begin{itemize}
	\item \texttt{quote}: short quotations (indented, typically one paragraph).
	\item \texttt{quotation}: longer quotations (indented, with paragraph formatting).
\end{itemize}

\begin{example}
\;\medskip	
	
\begin{lstlisting}
\begin{quote}
	A well-structured document is easier to write and easier to read.
\end{quote}
\end{lstlisting}
\end{example}

\subsubsection{Text Alignment}

Alignment environments are helpful for small blocks of text:
\begin{itemize}
	\item \verb|\begin{center}...\end{center}| $\rightarrow$ centered text
	\item \verb|\begin{flushleft}...\end{flushleft}| $\rightarrow$ left-aligned text
	\item \verb|\begin{flushright}...\end{flushright}| $\rightarrow$ right-aligned text
\end{itemize}

\subsubsection{Font Sizes}

\LaTeX{} provides relative font size switches that affect the enclosed group:

\begin{itemize}
	\item \verb|\tiny|, \verb|\scriptsize|, \verb|\footnotesize|
	\item \verb|\small|, \verb|\normalsize|, \verb|\large|, \verb|\Large|
	\item \verb|\LARGE|, \verb|\huge|, \verb|\Huge|
\end{itemize}

\begin{example}
\;\medskip

\noindent
{\tiny This is tiny text.}\\
{\scriptsize This is scriptsize text.}\\
{\footnotesize This is footnotesize text.}\\
{\small This is small text.}\\
{\normalsize This is normalsize text.}\\
{\large This is large text.}\\
{\Large This is Large text.}\\
{\LARGE This is Large text.}\\
{\huge This is huge text.}\\
{\Huge This is Huge text.}
\end{example}

\subsubsection{Special Characters}
Some characters are reserved because \LaTeX{} uses them for commands or syntax.
\begin{center}
	\begin{tabular}{ll}
		\verb|\%| & \% (percent) \\
		\verb|\$| & \$ (dollar) \\
		\verb|\#| & \# (hash) \\
		\verb|\_| & \_ (underscore) \\
		\verb|\&| & \& (ampersand) \\
		\verb|\{| \verb|\}| & \{ \} (braces) \\
	\end{tabular}
\end{center}
Use \verb|%| to write comments that will not appear in the output PDF:
\begin{lstlisting}
% This is a comment and will not appear in the output
\end{lstlisting}
