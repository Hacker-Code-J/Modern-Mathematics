\subsection{Theorems, Definitions, and Proofs}

\subsubsection{Package: \texttt{amsthm}}

Mathematical writing often needs structured statements such as definitions, theorems, lemmas, corollaries, and remarks.
%\begin{table}[t]
%	\centering
%%	\begin{threeparttable}
%%		\caption{Catalogue of formal statements used in this chapter}
%		\label{tab:formal-statements}
%		\begin{tabularx}{\textwidth}{@{}>{\raggedright\bfseries}p{0.175\textwidth}| X@{}}
%			\toprule
%			Statement type & Typical role \\ \midrule
%			Definition     & Introduces terminology and fixes conventions; may specify objects and notation. \\
%			Theorem        & Principal result; usually of independent interest and proved in full detail. \\
%			Lemma          & Auxiliary result serving the proof of a theorem; often technical. \\
%			Corollary      & Immediate consequence of a theorem, typically obtained with minimal additional work. \\
%			Proposition    & Intermediate result: stronger than a remark, often weaker or more specialized than a theorem. \\
%			Claim          & Local assertion used within a proof; commonly proved on the spot. \\
%			Remark         & Explanatory comment, intuition, caveat, or discussion of scope and sharpness. \\
%			Example        & Concrete instance illustrating definitions or results; may show necessity of hypotheses. \\
%			Notation       & Declares symbols and conventions to be used subsequently. \\
%			Assumption     & Standing hypotheses imposed throughout a section/chapter. \\
%			Observation    & Minor fact or perspective that clarifies structure without warranting theorem status. \\
%			\bottomrule
%		\end{tabularx}
%%		\begin{tablenotes}[para,flushleft]
%%			\footnotesize
%%			\emph{Note.} The hierarchy “theorem–proposition–lemma–corollary” is stylistic and varies by author; the guiding principle is readability.
%%		\end{tablenotes}
%%	\end{threeparttable}
%\end{table}
\begin{table}[h]
	\setlength{\tabcolsep}{4pt}        % default ~6pt
	\renewcommand{\arraystretch}{1.25} % default 1.0; keep only slightly >1
	\centering
	\begin{threeparttable}
%		\caption{Catalogue of formal statements}
		\label{tab:formal-statements}
		\small
		\begin{tabularx}{\textwidth}{@{}>{\raggedright\bfseries}p{0.22\textwidth}| X@{}}
			\toprule[1.2pt]
			Statement type & Typical role \\ \midrule
			Definition     & Introduces terminology; fixes conventions and notation. \\ \hline
			Theorem        & Principal result; proved in full detail. \\ \hline
			Lemma          & Auxiliary/technical result used in proofs. \\ \hline
			Corollary      & Immediate consequence of a theorem. \\ \hline
			Proposition    & Intermediate result; often specialized or less central than a theorem. \\  \hline
			Claim          & Local assertion inside a proof; proved on the spot. \\  \hline
			Remark         & Intuition, caveat, discussion of scope or sharpness. \\  \hline
			Example        & Illustrative instance; may justify hypotheses. \\  \hline
			Notation       & Declares symbols and standing conventions. \\  \hline
			Assumption     & Standing hypotheses for a section/chapter. \\  \hline
			Observation    & Minor fact clarifying structure. \\
			\bottomrule[1.2pt]
		\end{tabularx}
		
		\begin{tablenotes}[para,flushleft]
			\footnotesize
			\emph{Note.} The naming hierarchy is conventional and varies by author; prioritize clarity.
		\end{tablenotes}
	\end{threeparttable}
\end{table}

\noindent
In \LaTeX{}, the most standard tool for this is the \texttt{amsthm} package, which provides:

\begin{itemize}
	\item theorem-like environments with automatic numbering,
	\item consistent formatting (bold theorem headings, italic theorem body, etc.),
	\item a built-in \texttt{proof} environment that automatically prints a QED symbol.
\end{itemize}

\begin{example} In our preamble, we define theorem-style environments like this:
\begin{lstlisting}
\usepackage{amsthm}

\newtheorem{theorem}{Theorem}[section]
\newtheorem{lemma}[theorem]{Lemma}
\newtheorem{definition}[theorem]{Definition}
\newtheorem{corollary}[theorem]{Corollary}

\theoremstyle{remark}
\newtheorem*{remark}{Remark}
\end{lstlisting}
\end{example}


\begin{itemize}
	\item \verb|\usepackage{amsthm}| loads the theorem framework.
	\item \verb|\newtheorem{theorem}{Theorem}[chapter]| creates an environment named \texttt{theorem}.
	Its printed title is “Theorem”, and it resets numbering each chapter.
	So you get Theorem 1.1, 1.2, then Theorem 2.1, etc.
	\item \verb|\newtheorem{lemma}[theorem]{Lemma}| makes \texttt{lemma} share the same counter as \texttt{theorem}.
	That means a lemma can be Lemma 2.3 right after Theorem 2.2.
%	\item \verb|\newtheorem{definition}[theorem]{Definition}| and \verb|\newtheorem{corollary}[theorem]{Corollary}|
%	do the same counter-sharing for definitions and corollaries.
	\item \verb|\theoremstyle{remark}| switches to a different style (usually non-italic body text).
	\item \verb|\newtheorem*{remark}{Remark}| creates an unnumbered environment \texttt{remark}.
	The star \verb|*| means “no numbering”.
\end{itemize}

\begin{table}[h]
	\centering
	\setlength{\tabcolsep}{4pt}
	\renewcommand{\arraystretch}{1.05}
%	\caption{Built-in \texttt{\string\theoremstyle} options in \textsf{amsthm}}
	\label{tab:amsthm-theoremstyles}
	\begin{tabular}{@{}l|ll@{}}
		\toprule[1.2pt]
		\textbf{\texttt{\string\theoremstyle\{...\}}} & \textbf{Heading} & \textbf{Body} \\ \midrule
		\texttt{plain}      & Bold (upright) & Italic \\
		\texttt{definition} & Bold (upright) & Upright (roman) \\
		\texttt{remark}     & Italic          & Upright (roman) \\
		\bottomrule[1.2pt]
	\end{tabular}
\end{table}

\begin{table}[h]
	\centering
	\setlength{\tabcolsep}{4pt}
	\renewcommand{\arraystretch}{1.25}
%	\caption{Typical mapping of theorem-like environments to \texttt{\string\theoremstyle}}
	\label{tab:amsthm-mapping}
	\begin{tabular}{@{}p{0.275\linewidth}|p{0.125\linewidth}|p{0.6\linewidth}@{}}
		\toprule[1.2pt]
		\textbf{Environment (examples)} & \textbf{Style} & \textbf{Rationale} \\ \midrule
		Theorem, Lemma, & \multirow{2}{*}{\texttt{plain}} & \multirow{2}{*}{Results/proofs read well with italic body.} \\
		Proposition, Corollary & & \\ \hline
		Definition, Assumption, & \multirow{2}{*}{\texttt{definition}} & \multirow{2}{*}{Declarative text is usually upright.} \\
		Notation & & \\ \hline
		Remark, Observation, & \multirow{2}{*}{\texttt{remark}} & \multirow{2}{*}{Commentary often uses italic heading and upright body.} \\
		Example & & \\
		\bottomrule[1.2pt]
	\end{tabular}
\end{table}

\paragraph{Numbered vs. Unnumbered Statements}

\begin{itemize}
	\item Use numbered statements for items you will reference later.
	\item Use unnumbered statements for short comments, informal notes, or one-off remarks.
\end{itemize}

\subsubsection{Basic Environments}

A good theorem section begins with clear definitions.
Definitions tell the reader what words mean before you prove anything about them.

\begin{example}[Definition Statements]
\;\medskip

\begin{lstlisting}
\begin{definition}[Even Number]
	An integer $n$ is called \emph{even} if there exists an integer $k$ such that $n = 2k$.
\end{definition}
\end{lstlisting}
\begin{definition}[Even Number]
	An integer $n$ is called \emph{even} if there exists an integer $k$ such that $n = 2k$.
\end{definition}
\end{example}

\newpage
\begin{example}[Theorem Statements]
A theorem states a mathematical fact we will justify with a proof.

\begin{lstlisting}
\begin{theorem}[Sum of Evens]
	The sum of two even integers is even.
\end{theorem}
\begin{proof}
	Let $a = 2m$ and $b = 2n$ for some integers $m$ and $n$.
	Then $a + b = 2m + 2n = 2(m + n)$, which is divisible by $2$.
	Therefore, $a + b$ is even.
\end{proof}
\end{lstlisting}
\begin{theorem}[Sum of Evens]
	The sum of two even integers is even.
\end{theorem}
\begin{proof}
	Let $a = 2m$ and $b = 2n$ for some integers $m$ and $n$.
	Then $a + b = 2m + 2n = 2(m + n)$, which is divisible by $2$.
	Therefore, $a + b$ is even.
\end{proof}
\end{example}


\subsubsection{Referencing Theorems and Definitions}

In real writing, we should label key statements and refer back to them later.
Use \verb|\label| inside the theorem-like environment, then refer using \verb|\ref|.

\begin{example}[Label and reference]
\;\medskip	
	
\begin{lstlisting}
\begin{theorem}\label{thm:sum-even}
	The sum of two even integers is even.
\end{theorem}

By Theorem~\ref{thm:sum-even}, we conclude that ...
\end{lstlisting}	
\begin{theorem}\label{thm:sum-even}
	The sum of two even integers is even.
\end{theorem}

By Theorem~\ref{thm:sum-even}, we conclude that ...
\end{example}


%\section{Exercises}
%
%\begin{exercise}
%	Define your own:
%	\begin{itemize}
%		\item A definition for a prime number (as an integer greater than $1$ with exactly two positive divisors).
%		\item A theorem stating: “If $p$ is prime and $p$ divides $ab$, then $p$ divides $a$ or $p$ divides $b$.”
%		\item A brief proof using basic divisibility ideas.
%	\end{itemize}
%\end{exercise}
%
%\begin{exercise}
%	Create a labeled theorem and refer to it later:
%	\begin{itemize}
%		\item Write a theorem of your choice and label it (e.g., \verb|\label{thm:my-result}|).
%		\item Write a second theorem whose proof references the first using \verb|Theorem~\ref{thm:my-result}|.
%	\end{itemize}
%\end{exercise}
%
%\begin{exercise}
%	Write one proof in each style:
%	\begin{itemize}
%		\item direct proof,
%		\item proof by contradiction.
%	\end{itemize}
%	Use the \texttt{proof} environment for both and keep each proof under 10 lines.
%\end{exercise}
