\subsection{Figures and Graphics}

\subsubsection{Packages: \texttt{graphicx}}

To insert images in \LaTeX{}, you typically load the \texttt{graphicx} package.
This package provides the command \verb|\includegraphics|, which is the main tool for importing external images.
\begin{lstlisting}
\usepackage{graphicx}
\end{lstlisting}
A practical beginner rule:
\begin{itemize}
	\item Put image files in the same folder as our \texttt{.tex} file, or in a dedicated folder like \texttt{figures/}.
	\item If using a folder, include it in the path (e.g., \verb|figures/plot.pdf|).
\end{itemize}

\paragraph{Common File Formats}

Most workflows accept:
\begin{itemize}
	\item \texttt{.png}, \texttt{.jpg} for raster images (screenshots, photos),
	\item \texttt{.pdf} for vector graphics (plots, diagrams exported from tools).
\end{itemize}

\subsubsection{Basic Usage: \texttt{\textbackslash includegraphics}}

The most common professional pattern is to place an image inside a \texttt{figure} environment,
center it, add a caption, add a label, and then reference it in the text.

\begin{example}
\;\medskip	
	
\begin{lstlisting}
\begin{figure}[h]
	\centering
	\includegraphics[width=.6\linewidth]{example-image-a}
	\caption{A sample figure}
	\label{fig:sample}
\end{figure}
See Figure~\ref{fig:sample}.
\end{lstlisting}
\begin{itemize}
	\item \verb|\begin{figure}...\end{figure}| creates a \emph{float} (LaTeX may move it for better layout).
	\item \verb|\centering| centers the image.
	\item \verb|\caption| creates a numbered caption.
	\item \verb|\label| creates a reference key.
	\item \verb|\ref| prints the correct figure number (and updates automatically if numbering changes).
\end{itemize}
\end{example}

\newpage
\begin{example}[Controlling Image Size]
\;
\begin{itemize}
	\item \verb|width=...| e.g., \verb|width=0.6\textwidth|
	\item \verb|height=...| e.g., \verb|height=4cm|
	\item \verb|scale=...| e.g., \verb|scale=0.8|
\end{itemize}

\begin{lstlisting}
	...
	\includegraphics[scale=0.5]{diagram}
	...
\end{lstlisting}
\end{example}

\begin{example}
For consistent layout, \verb|width=...| is usually the best choice. A common pattern is:
\begin{itemize}
	\item \verb|width=\linewidth| inside a minipage or narrow container,
	\item \verb|width=0.6\textwidth| for a medium figure in normal text.
\end{itemize}
If we specify both width and height, the image can stretch.
A safer pattern is:
\begin{lstlisting}
	...
	\includegraphics[width=0.7\textwidth, keepaspectratio]{plot}
	...
\end{lstlisting}
\end{example}

\subsubsection{Positioning Figures}

Figures are floats, so \LaTeX{} chooses a good placement by default.
We can provide placement hints with:

\begin{itemize}
	\item \texttt{h} --- here (approximately)
	\item \texttt{t} --- top of page
	\item \texttt{b} --- bottom of page
	\item \texttt{p} --- on a separate float page
\end{itemize}

\subsubsection{Referencing Figures}
	
Use \verb|\label| and \verb|\ref| to refer to figures.
The reference notes explicitly demonstrate referencing with \verb|Figure~\ref{...}|.
	
\begin{lstlisting}
See Figure~\ref{fig:sample} for a visual illustration.
\end{lstlisting}
Use prefixes:
\begin{itemize}
	\item \verb|fig:...| for figures,
	\item \verb|tab:...| for tables,
	\item \verb|eq:...| for equations.
\end{itemize}
	
\subsubsection{Side-by-Side Images}
	
Side-by-side layouts are useful for comparisons (before/after, method A vs. method B).
A simple approach is to use \texttt{minipage} blocks inside one \texttt{figure} float.
	
\begin{example}
\;\medskip

\begin{lstlisting}
\begin{figure}[h]
	\centering
	\begin{minipage}{0.48\textwidth}
		\centering
		\includegraphics[width=\linewidth]{img1}
		\caption{Image 1}
		\label{fig:img1}
	\end{minipage}
	\hfill
	\begin{minipage}{0.48\textwidth}
		\centering
		\includegraphics[width=\linewidth]{img2}
		\caption{Image 2}
		\label{fig:img2}
	\end{minipage}
\end{figure}
\end{lstlisting}
The above produces two captions in one float, which can be acceptable but may look unusual in some styles.
A more ``textbook/paper'' solution is to use subfigures (commonly via the \texttt{subcaption} package), so we get one main caption plus (a)/(b) subcaptions.

\begin{lstlisting}
\begin{figure}[h]
	\centering
	% subfigure A here
	% subfigure B here
	\caption{Comparison of two methods}
	\label{fig:comparison}
\end{figure}
\end{lstlisting}	
\end{example}	

%	\section{Troubleshooting Figures (Common Beginner Issues)}
%	
%	\subsection{File Not Found}
%	
%	If you see an error like “File `img1` not found”:
%	\begin{itemize}
%		\item check the filename spelling exactly (case-sensitive on many systems),
%		\item confirm the file is in the correct folder,
%		\item try specifying the extension explicitly (e.g., \verb|img1.png|).
%	\end{itemize}
%	
%	\subsection{Figure Appears in the Wrong Place}
%	
%	This is normal float behavior. Try:
%	\begin{itemize}
%		\item using \verb|[htbp]| instead of \verb|[h]|,
%		\item moving the figure earlier/later in the source,
%		\item adding a little more text before the figure so LaTeX has flexibility.
%	\end{itemize}
%	
%	\section{Exercise}
%	
%	\begin{exercise}
%		Create a \LaTeX{} document that:
%		\begin{itemize}
%			\item loads the \texttt{graphicx} package,
%			\item inserts an image of your choice with caption and label,
%			\item refers to the image using \verb|\ref| (e.g., \verb|Figure~\ref{fig:...}|),
%			\item uses the standard structure shown in the reference pattern:contentReference[oaicite:5]{index=5}.
%		\end{itemize}
%	\end{exercise}
%	
%	\begin{exercise}
%		Create a comparison figure:
%		\begin{itemize}
%			\item place two images side by side using \texttt{minipage},
%			\item ensure both images have the same width (use \verb|\linewidth|),
%			\item write one sentence referencing both figures by number.
%		\end{itemize}
%	\end{exercise}
