\subsection{Custom Commands and Macros}
As our documents grow, we will repeat the same notation, symbols, and formatting patterns many times.
Custom commands (macros) let you write once and reuse everywhere.

Suppose we write the real numbers many times as $\mathbb{R}$.
If we define a macro \verb|\R|, then our source becomes shorter and clearer:
\begin{itemize}
	\item without macros: \verb|$\mathbb{R}$|
	\item with macros: \verb|$\R$|
\end{itemize}

\subsubsection{Defining New Commands}

Use \verb|\newcommand| in the preamble to define your own commands.
A command name starts with a backslash and usually uses letters only (e.g., \verb|\R|, \verb|\vect|).

\paragraph{Syntax}
\;\medskip
\begin{lstlisting}
\newcommand{\name}{replacement text}
\newcommand{\name}[n]{replacement with #1 to #n}
\end{lstlisting}
\begin{itemize}
	\item \verb|\name| is the command we will type in the document.
	\item \verb|{replacement text}| is what \LaTeX{} will insert when the command is used.
	\item \verb|[n]| declares the number of required arguments.
	\item \verb|#1|, \verb|#2|, \dots{} are placeholders for those arguments.
\end{itemize}

\begin{example}
\;\medskip
\begin{lstlisting}
\newcommand{\R}{\mathbb{R}}                 % real numbers
\newcommand{\vect}[1]{\mathbf{#1}}          % bold vectors
\newcommand{\abs}[1]{\left|#1\right|}       % absolute value
\end{lstlisting}
Once defined, you can use them in math mode:
\begin{itemize}
	\item $\R$ denotes the set of real numbers.
	\item $\mathbf{v}$ denotes a vector $v$ typeset in bold.
	\item $\left|{x}\right|$ produces the absolute value of $x$.
\end{itemize}
\end{example}


\subsubsection{A Note on Math Mode}
Sometimes we want a command with a default argument.
we can define an optional argument using \verb|\newcommand| by putting a default value in brackets.

\begin{example}
\;\medskip

\begin{lstlisting}
\newcommand{\seq}[2][n]{\{#2_1, #2_2, \dots, #2_#1\}}
\end{lstlisting}
\begin{itemize}
	\item \verb|\seq{a}| produces $\{a_1, a_2, \dots, a_n\}$.
	\item \verb|\seq[3]{x}| produces $\{x_1, x_2, x_3\}$.
\end{itemize}
\end{example}

%\subsection{Best Practice: Keep Optional Arguments Simple}
%
%Optional arguments are great for small variations (like a default size, a default exponent limit, etc.).
%If your macro becomes too complex, consider writing a clearer command name rather than many options.
%
%\section{Renewing Existing Commands}
%
%\subsection{When You Might Need \texttt{\textbackslash renewcommand}}
%
%If a command already exists and you want to redefine it, use \verb|\renewcommand|.
%For example, some authors prefer $\varphi$ instead of $\phi$:
%
%\begin{verbatim}
%	\renewcommand{\phi}{\varphi}
%\end{verbatim}
%
%\subsection{Warning About Redefinitions}
%
%Redefining built-in commands can break packages or change meaning unexpectedly.
%Use \verb|\renewcommand| only when you are confident:
%\begin{itemize}
%	\item you understand the original meaning,
%	\item you will not confuse collaborators or readers of the source,
%	\item the redefinition does not conflict with packages you use.
%\end{itemize}
%
%\subsection{Safer Alternative: Make a New Name}
%
%Instead of redefining \verb|\phi| globally, you can define your own macro:
%
%\begin{verbatim}
%	\newcommand{\phivar}{\varphi}
%\end{verbatim}
%
%Then use $\phivar$ when you want the variant.
%
\subsubsection{Declaring Math Operators}
%
%\subsection{Why Operators Need Special Handling}
%
Functions like $\sin(x)$, $\log(x)$, and $\lim$ are typeset as operators with correct spacing.
If we write a function name as plain letters in math mode (like $Var(X)$), spacing and font may look wrong.

\begin{example}
To define operators with proper spacing and upright font, use \verb|\DeclareMathOperator|:
\begin{lstlisting}
	\DeclareMathOperator{\Var}{Var}
	\DeclareMathOperator*{\argmax}{arg\,max}
\end{lstlisting}
Inline usage:
\begin{itemize}
	\item $\Var(X)$
	\item $\argmax_{x \in A} f(x)$
\end{itemize}
Display usage (showing typical operator styling):
\[
\argmax_{x \in A} f(x)
\qquad
\Var(X)
\]
\end{example}
\begin{remark}
\;
\begin{itemize}
	\item \verb|\DeclareMathOperator| behaves like $\sin$: subscripts appear to the side in display math.
	\item \verb|\DeclareMathOperator*| behaves like $\lim$: subscripts can appear underneath in display math.
\end{itemize}
\end{remark}

%\subsection{Usage Examples}
%

%
%\section{Best Practices}
%
%\subsection{Naming and Organization}
%
%\begin{itemize}
%	\item Use clear, descriptive names (e.g., \verb|\vect| instead of \verb|\v|).
%	\item Prefer consistent prefixes for project macros (e.g., \verb|\mR|, \verb|\mVect|) to avoid name conflicts.
%	\item Group macro definitions by purpose (sets, operators, formatting) and comment them.
%\end{itemize}
%
%\subsection{Avoid Conflicts}
%
%\begin{itemize}
%	\item If \verb|\newcommand| gives an error (“already defined”), do not switch to \verb|\renewcommand| immediately.
%	First check whether a package already uses that name.
%	\item Prefer longer names if necessary: \verb|\RealNumbers| is safer than \verb|\R| in complex projects.
%\end{itemize}
%
%\subsection{Write Macros That Match Meaning}
%
%A macro should represent meaning, not appearance.
%For example, \verb|\vect{v}| expresses “vector $v$”, not “bold $v$”.
%If you later decide vectors should be arrows (like $\vec{v}$), you change the macro definition once.
%
%\begin{example}[Semantic macro design]
%	\begin{verbatim}
%		\newcommand{\vect}[1]{\mathbf{#1}} % now bold
%		% later you can switch to:
%		% \newcommand{\vect}[1]{\vec{#1}}  % now arrow style
%	\end{verbatim}
%\end{example}
%
%\section{Exercises}
%
%\begin{exercise}
%	In a \LaTeX{} document:
%	\begin{itemize}
%		\item Define the following custom commands:
%		\begin{itemize}
%			\item \verb|\Z| for $\mathbb{Z}$
%			\item \verb|\norm{#1}| for $\left\|#1\right\|$
%			\item \verb|\inner{#1}{#2}| for the inner product $\langle #1, #2 \rangle$
%		\end{itemize}
%		\item Use each command in a displayed equation and in one sentence with inline math.
%	\end{itemize}
%\end{exercise}
%
%\begin{exercise}
%	Define a macro for probability and expectation operators:
%	\begin{itemize}
%		\item \verb|\DeclareMathOperator{\E}{\mathbb{E}}| is not correct because \verb|\mathbb| is not plain text.
%		Instead, create a command \verb|\E| that prints $\mathbb{E}$ and then use it as an operator in expressions like $\E[X]$.
%	\end{itemize}
%	Write two example formulas using your macro, including one with a subscript like $\E_{X \sim P}[f(X)]$.
%\end{exercise}
%
%\begin{exercise}
%	Create a “notation layer” for your document:
%	\begin{itemize}
%		\item Define macros for $\mathbb{R}$, $\mathbb{N}$, and $\mathbb{Z}$.
%		\item Define \verb|\abs| and \verb|\norm|.
%		\item Rewrite a short paragraph of math so that all sets and absolute values/norms use your macros.
%	\end{itemize}
%	Explain (in one paragraph) how this improves maintainability.
%\end{exercise}
