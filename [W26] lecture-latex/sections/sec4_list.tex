\subsection{Lists and Tables}

\subsubsection{Unordered Lists (Bulleted)}

Use the \texttt{itemize} environment to create bulleted lists. Each list item begins with \verb|\item|.

\begin{example}
\;\medskip

\begin{lstlisting}
\begin{itemize}
	\item Apples
	\item Bananas
	\item Cherries
\end{itemize}
\end{lstlisting}
\begin{itemize}
	\item Apples
	\item Bananas
	\item Cherries
\end{itemize}
\end{example}

\begin{example}
We can nest \texttt{itemize} to create sub-bullets.
\;\medskip

\begin{lstlisting}
\begin{itemize}
	\item Fruit
	\begin{itemize}
		\item Apples
		\item Bananas
	\end{itemize}
	\item Vegetables
\end{itemize}
\end{lstlisting}
\begin{itemize}
	\item Fruit
	\begin{itemize}
		\item Apples
		\item Bananas
	\end{itemize}
	\item Vegetables
\end{itemize}
\end{example}
%
%\subsection{Practical Notes (Beginner-Friendly)}
%
%\begin{itemize}
%	\item Use lists when you want readability: requirements, key points, pros/cons.
%	\item Keep items grammatically consistent (all nouns, or all full sentences).
%	\item Avoid extremely long items; if needed, split into sub-bullets.
%\end{itemize}
%

\newpage
\subsubsection{Ordered Lists (Numbered)}

Use the \texttt{enumerate} environment for numbered lists. \LaTeX{} automatically handles numbering, which is exactly the beginner-friendly point emphasized in the reference notes.

\begin{example}
\;\medskip

\begin{lstlisting}
\begin{enumerate}
	\item First
	\item Second
	\item Third
\end{enumerate}
\end{lstlisting}
\begin{enumerate}
	\item First
	\item Second
	\item Third
\end{enumerate}
\end{example}

\begin{example}
We can nest \texttt{enumerate} just like \texttt{itemize}:
\begin{lstlisting}
\begin{enumerate}
	\item Main step
	\begin{enumerate}
		\item Sub-step
		\item Sub-step
	\end{enumerate}
	\item Next main step
\end{enumerate}
\end{lstlisting}
\begin{enumerate}
	\item Main step
	\begin{enumerate}
		\item Sub-step
		\item Sub-step
	\end{enumerate}
	\item Next main step
\end{enumerate}
\end{example}

\subsubsection{Description Lists}

Use the \texttt{description} environment when each item needs a label.

\begin{example}
\;\medskip

\begin{lstlisting}
\begin{description}
	\item[Dog] A friendly animal.
	\item[Cat] A curious animal.
\end{description}
\end{lstlisting}
\begin{description}
	\item[Dog] A friendly animal.
	\item[Cat] A curious animal.
\end{description}
\end{example}

\subsubsection{Tables}

Tables in \LaTeX{} are typically built in two layers:
\begin{itemize}
	\item \texttt{tabular}: the grid itself (rows/columns).
	\item \texttt{table}: an optional floating wrapper for captions and labels.
\end{itemize}

\begin{example}
\;\medskip

\begin{lstlisting}
\begin{tabular}{|l|c|r|}
	\hline
	Name & Age & Score \\
	\hline
	Alice & 24 & 95 \\
	Bob & 22 & 88 \\
	\hline
\end{tabular}
\end{lstlisting}
\begin{center}
	\begin{tabular}{|l|c|r|}
		\hline
		Name & Age & Score \\
		\hline
		Alice & 24 & 95 \\
		Bob & 22 & 88 \\
		\hline
	\end{tabular}
\end{center}
\begin{itemize}
\item \verb|&| separates columns.
\item \verb|\\| ends a row.
\item \verb|\hline| draws a horizontal line.
\end{itemize}
\end{example}

\paragraph{Alignment Options in Tables}
The column specification controls alignment:

\begin{itemize}
	\item \texttt{l} -- left aligned
	\item \texttt{c} -- centered
	\item \texttt{r} -- right aligned
	\item \texttt{|} -- vertical line between columns
\end{itemize}
Beyond \texttt{l}, \texttt{c}, \texttt{r}, we will often need fixed-width columns:
\begin{itemize}
	\item \verb|p{3cm}|: a paragraph column of width 3cm (text wraps automatically).
\end{itemize}

%\newpage
\begin{example}[Fixed-width column]
\;\medskip	
	
\begin{lstlisting}
\begin{tabular}{|l|p{6cm}|}
	\hline
	Term & Explanation \\
	\hline
	LaTeX & A document preparation system for professional typesetting. \\
	\hline
\end{tabular}
\end{lstlisting}
\begin{center}
\begin{tabular}{|l|p{6cm}|}
	\hline
	Term & Explanation \\
	\hline
	LaTeX & A document preparation system for professional typesetting. \\
	\hline
\end{tabular}
\end{center}
\end{example}

\subsubsection{Floating Tables}

A \texttt{table} environment is a \emph{float}: \LaTeX{} may move it to an optimal location for page layout.

%\paragraph{Why Float?}
%Floating tables:
%\begin{itemize}
%	\item prevent ugly page breaks,
%	\item allow captions and labels,
%	\item let \LaTeX{} place the table where it fits best.
%\end{itemize}

\begin{example}
Standard Pattern: \texttt{table} + \texttt{tabular} + caption/label
\medskip

\begin{lstlisting}
\begin{table}[h]
	\centering
	\begin{tabular}{lrr}
		\hline
		Item & Qty & Price \\
		\hline
		Apples & 3 & 1.20 \\
		Bananas & 5 & 0.80 \\
		\hline
	\end{tabular}
	\caption{Fruit Inventory}
	\label{tab:fruit}
\end{table}
\end{lstlisting}
		\begin{table}[h]
	\centering
	\begin{tabular}{lrr}
		\hline
		Item & Qty & Price \\
		\hline
		Apples & 3 & 1.20 \\
		Bananas & 5 & 0.80 \\
		\hline
	\end{tabular}
	\caption{Fruit Inventory}
	\label{tab:fruit}
\end{table}
\end{example}

\paragraph{Placement Options}

The optional argument \verb|[h]| suggests ``place here''. Common options:
\begin{table}[h!]\centering
\renewcommand{\arraystretch}{1.25}
\begin{tabular}{@{}l|l@{}}
	\toprule[1.2pt]
	\texttt{h} & here \\ \hline
	\texttt{t} & top of page \\ \hline
	\texttt{b} & bottom of page \\ \hline
	\texttt{p} & float page \\
	\bottomrule[1.2pt]
\end{tabular}
\end{table}	
\begin{itemize}
	\item \texttt{h}: here
	\item \texttt{t}: top of page
	\item \texttt{b}: bottom of page
	\item \texttt{p}: float page
\end{itemize}
A common practical choice is \verb|[htbp]| to give \LaTeX{} flexibility.

\paragraph{Captions, Labels, and Referencing}

A best practice:
\begin{itemize}
	\item Put \verb|\label| after \verb|\caption|.
	\item Refer to the table using \verb|Table~\ref{tab:fruit}|. (Table~\ref{tab:fruit})
\end{itemize}