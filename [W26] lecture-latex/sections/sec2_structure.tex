\subsection{Document Structure and Classes}

%\section{Anatomy of a LaTeX Document}

A \LaTeX{} source file is a plain text file, usually saved with the extension \texttt{.tex}.
When you compile it, \LaTeX{} produces a beautifully typeset output (most commonly a PDF).

\subsubsection{Setup and Content}

A \LaTeX{} document is built from two main parts:
\begin{itemize}
	\item \textbf{Preamble} (before \verb|\begin{document}|): chooses the document class and loads packages.
		\item \textbf{Document body} (between \verb|\begin{document}| and \verb|\end{document}|): contains the content you want printed (text, math, figures, tables, etc.).
	\end{itemize}
	This “preamble + body” split is the standard structure shown in the reference material.
	
\begin{example}[Minimal Structure]
\;
\medskip

\begin{lstlisting}
\documentclass{article}

\begin{document}
	This is a LaTeX document.
\end{document}
\end{lstlisting}
\end{example}

	
\subsubsection{Document Classes}

The first line of a \LaTeX{} file is often the most important:
\begin{center}
	\verb|\documentclass[options]{class}|
\end{center}
It sets the foundation for the entire document.
\begin{table}[h!]\centering
\renewcommand{\arraystretch}{1.25}
\begin{tabularx}{\textwidth}{@{}l|X@{}}
\toprule[1.2pt]
\texttt{article}& best for \textbf{shorter documents} such as papers, essays, homework, and reports.\\ \hline
\texttt{report}& used for \textbf{longer documents} that need chapters (e.g., theses).\\ \hline
\texttt{book}& designed for \textbf{books}; supports chapters and is set up for two-sided printing.\\ \hline
\texttt{beamer}& used for \textbf{presentations}; the output becomes slide-based rather than page-based.\\
\bottomrule[1.2pt]
\end{tabularx}
\end{table}	

\noindent
Class options customize the class behavior.
\begin{table}[h!]\centering
\renewcommand{\arraystretch}{1.25}
\begin{tabularx}{\textwidth}{@{}l|X@{}}
	\toprule[1.2pt]
	\texttt{10pt}, \texttt{11pt}, \texttt{12pt}& base font size (default is commonly \texttt{10pt}). \\ \hline
	\texttt{a4paper}, \texttt{letterpaper}& paper size \\ \hline
	\texttt{twocolumn}& two-column layout. \\ \hline
	\texttt{oneside}, \texttt{twoside}& single-sided or double-sided layout. \\
	\bottomrule[1.2pt]
\end{tabularx}
\end{table}	
\begin{example}
	\verb|\documentclass[12pt, a4paper, twoside]{article}|
\end{example}
	
\begin{remark}
A beginner-friendly recommendation:
\begin{itemize}
	\item Use \texttt{11pt} or \texttt{12pt} for readable notes.
	\item Use the correct paper size for your country/institution.
	\item Avoid \texttt{twocolumn} until you are comfortable with layout.
\end{itemize}
\end{remark}	

\subsubsection{Packages}
	
Packages are extensions that add new features and commands. We load them in the preamble using \verb|\usepackage{...}|.
%\begin{definition}[Package]
%	A \LaTeX{} package is a collection of macros and commands that adds functionality to \LaTeX{}. Load it using \verb|\usepackage{...}|.
%\end{definition}
\medskip

\begin{lstlisting}
\usepackage{amsmath, amssymb}
\usepackage{graphicx}
\usepackage{hyperref}
\end{lstlisting}
\begin{table}[h!]\centering
\renewcommand{\arraystretch}{1.25}
\begin{tabularx}{\textwidth}{@{}l|X@{}}
	\toprule[1.2pt]
	\texttt{amsmath}, \texttt{amssymb}& improved math environments and symbols. \\ \hline
	\texttt{graphicx}& enables \verb|\includegraphics| for inserting images. \\ \hline
	\texttt{hyperref}& clickable links and clickable cross-references in the PDF. \\
	\bottomrule[1.2pt]
\end{tabularx}
\end{table}	
%	The reference notes show \texttt{geometry} as a standard package to control margins easily:contentReference[oaicite:15]{index=15}.
\begin{example}[Margins with \texttt{geometry}]
\; \medskip

\begin{lstlisting}
\usepackage[left=1.5in, right=1.5in, top=1in, bottom=1in]{geometry}
\end{lstlisting}
\end{example}
	
\subsubsection{Skeleton Template}
	
A typical skeleton combines class selection, packages, metadata (title/author/date), and document body commands. The reference provides a standard example.

\begin{example}[Typical Structure]
\; \medskip
	
\begin{lstlisting}
\documentclass[11pt]{article}
\usepackage[utf8]{inputenc}
\usepackage{amsmath, amssymb}
\usepackage{graphicx}
\usepackage{hyperref}

\title{My First Document}
\author{Student Name}
\date{\today}

\begin{document}
	\maketitle
	
	Hello, LaTeX World!
	
\end{document}
\end{lstlisting}
\end{example}

%	\begin{itemize}
%		\item \verb|\documentclass[11pt]{article}| selects the class and base font size:contentReference[oaicite:17]{index=17}.
%		\item \verb|\usepackage...| loads tools for math, graphics, and hyperlinks:contentReference[oaicite:18]{index=18}.
%		\item \verb|\title|, \verb|\author|, \verb|\date| define metadata used by \verb|\maketitle|:contentReference[oaicite:19]{index=19}.
%		\item \verb|\begin{document}| starts the content and \verb|\end{document}| ends it:contentReference[oaicite:20]{index=20}.
%	\end{itemize}

%
%\subsubsection{Exercises}
%
%\begin{exercise}
%	Create a new \LaTeX{} file with:
%	\begin{itemize}
%		\item Class: \texttt{report}
%		\item Title: “My LaTeX Project”
%		\item Packages: \texttt{amsmath}, \texttt{graphicx}, \texttt{hyperref}
%		\item A simple sentence: “This is my LaTeX report.”
%	\end{itemize}
%	Compile it twice and confirm that the PDF title page appears correctly (reports often differ from articles).
%\end{exercise}
%
%\begin{exercise}
%	Experiment with class options:
%	\begin{itemize}
%		\item Start with \verb|\documentclass[10pt]{article}|
%		\item Change it to \verb|\documentclass[12pt]{article}|
%	\end{itemize}
%	Observe how font size and page feel change. Then try adding \texttt{a4paper} or \texttt{letterpaper} and notice the difference.
%\end{exercise}
%
%\begin{exercise}
%	Add the \texttt{geometry} package and set custom margins (e.g., left and right larger than top and bottom).
%\end{exercise}