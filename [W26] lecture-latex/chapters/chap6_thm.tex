\chapter{Theorems, Definitions, and Proofs}

\section{Using the \texttt{amsthm} Package}

To define theorem-style environments in LaTeX, we use the \texttt{amsthm} package.

In your preamble, define environments like so:

\begin{verbatim}
	\usepackage{amsthm}
	
	\newtheorem{theorem}{Theorem}[chapter]
	\newtheorem{lemma}[theorem]{Lemma}
	\newtheorem{definition}[theorem]{Definition}
	\newtheorem{corollary}[theorem]{Corollary}
	\theoremstyle{remark}
	\newtheorem*{remark}{Remark}
\end{verbatim}

\section{Basic Environments}

\begin{definition}[Even Number]
	An integer $n$ is called \emph{even} if there exists an integer $k$ such that $n = 2k$.
\end{definition}

\begin{theorem}[Sum of Evens]
	The sum of two even integers is even.
\end{theorem}

\begin{proof}
	Let $a = 2m$ and $b = 2n$ for some integers $m$ and $n$.  
	Then $a + b = 2m + 2n = 2(m + n)$, which is divisible by $2$.  
	Therefore, $a + b$ is even.
\end{proof}

\section{More Structures}

\begin{lemma}[Basic Algebra]
	If $a^2 = b^2$, then $a = b$ or $a = -b$.
\end{lemma}

\begin{corollary}
	If $x^2 = 9$, then $x = 3$ or $x = -3$.
\end{corollary}

\begin{remark}
	You can group lemmas, theorems, and corollaries using the same counter by specifying it in square brackets.
\end{remark}

\section{Proof Formatting Tips}

\begin{itemize}
	\item Use short paragraphs inside proofs.
	\item Always close proofs with \verb|\qed| or let LaTeX do it automatically.
	\item Avoid long inline equations in proofs; prefer display mode if clarity is needed.
\end{itemize}

\section{Example: The Pythagorean Theorem}

\begin{theorem}[Pythagorean Theorem]
	In a right triangle with legs $a$, $b$ and hypotenuse $c$, we have:
	\[
	c^2 = a^2 + b^2
	\]
\end{theorem}

\begin{proof}
	There are many geometric and algebraic proofs. One method involves comparing the area of squares built on each side.
\end{proof}

\section{Exercise}

\begin{exercise}
	Define your own:
	\begin{itemize}
		\item A definition for a "prime number"
		\item A theorem stating: “If $p$ is prime and $p$ divides $ab$, then $p$ divides $a$ or $p$ divides $b$.”
		\item A brief proof using basic divisibility
	\end{itemize}
\end{exercise}
