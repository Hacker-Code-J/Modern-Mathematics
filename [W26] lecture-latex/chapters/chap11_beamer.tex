\chapter{Presentations with Beamer}

\section{What is Beamer?}

\texttt{beamer} is a LaTeX document class for creating high-quality presentations. It uses frames (slides) instead of pages and provides built-in themes, templates, and transitions.

\section{Basic Beamer Document}

\begin{verbatim}
	\documentclass{beamer}
	\usetheme{Madrid}
	
	\title{My First Beamer Presentation}
	\author{Your Name}
	\date{\today}
	
	\begin{document}
		
		\frame{\titlepage}
		
		\begin{frame}{Outline}
			\tableofcontents
		\end{frame}
		
		\section{Introduction}
		
		\begin{frame}{Why Beamer?}
			\begin{itemize}
				\item Professional slides
				\item Integrated with math and references
				\item Fully customizable
			\end{itemize}
		\end{frame}
		
	\end{document}
\end{verbatim}

\section{The \texttt{frame} Environment}

Each slide is enclosed in a \texttt{frame}:

\begin{verbatim}
	\begin{frame}{Slide Title}
		Slide content goes here.
	\end{frame}
\end{verbatim}

You can also use:

\begin{verbatim}
	\frame{\frametitle{Title} Slide content}
\end{verbatim}

\section{Itemized and Numbered Lists}

\begin{frame}{Lists}
	\begin{itemize}
		\item Bullet 1
		\item Bullet 2
	\end{itemize}
	
	\begin{enumerate}
		\item First
		\item Second
	\end{enumerate}
\end{frame}

\section{Math in Beamer}

You can include math just like in articles:

\begin{frame}{Math Example}
	The quadratic formula:
	\[
	x = \frac{-b \pm \sqrt{b^2 - 4ac}}{2a}
	\]
\end{frame}

\section{Themes and Color Schemes}

Use \verb|\usetheme| in the preamble to set the theme:

\begin{verbatim}
	\usetheme{Madrid}     % clean and simple
	\usetheme{Boadilla}   % serifed look
	\usetheme{AnnArbor}   % bold, yellow theme
	\usetheme{Warsaw}     % darker theme
\end{verbatim}

You can combine themes with:

\begin{verbatim}
	\usecolortheme{seahorse}
\end{verbatim}

\section{Blocks for Emphasis}

%\begin{frame}{Blocks}
%	\begin{block}{Standard Block}
%		This is a regular block.
%	\end{block}
%	
%	\begin{alertblock}{Alert Block}
%		This is a red alert!
%	\end{alertblock}
%	
%	\begin{exampleblock}{Example Block}
%		This is an example.
%	\end{exampleblock}
%\end{frame}

\section{Pause and Uncovering Content}

\begin{itemize}
	\item \verb|\pause| reveals the next item on each click.
	\item \verb|\uncover<2->{...}| shows content starting from slide step 2.
\end{itemize}

\section{Exercise}

\begin{exercise}
	Create a presentation with:
	\begin{itemize}
		\item A title slide and table of contents
		\item At least two sections with one frame each
		\item An equation slide
		\item A block with an important note
		\item Use the theme \texttt{Warsaw}
	\end{itemize}
\end{exercise}
