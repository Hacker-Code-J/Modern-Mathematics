\chapter{Cross-Referencing and Labels}

\section{Why Use Cross-References?}

Cross-referencing allows you to:
\begin{itemize}
	\item Refer to sections, figures, tables, and equations dynamically
	\item Automatically update numbers when structure changes
	\item Improve document navigability, especially in long documents
\end{itemize}

\section{Basic Labeling Syntax}

Use \verb|\label{key}| to mark a location, and use \verb|\ref{key}| to refer to it later.

\begin{example}
	\begin{verbatim}
		\section{Methodology}
		\label{sec:method}
		
		As discussed in Section~\ref{sec:method}, we use...
	\end{verbatim}
\end{example}

\section{Referencing Figures and Tables}

\begin{itemize}
	\item Place \verb|\label| after \verb|\caption|
	\item Reference with \verb|\ref|
\end{itemize}

\begin{example}
	\begin{verbatim}
		\begin{figure}[h]
			\centering
			\includegraphics[width=0.5\textwidth]{plot}
			\caption{Data trend}
			\label{fig:trend}
		\end{figure}
		
		See Figure~\ref{fig:trend}.
	\end{verbatim}
\end{example}

\section{Referencing Equations}

For numbered equations:

\begin{verbatim}
	\begin{equation}
		a^2 + b^2 = c^2
		\label{eq:pythagoras}
	\end{equation}
	
	From Eq.~\eqref{eq:pythagoras}, we know...
\end{verbatim}

\section{Referencing Theorems and Definitions}

\begin{example}
	\begin{verbatim}
		\begin{theorem}
			\label{thm:addition}
			Addition of even numbers is even.
		\end{theorem}
		
		Theorem~\ref{thm:addition} proves the statement.
	\end{verbatim}
\end{example}

\section{Hyperlinks with \texttt{hyperref}}

To make all references clickable in the PDF, use the \texttt{hyperref} package:

\begin{verbatim}
	\usepackage{hyperref}
\end{verbatim}

\begin{itemize}
	\item \verb|\href{URL}{text}| — insert a clickable external link
	\item \verb|\url{URL}| — insert a plain clickable URL
\end{itemize}

\begin{example}
	\href{https://www.latex-project.org}{Visit the LaTeX Project}
\end{example}

\section{Best Practices}

\begin{itemize}
	\item Prefix label names: \texttt{sec:}, \texttt{fig:}, \texttt{eq:}, \texttt{thm:}, etc.
	\item Always run \LaTeX{} twice to update references.
\end{itemize}

\section{Exercise}

\begin{exercise}
	Write a LaTeX document that includes:
	\begin{itemize}
		\item A labeled section titled “Results”
		\item A numbered and labeled equation
		\item A figure with caption and label
		\item A clickable link to an external website
		\item Proper references using \verb|\ref| and \verb|\eqref|
	\end{itemize}
\end{exercise}
