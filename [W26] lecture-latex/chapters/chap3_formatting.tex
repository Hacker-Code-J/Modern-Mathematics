\chapter{Text Formatting and Layout}

\section{Basic Text Commands}

LaTeX provides a variety of text formatting commands to style your writing.

\begin{itemize}
	\item \verb|\textbf{bold}| → \textbf{bold}
	\item \verb|\textit{italic}| → \textit{italic}
	\item \verb|\underline{underline}| → \underline{underline}
	\item \verb|\emph{emphasized}| → \emph{emphasized} (contextual)
\end{itemize}

\begin{example}
	This is \textbf{bold}, \textit{italic}, and \underline{underlined}.
\end{example}

\section{Paragraphs and Line Breaks}

\begin{itemize}
	\item A blank line starts a new paragraph.
	\item Use \verb|\\| for a manual line break.
\end{itemize}

\begin{example}
	This is the first line.\\
	This is the second line.
\end{example}

\section{Quotes and Verbatim Text}

\begin{itemize}
	\item Use \verb|\verb| for short inline code: \verb|\LaTeX|.
	\item Use the \texttt{verbatim} environment for code blocks.
	\item Use \texttt{quote} or \texttt{quotation} for indented blocks.
\end{itemize}

\begin{example}
	\begin{verbatim}
		\begin{itemize}
			\item First
			\item Second
		\end{itemize}
	\end{verbatim}
\end{example}

\section{Text Alignment}

\begin{itemize}
	\item \verb|\begin{center}...\end{center}| → centered text
	\item \verb|\begin{flushleft}...\end{flushleft}| → left-aligned
	\item \verb|\begin{flushright}...\end{flushright}| → right-aligned
\end{itemize}

\begin{example}
	\begin{center}
		This is centered.
	\end{center}
\end{example}

\section{Font Sizes}

\LaTeX{} supports relative font sizes:

\begin{itemize}
	\item \verb|\tiny|, \verb|\scriptsize|, \verb|\footnotesize|
	\item \verb|\small|, \verb|\normalsize|, \verb|\large|, \verb|\Large|
	\item \verb|\LARGE|, \verb|\huge|, \verb|\Huge|
\end{itemize}

\begin{example}
	{\Large This is large text.}
\end{example}

\section{Special Characters}

Some characters must be escaped using a backslash:

\begin{tabular}{ll}
	\verb|\%| & \% (percent) \\
	\verb|\$| & \$ (dollar) \\
	\verb|\#| & \# (hash) \\
	\verb|\_| & \_ (underscore) \\
	\verb|\&| & \& (ampersand) \\
	\verb|\{| \verb|\}| & \{ \}, braces
\end{tabular}

\section{Comments}

Use the percent symbol (\verb|%|) to add comments:

\begin{verbatim}
	% This is a comment and will not appear in the output
\end{verbatim}

\section{Exercise}

\begin{exercise}
	Create a document with:
	\begin{itemize}
		\item A centered title using \verb|\begin{center}|
			\item A paragraph using \verb|\textit| and \verb|\textbf|
			\item One inline verbatim command using \verb|\verb|
			\item A list of three escaped special characters
		\end{itemize}
	\end{exercise}
