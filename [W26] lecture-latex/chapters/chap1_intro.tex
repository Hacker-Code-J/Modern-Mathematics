\chapter{Introduction to \LaTeX}

\section{What is \LaTeX?}

\LaTeX{} (pronounced either “Lay-tek” or “Lah-tek”) is a high-quality typesetting system designed for producing scientific and technical documents. It separates content from formatting, allowing authors to focus on writing rather than visual layout.

\begin{definition}[LaTeX]
	\LaTeX{} is a document preparation system based on \TeX{}, designed for the creation of beautifully typeset documents, especially those containing complex mathematical expressions.
\end{definition}

\subsection{Why Use \LaTeX?}

Some benefits of using \LaTeX{} include:

\begin{itemize}
	\item Professional-quality typesetting
	\item Excellent mathematical rendering (e.g., $a^2 + b^2 = c^2$)
	\item Automated references, bibliographies, and table of contents
	\item Consistent document structure and style
\end{itemize}

\section{Compiling a Document}

To convert a `.tex` file into a PDF, you need to compile it using a LaTeX engine such as `pdflatex`.

\begin{verbatim}
	pdflatex file.tex
	pdflatex file.tex  % Run twice for updated references
\end{verbatim}

\subsection{Recommended Tools}

\begin{itemize}
	\item \textbf{Overleaf}: Online editor, no installation required.
	\item \textbf{TeX Live}: Comprehensive distribution for Linux/macOS.
	\item \textbf{MiKTeX}: User-friendly distribution for Windows.
\end{itemize}

\section{Minimal Example}

Every LaTeX document consists of two main parts:

\begin{itemize}
	\item \textbf{Preamble} (before \verb|\begin{document}|): specifies the document class and packages
		\item \textbf{Body} (between \verb|\begin{document}| and \verb|\end{document}|): contains the actual content
	\end{itemize}
	
	\begin{example}[Minimal Document]
		\begin{verbatim}
			\documentclass{article}
			
			\begin{document}
				Hello, world! $E = mc^2$
			\end{document}
		\end{verbatim}
	\end{example}
	
	\section{Chapter Summary}
	
	\begin{itemize}
		\item LaTeX is a powerful typesetting system.
		\item Compilation turns source code into a PDF.
		\item Use \verb|pdflatex| or Overleaf to build your documents.
	\end{itemize}
	
	\begin{exercise}
		Install Overleaf or a local TeX distribution. Create a `.tex` file with a simple "Hello, world!" message and compile it.
	\end{exercise}
