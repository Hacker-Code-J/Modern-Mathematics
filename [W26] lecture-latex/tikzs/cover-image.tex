\documentclass[12pt]{article}

% --- 1. Page Setup for Letter Size ---
\usepackage[letterpaper, margin=0pt]{geometry}
\usepackage[T1]{fontenc}
%\usepackage{mathpazo} % Palatino font
\usepackage{lmodern}
\usepackage{anyfontsize}
\usepackage{tikz}
\usetikzlibrary{calc, fadings, decorations.markings, backgrounds}

% --- 2. Color Palette (Solarized Dark Inspired) ---
\definecolor{bgbase}{HTML}{002B36}      % Deep Green-Blue
\definecolor{accentcyan}{HTML}{2AA198}  % Cyan
\definecolor{accentblue}{HTML}{268BD2}  % Blue
\definecolor{textlight}{HTML}{FDF6E3}   % Cream
\definecolor{textdim}{HTML}{93A1A1}     % Dim Grey

\begin{document}
	
% No page numbers
\pagestyle{empty}

\begin{tikzpicture}[remember picture, overlay]
	
	% =========================================================
	% 1. BACKGROUND
	% =========================================================
	\fill[bgbase] (current page.south west) rectangle (current page.north east);
	
	% Optional: Add subtle "code" texture in background
	\node[textdim, opacity=0.05, font=\ttfamily\tiny, anchor=north west, align=left, rotate=0] 
	at ([xshift=0.5in, yshift=-0.5in]current page.north west) 
	{
		\\documentclass\{book\} \\
		\\usepackage\{tikz\} \\
		\\begin\{document\} \\
		\% Creating beautiful documents... \\
		\\section\{Introduction\} \\
		The art of typesetting...
	};
	
	
	% =========================================================
	% 2. ABSTRACT GEOMETRIC ART (The "Wireframe Sphere")
	% =========================================================
	% We draw a parametric spirograph/sphere pattern to show off TikZ capabilities
	\begin{scope}[shift={([xshift=2.5in, yshift=-4in]current page.north west)}, scale=2.5]
		
		% Fading circles in background
		\fill[accentblue, opacity=0.1] (0,0) circle (3.5);
		\fill[accentcyan, opacity=0.1] (0,0) circle (2.8);
		
		% The Wireframe Pattern
		\draw[accentcyan, thick, opacity=0.6, samples=100, domain=0:360, variable=\t, smooth]
		plot ({3*cos(\t)}, {3*sin(\t)*cos(2*\t)});
		
		\draw[accentblue, thick, opacity=0.6, samples=100, domain=0:360, variable=\t, smooth, rotate=45]
		plot ({3*cos(\t)}, {3*sin(\t)*cos(2*\t)});
		
		\draw[white, thick, opacity=0.4, samples=100, domain=0:360, variable=\t, smooth, rotate=90]
		plot ({3*cos(\t)}, {3*sin(\t)*cos(2*\t)});
		
		\draw[accentcyan!50!white, thick, opacity=0.6, samples=100, domain=0:360, variable=\t, smooth, rotate=135]
		plot ({3*cos(\t)}, {3*sin(\t)*cos(2*\t)});
		
		% Central Core
		\fill[white, opacity=0.9] (0,0) circle (2pt);
%		\node[textdim, font=\tiny, anchor=north, yshift=-10pt] {TikZ Render Engine};
	\end{scope}
	
	% =========================================================
	% 3. TYPOGRAPHY (Modern Sans-Serif)
	% =========================================================
	
	% Main Title
	\node[anchor=west, align=left] 
	at ([xshift=1in, yshift=-7.5in]current page.north west) 
	{
		\sffamily
		\fontsize{50}{60}\selectfont \bfseries \textcolor{textlight}{The Art of}\\[0.2em]
		\fontsize{50}{60}\selectfont \bfseries \textcolor{accentcyan}{Modern \LaTeX}
	};
	
	% Separator Line
	\draw[accentblue, thick] 
	([xshift=1in, yshift=-8.5in]current page.north west) -- 
	([xshift=7.5in, yshift=-8.5in]current page.north west);
	
	% Subtitle & Subsubtitle
	\node[anchor=north west, align=left] 
	at ([xshift=1in, yshift=-8.7in]current page.north west) 
	{
		\sffamily
		\fontsize{20}{24}\selectfont \textcolor{textlight}{Professional Typesetting \& Vector Graphics}\\[0.5em]
		\fontsize{14}{18}\selectfont \textcolor{textdim}{From Standard Documents to Advanced TikZ}\\[1.5em]
		\fontsize{16}{20}\selectfont \textcolor{white}{Ji, Yonghyeon}
	};
	
	% =========================================================
	% 4. FOOTER / BADGE
	% =========================================================
	
	% "Lecture Notes" Badge (Top Right)
	\node[
	fill=accentblue, 
	text=bgbase, 
	font=\sffamily\bfseries\small, 
	anchor=north east, 
	minimum height=3em, 
	minimum width=8em
	] 
	at ([xshift=0in, yshift=-1in]current page.north east) 
	{LECTURE NOTES};
	
	% Bottom Footer
	\node[textdim, font=\sffamily\footnotesize, anchor=south] 
	at ([yshift=0.5in]current page.south) 
	{A Comprehensive Guide to \LaTeX};
\end{tikzpicture}
	
\end{document}