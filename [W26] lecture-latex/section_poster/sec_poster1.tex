A poster is a large-format document (often A0, A1, or 36in $\times$ 48in) designed for fast reading at a distance.
A good poster is \textbf{not} a paper on a big page. It is a visual summary:
\begin{itemize}
	\item a clear title and author line,
	\item 3--6 main blocks (Problem, Method, Results, Conclusion),
	\item a few strong figures and one key table,
	\item minimal text, large fonts, and consistent spacing.
\end{itemize}
For beginners, use \textbf{beamerposter} because it is stable, well-documented, and uses the familiar \texttt{beamer} “block” style. 
%Other popular options:
%\begin{itemize}
%	\item \texttt{beamerposter}: flexible blocks and columns; very common in academia.
%	\item \texttt{tikzposter}: attractive themes; good for visual posters.
%	\item \texttt{baposter}: classic scientific posters; more manual tuning.
%\end{itemize}
%In these notes, we focus on \textbf{beamerposter}.

%\subsection{Goal of This Guide}
%
%This guide explains how the following poster template works and how to customize it safely:
%\begin{itemize}
%	\item Class: \texttt{beamer} + \texttt{beamerposter}
%	\item Layout: portrait (vertical), A0, two columns
%	\item Content containers: \texttt{block}, \texttt{exampleblock}, \texttt{alertblock}
%\end{itemize}
%
%\subsection{What a Poster File Looks Like}
%
%A \LaTeX{} poster file typically has:
%\begin{itemize}
%	\item a \textbf{setup part} (document class, poster package settings, theme, packages, title info),
%	\item a \textbf{single poster page} made as one \texttt{frame},
%	\item a \textbf{layout system} using \texttt{columns} and blocks.
%\end{itemize}
%
\subsection{The Poster Template}

\begin{center}
\includegraphics[width=\textwidth]{samples/poster.png}
\end{center}
\begin{lstlisting}
\documentclass[final]{beamer}

\usepackage[size=a0,orientation=portrait,scale=1]{beamerposter}

\usetheme{default}
\usecolortheme{seahorse}

\usepackage{graphicx}
\usepackage{booktabs}
\usepackage{amsmath,amssymb}
\usepackage{hyperref}

\title{A \LaTeX{} Poster Example}
\author{Student Name \and Coauthor Name}
\institute{Department of Something, University Name \\ \texttt{email@domain.com}}
\date{Conference / Event, 2026}

\setlength{\parskip}{0.35em}

\begin{document}
	\begin{frame}[t]
		% header columns ...
		% main body columns ...
	\end{frame}
\end{document}
\end{lstlisting}

\subsection{Step 1: Document Class}
\begin{lstlisting}
\documentclass[final]{beamer}
\end{lstlisting}
\begin{itemize}
	\item \texttt{beamer} is usually for slides.
	\item With \texttt{beamerposter}, one \texttt{frame} becomes one large-format poster page.
	\item \texttt{final} reduces draft-style marks and is appropriate for submission/printing.
\end{itemize}

\newpage
\subsection{Step 2: Make It Vertical (Portrait)}
\begin{lstlisting}
\usepackage[size=a0,orientation=portrait,scale=1]{beamerposter}
\end{lstlisting}
\begin{itemize}
	\item `\texttt{size=a0}': the poster paper size (A0 is common for conferences).
	\item `\texttt{orientation=portrait}': vertical poster (taller than wide).
	\item `\texttt{scale=1}': global scaling of fonts and layout elements.
\end{itemize}

\begin{table}[h]
	\centering
	\begin{tabular*}{\textwidth}{@{\extracolsep{\fill}}lll}
		\hline
		\textbf{Option} & \textbf{Example Value(s)} & \textbf{Meaning / Effect} \\
		\hline
		\texttt{size} &
		\texttt{a0}, \texttt{a1}, \texttt{a2}, \texttt{a3}, \texttt{a4}, \texttt{letter} &
		Sets the poster paper size. \\
		\texttt{orientation} &
		\texttt{portrait}, \texttt{landscape} &
		Controls layout direction. \\
		\texttt{scale} &
		\texttt{0.95}, \texttt{1.0}, \texttt{1.05}, \texttt{1.1} &
		Global scaling of fonts and layout elements. \\
		\hline
	\end{tabular*}
	\caption{Common \texttt{beamerposter} options for size, orientation, and scale}
	\label{tab:beamerposter-options}
\end{table}
\noindent
The `\texttt{size}' option sets the poster paper size. Common values include:
\begin{itemize}
	\item \texttt{a0}, \texttt{a1}, \texttt{a2}, \texttt{a3}, \texttt{a4} \quad (ISO paper sizes)
	\item \texttt{letter} \quad (common US sizes)
\end{itemize}
The `\texttt{orientation}' option controls whether the poster is vertical or horizontal:

\begin{itemize}
	\item \texttt{orientation=portrait} \quad (vertical, tall)
	\item \texttt{orientation=landscape} \quad (horizontal, wide)
\end{itemize}

\begin{table}[h]
	\centering
	\begin{tabular*}{\textwidth}{@{\extracolsep{\fill}}lll}
		\hline
		\textbf{Category} & \textbf{Setting} & \textbf{Description} \\
		\hline
		Size (ISO) & \texttt{size=a0} & Very large poster size \\
		Size (ISO) & \texttt{size=a1} & Large poster size \\
		Size (ISO) & \texttt{size=a2} & Medium poster size \\
		Size (ISO) & \texttt{size=a3} & Small poster size \\
		Size (ISO) & \texttt{size=a4} & Page size (not typical for posters) \\
		Size (US)  & \texttt{size=letter} & US letter size \\
		\hline
		Orientation & \texttt{orientation=portrait} & Vertical poster (taller than wide) \\
		Orientation & \texttt{orientation=landscape} & Horizontal poster (wider than tall) \\
		\hline
	\end{tabular*}
	\caption{Typical values for \texttt{size} and \texttt{orientation} in \texttt{beamerposter}}
	\label{tab:beamerposter-size-orientation}
\end{table}

\newpage
\subsection{Step 3: Theme and Colors}
\begin{lstlisting}
\usetheme{default}
\usecolortheme{seahorse}
\end{lstlisting}
\begin{itemize}
	\item The theme controls the general style of blocks and typography.
	\item The color theme changes the colors of block headers and accents.
	\item Beginners should keep themes simple and prioritize readability.
\end{itemize}
%We can replace \verb|default| with many built-in beamer themes. Popular examples:
%\begin{itemize}
%	\item \verb|\usetheme{default}|: clean, minimal, safe for posters.
%	\item \verb|\usetheme{Madrid}|: clear navigation style, common in slides.
%	\item \verb|\usetheme{Boadilla}|: simple and modern look.
%	\item \verb|\usetheme{Warsaw}|: darker style, strong contrast.
%	\item \verb|\usetheme{AnnArbor}|: bold headings, more colorful.
%\end{itemize}
%Color themes adjust block headers, accents, and some text colors. Examples:
%\begin{itemize}
%	\item \verb|\usecolortheme{seahorse}|: soft, readable tones.
%	\item \verb|\usecolortheme{dolphin}|: stronger contrast.
%	\item \verb|\usecolortheme{whale}|: darker, heavier style.
%	\item \verb|\usecolortheme{orchid}|: more colorful accents.
%	\item \verb|\usecolortheme{beaver}|: red/brown academic look.
%\end{itemize}
\begin{table}[h]
	\centering
	\begin{tabular*}{\textwidth}{@{\extracolsep{\fill}}lll}
		\hline
		\textbf{Type} & \textbf{Command} & \textbf{Typical Use / Look} \\
		\hline
		Theme & \verb|\usetheme{default}| & Clean, minimal, safe for posters \\
		Theme & \verb|\usetheme{Madrid}| & Clear navigation style, common in slides \\
		Theme & \verb|\usetheme{Boadilla}| & Simple, modern look \\
		Theme & \verb|\usetheme{Warsaw}| & Darker style, strong contrast \\
		Theme & \verb|\usetheme{AnnArbor}| & Bold headings, more colorful \\
		\hline
		Color theme & \verb|\usecolortheme{seahorse}| & Soft, readable tones \\
		Color theme & \verb|\usecolortheme{dolphin}| & Stronger contrast \\
		Color theme & \verb|\usecolortheme{whale}| & Darker, heavier style \\
		Color theme & \verb|\usecolortheme{orchid}| & More colorful accents \\
		Color theme & \verb|\usecolortheme{beaver}| & Red/brown academic look \\
		\hline
	\end{tabular*}
	\caption{Examples of Beamer themes and color themes for posters}
	\label{tab:beamer-themes-colorthemes}
\end{table}


\subsection{Step 4: Packages}
\begin{lstlisting}
\usepackage{graphicx}
\usepackage{booktabs}
\usepackage{amsmath,amssymb}
\usepackage{hyperref}
\end{lstlisting}

\begin{itemize}
	\item `\texttt{graphicx}': insert images with `\verb|\includegraphics|'.
	\item `\texttt{booktabs}': clean tables using `\verb|\toprule|', `\verb|\midrule|', `\verb|\bottomrule|'.
	\item `\texttt{amsmath,amssymb}': better math support (align, symbols, `\verb|\arg\min|' patterns).
	\item `\texttt{hyperref}': clickable links via `\verb|\href|' and `\verb|\url|'.
\end{itemize}

\subsection{Step 5: Title, Author, Institute, Date}
\begin{lstlisting}
\title{A \LaTeX{} Poster Example}
\author{Student Name \and Coauthor Name}
\institute{Department ... \\ \texttt{email@domain.com}}
\date{Conference / Event, 2026}
\end{lstlisting}

\subsection{Step 6: One Frame = One Poster Page}
\begin{lstlisting}
\begin{document}
	\begin{frame}[t]
		...
	\end{frame}
\end{document}
\end{lstlisting}
\begin{itemize}
	\item A poster is usually one single \texttt{frame}.
	\item The option \texttt{[t]} top-aligns the content, which is helpful for column layouts.
\end{itemize}

\subsection{Step 7: The Header}
Our header uses \texttt{columns} but only one full-width column:
\begin{lstlisting}
\begin{columns}[t]
	\begin{column}{1.0\textwidth}
		\begin{block}{}
			\centering
			{\VeryHuge \textbf{\inserttitle}\par}
			\vspace{0.35em}
			{\Large \insertauthor\par}
			{\large \insertinstitute\par}
			{\large \insertdate\par}
		\end{block}
	\end{column}
\end{columns}
\end{lstlisting}
\begin{itemize}
	\item \verb|\inserttitle| prints the value set by \verb|\title|.
	\item \verb|\insertauthor| prints the value set by \verb|\author|.
	\item \verb|\insertinstitute| prints the value set by \verb|\institute|.
	\item \verb|\insertdate| prints the value set by \verb|\date|.
	\item \verb|\VeryHuge| is useful for posters because titles must be readable from far away.
\end{itemize}

\subsection{Step 8: The Main Body Layout (Two Columns)}
\begin{lstlisting}
\begin{columns}[t]
	\begin{column}{0.49\textwidth}
		% left column blocks...
	\end{column}
	\begin{column}{0.49\textwidth}
		% right column blocks...
	\end{column}
\end{columns}
\end{lstlisting}
Why `\verb|0.49\textwidth|'?
\begin{itemize}
	\item It leaves a small gap so the columns do not collide.
	\item Using \verb|0.50| + \verb|0.50| often causes overflow in posters.
\end{itemize}

\subsection{Step 9: Block Boxes}
Beamer provides three useful boxed environments:
\begin{itemize}
	\item \texttt{block}: normal content
	\item \texttt{exampleblock}: examples/demos
	\item \texttt{alertblock}: emphasis (main result, conclusion, warning)
\end{itemize}
Example patterns:

\begin{lstlisting}
\begin{block}{Title} ... \end{block}
\begin{exampleblock}{Example} ... \end{exampleblock}
\begin{alertblock}{Key Message} ... \end{alertblock}
\end{lstlisting}