\documentclass[tikz, border=0pt]{standalone}
\usepackage{pgfplots}
\pgfplotsset{compat=1.18}
\usetikzlibrary{arrows.meta, backgrounds, calc, shadings}

% --- Define Custom Colormaps for a "Dramatic" look ---
\pgfplotsset{
	colormap={cover_ice_fire}{
		color(0cm)=(blue!40!black);
		color(1cm)=(blue!80!cyan);
		color(2cm)=(cyan!80!white);
		color(3cm)=(white!90!yellow);
		color(4cm)=(orange!80!yellow);
		color(5cm)=(red!80!orange);
		color(6cm)=(red!50!black)
	}
}

\begin{document}
	% Use a dark background to make the colors pop, typical for modern textbooks
	\begin{tikzpicture}[
%		background rectangle/.style={
%			top color=gray!10!black,
%			bottom color=gray!30!black,
%			shading angle=45
%		},
%		show background rectangle,
		font=\sffamily
		]
		
		\begin{axis}[
			view={120}{30},    % A dramatic, slightly elevated angle
			axis lines=none,   % Hide axes for a clean artistic look
			width=16cm, height=14cm,
			z buffer=sort,     % Ensure correct 3D depth sorting
			xmin=-2.5, xmax=2.5,
			ymin=-2.5, ymax=2.5,
			zmin=-5, zmax=5,
			colormap name=viridis,
			]
			
			% ==================================================
			% 1. The Main Surface: A Hyperbolic Paraboloid (Saddle)
			% Function: z = y^2 - x^2
			% ==================================================
			\addplot3[
			surf,
			shader=interp,         % Smooth gradient shading
			opacity=0.9,           % Slight transparency
			domain=-2.2:2.2,       % Plot domain
			y domain=-2.2:2.2,
			samples=50,            % High sample count for smooth look (adjust if memory issues arise)
			samples y=50,
			] {y^2 - x^2};
			
			% ==================================================
			% 2. The Vector Field (Flow)
			% Visualizing a flow over the surface.
			% We'll use a flow that spirals slightly towards the saddle point for visual interest.
			% Vector V = <-y - 0.5x, x - 0.5y, dz> (Approximate spiral inflow)
			% ==================================================
			\def\sf{0.25} % Scale factor for arrows
			\addplot3[
			quiver={
				u={\sf*(-y - 0.3*x)},    % x-component of vector
				v={\sf*(x - 0.3*y)},     % y-component of vector
				w={\sf*(2*y*(x - 0.3*y) - 2*x*(-y - 0.3*x))}, % z-component (chain rule dz = df/dx*u + df/dy*v)
				scale arrows=1,
				every arrow/.append style={
					-Stealth, 
					opacity=0.6, 
					color=white!90!cyan, % Light color to contrast with the surface
					line width=0.8pt
				}
			},
			samples=15, % Fewer samples for arrows to avoid clutter
			samples y=15,
			domain=-2:2,
			y domain=-2:2,
			] {y^2 - x^2};
			
			
			% ==================================================
			% 3. Focal Point: Tangent Plane and Normal Vector
			% Pick a point P at (x_0, y_0) = (1.2, 1.5)
			% z_0 = 1.5^2 - 1.2^2 = 2.25 - 1.44 = 0.81
			% Partial dx = -2x = -2.4
			% Partial dy =  2y =  3.0
			% Normal Vector N = <-dx, -dy, 1> = <2.4, -3.0, 1>
			% ==================================================
			
			% Define the point P
			\coordinate (P) at (axis cs: 1.2, 1.5, 0.81);
			
			% --- Draw the Normal Vector ---
%			\draw[->, line width=2.5pt, color=red!90!white, glow={white}] 
%			(P) -- +(axis direction cs: 2.4, -3.0, 1); 
%			node[above right, text=white] {\large $\mathbf{n}$};
			
			% --- Draw a Patch of the Tangent Plane ---
			% Plane Eq: z - 0.81 = -2.4(x - 1.2) + 3.0(y - 1.5)
			% We draw a small square around P lying on this plane.
			\fill[red, opacity=0.4, shininess=20]
			(axis cs: 1.2-0.5, 1.5-0.5, {0.81 - 2.4*(-0.5) + 3.0*(-0.5)}) --
			(axis cs: 1.2+0.5, 1.5-0.5, {0.81 - 2.4*(+0.5) + 3.0*(-0.5)}) --
			(axis cs: 1.2+0.5, 1.5+0.5, {0.81 - 2.4*(+0.5) + 3.0*(+0.5)}) --
			(axis cs: 1.2-0.5, 1.5+0.5, {0.81 - 2.4*(-0.5) + 3.0*(+0.5)}) -- cycle;
			
			% Mark the point P
			\fill[red!30!white] (P) circle (2pt);
			
		\end{axis}
		
%		% ==================================================
%		% 4. Cover Text Placeholders (Optional Overlay)
%		% ==================================================
%		\node[
%		text=white, 
%		anchor=north west, 
%		align=left, 
%		font=\sffamily\bfseries\huge
%		] at ($(current bounding box.north west) + (1, -1)$) {
%			MULTIVARIABLE\\CALCULUS
%		};
%		
%		\node[
%		text=gray!30, 
%		anchor=north west, 
%		font=\sffamily\Large
%		] at ($(current bounding box.north west) + (1, -3.5)$) {
%			Early Transcendentals
%		};
%		
%		\node[
%		text=white, 
%		anchor=south east, 
%		font=\sffamily\large
%		] at ($(current bounding box.south east) + (-1, 1)$) {
%			Fourth Edition
%		};
		
	\end{tikzpicture}
\end{document}