\documentclass[12pt]{article}

% --- 1. Page Setup for Letter Size ---
\usepackage[letterpaper,margin=0in]{geometry}
\usepackage{amsmath, amssymb}
\usepackage[T1]{fontenc}
\usepackage{lmodern}
\usepackage{anyfontsize}
\usepackage{tikz}
\usetikzlibrary{
	calc, fadings, decorations.markings, backgrounds, arrows.meta
}

% --- 2. Color Palette (Solarized Dark Inspired) ---
\definecolor{bgbase}{HTML}{002B36}      % Deep Green-Blue
\definecolor{accentcyan}{HTML}{2AA198}  % Cyan
\definecolor{accentblue}{HTML}{268BD2}  % Blue
\definecolor{textlight}{HTML}{FDF6E3}   % Cream
\definecolor{textdim}{HTML}{93A1A1}     % Dim Grey

% =========================================================
% 3. Helpers for the NEW ABSTRACT GEOMETRIC ART
% =========================================================

% A small orbit equation node: bubble + label + formula (math)
% Usage: \EqNode{<x>}{<y>}{<label>}{<math formula>}
\newcommand{\EqNode}[4]{%
	\begin{scope}[shift={(#1,#2)}]
		\fill[white, opacity=0.03] (0,0) circle (0.44);
		\draw[white, opacity=0.16, line width=0.9pt] (0,0) circle (0.44);
		
%		% tiny "glyph" strokes (purely decorative)
%		\draw[accentcyan, opacity=0.55, line width=1.0pt] (-0.20,0.06) -- (0.20,0.06);
%		\draw[accentblue, opacity=0.55, line width=1.0pt] (-0.20,-0.06) -- (0.20,-0.06);
%		\fill[white, opacity=0.55] (0,0) circle (0.035);
		
		% Label + formula (math)
		\node[textdim, opacity=1, font=\sffamily\tiny, yshift=5pt] at (0,0) {#3};
		\node[textlight, opacity=1, font=\ttfamily\tiny, yshift=-6pt] at (0,0) {$#4$};
	\end{scope}%
}

% A curved-arrow style for "reductions"
\tikzset{
	reduction/.style={
		line width=1.0pt,
		draw=white,
		draw opacity=0.16,
		postaction={
			decorate,
			decoration={
				markings,
				mark=at position 0.60 with {\arrow[white, opacity=0.25]{Stealth[length=4pt]}}
			}
		}
	}
}

\begin{document}
	\pagestyle{empty}
	
	\begin{tikzpicture}[remember picture, overlay]
		
		% =========================================================
		% 1. BACKGROUND
		% =========================================================
		\fill[bgbase] (current page.south west) rectangle (current page.north east);
		
%		% Subtle "crypto/pqc" texture (optional)
%		\node[
%		textdim, opacity=1, font=\ttfamily\tiny,
%		anchor=north west, align=left
%		]
%		at ([xshift=0.55in, yshift=-0.55in]current page.north west)
%		{
%			\% Classical assumptions \& post-quantum foundations \\
%			Factoring:\ $N=pq$ \qquad DLP:\ $g^x=h$ \qquad ECC:\ $[x]P=Q$ \\
%			Lattices:\ $\lambda_1(\Lambda)$,\ $\mathrm{CVP}$ \qquad LWE:\ $As+e\equiv b\ (\bmod\ q)$ \\
%			Codes:\ $He^T=s$ \qquad Hash:\ $H(m)=H(m')$ \qquad Isogeny:\ $\varphi:E_1\to E_2$
%		};
		
		% =========================================================
		% 2. NEW ABSTRACT GEOMETRIC ART (embedded hard problems + formulas)
		%    "Reduction Mandala: lattices + interference + equation orbit"
		% =========================================================
		\begin{scope}[shift={([xshift=2.55in, yshift=-4.00in]current page.north west)}, scale=2.55]
		%[shift={(10,-10)}, scale=2.55]
			
			% --- Soft halo (two-tone)
			\fill[accentblue, opacity=0.11] (0,0) circle (3.60);
			\fill[accentcyan, opacity=0.09] (0,0) circle (2.90);
			\fill[black, opacity=0.14] (0,0) circle (3.45);
			
			% --- Outer rings
			\draw[white, opacity=0.10, line width=0.9pt] (0,0) circle (3.15);
			\draw[white, opacity=0.07, line width=0.9pt] (0,0) circle (2.45);
			\draw[white, opacity=0.05, line width=0.9pt] (0,0) circle (1.80);
			
			% --- Clip the globe
			\begin{scope}
				\clip (0,0) circle (3.18);
				% Dual-lattice field (primal + dual, rotated)
				\begin{scope}[opacity=0.070]
					\draw[accentcyan, line width=0.6pt] (-3.4,-3.4) grid[step=0.50] (3.4,3.4);
				\end{scope}
				\begin{scope}[rotate=31, opacity=0.055]
					\draw[accentblue, line width=0.6pt] (-3.4,-3.4) grid[step=0.50] (3.4,3.4);
				\end{scope}

				\foreach \a in {0,6,...,174} {
					\pgfmathsetmacro{\op}{0.03 + 0.0008*\a}
					\begin{scope}[rotate=\a]
						\draw[accentcyan, opacity=\op, line width=1.0pt, samples=240, smooth, domain=-3.2:3.2]
						plot (\x, {0.55*sin(0.95*\x r) + 0.12*sin(2.9*\x r + 20)});
					\end{scope}
				}
				
				% a second, thinner family to create “interference”
				\foreach \a in {3,9,...,177} {
					\pgfmathsetmacro{\op}{0.018 + 0.0006*\a}
					\begin{scope}[rotate=\a]
						\draw[accentblue, opacity=\op, line width=0.9pt, samples=240, smooth, domain=-3.2:3.2]
						plot (\x, {0.48*sin(1.05*\x r + 12) + 0.10*sin(3.2*\x r)});
					\end{scope}
				}

				
				% Interference bands (suggesting noise / averaging)
%				\draw[accentcyan, opacity=0.26, line width=1.1pt, samples=280, smooth, domain=0:360]
%				plot ({2.60*cos(\x)}, {1.15*sin(\x) + 0.42*sin(3*\x)});
%				\draw[accentblue, opacity=0.24, line width=1.1pt, samples=280, smooth, domain=0:360]
%				plot ({1.15*sin(\x) + 0.42*sin(3*\x)}, {2.60*cos(\x)});
%				\draw[white, opacity=0.12, line width=1.0pt, samples=280, smooth, domain=0:360]
%				plot ({2.10*cos(\x)}, {0.85*sin(\x) + 0.34*sin(5*\x)});
%				\draw[white!50!accentblue, opacity=0.12, line width=1.0pt, samples=280, smooth, domain=0:360]
%				plot ({0.85*sin(\x) + 0.34*sin(5*\x)}, {2.10*cos(\x)});
				
				% Central “axioms” (prominent formulas in the core)
				\fill[white, opacity=0.92] (0,0) circle (2.2pt);
				\fill[accentcyan, opacity=0.26] (0,0) circle (0.15);
				\draw[white, opacity=0.15, line width=0.9pt] (0,0) circle (0.32);
				
%				\node[textlight, opacity=0.85, font=\sffamily\bfseries\scriptsize] at (0,0.55)
%				{Hardness};
%				\node[textdim, opacity=0.80, font=\ttfamily\tiny, align=center] at (0,0.18)
%				{$\lambda_1(\Lambda)$,\ \ $\mathrm{CVP}(\Lambda,t)$};
%				\node[textdim, opacity=0.80, font=\ttfamily\tiny, align=center] at (0,-0.05)
%				{$As+e\equiv b\pmod q$};
%				\node[textdim, opacity=0.80, font=\ttfamily\tiny, align=center] at (0,-0.28)
%				{$N=pq\ \ \ \bullet\ \ \ g^x=h$};
				
%				% (E) A faint “equation ring” (text-only, integrated)
%				\draw[white, opacity=0.06, line width=0.9pt] (0,0) circle (1.25);
%				\node[textdim, opacity=0.25, font=\ttfamily\tiny] at (0, 1.33) {$H(m)=H(m')$};
%				\node[textdim, opacity=0.25, font=\ttfamily\tiny] at (1.38, 0.00) {$[x]P=Q$};
%				\node[textdim, opacity=0.25, font=\ttfamily\tiny] at (0,-1.33) {$He^T=s$};
%				\node[textdim, opacity=0.25, font=\ttfamily\tiny] at (-1.48,0.00) {$\varphi:E_1\to E_2$};
				
			\end{scope} % end clip
		
%			\begin{scope}[shift={(0,0)}, scale=3]
%				\draw[white, opacity=0.16, line width=0.9pt] (-1.25,0.40) -- (1.25,0.40);
%				\draw[white, opacity=0.16, line width=0.9pt] (-1.25,0.10) -- (1.25,0.10);
%				\draw[white, opacity=0.16, line width=0.9pt] (-1.25,-0.20) -- (1.25,-0.20);
%				\foreach \xx/\yy in {-0.85/0.40, -0.20/0.10, 0.55/-0.20, 0.95/0.10} {
%					\draw[accentblue, opacity=0.50, line width=0.9pt] (\xx-0.10,\yy-0.10) rectangle (\xx+0.10,\yy+0.10);
%				}
%				\fill[accentcyan, opacity=0.50] (0.15,0.40) circle (0.04);
%				\draw[white, opacity=0.22, line width=0.9pt] (0.15,0.40) -- (0.15,-0.20);
%				\draw[accentcyan, opacity=0.50, line width=0.9pt] (0.15,-0.20) circle (0.08);
%				\draw[accentcyan, opacity=0.50, line width=0.9pt] (0.11,-0.20) -- (0.19,-0.20);
%				\draw[accentcyan, opacity=0.50, line width=0.9pt] (0.15,-0.24) -- (0.15,-0.16);
%			\end{scope}
			
			% --- OUTER ORBIT OF HARD PROBLEMS (equation bubbles; outside clip OK)
			%     This is the primary "embedded" formula layer.
			\def\Rorb{4}
			\draw[white, opacity=0.08, line width=0.9pt] (0,0) circle (\Rorb);
			
			% Connectors (subtle)
			\foreach \ang in {60, 45, 30, 15, 0, 345, 330, 315} {
				\draw[white, opacity=0.10, line width=0.9pt] (0,0) -- ({\Rorb*cos(\ang)},{\Rorb*sin(\ang)});
			}
			
			% Place equation nodes (computed coordinates to avoid pgf "shape" errors)
			\foreach \ang/\lab/\form in {
				60/{Factoring}/{N=pq},
%				40/{RSA}/{x\mapsto x^e \bmod N},
				45/{DLP}/{g^x=h},
%				20/{ECDLP}/{[x]P=Q},
				30/{SVP}/{\lambda_1(\Lambda)=\min_{v\in\Lambda\setminus\{0\}}\|v\|},
				15/{CVP}/{\mathrm{dist}(t,\Lambda)=\min_{v\in\Lambda}\|t-v\|},
				0/{SIS}/{A\mathbf{x}\equiv 0\ (\bmod\ q),\ \|\mathbf{x}\|\ \text{small}},
				345/{LWE}/{As+e\equiv b\ (\bmod\ q)},
				330/{Codes}/{H e^T=s},
				315/{Isogeny}/{\varphi:E_1\to E_2}
			}{
				\pgfmathsetmacro{\xx}{\Rorb*cos(\ang)}
				\pgfmathsetmacro{\yy}{\Rorb*sin(\ang)}
				\EqNode{\xx}{\yy}{\lab}{\form}
			}
			
		\end{scope}
		
		% =========================================================
		% 3. TYPOGRAPHY
		% =========================================================
		\node[anchor=west, align=left]
		at ([xshift=1.00in, yshift=-7.45in]current page.north west)
		{
			\sffamily
			\fontsize{38}{46}\selectfont \bfseries \textcolor{textlight}{Hard Problems in}\\[0.22em]
			\fontsize{38}{46}\selectfont \bfseries \textcolor{accentcyan}{Cryptography}
		};
		
		% Separator Line
		\draw[accentblue, thick]
		([xshift=1.00in, yshift=-8.35in]current page.north west) --
		([xshift=7.85in, yshift=-8.35in]current page.north west);
		
		% Subtitle & Author
		\node[anchor=north west, align=left]
		at ([xshift=1.00in, yshift=-8.55in]current page.north west)
		{
			\sffamily
			\fontsize{18}{22}\selectfont \textcolor{textlight}{\bfseries Classical Assumptions and Post-Quantum Foundations}\\[0.55em]
			\fontsize{13}{17}\selectfont \textcolor{textdim}{Factoring $\cdot$ Discrete Log $\cdot$ Lattices $\cdot$ Codes $\cdot$ Hash-Based $\cdot$ Isogenies}\\[1.10em]
			\fontsize{15}{19}\sffamily\textcolor{white}{\bfseries Ji, Yonghyeon}
		};
		
		% =========================================================
		% 4. FOOTER / BADGE
		% =========================================================
		\node[
		fill=accentblue,
		text=bgbase,
		font=\sffamily\bfseries\small,
		anchor=north east,
		minimum height=3em,
		minimum width=10em
		]
		at ([xshift=0in, yshift=-1.00in]current page.north east)
		{LECTURE NOTES};
		
		\node[textdim, font=\sffamily\footnotesize, anchor=south]
		at ([yshift=0.52in]current page.south)
		{Classical hardness assumptions and post-quantum reductions};
		
	\end{tikzpicture}
	
\end{document}
