\paragraph{Setting (systems of polynomial equations).}
Let $\mathbb{F}_q$ be a finite field. Let $f_1,\dots,f_m\in \mathbb{F}_q[x_1,\dots,x_n]$ be polynomials.

\paragraph{MQ (Multivariate Quadratic) --- search problem.}
Given $m$ quadratic polynomials
\[
f_i(x_1,\dots,x_n) \in \mathbb{F}_q[x_1,\dots,x_n],\quad \deg(f_i)\le 2,
\]
find a solution $x=(x_1,\dots,x_n)\in\mathbb{F}_q^n$ such that
\[
f_i(x)=0 \quad\text{for all } i=1,\dots,m.
\]
(Decision variant: decide whether such an $x$ exists.)

\paragraph{Standard attacks (classical).}
\begin{itemize}\setlength\itemsep{0.2em}
	\item \textbf{Gr\"obner basis} methods: F4/F5; complexity driven by degree of regularity and sparsity.
	\item \textbf{XL / variants} (eXtended Linearization) and relinearization methods.
	\item \textbf{Hybrid attacks}: guess some variables to reduce to smaller systems, then algebraic solve.
	\item \textbf{Rank attacks / MinRank} reductions for structured schemes (oil-vinegar-type, etc.).
	\item \textbf{Linearization traps}: if parameters make the system overdetermined/easy.
\end{itemize}

\paragraph{Quantum note.}
Grover can speed up variable-guessing layers; algebraic-solving quantum speedups are limited and highly model/instance-dependent.
