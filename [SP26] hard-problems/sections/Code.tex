\chapter{Code-Based Hard Problems}

\section{Linear Codes and Syndromes}
Let $\mathcal{C}\subseteq \Ftwo^n$ be a linear $[n,k]$ code with parity-check matrix $H\in\Ftwo^{(n-k)\times n}$.
For $e\in\Ftwo^n$, the \emph{syndrome} is $s = He^\top\in\Ftwo^{n-k}$.
Syndrome decoding asks: given $H$ and $s$, find a low-weight $e$ with syndrome $s$.

\section{Syndrome Decoding (SD)}

\subsection{Formal Statements}
\begin{definition}[Search-SD]
	Given $H\in\Ftwo^{(n-k)\times n}$, a syndrome $s\in\Ftwo^{n-k}$, and an integer weight $w$,
	find $e\in\Ftwo^n$ such that
	\[
	He^\top = s
	\quad\text{and}\quad
	\wt(e)=w.
	\]
\end{definition}

\begin{definition}[Decisional SD (DSD)]
	Distinguish:
	\begin{itemize}[leftmargin=*]
		\item $\calD_0$: $e\from \Ftwo^n$ uniform subject to $\wt(e)=w$, and $s=He^\top$.
		\item $\calD_1$: $s\from \Ftwo^{n-k}$ uniform.
	\end{itemize}
	Given $(H,s)$ output whether $s$ is a syndrome of a weight-$w$ error vector with non-negligible advantage.
\end{definition}

\subsection{Why SD is Hard}
For random $H$, the mapping $e\mapsto He^\top$ is linear and many-to-one. The hardness comes from the combinatorial explosion of possible $e$ of weight $w$:
\[
\#\{e\in\Ftwo^n:\wt(e)=w\} = \binom{n}{w}.
\]
Brute force is exponential in $n$ for typical $w$ scaling.

\subsection{Information Set Decoding (ISD) Family (High-Level)}
The dominant attacks are \emph{information set decoding} and its refinements (Prange, Stern, Dumer, BJMM, and modern variants). The meta-idea:
\begin{itemize}[leftmargin=*]
	\item guess an ``information set'' of coordinates where the error is assumed sparse/structured,
	\item reduce the decoding task to a smaller combinatorial search,
	\item repeat until success with certain probability.
\end{itemize}
Complexities are typically $2^{c n}$ with constant $c$ depending on rate $k/n$ and relative weight $w/n$.

\subsection{Exercises (SD)}
\begin{exercise}[Upper-undergrad: syndrome as coset]
	Fix $H$. Show that the set $\{e\in\Ftwo^n : He^\top = s\}$ is an affine subspace (a coset of $\ker(H)$). What is its size?
\end{exercise}

\begin{exercise}[Masters: counting solutions]
	Assume $H$ is full rank. For random $s$, what is the expected number of solutions $e$ of weight exactly $w$? Express it using $\binom{n}{w}$ and $2^{n-k}$ and justify the approximation.
\end{exercise}

\begin{exercise}[PhD: ISD success probability sketch]
	In Prange's algorithm, one chooses a set $I$ of $k$ positions and hopes the error is zero on $I$. Derive the success probability in terms of $n,k,w$ and the expected work factor.
\end{exercise}

\section{QC Syndrome Decoding (QCSD)}

\subsection{Quasi-Cyclic Structure}
A binary quasi-cyclic (QC) code often uses a parity-check matrix built from circulant blocks.
For block size $p$, a circulant matrix is determined by its first row; multiplication corresponds to polynomial multiplication modulo $x^p-1$.

\subsection{Problem Statement}
\begin{definition}[QCSD / DQCSD]
	Same as SD/DSD, except $H$ is drawn from a QC ensemble (block-circulant structure), and sometimes $e$ is restricted to QC form. Given $(H,s,w)$ find $e$ with $He^\top=s$ and $\wt(e)=w$, or distinguish structured syndromes from uniform.
\end{definition}

\subsection{Security Subtleties}
QC structure reduces public key sizes dramatically but introduces algebraic symmetry. Best practice is to choose parameters so that known structural attacks (e.g., exploiting cyclic shifts, folding, or module-based speedups) do not reduce security below target.

\subsection{Exercises (QCSD)}
\begin{exercise}[Masters: circulant-as-polynomial]
	Show how multiplying a circulant matrix by a vector corresponds to polynomial multiplication modulo $x^p-1$.
\end{exercise}

\begin{exercise}[PhD: symmetry and attack surface]
	Explain how cyclic symmetry can introduce additional low-weight codewords or enable collision-style shortcuts. Give at least one concrete avenue (high-level) by which QC structure can be exploited.
\end{exercise}


%\subsection*{4.1\ Linear codes}
%
%\paragraph{Definition (linear code).}
%A linear $[n,k]_q$ code is a $k$-dimensional $\mathbb{F}_q$-subspace
%$C\subseteq \mathbb{F}_q^n$.
%A \emph{generator matrix} $G\in \mathbb{F}_q^{k\times n}$ satisfies
%\[
%C = \{ uG : u\in \mathbb{F}_q^k\}.
%\]
%A \emph{parity-check matrix} $H\in \mathbb{F}_q^{(n-k)\times n}$ satisfies
%\[
%C = \{ c\in \mathbb{F}_q^n : Hc^\top = 0\}.
%\]
%
%\paragraph{Hamming weight and distance.}
%For $x\in\mathbb{F}_q^n$, the Hamming weight is
%$w_H(x)=|\{i: x_i\neq 0\}|$.
%The Hamming distance is $d_H(x,y)=w_H(x-y)$.
%The minimum distance of $C$ is
%\[
%d(C)=\min\{w_H(c): c\in C\setminus\{0\}\}.
%\]
%
%\subsection*{4.2\ Syndrome Decoding (SD)}
%
%\paragraph{Search SD.}
%Given $H\in \mathbb{F}_q^{(n-k)\times n}$, a syndrome $s\in \mathbb{F}_q^{n-k}$,
%and an integer $t$, find $e\in \mathbb{F}_q^n$ such that
%\[
%He^\top = s
%\quad\text{and}\quad
%w_H(e)\le t.
%\]
%(Over $\mathbb{F}_2$, this is the classical binary syndrome decoding problem.)
%
%\paragraph{Decision SD.}
%Given $(H,s,t)$, decide whether there exists $e$ with $He^\top=s$ and $w_H(e)\le t$.
%
%\subsection*{4.3\ Codeword Finding / Minimum Distance Problem}
%
%\paragraph{Minimum distance (decision form).}
%Given a linear code $C$ (via $G$ or $H$) and integer $t$, decide whether
%\[
%\exists\, c\in C\setminus\{0\}\ \text{such that}\ w_H(c)\le t.
%\]
%The corresponding search problem asks to find such a codeword.









%\paragraph{Linear code.}
%A linear $[n,k]_q$ code is a $k$-dimensional subspace $C\subseteq\mathbb{F}_q^n$.
%A parity-check matrix $H\in\mathbb{F}_q^{(n-k)\times n}$ satisfies
%\[
%C=\{c\in\mathbb{F}_q^n: Hc^\top=0\}.
%\]
%Hamming weight: $w_H(x)=|\{i:x_i\neq 0\}|$.
%
%\paragraph{Syndrome Decoding (SD) --- search form.}
%Given $H\in\mathbb{F}_q^{(n-k)\times n}$, syndrome $s\in\mathbb{F}_q^{(n-k)}$, and bound $t$,
%find $e\in\mathbb{F}_q^n$ such that
%\[
%He^\top=s \quad\text{and}\quad w_H(e)\le t.
%\]
%
%\paragraph{Minimum Distance Problem (MDP) --- decision form.}
%Given a linear code $C$ and integer $t$, decide whether
%\[
%\exists\,c\in C\setminus\{0\}\ \text{with}\ w_H(c)\le t.
%\]
%
%\paragraph{Standard attacks (classical).}
%\begin{itemize}\setlength\itemsep{0.2em}
%	\item \textbf{Information Set Decoding (ISD)} family: Prange, Stern, Dumer, BJMM-style improvements.
%	\item \textbf{Structural attacks}: if the public code is not pseudorandom (hidden algebraic structure, rank defects, etc.).
%	\item \textbf{Side-channel/implementation}: leakage from decoding routines or masking failures.
%\end{itemize}
%
%\paragraph{Quantum note.}
%Grover-type search can improve brute-force components; quantum ISD analyses give model-dependent exponent reductions.
%
