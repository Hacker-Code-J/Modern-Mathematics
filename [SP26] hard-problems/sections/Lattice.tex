\chapter{Lattice-Based Hard Problems}

\section{Background: Lattices, Duality, and Problems}
A (full-rank) lattice $\Lambda \subset \mathbb{R}^n$ is $\Lambda = \{Bz : z\in\Z^n\}$ for some basis matrix $B\in\mathbb{R}^{n\times n}$.

Classic algorithmic problems:
\begin{itemize}[leftmargin=*]
	\item \textbf{SVP (Shortest Vector Problem):} Find $0\neq v\in\Lambda$ minimizing $\norm{v}_2$.
	\item \textbf{GapSVP (Decision/SVP approximation):} Given $(\Lambda, d)$ decide whether $\lambda_1(\Lambda)\le d$ or $\lambda_1(\Lambda)> \gamma d$.
	\item \textbf{SIVP (Shortest Independent Vectors):} Find $n$ linearly independent vectors of length $\le \gamma\cdot \lambda_n(\Lambda)$.
\end{itemize}
The importance for cryptography: average-case problems (LWE/SIS) reduce from worst-case lattice problems (GapSVP/SIVP) under suitable parameters.

\section{Learning With Errors (LWE)}

\subsection{Formal Statements}
\begin{definition}[Search-LWE]
	Fix integers $n,m,q\in\mathbb{N}$ and an error distribution $\chi$ over $\Z$ (typically supported on small integers).
	Sample $A\from \Z_q^{m\times n}$ uniformly, secret $s\from \Z_q^n$ (usually uniform), and error $e\from \chi^m$.
	Given $(A,b)$ where
	\[
	b = As + e \bmod q \in \Z_q^m,
	\]
	output the secret $s$ (or equivalently recover $e$).
\end{definition}

\begin{definition}[Decision-LWE]
	Under the same parameterization, consider two distributions over $(A,b)$:
	\[
	\calD_0: (A, As+e \bmod q), \qquad
	\calD_1: (A, u),\ \ u\from \Z_q^m.
	\]
	Given $(A,b)$, output a bit indicating whether $(A,b)\from\calD_0$ or $(A,b)\from\calD_1$ with non-negligible advantage.
\end{definition}

\subsection{Geometric / Statistical Intuition}
Each equation is:
\[
\langle a_i, s \rangle + e_i \equiv b_i \pmod q.
\]
If errors were $0$, this is solving a linear system over $\Z_q$. Errors make it an instance of \emph{noisy linear equations}, and (crucially) hide $s$.

A typical heuristic: if $e$ is small in $\Z$ and $q$ is large, the mapping $s\mapsto As+e$ looks like ``random'' without knowing $s$, but still allows decryption by rounding in cryptosystems.

\subsection{Decision vs Search; Standard Relationships}
Cryptographic constructions often assume decision-LWE hardness (for pseudorandomness) and search-LWE hardness (for extracting secrets). Under many standard parameter regimes, one can relate them:

\begin{remark}[Search-to-decision (informal)]
	For prime $q$, there are classical reductions showing decision-LWE is no harder than search-LWE and vice versa (up to losses), under mild conditions. Intuitively, if you can recover $s$ then you can distinguish; conversely, if you can distinguish, you can often recover $s$ coordinate-by-coordinate using hybrid and rerandomization tricks.
\end{remark}

\subsection{Worst-Case to Average-Case Reductions (High-Level)}
A landmark result (Regev-style) shows that (for appropriate $\alpha$ where errors have size about $\alpha q$) decision-LWE is at least as hard as approximating worst-case lattice problems (GapSVP/SIVP) in dimension $n$ within poly factors. The technical conditions tie $\alpha$, $q$, and $n$.

\begin{remark}[What you should remember]
	The security story is: \emph{if LWE is easy on average, then certain canonical lattice problems are easy in the worst case}. This is why LWE is a central conservative assumption.
\end{remark}

\subsection{Parameter Regimes (Conceptual)}
Let error magnitude scale be $\sigma$ (e.g., standard deviation for discrete Gaussian). Often one defines $\alpha = \sigma/q$.
\begin{itemize}[leftmargin=*]
	\item \textbf{Correctness in encryption}: needs $\sigma \ll q$ so small errors can be rounded.
	\item \textbf{Security}: needs $\sigma$ large enough that $As+e$ hides $s$; also $m$ large enough to prevent solving.
	\item \textbf{Typical cryptosystems}: use $m\approx n\log q$ or $m$ a constant multiple of $n$ in module/ring variants.
\end{itemize}

\subsection{Attack Taxonomy (What to Teach)}
Best-known attacks broadly fall into:
\begin{itemize}[leftmargin=*]
	\item \textbf{Lattice reduction (primal):} view LWE as finding a close vector / BDD instance; build a lattice from $A$ and $b$ and run BKZ-type reduction; recover $s$ by nearest-plane or enumeration.
	\item \textbf{Dual attack:} find a short vector $y$ in the dual lattice such that $y^\top A \equiv 0$ mod $q$, then test $y^\top b$ for smallness vs uniform.
	\item \textbf{BKW / combinatorial:} reduce dimension via collision-finding on $A$ rows; grows fast with $q$ and noise but can matter for small moduli.
	\item \textbf{Arora--Ge (algebraic):} for very small error alphabets and special parameter settings, solve polynomial system.
\end{itemize}

\subsection{Exercises (LWE)}
\begin{exercise}[Upper-undergrad: noiseless baseline]
	Assume $e=0$ and $m\ge n$. Show how to recover $s$ efficiently from $(A,As\bmod q)$ when $q$ is prime and $A$ has full rank.
\end{exercise}

\begin{exercise}[Masters: distinguishing via dual vector]
	Let $y\in \Z^m$ satisfy $y^\top A\equiv 0\pmod q$. Show that if $(A,b)\from \calD_0$ then
	\[
	y^\top b \equiv y^\top e \pmod q,
	\]
	and argue heuristically why $y^\top b$ is statistically closer to small integers mod $q$ than uniform if $y$ is short.
\end{exercise}

\begin{exercise}[PhD: hybrid for search-to-decision sketch]
	Assume $q$ prime. Outline a reduction strategy that recovers $s_i$ (the $i$-th coordinate of $s$) using a decision oracle by embedding a guess into the distribution and using hybrids.
\end{exercise}

\section{Short Integer Solution (SIS)}

\subsection{Formal Statement}
\begin{definition}[Search-SIS]
	Let $q\in\mathbb{N}$, $n,m\in\mathbb{N}$, and bound $\beta\in\mathbb{N}$.
	Sample $A\from \Z_q^{n\times m}$ uniformly.
	Find a nonzero vector $x\in \Z^m\setminus\{0\}$ such that
	\[
	Ax \equiv 0 \pmod q
	\quad\text{and}\quad
	\norm{x}\le \beta,
	\]
	where $\norm{\cdot}$ is typically $\ell_2$ or $\ell_\infty$.
\end{definition}

\subsection{Interpretation as Finding Short Relations}
The condition $Ax\equiv 0\pmod q$ means $x$ is an integer relation among the columns of $A$ modulo $q$. Without the shortness constraint, there are many solutions. The hardness is to find a \emph{short} one.

\subsection{Connection to Hash-and-Sign / Commitments}
SIS underlies:
\begin{itemize}[leftmargin=*]
	\item lattice-based hash functions: mapping $x\mapsto Ax\bmod q$; collisions correspond to short $x$ with $Ax\equiv 0$.
	\item commitments: binding reduces to SIS.
	\item signatures (e.g., GPV-style): produce short preimages under a public linear map.
\end{itemize}

\subsection{Worst-Case Reductions (High-Level)}
SIS is related to worst-case lattice problems as well: if SIS is easy for certain $(n,m,q,\beta)$, then approximating certain lattice problems is easy in the worst case. The parameter tradeoff differs from LWE.

\subsection{Attack Taxonomy}
\begin{itemize}[leftmargin=*]
	\item \textbf{Lattice reduction:} interpret SIS as finding a short vector in a lattice of solutions; build lattice basis and run BKZ.
	\item \textbf{Combinatorial / meet-in-the-middle:} sometimes applicable for $\ell_\infty$ and special constraints (rare in standard parameters).
\end{itemize}

\subsection{Exercises (SIS)}
\begin{exercise}[Upper-undergrad: pigeonhole existence]
	Let $A\in\Z_q^{n\times m}$ and consider all $x\in\{0,1\}^m$. Show that if $2^m > q^n$, then there exist distinct $x\neq x'$ with $Ax\equiv Ax'\pmod q$. Deduce existence of a nonzero $\{-1,0,1\}^m$ solution to $A(x-x')\equiv 0$.
\end{exercise}

\begin{exercise}[Masters: collision-resistance from SIS]
	Define $H(x)=Ax\bmod q$ for short $x$ in some domain. Formalize how a collision $(x\neq x')$ yields an SIS solution.
\end{exercise}

\begin{exercise}[PhD: parameter reasoning]
	For fixed $n,q$, explain qualitatively why increasing $m$ makes SIS \emph{easier} (more relations exist), but also allows setting smaller $\beta$ while maintaining existence of solutions.
\end{exercise}

\section{NTRU Search Problem}

\subsection{Ring Setting}
Let $f(x)$ be a cyclotomic-like polynomial (e.g., $x^N+1$ with $N$ power of 2, or $x^N-1$ for classical NTRU variants). Define
\[
R = \Z[x]/(f(x)),\qquad
R_q = R / qR \cong \Z_q[x]/(f(x)).
\]
Elements are represented by degree-$<N$ polynomials; ``small'' typically refers to small coefficients.

\subsection{Formal Problem (One Common Form)}
\begin{definition}[NTRU Search (informal canonical form)]
	Sample $f,g \in R$ from a ``small'' distribution such that $f$ is invertible in $R_q$.
	Publish
	\[
	h \equiv g f^{-1} \pmod q \in R_q.
	\]
	Given $h$, recover a short pair $(f,g)$ (or an equivalent short representation) satisfying $h f \equiv g \pmod q$ under the same smallness constraints.
\end{definition}

\subsection{Key Ambiguities in NTRU Statements}
NTRU has many instantiations; the exact hardness depends on:
\begin{itemize}[leftmargin=*]
	\item ring choice ($x^N\pm 1$; $N$ prime; etc.),
	\item modulus structure (prime $q$ vs power-of-two),
	\item distribution of $(f,g)$ (ternary, Gaussian, centered binomial),
	\item norm and acceptance region (e.g., $\ell_\infty$ bounds on coefficients),
	\item whether we are in \emph{ring} vs \emph{module} setting.
\end{itemize}

\subsection{NTRU as a Lattice Problem}
Given $h$, consider the \emph{NTRU lattice}:
\[
\Lambda_h = \left\{(u,v)\in R^2 : u - hv \equiv 0 \pmod q \right\}.
\]
A secret key corresponds to a short vector $(g,f)\in \Lambda_h$. Thus, breaking NTRU is (at high level) a shortest-vector style task in a structured lattice.

\subsection{Attack Taxonomy}
\begin{itemize}[leftmargin=*]
	\item \textbf{Lattice reduction on NTRU lattice:} embed $\Lambda_h$ into an integer lattice of dimension $2N$ and use BKZ; recover short $(f,g)$.
	\item \textbf{Hybrid attacks:} partial guessing of coefficients + lattice reduction for the remaining.
	\item \textbf{Subfield / algebraic structure attacks:} exploit ring structure if parameters are ill-chosen (historically important lesson: structure can leak).
\end{itemize}

\subsection{Exercises (NTRU)}
\begin{exercise}[Upper-undergrad: derive the NTRU relation]
	Show that $h \equiv gf^{-1}\pmod q$ implies $hf\equiv g\pmod q$. Explain why small $(f,g)$ is a ``short relation'' between $1$ and $h$.
\end{exercise}

\begin{exercise}[Masters: NTRU lattice membership]
	Define $\Lambda_h$ as above. Prove that $(g,f)\in \Lambda_h$. What other pairs are in $\Lambda_h$? Characterize them modulo $q$.
\end{exercise}

\begin{exercise}[PhD: compare NTRU vs LWE intuition]
	Give a conceptual comparison: NTRU keys correspond to short vectors in a structured lattice tied to one public ring element; LWE hides a secret with additive noise across many samples. Discuss how this affects the style of security reductions and the known attacks.
\end{exercise}






%\subsection*{3.1\ Lattices}
%
%\paragraph{Definition (lattice).}
%A (full-rank) lattice $\mathcal{L}\subset \mathbb{R}^n$ is a discrete additive subgroup of $\mathbb{R}^n$.
%Equivalently, given linearly independent vectors $b_1,\dots,b_n\in \mathbb{R}^n$, the set
%\[
%\mathcal{L}(B)=\left\{ \sum_{i=1}^n z_i b_i \ :\ z_i\in \mathbb{Z} \right\}
%\]
%is a lattice; $B=[b_1|\cdots|b_n]$ is a basis of $\mathcal{L}$.
%The \emph{determinant} is $\det(\mathcal{L}) = |\det(B)|$ (basis-independent).
%
%\paragraph{Norm.}
%Typically $\|\cdot\|$ denotes the Euclidean norm $\|\cdot\|_2$, unless otherwise stated.
%
%\subsection*{3.2\ Shortest Vector Problem (SVP)}
%
%\paragraph{Search SVP.}
%Given a basis $B$ of a lattice $\mathcal{L}=\mathcal{L}(B)$, find a nonzero vector
%$v\in \mathcal{L}\setminus\{0\}$ minimizing $\|v\|$, i.e.
%\[
%\|v\| = \lambda_1(\mathcal{L}) \;:=\; \min\{\|x\|: x\in \mathcal{L}\setminus\{0\}\}.
%\]
%
%\paragraph{Approximate SVP ($\gamma$-SVP).}
%Given $B$ and an approximation factor $\gamma=\gamma(n)\ge 1$, output
%$v\in \mathcal{L}\setminus\{0\}$ such that
%\[
%\|v\|\le \gamma\cdot \lambda_1(\mathcal{L}).
%\]
%
%\subsection*{3.3\ Closest Vector Problem (CVP)}
%
%\paragraph{Search CVP.}
%Given a basis $B$ of $\mathcal{L}=\mathcal{L}(B)$ and a target $t\in \mathbb{R}^n$,
%find $v\in \mathcal{L}$ minimizing $\|t-v\|$, i.e.
%\[
%\|t-v\| = \mathrm{dist}(t,\mathcal{L}) \;:=\; \min\{\|t-x\|: x\in \mathcal{L}\}.
%\]
%
%\paragraph{Approximate CVP ($\gamma$-CVP).}
%Output $v\in\mathcal{L}$ with $\|t-v\|\le \gamma\cdot \mathrm{dist}(t,\mathcal{L})$.
%
%\subsection*{3.4\ Learning With Errors (LWE) (cryptographic canonical lattice problem)}
%
%\paragraph{Decision LWE.}
%Fix parameters $n\in\mathbb{N}$, modulus $q\ge 2$, and an error distribution $\chi$ over $\mathbb{Z}_q$.
%Given $m$ samples $(a_i,b_i)\in \mathbb{Z}_q^n\times \mathbb{Z}_q$, distinguish between:
%\begin{align*}
%	\text{(LWE)}\quad & a_i \xleftarrow{\$} \mathbb{Z}_q^n,\;\; s \xleftarrow{\$} \mathbb{Z}_q^n,\;\;
%	e_i \xleftarrow{\$}\chi,\;\; b_i = \langle a_i,s\rangle + e_i \pmod q;\\
%	\text{(Uniform)}\quad & (a_i,b_i) \xleftarrow{\$} \mathbb{Z}_q^n\times \mathbb{Z}_q,
%\end{align*}
%where $\langle\cdot,\cdot\rangle$ is the standard inner product modulo $q$.
%The \emph{search} version asks to recover $s$ from LWE samples.








%\subsection*{3.1 Lattices}
%
%\paragraph{Definition (lattice).}
%A full-rank lattice $\mathcal{L}\subset \mathbb{R}^n$ is a discrete subgroup:
%given linearly independent $b_1,\dots,b_n\in\mathbb{R}^n$,
%\[
%\mathcal{L}=\mathcal{L}(B)=\left\{\sum_{i=1}^n z_i b_i : z_i\in\mathbb{Z}\right\},\quad
%B=[b_1|\cdots|b_n].
%\]
%Define the first successive minimum
%\[
%\lambda_1(\mathcal{L})=\min\{\|v\|_2: v\in \mathcal{L}\setminus\{0\}\}.
%\]
%
%\subsection*{3.2 SVP and CVP}
%
%\paragraph{SVP (Shortest Vector Problem).}
%Given a basis $B$ of $\mathcal{L}$, find $v\in\mathcal{L}\setminus\{0\}$ such that $\|v\|_2=\lambda_1(\mathcal{L})$.
%
%\paragraph{$\gamma$-SVP (Approximate SVP).}
%Given $B$ and $\gamma\ge 1$, find $v\in\mathcal{L}\setminus\{0\}$ with
%\[
%\|v\|_2 \le \gamma\cdot \lambda_1(\mathcal{L}).
%\]
%
%\paragraph{CVP (Closest Vector Problem).}
%Given $B$ and a target $t\in\mathbb{R}^n$, find $v\in\mathcal{L}$ minimizing $\|t-v\|_2$.
%
%\subsection*{3.3 LWE (Learning With Errors)}
%
%\paragraph{Decision-LWE.}
%Fix $n,q\ge 2$ and an error distribution $\chi$ over $\mathbb{Z}_q$.
%Given $m$ samples $(a_i,b_i)\in \mathbb{Z}_q^n\times\mathbb{Z}_q$, distinguish:
%\begin{align*}
%	\text{(LWE)}\quad & a_i \xleftarrow{\$} \mathbb{Z}_q^n,\; s \xleftarrow{\$} \mathbb{Z}_q^n,\;
%	e_i\xleftarrow{\$}\chi,\; b_i=\langle a_i,s\rangle+e_i \pmod q;\\
%	\text{(Uniform)}\quad & (a_i,b_i)\xleftarrow{\$}\mathbb{Z}_q^n\times\mathbb{Z}_q.
%\end{align*}
%The \emph{search} version asks to recover $s$.
%
%\paragraph{Standard attacks (classical).}
%\begin{itemize}\setlength\itemsep{0.2em}
%	\item \textbf{LLL/BKZ reduction} as the main engine (produces short/near-short vectors).
%	\item \textbf{Primal attacks} (embedding to SVP/CVP) + enumeration/sieving.
%	\item \textbf{Dual attacks} (find short dual vectors to distinguish LWE from uniform).
%	\item \textbf{Hybrid attacks} (guess part of secret + lattice reduction).
%	\item \textbf{BKW / combinatorial} attacks (parameter-dependent).
%	\item \textbf{Algebraic attacks} in certain regimes (e.g.\ over-defined systems).
%\end{itemize}
%
%\paragraph{Quantum note.}
%No known Shor-like polynomial-time algorithm for general lattice problems; quantum speedups mainly
%affect search/sieving constants/exponents.
