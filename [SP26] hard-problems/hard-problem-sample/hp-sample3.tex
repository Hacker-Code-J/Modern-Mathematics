% ============================================================
% TEACHING PACKET (7 weeks)
% - Slides-outline (per lecture)
% - Lecture scripts (instructor notes)
% - Recitation worksheets (student-facing)
% - Homework + solution sketches
%
% Audience: mathematicians; cryptography hard problems + attacks
% ============================================================
\documentclass[11pt]{article}

\usepackage[a4paper,margin=1in]{geometry}
\usepackage{amsmath,amssymb,amsthm,mathtools}
\usepackage{hyperref}
\usepackage{enumitem}
\usepackage{array}
\usepackage{longtable}
\usepackage{multirow}
\usepackage{bm}

% ------------------ Environments ------------------
\newtheorem{definition}{Definition}[section]
\newtheorem{theorem}[definition]{Theorem}
\newtheorem{lemma}[definition]{Lemma}
\newtheorem{proposition}[definition]{Proposition}
\newtheorem{corollary}[definition]{Corollary}
\newtheorem{remark}[definition]{Remark}
\newtheorem{example}[definition]{Example}
\newtheorem{exercise}{Exercise}[section]

% ------------------ Notation ------------------
\newcommand{\Z}{\mathbb{Z}}
\newcommand{\R}{\mathbb{R}}
\newcommand{\F}{\mathbb{F}}
\newcommand{\bits}{\{0,1\}}
\newcommand{\negl}{\mathrm{negl}}
\newcommand{\poly}{\mathrm{poly}}
\newcommand{\getsR}{\xleftarrow{\$}}
\newcommand{\norm}[1]{\left\lVert#1\right\rVert}
\newcommand{\ip}[2]{\left\langle #1,#2\right\rangle}
\newcommand{\Lnot}[3]{L_{#1}\!\left[#2,#3\right]}

\newcommand{\Slide}[1]{\paragraph{\textbf{Slide:} #1}}
\newcommand{\Instructor}[1]{\paragraph{\textbf{Instructor notes.}} #1}
\newcommand{\Worksheet}[1]{\paragraph{\textbf{Worksheet.}} #1}
\newcommand{\HW}[1]{\paragraph{\textbf{Homework.}} #1}
\newcommand{\Sol}[1]{\paragraph{\textbf{Solution sketch.}} #1}

\title{Teaching Packet: Hard Problems in Cryptography (7 Weeks)}
\author{}
\date{}

\begin{document}
	\maketitle
	\tableofcontents
	
	% ============================================================
	\section{How to Use This Packet}
	% ============================================================
	
	\begin{itemize}[leftmargin=2em]
		\item Each week contains \textbf{3 lectures} (A/B/C). For each lecture:
		\begin{itemize}
			\item \textbf{Slides-outline}: bullet-point list suitable for beamer slides.
			\item \textbf{Instructor notes}: a script-like narrative + emphasis points.
			\item \textbf{Recitation worksheet}: student-facing problems for a 50--90 minute session.
		\end{itemize}
		\item Each week ends with \textbf{homework} plus \textbf{solution sketches} (not fully worked, but enough to grade).
		\item Notation is consistent across topics; assumptions are made explicit.
	\end{itemize}
	
	% ============================================================
	\section{Global Preliminaries (Week 0 / Lecture 0)}
	% ============================================================
	
	\subsection*{Slides-outline}
	\Slide{Course framing}
	\begin{itemize}[leftmargin=2em]
		\item ``Hard problem'' = conjectured infeasible for PPT adversary at chosen security parameter $\lambda$.
		\item Distinguish \emph{mathematical} hardness vs \emph{implementation} failures.
		\item Families: factoring, discrete log, lattices, codes, isogenies, MQ, hash.
	\end{itemize}
	
	\Slide{Complexity language}
	\begin{itemize}[leftmargin=2em]
		\item Negligible $\negl(\lambda)$; polynomial $\poly(\lambda)$; security parameter $\lambda$.
		\item Subexponential $L$-notation: $L_N[\alpha,c]=\exp((c+o(1))(\log N)^\alpha(\log\log N)^{1-\alpha})$.
		\item Search vs decision vs distinguishing formulations.
	\end{itemize}
	
	\subsection*{Instructor notes}
	\Instructor{
		Set expectations: we care about \emph{best known attacks}, not absolute impossibility.
		Emphasize that a scheme can be broken even if the underlying ``family'' remains plausible (e.g.\ SIDH/SIKE).
		Explain why mathematicians like formal problem definitions and reductions, while cryptanalysts speak in attack taxonomies.
	}
	
	\subsection*{Worksheet}
	\Worksheet{
		\begin{enumerate}[leftmargin=2em]
			\item Give one example each of search, decision, distinguishing.
			\item Show that if $p,q$ are primes and you know $N=pq$ and $p+q$, then you can recover $p,q$.
			\item (Short) Estimate collision probability after $q$ hashes into $n$ bits using the birthday heuristic.
		\end{enumerate}
	}
	
	% ============================================================
	\section{Week 1: Integer Factorization (RSA/Rabin)}
	% ============================================================
	
	\subsection{Lecture 1A: Definitions and Reductions}
	
	\subsubsection*{Slides-outline}
	\Slide{Problem statement}
	\begin{itemize}[leftmargin=2em]
		\item Factoring (search): given composite $N$, output nontrivial divisor.
		\item RSA distribution: $N=pq$ with $p,q$ random $\lambda$-bit primes.
	\end{itemize}
	
	\Slide{Reductions used in crypto}
	\begin{itemize}[leftmargin=2em]
		\item Knowing $\varphi(N)$ factors semiprimes.
		\item Order-finding $\Rightarrow$ factoring (random $a$).
		\item Rabin inversion $\Rightarrow$ factoring.
	\end{itemize}
	
	\subsubsection*{Instructor notes}
	\Instructor{
		Do not overclaim ``RSA $\Leftrightarrow$ factoring''; explain the nuance:
		RSA inversion is \emph{believed} equivalent to factoring but not proved in general.
		However, Rabin inversion is provably as hard as factoring for Blum integers.
		Use the $p+q$ trick to show $\varphi(N)$ is enough.
	}
	
	\subsubsection*{Worksheet}
	\Worksheet{
		\begin{enumerate}[leftmargin=2em]
			\item Prove: if $N=pq$ and $\varphi(N)$ is known, then $p,q$ can be recovered.
			\item Show: if you can compute $\lambda(N)$ (Carmichael), you can factor $N=pq$.
			\item For $N=77$, compute $\varphi(N)$ and list $(\Z/N\Z)^\times$ orders for $a\in\{2,3,5,6\}$.
		\end{enumerate}
	}
	
	\subsection{Lecture 1B: Classical Attacks (ECM, QS, GNFS)}
	
	\subsubsection*{Slides-outline}
	\Slide{Landscape}
	\begin{itemize}[leftmargin=2em]
		\item ``Small factor'' methods: trial division, Pollard $\rho$, Pollard $p-1$, ECM.
		\item ``Sieve'' methods: QS ($L_N[1/2,1]$), GNFS ($L_N[1/3,(64/9)^{1/3}]$).
	\end{itemize}
	
	\Slide{ECM intuition}
	\begin{itemize}[leftmargin=2em]
		\item Replace $a^{M}\bmod p$ smoothness with elliptic curve group order smoothness.
		\item Expected time depends on size of smallest prime factor.
	\end{itemize}
	
	\Slide{QS/GNFS at 30,000 feet}
	\begin{itemize}[leftmargin=2em]
		\item Collect relations $\Rightarrow$ sparse linear algebra over $\F_2$.
		\item Square root step produces congruence of squares.
	\end{itemize}
	
	\subsubsection*{Instructor notes}
	\Instructor{
		Keep QS/GNFS black-box but conceptually correct: relations, smoothness probability,
		linear algebra in exponent vectors mod 2. For mathematicians: relate to ideal factorization
		language (NFS) without drowning in details.
	}
	
	\subsubsection*{Worksheet}
	\Worksheet{
		\begin{enumerate}[leftmargin=2em]
			\item Explain why QS needs linear algebra over $\F_2$.
			\item Run a toy QS by hand for $N=77$: try $x^2-N$ for several $x$ and look for squares/smooth values.
			\item Compare Pollard $\rho$ expected time for a 20-bit factor vs a 40-bit factor (order-of-magnitude).
		\end{enumerate}
	}
	
	\subsection{Lecture 1C: Quantum Factoring (Shor) at Concept Level}
	
	\subsubsection*{Slides-outline}
	\Slide{Reduction}
	\begin{itemize}[leftmargin=2em]
		\item Factoring $\rightarrow$ order-finding in $(\Z/N\Z)^\times$.
		\item Order-finding via period finding for $f(x)=a^x \bmod N$.
	\end{itemize}
	
	\Slide{QFT intuition}
	\begin{itemize}[leftmargin=2em]
		\item Fourier sampling reveals period $r$ with high probability.
		\item Classical post-processing: if $r$ even, use $\gcd(a^{r/2}\pm 1,N)$.
	\end{itemize}
	
	\subsubsection*{Instructor notes}
	\Instructor{
		Avoid full quantum circuit details; emphasize the mathematical structure:
		hidden periodicity and Fourier analysis on cyclic groups. Mention that asymptotically
		it is polynomial in $\log N$ but requires fault-tolerant qubits.
	}
	
	\subsubsection*{Worksheet}
	\Worksheet{
		\begin{enumerate}[leftmargin=2em]
			\item Prove: if $r=\mathrm{ord}_N(a)$ is even and $a^{r/2}\not\equiv -1\pmod N$, then $\gcd(a^{r/2}-1,N)$ yields a nontrivial factor.
			\item Compute order of $a=2$ modulo $N=15$ and recover factors using the above step.
		\end{enumerate}
	}
	
	\subsection{Week 1 Homework + solution sketches}
	
	\HW{
		\begin{enumerate}[leftmargin=2em]
			\item (Reduction) Prove $\varphi(N)$ factors semiprimes; implement in pseudocode.
			\item (Attack taxonomy) For each of Pollard $\rho$, ECM, QS, GNFS: state what property makes it effective and what input sizes it targets.
			\item (Order-finding) Show how order-finding implies factoring for random $a$ (state probability assumptions clearly).
		\end{enumerate}
	}
	\Sol{
		(1) Use $p+q=N-\varphi(N)+1$ and solve quadratic.  
		(2) Pollard $\rho$/ECM: small factors; QS: mid-size; GNFS: largest general.  
		(3) Standard argument: random $a$ has even order with decent probability; if $a^{r/2}\neq -1$ mod $N$ then gcd gives factor.
	}
	
	% ============================================================
	\section{Week 2: Discrete Logarithms (Finite Fields \& Elliptic Curves)}
	% ============================================================
	
	\subsection{Lecture 2A: DLP/CDH/DDH and Generic Algorithms}
	
	\subsubsection*{Slides-outline}
	\Slide{Definitions}
	\begin{itemize}[leftmargin=2em]
		\item DLP: given $g,h$, find $x$ with $g^x=h$ in cyclic group $G$ of order $n$.
		\item CDH/DDH: compute $g^{ab}$ / distinguish $g^{ab}$ from random.
	\end{itemize}
	
	\Slide{Generic algorithms}
	\begin{itemize}[leftmargin=2em]
		\item Baby-step/giant-step: $\tilde O(\sqrt{n})$ time+memory.
		\item Pollard $\rho$: $\tilde O(\sqrt{n})$ time, low memory.
		\item Generic lower bound idea: need $\Omega(\sqrt{n})$ in black-box groups.
	\end{itemize}
	
	\subsubsection*{Instructor notes}
	\Instructor{
		Drive home: in \emph{generic} groups ECDLP is not easier than $\sqrt{n}$.
		Hence curves choose $n\approx 2^{256}$ for 128-bit classical security.
		Explain random-walk collision philosophy.
	}
	
	\subsubsection*{Worksheet}
	\Worksheet{
		\begin{enumerate}[leftmargin=2em]
			\item Work baby-step/giant-step on $\Z_{29}^\times$ with generator $g=2$, target $h=18$.
			\item Explain why Pollard $\rho$ is a collision-finding algorithm on a pseudorandom map.
		\end{enumerate}
	}
	
	\subsection{Lecture 2B: Pohlig--Hellman and Subgroup Attacks}
	
	\subsubsection*{Slides-outline}
	\Slide{Pohlig--Hellman}
	\begin{itemize}[leftmargin=2em]
		\item If $n=\prod p_i^{e_i}$ then DLP reduces to each prime power.
		\item Solve residues, combine via CRT.
		\item Implication: choose prime-order subgroup (or with one large prime factor).
	\end{itemize}
	
	\subsubsection*{Instructor notes}
	\Instructor{
		Provide a worked example with $n$ having small factors.
		Emphasize that many protocol failures come from wrong subgroup choice or missing validation.
	}
	
	\subsubsection*{Worksheet}
	\Worksheet{
		\begin{enumerate}[leftmargin=2em]
			\item Do Pohlig--Hellman in a toy group where $n=2^2\cdot 3\cdot 5$.
			\item Explain what can go wrong in Diffie--Hellman if group membership is not validated.
		\end{enumerate}
	}
	
	\subsection{Lecture 2C: Index Calculus vs ECDLP, Pairing Reductions, Shor}
	
	\subsubsection*{Slides-outline}
	\Slide{Finite-field DLP}
	\begin{itemize}[leftmargin=2em]
		\item Index calculus: factor base, relations, linear algebra, individual logs.
		\item Best-known in prime fields: NFS-DL ($L_p[1/3,(64/9)^{1/3}]$).
	\end{itemize}
	
	\Slide{ECDLP}
	\begin{itemize}[leftmargin=2em]
		\item Generic attacks dominate for well-chosen curves: $\tilde O(\sqrt{n})$.
		\item MOV/Frey--R\"uck: special curves reduce to finite-field DLP via pairings.
	\end{itemize}
	
	\Slide{Quantum}
	\begin{itemize}[leftmargin=2em]
		\item Shor solves DLP in abelian groups in $\poly(\log n)$ time.
	\end{itemize}
	
	\subsubsection*{Instructor notes}
	\Instructor{
		Stress that “ECDLP is harder” is conditional: it avoids known index-calculus subexponential methods.
		But special curves (supersingular / small embedding degree) can invalidate this.
	}
	
	\subsection{Week 2 Homework + solution sketches}
	
	\HW{
		\begin{enumerate}[leftmargin=2em]
			\item Prove correctness of Pohlig--Hellman and give runtime in terms of factorization of $n$.
			\item Compare DLP hardness in $\F_p^\times$ vs elliptic curves of comparable size; justify using attack classes.
			\item Show DDH $\Rightarrow$ IND-CPA security of ElGamal (standard reduction outline).
		\end{enumerate}
	}
	\Sol{
		(1) Use lifting to prime powers + CRT.  
		(2) Finite fields admit index calculus; generic for EC.  
		(3) Hybrid argument: replace $g^{ab}$ with random if DDH hard.
	}
	
	% ============================================================
	\section{Week 3: Lattices (SVP/CVP, SIS/LWE) and Cryptanalysis Toolkit}
	% ============================================================
	
	\subsection{Lecture 3A: Geometry of Numbers Essentials}
	
	\subsubsection*{Slides-outline}
	\Slide{Lattices}
	\begin{itemize}[leftmargin=2em]
		\item $\mathcal{L}(B)=\{Bz:z\in\Z^d\}$, determinant/covolume, dual lattice.
		\item Successive minima $\lambda_1,\lambda_2,\dots$.
	\end{itemize}
	
	\Slide{Minkowski}
	\begin{itemize}[leftmargin=2em]
		\item Statement: $\lambda_1(\mathcal{L}) \le \sqrt{d}\det(\mathcal{L})^{1/d}$.
		\item Interpret: short vectors exist but finding them is hard.
	\end{itemize}
	
	\subsubsection*{Instructor notes}
	\Instructor{
		Give geometric intuition: fundamental parallelepiped volume; convex body argument.
		Make sure students can compute determinants and duals in low dimensions.
	}
	
	\subsubsection*{Worksheet}
	\Worksheet{
		\begin{enumerate}[leftmargin=2em]
			\item For $B=\begin{pmatrix}2&1\\0&3\end{pmatrix}$ compute $\det(\mathcal{L})$ and one nonzero short vector.
			\item Compute the dual lattice basis $B^{-\top}$ and verify pairing integrality.
		\end{enumerate}
	}
	
	\subsection{Lecture 3B: SVP/CVP, LLL/BKZ, Enumeration/Sieving}
	
	\subsubsection*{Slides-outline}
	\Slide{Problems}
	\begin{itemize}[leftmargin=2em]
		\item SVP/CVP and approximation $\gamma$-SVP/$\gamma$-CVP.
		\item Algorithm families: reduction (LLL/BKZ), enumeration, sieving.
	\end{itemize}
	
	\Slide{LLL vs BKZ}
	\begin{itemize}[leftmargin=2em]
		\item LLL: poly-time, exponential approximation factor.
		\item BKZ: parameter $\beta$ improves quality; dominates real cryptanalysis.
	\end{itemize}
	
	\subsubsection*{Instructor notes}
	\Instructor{
		Keep BKZ “concept-only”: local SVP on blocks, iterative.
		If asked for numbers: mention that security estimates use BKZ blocksize $\beta$ as main knob.
	}
	
	\subsubsection*{Worksheet}
	\Worksheet{
		\begin{enumerate}[leftmargin=2em]
			\item Run (by hand) a single LLL size-reduction + swap step on a 2D basis.
			\item Explain why enumeration complexity drops after basis reduction.
		\end{enumerate}
	}
	
	\subsection{Lecture 3C: SIS/LWE + Attack Taxonomy (Primal/Dual/Hybrid/BKW)}
	
	\subsubsection*{Slides-outline}
	\Slide{SIS}
	\begin{itemize}[leftmargin=2em]
		\item Given $A\in\Z_q^{n\times m}$ find short nonzero $x$ with $Ax\equiv 0\pmod q$.
	\end{itemize}
	
	\Slide{LWE}
	\begin{itemize}[leftmargin=2em]
		\item Distinguish $(a, \ip{a}{s}+e)$ from uniform; search-LWE recovers $s$.
	\end{itemize}
	
	\Slide{Attacks}
	\begin{itemize}[leftmargin=2em]
		\item Primal: embed to CVP/SVP, solve with BKZ+enum/sieve.
		\item Dual: find short dual vector to distinguish.
		\item Hybrid: guess some secret coordinates + reduce dimension.
		\item BKW: combinatorial sample combining; parameter-dependent.
	\end{itemize}
	
	\subsubsection*{Instructor notes}
	\Instructor{
		This lecture is about how cryptanalysts reason: “dimension drives security”.
		Explain qualitatively how $q$, noise $\alpha$, and dimension interact in primal/dual attacks.
	}
	
	\subsection{Week 3 Homework + solution sketches}
	
	\HW{
		\begin{enumerate}[leftmargin=2em]
			\item Prove $\det(\mathcal{L}^\ast)=1/\det(\mathcal{L})$ for full-rank lattices.
			\item In dimension 2, prove Minkowski’s bound using area and convexity.
			\item Give a one-page ``attack selection guide'' for LWE: when would you try primal vs dual vs hybrid vs BKW?
		\end{enumerate}
	}
	\Sol{
		(1) $\mathcal{L}=B\Z^d$, $\mathcal{L}^\ast=B^{-\top}\Z^d$, determinant transforms by $|\det(\cdot)|$.  
		(2) Use symmetric convex body disk of area $>4\det(\mathcal{L})$.  
		(3) Primal favored at certain noise; dual when short dual vectors exist; hybrid when secret small/structured; BKW when many samples and moderate noise.
	}
	
	% ============================================================
	\section{Week 4: Codes (Syndrome Decoding) and ISD Cryptanalysis}
	% ============================================================
	
	\subsection{Lecture 4A: Codes, Syndromes, Decoding Basics}
	
	\subsubsection*{Slides-outline}
	\Slide{Linear codes}
	\begin{itemize}[leftmargin=2em]
		\item $[n,k]_q$ linear code; generator $G$; parity-check $H$.
		\item Hamming weight/distance; decoding as nearest codeword problem.
	\end{itemize}
	
	\Slide{Syndrome}
	\begin{itemize}[leftmargin=2em]
		\item For $r=c+e$, $s=Hr^\top=He^\top$ depends only on error.
	\end{itemize}
	
	\subsubsection*{Instructor notes}
	\Instructor{
		Work a tiny $[7,4]$ Hamming code example if time; otherwise keep conceptual.
		Make students comfortable with matrix equations over $\F_2$.
	}
	
	\subsubsection*{Worksheet}
	\Worksheet{
		\begin{enumerate}[leftmargin=2em]
			\item Given $H$, compute syndrome of a received word and correct a single-bit error (toy).
			\item Show that syndrome decoding is solving for a low-weight vector in an affine subspace.
		\end{enumerate}
	}
	
	\subsection{Lecture 4B: Hard Problems (SD/MDP) and McEliece Context}
	
	\subsubsection*{Slides-outline}
	\Slide{Syndrome Decoding (SD)}
	\begin{itemize}[leftmargin=2em]
		\item Input: $(H,s,t)$; output $e$ with $He^\top=s$, $w_H(e)\le t$.
	\end{itemize}
	
	\Slide{McEliece}
	\begin{itemize}[leftmargin=2em]
		\item Public code should look random; secret structure allows fast decoding.
		\item Attacker: generic SD (ISD) unless structure leaks.
	\end{itemize}
	
	\subsection{Lecture 4C: ISD (Prange $\rightarrow$ Stern/Dumer/BJMM) + Quantum Notes}
	
	\subsubsection*{Slides-outline}
	\Slide{Prange ISD}
	\begin{itemize}[leftmargin=2em]
		\item Guess information set $I$ of size $k$ avoiding error positions.
		\item Success probability $\approx \binom{n-t}{k}/\binom{n}{k}$.
	\end{itemize}
	
	\Slide{Modern ISD}
	\begin{itemize}[leftmargin=2em]
		\item Stern/Dumer/BJMM: meet-in-the-middle improvements reduce exponent.
		\item Quantum: Grover speeds the guessing layers (model-dependent).
	\end{itemize}
	
	\subsubsection*{Instructor notes}
	\Instructor{
		Derive Prange probability in class; it is very accessible to mathematicians.
		Explain that modern ISD refinements optimize constant factors/exponents via clever splitting.
	}
	
	\subsection{Week 4 Homework + solution sketches}
	
	\HW{
		\begin{enumerate}[leftmargin=2em]
			\item Derive Prange expected work factor; plug in small toy parameters.
			\item Implement Prange on random binary codes (tiny) and compare to brute force.
			\item Explain what a ``structural attack'' means in code-based crypto and give one plausible distinguisher idea.
		\end{enumerate}
	}
	\Sol{
		(1) Inverse of success prob.  
		(2) Empirical scaling matches combinatorial estimates.  
		(3) Distinguisher examples: unusually low-weight dual codewords, rank properties, automorphism group size, etc.
	}
	
	% ============================================================
	\section{Week 5: Isogenies (Elliptic Curves, Graphs, Attacks)}
	% ============================================================
	
	\subsection{Lecture 5A: Elliptic Curve Essentials (finite fields)}
	
	\subsubsection*{Slides-outline}
	\Slide{Elliptic curves}
	\begin{itemize}[leftmargin=2em]
		\item $E/\F_q: y^2=x^3+ax+b$, $\Delta\neq 0$.
		\item Group law; torsion; Hasse bound (context).
	\end{itemize}
	
	\subsubsection*{Instructor notes}
	\Instructor{
		Don’t re-teach full EC theory; focus on what is needed:
		finite abelian group of points + morphisms.
		Optionally mention supersingular vs ordinary as a taxonomy.
	}
	
	\subsection{Lecture 5B: Isogenies (kernels, degrees, evaluation)}
	
	\subsubsection*{Slides-outline}
	\Slide{Isogeny definition}
	\begin{itemize}[leftmargin=2em]
		\item Group homomorphism given by rational maps; finite kernel; degree.
		\item Separable isogeny determined by its kernel; Vélu gives explicit formula.
	\end{itemize}
	
	\subsection{Lecture 5C: Hardness + Attacks (graph search, commutative actions, quantum)}
	
	\subsubsection*{Slides-outline}
	\Slide{Hard problems}
	\begin{itemize}[leftmargin=2em]
		\item Supersingular path-finding: find isogeny between $E$ and $E'$.
		\item CSIDH-style: recover class-group action element (commutative hidden shift flavor).
	\end{itemize}
	
	\Slide{Attacks}
	\begin{itemize}[leftmargin=2em]
		\item Meet-in-the-middle / bidirectional search (Delfs--Galbraith style).
		\item Protocol-specific breaks (e.g.\ SIDH/SIKE) vs generic problem.
		\item Quantum: Kuperberg-type subexponential for commutative hidden shift settings.
	\end{itemize}
	
	\subsection{Week 5 Homework + solution sketches}
	
	\HW{
		\begin{enumerate}[leftmargin=2em]
			\item Prove $\deg(\varphi\circ\psi)=\deg(\varphi)\deg(\psi)$.
			\item Explain why kernels classify separable isogenies (state carefully; prove a special case).
			\item Compare ``path-finding'' vs ``hidden shift'' formulations and their algorithmic consequences.
		\end{enumerate}
	}
	\Sol{
		(1) Degree of morphisms multiplies under composition.  
		(2) In separable case, quotient by finite subgroup yields isogeny; Vélu constructs it.  
		(3) Hidden shift allows Fourier methods (Kuperberg); generic path-finding is graph search.
	}
	
	% ============================================================
	\section{Week 6: Multivariate (MQ) --- Algebraic Attacks and Trapdoor Structure}
	% ============================================================
	
	\subsection{Lecture 6A: MQ as Polynomial System Solving}
	
	\subsubsection*{Slides-outline}
	\Slide{MQ}
	\begin{itemize}[leftmargin=2em]
		\item Given quadratic $f_1,\dots,f_m\in\F_q[x_1,\dots,x_n]$, find $x\in\F_q^n$ with $f_i(x)=0$.
		\item View as variety $V(I)$ for ideal $I=\langle f_1,\dots,f_m\rangle$.
	\end{itemize}
	
	\subsection{Lecture 6B: Gr\"obner Bases and Degree of Regularity}
	
	\subsubsection*{Slides-outline}
	\Slide{Gr\"obner}
	\begin{itemize}[leftmargin=2em]
		\item Term orders; leading term; elimination under lex.
		\item F4/F5 as efficient engines; complexity depends on degree of regularity.
	\end{itemize}
	
	\subsection{Lecture 6C: XL/Hybrid/MinRank (Structured Attacks)}
	
	\subsubsection*{Slides-outline}
	\Slide{Attack families}
	\begin{itemize}[leftmargin=2em]
		\item XL/relinearization: multiply, linearize, solve linear system.
		\item Hybrid: guess $k$ variables, solve remaining.
		\item MinRank/rank attacks exploit matrix structure of quadratic forms.
	\end{itemize}
	
	\subsection{Week 6 Homework + solution sketches}
	
	\HW{
		\begin{enumerate}[leftmargin=2em]
			\item Convert a quadratic system over odd characteristic into matrix form; identify rank conditions.
			\item Analyze hybrid complexity $q^k\cdot T(n-k)$; optimize $k$ for a toy model $T(t)=q^{ct}$.
			\item Solve a small MQ instance over $\F_2$ by linearization; compare to brute force.
		\end{enumerate}
	}
	\Sol{
		(1) Quadratic form $\leftrightarrow$ symmetric matrix after completing square; cross-terms map to off-diagonal.  
		(2) Minimize exponent: $k + c(n-k)=cn + (1-c)k$ so choose $k=0$ if $c<1$, etc.  
		(3) Linearization works if enough equations / low degree growth.
	}
	
	% ============================================================
	\section{Week 7: Hash Functions --- Games, Bounds, Constructions, Structural Attacks}
	% ============================================================
	
	\subsection{Lecture 7A: Formal Games (CR/SPR/OW)}
	
	\subsubsection*{Slides-outline}
	\Slide{Hash family}
	\begin{itemize}[leftmargin=2em]
		\item $H:\bits^\ast\to \bits^n$, security notions as games.
		\item Collision resistance, second-preimage, preimage.
	\end{itemize}
	
	\subsection{Lecture 7B: Generic Bounds (Birthday, Preimages) + Proofs}
	
	\subsubsection*{Slides-outline}
	\Slide{Birthday}
	\begin{itemize}[leftmargin=2em]
		\item Collision after $\approx 2^{n/2}$ queries.
		\item Approx formula: $1-\exp(-q(q-1)/2^{n+1})$.
	\end{itemize}
	
	\subsection{Lecture 7C: Merkle--Damg{\aa}rd, Length Extension, HMAC, Quantum}
	
	\subsubsection*{Slides-outline}
	\Slide{Merkle--Damg{\aa}rd}
	\begin{itemize}[leftmargin=2em]
		\item Iterated compression + padding.
		\item Length extension and why $H(k\|m)$ is a bad MAC.
		\item HMAC fixes it (double hash with keyed pads).
		\item Quantum: Grover preimages $\approx 2^{n/2}$.
	\end{itemize}
	
	\subsection{Week 7 Homework + solution sketches}
	
	\HW{
		\begin{enumerate}[leftmargin=2em]
			\item Prove the birthday bound formula (use occupancy or Poisson approximation).
			\item Demonstrate length extension in an idealized Merkle--Damg{\aa}rd model.
			\item Given $n$-bit hash output, compute classical vs quantum work for preimages and collisions; infer recommended $n$ for 128-bit post-quantum preimage security.
		\end{enumerate}
	}
	\Sol{
		(1) Probability no collision $\approx \prod_{i=0}^{q-1}(1-i/2^n)\approx e^{-q(q-1)/2^{n+1}}$.  
		(2) Internal chaining value after $m$ lets extend with known padding and extra blocks.  
		(3) Preimage: classical $2^n$, quantum $2^{n/2}$, so for 128-bit PQ preimage choose $n\approx 256$.
	}
	
	% ============================================================
	\section{Capstone (Optional): One Comparative Lecture + Exam-Style Questions}
	% ============================================================
	
	\subsection*{Slides-outline}
	\Slide{Compare families}
	\begin{itemize}[leftmargin=2em]
		\item Shor breaks factoring/DLP; Grover halves preimage exponent; others survive (no known poly-time).
		\item ``Security knob'': modulus size (factoring/DLP), group order (ECDLP), dimension (lattices),
		length/weight (codes), graph size/path length (isogenies), degree of regularity (MQ), output length (hash).
	\end{itemize}
	
	\subsection*{Exam-style questions}
	\begin{enumerate}[leftmargin=2em]
		\item Explain why Pohlig--Hellman forces cryptographers to use prime-order subgroups.
		\item Given an LWE instance, argue (qualitatively) how increasing $q$ changes primal vs dual attack feasibility.
		\item Compare birthday vs Grover and deduce hash output sizes for post-quantum targets.
	\end{enumerate}

	\newpage
	% ============================================================
	% Question-design playbook + detailed practice problems
	% For teaching hard problems in cryptography (math audience)
	% ============================================================
	
	\section{How to Design Good Questions (Instructor Toolkit)}
	
	\subsection{Learning objective $\rightarrow$ question template}
	
	For each topic, target a mix of:
	\begin{enumerate}[leftmargin=2em]
		\item \textbf{Definition checks} (precision): “State/derive the formal definition; identify inputs/outputs; specify distribution.”
		\item \textbf{Reduction problems} (mathematical thinking): “Show $A \le B$ via explicit oracle reduction; track success probability.”
		\item \textbf{Algorithm traces} (mechanics): “Run the algorithm on a toy instance; show intermediate steps.”
		\item \textbf{Complexity reasoning} (asymptotics): “Explain why runtime is $\tilde O(\sqrt{n})$ / $L_N[\alpha,c]$ / $2^{\Theta(d)}$.”
		\item \textbf{Attack selection} (cryptanalytic judgment): “Given parameters/structure, which attack dominates and why?”
		\item \textbf{Failure-mode questions} (engineering reality): “What breaks if validation/randomness is wrong? Provide counterexample.”
		\item \textbf{Proof-based extensions} (math depth): “Prove a standard lemma (Minkowski in 2D, Prange probability, birthday bound).”
	\end{enumerate}
	
	\subsection{Difficulty ladder (use for worksheets/homework/exams)}
	
	For each concept, create 4 tiers:
	\begin{itemize}[leftmargin=2em]
		\item \textbf{Tier 1 (warm-up):} recall/compute; single idea.
		\item \textbf{Tier 2 (core):} 2--3 steps; requires correct definitions.
		\item \textbf{Tier 3 (integration):} connects two concepts (e.g., DLP + subgroup structure; LWE + BKZ intuition).
		\item \textbf{Tier 4 (research-flavored):} open-ended but gradable: justify assumptions, compare attacks, critique parameter choices.
	\end{itemize}
	
	\subsection{Common pitfalls to avoid}
	
	\begin{itemize}[leftmargin=2em]
		\item Overly large toy numbers: keep hand-computable (e.g., primes $<50$; lattice dimension $2$ or $3$; codes length $\le 12$).
		\item Vague prompts: force explicit input/output and probability space.
		\item ``Prove hardness'': instead ask to prove \emph{reductions}, \emph{bounds}, or \emph{attack correctness}.
		\item Mixing security notions: be explicit about search vs decision vs distinguishing.
	\end{itemize}
	
	\subsection{Grading rubrics (quick)}
	
	\begin{itemize}[leftmargin=2em]
		\item \textbf{Definitions:} correct quantifiers, domains, modulo conventions.
		\item \textbf{Reductions:} explicit oracle calls; success probability; running time bound.
		\item \textbf{Algorithm traces:} correct intermediate computations; verify condition checks (gcd, smoothness, syndrome, etc.).
		\item \textbf{Attack selection:} justified by structure/parameters; not name-dropping.
	\end{itemize}
	
	% ============================================================
	\section{Practice Problems by Topic (with short solution notes)}
	% ============================================================
	
	% ------------------------------------------------------------
	\subsection{Week 1: Integer Factorization}
	% ------------------------------------------------------------
	
	\subsubsection*{Tier 1--2 (warm-up/core)}
	
	\begin{exercise}[Factoring vs Euler totient]
		Let $N=pq$ where $p,q$ are distinct odd primes.
		Show that knowing $\varphi(N)$ allows recovery of $p$ and $q$ in time polynomial in $\log N$.
	\end{exercise}
	\begin{remark}\textbf{Solution note.}
		Compute $S=p+q=N-\varphi(N)+1$ and solve $X^2-SX+N=0$.
	\end{remark}
	
	\begin{exercise}[Order-finding implies factoring]
		Let $N=pq$ be an RSA modulus. Suppose an oracle returns $\mathrm{ord}_N(a)$ for any $a\in(\Z/N\Z)^\times$.
		Give a randomized algorithm that factors $N$ using the oracle and analyze its success probability.
	\end{exercise}
	\begin{remark}\textbf{Solution note.}
		Pick random $a$; get $r=\mathrm{ord}_N(a)$. If $r$ even and $a^{r/2}\not\equiv -1\pmod N$ then
		$\gcd(a^{r/2}-1,N)$ yields a factor. Bound success away from $0$ under standard arguments.
	\end{remark}
	
	\begin{exercise}[Pollard $p-1$ success condition]
		State precisely the condition under which Pollard $p-1$ finds a factor $p\mid N$.
		Give a worked example with $N=187=11\cdot 17$ and a suitable smoothness bound.
	\end{exercise}
	\begin{remark}\textbf{Solution note.}
		If $p-1$ is $B$-smooth and $M=\mathrm{lcm}(1,\dots,B)$ then $a^M\equiv 1\pmod p$ for many $a$,
		so $\gcd(a^M-1,N)$ reveals $p$.
	\end{remark}
	
	\subsubsection*{Tier 3--4 (integration/research-flavored)}
	
	\begin{exercise}[Why linear algebra appears in QS]
		Explain why the Quadratic Sieve collects exponent vectors modulo $2$ over a factor base.
		Derive the linear algebra condition that guarantees a congruence of squares.
	\end{exercise}
	\begin{remark}\textbf{Solution note.}
		Smooth relations give $x_i^2-N=\prod p_j^{e_{ij}}$; if $\sum_i e_{ij}\equiv 0\pmod 2$ for all $j$,
		then $\prod_i(x_i^2-N)$ is a square; hence $x^2\equiv y^2\pmod N$.
	\end{remark}
	
	\begin{exercise}[Attack selection]
		You are given a 2048-bit RSA modulus $N$ and told it may have a 200-bit prime factor.
		Which attack do you try first and why? Contrast ECM vs GNFS.
	\end{exercise}
	\begin{remark}\textbf{Solution note.}
		ECM targets small/medium prime factors and is far cheaper than GNFS if such a factor exists.
	\end{remark}
	
	% ------------------------------------------------------------
	\subsection{Week 2: Discrete Logarithms (Finite Fields and Elliptic Curves)}
	% ------------------------------------------------------------
	
	\subsubsection*{Tier 1--2}
	
	\begin{exercise}[Baby-step/giant-step by hand]
		In $G=\Z_{29}^\times$, let $g=2$ and $h=18$.
		Compute $x$ such that $2^x\equiv 18\pmod{29}$ using baby-step/giant-step.
	\end{exercise}
	\begin{remark}\textbf{Solution note.}
		Take $m=\lceil\sqrt{28}\rceil=6$, build baby steps $g^0,\dots,g^5$,
		and giant steps $h g^{-6j}$ until collision.
	\end{remark}
	
	\begin{exercise}[Pohlig--Hellman core step]
		Let $G=\langle g\rangle$ have order $n=p^e$.
		Show how to recover $x \bmod p^e$ from an oracle that solves DLP modulo $p$ repeatedly (lifting).
	\end{exercise}
	\begin{remark}\textbf{Solution note.}
		Use base-$p$ expansion $x=\sum_{i=0}^{e-1} x_i p^i$ and solve successive digits by powering to $n/p$.
	\end{remark}
	
	\subsubsection*{Tier 3--4}
	
	\begin{exercise}[Why ECDLP avoids index calculus (concept)]
		Give a precise statement of what ``index calculus'' needs (smoothness notion and factor base),
		and explain why a naive analogue fails in generic elliptic-curve groups.
	\end{exercise}
	\begin{remark}\textbf{Solution note.}
		Finite fields admit unique factorization of ideals/elements and smoothness probabilities; generic EC groups do not
		provide comparable decomposition structure for random points.
	\end{remark}
	
	\begin{exercise}[MOV condition]
		State the condition (in terms of embedding degree) under which MOV/Frey--R\"uck reduces ECDLP to finite-field DLP.
		Why is this avoided in standard curve selection?
	\end{exercise}
	\begin{remark}\textbf{Solution note.}
		If there exists small $k$ with $n\mid (q^k-1)$, pairings map to $\F_{q^k}^\times$,
		where index calculus applies.
	\end{remark}
	
	\begin{exercise}[Subgroup-validation failure]
		Construct an explicit example where a DH implementation that fails to validate subgroup membership leaks information about the secret exponent.
	\end{exercise}
	\begin{remark}\textbf{Solution note.}
		Use small-subgroup confinement: attacker sends element of small-order subgroup; responses leak exponent mod that order.
	\end{remark}
	
	% ------------------------------------------------------------
	\subsection{Week 3: Lattices (SVP/CVP, SIS/LWE)}
	% ------------------------------------------------------------
	
	\subsubsection*{Tier 1--2}
	
	\begin{exercise}[Compute determinant and dual (2D)]
		Let $B=\begin{pmatrix}2&1\\0&3\end{pmatrix}$.
		Compute $\det(\mathcal{L}(B))$ and a basis of the dual lattice.
	\end{exercise}
	\begin{remark}\textbf{Solution note.}
		$\det=6$. Dual basis is $B^{-\top}$, scaled appropriately; verify inner products are integers.
	\end{remark}
	
	\begin{exercise}[Minkowski in dimension 2]
		Prove Minkowski’s first theorem bound in $\R^2$ using an area argument.
	\end{exercise}
	\begin{remark}\textbf{Solution note.}
		Use convex centrally symmetric body of area $>4\det(\mathcal{L})$ implies nonzero lattice point.
	\end{remark}
	
	\subsubsection*{Tier 3--4}
	
	\begin{exercise}[LWE distinguishing bias (dual attack intuition)]
		Suppose you find $y\in\Z_q^m$ such that $y^TA\equiv 0\pmod q$ and $y$ is short.
		Show that $y^Tb$ has a distributional bias when $(A,b)$ is LWE vs uniform.
	\end{exercise}
	\begin{remark}\textbf{Solution note.}
		If $b=As+e$, then $y^Tb\equiv y^Te \pmod q$; short $y$ keeps $y^Te$ small (non-uniform).
	\end{remark}
	
	\begin{exercise}[Attack selection guide]
		Given LWE parameters $(n,q,\alpha)$ (noise rate), explain when primal vs dual attacks are expected to dominate.
		Your answer should explicitly reference: dimension reduction quality, target vector norm, and sample count.
	\end{exercise}
	\begin{remark}\textbf{Solution note.}
		Primal: embedding finds close vector; dual: find short dual; hybrid trades dimension; BKW if many samples and moderate noise.
	\end{remark}
	
	% ------------------------------------------------------------
	\subsection{Week 4: Codes (Syndrome Decoding, ISD)}
	% ------------------------------------------------------------
	
	\subsubsection*{Tier 1--2}
	
	\begin{exercise}[Syndrome depends only on error]
		Let $H$ be a parity-check matrix and $r=c+e$ with $c\in C$.
		Show $Hr^\top = He^\top$.
	\end{exercise}
	\begin{remark}\textbf{Solution note.}
		$Hc^\top=0$, so $Hr^\top=H(c+e)^\top=He^\top$.
	\end{remark}
	
	\begin{exercise}[Prange success probability]
		In binary SD, assume an error vector has weight $t$.
		If an algorithm guesses an information set $I$ of size $k$ uniformly among $\binom{n}{k}$ choices,
		derive the probability that $I$ avoids all $t$ error positions.
	\end{exercise}
	\begin{remark}\textbf{Solution note.}
		$\Pr[I\cap \mathrm{supp}(e)=\emptyset]=\binom{n-t}{k}/\binom{n}{k}$.
	\end{remark}
	
	\subsubsection*{Tier 3--4}
	
	\begin{exercise}[Affine-subspace viewpoint]
		Show that the solution set to $He^\top=s$ is an affine subspace of $\F_2^n$ of dimension $k$.
		Interpret SD as finding a low-weight element in that affine space.
	\end{exercise}
	\begin{remark}\textbf{Solution note.}
		Fix one solution $e_0$; all solutions are $e_0 + \ker(H)$; $\dim \ker(H)=k$.
	\end{remark}
	
	\begin{exercise}[Structural vs generic attacks]
		Explain (with a concrete statistic) how one might distinguish a structured public code (e.g.\ with many low-weight dual codewords)
		from a uniformly random code of the same parameters.
	\end{exercise}
	\begin{remark}\textbf{Solution note.}
		Compute weight distribution of dual, automorphism group size, rank properties, etc.
	\end{remark}
	
	% ------------------------------------------------------------
	\subsection{Week 5: Isogenies}
	% ------------------------------------------------------------
	
	\subsubsection*{Tier 1--2}
	
	\begin{exercise}[Degree multiplicativity]
		Let $\varphi:E_1\to E_2$ and $\psi:E_2\to E_3$ be isogenies.
		Prove $\deg(\psi\circ \varphi)=\deg(\psi)\deg(\varphi)$.
	\end{exercise}
	\begin{remark}\textbf{Solution note.}
		Degree of morphisms multiplies under composition; can be shown via function field extensions.
	\end{remark}
	
	\begin{exercise}[Kernel determines separable isogeny (special case)]
		State and prove: for a finite subgroup $K\le E(\overline{\F}_q)$ of order coprime to $\mathrm{char}(\F_q)$,
		there exists a separable isogeny with kernel $K$.
	\end{exercise}
	\begin{remark}\textbf{Solution note.}
		Quotient curve $E/K$ exists; Vélu gives explicit formulas.
	\end{remark}
	
	\subsubsection*{Tier 3--4}
	
	\begin{exercise}[Graph search complexity heuristic]
		Model a supersingular $\ell$-isogeny graph as a random $d$-regular graph on $M$ vertices.
		Estimate the expected meet-in-the-middle time to find a path between two random vertices.
	\end{exercise}
	\begin{remark}\textbf{Solution note.}
		Bidirectional BFS to depth $\approx \frac{1}{2}\log_d M$ visits $\approx d^{\ell/2}\approx \sqrt{M}$ states;
		refine to $\tilde O(M^{1/2})$ or $p^{1/4}$ depending on the parameterization used.
	\end{remark}
	
	% ------------------------------------------------------------
	\subsection{Week 6: Multivariate (MQ)}
	% ------------------------------------------------------------
	
	\subsubsection*{Tier 1--2}
	
	\begin{exercise}[Linearization]
		Given quadratic equations over $\F_2$ in variables $x_1,\dots,x_n$,
		define new variables $y_{ij}=x_ix_j$ (for $i\le j$) and write the system as linear equations in the $y_{ij}$.
		When is this sufficient to solve the system?
	\end{exercise}
	\begin{remark}\textbf{Solution note.}
		If enough independent equations exist and consistency constraints are manageable; otherwise many spurious solutions.
	\end{remark}
	
	\subsubsection*{Tier 3--4}
	
	\begin{exercise}[Hybrid complexity optimization]
		Suppose solving MQ in $t$ variables costs $T(t)=q^{ct}$ operations.
		If you guess $k$ variables, derive total cost $q^k T(n-k)$ and find the optimal $k$.
	\end{exercise}
	\begin{remark}\textbf{Solution note.}
		Exponent is $k + c(n-k)=cn+(1-c)k$; if $c<1$, minimize at $k=0$; if $c>1$, at $k=n$ (toy model).
		Real models have non-linear $T(t)$ so optimization is nontrivial.
	\end{remark}
	
	\begin{exercise}[Matrix form of quadratic maps (odd characteristic)]
		Show that any quadratic polynomial $f(x)\in\F_q[x_1,\dots,x_n]$ (odd $q$)
		can be written as $x^\top A x + b^\top x + c$ with $A$ symmetric.
	\end{exercise}
	\begin{remark}\textbf{Solution note.}
		Use $x_ix_j$ cross terms; symmetrize using $(A+A^\top)/2$ since 2 is invertible.
	\end{remark}
	
	% ------------------------------------------------------------
	\subsection{Week 7: Hash Functions}
	% ------------------------------------------------------------
	
	\subsubsection*{Tier 1--2}
	
	\begin{exercise}[Birthday bound derivation]
		Let $H$ be a random function into $\bits^n$. After $q$ queries, show
		\[
		\Pr[\text{collision}] \approx 1-\exp\!\left(-\frac{q(q-1)}{2^{n+1}}\right).
		\]
	\end{exercise}
	\begin{remark}\textbf{Solution note.}
		Probability of no collision $\approx \prod_{i=0}^{q-1}(1-i/2^n)$ and use $\log(1-x)\approx -x$.
	\end{remark}
	
	\begin{exercise}[Length extension (Merkle--Damg{\aa}rd)]
		In an iterated hash $H(m)=f(\cdots f(IV,m_1),\dots,m_t)$ with MD padding,
		explain how $H(m\|pad(m)\|m')$ can be computed from $H(m)$ and $|m|$ without knowing $m$.
	\end{exercise}
	\begin{remark}\textbf{Solution note.}
		$H(m)$ is the internal chaining value after padding; reuse as IV for extra blocks.
	\end{remark}
	
	\subsubsection*{Tier 3--4}
	
	\begin{exercise}[Post-quantum sizing]
		If Grover gives preimages in $\Theta(2^{n/2})$ quantum queries, what output length $n$
		is needed for $\approx 128$-bit post-quantum preimage security? Compare to collision security.
	\end{exercise}
	\begin{remark}\textbf{Solution note.}
		Need $2^{n/2}\approx 2^{128}\Rightarrow n\approx 256$ for PQ preimages; collisions require larger for PQ depending on collision algorithm model.
	\end{remark}
	
	% ============================================================
	\section{Ready-to-Use Question Sets (by class type)}
	% ============================================================
	
	\subsection{Quick in-class checks (5--10 minutes each)}
	\begin{enumerate}[leftmargin=2em]
		\item (Factoring) State the exact condition that makes Pollard $p-1$ succeed.
		\item (DLP) Why does Pohlig--Hellman force prime-order subgroups?
		\item (Lattices) Define the dual lattice and compute it for a given basis.
		\item (Codes) Derive Prange success probability.
		\item (MQ) Explain relinearization in one paragraph.
		\item (Hash) Derive birthday bound in two lines using $\log(1-x)\approx -x$.
	\end{enumerate}
	
	\subsection{Recitation set (60--90 minutes)}
	Pick 1--2 per topic:
	\begin{enumerate}[leftmargin=2em]
		\item Baby-step/giant-step on a small finite field.
		\item One LLL step on a 2D lattice basis + interpret geometric meaning.
		\item Prange ISD expected trials for a toy $(n,k,t)$.
		\item A small MQ system over $\F_2$ solved by linearization.
		\item Birthday bound + compute $q$ for 50\% collision probability.
	\end{enumerate}
	
	\subsection{Exam-style integrators}
	\begin{enumerate}[leftmargin=2em]
		\item Compare classical vs quantum asymptotics for factoring, DLP, hash preimages, and one PQ family (lattices/codes/isogenies/MQ).
		\item Given a parameter set for an LWE-based KEM, explain qualitatively which attack is expected to dominate and what parameter changes would harden it.
		\item Given a hash output length, infer collision vs preimage security in classical and quantum models and recommend a safe output length.
	\end{enumerate}
	
	% ============================================================
	% End
	% ============================================================
	
	
\end{document}
