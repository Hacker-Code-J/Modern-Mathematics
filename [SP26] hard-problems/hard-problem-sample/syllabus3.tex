% ============================================================
% One-page Flyer / Announcement (LaTeX)
% Colloquium: Hard Problems in Cryptography
% ============================================================
\documentclass[11pt]{article}

\usepackage[a4paper,margin=1in]{geometry}
\usepackage{hyperref}
\usepackage{enumitem}
\usepackage{xcolor}
\usepackage{array}

\setlength{\parindent}{0pt}
\setlength{\parskip}{6pt}

% ---------- Fill these in ----------
\newcommand{\groupname}{\textbf{Hard Problems in Cryptography Colloquium}}
\newcommand{\term}{\textbf{[Insert Term / Date Range]}}
\newcommand{\whenwhere}{\textbf{[Day, Time] @ [Room] / Zoom: [Link]}}
\newcommand{\organizer}{\textbf{[Your Name]}}
\newcommand{\contact}{\textbf{[Email]}}
\newcommand{\signup}{\textbf{[Sign-up link / form]}}
\newcommand{\repo}{\textbf{[Shared notes link]}}
\newcommand{\deadline}{\textbf{[Sign-up deadline]}}
\newcommand{\audience}{\textbf{Masters/advanced undergrad; PhD welcome}}

% ---------- Styling ----------
\definecolor{accent}{RGB}{20,75,140}

\begin{document}
	
	% ---------- Header ----------
	{\Large \color{accent}\textbf{\groupname}}\\
	{\normalsize \term}\\[4pt]
	\textbf{Meetings:} \whenwhere\\
	\textbf{Organizer:} \organizer \quad $\mid$ \quad \textbf{Contact:} \contact\\
	\textbf{Sign-up:} \signup \quad $\mid$ \quad \textbf{Deadline:} \deadline\\
	\textbf{Shared notes/repo:} \repo
	
	\vspace{8pt}
	\hrule
	\vspace{10pt}
	
	% ---------- Body ----------
	\textbf{What is this?}\\
	A structured, discussion-driven study group on the \emph{hard computational problems} that underpin classical and post-quantum cryptography.
	Each week features a short talk by a participant followed by a guided board session (proof sketch + toy attack + Q\&A).
	
	\textbf{Topics (tentative):}
	\begin{itemize}[leftmargin=2em]
		\item Integer factorization (RSA/Rabin): ECM, QS, GNFS, Shor overview
		\item Discrete logarithms (finite fields \& elliptic curves): Pollard $\rho$, Pohlig--Hellman, index calculus, pairing pitfalls
		\item Lattices (SVP/CVP, SIS/LWE): LLL/BKZ intuition; primal/dual/hybrid attacks
		\item Codes (syndrome decoding): McEliece context; ISD (Prange $\rightarrow$ BJMM)
		\item Isogenies: supersingular graphs; CSIDH-style actions; attack heuristics
		\item Multivariate (MQ): Gr\"obner/XL/hybrid; structural pitfalls
		\item Hash: CR/SPR/OW games; birthday bound; length extension; HMAC; quantum impacts
	\end{itemize}
	
	\textbf{Audience / prerequisites}\\
	\audience. You should be comfortable with modular arithmetic, basic linear algebra, and proof writing.
	We will share a short preliminaries handout.
	
	\textbf{Format (90 minutes/week)}
	\begin{itemize}[leftmargin=2em]
		\item 30--40 min participant talk (rotating presenters)
		\item 30--40 min board session (practice problems / proof sketches)
		\item 10--15 min research-style discussion (assumptions, best-known attacks, open questions)
	\end{itemize}
	
	\textbf{Why join?}
	\begin{itemize}[leftmargin=2em]
		\item Build a coherent map of cryptographic hardness assumptions and their best-known attacks.
		\item Improve paper-reading and presentation skills in a supportive setting.
		\item Leave with a shared set of clean notes, a glossary, and a curated reference list.
	\end{itemize}
	
	\textbf{How to participate}\\
	Sign up by \deadline\ using \signup. Indicate whether you can present (recommended but optional).
	Presenters will get a template (definition + reduction + attack mechanism + 2 practice problems).
	
	\vfill
	\hrule
	\vspace{6pt}
	{\small \textbf{Note:} This is an educational colloquium. We focus on publicly documented algorithms/attacks and do not discuss wrongdoing or misuse.}
	
\end{document}
