% main.tex
\documentclass[11pt]{article}

% ---------- Packages ----------
\usepackage[letterpaper,margin=1in]{geometry}
\usepackage[utf8]{inputenc}
\usepackage[T1]{fontenc}
\usepackage{mathpazo} % Palatino font
%\usepackage{microtype}
%\usepackage{lmodern}
\usepackage{amsmath,amssymb,amsthm,mathtools}
\usepackage{bbm}           % indicator 1_{}
\usepackage{hyperref}
\usepackage{cleveref}
\usepackage{graphicx}
\usepackage{xcolor}
\usepackage{enumitem}
\usepackage{todonotes}
\usepackage{booktabs}
\usepackage{multirow}
\usepackage{longtable}
\usepackage{array}

% If you want pseudocode:
\usepackage{algorithm}
\usepackage{algpseudocode}

% If you want cryptography notation:
\usepackage{cryptocode}  % great for games and proofs (if installed)

\usepackage{pdfpages}

% ---------- Hyperref setup ----------
\hypersetup{
	colorlinks=true,
	linkcolor=blue,
	citecolor=blue,
	urlcolor=blue,
	pdftitle={Hard Problems in Cryptography: Classical and Post-Quantum},
}

% ---------- Theorem environments ----------
\theoremstyle{plain}
\newtheorem{theorem}{Theorem}[section]
\newtheorem{lemma}[theorem]{Lemma}
\newtheorem{proposition}[theorem]{Proposition}
\newtheorem{corollary}[theorem]{Corollary}

\theoremstyle{definition}
\newtheorem{definition}[theorem]{Definition}
\newtheorem{game}[theorem]{Game}
\newtheorem{assumption}[theorem]{Assumption}

\theoremstyle{remark}
\newtheorem{remark}[theorem]{Remark}
\newtheorem{example}[theorem]{Example}

% ---------- Common macros ----------
\newcommand{\negl}{\mathsf{negl}}
\newcommand{\poly}{\mathsf{poly}}
\newcommand{\ppt}{\mathsf{ppt}}
\newcommand{\Z}{\mathbb{Z}}
\newcommand{\R}{\mathbb{R}}
\newcommand{\bits}{\{0,1\}}
\newcommand{\Adv}{\mathsf{Adv}}

% Distributions and sampling
\newcommand{\getsr}{\xleftarrow{\$}}

% ---------- Bibliography (biblatex) ----------
\usepackage[
backend=biber,
style=alphabetic,
maxbibnames=99
]{biblatex}
\addbibresource{refs.bib}

% ---------- Document ----------
\title{Hard Problems in Cryptography:\\ Classical Assumptions and Post-Quantum Foundations}
\author{Ji, Yonghyeon}
\date{\today}

\begin{document}
	\includepdf[pages=-, fitpaper=true]{tikzs/cover.pdf}
	\cleardoublepage
	\newpage	
	
	\maketitle
	\tableofcontents
	
	\newpage
%	% ============================================================
%	% Summary table: hard problems / primitives and main attacks
%	% ============================================================
%	
%	\section*{Summary of Hard Problems and Main Attacks}
%	
%	\renewcommand{\arraystretch}{1.25}
%	\begin{tabular}{|p{2.8cm}|p{4.4cm}|p{6.8cm}|p{3.8cm}|}
%		\hline
%		\textbf{Problem / primitive} & \textbf{Canonical instance} & \textbf{Best-known / standard classical attacks (high level)} & \textbf{Quantum attacks / notes} \\
%		\hline
%		
%		\textbf{Integer Factorization (IFP)} &
%		Given $N=pq$ (RSA-type modulus), recover $p,q$ &
%		\begin{itemize}\setlength\itemsep{0.15em}
%			\item \emph{General Number Field Sieve (GNFS)} (asymptotically fastest for general $N$).
%			\item \emph{Special NFS (SNFS)} when $N$ has special form.
%			\item \emph{ECM} (Elliptic Curve Method) for finding relatively small prime factors.
%			\item Exploit bad key generation: shared primes, partial leakage, or smoothness.
%		\end{itemize}
%		&
%		\begin{itemize}\setlength\itemsep{0.15em}
%			\item \emph{Shor}: polynomial-time factoring on a fault-tolerant quantum computer.
%		\end{itemize}
%		\\
%		\hline
%		
%		\textbf{Discrete Logarithm (DLP)} &
%		Given cyclic group $G=\langle g\rangle$ and $h=g^x$, recover $x$ &
%		\begin{itemize}\setlength\itemsep{0.15em}
%			\item \emph{Generic} (works in any group): Baby-step/giant-step; Pollard $\rho$ ($\tilde O(\sqrt{|G|})$).
%			\item \emph{Pohlig--Hellman} when $|G|$ is smooth-ish (reduces to prime-power factors).
%			\item \emph{Index calculus / NFS-DL} in $\mathbb{F}_p^\times$ and related finite fields (subexponential).
%			\item For elliptic curves: generally only generic attacks apply; plus \emph{MOV/Frey--R\"uck} reductions for special curves (pairing-friendly/pathological cases).
%		\end{itemize}
%		&
%		\begin{itemize}\setlength\itemsep{0.15em}
%			\item \emph{Shor}: polynomial-time discrete log in abelian groups.
%		\end{itemize}
%		\\
%		\hline
%		
%		\textbf{Lattices (SVP/CVP/LWE)} &
%		SVP/CVP on $\mathcal{L}(B)\subset\mathbb{R}^n$, or LWE over $\mathbb{Z}_q$ &
%		\begin{itemize}\setlength\itemsep{0.15em}
%			\item \emph{Basis reduction (LLL/BKZ)} + enumeration (primal attacks); \emph{sieving} for SVP in moderate $n$.
%			\item \emph{Dual attacks} (distinguish LWE via short dual vectors) using BKZ.
%			\item \emph{Hybrid attacks} (meet-in-the-middle on part of secret + dual/primal).
%			\item \emph{Combinatorial attacks} such as BKW (parameter-dependent).
%			\item \emph{Algebraic attacks} (e.g.\ Arora--Ge style) for certain parameter regimes.
%		\end{itemize}
%		&
%		\begin{itemize}\setlength\itemsep{0.15em}
%			\item Quantum speedups exist for some subroutines (e.g.\ search/sieving), but no known Shor-like poly-time algorithm for general lattice problems.
%		\end{itemize}
%		\\
%		\hline
%		
%		\textbf{Codes (Syndrome Decoding / Min Distance)} &
%		Given $H$, syndrome $s$, and weight bound $t$, find $e$ with $He^\top=s$ and $w_H(e)\le t$ &
%		\begin{itemize}\setlength\itemsep{0.15em}
%			\item \emph{Information Set Decoding (ISD)} family: Prange, Stern/Dumer and many improvements (incl.\ BJMM-style).
%			\item \emph{Structural attacks} if the public code has exploitable structure (distinguish-and-recover hidden structure).
%		\end{itemize}
%		&
%		\begin{itemize}\setlength\itemsep{0.15em}
%			\item Grover-type speedups can reduce exhaustive-search exponents (often modeled as square-root improvements), but ISD has more nuanced quantum analyses.
%		\end{itemize}
%		\\
%		\hline
%		
%		\textbf{Isogenies (supersingular)} &
%		Given supersingular $E,E'$ (often over $\mathbb{F}_{p^2}$), find an isogeny $\varphi:E\to E'$ of prescribed degree &
%		\begin{itemize}\setlength\itemsep{0.15em}
%			\item Model as path-finding in the supersingular isogeny graph.
%			\item \emph{Meet-in-the-middle} graph search for isogeny path (Delfs--Galbraith-type approaches and refinements).
%			\item \textbf{Protocol-specific breaks:} SIDH/SIKE was broken by the Castryck--Decru attack (key-recovery), so those assumptions/instantiations are not considered secure.
%		\end{itemize}
%		&
%		\begin{itemize}\setlength\itemsep{0.15em}
%			\item Known quantum algorithms can improve the exponent for the generic supersingular isogeny problem (still superpolynomial in $\log p$).
%		\end{itemize}
%		\\
%		\hline
%		
%		\textbf{Hash functions} &
%		$H:\{0,1\}^\ast\to\{0,1\}^n$ with CR/SPR/OW notions &
%		\begin{itemize}\setlength\itemsep{0.15em}
%			\item \emph{Generic bounds:} collisions in $\tilde O(2^{n/2})$ (birthday); preimages in $\tilde O(2^{n})$.
%			\item \emph{Structural attacks:} differential/boomerang-style cryptanalysis on specific designs (e.g.\ MD5, SHA-1).
%			\item \emph{Chosen-prefix collisions} (practically demonstrated for SHA-1).
%			\item \emph{Length extension} against Merkle--Damg{\aa}rd hashes when used as MAC via $H(k\|m)$ (use HMAC to avoid).
%		\end{itemize}
%		&
%		\begin{itemize}\setlength\itemsep{0.15em}
%			\item \emph{Grover}: quadratic speedup for preimage search ($\tilde O(2^{n/2})$).
%			\item Quantum collision finding can also improve over birthday in some models, but collision resistance still requires larger output than preimage resistance under quantum.
%		\end{itemize}
%		\\
%		\hline
%	\end{tabular}

%	\renewcommand{\arraystretch}{1.15}
%	\setlength{\tabcolsep}{5pt}
%	
%	\begin{tabular}{|p{3.1cm}|p{7.3cm}|p{5.3cm}|}
%		\hline
%		\textbf{Problem / primitive} & \textbf{Main classical attacks (keywords)} & \textbf{Quantum attack / note} \\
%		\hline
%		Integer Factorization (IFP) &
%		GNFS; SNFS (special form); ECM (small factors); key-generation flaws (shared primes, leakage) &
%		Shor (poly-time) \\
%		\hline
%		Discrete Logarithm (DLP) &
%		BSGS; Pollard $\rho$ (generic $\tilde O(\sqrt{|G|})$); Pohlig--Hellman (smooth order); index calculus / NFS-DL (finite fields); MOV/Frey--R\"uck (special EC cases) &
%		Shor (poly-time) \\
%		\hline
%		Lattices (SVP/CVP/LWE) &
%		LLL/BKZ; enumeration; sieving; LWE primal/dual/hybrid; BKW; parameter-dependent algebraic attacks &
%		No known Shor-like poly-time; quantum can speed up some search/sieving subroutines \\
%		\hline
%		Codes (Syndrome Decoding) &
%		Information Set Decoding (Prange, Stern/Dumer, BJMM-style); structural distinguishers (if non-random structure) &
%		Grover-type speedups reduce exponents (model-dependent) \\
%		\hline
%		Isogenies (supersingular) &
%		Isogeny-graph path search; meet-in-the-middle (Delfs--Galbraith-type); protocol-specific breaks (SIDH/SIKE broken) &
%		Quantum improves path-search exponents; not known poly-time \\
%		\hline
%		Hash functions &
%		Generic: collision $\approx 2^{n/2}$ (birthday), preimage $\approx 2^{n}$; differential/chosen-prefix collisions (design-specific); length extension (Merkle--Damg{\aa}rd misuse) &
%		Grover: preimage $\approx 2^{n/2}$; quantum collision-finding improves over birthday in some models \\
%		\hline
%	\end{tabular}

%\begin{table}[h]
%	\centering
%	\renewcommand{\arraystretch}{1.5} % Adjusts row height for better readability
%	\begin{tabular}{|l|l|l|l|}
%		\hline
%		\textbf{Hard Problem} & \textbf{Cryptosystems} & \textbf{Major Attacks} & \textbf{Complexity (Best Known)} \\ \hline
%		\hline
%		\multirow{4}{*}{\textbf{Integer Factorization}} & \multirow{4}{*}{RSA, Rabin} & General Number Field Sieve & Sub-exponential (Classical) \\
%		& & (GNFS) & $L_n[1/3, \sqrt[3]{64/9}]$ \\ \cline{3-4}
%		& & Quadratic Sieve (QS) & Sub-exponential (Classical) \\
%		& & & $L_n[1/2, 1]$ \\ \cline{3-4}
%		& & Shor's Algorithm & Polynomial (Quantum) \\
%		& & & $O((\log N)^2 (\log \log N))$ \\ \hline
%		\hline
%		\multirow{4}{*}{\textbf{Discrete Logarithm}} & \multirow{4}{*}{Diffie-Hellman,} & Number Field Sieve (NFS) & Sub-exponential (Classical) \\
%		\textbf{(DLP)} & DSA, ElGamal & & $L_p[1/3, \sqrt[3]{64/9}]$ \\ \cline{3-4}
%		\text{(Finite Fields)} & & Index Calculus & Sub-exponential (Classical) \\
%		& & & $L_p[1/2, \sqrt{2}]$ \\ \cline{3-4}
%		& & Shor's Algorithm & Polynomial (Quantum) \\
%		& & & $O((\log N)^2 (\log \log N))$ \\ \hline
%		\hline
%		\multirow{4}{*}{\textbf{Elliptic Curve DLP}} & \multirow{4}{*}{ECDH, ECDSA} & Pollard's Rho & Exponential (Classical) \\
%		\textbf{(ECDLP)} & & & $O(\sqrt{n})$ \\ \cline{3-4}
%		& & MOV Attack (Supersingular) & Sub-exponential \\
%		& & (Reduces to finite field DLP) & $L_n[1/3, c]$ \\ \cline{3-4}
%		& & Shor's Algorithm & Polynomial (Quantum) \\
%		& & & $O((\log n)^2)$ \\ \hline
%		\hline
%		\multirow{3}{*}{\textbf{Lattice Problems}} & Kyber (ML-KEM), & Primal/Dual Lattice Reduction & Exponential (Classical \& Quantum) \\
%		\text{(LWE, SIS, SVP)} & Dilithium (ML-DSA), & (using BKZ, LLL) & $2^{O(d)}$ \\ \cline{3-4}
%		& NTRU, FHE & Sieve Algorithms & Exponential (Classical) \\
%		& & & $2^{O(d)}$ \\ \hline
%		\hline
%		\multirow{2}{*}{\textbf{Isogeny Problems}} & CSIDH, & Meet-in-the-Middle & Exponential (Classical) \\
%		\text{(Path Finding)} & SQISign & Kuperberg's Algorithm & Sub-exponential (Quantum) \\
%		& & & $L[1/2]$ (commutative groups) \\ \hline
%	\end{tabular}
%	\caption{Summary of Hard Problems in Cryptography and Attacks}
%	\label{tab:crypto_problems}
%\end{table}

\begin{longtable}{|p{3.0cm}|p{2.8cm}|p{4.2cm}|p{4.0cm}|}
	\hline
	\textbf{Hard Problem} & \textbf{Cryptosystems} & \textbf{Major Attacks} & \textbf{Complexity (Best Known)} \\ \hline
	\endhead
	
	\hline
	\multicolumn{4}{|r|}{{Continued on next page}} \\ \hline
	\endfoot
	
	\hline
	\endlastfoot
	
	% 1. Integer Factorization
	\hline
	\multirow{5}{=}{\textbf{Integer Factorization}} 
	& \multirow{5}{*}{RSA, Rabin} 
	& General Number Field Sieve (GNFS) 
	& Sub-exponential (Classical) \newline $L_n[1/3, \sqrt[3]{64/9}]$ \\ \cline{3-4}
	& & Quadratic Sieve (QS) 
	& Sub-exponential (Classical) \newline $L_n[1/2, 1]$ \\ \cline{3-4}
	& & Shor's Algorithm 
	& Polynomial (Quantum) \newline $O((\log N)^2)$ \\ \hline
	
	% 2. Discrete Logarithm (DLP)
	\hline
	\multirow{5}{=}{\textbf{Discrete Logarithm (DLP)} \newline \textit{(Finite Fields)}} 
	& \multirow{5}{=}{Diffie-Hellman, DSA, ElGamal} 
	& Number Field Sieve (NFS) 
	& Sub-exponential (Classical) \newline $L_p[1/3, \sqrt[3]{64/9}]$ \\ \cline{3-4}
	& & Index Calculus 
	& Sub-exponential (Classical) \newline $L_p[1/2, \sqrt{2}]$ \\ \cline{3-4}
	& & Shor's Algorithm 
	& Polynomial (Quantum) \newline $O((\log N)^2)$ \\ \hline
	
	% 3. Elliptic Curve DLP (ECDLP)
	\hline
	\multirow{4}{=}{\textbf{Elliptic Curve DLP (ECDLP)}} 
	& \multirow{4}{=}{ECDH, ECDSA, EdDSA} 
	& Pollard's Rho / Kangaroo 
	& Exponential (Classical) \newline $O(\sqrt{n})$ \\ \cline{3-4}
	& & MOV Attack (Supersingular curves only) 
	& Reduces to Finite Field DLP \newline $L_n[1/3, c]$ \\ \cline{3-4}
	& & Shor's Algorithm 
	& Polynomial (Quantum) \newline $O((\log n)^2)$ \\ \hline
	
	% 4. Lattice Problems
	\hline
	\multirow{4}{=}{\textbf{Lattice Problems} \newline (LWE, SIS, SVP, CVP)} 
	& Kyber (ML-KEM), Dilithium (ML-DSA), NTRU, FHE 
	& Lattice Reduction (LLL, BKZ) 
	& Exponential (Classical/Quantum) \newline $2^{O(d)}$ \\ \cline{3-4}
	& & Sieve Algorithms 
	& Exponential (Classical) \newline $2^{O(d)}$ \\ \hline
	
	% 5. Code-Based Problems
	\hline
	\multirow{4}{=}{\textbf{Code-Based} \newline (Syndrome Decoding)} 
	& McEliece, HQC, BIKE 
	& Information Set Decoding (ISD) \newline (e.g., Stern, Lee-Brickell) 
	& Exponential (Classical/Quantum) \newline $2^{O(n / \log n)}$ \\ \cline{3-4}
	& & Grover's Search 
	& Quadratic Speedup only \newline (Halves the exponent) \\ \hline
	
	% 6. Isogeny Problems
	\hline
	\multirow{5}{=}{\textbf{Isogeny Problems}} 
	& \multirow{5}{=}{SQISign, CSIDH \newline \textit{(SIKE is broken)}} 
	& Castryck-Decru Attack \newline (Specific to SIDH/SIKE) 
	& Polynomial (Classical) \newline \textbf{Broken} \\ \cline{3-4}
	& & Meet-in-the-Middle / Delfs-Galbraith 
	& Exponential (Classical) \newline $O(p^{1/4})$ \\ \cline{3-4}
	& & Kuperberg's Algorithm 
	& Sub-exponential (Quantum) \newline $L[1/2]$ (Commutative only) \\ \hline
	
	% 7. Hash-Based
	\hline
	\multirow{4}{=}{\textbf{Hash-Based} \newline (Collision / Preimage)} 
	& \multirow{4}{=}{SPHINCS+, XMSS, LMS} 
	& Birthday Attack (Collision) 
	& Exponential \newline $O(2^{n/2})$ \\ \cline{3-4}
	& & Grover's Algorithm (Preimage) 
	& Exponential (Quantum) \newline $O(2^{n/2})$ \\ \hline
	
	% 8. Multivariate
	\hline
	\multirow{4}{=}{\textbf{Multivariate (MQ)} \newline (Solving Quadratic Systems)} 
	& UOV, MAYO \newline \textit{(Rainbow is broken)} 
	& MinRank / Kipnis-Shamir 
	& Exponential (Classical) \\ \cline{3-4}
	& & XL Algorithm (Groebner Basis) 
	& Exponential (generally) \\ \cline{3-4}
	& & Beullens' Attack \newline (Specific to Rainbow) 
	& Polynomial (Classical) \newline \textbf{Broken} \\ \hline
	
	\caption{Comprehensive Summary of Hard Cryptographic Problems and Attacks}
	\label{tab:crypto_comprehensive}
\end{longtable}

\newpage
% ============================================================
% Recommended packages (preamble)
% \usepackage{longtable}
% \usepackage{multirow}
% \usepackage{array}
% ============================================================

% Notation for subexponential L-notation:
% L_N[\alpha,c] := \exp\!\big((c+o(1))(\log N)^\alpha(\log\log N)^{1-\alpha}\big)

\begin{longtable}{|p{3.0cm}|p{3.0cm}|p{4.4cm}|p{3.6cm}|}
	\hline
	\textbf{Hard Problem} & \textbf{Cryptosystems} & \textbf{Major Attacks} & \textbf{Complexity (best known)} \\
	\hline
	\endhead
	
	\hline
	\multicolumn{4}{|r|}{{Continued on next page}} \\
	\hline
	\endfoot
	
	\hline
	\endlastfoot
	
	% ------------------------------------------------------------
	% 1. Integer Factorization
	% ------------------------------------------------------------
	\hline
	\multirow{4}{=}{\textbf{Integer Factorization}} 
	& \multirow{4}{=}{RSA, Rabin} 
	& General Number Field Sieve (GNFS) 
	& Sub-exponential (classical)\newline $L_N\!\left[ \tfrac{1}{3},\sqrt[3]{\tfrac{64}{9}} \right]$ \\ \cline{3-4}
	& & Quadratic Sieve (QS) 
	& Sub-exponential (classical)\newline $L_N\!\left[\tfrac{1}{2},1\right]$ \\ \cline{3-4}
	& & ECM (small prime factors) 
	& Heuristic $\approx \exp\!\big(\sqrt{2\log p\log\log p}\big)$ for factor $p$ \\ \cline{3-4}
	& & Shor (quantum) 
	& Polynomial in $\log N$ (quantum)\newline $\mathrm{poly}(\log N)$ \\ 
	\hline
	
	% ------------------------------------------------------------
	% 2. DLP in finite fields
	% ------------------------------------------------------------
	\hline
	\multirow{4}{=}{\textbf{Discrete Logarithm (DLP)}\\\textit{(finite fields)}} 
	& \multirow{4}{=}{Diffie--Hellman,\newline DSA, ElGamal} 
	& Number Field Sieve for DL (NFS-DL) 
	& Sub-exponential (classical)\newline $L_p\!\left[ \tfrac{1}{3},\sqrt[3]{\tfrac{64}{9}} \right]$ \\ \cline{3-4}
	& & Index Calculus (classical; group-dependent) 
	& Sub-exponential (classical)\newline typically $L_p\!\left[\tfrac{1}{2},c\right]$ \\ \cline{3-4}
	& & Pohlig--Hellman (if $|G|$ smooth) 
	& Polynomial in $\log p$ given smooth factorization of $|G|$ \\ \cline{3-4}
	& & Shor (quantum) 
	& Polynomial in $\log p$ (quantum)\newline $\mathrm{poly}(\log p)$ \\
	\hline
	
	% ------------------------------------------------------------
	% 3. ECDLP
	% ------------------------------------------------------------
	\hline
	\multirow{3}{=}{\textbf{Elliptic Curve DLP (ECDLP)}} 
	& \multirow{3}{=}{ECDH,\newline ECDSA, EdDSA} 
	& Pollard $\rho$ / Kangaroo (generic) 
	& Exponential (classical)\newline $\tilde O(\sqrt{n})$ group ops \\ \cline{3-4}
	& & MOV / Frey--R\"uck (special curves only) 
	& Reduces to finite-field DLP (curve-dependent) \\ \cline{3-4}
	& & Shor (quantum) 
	& Polynomial in $\log n$ (quantum)\newline $\mathrm{poly}(\log n)$ \\
	\hline
	
	% ------------------------------------------------------------
	% 4. Lattices (SVP/CVP/LWE/SIS)
	% ------------------------------------------------------------
	\hline
	\multirow{4}{=}{\textbf{Lattice Problems}\\(SVP, CVP, LWE, SIS)} 
	& Kyber (ML-KEM), Dilithium (ML-DSA),\newline NTRU, FHE 
	& BKZ/LLL basis reduction (core primitive) 
	& Parameterized by dimension $d$:\newline typically $2^{\Theta(d)}$ (classical) \\ \cline{3-4}
	& & Enumeration (primal) / Dual attacks (LWE) 
	& $2^{\Theta(d)}$ (classical), constants depend on BKZ blocksize \\ \cline{3-4}
	& & Sieving for SVP (heuristic) 
	& $\approx 2^{c d}$ with known $c<1$ (classical heuristic) \\ \cline{3-4}
	& & Quantum speedups (search/sieving subroutines) 
	& No known Shor-like poly-time; typically improves constants/exponents in $2^{\Theta(d)}$ \\
	\hline
	
	% ------------------------------------------------------------
	% 5. Codes (Syndrome Decoding)
	% ------------------------------------------------------------
	\hline
	\multirow{3}{=}{\textbf{Code-Based}\\(Syndrome Decoding)} 
	& McEliece,\newline HQC, BIKE 
	& Information Set Decoding (ISD)\newline (Prange, Stern, Dumer, BJMM, \dots) 
	& Exponential (classical)\newline $\approx 2^{\Theta(n)}$ (instance/params dependent) \\ \cline{3-4}
	& & Structural attacks (if code not pseudorandom) 
	& Often polynomial if exploitable structure exists \\ \cline{3-4}
	& & Grover-type quantum speedups 
	& Typically reduces brute-force layers (often ``halves'' exponents in idealized models) \\
	\hline
	
	% ------------------------------------------------------------
	% 6. Isogenies (supersingular/abelian class group actions)
	% ------------------------------------------------------------
	\hline
	\multirow{4}{=}{\textbf{Isogeny Problems}} 
	& SQISign, CSIDH\newline \textit{(SIDH/SIKE broken)} 
	& Castryck--Decru (SIDH/SIKE-specific) 
	& Polynomial-time key recovery for those schemes\newline \textbf{Broken (protocol-specific)} \\ \cline{3-4}
	& & Meet-in-the-middle / Delfs--Galbraith (path finding) 
	& Heuristic $\tilde O(p^{1/4})$ for supersingular path-finding variants \\ \cline{3-4}
	& & Classical random-walk / graph search 
	& Roughly $\tilde O(p^{1/2})$ in naive models \\ \cline{3-4}
	& & Kuperberg-type (quantum; abelian hidden shift) 
	& Sub-exponential (quantum)\newline $\exp\!\big(O(\sqrt{\log p\,\log\log p})\big)$ (commutative settings) \\
	\hline
	
	% ------------------------------------------------------------
	% 7. Hash
	% ------------------------------------------------------------
	\hline
	\multirow{3}{=}{\textbf{Hash-Based}\\(Collision / Preimage)} 
	& SPHINCS+, XMSS, LMS 
	& Birthday attack (collisions) 
	& $\Theta(2^{n/2})$ evaluations for $n$-bit outputs \\ \cline{3-4}
	& & Generic preimage search 
	& $\Theta(2^{n})$ evaluations (classical) \\ \cline{3-4}
	& & Grover (quantum preimage) 
	& $\Theta(2^{n/2})$ quantum queries (idealized) \\
	\hline
	
	% ------------------------------------------------------------
	% 8. Multivariate (MQ)
	% ------------------------------------------------------------
	\hline
	\multirow{4}{=}{\textbf{Multivariate (MQ)}\\(Quadratic systems)} 
	& UOV, MAYO\newline \textit{(Rainbow broken)} 
	& Gr\"obner basis (F4/F5), XL / relinearization 
	& Exponential in $n$ in general;\newline governed by degree of regularity \\ \cline{3-4}
	& & Hybrid attacks (guess variables + algebraic solve) 
	& Exponential; trades time for memory / guessing \\ \cline{3-4}
	& & MinRank / Kipnis--Shamir (scheme-structure dependent) 
	& Often subexponential-to-exponential; can be polynomial for weak parameters \\ \cline{3-4}
	& & Beullens-type attacks (Rainbow-specific) 
	& Practical/polynomial-time breaks for Rainbow variants\newline \textbf{Broken (protocol-specific)} \\
	\hline
	
	\caption{Hard problems in cryptography: canonical systems, major attacks, and best-known asymptotic complexities (high-level).}
	\label{tab:crypto_comprehensive_refined}
\end{longtable}

	
	\newpage
	\section{Preliminaries}
	% ============================================================
% Preliminaries (mathematician-friendly) for common
% cryptographic hard problems and primitives.
% ============================================================

%\section*{Preliminaries}

\subsection*{Security parameter and asymptotics}

\paragraph{Security parameter.}
Cryptographic families are indexed by a security parameter $\lambda\in\mathbb{N}$.
All objects (groups, moduli, dimensions, etc.) are efficiently generated from $1^\lambda$.
A function $\mu:\mathbb{N}\to \mathbb{R}_{\ge 0}$ is \emph{negligible} if for every $c>0$,
there exists $\lambda_0$ such that for all $\lambda\ge \lambda_0$,
\[
\mu(\lambda) < \lambda^{-c}.
\]
An event happens with \emph{non-negligible} probability if its probability is not negligible.

\paragraph{Efficient algorithms and PPT adversaries.}
A \emph{probabilistic polynomial-time} (PPT) algorithm $\mathcal{A}$ is a randomized algorithm
running in time $\mathrm{poly}(\lambda)$ on inputs generated at security level $\lambda$.
Probabilities are taken over the internal randomness of $\mathcal{A}$ and over all random choices
made by experiment distributions.

\subsection*{Basic algebraic structures}

\paragraph{Rings and fields.}
$\mathbb{Z}$ denotes the integers. For $q\ge 2$,
\[
\mathbb{Z}_q := \mathbb{Z}/q\mathbb{Z}
\]
is the ring of integers modulo $q$. If $q=p$ is prime, then $\mathbb{Z}_p \cong \mathbb{F}_p$ is a field.
For a finite field $\mathbb{F}_q$ (with $q=p^r$), $\mathbb{F}_q^\times := \mathbb{F}_q\setminus\{0\}$
denotes its multiplicative group.

\paragraph{Groups.}
A (finite) group is a pair $(G,\circ)$ with associative operation, identity element $e$,
and inverses for all elements. A group is \emph{cyclic} if $G=\langle g\rangle$ for some $g\in G$.
If $|G|=n$, then exponents are interpreted modulo $n$:
\[
g^x := \underbrace{g\circ\cdots\circ g}_{x \text{ times}} \quad\text{and}\quad g^{x+n}=g^x.
\]

\paragraph{Homomorphisms.}
A map $\varphi:G\to H$ between groups is a homomorphism if
$\varphi(x\circ_G y)=\varphi(x)\circ_H \varphi(y)$.
Its kernel is $\ker(\varphi)=\{x\in G:\varphi(x)=e_H\}$.

\subsection*{Probability and sampling notation}

\paragraph{Uniform sampling.}
For a finite set $S$, the notation $x\xleftarrow{\$} S$ means $x$ is sampled uniformly from $S$.
More generally, $x\leftarrow \mathcal{D}$ means $x$ is sampled from distribution $\mathcal{D}$.

\paragraph{Advantage in a distinguishing task.}
For distributions $\mathcal{D}_0,\mathcal{D}_1$ on a common sample space and a (randomized) distinguisher
$\mathcal{A}$ outputting a bit, define
\[
\mathrm{Adv}_{\mathcal{A}}(\mathcal{D}_0,\mathcal{D}_1)
:= \left|\Pr_{x\leftarrow \mathcal{D}_0}[\mathcal{A}(x)=1]-\Pr_{x\leftarrow \mathcal{D}_1}[\mathcal{A}(x)=1]\right|.
\]
A distinguishing advantage is \emph{negligible} if it is negligible in $\lambda$.

\subsection*{Bitstrings and encodings}

\paragraph{Bitstrings.}
$\{0,1\}^\ast$ is the set of all finite bitstrings; $\{0,1\}^n$ denotes bitstrings of length $n$.
Concrete mathematical objects (integers, group elements, matrices) are assumed to have fixed,
efficient encodings as bitstrings, so they can be given to algorithms.

\subsection*{Integers and arithmetic}

\paragraph{Divisibility and gcd.}
For $a,b\in\mathbb{Z}$, $a\mid b$ means $\exists k\in\mathbb{Z}$ with $b=ak$.
The greatest common divisor is $\gcd(a,b)$.

\paragraph{RSA-type moduli.}
A common distribution for factorization hardness is
\[
N=pq
\]
where $p,q$ are distinct random primes of prescribed bit-length.

\subsection*{Linear algebra over finite fields}

\paragraph{Vector spaces.}
For a finite field $\mathbb{F}_q$, $\mathbb{F}_q^n$ is an $n$-dimensional vector space.
Matrices $A\in\mathbb{F}_q^{m\times n}$ act on vectors by multiplication.

\paragraph{Inner product modulo $q$.}
For $a,s\in \mathbb{Z}_q^n$,
\[
\langle a,s\rangle := \sum_{i=1}^n a_i s_i \pmod q.
\]

\subsection*{Normed spaces and geometry of numbers}

\paragraph{Euclidean norm.}
For $x\in\mathbb{R}^n$,
\[
\|x\|_2 := \sqrt{\sum_{i=1}^n x_i^2}.
\]
(Other norms, e.g.\ $\|\cdot\|_\infty$, may be used depending on the lattice problem.)

\paragraph{Distance to a set.}
For $t\in\mathbb{R}^n$ and $S\subseteq\mathbb{R}^n$,
\[
\mathrm{dist}(t,S) := \inf_{x\in S}\|t-x\|.
\]

\subsection*{Lattices (basic definitions used later)}

\paragraph{Lattice and basis.}
A full-rank lattice $\mathcal{L}\subset \mathbb{R}^n$ is
\[
\mathcal{L}(B)=\{Bz: z\in \mathbb{Z}^n\}
\]
for an invertible matrix $B\in\mathbb{R}^{n\times n}$ whose columns form a basis.

\paragraph{Determinant / covolume.}
$\det(\mathcal{L}) := |\det(B)|$ is independent of the chosen basis and equals the volume of a
fundamental parallelepiped.

\paragraph{Successive minima.}
The first successive minimum is
\[
\lambda_1(\mathcal{L}) := \min\{\|x\|_2 : x\in \mathcal{L}\setminus\{0\}\}.
\]

\subsection*{Coding theory preliminaries}

\paragraph{Hamming weight and distance.}
For $x\in\mathbb{F}_q^n$,
\[
w_H(x)=|\{i: x_i\neq 0\}|,
\qquad
d_H(x,y)=w_H(x-y).
\]

\paragraph{Linear codes.}
A linear $[n,k]_q$ code is a $k$-dimensional subspace $C\subseteq \mathbb{F}_q^n$.
Generator and parity-check descriptions are equivalent:
\[
C=\{uG:u\in\mathbb{F}_q^k\}
\quad\text{and}\quad
C=\{c\in\mathbb{F}_q^n:Hc^\top=0\}.
\]

\subsection*{Elliptic curves and finite-field preliminaries}

\paragraph{Finite fields and extensions.}
For prime $p$, $\mathbb{F}_{p^2}$ is the degree-$2$ extension of $\mathbb{F}_p$.
Elliptic curves in isogeny-based cryptography are often defined over $\mathbb{F}_{p^2}$.

\paragraph{Elliptic curves (minimal facts).}
An elliptic curve $E/\mathbb{F}_q$ is a smooth projective genus-one curve with a chosen base point,
whose $\mathbb{F}_q$-rational points $E(\mathbb{F}_q)$ form a finite abelian group.

\paragraph{Morphisms and degree.}
A nonconstant rational map between curves has an associated (algebraic) degree.
An \emph{isogeny} is a morphism $E_1\to E_2$ that is also a group homomorphism.

\subsection*{Hash-function preliminaries}

\paragraph{Function families.}
A hash is typically modeled as a family $\{H_\lambda\}$ where each
\[
H_\lambda:\{0,1\}^\ast\to \{0,1\}^{n(\lambda)}
\]
is efficiently computable.

\paragraph{Search vs.\ decision vs.\ distinguishing.}
Many hardness notions can be expressed as:
\begin{itemize}
	\item \emph{Search}: output a witness (e.g.\ a factor, a discrete log, an error vector).
	\item \emph{Decision}: decide existence of a witness.
	\item \emph{Distinguishing}: tell apart two distributions (e.g.\ LWE vs.\ uniform).
\end{itemize}

% ============================================================
% End preliminaries
% ============================================================

	
	\newpage
	\section{Integer Factorization Problem (IFP)}
	%\paragraph{Setting.}
%Let $N\in\mathbb{Z}_{\ge 2}$ be an odd composite integer, typically $N=pq$ with distinct primes
%$p,q$ of comparable bit-length.
%
%\paragraph{Search problem (Factorization).}
%Given $N$, output a nontrivial factor of $N$, i.e.\ an integer $d$ such that
%\[
%1< d < N
%\quad\text{and}\quad
%d \mid N.
%\]
%Equivalently, output the full prime factorization of $N$.
%
%\paragraph{Decision variant.}
%Given $(N,d)$, decide whether $d\mid N$ and $1<d<N$ (trivial), or more meaningfully:
%given $N$, decide whether $N$ is prime (primality) or has a factor in a specified interval.
%
%\paragraph{Hardness assumption (typical cryptographic form).}
%For a distribution $\mathcal{D}$ over composites (e.g.\ RSA moduli $N=pq$),
%no probabilistic polynomial-time (PPT) algorithm factors $N\leftarrow\mathcal{D}$
%with non-negligible probability.

\paragraph{Search problem (Factorization).}
Given an odd composite integer $N\in\mathbb{Z}_{\ge 2}$ (often $N=pq$ with distinct primes $p,q$),
output a nontrivial factor $d$ such that
\[
1<d<N \quad\text{and}\quad d\mid N.
\]
Equivalently, output the prime factorization of $N$.

\paragraph{Hardness assumption (RSA distribution).}
Let $\mathcal{D}_\lambda$ output $N=pq$ where $p,q$ are random $\lambda$-bit primes.
The assumption states: no PPT algorithm factors $N\leftarrow \mathcal{D}_\lambda$
with non-negligible probability in $\lambda$.

\paragraph{Standard attacks (classical).}
\begin{itemize}\setlength\itemsep{0.2em}
	\item \textbf{Trial division / Pollard $p-1$}: effective when $p-1$ is smooth.
	\item \textbf{Pollard $\rho$}: heuristic $O(\sqrt{p})$ time to find a factor $p$.
	\item \textbf{ECM (Elliptic Curve Method)}: best for finding relatively small prime factors.
	\item \textbf{QS (Quadratic Sieve)}: subexponential; good for moderate sizes.
	\item \textbf{GNFS (General Number Field Sieve)}: asymptotically fastest for general $N$.
	\item \textbf{SNFS (Special NFS)}: faster than GNFS when $N$ has special form.
	\item \textbf{Implementation/key-gen weaknesses}: shared primes, partial key leakage, smoothness, side-channels.
\end{itemize}

\paragraph{Quantum attack.}
\textbf{Shor's algorithm} factors in time polynomial in $\log N$ on a fault-tolerant quantum computer.

	
	\newpage
	\section{Discrete Logarithm Problem (DLP)}
	%\paragraph{Setting.}
%Let $G$ be a finite cyclic group written multiplicatively, with generator $g\in G$.
%Let $|G|=n$.
%
%\paragraph{Search problem (Discrete logarithm).}
%Given $g$ and $h\in \langle g\rangle = G$, find an integer $x\in \mathbb{Z}_n$ such that
%\[
%g^{x} = h.
%\]
%Any such $x$ is unique modulo $n$ and is denoted $\log_g(h)$.
%
%\paragraph{Computational DLP.}
%The above search problem; often instantiated in $G=\mathbb{F}_p^\times$ or an elliptic-curve group.
%
%\paragraph{Hardness assumption.}
%For a family of groups $\{G_\lambda\}$ indexed by security parameter $\lambda$,
%no PPT algorithm solves DLP in $G_\lambda$ with non-negligible probability over random
%$g$ (generator) and random $x\leftarrow \mathbb{Z}_{|G_\lambda|}$, where $h=g^x$.

\paragraph{Setting.}
Let $G$ be a finite cyclic group (written multiplicatively) of order $n$, with generator $g\in G$.

\paragraph{Search DLP.}
Given $g$ and $h\in G$, find $x\in \mathbb{Z}_n$ such that
\[
g^x = h.
\]
The solution is unique modulo $n$ and is denoted $x=\log_g(h)$.

\paragraph{Hardness assumption.}
For a family $\{G_\lambda\}$, no PPT algorithm recovers $x$ from $(g,h=g^x)$
with non-negligible probability over random $g$ (generator) and random $x$.

\paragraph{Standard attacks (classical).}
\begin{itemize}\setlength\itemsep{0.2em}
	\item \textbf{Generic attacks} (all groups): baby-step/giant-step; Pollard $\rho$ in $\tilde O(\sqrt{n})$ time.
	\item \textbf{Pohlig--Hellman}: reduces DLP to prime-power factors of $n$; devastating if $n$ is smooth.
	\item \textbf{Index calculus} (finite fields): subexponential; includes \textbf{NFS-DL} variants.
	\item \textbf{Special-curve pitfalls} (elliptic curves): MOV/Frey--R\"uck reductions for pairing-friendly/special curves.
	\item \textbf{Side-channel/implementation}: timing, power, fault attacks against exponentiation/scalar multiplication.
\end{itemize}

\paragraph{Quantum attack.}
\textbf{Shor's algorithm} solves DLP in abelian groups in polynomial time (in $\log n$).

	
	\newpage
	\section{Lattices (SVP/CVP and LWE)}
	\chapter{Lattice-Based Hard Problems}

\section{Background: Lattices, Duality, and Problems}
A (full-rank) lattice $\Lambda \subset \mathbb{R}^n$ is $\Lambda = \{Bz : z\in\Z^n\}$ for some basis matrix $B\in\mathbb{R}^{n\times n}$.

Classic algorithmic problems:
\begin{itemize}[leftmargin=*]
	\item \textbf{SVP (Shortest Vector Problem):} Find $0\neq v\in\Lambda$ minimizing $\norm{v}_2$.
	\item \textbf{GapSVP (Decision/SVP approximation):} Given $(\Lambda, d)$ decide whether $\lambda_1(\Lambda)\le d$ or $\lambda_1(\Lambda)> \gamma d$.
	\item \textbf{SIVP (Shortest Independent Vectors):} Find $n$ linearly independent vectors of length $\le \gamma\cdot \lambda_n(\Lambda)$.
\end{itemize}
The importance for cryptography: average-case problems (LWE/SIS) reduce from worst-case lattice problems (GapSVP/SIVP) under suitable parameters.

\section{Learning With Errors (LWE)}

\subsection{Formal Statements}
\begin{definition}[Search-LWE]
	Fix integers $n,m,q\in\mathbb{N}$ and an error distribution $\chi$ over $\Z$ (typically supported on small integers).
	Sample $A\from \Z_q^{m\times n}$ uniformly, secret $s\from \Z_q^n$ (usually uniform), and error $e\from \chi^m$.
	Given $(A,b)$ where
	\[
	b = As + e \bmod q \in \Z_q^m,
	\]
	output the secret $s$ (or equivalently recover $e$).
\end{definition}

\begin{definition}[Decision-LWE]
	Under the same parameterization, consider two distributions over $(A,b)$:
	\[
	\calD_0: (A, As+e \bmod q), \qquad
	\calD_1: (A, u),\ \ u\from \Z_q^m.
	\]
	Given $(A,b)$, output a bit indicating whether $(A,b)\from\calD_0$ or $(A,b)\from\calD_1$ with non-negligible advantage.
\end{definition}

\subsection{Geometric / Statistical Intuition}
Each equation is:
\[
\langle a_i, s \rangle + e_i \equiv b_i \pmod q.
\]
If errors were $0$, this is solving a linear system over $\Z_q$. Errors make it an instance of \emph{noisy linear equations}, and (crucially) hide $s$.

A typical heuristic: if $e$ is small in $\Z$ and $q$ is large, the mapping $s\mapsto As+e$ looks like ``random'' without knowing $s$, but still allows decryption by rounding in cryptosystems.

\subsection{Decision vs Search; Standard Relationships}
Cryptographic constructions often assume decision-LWE hardness (for pseudorandomness) and search-LWE hardness (for extracting secrets). Under many standard parameter regimes, one can relate them:

\begin{remark}[Search-to-decision (informal)]
	For prime $q$, there are classical reductions showing decision-LWE is no harder than search-LWE and vice versa (up to losses), under mild conditions. Intuitively, if you can recover $s$ then you can distinguish; conversely, if you can distinguish, you can often recover $s$ coordinate-by-coordinate using hybrid and rerandomization tricks.
\end{remark}

\subsection{Worst-Case to Average-Case Reductions (High-Level)}
A landmark result (Regev-style) shows that (for appropriate $\alpha$ where errors have size about $\alpha q$) decision-LWE is at least as hard as approximating worst-case lattice problems (GapSVP/SIVP) in dimension $n$ within poly factors. The technical conditions tie $\alpha$, $q$, and $n$.

\begin{remark}[What you should remember]
	The security story is: \emph{if LWE is easy on average, then certain canonical lattice problems are easy in the worst case}. This is why LWE is a central conservative assumption.
\end{remark}

\subsection{Parameter Regimes (Conceptual)}
Let error magnitude scale be $\sigma$ (e.g., standard deviation for discrete Gaussian). Often one defines $\alpha = \sigma/q$.
\begin{itemize}[leftmargin=*]
	\item \textbf{Correctness in encryption}: needs $\sigma \ll q$ so small errors can be rounded.
	\item \textbf{Security}: needs $\sigma$ large enough that $As+e$ hides $s$; also $m$ large enough to prevent solving.
	\item \textbf{Typical cryptosystems}: use $m\approx n\log q$ or $m$ a constant multiple of $n$ in module/ring variants.
\end{itemize}

\subsection{Attack Taxonomy (What to Teach)}
Best-known attacks broadly fall into:
\begin{itemize}[leftmargin=*]
	\item \textbf{Lattice reduction (primal):} view LWE as finding a close vector / BDD instance; build a lattice from $A$ and $b$ and run BKZ-type reduction; recover $s$ by nearest-plane or enumeration.
	\item \textbf{Dual attack:} find a short vector $y$ in the dual lattice such that $y^\top A \equiv 0$ mod $q$, then test $y^\top b$ for smallness vs uniform.
	\item \textbf{BKW / combinatorial:} reduce dimension via collision-finding on $A$ rows; grows fast with $q$ and noise but can matter for small moduli.
	\item \textbf{Arora--Ge (algebraic):} for very small error alphabets and special parameter settings, solve polynomial system.
\end{itemize}

\subsection{Exercises (LWE)}
\begin{exercise}[Upper-undergrad: noiseless baseline]
	Assume $e=0$ and $m\ge n$. Show how to recover $s$ efficiently from $(A,As\bmod q)$ when $q$ is prime and $A$ has full rank.
\end{exercise}

\begin{exercise}[Masters: distinguishing via dual vector]
	Let $y\in \Z^m$ satisfy $y^\top A\equiv 0\pmod q$. Show that if $(A,b)\from \calD_0$ then
	\[
	y^\top b \equiv y^\top e \pmod q,
	\]
	and argue heuristically why $y^\top b$ is statistically closer to small integers mod $q$ than uniform if $y$ is short.
\end{exercise}

\begin{exercise}[PhD: hybrid for search-to-decision sketch]
	Assume $q$ prime. Outline a reduction strategy that recovers $s_i$ (the $i$-th coordinate of $s$) using a decision oracle by embedding a guess into the distribution and using hybrids.
\end{exercise}

\section{Short Integer Solution (SIS)}

\subsection{Formal Statement}
\begin{definition}[Search-SIS]
	Let $q\in\mathbb{N}$, $n,m\in\mathbb{N}$, and bound $\beta\in\mathbb{N}$.
	Sample $A\from \Z_q^{n\times m}$ uniformly.
	Find a nonzero vector $x\in \Z^m\setminus\{0\}$ such that
	\[
	Ax \equiv 0 \pmod q
	\quad\text{and}\quad
	\norm{x}\le \beta,
	\]
	where $\norm{\cdot}$ is typically $\ell_2$ or $\ell_\infty$.
\end{definition}

\subsection{Interpretation as Finding Short Relations}
The condition $Ax\equiv 0\pmod q$ means $x$ is an integer relation among the columns of $A$ modulo $q$. Without the shortness constraint, there are many solutions. The hardness is to find a \emph{short} one.

\subsection{Connection to Hash-and-Sign / Commitments}
SIS underlies:
\begin{itemize}[leftmargin=*]
	\item lattice-based hash functions: mapping $x\mapsto Ax\bmod q$; collisions correspond to short $x$ with $Ax\equiv 0$.
	\item commitments: binding reduces to SIS.
	\item signatures (e.g., GPV-style): produce short preimages under a public linear map.
\end{itemize}

\subsection{Worst-Case Reductions (High-Level)}
SIS is related to worst-case lattice problems as well: if SIS is easy for certain $(n,m,q,\beta)$, then approximating certain lattice problems is easy in the worst case. The parameter tradeoff differs from LWE.

\subsection{Attack Taxonomy}
\begin{itemize}[leftmargin=*]
	\item \textbf{Lattice reduction:} interpret SIS as finding a short vector in a lattice of solutions; build lattice basis and run BKZ.
	\item \textbf{Combinatorial / meet-in-the-middle:} sometimes applicable for $\ell_\infty$ and special constraints (rare in standard parameters).
\end{itemize}

\subsection{Exercises (SIS)}
\begin{exercise}[Upper-undergrad: pigeonhole existence]
	Let $A\in\Z_q^{n\times m}$ and consider all $x\in\{0,1\}^m$. Show that if $2^m > q^n$, then there exist distinct $x\neq x'$ with $Ax\equiv Ax'\pmod q$. Deduce existence of a nonzero $\{-1,0,1\}^m$ solution to $A(x-x')\equiv 0$.
\end{exercise}

\begin{exercise}[Masters: collision-resistance from SIS]
	Define $H(x)=Ax\bmod q$ for short $x$ in some domain. Formalize how a collision $(x\neq x')$ yields an SIS solution.
\end{exercise}

\begin{exercise}[PhD: parameter reasoning]
	For fixed $n,q$, explain qualitatively why increasing $m$ makes SIS \emph{easier} (more relations exist), but also allows setting smaller $\beta$ while maintaining existence of solutions.
\end{exercise}

\section{NTRU Search Problem}

\subsection{Ring Setting}
Let $f(x)$ be a cyclotomic-like polynomial (e.g., $x^N+1$ with $N$ power of 2, or $x^N-1$ for classical NTRU variants). Define
\[
R = \Z[x]/(f(x)),\qquad
R_q = R / qR \cong \Z_q[x]/(f(x)).
\]
Elements are represented by degree-$<N$ polynomials; ``small'' typically refers to small coefficients.

\subsection{Formal Problem (One Common Form)}
\begin{definition}[NTRU Search (informal canonical form)]
	Sample $f,g \in R$ from a ``small'' distribution such that $f$ is invertible in $R_q$.
	Publish
	\[
	h \equiv g f^{-1} \pmod q \in R_q.
	\]
	Given $h$, recover a short pair $(f,g)$ (or an equivalent short representation) satisfying $h f \equiv g \pmod q$ under the same smallness constraints.
\end{definition}

\subsection{Key Ambiguities in NTRU Statements}
NTRU has many instantiations; the exact hardness depends on:
\begin{itemize}[leftmargin=*]
	\item ring choice ($x^N\pm 1$; $N$ prime; etc.),
	\item modulus structure (prime $q$ vs power-of-two),
	\item distribution of $(f,g)$ (ternary, Gaussian, centered binomial),
	\item norm and acceptance region (e.g., $\ell_\infty$ bounds on coefficients),
	\item whether we are in \emph{ring} vs \emph{module} setting.
\end{itemize}

\subsection{NTRU as a Lattice Problem}
Given $h$, consider the \emph{NTRU lattice}:
\[
\Lambda_h = \left\{(u,v)\in R^2 : u - hv \equiv 0 \pmod q \right\}.
\]
A secret key corresponds to a short vector $(g,f)\in \Lambda_h$. Thus, breaking NTRU is (at high level) a shortest-vector style task in a structured lattice.

\subsection{Attack Taxonomy}
\begin{itemize}[leftmargin=*]
	\item \textbf{Lattice reduction on NTRU lattice:} embed $\Lambda_h$ into an integer lattice of dimension $2N$ and use BKZ; recover short $(f,g)$.
	\item \textbf{Hybrid attacks:} partial guessing of coefficients + lattice reduction for the remaining.
	\item \textbf{Subfield / algebraic structure attacks:} exploit ring structure if parameters are ill-chosen (historically important lesson: structure can leak).
\end{itemize}

\subsection{Exercises (NTRU)}
\begin{exercise}[Upper-undergrad: derive the NTRU relation]
	Show that $h \equiv gf^{-1}\pmod q$ implies $hf\equiv g\pmod q$. Explain why small $(f,g)$ is a ``short relation'' between $1$ and $h$.
\end{exercise}

\begin{exercise}[Masters: NTRU lattice membership]
	Define $\Lambda_h$ as above. Prove that $(g,f)\in \Lambda_h$. What other pairs are in $\Lambda_h$? Characterize them modulo $q$.
\end{exercise}

\begin{exercise}[PhD: compare NTRU vs LWE intuition]
	Give a conceptual comparison: NTRU keys correspond to short vectors in a structured lattice tied to one public ring element; LWE hides a secret with additive noise across many samples. Discuss how this affects the style of security reductions and the known attacks.
\end{exercise}






%\subsection*{3.1\ Lattices}
%
%\paragraph{Definition (lattice).}
%A (full-rank) lattice $\mathcal{L}\subset \mathbb{R}^n$ is a discrete additive subgroup of $\mathbb{R}^n$.
%Equivalently, given linearly independent vectors $b_1,\dots,b_n\in \mathbb{R}^n$, the set
%\[
%\mathcal{L}(B)=\left\{ \sum_{i=1}^n z_i b_i \ :\ z_i\in \mathbb{Z} \right\}
%\]
%is a lattice; $B=[b_1|\cdots|b_n]$ is a basis of $\mathcal{L}$.
%The \emph{determinant} is $\det(\mathcal{L}) = |\det(B)|$ (basis-independent).
%
%\paragraph{Norm.}
%Typically $\|\cdot\|$ denotes the Euclidean norm $\|\cdot\|_2$, unless otherwise stated.
%
%\subsection*{3.2\ Shortest Vector Problem (SVP)}
%
%\paragraph{Search SVP.}
%Given a basis $B$ of a lattice $\mathcal{L}=\mathcal{L}(B)$, find a nonzero vector
%$v\in \mathcal{L}\setminus\{0\}$ minimizing $\|v\|$, i.e.
%\[
%\|v\| = \lambda_1(\mathcal{L}) \;:=\; \min\{\|x\|: x\in \mathcal{L}\setminus\{0\}\}.
%\]
%
%\paragraph{Approximate SVP ($\gamma$-SVP).}
%Given $B$ and an approximation factor $\gamma=\gamma(n)\ge 1$, output
%$v\in \mathcal{L}\setminus\{0\}$ such that
%\[
%\|v\|\le \gamma\cdot \lambda_1(\mathcal{L}).
%\]
%
%\subsection*{3.3\ Closest Vector Problem (CVP)}
%
%\paragraph{Search CVP.}
%Given a basis $B$ of $\mathcal{L}=\mathcal{L}(B)$ and a target $t\in \mathbb{R}^n$,
%find $v\in \mathcal{L}$ minimizing $\|t-v\|$, i.e.
%\[
%\|t-v\| = \mathrm{dist}(t,\mathcal{L}) \;:=\; \min\{\|t-x\|: x\in \mathcal{L}\}.
%\]
%
%\paragraph{Approximate CVP ($\gamma$-CVP).}
%Output $v\in\mathcal{L}$ with $\|t-v\|\le \gamma\cdot \mathrm{dist}(t,\mathcal{L})$.
%
%\subsection*{3.4\ Learning With Errors (LWE) (cryptographic canonical lattice problem)}
%
%\paragraph{Decision LWE.}
%Fix parameters $n\in\mathbb{N}$, modulus $q\ge 2$, and an error distribution $\chi$ over $\mathbb{Z}_q$.
%Given $m$ samples $(a_i,b_i)\in \mathbb{Z}_q^n\times \mathbb{Z}_q$, distinguish between:
%\begin{align*}
%	\text{(LWE)}\quad & a_i \xleftarrow{\$} \mathbb{Z}_q^n,\;\; s \xleftarrow{\$} \mathbb{Z}_q^n,\;\;
%	e_i \xleftarrow{\$}\chi,\;\; b_i = \langle a_i,s\rangle + e_i \pmod q;\\
%	\text{(Uniform)}\quad & (a_i,b_i) \xleftarrow{\$} \mathbb{Z}_q^n\times \mathbb{Z}_q,
%\end{align*}
%where $\langle\cdot,\cdot\rangle$ is the standard inner product modulo $q$.
%The \emph{search} version asks to recover $s$ from LWE samples.








%\subsection*{3.1 Lattices}
%
%\paragraph{Definition (lattice).}
%A full-rank lattice $\mathcal{L}\subset \mathbb{R}^n$ is a discrete subgroup:
%given linearly independent $b_1,\dots,b_n\in\mathbb{R}^n$,
%\[
%\mathcal{L}=\mathcal{L}(B)=\left\{\sum_{i=1}^n z_i b_i : z_i\in\mathbb{Z}\right\},\quad
%B=[b_1|\cdots|b_n].
%\]
%Define the first successive minimum
%\[
%\lambda_1(\mathcal{L})=\min\{\|v\|_2: v\in \mathcal{L}\setminus\{0\}\}.
%\]
%
%\subsection*{3.2 SVP and CVP}
%
%\paragraph{SVP (Shortest Vector Problem).}
%Given a basis $B$ of $\mathcal{L}$, find $v\in\mathcal{L}\setminus\{0\}$ such that $\|v\|_2=\lambda_1(\mathcal{L})$.
%
%\paragraph{$\gamma$-SVP (Approximate SVP).}
%Given $B$ and $\gamma\ge 1$, find $v\in\mathcal{L}\setminus\{0\}$ with
%\[
%\|v\|_2 \le \gamma\cdot \lambda_1(\mathcal{L}).
%\]
%
%\paragraph{CVP (Closest Vector Problem).}
%Given $B$ and a target $t\in\mathbb{R}^n$, find $v\in\mathcal{L}$ minimizing $\|t-v\|_2$.
%
%\subsection*{3.3 LWE (Learning With Errors)}
%
%\paragraph{Decision-LWE.}
%Fix $n,q\ge 2$ and an error distribution $\chi$ over $\mathbb{Z}_q$.
%Given $m$ samples $(a_i,b_i)\in \mathbb{Z}_q^n\times\mathbb{Z}_q$, distinguish:
%\begin{align*}
%	\text{(LWE)}\quad & a_i \xleftarrow{\$} \mathbb{Z}_q^n,\; s \xleftarrow{\$} \mathbb{Z}_q^n,\;
%	e_i\xleftarrow{\$}\chi,\; b_i=\langle a_i,s\rangle+e_i \pmod q;\\
%	\text{(Uniform)}\quad & (a_i,b_i)\xleftarrow{\$}\mathbb{Z}_q^n\times\mathbb{Z}_q.
%\end{align*}
%The \emph{search} version asks to recover $s$.
%
%\paragraph{Standard attacks (classical).}
%\begin{itemize}\setlength\itemsep{0.2em}
%	\item \textbf{LLL/BKZ reduction} as the main engine (produces short/near-short vectors).
%	\item \textbf{Primal attacks} (embedding to SVP/CVP) + enumeration/sieving.
%	\item \textbf{Dual attacks} (find short dual vectors to distinguish LWE from uniform).
%	\item \textbf{Hybrid attacks} (guess part of secret + lattice reduction).
%	\item \textbf{BKW / combinatorial} attacks (parameter-dependent).
%	\item \textbf{Algebraic attacks} in certain regimes (e.g.\ over-defined systems).
%\end{itemize}
%
%\paragraph{Quantum note.}
%No known Shor-like polynomial-time algorithm for general lattice problems; quantum speedups mainly
%affect search/sieving constants/exponents.

	
	\newpage
	\section{Codes (Syndrome Decoding / Min Distance)}
	\chapter{Code-Based Hard Problems}

\section{Linear Codes and Syndromes}
Let $\mathcal{C}\subseteq \Ftwo^n$ be a linear $[n,k]$ code with parity-check matrix $H\in\Ftwo^{(n-k)\times n}$.
For $e\in\Ftwo^n$, the \emph{syndrome} is $s = He^\top\in\Ftwo^{n-k}$.
Syndrome decoding asks: given $H$ and $s$, find a low-weight $e$ with syndrome $s$.

\section{Syndrome Decoding (SD)}

\subsection{Formal Statements}
\begin{definition}[Search-SD]
	Given $H\in\Ftwo^{(n-k)\times n}$, a syndrome $s\in\Ftwo^{n-k}$, and an integer weight $w$,
	find $e\in\Ftwo^n$ such that
	\[
	He^\top = s
	\quad\text{and}\quad
	\wt(e)=w.
	\]
\end{definition}

\begin{definition}[Decisional SD (DSD)]
	Distinguish:
	\begin{itemize}[leftmargin=*]
		\item $\calD_0$: $e\from \Ftwo^n$ uniform subject to $\wt(e)=w$, and $s=He^\top$.
		\item $\calD_1$: $s\from \Ftwo^{n-k}$ uniform.
	\end{itemize}
	Given $(H,s)$ output whether $s$ is a syndrome of a weight-$w$ error vector with non-negligible advantage.
\end{definition}

\subsection{Why SD is Hard}
For random $H$, the mapping $e\mapsto He^\top$ is linear and many-to-one. The hardness comes from the combinatorial explosion of possible $e$ of weight $w$:
\[
\#\{e\in\Ftwo^n:\wt(e)=w\} = \binom{n}{w}.
\]
Brute force is exponential in $n$ for typical $w$ scaling.

\subsection{Information Set Decoding (ISD) Family (High-Level)}
The dominant attacks are \emph{information set decoding} and its refinements (Prange, Stern, Dumer, BJMM, and modern variants). The meta-idea:
\begin{itemize}[leftmargin=*]
	\item guess an ``information set'' of coordinates where the error is assumed sparse/structured,
	\item reduce the decoding task to a smaller combinatorial search,
	\item repeat until success with certain probability.
\end{itemize}
Complexities are typically $2^{c n}$ with constant $c$ depending on rate $k/n$ and relative weight $w/n$.

\subsection{Exercises (SD)}
\begin{exercise}[Upper-undergrad: syndrome as coset]
	Fix $H$. Show that the set $\{e\in\Ftwo^n : He^\top = s\}$ is an affine subspace (a coset of $\ker(H)$). What is its size?
\end{exercise}

\begin{exercise}[Masters: counting solutions]
	Assume $H$ is full rank. For random $s$, what is the expected number of solutions $e$ of weight exactly $w$? Express it using $\binom{n}{w}$ and $2^{n-k}$ and justify the approximation.
\end{exercise}

\begin{exercise}[PhD: ISD success probability sketch]
	In Prange's algorithm, one chooses a set $I$ of $k$ positions and hopes the error is zero on $I$. Derive the success probability in terms of $n,k,w$ and the expected work factor.
\end{exercise}

\section{QC Syndrome Decoding (QCSD)}

\subsection{Quasi-Cyclic Structure}
A binary quasi-cyclic (QC) code often uses a parity-check matrix built from circulant blocks.
For block size $p$, a circulant matrix is determined by its first row; multiplication corresponds to polynomial multiplication modulo $x^p-1$.

\subsection{Problem Statement}
\begin{definition}[QCSD / DQCSD]
	Same as SD/DSD, except $H$ is drawn from a QC ensemble (block-circulant structure), and sometimes $e$ is restricted to QC form. Given $(H,s,w)$ find $e$ with $He^\top=s$ and $\wt(e)=w$, or distinguish structured syndromes from uniform.
\end{definition}

\subsection{Security Subtleties}
QC structure reduces public key sizes dramatically but introduces algebraic symmetry. Best practice is to choose parameters so that known structural attacks (e.g., exploiting cyclic shifts, folding, or module-based speedups) do not reduce security below target.

\subsection{Exercises (QCSD)}
\begin{exercise}[Masters: circulant-as-polynomial]
	Show how multiplying a circulant matrix by a vector corresponds to polynomial multiplication modulo $x^p-1$.
\end{exercise}

\begin{exercise}[PhD: symmetry and attack surface]
	Explain how cyclic symmetry can introduce additional low-weight codewords or enable collision-style shortcuts. Give at least one concrete avenue (high-level) by which QC structure can be exploited.
\end{exercise}


%\subsection*{4.1\ Linear codes}
%
%\paragraph{Definition (linear code).}
%A linear $[n,k]_q$ code is a $k$-dimensional $\mathbb{F}_q$-subspace
%$C\subseteq \mathbb{F}_q^n$.
%A \emph{generator matrix} $G\in \mathbb{F}_q^{k\times n}$ satisfies
%\[
%C = \{ uG : u\in \mathbb{F}_q^k\}.
%\]
%A \emph{parity-check matrix} $H\in \mathbb{F}_q^{(n-k)\times n}$ satisfies
%\[
%C = \{ c\in \mathbb{F}_q^n : Hc^\top = 0\}.
%\]
%
%\paragraph{Hamming weight and distance.}
%For $x\in\mathbb{F}_q^n$, the Hamming weight is
%$w_H(x)=|\{i: x_i\neq 0\}|$.
%The Hamming distance is $d_H(x,y)=w_H(x-y)$.
%The minimum distance of $C$ is
%\[
%d(C)=\min\{w_H(c): c\in C\setminus\{0\}\}.
%\]
%
%\subsection*{4.2\ Syndrome Decoding (SD)}
%
%\paragraph{Search SD.}
%Given $H\in \mathbb{F}_q^{(n-k)\times n}$, a syndrome $s\in \mathbb{F}_q^{n-k}$,
%and an integer $t$, find $e\in \mathbb{F}_q^n$ such that
%\[
%He^\top = s
%\quad\text{and}\quad
%w_H(e)\le t.
%\]
%(Over $\mathbb{F}_2$, this is the classical binary syndrome decoding problem.)
%
%\paragraph{Decision SD.}
%Given $(H,s,t)$, decide whether there exists $e$ with $He^\top=s$ and $w_H(e)\le t$.
%
%\subsection*{4.3\ Codeword Finding / Minimum Distance Problem}
%
%\paragraph{Minimum distance (decision form).}
%Given a linear code $C$ (via $G$ or $H$) and integer $t$, decide whether
%\[
%\exists\, c\in C\setminus\{0\}\ \text{such that}\ w_H(c)\le t.
%\]
%The corresponding search problem asks to find such a codeword.









%\paragraph{Linear code.}
%A linear $[n,k]_q$ code is a $k$-dimensional subspace $C\subseteq\mathbb{F}_q^n$.
%A parity-check matrix $H\in\mathbb{F}_q^{(n-k)\times n}$ satisfies
%\[
%C=\{c\in\mathbb{F}_q^n: Hc^\top=0\}.
%\]
%Hamming weight: $w_H(x)=|\{i:x_i\neq 0\}|$.
%
%\paragraph{Syndrome Decoding (SD) --- search form.}
%Given $H\in\mathbb{F}_q^{(n-k)\times n}$, syndrome $s\in\mathbb{F}_q^{(n-k)}$, and bound $t$,
%find $e\in\mathbb{F}_q^n$ such that
%\[
%He^\top=s \quad\text{and}\quad w_H(e)\le t.
%\]
%
%\paragraph{Minimum Distance Problem (MDP) --- decision form.}
%Given a linear code $C$ and integer $t$, decide whether
%\[
%\exists\,c\in C\setminus\{0\}\ \text{with}\ w_H(c)\le t.
%\]
%
%\paragraph{Standard attacks (classical).}
%\begin{itemize}\setlength\itemsep{0.2em}
%	\item \textbf{Information Set Decoding (ISD)} family: Prange, Stern, Dumer, BJMM-style improvements.
%	\item \textbf{Structural attacks}: if the public code is not pseudorandom (hidden algebraic structure, rank defects, etc.).
%	\item \textbf{Side-channel/implementation}: leakage from decoding routines or masking failures.
%\end{itemize}
%
%\paragraph{Quantum note.}
%Grover-type search can improve brute-force components; quantum ISD analyses give model-dependent exponent reductions.
%

	
	\newpage
	\section{Isogenies (Supersingular Isogeny Problems)}
	%\subsection*{5.1\ Elliptic curves and isogenies}
%
%\paragraph{Elliptic curve.}
%Let $\mathbb{F}_q$ be a finite field. An elliptic curve $E/\mathbb{F}_q$ is a smooth projective
%genus-one curve with a specified $\mathbb{F}_q$-rational point; equivalently (for $\mathrm{char}(\mathbb{F}_q)\neq 2,3$),
%a nonsingular Weierstrass equation
%\[
%E: y^2 = x^3 + ax + b,\qquad a,b\in \mathbb{F}_q,
%\]
%with discriminant $\Delta\neq 0$.
%
%\paragraph{Isogeny.}
%An isogeny $\varphi:E_1\to E_2$ of elliptic curves over $\mathbb{F}_q$ is a nonconstant morphism
%defined over $\mathbb{F}_q$ that is also a group homomorphism on $\mathbb{F}_q$-rational points.
%Its degree $\deg(\varphi)$ is its degree as a morphism; the kernel $\ker(\varphi)$ is a finite subgroup of $E_1$.
%
%\subsection*{5.2\ Isogeny path problem (informal canonical hardness)}
%
%\paragraph{Isogeny path problem (IPP).}
%Fix a class of curves (e.g.\ supersingular elliptic curves over $\mathbb{F}_{p^2}$) and a small prime $\ell$.
%Given elliptic curves $E_0,E_1$ in the same isogeny class, find an explicit isogeny
%\[
%\varphi:E_0\to E_1
%\]
%that decomposes as a composition of $\ell$-isogenies, i.e.\ $\deg(\varphi)=\ell^r$ for some $r$,
%and output a representation sufficient to evaluate $\varphi$ (e.g.\ the sequence of intermediate curves / kernels).
%
%\subsection*{5.3\ Supersingular Isogeny (computational) problem}
%
%\paragraph{Supersingular isogeny problem (one common formulation).}
%Let $p$ be prime and work over $\mathbb{F}_{p^2}$.
%Given two supersingular elliptic curves $E,E'$ over $\mathbb{F}_{p^2}$,
%find an isogeny $\varphi:E\to E'$ of prescribed smooth degree (often $\ell^r$ for small $\ell$),
%represented so that $\varphi$ can be evaluated.
%
%\paragraph{Decision variants.}
%Distinguish whether two given supersingular curves are connected by an isogeny of degree $\ell^r$,
%or distinguish distributions induced by random walks in the isogeny graph.
%


\paragraph{Elliptic curve over a finite field.}
Over $\mathbb{F}_q$ (characteristic $\neq 2,3$), an elliptic curve can be given by
\[
E: y^2 = x^3 + ax + b,\quad a,b\in\mathbb{F}_q,\quad \Delta\neq 0,
\]
and $E(\mathbb{F}_q)$ is a finite abelian group.

\paragraph{Isogeny.}
An isogeny $\varphi:E_1\to E_2$ over $\mathbb{F}_q$ is a nonconstant morphism defined over $\mathbb{F}_q$
that is also a group homomorphism. Its kernel is finite; $\deg(\varphi)$ is its morphism degree.

\paragraph{Supersingular Isogeny Problem (one common search form).}
Work over $\mathbb{F}_{p^2}$. Given supersingular elliptic curves $E,E'/\mathbb{F}_{p^2}$,
find an explicit isogeny
\[
\varphi:E\to E'
\]
of prescribed smooth degree (often $\ell^r$ for small prime $\ell$), represented so $\varphi$ is evaluable.

\paragraph{Standard attacks (classical).}
\begin{itemize}\setlength\itemsep{0.2em}
	\item \textbf{Isogeny-graph path search}: treat curves as vertices, $\ell$-isogenies as edges.
	\item \textbf{Meet-in-the-middle} / bidirectional search to find paths faster than naive random walk.
	\item \textbf{Protocol-specific cryptanalysis}: some isogeny protocols have been broken (do not assume all are secure).
\end{itemize}

\paragraph{Quantum note.}
Known quantum algorithms can improve generic path-finding exponents in isogeny graphs; no known general polynomial-time algorithm.

	
	\newpage
	\section{Multivariable (Multivariate Quadratic, MQ)}
	\paragraph{Setting (systems of polynomial equations).}
Let $\mathbb{F}_q$ be a finite field. Let $f_1,\dots,f_m\in \mathbb{F}_q[x_1,\dots,x_n]$ be polynomials.

\paragraph{MQ (Multivariate Quadratic) --- search problem.}
Given $m$ quadratic polynomials
\[
f_i(x_1,\dots,x_n) \in \mathbb{F}_q[x_1,\dots,x_n],\quad \deg(f_i)\le 2,
\]
find a solution $x=(x_1,\dots,x_n)\in\mathbb{F}_q^n$ such that
\[
f_i(x)=0 \quad\text{for all } i=1,\dots,m.
\]
(Decision variant: decide whether such an $x$ exists.)

\paragraph{Standard attacks (classical).}
\begin{itemize}\setlength\itemsep{0.2em}
	\item \textbf{Gr\"obner basis} methods: F4/F5; complexity driven by degree of regularity and sparsity.
	\item \textbf{XL / variants} (eXtended Linearization) and relinearization methods.
	\item \textbf{Hybrid attacks}: guess some variables to reduce to smaller systems, then algebraic solve.
	\item \textbf{Rank attacks / MinRank} reductions for structured schemes (oil-vinegar-type, etc.).
	\item \textbf{Linearization traps}: if parameters make the system overdetermined/easy.
\end{itemize}

\paragraph{Quantum note.}
Grover can speed up variable-guessing layers; algebraic-solving quantum speedups are limited and highly model/instance-dependent.

	
	\newpage
	\section{Hash functions (Formal notions and Generic bounds)}
	%\subsection*{6.1\ Hash function as a deterministic map}
%
%\paragraph{Definition (hash function).}
%A hash function is a deterministic function
%\[
%H:\{0,1\}^\ast \to \{0,1\}^n
%\]
%mapping arbitrary-length bitstrings to fixed-length digests of $n$ bits.
%
%\subsection*{6.2\ Collision resistance}
%
%\paragraph{Collision resistance (CR).}
%A family of functions $\{H_\lambda\}$ (indexed by security parameter $\lambda$)
%is collision resistant if for every PPT adversary $\mathcal{A}$,
%\[
%\Pr\Big[(x,x')\leftarrow \mathcal{A}(1^\lambda): x\neq x' \ \wedge\ H_\lambda(x)=H_\lambda(x')\Big]
%\]
%is negligible in $\lambda$.
%
%\subsection*{6.3\ Second-preimage resistance}
%
%\paragraph{Second-preimage resistance (SPR).}
%$\{H_\lambda\}$ is second-preimage resistant if for every PPT adversary $\mathcal{A}$,
%\[
%\Pr\Big[x\leftarrow \{0,1\}^\ast;\ x'\leftarrow \mathcal{A}(1^\lambda,x):
%x'\neq x \ \wedge\ H_\lambda(x')=H_\lambda(x)\Big]
%\]
%is negligible (with respect to a specified distribution over $x$, often uniform over fixed length).
%
%\subsection*{6.4\ Preimage resistance / one-wayness}
%
%\paragraph{Preimage resistance (OW).}
%$\{H_\lambda\}$ is preimage resistant if for every PPT adversary $\mathcal{A}$,
%\[
%\Pr\Big[y\leftarrow \{0,1\}^n;\ x\leftarrow \mathcal{A}(1^\lambda,y):
%H_\lambda(x)=y\Big]
%\]
%is negligible in $\lambda$ (with $y$ uniform in the range; equivalently, $y=H_\lambda(x)$ for random $x$ in some domain).
%
%\subsection*{6.5\ Random oracle model (optional idealization)}
%
%\paragraph{Random oracle.}
%An ideal hash $H$ is modeled as a uniformly random function
%$H:\{0,1\}^\ast\to \{0,1\}^n$ to which all parties have oracle access; i.e.\ consistent
%answers on repeated queries, independent uniform outputs on new inputs.

\paragraph{Hash family.}
A hash family $\{H_\lambda\}$ is a set of efficiently computable functions
\[
H_\lambda:\{0,1\}^\ast\to \{0,1\}^{n(\lambda)}.
\]

\paragraph{Collision resistance (CR).}
$\{H_\lambda\}$ is collision resistant if for every PPT adversary $\mathcal{A}$,
\[
\Pr\Big[(x,x')\leftarrow \mathcal{A}(1^\lambda): x\neq x' \ \wedge\ H_\lambda(x)=H_\lambda(x')\Big]
\]
is negligible in $\lambda$.

\paragraph{Second-preimage resistance (SPR).}
For every PPT $\mathcal{A}$,
\[
\Pr\Big[x\leftarrow \mathcal{D};\ x'\leftarrow \mathcal{A}(1^\lambda,x):
x'\neq x \ \wedge\ H_\lambda(x')=H_\lambda(x)\Big]
\]
is negligible (for a specified distribution $\mathcal{D}$ over inputs).

\paragraph{Preimage resistance (one-wayness, OW).}
For every PPT $\mathcal{A}$,
\[
\Pr\Big[y\leftarrow \{0,1\}^{n(\lambda)};\ x\leftarrow \mathcal{A}(1^\lambda,y):
H_\lambda(x)=y\Big]
\]
is negligible.

\paragraph{Standard attacks.}
\begin{itemize}\setlength\itemsep{0.2em}
	\item \textbf{Generic bounds}: collisions in about $2^{n/2}$ evaluations (birthday paradox),
	preimages in about $2^{n}$ evaluations.
	\item \textbf{Design-specific cryptanalysis}: differential/boomerang-style attacks; rotational/symmetry attacks; etc.
	\item \textbf{Chosen-prefix collisions} (for weakened designs).
	\item \textbf{Length extension}: for Merkle--Damg{\aa}rd hashes if misused as $H(k\|m)$; use HMAC to avoid.
\end{itemize}

\paragraph{Quantum attacks.}
\textbf{Grover} gives preimages in about $2^{n/2}$ quantum queries.
(Quantum collision-finding can also improve over the classical birthday bound in some models.)

	
	
	\newpage
%	\section{Classical Hard Problems}
%	\input{sections/02-classical.tex}
%	
%	\section{Quantum Threat Model}
%	\input{sections/03-quantum.tex}
%	
%	\section{Post-Quantum Hard Problems}
%	\input{sections/04-pqc.tex}
%	
%	\section{From Hard Problems to Standards}
%	\input{sections/05-standards.tex}
	
	\printbibliography
\end{document}
