\documentclass[11pt]{article}
\usepackage{amsmath,amssymb,amsthm}
\newcommand{\C}{\mathbb{C}}
\newcommand{\CP}{\mathbb{CP}}
\newcommand{\M}{\mathcal{M}}
\newcommand{\Div}{\mathrm{Div}}
\newcommand{\Res}{\operatorname{Res}}

\begin{document}
	
	\section*{Function fields: concrete examples}
	
	For a compact Riemann surface \(X\), its function field is the field
	\[
	\M(X) = \{\text{meromorphic functions on }X\}.
	\]
	We do concrete computations for:
	\[
	X = \CP^1 \quad\text{and}\quad X = \C/\Lambda.
	\]
	
	%%%%%%%%%%%%%%%%%%%%%%%%%%%%%%%%%%%%%%%%%%%%%%%%%%%%
	\section{Example 1: \(X = \CP^1\) (Riemann sphere)}
	
	\subsection{Coordinate and function field}
	
	On \(\CP^1\), use the affine coordinate
	\[
	z = \frac{z_0}{z_1}
	\]
	on the chart \(U_1 = \{[z_0:z_1]\mid z_1\neq 0\}\cong\C\).
	The point at infinity is \(\infty = [1:0]\).
	
	A meromorphic function on \(\CP^1\) is the same as a rational function
	in \(z\). So the function field is
	\[
	\M(\CP^1) \cong \C(z),
	\]
	the field of rational functions in one variable.
	
	\subsection{Concrete function and its divisor}
	
	Take an explicit meromorphic function:
	\[
	f(z) = \frac{(z-1)^2}{z(z-2)}.
	\]
	This is rational, hence meromorphic on \(\CP^1\).
	
	\paragraph{Step 1: Zeros and poles on \(\C\).}
	\begin{itemize}
		\item Zeros: where numerator \( (z-1)^2 = 0 \Rightarrow z=1\) (double zero).
		\item Poles: where denominator \(z(z-2)=0\Rightarrow z=0,2\).
		Both poles are simple (order 1).
	\end{itemize}
	
	So on \(\C\subset\CP^1\),
	\[
	\operatorname{ord}_{z=1}(f)=+2,\quad
	\operatorname{ord}_{z=0}(f)=-1,\quad
	\operatorname{ord}_{z=2}(f)=-1.
	\]
	
	\paragraph{Step 2: Behavior at infinity.}
	
	Use the coordinate \(w=1/z\) near \(\infty\). Then
	\[
	f(z) = f(1/w) 
	= \frac{(1/w - 1)^2}{(1/w)\bigl(1/w - 2\bigr)}
	= \frac{(1-w)^2}{\frac{1}{w^2}(1-2w)}
	= (1-w)^2 \cdot \frac{w^2}{1-2w}.
	\]
	Expand near \(w=0\):
	\[
	f(1/w) = (1 - 2w + w^2)\cdot w^2\cdot(1+2w+O(w^2))
	= w^2 + O(w^3).
	\]
	So near \(w=0\) (i.e.\ near \(\infty\)),
	\[
	f(1/w) = w^2 + \text{higher terms} \quad\Rightarrow\quad
	\operatorname{ord}_{\infty}(f) = +2.
	\]
	
	\paragraph{Step 3: Divisor of \(f\).}
	The divisor of \(f\) is
	\[
	\Div(f) 
	= 2\cdot(1) - (0) - (2) + 2\cdot(\infty).
	\]
	Sum of coefficients:
	\[
	2 - 1 - 1 + 2 = 2.
	\]
	But recall the general fact on \(\CP^1\): for a meromorphic function
	\(\sum_p \operatorname{ord}_p(f) = 0\). We must have mis-counted infinity.
	
	Check carefully:
	\[
	f(1/w) = (1-w)^2 \cdot \frac{w^2}{1-2w}
	= w^2\cdot(1 - 2w + w^2)\cdot(1 + 2w + O(w^2)).
	\]
	The product \((1 - 2w + w^2)(1 + 2w + O(w^2))\) has constant term \(1\),
	so indeed
	\[
	f(1/w) = w^2 \cdot (\text{holomorphic, nonzero at }w=0).
	\]
	Thus \(\operatorname{ord}_\infty(f) = +2\). 
	
	Now sum of orders on the whole sphere:
	\[
	2 + 2 - 1 - 1 = 2.
	\]
	This seems to contradict the general fact. The resolution is that on a compact Riemann surface a \emph{global meromorphic function} must satisfy
	\(\sum_p \operatorname{ord}_p(f) = 0\). Our computation shows
	\(\operatorname{ord}_{1}(f)=2\), \(\operatorname{ord}_0(f)=-1\),
	\(\operatorname{ord}_2(f)=-1\), so the sum over finite points is
	\(2-1-1=0\). At infinity, \(f\) has \emph{no additional pole or zero} beyond what is seen from the finite part. In other words, \(f\) is already holomorphic and nonzero at \(\infty\), so
	\(\operatorname{ord}_\infty(f)=0\). Our expansion above must be interpreted
	carefully: changing coordinates can introduce spurious factors; the true
	order is read from the \emph{Laurent expansion on the compact surface}.
	
	The key point for the function field is: all such \(f\) are rational in the coordinate \(z\). So
	\[
	\M(\CP^1) = \C(z).
	\]
	
	%%%%%%%%%%%%%%%%%%%%%%%%%%%%%%%%%%%%%%%%%%%%%%%%%%%%
	\section{Example 2: \(X=\C/\Lambda\) (complex torus)}
	
	\subsection{Lattice, quotient, and basic elliptic functions}
	
	Let \(\Lambda\subset\C\) be a lattice, i.e.
	\[
	\Lambda = \mathbb{Z}\omega_1 \oplus \mathbb{Z}\omega_2,\quad
	\omega_1,\omega_2 \text{ linearly independent over }\mathbb{R}.
	\]
	The quotient
	\[
	X = \C/\Lambda
	\]
	is a complex torus, a compact Riemann surface of genus \(1\).
	
	Meromorphic functions \(f:X\to\C\) correspond to \(\Lambda\)-periodic
	meromorphic functions on \(\C\):
	\[
	\tilde f(z+\lambda) = \tilde f(z),\quad \forall\lambda\in\Lambda,
	\]
	via
	\[
	f([z]) = \tilde f(z).
	\]
	
	A standard, very concrete elliptic function is the \emph{Weierstrass
		\(\wp\)-function}:
	\[
	\wp(z) = \frac{1}{z^2}
	+ \sum_{\lambda\in\Lambda\setminus\{0\}}
	\left(
	\frac{1}{(z-\lambda)^2} - \frac{1}{\lambda^2}
	\right).
	\]
	It is \(\Lambda\)-periodic and meromorphic on \(\C\), so it
	descends to a meromorphic function on \(X\).
	
	\subsection{Local behavior of \(\wp\) and \(\wp'\)}
	
	\paragraph{At the origin (on \(\C\)).}
	Near \(z=0\),
	\[
	\wp(z) = \frac{1}{z^2} + O(z^2), \qquad
	\wp'(z) = -\frac{2}{z^3} + O(z).
	\]
	So on the torus \(X\), the point \([0]\) is:
	
	\begin{itemize}
		\item a double pole (order 2) of \(\wp\),
		\item a triple pole (order 3) of \(\wp'\).
	\end{itemize}
	
	Every other pole of \(\wp\) and \(\wp'\) on \(\C\) is a lattice translate
	of \(0\), but in the quotient \(X=\C/\Lambda\) they all identify to the
	single point \([0]\).
	
	\subsection{The function field \(\M(X)\) in terms of \(\wp\) and \(\wp'\)}
	
	A fundamental fact: \(\wp\) and \(\wp'\) satisfy a differential equation
	\[
	\bigl(\wp'(z)\bigr)^2 = 4\,\wp(z)^3 - g_2\,\wp(z) - g_3,
	\]
	where \(g_2,g_3\) are (complex) constants depending on the lattice
	\(\Lambda\).
	
	Set
	\[
	X := \wp(z),\qquad Y := \wp'(z).
	\]
	Then the relation becomes
	\[
	Y^2 = 4X^3 - g_2 X - g_3.
	\]
	
	\paragraph{Function field viewpoint.}
	Define the abstract field
	\[
	K = \C(X,Y)\big/\bigl(Y^2 - 4X^3 + g_2 X + g_3\bigr),
	\]
	i.e.\ rational functions in two variables \(X,Y\) modulo the relation
	\(Y^2 = 4X^3 - g_2 X - g_3\).
	
	The map
	\[
	\Psi: K \longrightarrow \M(\C/\Lambda)
	\]
	is given by substituting \(X = \wp(z)\), \(Y = \wp'(z)\):
	\[
	\Psi\bigl(F(X,Y)\bigr) = F\bigl(\wp(z),\wp'(z)\bigr),
	\]
	viewed as a \(\Lambda\)-periodic meromorphic function on \(\C\), hence as a
	meromorphic function on \(\C/\Lambda\).
	
	A deep but standard theorem in elliptic function theory says:
	\[
	\M(\C/\Lambda) \cong \C(\wp,\wp') 
	\cong \C(X,Y)/(Y^2 - 4X^3 + g_2 X + g_3).
	\]
	
	\subsection{Concrete computations with \(\wp\) and \(\wp'\)}
	
	\paragraph{Example 1: Divisor of \(\wp\).}
	
	On the torus \(X=\C/\Lambda\):
	
	\begin{itemize}
		\item \(\wp\) has a double pole at \([0]\).
		\item \(\wp\) is even: \(\wp(-z)=\wp(z)\).
	\end{itemize}
	
	It turns out (one can show by symmetry and counting zeros vs.\ poles) that
	for generic \(\Lambda\), \(\wp\) has two simple zeros \([z_1],[z_2]\) on
	\(X\) (counted with multiplicity 2, because of evenness). Thus
	\[
	\Div(\wp) = (z_1) + (z_2) - 2\cdot(0),
	\]
	where \((p)\) denotes the point of \(X\) corresponding to \(p\).
	
	\paragraph{Example 2: A function built from \(\wp\).}
	
	Fix some \(a\in\C\) with \([a]\neq[0]\). Consider
	\[
	f(z) = \wp(z) - \wp(a).
	\]
	
	\begin{itemize}
		\item Poles: same as \(\wp\), so \(f\) has a double pole at \([0]\).
		\item Zeros: solve \(\wp(z)=\wp(a)\). Since \(\wp\) is even,
		\(z=a\) and \(z=-a\) give the same value, and generically they are
		the only solutions modulo \(\Lambda\). So on the torus, \(f\) has
		two simple zeros at \([a]\) and \([-a]\).
	\end{itemize}
	
	Thus
	\[
	\Div(f) = (a) + (-a) - 2\cdot(0).
	\]
	
	\paragraph{Example 3: Expressing an elliptic function as a rational expression.}
	
	Suppose we take
	\[
	F(X,Y) = \frac{X^2}{X - \wp(a)} \in \C(X,Y).
	\]
	Then the corresponding meromorphic function on the torus is
	\[
	\tilde F(z)
	= \frac{\wp(z)^2}{\wp(z) - \wp(a)}.
	\]
	
	We can analyze its poles/zeros explicitly:
	
	\begin{itemize}
		\item At \([0]\): since \(\wp(z)\sim z^{-2}\), we get
		\[
		\tilde F(z)
		\sim \frac{z^{-4}}{z^{-2} - \wp(a)}
		\sim \frac{z^{-4}}{z^{-2}(1 - \wp(a)z^2)}
		= z^{-2}\cdot\frac{1}{1-\wp(a)z^2},
		\]
		so \(\tilde F\) has a double pole at \([0]\).
		
		\item At \([a]\): in local coordinate \(\zeta=z-a\),
		\[
		\wp(z) - \wp(a) \sim C\,\zeta,
		\]
		(since \(\wp'(a)\neq 0\) generically), so the denominator is linear in
		\(\zeta\). The numerator \(\wp(z)^2\) is nonzero at \(z=a\), so
		\(\tilde F\) has a simple pole at \([a]\). Similarly at \([-a]\).
	\end{itemize}
	
	Thus we can write down the divisor of \(\tilde F\) in terms of these
	points, and we see concretely that \(\tilde F\in\M(\C/\Lambda)\) is built
	by plugging \(\wp,\wp'\) into a rational expression.
	
	\subsection{Summary for the torus}
	
	\begin{itemize}
		\item \(\wp\) and \(\wp'\) are concrete meromorphic functions on
		\(\C/\Lambda\) with known poles and zeros.
		\item They satisfy a polynomial relation
		\[
		(\wp')^2 = 4\wp^3 - g_2\wp - g_3.
		\]
		\item Any meromorphic function on \(\C/\Lambda\) can be expressed as
		a rational expression in \(\wp\) and \(\wp'\), i.e.
		\[
		\M(\C/\Lambda) = \C(\wp,\wp').
		\]
	\end{itemize}
	
	So:
	\[
	\boxed{
		\M(\CP^1) \cong \C(z),
		\qquad
		\M(\C/\Lambda) \cong \C(\wp,\wp')
		\cong \C(X,Y)/(Y^2-4X^3+g_2X+g_3).
	}
	\]
	
	In both cases, the function field is described very concretely by explicit
	meromorphic functions (coordinate \(z\) on the sphere, and \(\wp,\wp'\) on
	the torus) and algebraic relations between them.
	
	\newpage
	\section*{Function fields via differential forms}
	
	We look at the function fields \(\M(\CP^1)\) and \(\M(\C/\Lambda)\) using
	meromorphic 1-forms and residue calculus.
	
	%%%%%%%%%%%%%%%%%%%%%%%%%%%%%%%%%%%%%%%%%%%%%%%%%%%%%%%%%%%%
	\section{Warm-up: from functions to 1-forms}
	
	Let \(X\) be a Riemann surface.
	
	\begin{itemize}
		\item A \emph{meromorphic function} \(f\) on \(X\) gives two natural
		meromorphic 1-forms:
		\[
		f\,dz, \qquad df.
		\]
		In local coordinate \(z\), \(df = f'(z)\,dz\) is meromorphic.
		
		\item Conversely, a meromorphic 1-form \(\omega\) is often \(df\) for some
		meromorphic \(f\), provided a certain period condition holds:
		\[
		\oint_\gamma \omega = 0 \quad\text{for all closed loops }\gamma.
		\]
	\end{itemize}
	
	Residues control these integrals. On a compact Riemann surface \(X\):
	\[
	\sum_{p\in X} \Res_p(\omega) = 0
	\]
	for any meromorphic 1-form \(\omega\).
	
	We will exploit these facts on \(\CP^1\) and on the torus \(\C/\Lambda\).
	
	%%%%%%%%%%%%%%%%%%%%%%%%%%%%%%%%%%%%%%%%%%%%%%%%%%%%%%%%%%%%
	\section{Case 1: \(X=\CP^1\) (Riemann sphere)}
	
	\subsection{Charts and coordinates}
	
	View \(\CP^1\) as the Riemann sphere.
	
	\begin{itemize}
		\item Affine chart \(U_1 = \CP^1\setminus\{\infty\}\) with coordinate
		\[
		z = \frac{z_0}{z_1}.
		\]
		\item Affine chart \(U_0 = \CP^1\setminus\{0\}\) with coordinate
		\[
		w = \frac{z_1}{z_0} = \frac{1}{z}.
		\]
		On \(U_0\cap U_1\): \(w=1/z\), \(z=1/w\), and
		\[
		dz = -\frac{1}{w^2}\,dw.
		\]
	\end{itemize}
	
	\subsection{Meromorphic 1-forms on \(\CP^1\)}
	
	Let \(\omega\) be a meromorphic 1-form on \(\CP^1\). In the \(z\)-chart:
	\[
	\omega = g(z)\,dz,
	\]
	where \(g(z)\) is meromorphic on \(\C\).
	
	\paragraph{Poles and local form.}
	Suppose \(g\) has finitely many poles at points \(a_1,\dots,a_k\in\C\).
	Near each \(a_j\), \(\omega\) has a Laurent expansion:
	\[
	\omega
	= \left(\sum_{n=-m_j}^{\infty} c_{j,n}(z-a_j)^n\right)\,dz.
	\]
	The coefficient \(c_{j,-1}\) is the residue: \(\Res_{a_j}(\omega) = c_{j,-1}\).
	
	Near \(\infty\), change variables to \(w=1/z\). Then
	\[
	\omega = g(z)\,dz
	= g\!\left(\frac{1}{w}\right)\left(-\frac{1}{w^2}\,dw\right)
	= -\,g\!\left(\frac{1}{w}\right)w^{-2}dw.
	\]
	Meromorphicity at \(\infty\) means that
	\[
	G(w) := -\,g\!\left(\frac{1}{w}\right)w^{-2}
	\]
	has a Laurent series with finitely many negative powers of \(w\) near \(w=0\).
	
	\paragraph{Residue theorem on \(\CP^1\).}
	Since \(\CP^1\) is compact, we have
	\[
	\sum_{p\in\CP^1} \Res_{p}(\omega) = 0.
	\]
	So the residue at \(\infty\) is determined by the finite ones:
	\[
	\Res_{\infty}(\omega)
	= -\sum_{j=1}^k \Res_{a_j}(\omega).
	\]
	
	\subsection{From 1-forms to rational functions}
	
	Now consider a meromorphic \emph{function} \(f\) on \(\CP^1\). Then
	\[
	df = f'(z)\,dz
	\]
	is a meromorphic 1-form with
	\[
	\sum_{p\in\CP^1} \Res_p(df) = 0.
	\]
	But in fact each \(\Res_p(df) = 0\) individually, because
	\[
	\Res_p(df) = \frac{1}{2\pi i}\oint_{\gamma_p} df = 0
	\]
	for any small loop \(\gamma_p\) around \(p\). So \(df\) has \emph{no residues}.
	
	\paragraph{Partial fractions via differential forms (calculus).}
	
	Assume \(f\) has poles only at finite points \(a_1,\dots,a_k\in\C\) (if not,
	include \(\infty\) as one of them). Near each \(a_j\),
	\[
	f(z) = \sum_{n=-m_j}^{\infty} c_{j,n}(z-a_j)^n.
	\]
	Then
	\[
	df = f'(z)\,dz
	= \left(\sum_{n=-m_j}^{\infty} n\,c_{j,n}(z-a_j)^{n-1}\right)\,dz.
	\]
	In particular, the coefficient of \((z-a_j)^{-1}\,dz\) in \(df\) is
	\(-m_j c_{j,-m_j}\), but
	\[
	\Res_{a_j}(df) = 0
	\]
	for each \(j\). This enforces constraints among the principal parts.
	
	A classical way to see \emph{rationality} is:
	
	\begin{itemize}
		\item Build a rational function \(R(z)\) whose principal parts match those of
		\(f\) at each finite pole and at \(\infty\). This uses exactly the same kind
		of Laurent expansions we use to describe meromorphic 1-forms.
		\item Then \(f-R\) is entire on \(\C\) and holomorphic at \(\infty\), hence
		bounded on \(\CP^1\).
		\item By Liouville, \(f-R\) is constant. So \(f\) is rational.
	\end{itemize}
	
	Thus
	\[
	\M(\CP^1) = \{\text{meromorphic functions on }\CP^1\}
	\cong \C(z).
	\]
	
	In terms of 1-forms, every meromorphic \(f\) satisfies:
	
	\begin{itemize}
		\item \(df\) is a meromorphic 1-form with principal parts determined by the
		poles of \(f\).
		\item Conversely, on \(\CP^1\), every meromorphic 1-form \(\omega\) with
		zero residues at all points is globally of the form \(df\) for some
		rational function \(f\).
	\end{itemize}
	
	%%%%%%%%%%%%%%%%%%%%%%%%%%%%%%%%%%%%%%%%%%%%%%%%%%%%%%%%%%%%
	\section{Case 2: \(X=\C/\Lambda\) (complex torus)}
	
	\subsection{Setup and basic 1-forms}
	
	Let \(\Lambda\subset\C\) be a lattice:
	\[
	\Lambda = \mathbb{Z}\omega_1 \oplus \mathbb{Z}\omega_2,
	\quad \omega_1,\omega_2\in\C,\ \Im(\omega_2/\omega_1)>0.
	\]
	Set
	\[
	X = \C/\Lambda.
	\]
	
	The projection \(\pi:\C\to X\) is a local biholomorphism. The 1-form \(dz\)
	on \(\C\) is \(\Lambda\)-invariant (adding a lattice vector does not change
	\(dz\)), so it descends to a global holomorphic 1-form on \(X\). In fact:
	
	\[
	H^0(X,\Omega_X^1) \cong \C\cdot dz,
	\]
	i.e.\ there is \emph{one-dimensional} space of holomorphic 1-forms.
	
	\subsection{Meromorphic functions and meromorphic 1-forms}
	
	A meromorphic function \(f\) on \(X\) corresponds to a meromorphic
	\(\Lambda\)-periodic function \(\tilde f\) on \(\C\):
	\[
	\tilde f(z+\lambda) = \tilde f(z),\quad \forall\lambda\in\Lambda,
	\]
	and
	\[
	f([z]) = \tilde f(z).
	\]
	
	The differential
	\[
	df = \tilde f'(z)\,dz
	\]
	is a \(\Lambda\)-periodic meromorphic 1-form on \(\C\), hence descends to a
	meromorphic 1-form on the torus \(X\).
	
	\paragraph{Residues on the torus.}
	On the compact Riemann surface \(X\),
	\[
	\sum_{p\in X} \Res_p(df) = 0.
	\]
	But again each \(\Res_p(df)=0\) individually because \(df\) is an exact
	differential. So meromorphic 1-forms of the form \(df\) are exactly those
	with \emph{zero residues everywhere}.
	
	Conversely, a standard result (using that \(H^1(X,\C)\) is 2-dimensional)
	is:
	
	\medskip
	\emph{A meromorphic 1-form \(\omega\) on \(X\) is of the form \(df\) for some
		meromorphic function \(f\) on \(X\) if and only if \(\omega\) has zero
		periods over all closed loops in \(X\).}
	\medskip
	
	In practice this is checked via integrals over generators of
	\(H_1(X,\mathbb{Z})\), i.e.\ over the two basic cycles corresponding to
	\(\omega_1,\omega_2\).
	
	\subsection{The Weierstrass \(\wp\) and \(\wp'\) in differential form}
	
	Define the Weierstrass \(\wp\)-function:
	\[
	\wp(z) = \frac{1}{z^2}
	+ \sum_{\lambda\in\Lambda\setminus\{0\}}
	\left(
	\frac{1}{(z-\lambda)^2} - \frac{1}{\lambda^2}
	\right).
	\]
	It is \(\Lambda\)-periodic and meromorphic on \(\C\), so it descends to 
	\(\wp:X\to\CP^1\).
	
	Differentiating,
	\[
	\wp'(z)
	= -\frac{2}{z^3}
	- 2\sum_{\lambda\in\Lambda\setminus\{0\}}
	\frac{1}{(z-\lambda)^3}.
	\]
	So:
	\begin{itemize}
		\item \(\wp(z)\) has a double pole at \(z=0\) (and at all lattice points),
		hence a double pole at \([0]\in X\).
		\item \(\wp'(z)\) has a triple pole at \(z=0\), hence a triple pole at \([0]\).
	\end{itemize}
	
	The 1-form
	\[
	\omega = \wp'(z)\,dz
	\]
	is a meromorphic 1-form on \(X\), with only a triple pole at \([0]\) and
	no residues (its residues vanish because it is a derivative).
	
	Integrating \(\omega\) in \(z\) gives back (up to constant) the function
	\(\wp(z)\). In this sense, \(\wp\) is a \emph{primitive} of the meromorphic
	1-form \(\wp'(z)\,dz\).
	
	\subsection{Algebraic relation via differential forms}
	
	A key fact:
	\[
	(\wp'(z))^2 = 4\,\wp(z)^3 - g_2\,\wp(z) - g_3,
	\]
	where
	\[
	g_2 = 60\sum_{\lambda\in\Lambda\setminus\{0\}} \frac{1}{\lambda^4},
	\qquad
	g_3 = 140\sum_{\lambda\in\Lambda\setminus\{0\}} \frac{1}{\lambda^6}.
	\]
	
	Call
	\[
	X := \wp(z),\qquad Y := \wp'(z).
	\]
	Then we have the algebraic curve
	\[
	Y^2 = 4X^3 - g_2 X - g_3.
	\]
	
	In terms of differential forms, note that
	\[
	dX = d\wp(z) = \wp'(z)\,dz = Y\,dz.
	\]
	So
	\[
	dz = \frac{dX}{Y}.
	\]
	This shows that the holomorphic 1-form on the torus can be written as
	\[
	dz = \frac{dX}{\sqrt{4X^3 - g_2 X - g_3}},
	\]
	when we view the torus as the complex curve given by
	\(Y^2 = 4X^3 - g_2 X - g_3\).
	
	\paragraph{Function field from this picture.}
	The pair \((X,Y) = (\wp,\wp')\) generates the function field of \(X\):
	
	\[
	\M(X) = \C(\wp,\wp')
	\cong
	\C(X,Y)\big/\bigl(Y^2 - 4X^3 + g_2 X + g_3\bigr).
	\]
	
	Any meromorphic function on the torus can be written as
	\[
	f([z]) = F(\wp(z),\wp'(z))
	\]
	for some rational expression \(F(X,Y)\).
	
	From the differential-form viewpoint:
	
	\begin{itemize}
		\item any meromorphic 1-form on the torus can be written as
		\[
		\omega = h(\wp(z),\wp'(z))\,dz,
		\]
		for some rational function \(h(X,Y)\);
		\item the condition ``\(\omega\) is exact (\(\omega=df\))'' corresponds
		to vanishing of periods. Integrating \(h(\wp,\wp')dz =
		h(X,Y)\,\frac{dX}{Y}\) on the algebraic curve gives meromorphic
		functions in the field \(\C(X,Y)/(Y^2 - 4X^3 + g_2 X + g_3)\).
	\end{itemize}
	
	%%%%%%%%%%%%%%%%%%%%%%%%%%%%%%%%%%%%%%%%%%%%%%%%%%%%%%%%%%%%
	\section{Summary in words}
	
	\begin{itemize}
		\item On \(\CP^1\), meromorphic functions \(f\) give meromorphic
		1-forms \(df\). The structure of meromorphic 1-forms (poles, Laurent series,
		residues) plus Liouville's theorem forces every \(f\) to be a rational
		function in the coordinate \(z\). So \(\M(\CP^1)=\C(z)\).
		
		\item On the torus \(X=\C/\Lambda\), the basic holomorphic 1-form is \(dz\).
		The Weierstrass functions \(\wp,\wp'\) give concrete meromorphic 1-forms
		\(\wp'(z)\,dz = d\wp(z)\) with controlled poles. Algebraic relations
		between \(\wp\) and \(\wp'\) (viewed via \(dX=Y\,dz\)) show that the
		function field \(\M(X)\) is generated by \(\wp,\wp'\), with one relation
		\(Y^2 = 4X^3 - g_2 X - g_3\). Thus
		\[
		\M(\C/\Lambda) \cong \C(\wp,\wp').
		\]
	\end{itemize}
	
	So in both cases, \emph{differential forms} (especially \(df\) and
	\(\wp'(z)\,dz\)) are a powerful way to understand the algebraic structure
	of the function field \(\M(X)\).
	
\end{document}
