% !TEX TS-program = pdflatex
\documentclass[11pt, a4paper]{article}
\usepackage{amsmath,amssymb,amsthm}
\usepackage{mathtools}

\newtheorem{theorem}{Theorem}
\newtheorem{lemma}{Lemma}

\newcommand{\CP}{\mathbb{CP}}
\newcommand{\C}{\mathbb{C}}
\newcommand{\M}{\mathcal{M}}

\begin{document}
	
\subsection*{Step 0: What are the two objects?}

\begin{itemize}
	\item \(\C(x)\): the field of rational functions in one variable \(x\):
	\[
	\C(x)
	=
	\left\{
	\frac{p(x)}{q(x)}
	\;\middle|\;
	p,q\in\C[x],\ q\not\equiv 0
	\right\}\Big/\sim,
	\]
	where \(\frac{p}{q}\sim\frac{p'}{q'}\) if \(p(x)q'(x)=p'(x)q(x)\).
	
	\item \(\M(\CP^1)\): the field of meromorphic functions on the Riemann sphere
	\(\CP^1\). We view a meromorphic function as a holomorphic map
	\[
	F:\CP^1\to\CP^1
	\]
	(finite values in \(\C\) are written as \([F(p):1]\), poles as \([1:0]\)).
\end{itemize}

We want a \emph{field isomorphism}
\[
\Phi:\C(x)\xrightarrow{\ \simeq\ }\M(\CP^1).
\]

\subsection*{Step 1: Fix the affine chart on \(\CP^1\)}

Consider the standard chart
\[
U_1 := \{[z_0:z_1]\in\CP^1 \mid z_1\neq 0\},
\qquad
\phi_1:U_1\to\C,\quad
\phi_1([z_0:z_1]) = \frac{z_0}{z_1}.
\]
We write
\[
x := \phi_1,
\]
and think of \(x\) as a \emph{coordinate function} on \(U_1\). It extends
meromorphically to all of \(\CP^1\) with a simple pole at \(\infty=[1:0]\).

Intuitively: \(U_1 \simeq \C\) and \(x\) is the coordinate \(z\).

\subsection*{Step 2: From \(\C(x)\) to \(\M(\CP^1)\)}

Let
\[
R(x) = \frac{p(x)}{q(x)} \in \C(x),\quad p,q\in\C[x],\ q\not\equiv 0.
\]

\paragraph{2a. Define \(F_R\) on the affine chart \(U_1\).}

Take a point \([z_0:z_1]\in U_1\). Write
\[
x([z_0:z_1]) = \phi_1([z_0:z_1]) = \frac{z_0}{z_1} =: z.
\]

We \emph{define} a map \(F_R\) on \(U_1\) by
\[
\phi_1\big(F_R([z_0:z_1])\big) = R\big(\phi_1([z_0:z_1])\big) = R(z),
\]
i.e.
\[
F_R|_{U_1} = \phi_1^{-1}\circ R\circ \phi_1.
\]

Concretely, for \(z_1\neq 0\) and \(R(z)\neq\infty\),
\[
F_R([z_0:z_1]) = [R(z_0/z_1):1].
\]
If \(R(z)=\infty\) (i.e.\ \(q(z)=0\)), we set
\[
F_R([z_0:z_1]) = [1:0].
\]

So on \(\CP^1\setminus\{\infty\}\cong\C\), \(F_R\) is just ``apply \(R\) in the
coordinate \(x\)''.

\paragraph{2b. Extend \(F_R\) globally using homogeneous polynomials.}

Let
\[
m = \max\{\deg p,\,\deg q\},
\]
and define homogeneous polynomials of degree \(m\) by
\[
P(z_0,z_1) = z_1^m\, p\!\left(\frac{z_0}{z_1}\right),
\qquad
Q(z_0,z_1) = z_1^m\, q\!\left(\frac{z_0}{z_1}\right).
\]
Then for any \([z_0:z_1]\in\CP^1\), define
\[
F_R([z_0:z_1]) =
\begin{cases}
	[P(z_0,z_1):Q(z_0,z_1)], & Q(z_0,z_1)\neq 0,\\[4pt]
	[1:0], & Q(z_0,z_1)=0.
\end{cases}
\]

Facts:
\begin{itemize}
	\item \([P:Q]\) is a well-defined point of \(\CP^1\) (projective coordinates).
	\item Since \(P,Q\) are homogeneous polynomials, this gives a holomorphic
	map \(\CP^1\to\CP^1\).
	\item On \(U_1\), this agrees with \(\phi_1^{-1}\circ R\circ\phi_1\).
\end{itemize}

Thus \(F_R\) is a meromorphic function on \(\CP^1\), i.e.\ \(F_R\in\M(\CP^1)\).

\paragraph{2c. Define the map \(\Phi\).}

We have a function
\[
\Phi:\C(x)\longrightarrow \M(\CP^1),\qquad
\Phi(R)=F_R.
\]
One checks directly:
\[
\Phi(R_1+R_2) = \Phi(R_1)+\Phi(R_2),\qquad
\Phi(R_1R_2) = \Phi(R_1)\,\Phi(R_2),
\]
so \(\Phi\) is a field homomorphism.

\subsection*{Step 3: Every meromorphic function on \(\CP^1\) is rational}

Now we go in the other direction.

Let \(F\in \M(\CP^1)\), so \(F:\CP^1\to\CP^1\) is holomorphic.

\paragraph{3a. Restrict to the affine chart \(\CP^1\setminus\{\infty\}\cong\C\).}

On the chart \(U_1\) with coordinate \(x=\phi_1\), consider
\[
f := \phi_1\circ F\circ \phi_1^{-1} : \C\cup\{\infty\}\to\C\cup\{\infty\}.
\]

In words: write \(F\) in the coordinate \(x\). Then \(f\) is a meromorphic
function on the Riemann sphere \(\C\cup\{\infty\}\).

\paragraph{3b. Use the theorem from complex analysis.}

A standard result: a meromorphic function on the Riemann sphere
\(\C\cup\{\infty\}\) is a rational function. Therefore there exists some
\(R(x)\in\C(x)\) such that
\[
f(x) = R(x)\quad\text{for all }x\in\C\cup\{\infty\}.
\]

Translating back through \(\phi_1\), this means
\[
\phi_1\circ F\circ \phi_1^{-1} = R
\quad\Longleftrightarrow\quad
F = \phi_1^{-1}\circ R\circ \phi_1 = F_R.
\]

Thus every \(F\in\M(\CP^1)\) is of the form \(F_R\) for a unique \(R\in\C(x)\).

\subsection*{Step 4: \(\Phi\) is an isomorphism}

We already defined
\[
\Phi:\C(x)\to\M(\CP^1),\quad R\mapsto F_R.
\]

\begin{itemize}
	\item \textbf{Injective:} If \(\Phi(R_1)=\Phi(R_2)\), then
	\(F_{R_1}=F_{R_2}\). On \(U_1\simeq\C\),
	\[
	\phi_1\circ F_{R_1} = R_1\circ\phi_1,\qquad
	\phi_1\circ F_{R_2} = R_2\circ\phi_1.
	\]
	So \(R_1=R_2\) as functions on \(\C\), hence as elements of \(\C(x)\).
	Therefore \(R_1=R_2\).
	
	\item \textbf{Surjective:} Given any \(F\in\M(\CP^1)\), Step~3 produced a
	unique \(R\in\C(x)\) with \(F=F_R\). So every \(F\) is in the image of
	\(\Phi\).
\end{itemize}

Hence \(\Phi\) is a bijective field homomorphism:
\[
\boxed{\C(x)\;\cong\;\M(\CP^1)}.
\]

\subsection*{Step 5: Intuitive summary}

\begin{itemize}
	\item The chart \(\phi_1\) identifies \(\CP^1\setminus\{\infty\}\) with \(\C\).
	\item A rational function \(R(x)\) is just an ordinary meromorphic function on
	\(\C\cup\{\infty\}\).
	\item Using \(\phi_1\), we can transport \(R\) to a map \(F_R:\CP^1\to\CP^1\),
	which is meromorphic on the sphere.
	\item Conversely, any meromorphic map \(F:\CP^1\to\CP^1\) becomes a
	meromorphic \(f\) on \(\C\cup\{\infty\}\), hence a rational function \(R(x)\).
\end{itemize}

So “rational functions in the variable \(x\)” and “meromorphic functions on the
Riemann sphere” are two different \emph{descriptions} of the same field.
\end{document}
