\documentclass[11pt]{article}
\usepackage{amsmath,amssymb,amsthm}
\usepackage[margin=1in]{geometry}

\newcommand{\C}{\mathbb{C}}
\newcommand{\CP}{\mathbb{CP}}
\newcommand{\M}{\mathcal{M}}
\newcommand{\Res}{\operatorname{Res}}

\begin{document}
	
	\title{Why $\M(\CP^1)\cong\C(z)$ via Differential Forms and Integrals}
	\author{}
	\date{}
	\maketitle
	
	\section{Setup: $\CP^1$ as the Riemann sphere}
	
	We identify the complex projective line with the Riemann sphere:
	\[
	\CP^1 \cong \widehat{\C} := \C \cup \{\infty\}.
	\]
	On $\C \subset \widehat{\C}$ we use the usual complex coordinate $z$. The point
	$\infty$ is the ``point at infinity''.
	
	A \emph{meromorphic function} on $\CP^1$ is then a function
	\[
	f:\widehat{\C}\to\widehat{\C}
	\]
	that is holomorphic on $\widehat{\C}$ except for isolated poles (including possibly a pole at $\infty$).
	
	Our goal is to prove:
	\[
	\M(\CP^1) = \{\text{meromorphic functions on }\CP^1\}
	\cong \C(z),
	\]
	that is, every meromorphic $f$ on $\CP^1$ is a rational function in the
	coordinate $z$.
	
	We will base the proof on:
	
	\begin{itemize}
		\item meromorphic $1$-forms $\omega = f(z)\,dz$,
		\item residues and contour integrals,
		\item Laurent expansions defined by integrals.
	\end{itemize}
	
	\section{Meromorphic $1$-forms and residues}
	
	Let $f$ be a meromorphic function on $\widehat{\C}$, and consider the meromorphic $1$-form
	\[
	\omega = f(z)\,dz.
	\]
	For any piecewise smooth closed loop $\gamma$ in $\C$ avoiding the poles of $f$,
	we can form the integral
	\[
	\oint_\gamma \omega = \oint_\gamma f(z)\,dz.
	\]
	
	\subsection{Residues at finite poles (Cauchy point of view)}
	
	Let $a\in\C$ be a pole of $f$. Take a small positively oriented circle
	\[
	\gamma_a : z = a + r e^{it},\quad 0\le t\le 2\pi,
	\]
	small enough that it encloses no other poles of $f$. The \emph{residue} of
	$\omega$ at $a$ is defined by
	\[
	\Res_{z=a}(f(z)\,dz)
	:= \frac{1}{2\pi i}\oint_{\gamma_a} f(z)\,dz.
	\]
	
	Equivalently, on an annulus $0<|z-a|<\varepsilon$, $f$ has a Laurent
	expansion
	\[
	f(z) = \sum_{n=-m}^{\infty} c_n (z-a)^n,
	\]
	then the coefficient $c_{-1}$ of $(z-a)^{-1}$ is the residue:
	\[
	\Res_{z=a}(f(z)\,dz) = c_{-1}.
	\]
	
	These coefficients $c_n$ can be obtained from integrals. For each integer $n$,
	\[
	c_n = \frac{1}{2\pi i} \oint_{\gamma_a}
	\frac{f(\zeta)}{(\zeta-a)^{n+1}}\,d\zeta,
	\]
	so in particular
	\[
	c_{-1} = \frac{1}{2\pi i}\oint_{\gamma_a} f(\zeta)\,d\zeta.
	\]
	Thus the principal part of $f$ at a finite pole is determined entirely by
	integrals of the $1$-form $f(\zeta)\,d\zeta$.
	
	\subsection{Residue at infinity}
	
	We also need the residue at $\infty$. There are two equivalent ways to define
	it.
	
	\subsubsection*{Method A: change variable $w = 1/z$}
	
	Define $w = 1/z$, so $z=1/w$ and $dz = -w^{-2}\,dw$. Let
	\[
	F(w) := f\!\left(\frac{1}{w}\right).
	\]
	Then the $1$-form $\omega$ in terms of $w$ is
	\[
	\omega = f(z)\,dz
	= f\!\left(\frac{1}{w}\right)\left(-\frac{1}{w^2}\,dw\right)
	= -F(w)w^{-2}\,dw.
	\]
	Since $f$ is meromorphic at $\infty$, the function $F(w)w^{-2}$ has a Laurent
	expansion near $w=0$:
	\[
	F(w)w^{-2} = \sum_{n=-M}^{\infty} a_n w^n
	\]
	with finitely many negative powers. Then the residue at $\infty$ is defined as
	\[
	\Res_{z=\infty}(f(z)\,dz)
	:= -\,\Res_{w=0}(F(w)w^{-2}\,dw).
	\]
	If
	\[
	F(w)w^{-2} = \cdots + a_{-1}w^{-1} + a_0 + a_1 w + \cdots,
	\]
	then
	\[
	\Res_{z=\infty}(f(z)\,dz) = -a_{-1}.
	\]
	
	\subsubsection*{Method B: big circle and global residue theorem}
	
	Let $R$ be so large that the circle
	\[
	\Gamma_R: z = Re^{it},\quad 0\le t\le 2\pi,
	\]
	encloses all finite poles of $f$. Then the residue theorem on $\C$ gives
	\[
	\oint_{\Gamma_R} f(z)\,dz
	= 2\pi i \sum_{a_j\in\C} \Res_{z=a_j}(f(z)\,dz),
	\]
	where the sum is over all finite poles $a_j$ of $f$.
	
	On the Riemann sphere $\widehat{\C}$, the \emph{global residue theorem} says:
	\[
	\sum_{p\in\widehat{\C}} \Res_p(f(z)\,dz) = 0.
	\]
	Thus
	\[
	\Res_{z=\infty}(f(z)\,dz)
	= -\sum_{a_j\in\C} \Res_{z=a_j}(f(z)\,dz).
	\]
	So the residue at infinity is completely determined by the residues at finite
	poles (and vice versa).
	
	\section{Meromorphic $f$ on $\widehat{\C}$: poles and Laurent expansions}
	
	Let $f$ be a meromorphic function on $\widehat{\C}$.
	
	\subsection{Finitely many poles}
	
	Since $\widehat{\C}$ is compact and poles are isolated, $f$ has only finitely
	many poles. Thus there exist points
	\[
	a_1,\dots,a_N \in \C\cup\{\infty\}
	\]
	such that all poles of $f$ are among these $a_j$.
	
	\subsection{Laurent expansions at finite poles (via integrals)}
	
	Fix a finite pole $a_j\in\C$. There exists a small circle $\gamma_j$ around
	$a_j$ enclosing no other poles. On an annulus $0<|z-a_j|<\varepsilon$, $f$ has
	a Laurent expansion
	\[
	f(z) = \sum_{n=-m_j}^\infty c_{j,n}(z-a_j)^n,
	\]
	where $m_j\ge 1$ and $c_{j,-m_j}\neq 0$. Each coefficient is given by
	\[
	c_{j,n} = \frac{1}{2\pi i} \oint_{\gamma_j}
	\frac{f(\zeta)}{(\zeta-a_j)^{n+1}}\,d\zeta.
	\]
	
	The \emph{principal part} of $f$ at $a_j$ is
	\[
	\operatorname{PP}_{a_j}(f)(z)
	:= \sum_{n=-m_j}^{-1} c_{j,n}(z-a_j)^n.
	\]
	This is a finite sum of negative powers of $(z-a_j)$, with coefficients
	defined by integrals of $f(\zeta)\,d\zeta$.
	
	\subsection{Laurent expansion at infinity (via integrals)}
	
	At $\infty$, use $w = 1/z$ as local coordinate. Define
	\[
	F(w) := f\!\left(\frac{1}{w}\right).
	\]
	Since $f$ is meromorphic at $\infty$, there is an integer $M\ge 0$ and
	coefficients $b_n$ such that
	\[
	F(w) = \sum_{n=-M}^\infty b_n w^n
	\]
	for $0<|w|<\varepsilon$. These $b_n$ are also given by Cauchy integrals:
	\[
	b_n = \frac{1}{2\pi i} \oint_{|\xi|=\rho} 
	\frac{F(\xi)}{\xi^{n+1}}\,d\xi,
	\]
	for sufficiently small $\rho>0$.
	
	The \emph{principal part} at $\infty$ is
	\[
	\operatorname{PP}_\infty(f)(w)
	:= \sum_{n=-M}^{-1} b_n w^n.
	\]
	In terms of the original variable $z=1/w$, note that $w^{-k} = z^k$, so
	$\operatorname{PP}_\infty(f)$ corresponds to a \emph{polynomial} in $z$:
	\[
	P(z) = \sum_{k=1}^M \tilde b_k\,z^k.
	\]
	
	\section{Constructing a rational function $R(z)$ with the same principal parts}
	
	We now build a single rational function $R(z)$ that has exactly the same
	principal parts as $f$ at each pole $a_j$ (including $\infty$).
	
	\subsection{Definition of $R(z)$}
	
	For each finite pole $a_j\in\C$, write the principal part as
	\[
	\operatorname{PP}_{a_j}(f)(z)
	= \sum_{k=1}^{m_j} \frac{c_{j,-k}}{(z-a_j)^k}.
	\]
	For the pole at $\infty$, we have a polynomial $P(z)$ as above.
	
	Define
	\[
	R(z) :=
	P(z) + \sum_{j=1}^N \operatorname{PP}_{a_j}(f)(z)
	= P(z) + \sum_{j=1}^N\sum_{k=1}^{m_j} \frac{c_{j,-k}}{(z-a_j)^k}.
	\]
	
	Each term is a rational function in $z$. Hence
	\[
	R(z) \in \C(z).
	\]
	
	By construction:
	
	\begin{itemize}
		\item At each finite pole $a_j$, $f$ and $R$ have the same principal part.
		\item At $\infty$, $f$ and $R$ also have the same principal part.
	\end{itemize}
	
	\section{The difference $g = f - R$ is holomorphic everywhere}
	
	Define
	\[
	g(z) := f(z) - R(z).
	\]
	
	\subsection{Behavior at finite points}
	
	At each finite pole $a_j\in\C$, $f$ and $R$ have the same principal part, so
	the negative powers in the Laurent expansion cancel. Thus near $a_j$ we have
	\[
	g(z) = \sum_{n=0}^\infty d_{j,n}(z-a_j)^n,
	\]
	i.e.\ $g$ is holomorphic at $a_j$.
	
	At points where $f$ (hence $R$) is already holomorphic, clearly $g$ is also
	holomorphic. Therefore $g$ is holomorphic on all of $\C$.
	
	\subsection{Behavior at infinity}
	
	At $\infty$, in the coordinate $w=1/z$, $f$ and $R$ have the same principal
	part in $w$. Thus the Laurent expansion of $g(1/w)$ at $w=0$ has no negative
	powers:
	\[
	g\!\left(\frac{1}{w}\right)
	= \sum_{n=0}^\infty d_n w^n.
	\]
	So $g$ is holomorphic at $w=0$, i.e.\ at $z=\infty$.
	
	Therefore $g$ is holomorphic at every point of $\widehat{\C}$. In other words,
	$g$ is a global holomorphic function
	\[
	g:\CP^1\to\C.
	\]
	
	\section{Holomorphic on $\CP^1$ implies constant}
	
	The Riemann sphere $\CP^1$ is compact. A holomorphic function on a compact
	Riemann surface is bounded. By the maximum modulus principle (or Liouville's
	theorem), such a function must be constant.
	
	Therefore there exists $C\in\C$ with
	\[
	g(z) \equiv C,
	\]
	i.e.
	\[
	f(z) = R(z) + C.
	\]
	
	Since $R(z)\in\C(z)$, we conclude that $f(z)$ is also rational:
	\[
	f(z) \in \C(z).
	\]
	
	This proves that \emph{every} meromorphic function on $\CP^1$ is a rational
	function in $z$.
	
	\section{Conclusion}
	
	We have shown, using only:
	
	\begin{itemize}
		\item meromorphic $1$-forms $\omega = f(z)\,dz$,
		\item residues, defined as contour integrals,
		\item Laurent coefficients extracted from integrals,
		\item and the maximum modulus principle on the compact surface $\CP^1$,
	\end{itemize}
	
	that
	\[
	\M(\CP^1) = \C(z).
	\]
	
	Thus there is an isomorphism of fields
	\[
	\M(\CP^1) \xrightarrow{\ \cong\ } \C(z),
	\]
	sending a meromorphic function $f$ on the sphere to its expression as a
	rational function in the affine (stereographic) coordinate $z$.
	
\end{document}
``
