\documentclass[11pt]{article}
\usepackage{amsmath,amssymb,amsthm}
\newcommand{\C}{\mathbb{C}}
\newcommand{\CP}{\mathbb{CP}}
\newcommand{\M}{\mathcal{M}}

\begin{document}
	
	\section*{Calculus / differential-form viewpoint on \(\C(x)\simeq \M(\CP^1)\)}
	
	We want to show:
	\[
	\M(\CP^1) \;\cong\; \C(x),
	\]
	using mainly complex analysis (Laurent series, Liouville, change of
	variables) and a bit of differential-form language.
	
	\subsection*{1. Charts on \(\CP^1\) and the coordinate \(x\)}
	
	Think of \(\CP^1\) as the Riemann sphere \(\C\cup\{\infty\}\).
	Use two standard charts:
	
	\begin{itemize}
		\item \(U_1 = \CP^1\setminus\{\infty\}\) with coordinate
		\[
		z = \frac{z_0}{z_1} : U_1\to\C.
		\]
		We call this coordinate \(x\). So \(x=z\) on \(\C\).
		
		\item \(U_0 = \CP^1\setminus\{0\}\) with coordinate
		\[
		w = \frac{z_1}{z_0} : U_0\to\C,
		\]
		so that on the overlap \(U_0\cap U_1\) we have
		\[
		w = \frac{1}{z},\qquad z=\frac{1}{w}.
		\]
		Here \(w\) is a coordinate near \(\infty\) (since \(\infty=[1:0]\)
		corresponds to \(w=0\)).
	\end{itemize}
	
	\subsection*{2. Meromorphic functions and meromorphic 1-forms}
	
	A \emph{meromorphic function} \(f\) on \(\CP^1\) is the same as a
	holomorphic map
	\[
	F:\CP^1\to\CP^1,
	\]
	whose restriction to \(\C\subset\CP^1\) is an ordinary meromorphic function
	\(f(z)\) on \(\C\), and with some allowed behavior at \(\infty\).
	
	From the viewpoint of calculus/differential forms, you often look at the
	1-form
	\[
	\omega = f(z)\,dz.
	\]
	On \(\C\), this is a meromorphic 1-form. To extend \(f\) (or \(\omega\)) to the
	sphere, we must check what happens near \(\infty\). That means rewriting
	everything in the coordinate \(w=1/z\).
	
	On the overlap:
	\[
	z = \frac{1}{w},\qquad dz = -\frac{1}{w^2}\,dw.
	\]
	Thus
	\[
	\omega = f(z)\,dz
	= f\!\left(\frac{1}{w}\right)\cdot\left(-\frac{1}{w^2}\,dw\right)
	= -\,f\!\left(\frac{1}{w}\right)\,w^{-2}\,dw.
	\]
	So the expression of \(\omega\) near \(\infty\) (in the coordinate \(w\)) is
	\[
	\omega = G(w)\,dw,\qquad
	G(w) = -\,f\!\left(\frac{1}{w}\right)\,w^{-2}.
	\]
	
	\emph{Meromorphicity at \(\infty\)} means: \(G(w)\) has a Laurent series at
	\(w=0\) with only finitely many negative powers.
	
	\subsection*{3. Global calculus proof that meromorphic on \(\CP^1\) is rational}
	
	Let \(f:\CP^1\to\CP^1\) be meromorphic. Restricting to \(\C\subset\CP^1\),
	we view \(f\) as a meromorphic function \(f(z)\) on \(\C\).
	
	\begin{itemize}
		\item On a compact Riemann surface like \(\CP^1\), a meromorphic function
		has only finitely many poles. So there exist points
		\(a_1,\dots,a_k\in\C\cup\{\infty\}\) and positive integers \(m_j\) such that
		all poles of \(f\) lie at the \(a_j\).
		
		\item For each finite pole \(a_j\in\C\) (with local coordinate \(z-a_j\)),
		\(f\) has a Laurent expansion
		\[
		f(z) = \sum_{n=-m_j}^{\infty} c_{j,n}\,(z-a_j)^n
		\]
		near \(z=a_j\). Its \emph{principal part} at \(a_j\) is
		\[
		P_j(z) := \sum_{n=-m_j}^{-1} c_{j,n}\,(z-a_j)^n.
		\]
		
		\item Near \(\infty\), change variables to \(w=1/z\). Then
		\[
		f(z) = f\!\left(\tfrac{1}{w}\right)
		\]
		has a Laurent expansion in \(w\):
		\[
		f\!\left(\tfrac{1}{w}\right) = \sum_{n=-M}^{\infty} b_n\,w^n,
		\]
		with finitely many negative powers \(w^n\) (this is the definition of
		``meromorphic at \(\infty\)''). Equivalently,
		\[
		f(z) = \sum_{n=-M}^{\infty} b_n\,z^{-n}
		\]
		for large \(|z|\). Its principal part at \(\infty\) is
		\[
		P_\infty(z) := \sum_{n=-M}^{-1} b_n\,z^{-n}.
		\]
	\end{itemize}
	
	Now define a \emph{rational function} \(R(z)\) by summing all principal parts:
	\[
	R(z) = \sum_{j=1}^k P_j(z) + P_\infty(z).
	\]
	This is a rational function because it is a finite sum of expressions of
	the form \((z-a_j)^{-n}\) and powers of \(z\) (which are all rational).
	
	Then consider
	\[
	g(z) := f(z) - R(z).
	\]
	By construction:
	
	\begin{itemize}
		\item At each finite pole \(a_j\), \(g\) has no negative powers in its
		Laurent expansion, hence is holomorphic at \(a_j\).
		\item At \(\infty\), subtracting the principal part \(P_\infty(z)\) removes
		all negative powers in the expansion in \(w=1/z\), so \(g\) is
		holomorphic at \(\infty\) as well.
	\end{itemize}
	
	Thus \(g\) is entire on \(\C\) and \emph{holomorphic at \(\infty\)}. Being
	holomorphic at \(\infty\) means exactly that \(g\) is \emph{bounded} for large
	\(|z|\). By Liouville’s theorem:
	\[
	g \text{ entire and bounded } \Longrightarrow g \text{ is constant.}
	\]
	
	So there exists \(C\in\C\) such that
	\[
	f(z) = R(z) + C.
	\]
	But \(R(z)+C\) is still rational, so \(f\) is a rational function:
	\[
	f(z)\in\C(z).
	\]
	
	Therefore, any meromorphic function \(f\) on \(\CP^1\) is a rational function
	in the coordinate \(x=z\). This shows \(\M(\CP^1) \subset \C(x)\), and the
	converse inclusion (every rational function defines a meromorphic map
	\(\CP^1\to\CP^1\)) is straightforward. Hence
	\[
	\M(\CP^1) \cong \C(x).
	\]
	
	\subsection*{4. Differential-form language (optional viewpoint)}
	
	Instead of looking directly at \(f\), one may look at the meromorphic
	1-form
	\[
	\omega = f(z)\,dz.
	\]
	
	\begin{itemize}
		\item At each finite pole \(a_j\), \(\omega\) has a Laurent expansion
		\[
		\omega = \left( \sum_{n=-m_j}^{\infty} c_{j,n}(z-a_j)^n \right)\,dz.
		\]
		Its residue at \(a_j\) is \(c_{j,-1}\).
		
		\item Near \(\infty\), write \(z=1/w\), \(dz=-w^{-2}dw\). Then
		\[
		\omega = f(1/w)\,dz = -\,f(1/w)\,w^{-2}dw.
		\]
		Meromorphicity at \(\infty\) is the condition that the coefficient of
		\(dw\) has only finitely many negative powers of \(w\).
		
		\item On a compact Riemann surface (like \(\CP^1\)), the sum of the
		residues of any meromorphic 1-form is zero:
		\[
		\sum_{p\in\CP^1} \operatorname{Res}_p(\omega) = 0.
		\]
		This is a global version of the Cauchy integral theorem.
	\end{itemize}
	
	The partial fractions decomposition of \(f\) can be seen as reconstructing
	\(\omega\) from its local principal parts (residues, higher-order terms) at
	each pole, then subtracting this from the given \(\omega\) to get a
	holomorphic 1-form on \(\CP^1\). But any global holomorphic 1-form on
	\(\CP^1\) must be zero, so what remains is just a constant multiple of \(dz\),
	which corresponds to the polynomial part of \(f\). This reproduces the same
	``\(f = \text{rational} + \text{constant}\)'' conclusion.
	
	\medskip
	
	In summary, from a calculus/differential-forms viewpoint:
	
	\begin{itemize}
		\item Meromorphic \(f\) on \(\CP^1\) has finitely many poles (including
		possibly \(\infty\)).
		\item Using Laurent expansions and the change of variables \(w=1/z\), we
		build a rational function \(R(z)\) with the same principal parts as \(f\).
		\item Their difference \(g=f-R\) is entire and bounded at \(\infty\), hence
		constant by Liouville.
		\item Therefore \(f\) is rational: \(\M(\CP^1)=\C(x)\).
	\end{itemize}
	
\end{document}
