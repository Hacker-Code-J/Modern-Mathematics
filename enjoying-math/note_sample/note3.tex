\documentclass[11pt]{article}
\usepackage[margin=1in]{geometry}
\usepackage{amsmath,amssymb,amsthm}

\newtheorem{theorem}{Theorem}
\newtheorem{remark}{Remark}

\newcommand{\C}{\mathbb{C}}
\newcommand{\CP}{\mathbb{CP}}
\newcommand{\M}{\mathcal{M}}
\newcommand{\Res}{\operatorname{Res}}
\newcommand{\Divv}{\operatorname{Div}}
\newcommand{\ord}{\operatorname{ord}}

\begin{document}
	
	\title{From $\M(\CP^1)\cong\C(z)$ to $\M(\C/\Lambda)\cong\C(\wp,\wp')$\\
		with Detailed Computations}
	\author{}
	\date{}
	\maketitle
	
	\tableofcontents
	
	\bigskip
	
	%%%%%%%%%%%%%%%%%%%%%%%%%%%%%%%%%%%%%%%%%%%%%%%%%%%%%%%%%%%%
	\section{Reminder: $\M(\CP^1)\cong\C(z)$ as the genus $0$ prototype}
	
	We briefly recall the key fact:
	
	\begin{theorem}\label{thm:MCP1Cz}
		Let $\CP^1\cong\C\cup\{\infty\}$ be the Riemann sphere with affine
		coordinate $z$ on $\C$. Then
		\[
		\M(\CP^1) = \C(z),
		\]
		i.e.\ every meromorphic function on $\CP^1$ is a rational function in $z$.
	\end{theorem}
	
	\begin{proof}[Sketch (analytic)]
		Let $f$ be meromorphic on $\CP^1$. Then $f$ has finitely many poles
		$a_1,\dots,a_N\in\C\cup\{\infty\}$. Around each finite pole $a_j\in\C$, we
		have a Laurent expansion
		\[
		f(z) = \sum_{n=-m_j}^\infty c_{j,n}(z-a_j)^n,
		\]
		and similarly at $\infty$ in the coordinate $w=1/z$:
		\[
		f\!\left(\frac{1}{w}\right) = \sum_{n=-M}^\infty b_n w^n.
		\]
		
		The \emph{principal part} at each pole is a finite sum of negative powers.
		Using these principal parts, we build a rational function
		\[
		R(z) = P(z) + \sum_{j=1}^N \sum_{k=1}^{m_j}
		\frac{c_{j,-k}}{(z-a_j)^k},
		\]
		where $P(z)$ is the polynomial corresponding to the principal part at
		infinity. By construction, $R$ has exactly the same principal parts as $f$
		at all poles (finite and at infinity). Hence $g:=f-R$ has no pole anywhere
		on $\CP^1$, i.e.\ $g$ is holomorphic on the compact surface $\CP^1$.
		Therefore $g$ is constant, so $f=R+(\text{constant})$ is rational in $z$.
	\end{proof}
	
	Thus the sphere is characterized by having a \emph{single global coordinate}
	$z$ whose field of meromorphic functions is exactly $\C(z)$.
	
	%%%%%%%%%%%%%%%%%%%%%%%%%%%%%%%%%%%%%%%%%%%%%%%%%%%%%%%%%%%%
	\section{General theorem: $\M(X)$ is finite over $\C(f)$}
	
	Let $X$ be a compact Riemann surface and $f\in\M(X)$ a nonconstant meromorphic
	function. Then $f$ defines a holomorphic map
	\[
	f:X\to\CP^1,
	\]
	by
	\[
	f(p)=
	\begin{cases}
		[f(p):1], & \text{$p$ not a pole of $f$},\\[4pt]
		[1:0],    & \text{$p$ a pole of $f$}.
	\end{cases}
	\]
	
	\subsection{Degree of $f$ and the field extension}
	
	The map $f:X\to\CP^1$ is \emph{finite} and has a well-defined degree
	$\deg(f)\in\mathbb{N}$: for a generic $w\in\CP^1$, the fiber $f^{-1}(w)$
	consists of $\deg(f)$ points counted with multiplicity.
	
	On the level of function fields, $f$ induces an embedding
	\[
	f^*:\M(\CP^1)\hookrightarrow \M(X),\qquad
	\phi(z)\mapsto \phi\bigl(f\bigr).
	\]
	Identifying $\M(\CP^1)\cong\C(z)$ (Theorem~\ref{thm:MCP1Cz}), we get an
	inclusion
	\[
	\C(z)\hookrightarrow \M(X).
	\]
	
	\begin{theorem}[Standard fact]\label{thm:finiteextension}
		Let $X$ be a compact Riemann surface and $f:X\to\CP^1$ a nonconstant
		meromorphic function. Then:
		\begin{enumerate}
			\item $\M(X)$ is a finite extension of $\C(z)$, where $z$ is the affine
			coordinate on $\CP^1$ (pulled back by $f$).
			\item The degree of this field extension equals the degree of the map:
			\[
			[\M(X):\C(z)] = \deg(f).
			\]
		\end{enumerate}
	\end{theorem}
	
	\begin{proof}[Very sketchy idea]
		One shows that the transcendence degree of $\M(X)$ over $\C$ is $1$ (since
		$X$ is a one-dimensional complex manifold). The map $f$ gives a nonconstant
		element of $\M(X)$, so $\C(f)$ is isomorphic to $\C(z)$. Then every other
		meromorphic function $g\in\M(X)$ satisfies a polynomial relation
		\[
		P(f,g) = 0
		\]
		with $P\in\C[T,U]$ not the zero polynomial (this uses compactness and
		boundedness arguments or more algebraic geometry tools). This shows $g$ is
		algebraic over $\C(f)$, hence $\M(X)$ is algebraic over $\C(f)$.
		
		Finiteness and the equality of degrees $[\M(X):\C(f)]=\deg(f)$ can be
		shown by comparing zeros and poles of pullbacks of rational functions, or
		using Riemann--Hurwitz. For the purposes of this note, we accept this as a
		standard result from the theory of compact Riemann surfaces.
	\end{proof}
	
	\begin{remark}
		For $X=\CP^1$ and $f=z$, Theorem~\ref{thm:finiteextension} tells us
		$[\M(\CP^1):\C(z)]=1$, which matches $\M(\CP^1)=\C(z)$.
	\end{remark}
	
	%%%%%%%%%%%%%%%%%%%%%%%%%%%%%%%%%%%%%%%%%%%%%%%%%%%%%%%%%%%%
	\section{Torus case: $X=\C/\Lambda$ and the Weierstrass $\wp$}
	
	Now let $\Lambda\subset\C$ be a lattice generated by $\omega_1,\omega_2$ with
	$\Im(\omega_2/\omega_1)>0$. The complex torus is
	\[
	X = \C/\Lambda.
	\]
	We denote by $[z]$ the class of $z\in\C$ modulo $\Lambda$.
	
	\subsection{Definition and basic properties of $\wp$}
	
	The Weierstrass $\wp$-function associated with $\Lambda$ is defined by
	\[
	\wp(z)
	= \frac{1}{z^2}
	+ \sum_{\lambda\in\Lambda\setminus\{0\}}
	\left(
	\frac{1}{(z-\lambda)^2} - \frac{1}{\lambda^2}
	\right).
	\]
	
	Basic facts (all standard; we list them for context):
	
	\begin{itemize}
		\item The series converges normally on compact subsets of
		$\C\setminus\Lambda$, so $\wp$ is meromorphic on $\C$.
		\item $\wp$ is $\Lambda$-periodic:
		\[
		\wp(z+\lambda) = \wp(z),\quad\forall\lambda\in\Lambda.
		\]
		\item $\wp$ is even:
		\[
		\wp(-z) = \wp(z).
		\]
		\item The only poles of $\wp$ (modulo $\Lambda$) are at the lattice points,
		with a double pole at each $\lambda\in\Lambda$.
	\end{itemize}
	
	Because of periodicity, $\wp$ descends to a meromorphic function on the torus:
	\[
	\wp_X: X\to\CP^1,\quad \wp_X([z]) = \wp(z).
	\]
	We will usually just write $\wp$ for the descended function as well.
	
	\subsection{Local expansion of $\wp$ and $\wp'$ near $0$}
	
	By symmetry and periodicity, it suffices to study $\wp$ near $z=0$.
	
	One can show (expanding the series or using evenness and the type of pole)
	that near $z=0$,
	\begin{equation}\label{eq:wpLocalExpansion}
		\wp(z) = \frac{1}{z^2} + c_2 z^2 + c_4 z^4 + c_6 z^6 + \cdots,
	\end{equation}
	for some constants $c_{2k}\in\C$ depending on the lattice.
	
	Differentiating termwise,
	\[
	\wp'(z) = -\frac{2}{z^3} + 2c_2 z + 4c_4 z^3 + 6c_6 z^5 + \cdots.
	\]
	
	Thus:
	
	\begin{itemize}
		\item $\wp$ has a double pole at $z=0$:
		\[
		\ord_0(\wp) = -2.
		\]
		\item $\wp'$ has a triple pole at $z=0$:
		\[
		\ord_0(\wp') = -3.
		\]
		\item $\wp$ is even, $\wp'$ is odd: $\wp(-z)=\wp(z)$, $\wp'(-z)=-\wp'(z)$.
	\end{itemize}
	
	Because $\wp$ and $\wp'$ are $\Lambda$-periodic, in the quotient $X=\C/\Lambda$
	the function $\wp$ has a single double pole at $[0]$, and $\wp'$ has a
	single triple pole at $[0]$.
	
	\subsection{Degree of $\wp:X\to\CP^1$}
	
	We now compute the degree of the meromorphic map
	\[
	\wp:X\to\CP^1,\quad [z]\mapsto\wp(z).
	\]
	
	Fix a generic value $w\in\C$. The equation
	\[
	\wp(z) = w
	\]
	has solutions in pairs $\{z,-z\}$ because $\wp$ is even. For a generic value
	$w$ (i.e.\ provided $w$ is not a critical value), we have exactly two
	solutions modulo $\Lambda$.
	
	So the map $\wp:X\to\CP^1$ has degree
	\[
	\deg(\wp) = 2.
	\]
	
	By Theorem~\ref{thm:finiteextension}, the induced inclusion
	\[
	\C(x) \hookrightarrow \M(X),\quad x:=\wp,
	\]
	makes $\M(X)$ a field extension of degree $2$ over $\C(x)$:
	\[
	[\M(X):\C(\wp)] = 2.
	\]
	
	Thus $\M(X)$ is a \emph{quadratic extension} of the rational function field
	$\C(\wp)$, exactly analogous to an algebraic function field
	$\C(x,y)$ with a relation of the form $y^2=\cdots$.
	
	%%%%%%%%%%%%%%%%%%%%%%%%%%%%%%%%%%%%%%%%%%%%%%%%%%%%%%%%%%%%
	\section{The algebraic relation $(\wp')^2 = 4\wp^3 - g_2\wp - g_3$}
	
	We next derive the key differential equation satisfied by $\wp$ and $\wp'$.
	
	\subsection{Constructing an elliptic function with no poles}
	
	Consider the function
	\[
	F(z)
	:= \bigl(\wp'(z)\bigr)^2 - 4\bigl(\wp(z)\bigr)^3 + g_2\,\wp(z) + g_3,
	\]
	where $g_2,g_3\in\C$ are some constants we will choose.
	
	We want to show that for suitable $g_2,g_3$, this $F$ is identically zero.
	The strategy is:
	
	\begin{itemize}
		\item Note that $F$ is an elliptic (doubly periodic) meromorphic function.
		\item Show that $F$ has no poles (for appropriate $g_2,g_3$).
		\item Conclude that $F$ must be constant.
		\item Analyze the constant by looking at the expansion near $0$; choose
		$g_2,g_3$ so that the constant is $0$.
	\end{itemize}
	
	\subsection{Local expansion of $(\wp')^2$ and $\wp^3$ near $z=0$}
	
	From the expansions in \eqref{eq:wpLocalExpansion}, write
	\[
	\wp(z) = z^{-2} + a_2 z^2 + a_4 z^4 + a_6 z^6 + \cdots,
	\]
	\[
	\wp'(z) = -2 z^{-3} + b_1 z + b_3 z^3 + b_5 z^5 + \cdots,
	\]
	where $a_{2k},b_{2k-1}\in\C$.
	
	\paragraph{Compute $(\wp'(z))^2$ near $0$.}
	We have
	\begin{align*}
		(\wp'(z))^2
		&= \bigl(-2 z^{-3} + b_1 z + b_3 z^3 + \cdots\bigr)^2 \\
		&= 4 z^{-6} - 4b_1 z^{-2} + \text{(higher powers in $z$)}.
	\end{align*}
	More precisely, we can expand:
	\[
	(\wp'(z))^2
	= 4 z^{-6} + c_{-4} z^{-4} + c_{-2} z^{-2} + c_0 + c_2 z^2 + \cdots.
	\]
	
	\paragraph{Compute $4(\wp(z))^3$ near $0$.}
	First
	\[
	\wp(z)^3
	= (z^{-2} + a_2 z^2 + a_4 z^4 + \cdots)^3.
	\]
	Expanding:
	\begin{align*}
		\wp(z)^3
		&= z^{-6} + 3 a_2 z^{-2} + \text{(terms with $z^0,z^2,\dots$)}.
	\end{align*}
	Then
	\[
	4\wp(z)^3 = 4 z^{-6} + 12 a_2 z^{-2} + \text{(higher powers)}.
	\]
	
	So near $z=0$:
	\begin{align*}
		(\wp'(z))^2 - 4\wp(z)^3
		&= \bigl(4 z^{-6} + c_{-4} z^{-4} + c_{-2} z^{-2} + c_0 + \cdots\bigr)
		- \bigl(4 z^{-6} + 12 a_2 z^{-2} + \cdots\bigr) \\
		&= c_{-4} z^{-4} + (c_{-2} - 12 a_2) z^{-2} + c_0 + \cdots.
	\end{align*}
	
	The remarkable fact (which can be checked more systematically from the series
	definition of $\wp$) is that $c_{-4}=0$, so the $z^{-4}$ term vanishes.
	
	Even more structure: the combination
	\[
	(\wp'(z))^2 - 4\wp(z)^3
	\]
	has at worst a $z^{-2}$ term in its Laurent series. We then \emph{define}
	\[
	g_2 := -20 a_2,\qquad
	g_3 := \text{(suitable constant to kill the $z^0$ term)},
	\]
	so that
	\[
	(\wp'(z))^2 - 4\wp(z)^3 + g_2\wp(z) + g_3
	\]
	has no $z^{-6},z^{-4},z^{-2},z^0$ terms in its Laurent expansion at $z=0$,
	and in fact its Laurent expansion at $z=0$ starts at some positive power
	$z^2$ or higher.
	
	\subsection{No poles $\Rightarrow$ constant $\Rightarrow$ $0$}
	
	Because $\wp$ and $\wp'$ are elliptic, the function
	\[
	F(z) := \bigl(\wp'(z)\bigr)^2 - 4\bigl(\wp(z)\bigr)^3 + g_2\,\wp(z) + g_3
	\]
	is elliptic (meromorphic and $\Lambda$-periodic). We have arranged the
	coefficients $g_2,g_3$ so that \emph{near $z=0$} the Laurent expansion of
	$F$ has no negative powers. By periodicity, this is also true near any other
	lattice point (as the expansions are the same up to translation). Hence $F$
	has no poles on $\C$.
	
	Therefore $F$ is an elliptic function with no poles, hence entire and bounded
	(on the fundamental domain), thus constant. Let $C$ be this constant:
	\[
	F(z)\equiv C.
	\]
	
	Looking at the Laurent expansion at $z=0$, one finds that the constant term
	must be $0$ (we chose $g_3$ exactly to kill the constant term). Thus
	\[
	C=0,
	\]
	and we have the fundamental relation
	\[
	\boxed{
		\bigl(\wp'(z)\bigr)^2 = 4\bigl(\wp(z)\bigr)^3 - g_2\wp(z) - g_3.
	}
	\]
	
	Set
	\[
	x := \wp(z),\qquad y := \wp'(z).
	\]
	Then
	\[
	y^2 = 4x^3 - g_2 x - g_3.
	\]
	
	%%%%%%%%%%%%%%%%%%%%%%%%%%%%%%%%%%%%%%%%%%%%%%%%%%%%%%%%%%%%
	\section{Function field of the torus: $\M(\C/\Lambda)=\C(\wp,\wp')$}
	
	We already know $\C(\wp)\hookrightarrow\M(X)$ and that
	\[
	[\M(X):\C(\wp)] = 2.
	\]
	From the relation
	\[
	y^2 = 4x^3 - g_2 x - g_3,
	\]
	we see that $\wp'$ is \emph{algebraic} of degree $2$ over $\C(\wp)$, since it
	satisfies the quadratic equation
	\[
	Y^2 - (4x^3 - g_2 x - g_3) = 0.
	\]
	
	Thus
	\[
	\C(\wp,\wp') = \C(\wp)\bigl(\wp'\bigr)
	\]
	is a degree $\le 2$ extension of $\C(\wp)$. On the other hand, we already know
	from the degree of the map $\wp:X\to\CP^1$ that
	\[
	[\M(X):\C(\wp)] = 2.
	\]
	Hence
	\[
	\C(\wp,\wp') = \M(X).
	\]
	
	In more algebraic-geometric language, the pair $(x,y)=(\wp,\wp')$ gives an
	embedding of the torus as the complex curve
	\[
	Y^2 = 4X^3 - g_2 X - g_3
	\]
	in the affine plane, and the function field of that curve is exactly
	\[
	\C(X,Y)/(Y^2 - 4X^3 + g_2 X + g_3)
	\cong \C(\wp,\wp').
	\]
	
	\begin{theorem}\label{thm:torusfunctionfield}
		Let $X=\C/\Lambda$ be a complex torus, and $\wp,\wp'$ the associated
		Weierstrass functions. Then:
		\[
		\M(X) = \C(\wp,\wp'),
		\]
		with the single algebraic relation
		\[
		(\wp')^2 = 4\wp^3 - g_2\wp - g_3.
		\]
	\end{theorem}
	
	%%%%%%%%%%%%%%%%%%%%%%%%%%%%%%%%%%%%%%%%%%%%%%%%%%%%%%%%%%%%
	\section{Comparing with the sphere case}
	
	We can now directly compare:
	
	\begin{itemize}
		\item \textbf{Sphere}: $X=\CP^1$ with coordinate $z$ and function field
		\[
		\M(\CP^1) = \C(z).
		\]
		This is the \emph{rational function field} in one variable.
		
		\item \textbf{Torus}: $X=\C/\Lambda$ with Weierstrass functions
		\[
		x := \wp(z),\quad y := \wp'(z),
		\]
		and function field
		\[
		\M(\C/\Lambda) = \C(\wp,\wp')
		= \C(x,y)/(y^2 - 4x^3 + g_2 x + g_3),
		\]
		i.e.\ a quadratic extension of the rational function field $\C(x)$.
	\end{itemize}
	
	Thus the Weierstrass function $\wp$ plays the role of a ``base coordinate''
	(like $z$ on $\CP^1$), and the full function field is an \emph{extension} of
	$\C(\wp)$ obtained by adjoining the algebraic function $\wp'$.
	
	This is exactly the sense in which one can think of the torus case as an
	\emph{extended} version of the sphere case $\M(\CP^1)\cong\C(z)$:
	
	\[
	\boxed{
		\begin{aligned}
			&\text{Genus }0: && \M(\CP^1) = \C(z),\\[4pt]
			&\text{Genus }1: && \M(\C/\Lambda) = \C(\wp,\wp'),
			\quad (\wp')^2 = 4\wp^3 - g_2\wp - g_3.
		\end{aligned}
	}
	\]
	
\end{document}
