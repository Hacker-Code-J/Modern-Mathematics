\documentclass[11pt]{article}
\usepackage{amsmath,amssymb,amsthm,mathtools}
\usepackage[margin=1in]{geometry}

\newcommand{\Z}{\mathbb{Z}}
\newcommand{\R}{\mathbb{R}}
\newcommand{\C}{\mathbb{C}}
\newcommand{\K}{\mathbb{K}}
\newcommand{\dd}{\mathrm{d}}
\newcommand{\Om}{\Omega}
\newcommand{\coker}{\operatorname{coker}}
\newcommand{\im}{\operatorname{im}}
%\newcommand{\ker}{\operatorname{ker}}
\newcommand{\id}{\mathrm{id}}

\theoremstyle{definition}
\newtheorem{definition}{Definition}
\newtheorem{remark}{Remark}
\theoremstyle{plain}
\newtheorem{theorem}{Theorem}
\newtheorem{lemma}{Lemma}
\newtheorem{proposition}{Proposition}
\newtheorem{corollary}{Corollary}

\begin{document}
	
	\section*{Mayer--Vietoris as ``gluing'' in grad--curl--div language (with explicit connecting map)}
	
	\subsection*{1. The de Rham complex and the vector-calculus complex}
	
	Let $M$ be a smooth manifold. The de Rham complex is
	\[
	0 \longrightarrow \Omega^0(M)
	\xrightarrow{\ \dd\ }
	\Omega^1(M)
	\xrightarrow{\ \dd\ }
	\Omega^2(M)
	\xrightarrow{\ \dd\ }
	\Omega^3(M)
	\longrightarrow \cdots
	\]
	and its cohomology is
	\[
	H^k_{\mathrm{dR}}(M) \;=\; \frac{\ker\bigl(\dd:\Omega^k(M)\to\Omega^{k+1}(M)\bigr)}
	{\im\bigl(\dd:\Omega^{k-1}(M)\to\Omega^k(M)\bigr)}.
	\]
	
	\medskip
	On an oriented Riemannian $3$-manifold (in particular on domains in $\R^3$), using the metric and Hodge star and the standard identifications
	\[
	\Omega^0 \leftrightarrow \{\text{scalar fields}\},\quad
	\Omega^1 \leftrightarrow \{\text{vector fields}\},\quad
	\Omega^2 \leftrightarrow \{\text{vector fields}\},\quad
	\Omega^3 \leftrightarrow \{\text{scalar densities}\},
	\]
	the operator $\dd$ corresponds (up to conventional sign/identification choices) to the familiar grad--curl--div chain
	\[
	\Omega^0 \xrightarrow{\dd\;\sim\;\nabla}
	\Omega^1 \xrightarrow{\dd\;\sim\;\nabla\times}
	\Omega^2 \xrightarrow{\dd\;\sim\;\nabla\cdot}
	\Omega^3,
	\]
	so $\dd^2=0$ corresponds to
	\[
	\nabla\times(\nabla f)=0,\qquad \nabla\cdot(\nabla\times A)=0.
	\]
	
	\subsection*{2. The gluing short exact sequence of complexes}
	
	Let $M=U\cup V$ be an open cover. For each $k\ge 0$ define maps
	\[
	r:\Omega^k(M)\to\Omega^k(U)\oplus\Omega^k(V),\qquad
	r(\omega)=(\omega|_U,\omega|_V),
	\]
	and
	\[
	\delta:\Omega^k(U)\oplus\Omega^k(V)\to\Omega^k(U\cap V),\qquad
	\delta(\alpha,\beta)=\alpha|_{U\cap V}-\beta|_{U\cap V}.
	\]
	
	\begin{proposition}[Gluing exact sequence at the cochain level]
		For each $k$ there is a short exact sequence of vector spaces
		\[
		0 \longrightarrow \Omega^k(M)
		\xrightarrow{\ r\ }
		\Omega^k(U)\oplus\Omega^k(V)
		\xrightarrow{\ \delta\ }
		\Omega^k(U\cap V)
		\longrightarrow 0,
		\]
		and these assemble into a short exact sequence of cochain complexes
		\[
		0 \to \Omega^\bullet(M)\xrightarrow{r}\Omega^\bullet(U)\oplus\Omega^\bullet(V)\xrightarrow{\delta}\Omega^\bullet(U\cap V)\to 0
		\]
		because $\dd$ commutes with restriction and hence with $r,\delta$.
	\end{proposition}
	
	\begin{remark}[Calculus interpretation: ``agree on the overlap'']
		A pair of local fields $(\alpha,\beta)\in\Omega^k(U)\oplus\Omega^k(V)$ lies in $\ker(\delta)$
		iff $\alpha$ and $\beta$ \emph{agree on} $U\cap V$. Equivalently, they \emph{glue} to a global
		$k$-form on $M$. Thus the short exact sequence above is the formal encoding of the basic
		principle: global objects are exactly compatible local objects on an open cover.
	\end{remark}
	
	\subsection*{3. Mayer--Vietoris long exact sequence}
	
	\begin{theorem}[Mayer--Vietoris for de Rham cohomology]
		The short exact sequence of cochain complexes in \S2 induces a long exact sequence in cohomology:
		\[
		\cdots \to H^{k-1}_{\mathrm{dR}}(U\cap V)
		\xrightarrow{\ \partial\ }
		H^{k}_{\mathrm{dR}}(M)
		\xrightarrow{\ (r_U^\ast,r_V^\ast)\ }
		H^{k}_{\mathrm{dR}}(U)\oplus H^{k}_{\mathrm{dR}}(V)
		\xrightarrow{\ \delta^\ast\ }
		H^{k}_{\mathrm{dR}}(U\cap V)
		\xrightarrow{\ \partial\ }
		H^{k+1}_{\mathrm{dR}}(M)\to\cdots
		\]
		where $\delta^\ast([\alpha],[\beta])=[\alpha|_{U\cap V}-\beta|_{U\cap V}]$.
	\end{theorem}
	
	\subsection*{4. Explicit formula for the connecting morphism $\partial$ (the ``overlap mismatch'' map)}
	
	Fix a smooth partition of unity subordinate to the cover: choose $\rho_U,\rho_V\in C^\infty(M)$ such that
	\[
	\rho_U+\rho_V=1,\qquad \operatorname{supp}(\rho_U)\subset U,\qquad \operatorname{supp}(\rho_V)\subset V.
	\]
	
	\begin{proposition}[Concrete representative of $\partial$]
		Let $[\eta]\in H^{k-1}_{\mathrm{dR}}(U\cap V)$ with $\dd\eta=0$ on $U\cap V$.
		Extend $\eta$ to forms $\tilde\eta_U\in\Omega^{k-1}(U)$ and $\tilde\eta_V\in\Omega^{k-1}(V)$
		satisfying $\tilde\eta_U|_{U\cap V}=\eta=\tilde\eta_V|_{U\cap V}$ (existence holds by locality of forms).
		Define
		\[
		\omega_U \coloneqq \rho_V\cdot \tilde\eta_U\in\Omega^{k-1}(U),\qquad
		\omega_V \coloneqq -\rho_U\cdot \tilde\eta_V\in\Omega^{k-1}(V).
		\]
		Then on $U\cap V$ one has $\omega_U-\omega_V = \eta$ and hence $\delta(\omega_U,\omega_V)=\eta$.
		Moreover, the pair $(\dd\omega_U,\dd\omega_V)$ agrees on $U\cap V$, so it glues to a global closed $k$-form
		$\theta\in\Omega^k(M)$. The connecting map is
		\[
		\partial([\eta]) \;=\; [\theta]\in H^k_{\mathrm{dR}}(M),
		\]
		and this class is independent of all choices (extensions, partition of unity).
	\end{proposition}
	
	\begin{remark}[What $\partial$ \emph{means}]
		The element $[\eta]\in H^{k-1}(U\cap V)$ can be viewed as a ``transition datum'' on the overlap.
		The connecting map $\partial$ converts this overlap datum into a \emph{global obstruction class} on $M$:
		it is precisely the obstruction to gluing primitives/potentials globally.
	\end{remark}
	
	\subsection*{5. Translation to grad--curl--div: the case $k=1$ (curl-free vs global gradient)}
	
	Assume we are in the $3$-dimensional vector-calculus setting.
	
	\medskip
	Take $k=1$. Then $H^1_{\mathrm{dR}}$ detects the failure of a curl-free field to be a global gradient.
	
	\begin{itemize}
		\item A class in $H^0_{\mathrm{dR}}(U\cap V)$ is represented by a locally constant function on $U\cap V$
		(when $U\cap V$ is disconnected, this is where ``different constants on different components'' appears).
		
		\item The connecting map
		\[
		\partial: H^0_{\mathrm{dR}}(U\cap V)\longrightarrow H^1_{\mathrm{dR}}(M)
		\]
		takes an overlap ``constant mismatch'' and outputs a global cohomology class of closed $1$-forms,
		i.e.\ (under the identifications) a curl-free vector field modulo global gradients.
	\end{itemize}
	
	Concretely, let $c$ be a locally constant function on $U\cap V$. Choose extensions $\tilde c_U\in C^\infty(U)$
	and $\tilde c_V\in C^\infty(V)$ with $\tilde c_U|_{U\cap V}=c=\tilde c_V|_{U\cap V}$. Then
	\[
	\theta \;=\; \dd(\rho_V \tilde c_U) \;=\; -\dd(\rho_U \tilde c_V)
	\quad\text{on }U\cap V
	\]
	glues to a global closed $1$-form on $M$, and $\partial([c])=[\theta]$.
	In vector-calculus language, $\theta$ corresponds to a curl-free field $F$ (since $\dd\theta=0$ means $\nabla\times F=0$)
	whose failure to be a global gradient is encoded by the ``jump'' data $c$ on $U\cap V$.
	
	\subsection*{6. Worked example: $H^1(S^1)\cong \R$ from overlap-gluing (pure calculus intuition)}
	
	Let $M=S^1$. Cover $S^1$ by two open arcs $U,V$ such that $U\cap V$ is a disjoint union of two open arcs,
	hence has two connected components:
	\[
	U\cap V \;=\; W_1 \sqcup W_2.
	\]
	Then
	\[
	H^0_{\mathrm{dR}}(U)\cong \R,\quad H^0_{\mathrm{dR}}(V)\cong \R,\quad
	H^0_{\mathrm{dR}}(U\cap V)\cong \R\oplus\R,
	\]
	and (since $U,V,U\cap V$ are unions of contractible sets) one has
	\[
	H^1_{\mathrm{dR}}(U)=H^1_{\mathrm{dR}}(V)=H^1_{\mathrm{dR}}(U\cap V)=0.
	\]
	The Mayer--Vietoris segment for degrees $0$ and $1$ becomes
	\[
	0 \to H^0(S^1)
	\to H^0(U)\oplus H^0(V)
	\xrightarrow{\ \delta^\ast\ }
	H^0(U\cap V)
	\xrightarrow{\ \partial\ }
	H^1(S^1)
	\to 0.
	\]
	Identify $H^0(U)\oplus H^0(V)\cong \R^2$ and $H^0(U\cap V)\cong \R^2$ by sending a class to its constant value
	on each connected component. Then
	\[
	\delta^\ast(a,b) \;=\; (a-b,\; a-b),
	\]
	so $\im(\delta^\ast)=\{(t,t):t\in\R\}$ is the diagonal in $\R^2$. Hence
	\[
	H^1_{\mathrm{dR}}(S^1)\;\cong\; \frac{\R^2}{\{(t,t)\}} \;\cong\; \R,
	\]
	and the isomorphism is concretely given by the \emph{difference of overlap constants}:
	\[
	[(c_1,c_2)] \longmapsto c_1-c_2.
	\]
	
	\medskip
	\noindent
	\textbf{Calculus meaning.}
	Locally on $U$ and $V$, a curl-free $1$-field is a gradient of a potential (angle function).
	On $U\cap V$, the two local potentials differ by constants; because $U\cap V$ has \emph{two} components,
	there can be \emph{two} constants. If those constants disagree between the components, you cannot choose
	potentials that match globally; this is exactly the nontrivial class in $H^1(S^1)$, i.e.\ the global obstruction.
	
	\newpage
	\section*{Extremely detailed computations of $H^1_{\mathrm{dR}}(S^1)$ and $H^1_{\mathrm{dR}}(S^2)$}
	
	\subsection*{0. Preliminaries: de Rham cohomology and Mayer--Vietoris}
	
	\begin{definition}[de Rham cohomology]
		For a smooth manifold $M$, define
		\[
		H^k_{\mathrm{dR}}(M)\;=\;\frac{\ker(\dd:\Om^k(M)\to\Om^{k+1}(M))}
		{\im(\dd:\Om^{k-1}(M)\to\Om^k(M))}.
		\]
		Elements of $\ker(\dd)$ are called \emph{closed} $k$-forms; elements of $\im(\dd)$ are \emph{exact}.
	\end{definition}
	
	\begin{proposition}[Gluing exact sequence for an open cover]
		If $M=U\cup V$ is an open cover, then for each $k$ there is a short exact sequence
		\[
		0\to \Om^k(M)\xrightarrow{r}\Om^k(U)\oplus\Om^k(V)\xrightarrow{\delta}\Om^k(U\cap V)\to 0
		\]
		where $r(\omega)=(\omega|_U,\omega|_V)$ and $\delta(\alpha,\beta)=\alpha|_{U\cap V}-\beta|_{U\cap V}$.
		Moreover, this is a short exact sequence of \emph{cochain complexes} because $\dd$ commutes with restriction.
	\end{proposition}
	
	\begin{theorem}[Mayer--Vietoris long exact sequence in de Rham cohomology]
		From the short exact sequence of complexes above, one obtains a long exact sequence
		\[
		\cdots\to H^{k-1}(U\cap V)\xrightarrow{\partial}H^k(M)\xrightarrow{(r_U^\ast,r_V^\ast)}
		H^k(U)\oplus H^k(V)\xrightarrow{\delta^\ast}H^k(U\cap V)\xrightarrow{\partial}H^{k+1}(M)\to\cdots
		\]
		where $\delta^\ast([\alpha],[\beta])=[\alpha|_{U\cap V}-\beta|_{U\cap V}]$.
	\end{theorem}
	
	\begin{remark}[Concrete formula for $\partial$ via partition of unity]
		Fix $\rho_U,\rho_V\in C^\infty(M)$ with $\rho_U+\rho_V=1$ and $\mathrm{supp}(\rho_U)\subset U$,
		$\mathrm{supp}(\rho_V)\subset V$. If $[\eta]\in H^{k-1}(U\cap V)$ is represented by a closed form $\eta$,
		choose extensions $\tilde\eta_U\in\Om^{k-1}(U)$ and $\tilde\eta_V\in\Om^{k-1}(V)$ with
		$\tilde\eta_U|_{U\cap V}=\eta=\tilde\eta_V|_{U\cap V}$. Define
		\[
		\omega_U:=\rho_V\,\tilde\eta_U,\qquad \omega_V:=-\rho_U\,\tilde\eta_V.
		\]
		Then $\delta(\omega_U,\omega_V)=\eta$, and $(\dd\omega_U,\dd\omega_V)$ agrees on $U\cap V$, hence glues
		to a global closed $k$-form $\theta$ on $M$. One sets $\partial([\eta]):=[\theta]\in H^k(M)$.
	\end{remark}
	
	\subsection*{1. $H^1_{\mathrm{dR}}(S^1)$: two complementary computations}
	
	\subsubsection*{1A. Direct calculation using the angle coordinate}
	
	View $S^1\subset\R^2$ with standard angle coordinate $\theta\in\R/2\pi\Z$ and parametrization
	\[
	\gamma:\R/2\pi\Z\to S^1,\qquad \gamma(\theta)=(\cos\theta,\sin\theta).
	\]
	Then $\Om^1(S^1)$ is a rank-$1$ $C^\infty(S^1)$-module generated by $\dd\theta$ in the sense that any $1$-form
	$\omega\in\Om^1(S^1)$ can be uniquely written as
	\[
	\omega = f(\theta)\,\dd\theta,\qquad f\in C^\infty(S^1).
	\]
	
	\begin{lemma}[All $1$-forms on $S^1$ are closed]
		If $\omega\in\Om^1(S^1)$, then $\dd\omega=0$.
	\end{lemma}
	\begin{proof}
		On any $1$-manifold, $\Om^2=0$ identically (there are no nonzero $2$-forms), so $\dd:\Om^1\to\Om^2$ is the
		zero map. Hence every $1$-form is closed.
	\end{proof}
	
	Thus
	\[
	H^1_{\mathrm{dR}}(S^1)=\frac{\Om^1(S^1)}{\dd \Om^0(S^1)}.
	\]
	So we must characterize which $f(\theta)\dd\theta$ are exact.
	
	\begin{lemma}[Exactness criterion by period]
		Let $\omega=f(\theta)\dd\theta\in\Om^1(S^1)$. Then $\omega$ is exact iff
		\[
		\int_{S^1}\omega \;=\; \int_0^{2\pi} f(\theta)\, \dd\theta \;=\; 0.
		\]
	\end{lemma}
	\begin{proof}
		($\Rightarrow$) If $\omega=\dd g$ for some $g\in C^\infty(S^1)$, then by the fundamental theorem of calculus,
		\[
		\int_{S^1}\omega=\int_{S^1}\dd g = 0,
		\]
		because the integral of an exact $1$-form over a closed loop is zero.
		
		($\Leftarrow$) Assume $\int_0^{2\pi} f(\theta)\,\dd\theta=0$. Define
		\[
		G(\theta):=\int_0^\theta f(t)\,\dd t.
		\]
		Then $G$ is smooth on $\R$, and $G'(\theta)=f(\theta)$. Moreover,
		\[
		G(\theta+2\pi)-G(\theta)=\int_\theta^{\theta+2\pi} f(t)\,\dd t = \int_0^{2\pi} f(t)\,\dd t=0,
		\]
		so $G$ is $2\pi$-periodic and thus descends to a smooth function $g\in C^\infty(S^1)$ satisfying
		$\dd g = f(\theta)\dd\theta=\omega$.
	\end{proof}
	
	\begin{proposition}[The period map identifies $H^1_{\mathrm{dR}}(S^1)\cong\R$]
		Define the linear map (``period'' or ``circulation'')
		\[
		\mathcal{P}:\Om^1(S^1)\to\R,\qquad \mathcal{P}(\omega)=\int_{S^1}\omega.
		\]
		Then $\mathcal{P}$ vanishes on exact forms and hence induces a well-defined linear map
		\[
		\overline{\mathcal{P}}:H^1_{\mathrm{dR}}(S^1)\to\R,\qquad \overline{\mathcal{P}}([\omega])=\int_{S^1}\omega.
		\]
		Moreover, $\overline{\mathcal{P}}$ is an isomorphism. In particular,
		\[
		H^1_{\mathrm{dR}}(S^1)\cong\R,\qquad [\dd\theta]\ \text{is a generator with }\int_{S^1}\dd\theta=2\pi.
		\]
	\end{proposition}
	\begin{proof}
		Well-definedness is immediate because $\int_{S^1}\dd g=0$ for all $g$.
		
		Injectivity: If $\overline{\mathcal{P}}([\omega])=0$, then $\int_{S^1}\omega=0$, so by the exactness criterion
		$\omega$ is exact, hence $[\omega]=0$ in cohomology.
		
		Surjectivity: Given $c\in\R$, take $\omega=\frac{c}{2\pi}\dd\theta$. Then
		\[
		\overline{\mathcal{P}}([\omega])=\int_{S^1}\frac{c}{2\pi}\dd\theta=\frac{c}{2\pi}\cdot 2\pi=c.
		\]
	\end{proof}
	
	\begin{remark}[Vector-calculus reading in 1D]
		On a 1-dimensional manifold, ``curl'' is vacuous and every $1$-form is locally a gradient
		(Poincar\'e lemma in degree $1$). The obstruction to a \emph{global} potential is exactly the nonzero
		circulation $\int_{S^1}\omega$, which is the $H^1$ class.
	\end{remark}
	
	\subsubsection*{1B. Mayer--Vietoris computation for $S^1$ (extremely explicit)}
	
	Choose an open cover $S^1=U\cup V$ where $U,V$ are open arcs, each diffeomorphic to an interval,
	such that $U\cap V$ is the disjoint union of \emph{two} open arcs:
	\[
	U\cap V = W_1 \sqcup W_2,\qquad W_1\cap W_2=\varnothing.
	\]
	(Geometrically: take $U$ and $V$ as two large overlapping arcs whose overlap occurs in two separated regions.)
	
	\begin{lemma}[Cohomology of the pieces]
		Because $U,V,W_1,W_2$ are each diffeomorphic to an open interval, they are contractible. Hence:
		\[
		H^0(U)\cong\R,\quad H^0(V)\cong\R,\quad H^0(W_i)\cong\R,\qquad
		H^1(U)=H^1(V)=H^1(W_i)=0.
		\]
		Therefore
		\[
		H^0(U\cap V)\cong H^0(W_1)\oplus H^0(W_2)\cong \R\oplus\R,\qquad
		H^1(U\cap V)=0.
		\]
	\end{lemma}
	
	Now write out the Mayer--Vietoris long exact sequence in low degrees.
	The relevant segment is
	\[
	0 \to H^0(S^1)\xrightarrow{r^\ast} H^0(U)\oplus H^0(V)
	\xrightarrow{\delta^\ast} H^0(U\cap V)\xrightarrow{\partial} H^1(S^1)
	\xrightarrow{r^\ast} H^1(U)\oplus H^1(V)=0.
	\]
	Exactness at the last term implies $\partial$ is \emph{surjective}:
	\[
	\im(\partial)=H^1(S^1).
	\]
	Exactness at $H^0(U\cap V)$ gives
	\[
	\ker(\partial)=\im(\delta^\ast).
	\]
	Therefore
	\[
	H^1(S^1)\;\cong\;\frac{H^0(U\cap V)}{\im(\delta^\ast)}.
	\]
	
	It remains to compute $\delta^\ast$ explicitly on constants.
	
	\begin{lemma}[Explicit form of $\delta^\ast$ on $H^0$]
		Identify
		\[
		H^0(U)\oplus H^0(V)\cong \R\oplus\R
		\]
		by sending a class to its constant value on each connected set.
		Similarly, identify
		\[
		H^0(U\cap V)\cong \R\oplus\R
		\]
		by sending a class to its pair of constants on $(W_1,W_2)$.
		Then
		\[
		\delta^\ast(a,b) = (a-b,\ a-b).
		\]
	\end{lemma}
	\begin{proof}
		A class in $H^0(U)$ is represented by a locally constant function; since $U$ is connected, it is constant $a$.
		Similarly $b$ on $V$. On each overlap component $W_i\subset U\cap V$, the difference
		\[
		a|_{W_i}-b|_{W_i}=a-b
		\]
		is the constant value of the class in $H^0(W_i)$. Hence the pair is $(a-b,a-b)$.
	\end{proof}
	
	Thus
	\[
	\im(\delta^\ast)=\{(t,t):t\in\R\}\subset \R^2,
	\]
	the diagonal. Consequently
	\[
	H^1(S^1)\cong \frac{\R^2}{\{(t,t)\}}\cong \R,
	\]
	and a concrete isomorphism is given by the \emph{difference of overlap constants}:
	\[
	[(c_1,c_2)] \longmapsto c_1-c_2.
	\]
	This exhibits the slogan:
	\begin{quote}
		local potentials exist on $U$ and $V$, but their constants of integration on $W_1$ and $W_2$
		need not match; the mismatch $c_1-c_2$ is the $H^1$ obstruction.
	\end{quote}
	
	\subsection*{2. $H^1_{\mathrm{dR}}(S^2)=0$ (with Mayer--Vietoris and explicit maps)}
	
	Let $S^2\subset \R^3$ be the unit sphere. Let $N=(0,0,1)$ (north pole) and $S=(0,0,-1)$ (south pole).
	Define the standard two-chart cover
	\[
	U := S^2\setminus\{S\},\qquad V := S^2\setminus\{N\}.
	\]
	Then $U$ and $V$ are each diffeomorphic to $\R^2$ via stereographic projection. Their intersection is
	\[
	U\cap V = S^2\setminus\{N,S\}.
	\]
	
	\subsubsection*{2A. Cohomology of $U$, $V$, and $U\cap V$}
	
	\begin{lemma}[Cohomology of $U$ and $V$]
		Since $U\cong\R^2$ and $V\cong\R^2$ are contractible,
		\[
		H^0(U)\cong\R,\quad H^0(V)\cong\R,\qquad H^1(U)=H^1(V)=0.
		\]
	\end{lemma}
	
	\begin{lemma}[Homotopy type and $H^1$ of the overlap]
		The manifold $U\cap V=S^2\setminus\{N,S\}$ deformation retracts onto the equator $S^1\subset S^2$.
		Hence
		\[
		H^0(U\cap V)\cong\R,\qquad H^1(U\cap V)\cong H^1(S^1)\cong\R.
		\]
	\end{lemma}
	\begin{proof}
		Geometrically, remove the poles; every remaining point lies on a unique meridian segment crossing the equator.
		Define a deformation retraction by sliding points along meridians to the equator (keeping longitude fixed).
		Thus $U\cap V\simeq S^1$, and de Rham cohomology is homotopy invariant, giving the claims.
	\end{proof}
	
	\subsubsection*{2B. Mayer--Vietoris in degrees $0$ and $1$}
	
	Write the Mayer--Vietoris long exact sequence in low degrees:
	\[
	0 \to H^0(S^2)\xrightarrow{r^\ast} H^0(U)\oplus H^0(V)
	\xrightarrow{\delta^\ast} H^0(U\cap V)
	\xrightarrow{\partial} H^1(S^2)
	\xrightarrow{r^\ast} H^1(U)\oplus H^1(V).
	\]
	Substitute the computed groups:
	\[
	0 \to \R \xrightarrow{r^\ast} \R\oplus\R
	\xrightarrow{\delta^\ast} \R
	\xrightarrow{\partial} H^1(S^2)
	\xrightarrow{r^\ast} 0\oplus 0 = 0.
	\]
	Thus $\partial$ is surjective and
	\[
	H^1(S^2)\cong \frac{H^0(U\cap V)}{\im(\delta^\ast)} \cong \frac{\R}{\im(\delta^\ast)}.
	\]
	Therefore it suffices to show $\delta^\ast:\R^2\to\R$ is surjective.
	
	\begin{lemma}[Explicit computation of $\delta^\ast$ on $H^0$ for $S^2$ cover]
		Under the identifications $H^0(U)\cong\R$, $H^0(V)\cong\R$, $H^0(U\cap V)\cong\R$ via constants,
		\[
		\delta^\ast(a,b)=a-b.
		\]
		In particular, $\delta^\ast$ is surjective.
	\end{lemma}
	\begin{proof}
		As in the $S^1$ case, a class in $H^0(U)$ is represented by a constant $a$ on connected $U$, similarly $b$ on $V$.
		Since $U\cap V$ is connected, the difference $a|_{U\cap V}-b|_{U\cap V}$ is the constant $a-b$ in $H^0(U\cap V)$.
		Surjectivity: given $c\in\R$, choose $(a,b)=(c,0)$ so $a-b=c$.
	\end{proof}
	
	\begin{corollary}[$H^1_{\mathrm{dR}}(S^2)=0$]
		Because $\delta^\ast$ is surjective, $\im(\delta^\ast)=\R$, hence
		\[
		H^1_{\mathrm{dR}}(S^2)\cong \R/\R = 0.
		\]
	\end{corollary}
	
	\begin{remark}[Calculus reading: every curl-free tangent field has a global potential on $S^2$]
		Under the $1$-form/vector-field identification (with a metric), a closed $1$-form corresponds to a curl-free field.
		The statement $H^1(S^2)=0$ says:
		\[
		\text{every closed $1$-form on $S^2$ is exact, i.e.\ every curl-free field is a global gradient.}
		\]
		From the Mayer--Vietoris perspective: local potentials exist on $U$ and $V$ (since $H^1(U)=H^1(V)=0$),
		and because the overlap $U\cap V$ is connected, the only ambiguity is an \emph{additive constant}, which can be
		adjusted to make the two potentials agree globally. There is no ``two-component mismatch'' as in the $S^1$ case.
	\end{remark}
	
	\subsubsection*{2C. (Optional but illuminating) What happens one degree higher: $H^2(S^2)\cong \R$}
	
	Although you asked for $H^1$, it is instructive to record the adjacent MV segment:
	\[
	H^1(U)\oplus H^1(V)=0 \longrightarrow H^1(U\cap V)\cong \R
	\xrightarrow{\partial} H^2(S^2) \longrightarrow H^2(U)\oplus H^2(V)=0.
	\]
	Exactness forces $\partial:\R\to H^2(S^2)$ to be an isomorphism, so $H^2(S^2)\cong\R$.
	This is the formal place where the ``flux/area'' class of the sphere lives.
	
	
\end{document}
