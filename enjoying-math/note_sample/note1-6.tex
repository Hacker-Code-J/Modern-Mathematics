\documentclass[11pt]{article}
\usepackage{amsmath,amssymb,amsthm}
\newcommand{\C}{\mathbb{C}}
\newcommand{\CP}{\mathbb{CP}}
\newcommand{\M}{\mathcal{M}}
\newcommand{\Res}{\operatorname{Res}}
\newcommand{\Divv}{\operatorname{Div}}

\begin{document}
	
	\title{Concrete, calculus-style examples of \(\M(X)\) for \(X=\CP^1\) and \(X=\C/\Lambda\)}
	\author{}
	\date{}
	\maketitle
	
	%%%%%%%%%%%%%%%%%%%%%%%%%%%%%%%%%%%%%%%%%%%%%%%%%%%%%%%%%%%%
	\section{Case 1: \(X=\CP^1\) (Riemann sphere)}
	
	\subsection{Setup: coordinate and function field}
	
	View \(\CP^1\) as the Riemann sphere:
	\[
	\CP^1 \cong \C \cup \{\infty\}.
	\]
	Use the standard affine coordinate
	\[
	z = \frac{z_0}{z_1}
	\]
	on the chart \(U_1 = \{[z_0:z_1]\mid z_1\neq 0\}\cong \C\).
	The point at infinity is
	\[
	\infty = [1:0].
	\]
	
	A meromorphic function on \(\CP^1\) is the same as a rational function in \(z\), i.e.
	\[
	\M(\CP^1) \cong \C(z).
	\]
	
	We will \emph{explicitly} analyze one such function and its differential.
	
	\subsection{A concrete meromorphic function \(f\) on \(\CP^1\)}
	
	Let
	\[
	f(z) = \frac{(z-1)^2}{z(z-2)}.
	\]
	
	\subsubsection*{Step 1: Zeros and poles on \(\C\)}
	
	\begin{itemize}
		\item Zeros (where numerator vanishes):
		\[
		(z-1)^2 = 0 \quad\Rightarrow\quad z=1\quad\text{(double zero)}.
		\]
		So \(\operatorname{ord}_{z=1}(f) = +2\).
		
		\item Poles (where denominator vanishes):
		\[
		z(z-2)=0 \quad\Rightarrow\quad z=0,\ z=2.
		\]
		We check they are simple poles:
		
		Near \(z=0\): write \(z\) as local coordinate. Then
		\[
		f(z) = \frac{(z-1)^2}{z(z-2)}
		= \frac{z^2 - 2z + 1}{z(z-2)}
		\sim \frac{1}{z(-2)} = -\frac{1}{2z}
		\quad\text{as }z\to 0.
		\]
		So
		\[
		\operatorname{ord}_{z=0}(f) = -1.
		\]
		
		Near \(z=2\): set \(u = z-2\), so \(z = 2+u\). Then
		\[
		f(z) = f(2+u)
		= \frac{(2+u-1)^2}{(2+u)(2+u-2)}
		= \frac{(1+u)^2}{(2+u)u}
		\sim \frac{1}{2u}
		\quad\text{as }u\to 0.
		\]
		So
		\[
		\operatorname{ord}_{z=2}(f) = -1.
		\]
	\end{itemize}
	
	So on the finite plane:
	\[
	\operatorname{ord}_1(f) = +2,\quad
	\operatorname{ord}_0(f) = -1,\quad
	\operatorname{ord}_2(f) = -1.
	\]
	
	\subsubsection*{Step 2: Behavior at infinity via differential-calculus change of variable}
	
	To study \(z=\infty\), use \(w = 1/z\) as local coordinate near \(\infty\).
	
	Then \(z = 1/w\), and
	\[
	f(z) = f\Bigl(\frac{1}{w}\Bigr).
	\]
	
	Compute explicitly:
	\[
	f\Bigl(\frac{1}{w}\Bigr)
	= \frac{(1/w - 1)^2}{(1/w)(1/w - 2)}
	= \frac{(1-w)^2}{\frac{1}{w^2} (1 - 2w)} 
	= (1-w)^2 \cdot \frac{w^2}{1-2w}.
	\]
	Now expand near \(w=0\). First,
	\[
	(1-w)^2 = 1 - 2w + w^2,
	\]
	and
	\[
	\frac{1}{1-2w} = 1 + 2w + 4w^2 + O(w^3).
	\]
	So
	\[
	f\Bigl(\frac{1}{w}\Bigr)
	= w^2\,(1 - 2w + w^2)\,(1 + 2w + 4w^2 + O(w^3)).
	\]
	
	Multiply out just enough terms to see the leading behavior:
	\[
	(1 - 2w + w^2)(1 + 2w + 4w^2)
	= 1 + (2w - 2w) + (\cdots) = 1 + O(w^2),
	\]
	so overall
	\[
	f\Bigl(\frac{1}{w}\Bigr)
	= w^2\,(1 + O(w^2)) = w^2 + O(w^4).
	\]
	
	Thus near \(w=0\), \(f(1/w)\) has a \emph{zero of order 2} as a function of \(w\). But remember: the order at \(\infty\) as a point of \(\CP^1\) is the order of this function viewed on the sphere. More directly: in terms of the original coordinate \(z\), we see that as \(|z|\to\infty\),
	\[
	f(z) = 1 + O\Bigl(\frac{1}{z^2}\Bigr),
	\]
	so \(f\) is holomorphic and nonzero at \(\infty\). Hence
	\[
	\operatorname{ord}_{\infty}(f) = 0.
	\]
	
	\subsubsection*{Step 3: The divisor of \(f\)}
	
	The divisor of \(f\) on \(\CP^1\) is
	\[
	\Divv(f) = 2\cdot(1) - (0) - (2) + 0\cdot(\infty).
	\]
	Sum of coefficients:
	\[
	2 - 1 - 1 + 0 = 0,
	\]
	which matches the general fact that the sum of orders of a meromorphic
	function on a compact Riemann surface is zero.
	
	\subsection{Differential: computing \(df = f'(z)\,dz\) in detail}
	
	Now do calculus: compute the derivative.
	
	\subsubsection*{Step 1: Simplify \(f\) via partial fractions}
	
	We try to write
	\[
	f(z) = A + \frac{B}{z} + \frac{C}{z-2}.
	\]
	
	Compute directly:
	\[
	\frac{(z-1)^2}{z(z-2)}
	= \frac{z^2 - 2z + 1}{z(z-2)}.
	\]
	We look for \(A,B,C\) such that
	\[
	\frac{z^2 - 2z + 1}{z(z-2)}
	= A + \frac{B}{z} + \frac{C}{z-2}.
	\]
	
	Multiply both sides by \(z(z-2)\):
	\[
	z^2 - 2z + 1 = A z(z-2) + B(z-2) + Cz.
	\]
	Expand the right-hand side:
	\[
	A z(z-2) = A(z^2 - 2z) = A z^2 - 2A z,
	\]
	\[
	B(z-2) = Bz - 2B,
	\]
	\[
	Cz = Cz.
	\]
	So
	\[
	A z^2 - 2A z + Bz - 2B + Cz
	= A z^2 + (-2A + B + C)z - 2B.
	\]
	Equate coefficients with the left side \(z^2 - 2z + 1\):
	\[
	\begin{cases}
		A = 1,\\
		-2A + B + C = -2,\\
		-2B = 1.
	\end{cases}
	\]
	From \(-2B=1\), we get \(B = -\frac{1}{2}\). Then
	\[
	-2A + B + C = -2\quad\Rightarrow\quad
	-2(1) - \frac{1}{2} + C = -2
	\quad\Rightarrow\quad
	-2.5 + C = -2
	\quad\Rightarrow\quad
	C = \frac{1}{2}.
	\]
	So
	\[
	f(z) = 1 - \frac{1}{2z} + \frac{1}{2(z-2)}.
	\]
	
	\subsubsection*{Step 2: Differentiate term by term}
	
	Now
	\[
	f(z) = 1 - \frac{1}{2z} + \frac{1}{2(z-2)}.
	\]
	Differentiate:
	\[
	f'(z) = 0 + \frac{1}{2z^2} - \frac{1}{2(z-2)^2}.
	\]
	So the meromorphic 1-form
	\[
	df = f'(z)\,dz = \left( \frac{1}{2z^2} - \frac{1}{2(z-2)^2} \right)\,dz.
	\]
	
	\paragraph{Poles of \(df\).}
	Clearly:
	\[
	df \text{ has poles of order 2 at } z=0 \text{ and } z=2.
	\]
	No simple pole terms appear in the Laurent expansions; hence all residues are 0.
	
	\paragraph{Check residues explicitly via Laurent series.}
	
	Near \(z=0\):
	\[
	\frac{1}{2z^2} - \frac{1}{2(z-2)^2}
	= \frac{1}{2z^2} - \frac{1}{2(4 - 4z + z^2)}
	= \frac{1}{2z^2} - \frac{1}{8}\cdot\frac{1}{1 - z + z^2/4}.
	\]
	Expand \(\frac{1}{1 - z + z^2/4}\) as a power series in \(z\) (no negative powers),
	so near \(z=0\) this second term has no negative-power part. Thus the only
	negative-power part is \(\frac{1}{2z^2}\), which has no \((z-0)^{-1}\) term.
	So \(\Res_{z=0}(df) = 0\).
	
	Similarly near \(z=2\), set \(u=z-2\). Then
	\[
	df = \left( \frac{1}{2(2+u)^2} - \frac{1}{2u^2} \right)\,d(2+u)
	= \left( \frac{1}{8}\cdot\frac{1}{(1+u/2)^2} - \frac{1}{2u^2} \right)\,du.
	\]
	Again the first term has only nonnegative powers in \(u\), the second has
	\(u^{-2}\) but no \(u^{-1}\). So \(\Res_{z=2}(df)=0\).
	
	At \(\infty\), we could check via \(w=1/z\). A general fact: for any meromorphic function \(f\) on a compact Riemann surface,
	\[
	\sum_p \Res_p(df) = 0.
	\]
	Since the residues at \(0\) and \(2\) are zero, the residue at \(\infty\) is
	also zero.
	
	\subsection{Moral for \(\CP^1\): function field via differential forms}
	
	In general, if \(f\) is \emph{any} meromorphic function on \(\CP^1\), then
	
	\begin{itemize}
		\item \(df\) is a meromorphic 1-form whose residues all vanish.
		\item By analyzing the principal parts and using Liouville, one shows that
		\(f\) must be a rational function in the coordinate \(z\).
	\end{itemize}
	
	Thus the function field is
	\[
	\M(\CP^1) = \C(z).
	\]
	
	Our explicit \(f\) shows concretely how poles/zeros and the differential look,
	and how partial fractions naturally appear from calculus.
	
	%%%%%%%%%%%%%%%%%%%%%%%%%%%%%%%%%%%%%%%%%%%%%%%%%%%%%%%%%%%%
	\section{Case 2: \(X=\C/\Lambda\) (complex torus)}
	
	Now we move to a more subtle example and use differential forms very concretely.
	
	\subsection{The torus and the basic holomorphic 1-form}
	
	Let \(\Lambda\subset\C\) be a lattice:
	\[
	\Lambda = \mathbb{Z}\omega_1 \oplus \mathbb{Z}\omega_2,\quad
	\Im\left(\frac{\omega_2}{\omega_1}\right) > 0.
	\]
	Define the complex torus
	\[
	X = \C/\Lambda.
	\]
	
	The projection \(\pi:\C\to X\) is holomorphic and locally biholomorphic.
	The 1-form \(dz\) on \(\C\) is invariant under translations by \(\Lambda\), so it
	descends to a global holomorphic 1-form on \(X\). In fact,
	\[
	H^0(X,\Omega^1_X) = \C\cdot dz,
	\]
	i.e.\ there is a one-dimensional space of holomorphic 1-forms.
	
	\subsection{The Weierstrass \(\wp\) and its derivative \(\wp'\)}
	
	Define the Weierstrass \(\wp\)-function:
	\[
	\wp(z) = \frac{1}{z^2}
	+ \sum_{\lambda\in\Lambda\setminus\{0\}}
	\left(
	\frac{1}{(z-\lambda)^2} - \frac{1}{\lambda^2}
	\right).
	\]
	Facts:
	\begin{itemize}
		\item \(\wp(z+\lambda) = \wp(z)\) for all \(\lambda\in\Lambda\)
		(\(\Lambda\)-periodic).
		\item \(\wp\) is even: \(\wp(-z)=\wp(z)\).
		\item \(\wp\) is meromorphic on \(\C\) with poles of order 2 at each
		lattice point.
	\end{itemize}
	
	Differentiate term by term (justified by uniform convergence on compact
	sets away from poles):
	\[
	\wp'(z)
	= -\frac{2}{z^3}
	- 2\sum_{\lambda\in\Lambda\setminus\{0\}}
	\frac{1}{(z-\lambda)^3}.
	\]
	Then:
	\begin{itemize}
		\item \(\wp'\) is odd: \(\wp'(-z)=-\wp'(z)\).
		\item \(\wp'(z)\) has poles of order 3 at each lattice point.
	\end{itemize}
	
	Both \(\wp\) and \(\wp'\) are \(\Lambda\)-periodic, so they descend to meromorphic
	functions on \(X=\C/\Lambda\):
	\[
	\wp_X([z]) = \wp(z),\qquad (\wp_X)'([z]) = \wp'(z).
	\]
	
	\subsection{The meromorphic 1-form \(\wp'(z)\,dz = d\wp(z)\)}
	
	Consider the 1-form
	\[
	\omega = \wp'(z)\,dz.
	\]
	This is clearly
	\[
	\omega = d\wp(z),
	\]
	as in calculus: derivative times \(dz\).
	
	\subsubsection*{Local expansion near a pole}
	
	Focus on the point \([0]\in X\) (the image of \(0\in\C\)).
	
	Near \(z=0\),
	\[
	\wp(z) = \frac{1}{z^2} + c_2 z^2 + c_4 z^4 + \cdots
	\]
	for some complex coefficients \(c_k\) (coming from the lattice).
	Then
	\[
	\wp'(z) = -\frac{2}{z^3} + 2c_2 z + 4c_4 z^3 + \cdots.
	\]
	
	So near \(z=0\),
	\[
	\omega = \left(-\frac{2}{z^3} + 2c_2 z + 4c_4 z^3 + \cdots\right)\,dz.
	\]
	
	In Laurent series form:
	\[
	\omega = -2 z^{-3} dz + 2c_2 z\,dz + 4c_4 z^3 dz + \cdots.
	\]
	
	Note there is \emph{no} term of the form \(c_{-1} z^{-1}dz\), so
	\[
	\Res_{z=0}(\omega) = 0.
	\]
	Similarly, at any other lattice point \(\lambda\in\Lambda\), shifting \(z\mapsto z-\lambda\)
	gives the same type of expansion: pole of order 3, no \(1/(z-\lambda)\) term.
	Thus
	\[
	\Res_{[0]}(\omega) = 0,\quad
	\Res_{[\lambda]}(\omega)=0\quad\text{in the quotient }X.
	\]
	
	\subsubsection*{Exactness and periods}
	
	Since \(\omega=d\wp\), it is an exact differential. On the torus, this implies
	for any closed loop \(\gamma\) in \(X\),
	\[
	\oint_{\gamma}\omega = \oint_{\gamma}d\wp = 0.
	\]
	This is the global version of the fact that an exact differential has zero
	integral over closed paths.
	
	\subsection{A very concrete elliptic function built from \(\wp\)}
	
	Fix some \(a\in\C\) with \([a]\neq[0]\) in \(X\), and also \([a]\neq[-a]\) (i.e.\
	\(2a\notin\Lambda\)). Consider the function
	\[
	g(z) = \wp(z) - \wp(a),
	\]
	which is \(\Lambda\)-periodic and meromorphic.
	
	Define:
	\[
	f(z) = \frac{\wp'(z)}{\wp(z) - \wp(a)}.
	\]
	This is a classical elliptic function: it is the derivative of the logarithm of
	\(g(z)\):
	\[
	f(z) = \frac{d}{dz}\log(\wp(z)-\wp(a)).
	\]
	
	\subsubsection*{Poles of \(f(z)\) on the torus \(X\)}
	
	\paragraph{At \(z=a\).}
	Near \(z=a\), write \(\zeta = z-a\). Then expand:
	\[
	\wp(z) = \wp(a) + \wp'(a)\zeta + \frac{1}{2}\wp''(a)\zeta^2 + \cdots.
	\]
	So
	\[
	\wp(z) - \wp(a) = \wp'(a)\zeta + \frac{1}{2}\wp''(a)\zeta^2 + \cdots.
	\]
	Similarly
	\[
	\wp'(z) = \wp'(a) + \wp''(a)\zeta + \cdots.
	\]
	
	Thus
	\[
	f(z) 
	= \frac{\wp'(z)}{\wp(z)-\wp(a)}
	= \frac{\wp'(a) + \wp''(a)\zeta + \cdots}
	{\wp'(a)\zeta + \frac{1}{2}\wp''(a)\zeta^2 + \cdots}.
	\]
	Pull out \(\wp'(a)\zeta\) from the denominator:
	\[
	f(z)
	= \frac{\wp'(a) + \wp''(a)\zeta + \cdots}
	{\wp'(a)\zeta\left(1 + \frac{\wp''(a)}{2\wp'(a)}\zeta + \cdots\right)}.
	\]
	Now expand:
	\[
	\frac{1}{1 + \alpha\zeta + \cdots} = 1 - \alpha\zeta + \cdots,
	\]
	so near \(\zeta=0\),
	\[
	f(z)
	\sim \frac{1}{\zeta} \cdot
	\frac{\wp'(a)}{\wp'(a)} = \frac{1}{\zeta} = \frac{1}{z-a}.
	\]
	Thus \(f\) has a \emph{simple pole} at \(z=a\) with residue
	\[
	\Res_{z=a}\bigl(f(z)\,dz\bigr) = 1.
	\]
	
	\paragraph{At \(z=-a\).}
	Because \(\wp\) is even and \(\wp'\) is odd,
	\[
	\wp(-z) = \wp(z),\quad \wp'(-z) = -\wp'(z).
	\]
	The equation \(\wp(z) = \wp(a)\) has the two solutions \(z = \pm a\) modulo \(\Lambda\).
	So \(\wp(z)-\wp(a)\) also vanishes at \(z=-a\). Repeating the same expansion, or
	just using the symmetry, we find
	\[
	\Res_{z=-a}\bigl(f(z)\,dz\bigr) = 1.
	\]
	
	\paragraph{At \(z=0\) (and other lattice points).}
	Near \(z=0\),
	\[
	\wp(z) \sim \frac{1}{z^2},\quad \wp'(z)\sim -\frac{2}{z^3}.
	\]
	Then
	\[
	f(z) = \frac{\wp'(z)}{\wp(z)-\wp(a)}
	\sim \frac{-2 z^{-3}}{z^{-2} - \wp(a)}
	= \frac{-2 z^{-3}}{z^{-2}(1 - \wp(a)z^2)}
	= \frac{-2}{z}\cdot\frac{1}{1 - \wp(a)z^2}.
	\]
	Expand \(\frac{1}{1 - \wp(a)z^2} = 1 + \wp(a)z^2 + \cdots\), so
	\[
	f(z) \sim \frac{-2}{z} + (\text{holomorphic terms}).
	\]
	Thus at \(z=0\) we have a simple pole with residue
	\[
	\Res_{z=0}\bigl(f(z)\,dz\bigr) = -2.
	\]
	
	Similarly, at other lattice points \(\lambda\neq 0\), the local behavior is
	like near 0 shifted by \(\lambda\), giving the same residue pattern, but on the
	torus \(X=\C/\Lambda\) all lattice points project to a finite set of points,
	and you can group residues within one fundamental domain.
	
	\subsubsection*{Residue theorem on the torus}
	
	On the compact Riemann surface \(X\),
	\[
	\sum_{p\in X} \Res_p\bigl(f(z)\,dz\bigr) = 0.
	\]
	From the computations:
	
	\begin{itemize}
		\item \(\Res_{[a]} = 1\),
		\item \(\Res_{[-a]} = 1\),
		\item \(\Res_{[0]} = -2\)
	\end{itemize}
	(and no other poles in a fundamental parallelogram),
	so
	\[
	1 + 1 - 2 = 0.
	\]
	This is a very concrete check of the residue theorem on a torus for this
	particular elliptic function.
	
	\subsection{Function field \(\M(\C/\Lambda)\) via \(\wp\) and \(\wp'\)}
	
	A fundamental theorem: \(\wp\) and \(\wp'\) satisfy
	\[
	(\wp'(z))^2 = 4\wp(z)^3 - g_2 \wp(z) - g_3,
	\]
	where \(g_2,g_3\) are complex constants depending on \(\Lambda\).
	
	If we set
	\[
	X := \wp(z),\quad Y := \wp'(z),
	\]
	then
	\[
	Y^2 = 4X^3 - g_2 X - g_3.
	\]
	
	From the differential-form viewpoint:
	\[
	dX = d\wp(z) = \wp'(z)\,dz = Y\,dz
	\quad\Rightarrow\quad
	dz = \frac{dX}{Y}.
	\]
	So the basic holomorphic 1-form \(dz\) on the torus becomes
	\[
	dz = \frac{dX}{\sqrt{4X^3 - g_2 X - g_3}},
	\]
	when we view the torus as the algebraic curve \(Y^2 = 4X^3 - g_2 X - g_3\).
	
	The function field is then
	\[
	\M(\C/\Lambda)
	= \C(\wp,\wp')
	\cong \C(X,Y)\big/\bigl(Y^2 - 4X^3 + g_2X + g_3\bigr).
	\]
	
	Concretely, any meromorphic function \(F\) on \(X\) is of the form
	\[
	F([z]) = R\bigl(\wp(z),\wp'(z)\bigr)
	\]
	for some rational function \(R(X,Y)\).
	
	Our explicit example
	\[
	f(z) = \frac{\wp'(z)}{\wp(z)-\wp(a)} = \frac{Y}{X - \wp(a)}
	\]
	is exactly such a rational function in \((X,Y)=(\wp,\wp')\), and we computed
	its poles and residues using differential-form techniques in detail.
	
	%%%%%%%%%%%%%%%%%%%%%%%%%%%%%%%%%%%%%%%%%%%%%%%%%%%%%%%%%%%%
	\section*{Conclusion}
	
	\begin{itemize}
		\item For \(X=\CP^1\), we picked a very concrete rational function
		\(f(z)=(z-1)^2/(z(z-2))\), computed its divisor, wrote it via partial fractions,
		and computed \(df\) as a meromorphic 1-form, verifying the residue behavior.
		This illustrates \(\M(\CP^1)=\C(z)\).
		
		\item For \(X=\C/\Lambda\), we used \(\wp,\wp'\) and a very explicit elliptic
		function \(f(z)=\wp'(z)/(\wp(z)-\wp(a))\), expanded near poles, and computed residues of
		the meromorphic 1-form \(f(z)\,dz\). This illustrates concretely how
		\(\M(\C/\Lambda)\) is generated by \(\wp,\wp'\), and how differential forms
		(like \(\wp'(z)\,dz = d\wp(z)\)) control the structure of the function field.
	\end{itemize}
	
	\section*{Integrating $\omega = \dfrac{\wp'(z)}{\wp(z)-\wp(a)}\,dz$ around the torus}
	
	We continue with the torus
	\[
	X = \C/\Lambda, \qquad
	\Lambda = \mathbb Z\omega_1 \oplus \mathbb Z\omega_2,
	\quad \Im\left(\frac{\omega_2}{\omega_1}\right) > 0.
	\]
	
	Recall the elliptic function
	\[
	f(z) = \frac{\wp'(z)}{\wp(z) - \wp(a)},
	\]
	for some fixed \(a\in\C\) with \(2a\notin\Lambda\), and the meromorphic 1-form
	\[
	\omega = f(z)\,dz.
	\]
	
	We already computed the residues:
	\[
	\Res_{z=a}(\omega) = 1,\qquad
	\Res_{z=-a}(\omega) = 1,\qquad
	\Res_{z=0}(\omega) = -2
	\]
	(modulo the lattice). Now we will:
	
	\begin{itemize}
		\item Fix a fundamental parallelogram in \(\C\).
		\item Integrate \(\omega\) around its boundary.
		\item Use periodicity of \(f\) to show the boundary integral is \(0\).
		\item Use the residue theorem to show that this equals \(2\pi i\) times the sum of residues inside.
	\end{itemize}
	
	\subsection*{1. Fundamental parallelogram and its boundary}
	
	Pick a fundamental parallelogram (fundamental domain) for \(\Lambda\) in \(\C\):
	\[
	\mathcal{P}
	:= \left\{ s\omega_1 + t\omega_2 \;\middle|\; 0 \le s \le 1,\; 0\le t \le 1 \right\}.
	\]
	Its vertices are
	\[
	0,\quad \omega_1,\quad \omega_1+\omega_2,\quad \omega_2.
	\]
	
	Define the oriented boundary \(\partial\mathcal{P}\) as the piecewise smooth path:
	\[
	\partial\mathcal{P}
	= \gamma_1 + \gamma_2 + \gamma_3 + \gamma_4,
	\]
	where
	\begin{align*}
		\gamma_1 &: 0 \to \omega_1,\\
		\gamma_2 &: \omega_1 \to \omega_1 + \omega_2,\\
		\gamma_3 &: \omega_1 + \omega_2 \to \omega_2,\\
		\gamma_4 &: \omega_2 \to 0.
	\end{align*}
	More concretely, parametrize each side:
	
	\begin{itemize}
		\item \(\gamma_1(t) = t\,\omega_1,\quad 0\le t\le 1.\)
		\item \(\gamma_2(t) = \omega_1 + t\,\omega_2,\quad 0\le t\le 1.\)
		\item \(\gamma_3(t) = \omega_1 + \omega_2 - t\,\omega_1,\quad 0\le t\le 1.\)
		\item \(\gamma_4(t) = \omega_2 - t\,\omega_2,\quad 0\le t\le 1.\)
	\end{itemize}
	
	We will compute
	\[
	\oint_{\partial\mathcal{P}} \omega
	= \int_{\gamma_1}\omega + \int_{\gamma_2}\omega
	+ \int_{\gamma_3}\omega + \int_{\gamma_4}\omega.
	\]
	
	\subsection*{2. Periodicity of $f$ and cancellation of integrals on opposite sides}
	
	Since \(f\) is \(\Lambda\)-periodic, we have
	\[
	f(z+\omega_1) = f(z),\quad f(z+\omega_2) = f(z),\quad \forall z\in\C.
	\]
	We use this to relate the integrals on opposite sides of \(\mathcal{P}\).
	
	\subsubsection*{2.1. Compare $\int_{\gamma_1}\omega$ and $\int_{\gamma_3}\omega$}
	
	Write
	\[
	I_1 := \int_{\gamma_1}\omega,\qquad
	I_3 := \int_{\gamma_3}\omega.
	\]
	
	First compute \(I_1\) explicitly:
	\[
	\gamma_1(t) = t\,\omega_1,\quad
	\gamma_1'(t) = \omega_1,\quad 0\le t\le 1.
	\]
	Then
	\[
	I_1
	= \int_{\gamma_1} f(z)\,dz
	= \int_0^1 f(\gamma_1(t))\,\gamma_1'(t)\,dt
	= \int_0^1 f(t\omega_1)\,\omega_1\,dt
	= \omega_1 \int_0^1 f(t\omega_1)\,dt.
	\]
	
	Now compute \(I_3\). Parametrization:
	\[
	\gamma_3(t)
	= \omega_1 + \omega_2 - t\,\omega_1,\quad
	\gamma_3'(t) = -\omega_1,\quad 0\le t\le 1.
	\]
	So
	\[
	I_3
	= \int_{\gamma_3} f(z)\,dz
	= \int_0^1 f(\gamma_3(t))\,\gamma_3'(t)\,dt
	= \int_0^1 f(\omega_1+\omega_2 - t\omega_1)\cdot(-\omega_1)\,dt.
	\]
	
	We now use periodicity to simplify \(f(\omega_1+\omega_2 - t\omega_1)\):
	\[
	\omega_1+\omega_2 - t\omega_1
	= \omega_2 + (1-t)\omega_1.
	\]
	Since \(f\) is periodic with period \(\omega_2\), we can subtract \(\omega_2\):
	\[
	f(\omega_2 + (1-t)\omega_1)
	= f((1-t)\omega_1).
	\]
	Thus
	\[
	I_3
	= \int_0^1 f((1-t)\omega_1)\cdot(-\omega_1)\,dt.
	\]
	
	Make the change of variable
	\[
	s = 1-t \quad\Rightarrow\quad t = 1-s,\quad dt = -ds,
	\]
	and when \(t=0\), \(s=1\); when \(t=1\), \(s=0\). Then
	\[
	I_3
	= \int_{s=1}^{s=0} f(s\omega_1)\cdot(-\omega_1)\cdot(-ds)
	= \int_{s=1}^{s=0} f(s\omega_1)\,\omega_1\,ds.
	\]
	Swap limits:
	\[
	I_3
	= -\int_{s=0}^{s=1} f(s\omega_1)\,\omega_1\,ds
	= -\omega_1 \int_0^1 f(s\omega_1)\,ds.
	\]
	
	Comparing with
	\[
	I_1 = \omega_1 \int_0^1 f(t\omega_1)\,dt,
	\]
	we see
	\[
	I_3 = -I_1.
	\]
	
	So the integrals along the bottom and top sides cancel:
	\[
	\int_{\gamma_1}\omega + \int_{\gamma_3}\omega = I_1 + I_3 = 0.
	\]
	
	\subsubsection*{2.2. Compare $\int_{\gamma_2}\omega$ and $\int_{\gamma_4}\omega$}
	
	Similarly define
	\[
	I_2 := \int_{\gamma_2}\omega,\qquad
	I_4 := \int_{\gamma_4}\omega.
	\]
	
	Parametrize \(\gamma_2\):
	\[
	\gamma_2(t) = \omega_1 + t\omega_2,\quad
	\gamma_2'(t) = \omega_2,\quad 0\le t\le 1.
	\]
	Then
	\[
	I_2
	= \int_{\gamma_2} f(z)\,dz
	= \int_0^1 f(\omega_1 + t\omega_2)\,\omega_2\,dt.
	\]
	
	Parametrize \(\gamma_4\):
	\[
	\gamma_4(t) = \omega_2 - t\omega_2,\quad
	\gamma_4'(t) = -\omega_2,\quad 0\le t\le 1.
	\]
	Then
	\[
	I_4
	= \int_{\gamma_4} f(z)\,dz
	= \int_0^1 f(\omega_2 - t\omega_2)\cdot(-\omega_2)\,dt.
	\]
	
	Simplify \(f(\omega_2 - t\omega_2)\):
	\[
	\omega_2 - t\omega_2 = (1-t)\omega_2.
	\]
	Because \(f(z+\omega_1) = f(z)\), we can add \(\omega_1\) if we want:
	\[
	f((1-t)\omega_2)
	= f(\omega_1 + (1-t)\omega_2)
	\]
	as well. But we do not strictly need that here. Do the change of variable
	\[
	s = 1-t \quad\Rightarrow\quad dt = -ds,
	\]
	and when \(t=0\to s=1\), \(t=1\to s=0\). Then
	\[
	I_4
	= \int_{s=1}^{s=0} f(s\omega_2)\cdot(-\omega_2)\cdot(-ds)
	= \int_{s=1}^{s=0} f(s\omega_2)\,\omega_2\,ds
	= -\int_{s=0}^{s=1} f(s\omega_2)\,\omega_2\,ds.
	\]
	
	But
	\[
	I_2 = \int_0^1 f(\omega_1 + t\omega_2)\,\omega_2\,dt.
	\]
	Use periodicity with period \(\omega_1\):
	\[
	f(\omega_1 + t\omega_2) = f(t\omega_2).
	\]
	So
	\[
	I_2 = \int_0^1 f(t\omega_2)\,\omega_2\,dt = \omega_2\int_0^1 f(t\omega_2)\,dt.
	\]
	Using $s$ instead of $t$ as dummy variable,
	\[
	I_2 = \omega_2\int_0^1 f(s\omega_2)\,ds.
	\]
	
	Thus
	\[
	I_4 = -\omega_2\int_0^1 f(s\omega_2)\,ds = -I_2.
	\]
	
	So the integrals along the left and right sides cancel:
	\[
	\int_{\gamma_2}\omega + \int_{\gamma_4}\omega = I_2 + I_4 = 0.
	\]
	
	\subsubsection*{2.3. Total boundary integral}
	
	Putting together:
	\[
	\oint_{\partial\mathcal{P}} \omega
	= I_1 + I_2 + I_3 + I_4
	= (I_1 + I_3) + (I_2 + I_4)
	= 0 + 0 = 0.
	\]
	
	So we have shown purely from periodicity and calculus parametrizations:
	\[
	\oint_{\partial\mathcal{P}} f(z)\,dz = 0.
	\]
	
	\subsection*{3. Residue theorem inside the fundamental parallelogram}
	
	Now apply the residue theorem from complex analysis.
	
	\subsubsection*{3.1. Poles inside $\mathcal{P}$}
	
	We consider the poles of $\omega$ inside the parallelogram $\mathcal{P}$. We chose
	$a$ such that $0,a,-a$ represent three distinct points mod $\Lambda$, and (for
	a generic choice of $a$) we can assume that within the chosen fundamental domain
	$\mathcal{P}$, the only poles of $f(z)$ are at
	\[
	z=0,\quad z=a,\quad z=-a.
	\]
	(Any other poles are at lattice translates of these, lying in other translates of
	the parallelogram.)
	
	We computed the residues:
	\[
	\Res_{z=a}(\omega) = 1,\quad
	\Res_{z=-a}(\omega) = 1,\quad
	\Res_{z=0}(\omega) = -2.
	\]
	
	\subsubsection*{3.2. Residue theorem}
	
	The residue theorem applied to $\omega = f(z)\,dz$ on $\mathcal{P}$ states:
	\[
	\oint_{\partial\mathcal{P}} \omega
	= 2\pi i \sum_{p\in \mathcal{P}} \Res_{z=p}(\omega),
	\]
	where the sum is over all poles of $\omega$ inside $\mathcal{P}$.
	
	From the previous subsection:
	\[
	\oint_{\partial\mathcal{P}} \omega = 0.
	\]
	Thus
	\[
	0 = 2\pi i \left(
	\Res_{z=a}(\omega)
	+ \Res_{z=-a}(\omega)
	+ \Res_{z=0}(\omega)
	\right)
	= 2\pi i (1 + 1 - 2).
	\]
	So indeed
	\[
	1 + 1 - 2 = 0.
	\]
	
	\subsection*{4. Interpretation on the torus $X = \C/\Lambda$}
	
	The fundamental parallelogram $\mathcal{P}$ is a fundamental domain for the
	projection map $\pi:\C\to X=\C/\Lambda$. Its boundary edges are identified
	in pairs:
	
	\[
	\gamma_1\sim\gamma_3,\quad
	\gamma_2\sim\gamma_4.
	\]
	
	The fact that
	\[
	\int_{\gamma_1}\omega + \int_{\gamma_3}\omega = 0,\quad
	\int_{\gamma_2}\omega + \int_{\gamma_4}\omega = 0
	\]
	is exactly the statement that, when you go to the quotient torus $X$, the
	integral of $\omega$ over the resulting closed loop is well-defined and
	``wraps around'' consistently.
	
	On the compact Riemann surface $X$, the residue theorem becomes
	\[
	\sum_{p\in X} \Res_{p}(\omega) = 0.
	\]
	In our explicit example,
	\[
	\Res_{[a]}(\omega) + \Res_{[-a]}(\omega) + \Res_{[0]}(\omega)
	= 1 + 1 - 2 = 0,
	\]
	in perfect agreement with the global theory.
	
	\bigskip
	
	\noindent\textbf{Summary.}
	For the elliptic function
	\[
	f(z) = \frac{\wp'(z)}{\wp(z)-\wp(a)}
	\]
	on the torus $X=\C/\Lambda$, we have:
	
	\begin{itemize}
		\item an explicit meromorphic 1-form $\omega=f(z)\,dz$;
		\item an explicit fundamental parallelogram $\mathcal{P}$ in $\C$;
		\item a direct computation using periodicity shows $\displaystyle\oint_{\partial\mathcal{P}}\omega = 0$;
		\item the residue theorem computes the same integral as
		\(\displaystyle 2\pi i \sum \Res_p(\omega)\),
		giving $2\pi i(1+1-2)=0$.
	\end{itemize}
	
	This is a fully concrete, calculus-level demonstration of how residues and
	periodicity interact on the torus, and how global facts about the function
	field $\M(\C/\Lambda)$ arise from local Laurent expansions and contour
	integrals.
	
	
\end{document}
