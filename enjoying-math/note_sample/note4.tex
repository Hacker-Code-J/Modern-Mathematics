% =====================================================================
% OPTION B (editable, fully re-typeset): closed/exact forms and Poincaré.
% =====================================================================
\documentclass[11pt]{book}

\usepackage{amsmath,amssymb,amsthm,mathtools}
\usepackage[hidelinks]{hyperref}
\usepackage[margin=1.2in]{geometry}
\usepackage{xcolor}
%\usepackage{enumitem}
\usepackage{enumerate}
\usepackage{tcolorbox}
\tcbset{
	colback=white,
	colframe=teal
}

% ---------- Theorem style ----------
\theoremstyle{definition}
\newtheorem{definition}{Definition}[chapter]
\newtheorem{example}{Example}[chapter]
\newtheorem{exercise}{Exercise}[chapter]

\theoremstyle{plain}
\newtheorem{theorem}[definition]{Theorem}
\newtheorem{proposition}[definition]{Proposition}
\newtheorem{lemma}[definition]{Lemma}
\newtheorem{corollary}[definition]{Corollary}

%\theoremstyle{remark}
\newtheorem{remark}[definition]{Remark}

% ---------- Notation ----------
\newcommand{\sol}{\color{magenta}\normalfont\textbf{Sol}}
\newcommand{\R}{\mathbb{R}}
\newcommand{\C}{\mathbb{C}}
\newcommand{\id}{\mathrm{id}}
\newcommand{\dd}{\,\mathrm{d}}
\newcommand{\Om}{\Omega}
\newcommand{\pull}[1]{#1^{*}}
\newcommand{\into}{\mathbin{\lrcorner}}
\newcommand{\Lie}{\mathcal{L}}
\newcommand{\seg}[2]{\overline{#1#2}}

\renewcommand{\vec}[1]{\mathbf{#1}}
\newcommand{\TpM}{T_pM}

\begin{document}
	
\chapter{Closed Forms and Exact Forms}
	
%	\section{Closed and exact forms}
%	
%	Let $A\subseteq \R^{n}$ be open. For each integer $k\ge 0$, write $\Om^{k}(A)$
%	for the $\R$-vector space of smooth differential $k$-forms on $A$, and
%	\[
%	d:\Om^{k}(A)\longrightarrow \Om^{k+1}(A)
%	\]
%	for the exterior derivative. Recall that $d\circ d=0$.
%	
%	\begin{definition}[Closed and exact]
%		Let $\omega\in \Om^{k}(A)$.
%		\begin{enumerate}[label=\textup{(\roman*)},leftmargin=3.2em]
%			\item $\omega$ is \emph{closed} if $d\omega=0$.
%			\item $\omega$ is \emph{exact} if there exists $\eta\in \Om^{k-1}(A)$ such that
%			$\omega=d\eta$ (for $k\ge 1$).
%		\end{enumerate}
%	\end{definition}
%	
%	\begin{remark}
%		Exactness implies closedness by $d^{2}=0$. The converse is false on general domains;
%		the Poincar\'e lemma identifies a large and important class of domains on which the converse holds.
%	\end{remark}
%	
%	\section{Differentiable homotopy and the homotopy formula}
%	
%	\begin{definition}[Differentiable homotopy]
%		Let $A\subseteq \R^{n}$ and $B\subseteq \R^{m}$ be open, and let $g,h:A\to B$ be $C^{\infty}$.
%		We say that $g$ and $h$ are \emph{differentiably homotopic} if there exists a $C^{\infty}$ map
%		\[
%		H:A\times I \longrightarrow B,\qquad I\coloneqq [0,1],
%		\]
%		such that $H(x,0)=g(x)$ and $H(x,1)=h(x)$ for all $x\in A$.
%	\end{definition}
%	
%	We write $H_{t}:A\to B$ for the time-$t$ slice $H_{t}(x)\coloneqq H(x,t)$, so that
%	$H_{0}=g$ and $H_{1}=h$.
%	
%	\begin{definition}[Contraction with $\partial/\partial t$ and the homotopy operator]
%		Let $\eta\in \Om^{k}(B)$ with $k\ge 1$ and let $H:A\times I\to B$ be smooth.
%		Define a linear map
%		\[
%		P_{H}:\Om^{k}(B)\longrightarrow \Om^{k-1}(A)
%		\]
%		by
%		\[
%		P_{H}(\eta)\;\coloneqq\;\int_{0}^{1}
%		\Bigl(\,\iota_{\partial/\partial t}\,\pull{H}(\eta)\Bigr)\Big|_{A\times\{t\}}
%		\,\dd t.
%		\]
%		(Here $\iota_{\partial/\partial t}$ denotes interior product with the vector field $\partial/\partial t$
%		on $A\times I$.)
%	\end{definition}
%	
%	\begin{theorem}[Homotopy formula]
%		Let $A\subseteq \R^{n}$ and $B\subseteq \R^{m}$ be open, and let $g,h:A\to B$ be
%		differentiably homotopic via $H:A\times I\to B$. Then for every $k\ge 1$ and every
%		$\eta\in \Om^{k}(B)$,
%		\[
%		d\bigl(P_{H}\eta\bigr)\;+\;P_{H}(d\eta)\;=\;\pull{h}(\eta)\;-\;\pull{g}(\eta).
%		\]
%		For $k=0$ (i.e.\ $\eta=f\in C^{\infty}(B)$) one has
%		\[
%		P_{H}(df)=\pull{h}(f)-\pull{g}(f).
%		\]
%	\end{theorem}
%	
%	\begin{proof}
%		Let $\eta\in\Om^{k}(B)$ with $k\ge 1$. On $A\times I$ we have Cartan's identity
%		\[
%		\Lie_{\partial/\partial t}=\;d\circ \iota_{\partial/\partial t}+\iota_{\partial/\partial t}\circ d.
%		\]
%		Apply this to $\pull{H}(\eta)$:
%		\[
%		\Lie_{\partial/\partial t}\bigl(\pull{H}(\eta)\bigr)
%		= d\Bigl(\iota_{\partial/\partial t}\pull{H}(\eta)\Bigr)
%		+ \iota_{\partial/\partial t}\Bigl(d\pull{H}(\eta)\Bigr).
%		\]
%		Since exterior differentiation commutes with pullback,
%		$d\pull{H}(\eta)=\pull{H}(d\eta)$. Restrict to $A\times\{t\}$ and integrate in $t$ from $0$ to $1$:
%		\begin{align*}
%			\int_{0}^{1}\frac{\dd}{\dd t}\Bigl(\pull{H_{t}}(\eta)\Bigr)\dd t
%			&=
%			\int_{0}^{1} d\Bigl(\iota_{\partial/\partial t}\pull{H}(\eta)\Bigr)\Big|_{A\times\{t\}}\dd t
%			+
%			\int_{0}^{1} \Bigl(\iota_{\partial/\partial t}\pull{H}(d\eta)\Bigr)\Big|_{A\times\{t\}}\dd t \\
%			\pull{H_{1}}(\eta)-\pull{H_{0}}(\eta)
%			&=
%			d\left(\int_{0}^{1}\Bigl(\iota_{\partial/\partial t}\pull{H}(\eta)\Bigr)\Big|_{A\times\{t\}}\dd t\right)
%			+
%			P_{H}(d\eta).
%		\end{align*}
%		The first integral is precisely $P_{H}\eta$, and $H_{1}=h$, $H_{0}=g$, yielding
%		$d(P_{H}\eta)+P_{H}(d\eta)=\pull{h}(\eta)-\pull{g}(\eta)$.
%		
%		For $k=0$, the same computation applied to $df$ gives $P_{H}(df)=\pull{h}(f)-\pull{g}(f)$.
%	\end{proof}
%	
%	\section{Star-convex domains and the Poincar\'e lemma}
%	
%	\begin{definition}[Star-convex (star-shaped)]
%		Let $A\subseteq \R^{n}$ be open. We say $A$ is \emph{star-convex with respect to} $p\in A$
%		if for every $x\in A$ the line segment
%		\[
%		\{(1-t)p+tx:\;t\in[0,1]\}
%		\]
%		is contained in $A$.
%	\end{definition}
%	
%	\begin{theorem}[Poincar\'e lemma on star-convex sets]
%		Let $A\subseteq \R^{n}$ be a star-convex open set. If $\omega\in\Om^{k}(A)$ is closed and $k\ge 1$,
%		then $\omega$ is exact on $A$.
%		Equivalently,
%		\[
%		H^{k}_{\mathrm{dR}}(A)=0\qquad\text{for all }k\ge 1.
%		\]
%	\end{theorem}
%	
%	\begin{proof}
%		Choose $p\in A$ with respect to which $A$ is star-convex. Let $h:A\to A$ be the identity,
%		and let $g:A\to A$ be the constant map $g(x)\equiv p$. Define a smooth homotopy
%		\[
%		H:A\times I\to A,\qquad H(x,t)\coloneqq (1-t)p+tx.
%		\]
%		Star-convexity ensures $H(x,t)\in A$ for all $(x,t)\in A\times I$.
%		
%		Let $\omega\in\Om^{k}(A)$ be closed with $k\ge 1$. Apply the homotopy formula:
%		\[
%		d\bigl(P_{H}\omega\bigr)+P_{H}(d\omega)=\pull{h}(\omega)-\pull{g}(\omega).
%		\]
%		Since $d\omega=0$, the left side is $d(P_{H}\omega)$. Since $h=\id$, we have $\pull{h}(\omega)=\omega$.
%		Since $g$ is constant, $\pull{g}(\omega)=0$ for $k\ge 1$. Therefore
%		\[
%		d\bigl(P_{H}\omega\bigr)=\omega,
%		\]
%		so $\omega$ is exact, with primitive $\eta\coloneqq P_{H}\omega\in\Om^{k-1}(A)$.
%	\end{proof}
%	
%	\begin{theorem}[Uniqueness of primitives up to lower-order exactness/constancy]
%		Let $A\subseteq \R^{n}$ be star-convex and let $\omega\in\Om^{k}(A)$ be closed.
%		\begin{enumerate}[label=\textup{(\roman*)},leftmargin=3.2em]
%			\item If $k\ge 2$ and $\eta,\eta_{0}\in\Om^{k-1}(A)$ satisfy $d\eta=d\eta_{0}=\omega$,
%			then $\eta-\eta_{0}$ is exact.
%			\item If $k=1$ and $f,f_{0}\in C^{\infty}(A)$ satisfy $df=df_{0}=\omega$,
%			then $f-f_{0}$ is constant on $A$.
%		\end{enumerate}
%	\end{theorem}
%	
%	\begin{proof}
%		In either case, $d(\eta-\eta_{0})=0$ (or $d(f-f_{0})=0$). If $k\ge 2$, the Poincar\'e lemma
%		applied to the closed $(k-1)$-form $\eta-\eta_{0}$ yields exactness.
%		If $k=1$, $d(f-f_{0})=0$ implies $f-f_{0}$ is locally constant; since $A$ is connected along line segments
%		from the star point, it is constant on $A$.
%	\end{proof}

\begin{tcolorbox}[title=\bfseries Star-convex with respect to a point]
\begin{definition}
Let $U\subseteq\R^n$ be an open set in $\R^n$. We say that $U$ is \textbf{star-convex} with respect to the point $\textbf{p}$ of $U$ if for each $\textbf{x}\in U$, the line segment joining $\textbf{x}$ and $\textbf{p}$ lines in $U$.
\end{definition}
\end{tcolorbox}
\begin{remark}
Let $U\subseteq \R^n$ be open and let $\mathbf{p}\in U$.
We say that $U$ is \emph{star-convex with respect to $\mathbf{p}$} if \[
\forall\,\mathbf{x}\in U,\ \forall\,t\in[0,1],\qquad (1-t)\mathbf{p}+t\mathbf{x}\in U.
\] Equivalently, for each $\mathbf{x}\in U$, the line segment \[
\seg{\mathbf{p}}{\mathbf{x}}
=\bigl\{(1-t)\mathbf{p}+t\mathbf{x}:t\in[0,1]\bigr\}
\]
is contained in $U$.
\end{remark}

\begin{example}[Star-convex but not convex: an annulus with a slit]\label{ex:slit-annulus}
	Consider $n=2$ and define the open set
	\[
	U \;\coloneqq\;
	\Bigl\{(x,y)\in\R^2:\ 1<x^2+y^2<4\Bigr\}
	\ \setminus\
	\Bigl\{(x,0)\in\R^2:\ 1<x<2\Bigr\}.
	\]
	Geometrically, $U$ is the open annulus $\{1<r<2\}$ with the open radial segment on the
	positive $x$-axis removed. Fix the point
	\[
	\mathbf{p}\coloneqq\left(-\frac{3}{2},0\right)\in U.
	\]
	Then $U$ is star-convex with respect to $\mathbf{p}$, but $U$ is not convex.
	\begin{figure}[h!]
		\centering
		\includegraphics[scale=1]{note4-tikz1.pdf}
	\end{figure}
	\newpage
	\medskip
	\noindent\textbf{(i) Star-convexity with respect to $\mathbf{p}$.}
	For each $\mathbf{x}\in U$, define the line segment parameterization
	\[
	\gamma_{\mathbf{x}}(t)\coloneqq (1-t)\mathbf{p}+t\mathbf{x},\qquad t\in[0,1].
	\]
	The only obstruction to $\gamma_{\mathbf{x}}([0,1])\subseteq U$ (besides staying between the
	circles $r=1$ and $r=2$) is crossing the removed slit
	\[
	S\coloneqq \{(x,0):1<x<2\}.
	\]
	But $S$ lies on the positive $x$-axis, whereas $\mathbf{p}$ lies on the negative $x$-axis.
	Any segment from $\mathbf{p}$ to a point $\mathbf{x}\in U$ either:
	\begin{itemize}
		\item has $y\neq 0$ somewhere along the segment, hence cannot meet $S\subset\{y=0\}$; or
		\item stays on the $x$-axis, in which case $\mathbf{x}$ must lie on the \emph{negative} $x$-axis
		(since the positive portion $(1,2)$ has been removed), and again the segment avoids $S$.
	\end{itemize}
	Thus $\gamma_{\mathbf{x}}(t)\notin S$ for all $t$, and since both endpoints lie in the annulus
	$1<r<2$, the entire segment remains in $U$. Hence $U$ is star-convex with respect to $\mathbf{p}$.
	
	\medskip
	\noindent\textbf{(ii) Failure of convexity.}
	Choose two points on opposite sides of the slit, for instance
	\[
	\mathbf{a}\coloneqq(1.6,0.45)\in U,\qquad \mathbf{b}\coloneqq(1.6,-0.45)\in U.
	\]
	Then the segment $\seg{\mathbf{a}}{\mathbf{b}}$ is the vertical line at $x=1.6$ and it intersects
	the removed slit $S$ at $(1.6,0)\notin U$. Hence $\seg{\mathbf{a}}{\mathbf{b}}\nsubseteq U$,
	so $U$ is not convex.
\end{example}

\begin{center}

\end{center}

\newpage
\begin{tcolorbox}[title=\bfseries The Poincar\'{e} lemma]
\begin{theorem}
Let $U\subseteq\R^3$ be a star-convex (equivalently, contractible) open set in $\R^3$. If $\omega$ is a closed $k$-form on $U$, then $\omega$ is exact on $U$.
\end{theorem}
\end{tcolorbox}

\begin{exercise}
Translate the Poincare lemma for $k$-forms into theorem about scalar and vector fields in $R^3$. Consider cases $k=0,1,2,3$.
\begin{proof}[\sol]
Let $U\subseteq \R^{3}$ be a star-convex (equivalently, contractible) open set.
Identify differential forms on $U$ with scalar and vector fields via
\[
\Om^{0}(U)\cong C^{\infty}(U),\qquad
\Om^{1}(U)\cong C^{\infty}(U;\R^{3}),\qquad
\Om^{2}(U)\cong C^{\infty}(U;\R^{3}),\qquad
\Om^{3}(U)\cong C^{\infty}(U),
\]
by the explicit correspondences
\begin{align*}
	f\in C^{\infty}(U)
	&\longleftrightarrow
	\omega^{(0)} \coloneqq f,\\
	\mathbf{F}=(P,Q,R)\in C^{\infty}(U;\R^{3})
	&\longleftrightarrow
	\omega^{(1)} \coloneqq P\,dx+Q\,dy+R\,dz,\\
	\mathbf{G}=(U,V,W)\in C^{\infty}(U;\R^{3})
	&\longleftrightarrow
	\omega^{(2)} \coloneqq U\,dy\wedge dz+V\,dz\wedge dx+W\,dx\wedge dy,\\
	g\in C^{\infty}(U)
	&\longleftrightarrow
	\omega^{(3)} \coloneqq g\,dx\wedge dy\wedge dz.
\end{align*}
Under these identifications, the exterior derivative $d$ corresponds to the grad--curl--div operators:
\[
d f \longleftrightarrow \nabla f,\qquad
d\omega^{(1)} \longleftrightarrow \nabla\times \mathbf{F},\qquad
d\omega^{(2)} \longleftrightarrow \nabla\cdot \mathbf{G},\qquad
d\omega^{(3)}=0.
\]
Then the Poincar\'e lemma statement
\[
\omega\in\Om^{k}(U),\ d\omega=0\ \Longrightarrow\ \exists\,\eta\in\Om^{k-1}(U):\ \omega=d\eta
\qquad (k\ge 1)
\]
is equivalent (degree-by-degree) to the following assertions.

\medskip
\noindent\textbf{Case $k=0$ (scalar fields).}
If $f\in C^{\infty}(U)$ satisfies $df=0$ (equivalently $\nabla f=\mathbf{0}$), then $f$ is constant on each
connected component of $U$.

\medskip
\noindent\textbf{Case $k=1$ (curl-free vector fields are gradients).}
If $\mathbf{F}\in C^{\infty}(U;\R^{3})$ satisfies
\[
\nabla\times \mathbf{F}=\mathbf{0},
\]
then there exists a scalar potential $\phi\in C^{\infty}(U)$ such that
\[
\mathbf{F}=\nabla \phi.
\]
Moreover, $\phi$ is unique up to addition of a constant on each connected component of $U$.

\medskip
\noindent\textbf{Case $k=2$ (divergence-free vector fields are curls).}
If $\mathbf{G}\in C^{\infty}(U;\R^{3})$ satisfies
\[
\nabla\cdot \mathbf{G}=0,
\]
then there exists a vector potential $\mathbf{A}\in C^{\infty}(U;\R^{3})$ such that
\[
\mathbf{G}=\nabla\times \mathbf{A}.
\]
Moreover, $\mathbf{A}$ is not unique: if $\psi\in C^{\infty}(U)$ then
$\nabla\times(\mathbf{A}+\nabla\psi)=\nabla\times\mathbf{A}$.

\medskip
\noindent\textbf{Case $k=3$ (every $3$-form is exact; every scalar is a divergence).}
Every $3$-form on $U$ is closed (since $\Om^{4}(U)=0$). Hence for every $g\in C^{\infty}(U)$ there exists
$\mathbf{G}\in C^{\infty}(U;\R^{3})$ such that
\[
\nabla\cdot \mathbf{G}=g.
\]
Equivalently, every density $g\,dx\wedge dy\wedge dz$ is the exterior derivative of some $2$-form.

\end{proof}
\end{exercise}



\newpage
\begin{tcolorbox}[title=\bfseries Uniqueness of primitives on star-convex]
\begin{theorem}
Let $U\subseteq\R^n$ be a star-convex open set in $\R^n$. Let $\omega$ be a closed $k$-form on $U$. If $k>1$, and if $\eta$ and $\eta_0$ are two $k-1$ forms on $U$ with $d\eta=\omega=d\eta_0$, then \[
\eta=\eta_0+d\theta
\] for some $(k-2)$-form $\theta$ on $U$. If $k=1$, and if $f$ and $f_0$ are two $0$-forms on $U$ with $df=\omega=df_0$, then $f=f_0+c$ for some constant $c$.
\end{theorem}
\end{tcolorbox}
\begin{exercise}
%\begin{theorem}[Uniqueness of primitives on star-convex $U\subset\R^3$ (cases $k=1,2,3$)]
	\label{thm:uniq-primitives-R3-k123}
	Let $U\subseteq \R^{3}$ be a star-convex open set. Use the standard identifications
	\[
	\Om^{0}(U)\cong C^{\infty}(U),\qquad
	\Om^{1}(U)\cong C^{\infty}(U;\R^{3}),\qquad
	\Om^{2}(U)\cong C^{\infty}(U;\R^{3}),\qquad
	\Om^{3}(U)\cong C^{\infty}(U),
	\]
	given concretely by \begin{align*}
	(P,Q,R)\ &\longleftrightarrow\ P\,dx+Q\,dy+R\,dz,\qquad \\
	(U,V,W)\ &\longleftrightarrow\ U\,dy\wedge dz+V\,dz\wedge dx+W\,dx\wedge dy,\qquad \\
	g\ &\longleftrightarrow\ g\,dx\wedge dy\wedge dz,
	\end{align*}
	so that the exterior derivative corresponds to the operators
	\begin{align*}
	(d:\Om^{0}\to\Om^{1})\ &\longleftrightarrow\ (\nabla),\qquad \\
	(d:\Om^{1}\to\Om^{2})\ &\longleftrightarrow\ (\nabla\times),\qquad \\
	(d:\Om^{2}\to\Om^{3})\ &\longleftrightarrow\ (\nabla\cdot).
	\end{align*}
	Let $\omega$ be a closed $k$-form on $U$ (so $d\omega=0$). Suppose $\eta,\eta_{0}$ are two $(k-1)$-forms
	satisfying $d\eta=\omega=d\eta_{0}$. Then, in $\R^{3}$, this statement becomes the following
	degree-by-degree uniqueness assertions.
	
	\medskip
	\noindent\textbf{Case $k=1$ (scalar potentials for a curl-free vector field).}
	Let $\mathbf{F}\in C^{\infty}(U;\R^{3})$ satisfy
	\[
	\nabla\times \mathbf{F}=\mathbf{0}.
	\]
	If $\phi,\phi_{0}\in C^{\infty}(U)$ satisfy
	\[
	\nabla\phi=\mathbf{F}=\nabla\phi_{0},
	\]
	then there exists a constant $c\in\R$ (on each connected component of $U$) such that
	\[
	\phi=\phi_{0}+c.
	\]
	Equivalently, scalar potentials of the same vector field are unique up to addition of constants.
	
	\medskip
	\noindent\textbf{Case $k=2$ (vector potentials for a divergence-free vector field).}
	Let $\mathbf{G}\in C^{\infty}(U;\R^{3})$ satisfy
	\[
	\nabla\cdot \mathbf{G}=0.
	\]
	If $\mathbf{A},\mathbf{A}_{0}\in C^{\infty}(U;\R^{3})$ satisfy
	\[
	\nabla\times \mathbf{A}=\mathbf{G}=\nabla\times \mathbf{A}_{0},
	\]
	then there exists a scalar field $\psi\in C^{\infty}(U)$ such that
	\[
	\mathbf{A}=\mathbf{A}_{0}+\nabla\psi.
	\]
	Equivalently, vector potentials of the same field are unique up to ``gauge'' transformations
	$\mathbf{A}\mapsto \mathbf{A}+\nabla\psi$.
	
	\medskip
	\noindent\textbf{Case $k=3$ (primitives of a $3$-form / solutions to a divergence equation).}
	Let $g\in C^{\infty}(U)$ and suppose $\mathbf{H},\mathbf{H}_{0}\in C^{\infty}(U;\R^{3})$ satisfy
	\[
	\nabla\cdot \mathbf{H}=g=\nabla\cdot \mathbf{H}_{0}.
	\]
	Then there exists a vector field $\mathbf{B}\in C^{\infty}(U;\R^{3})$ such that
	\[
	\mathbf{H}=\mathbf{H}_{0}+\nabla\times \mathbf{B}.
	\]
	Equivalently, solutions to $\nabla\cdot \mathbf{H}=g$ are unique up to addition of a curl field,
	because $\nabla\cdot(\nabla\times \mathbf{B})=0$ and (on star-convex $U$) every divergence-free field
	is a curl.
%\end{theorem}
\end{exercise}

\newpage
\chapter{Differentiable Manifolds and Riemannian Manifolds}
\begin{exercise}
Show that if $\vec{v}\in T_p(M)$, then $\vec{v}$ is the velocity vector of some $C^\infty$ curve $\gamma$ in $M$ passing through $p$.
\begin{proof}[\sol]
Let $M$ be a smooth manifold, let $p\in M$, and let $\vec v\in T_pM$.
Then there exist an open interval $I\subset \R$ with $0\in I$ and a $C^\infty$ curve
$\gamma:I\to M$ such that
\[
\gamma(0)=p,\qquad \gamma'(0)=\vec v.
\]
Moreover, one may take $I=\R$.

\medskip\noindent
Choose a smooth chart $(U,\varphi)$ about $p$, with $p\in U$ and
$\varphi:U\to \varphi(U)\subset \R^n$ a diffeomorphism onto an open set.
Define
\[
a \;\coloneqq\; (d\varphi)_p(\vec v)\in T_{\varphi(p)}\R^n\cong \R^n.
\]
Since $\varphi(U)$ is open, there exists $\varepsilon>0$ such that
\[
\varphi(p)+t a \in \varphi(U)\qquad \text{for all }|t|<\varepsilon.
\]
Define a smooth curve $\widetilde\gamma:(-\varepsilon,\varepsilon)\to \R^n$ by
\[
\widetilde\gamma(t)\coloneqq \varphi(p)+t a,
\]
and define $\gamma:(-\varepsilon,\varepsilon)\to M$ by
\[
\gamma(t)\coloneqq \varphi^{-1}\bigl(\widetilde\gamma(t)\bigr)
=\varphi^{-1}\bigl(\varphi(p)+t a\bigr).
\]
Then $\gamma$ is $C^\infty$ (being the composition of smooth maps) and
\[
\gamma(0)=\varphi^{-1}(\varphi(p))=p.
\]

To prove $\gamma'(0)=\vec v$, let $f\in C^\infty(M)$ be arbitrary. By the chain rule,
\begin{align*}
	\left.\frac{d}{dt}\right|_{t=0}(f\circ \gamma)(t)
	&=
	\left.\frac{d}{dt}\right|_{t=0}
	\Bigl( (f\circ \varphi^{-1})\bigl(\varphi(p)+t a\bigr)\Bigr) \\
	&=
	d(f\circ \varphi^{-1})_{\varphi(p)}(a).
\end{align*}
Substituting $a=(d\varphi)_p(\vec v)$ and using the functoriality of the differential,
\[
d(f\circ \varphi^{-1})_{\varphi(p)}\circ (d\varphi)_p
=
d\bigl((f\circ \varphi^{-1})\circ \varphi\bigr)_p
=
(df)_p,
\]
we obtain
\[
\left.\frac{d}{dt}\right|_{t=0}(f\circ \gamma)(t)
=
(df)_p(\vec v)
=
\vec v(f).
\]
By the defining characterization of the velocity vector $\gamma'(0)\in T_pM$ as the derivation
$f\mapsto \left.\frac{d}{dt}\right|_{t=0}(f\circ\gamma)(t)$, this equality for all
$f\in C^\infty(M)$ implies $\gamma'(0)=\vec v$.

\medskip
Finally, to obtain a curve defined on all of $\R$, choose $\beta\in C^\infty(\R)$ such that
\[
\beta(t)=t\ \text{ for }|t|\ \text{small},\qquad |\beta(t)|<\varepsilon\ \text{ for all }t\in\R.
\]
(For example, one may take $\beta(t)=\varepsilon\tanh(t/\varepsilon)$.)
Define
\[
\Gamma(t)\coloneqq \varphi^{-1}\bigl(\varphi(p)+\beta(t)a\bigr),\qquad t\in \R.
\]
Then $\Gamma\in C^\infty(\R,M)$, $\Gamma(0)=p$, and $\Gamma'(0)=\vec v$ since $\beta'(0)=1$.
\end{proof}
\end{exercise}

\begin{example}[Unit circle version of the chart--pushforward construction]
Let
\[
M=S^1=\{(x,y)\in\R^2:\ x^2+y^2=1\}.
\]
Fix the point
\[
p=(1,0)\in S^1.
\]
Then
\[
T_pS^1=\{(0,\lambda):\lambda\in\R\}\subset \R^2.
\]
Choose the tangent vector
\[
\vec v=(0,1)\in T_pS^1.
\]

\subsection*{A concrete smooth chart around \(p\)}
Let
\[
U \coloneqq \{(x,y)\in S^1:\ x>0\},
\]
an open neighborhood of \(p\) in \(S^1\). Define
\[
\varphi:U\to (-1,1),\qquad \varphi(x,y)\coloneqq y.
\]
Then \(\varphi\) is a \(C^\infty\) diffeomorphism onto the open interval \((-1,1)\), with inverse
\[
\varphi^{-1}:(-1,1)\to U,\qquad \varphi^{-1}(u)=\bigl(\sqrt{1-u^{2}},\,u\bigr).
\]
Note that
\[
\varphi(p)=\varphi(1,0)=0\in (-1,1).
\]

\subsection*{Computing \(a=(d\varphi)_p(\vec v)\)}
Since \(\varphi\) is (the restriction to \(S^1\)) of the ambient coordinate function \((x,y)\mapsto y\),
the differential at \(p\) satisfies
\[
(d\varphi)_p(w)=w_y \qquad \text{for all } w=(w_x,w_y)\in T_pS^1.
\]
Hence
\[
a\;\coloneqq\;(d\varphi)_p(\vec v)=(d\varphi)_p(0,1)=1\in T_{\varphi(p)}\R\cong \R.
\]

\subsection*{The straight line in coordinates and its pullback to the manifold}
Because \(\varphi(U)=(-1,1)\) is open, there exists \(\varepsilon>0\) such that
\[
\varphi(p)+t a = 0+t\in (-1,1)\qquad\text{for all }|t|<\varepsilon.
\]
For instance, one may take \(\varepsilon=1\).

Define
\[
\widetilde\gamma:(-\varepsilon,\varepsilon)\to \R,\qquad
\widetilde\gamma(t)\coloneqq \varphi(p)+t a = t.
\]
Define a curve in \(S^1\) by pulling back via \(\varphi^{-1}\):
\[
\gamma:(-\varepsilon,\varepsilon)\to S^1,\qquad
\gamma(t)\coloneqq \varphi^{-1}\bigl(\widetilde\gamma(t)\bigr)
=\varphi^{-1}(t)=\bigl(\sqrt{1-t^{2}},\,t\bigr).
\]
Then \(\gamma\) is \(C^\infty\) and
\[
\gamma(0)=\bigl(\sqrt{1-0},\,0\bigr)=(1,0)=p.
\]

\subsection*{Verifying \(\gamma'(0)=\vec v\)}
For any \(f\in C^\infty(S^1)\), the chain rule gives
\begin{align*}
	\left.\frac{d}{dt}\right|_{t=0}(f\circ\gamma)(t)
	&=
	\left.\frac{d}{dt}\right|_{t=0}
	\Bigl((f\circ\varphi^{-1})\bigl(\varphi(p)+t a\bigr)\Bigr)\\
	&=
	d(f\circ\varphi^{-1})_{\varphi(p)}(a).
\end{align*}
Substituting \(a=(d\varphi)_p(\vec v)\) and using functoriality of differentials,
\[
d(f\circ\varphi^{-1})_{\varphi(p)}\circ (d\varphi)_p
=
d\bigl((f\circ\varphi^{-1})\circ\varphi\bigr)_p
=
(df)_p,
\]
hence
\[
\left.\frac{d}{dt}\right|_{t=0}(f\circ\gamma)(t)
=
(df)_p(\vec v)
=
\vec v(f).
\]
By the defining characterization of the velocity vector \(\gamma'(0)\in T_pS^1\) as the derivation
\(f\mapsto \left.\frac{d}{dt}\right|_{t=0}(f\circ\gamma)(t)\), it follows that
\[
\gamma'(0)=\vec v.
\]
(Indeed, differentiating \(\gamma(t)=(\sqrt{1-t^2},t)\) gives \(\gamma'(0)=(0,1)\).)

\subsection*{Extending to a globally defined smooth curve}
Choose \(\beta\in C^\infty(\R)\) such that
\[
\beta(t)=t\ \text{for }|t|\text{ small},\qquad |\beta(t)|<\varepsilon\ \text{for all }t\in\R.
\]
For example, if \(\varepsilon=1\), take \(\beta(t)=\tanh(t)\).
Define
\[
\Gamma:\R\to S^1,\qquad
\Gamma(t)\coloneqq \varphi^{-1}\bigl(\varphi(p)+\beta(t)a\bigr)
=\varphi^{-1}(\beta(t))
=\bigl(\sqrt{1-\beta(t)^2},\,\beta(t)\bigr).
\]
Then \(\Gamma\in C^\infty(\R,S^1)\), \(\Gamma(0)=p\), and \(\Gamma'(0)=\vec v\), since \(\beta'(0)=1\).
\end{example}

\begin{tcolorbox}[title=\bfseries The Poincar\'{e} lemma]
\end{tcolorbox}

\newpage
\section*{The derivation associated to a tangent vector}

Let $M$ be a $C^\infty$ manifold and let $p\in M$.
Assume the tangent space $\TpM$ is defined as equivalence classes of $C^\infty$ curves through $p$:
two curves $\gamma_1,\gamma_2:(-\varepsilon,\varepsilon)\to M$ with $\gamma_i(0)=p$ are equivalent if,
for some (hence any) chart $(U,\varphi)$ with $p\in U$,
\[
\left.\frac{d}{dt}\right|_{t=0}\bigl(\varphi\circ\gamma_1\bigr)(t)
=
\left.\frac{d}{dt}\right|_{t=0}\bigl(\varphi\circ\gamma_2\bigr)(t)\in \R^n.
\]
Let $\vec v\in \TpM$, and choose a representative curve $\gamma:(-\varepsilon,\varepsilon)\to M$
with $\gamma(0)=p$ and $[\gamma]=\vec v$.

Define an operator
\[
X_{\vec v}:\C^\infty(M)\longrightarrow \R
\]
by
\begin{equation}\label{eq:def-Xv}
	X_{\vec v}(f)\;\coloneqq\;\left.\frac{d}{dt}\right|_{t=0}\bigl(f\circ\gamma\bigr)(t),
	\qquad f\in\C^\infty(M).
\end{equation}

\begin{proposition}[Well-definedness and basic properties of $X_{\vec v}$]
	\label{prop:Xv}
	With notation as above:
	\begin{enumerate}
		\item[\textup{(a)}] The value $X_{\vec v}(f)$ is independent of the choice of representative curve $\gamma$
		for $\vec v$. Hence $X_{\vec v}$ is well-defined.
		\item[\textup{(b)}] The operator $X_{\vec v}$ satisfies:
		\begin{enumerate}
			\item[\textup{(1)}] \textbf{Locality:} If $f,g\in \C^\infty(M)$ agree on some neighborhood of $p$, then
			$X_{\vec v}(f)=X_{\vec v}(g)$.
			\item[\textup{(2)}] \textbf{Linearity:} For all $a,b\in \R$ and $f,g\in \C^\infty(M)$,
			\[
			X_{\vec v}(af+bg)=aX_{\vec v}(f)+bX_{\vec v}(g).
			\]
			\item[\textup{(3)}] \textbf{Product rule:} For all $f,g\in \C^\infty(M)$,
			\[
			X_{\vec v}(f\cdot g)=X_{\vec v}(f)\,g(p)+f(p)\,X_{\vec v}(g).
			\]
		\end{enumerate}
	\end{enumerate}
\end{proposition}

\begin{proof}
	\textbf{(a) Well-definedness.}
	Let $\gamma_1,\gamma_2:(-\varepsilon,\varepsilon)\to M$ be two $C^\infty$ curves with
	$\gamma_1(0)=\gamma_2(0)=p$ representing the same tangent vector $\vec v$.
	By definition of the equivalence relation, there exists (equivalently, for every) chart $(U,\varphi)$
	about $p$ such that
	\begin{equation}\label{eq:equal-vel}
		\left.\frac{d}{dt}\right|_{t=0}\bigl(\varphi\circ\gamma_1\bigr)(t)
		=
		\left.\frac{d}{dt}\right|_{t=0}\bigl(\varphi\circ\gamma_2\bigr)(t)\in \R^n.
	\end{equation}
	Fix such a chart $(U,\varphi)$ and let $f\in \C^\infty(M)$ be arbitrary.
	Set $F\coloneqq f\circ \varphi^{-1}\in \C^\infty(\varphi(U))$.
	For $i=1,2$ and $|t|$ small we have $\gamma_i(t)\in U$ and hence
	\[
	(f\circ\gamma_i)(t)=(F\circ \varphi\circ\gamma_i)(t).
	\]
	Applying the chain rule in $\R^n$ yields
	\[
	\left.\frac{d}{dt}\right|_{t=0}(f\circ\gamma_i)(t)
	=
	dF_{\varphi(p)}\!\left(\left.\frac{d}{dt}\right|_{t=0}(\varphi\circ\gamma_i)(t)\right).
	\]
	Using \eqref{eq:equal-vel} for the velocities, we conclude
	\[
	\left.\frac{d}{dt}\right|_{t=0}(f\circ\gamma_1)(t)
	=
	\left.\frac{d}{dt}\right|_{t=0}(f\circ\gamma_2)(t).
	\]
	Thus the right-hand side of \eqref{eq:def-Xv} depends only on $\vec v$, not on the representative curve.
	Therefore $X_{\vec v}$ is well-defined.
	
	\medskip
	\textbf{(b1) Locality.}
	Assume $f=g$ on a neighborhood $W$ of $p$. Since $\gamma(0)=p$ and $\gamma$ is continuous,
	there exists $\delta>0$ such that $\gamma((-\delta,\delta))\subset W$.
	Hence for $|t|<\delta$ we have $(f\circ\gamma)(t)=(g\circ\gamma)(t)$, and therefore
	\[
	X_{\vec v}(f)=\left.\frac{d}{dt}\right|_{t=0}(f\circ\gamma)(t)
	=
	\left.\frac{d}{dt}\right|_{t=0}(g\circ\gamma)(t)
	=X_{\vec v}(g).
	\]
	
	\medskip
	\textbf{(b2) Linearity.}
	Let $a,b\in \R$ and $f,g\in \C^\infty(M)$. Then for all $t$,
	\[
	(af+bg)\circ\gamma
	=
	a(f\circ\gamma)+b(g\circ\gamma).
	\]
	Differentiate at $t=0$ and use linearity of the derivative in $\R$:
	\[
	X_{\vec v}(af+bg)
	=
	\left.\frac{d}{dt}\right|_{t=0}\bigl(a(f\circ\gamma)(t)+b(g\circ\gamma)(t)\bigr)
	=
	aX_{\vec v}(f)+bX_{\vec v}(g).
	\]
	
	\medskip
	\textbf{(b3) Product rule.}
	For all $t$,
	\[
	(fg)\circ\gamma=(f\circ\gamma)\,(g\circ\gamma),
	\]
	so by the ordinary product rule for functions of one real variable,
	\begin{align*}
		X_{\vec v}(fg)
		&=
		\left.\frac{d}{dt}\right|_{t=0}\bigl((f\circ\gamma)(t)\,(g\circ\gamma)(t)\bigr) \\
		&=
		\left.\frac{d}{dt}\right|_{t=0}(f\circ\gamma)(t)\cdot (g\circ\gamma)(0)
		+
		(f\circ\gamma)(0)\cdot \left.\frac{d}{dt}\right|_{t=0}(g\circ\gamma)(t).
	\end{align*}
	Since $(g\circ\gamma)(0)=g(\gamma(0))=g(p)$ and $(f\circ\gamma)(0)=f(p)$, this becomes
	\[
	X_{\vec v}(fg)=X_{\vec v}(f)\,g(p)+f(p)\,X_{\vec v}(g).
	\]
	This proves (1)--(3).
\end{proof}


\section*{A concrete example of $X_{\vec v}$ on the unit circle}

Let
\[
M=S^1=\{(x,y)\in\R^2:\ x^2+y^2=1\}.
\]
Fix the point
\[
p=(1,0)\in S^1,
\]
and the tangent vector
\[
\vec v=(0,1)\in T_pS^1=\{(0,\lambda):\lambda\in\R\}.
\]
A natural smooth curve representing $\vec v$ is the counterclockwise parametrization
\[
\gamma(t)=(\cos t,\sin t),\qquad t\in\R.
\]
Indeed,
\[
\gamma(0)=(1,0)=p,
\qquad
\gamma'(0)=(-\sin 0,\cos 0)=(0,1)=\vec v.
\]

\subsection*{Definition of $X_{\vec v}$ in this example}
For any $f\in C^\infty(S^1)$ define
\[
X_{\vec v}(f)\coloneqq \left.\frac{d}{dt}\right|_{t=0} (f\circ\gamma)(t).
\]
Thus, in this concrete situation, $X_{\vec v}(f)$ is literally the directional derivative of $f$
along $S^1$ at $p$ in the counterclockwise direction.

\subsection*{Concrete computations}
Let $x,y:S^1\to\R$ be the coordinate restriction functions $x(x,y)=x$, $y(x,y)=y$.
Then
\[
(x\circ\gamma)(t)=\cos t,\qquad (y\circ\gamma)(t)=\sin t,
\]
so
\[
X_{\vec v}(x)=\left.\frac{d}{dt}\right|_{0}\cos t=0,
\qquad
X_{\vec v}(y)=\left.\frac{d}{dt}\right|_{0}\sin t=1.
\]

Now define two smooth functions on $S^1$ by
\[
f(x,y)\coloneqq y,\qquad g(x,y)\coloneqq x\,y.
\]
Then
\[
(f\circ\gamma)(t)=\sin t,\qquad (g\circ\gamma)(t)=(\cos t)(\sin t),
\]
so
\[
X_{\vec v}(f)=\left.\frac{d}{dt}\right|_{0}\sin t=1,
\qquad
X_{\vec v}(g)=\left.\frac{d}{dt}\right|_{0}(\cos t\sin t)
=\left.\frac{d}{dt}\right|_{0}\frac{1}{2}\sin(2t)=1.
\]
Also
\[
f(p)=f(1,0)=0,\qquad g(p)=g(1,0)=0.
\]

\subsection*{Verification of the three properties in this example}

\paragraph{(1) Locality.}
Let $f_1,f_2\in C^\infty(S^1)$ and suppose there exists an open arc $W\subset S^1$ containing $p$
such that $f_1|_{W}=f_2|_{W}$.
Since $\gamma(0)=p$ and $\gamma$ is continuous, there exists $\delta>0$ such that
$\gamma((-\delta,\delta))\subset W$.
Hence $(f_1\circ\gamma)(t)=(f_2\circ\gamma)(t)$ for $|t|<\delta$, and therefore
\[
X_{\vec v}(f_1)=\left.\frac{d}{dt}\right|_{0}(f_1\circ\gamma)(t)
=
\left.\frac{d}{dt}\right|_{0}(f_2\circ\gamma)(t)=X_{\vec v}(f_2).
\]

\paragraph{(2) Linearity.}
For $a,b\in\R$ and $f_1,f_2\in C^\infty(S^1)$,
\[
((a f_1+b f_2)\circ\gamma)(t)=a(f_1\circ\gamma)(t)+b(f_2\circ\gamma)(t),
\]
so differentiating at $0$ yields
\[
X_{\vec v}(a f_1+b f_2)=aX_{\vec v}(f_1)+bX_{\vec v}(f_2).
\]

\paragraph{(3) Product rule (verified by explicit functions).}
Take $f(x,y)=y$ and $g(x,y)=xy$ as above. Then
\[
X_{\vec v}(fg)=X_{\vec v}(y\cdot xy)=X_{\vec v}(x y^2).
\]
Compute:
\[
(fg)\circ\gamma(t)=(\sin t)\cdot(\cos t\sin t)=\cos t\,\sin^2 t,
\]
hence
\[
X_{\vec v}(fg)=\left.\frac{d}{dt}\right|_{0}\bigl(\cos t\,\sin^2 t\bigr)=0.
\]
On the other hand,
\[
X_{\vec v}(f)\,g(p)+f(p)\,X_{\vec v}(g)= (1)\cdot 0 + 0\cdot (1)=0,
\]
so indeed
\[
X_{\vec v}(fg)=X_{\vec v}(f)\,g(p)+f(p)\,X_{\vec v}(g).
\]
(The general product rule follows from the one-variable product rule for $(f\circ\gamma)(t)$ and
$(g\circ\gamma)(t)$.)


\newpage
\begin{tcolorbox}[title=\bfseries The Poincar\'{e} lemma]
\end{tcolorbox}

\begin{tcolorbox}[title=\bfseries The Poincar\'{e} lemma]
\end{tcolorbox}

\begin{tcolorbox}[title=\bfseries The Poincar\'{e} lemma]
\end{tcolorbox}

\begin{tcolorbox}[title=\bfseries The Poincar\'{e} lemma]
\end{tcolorbox}

\begin{tcolorbox}[title=\bfseries The Poincar\'{e} lemma]
\end{tcolorbox}

	
\end{document}
