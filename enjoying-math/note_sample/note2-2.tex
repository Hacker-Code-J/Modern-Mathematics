\documentclass[11pt]{article}
\usepackage[margin=1in]{geometry}
\usepackage{amsmath,amssymb,amsthm}
\usepackage{mathtools}

\newcommand{\C}{\mathbb{C}}
\newcommand{\CP}{\mathbb{CP}}
\newcommand{\M}{\mathcal{M}}
\newcommand{\Res}{\operatorname{Res}}
\newcommand{\Divv}{\operatorname{Div}}
\newcommand{\degd}{\operatorname{deg}}

\begin{document}
	
	\title{Extremely Detailed Explanation of $\M(\CP^1)\cong\C(z)$ and\\
		Why $\M(X)\cong\C(z)\iff X\cong\CP^1$}
	\author{}
	\date{}
	\maketitle
	
	\tableofcontents
	
	\bigskip
	
	%%%%%%%%%%%%%%%%%%%%%%%%%%%%%%%%%%%%%%%%%%%%%%%%%%%%%%%%%%%%
	\section{Basic objects and philosophy}
	
	We work over $\C$. There are three levels of structure:
	
	\begin{itemize}
		\item \textbf{Complex-analytic / calculus}: holomorphic and meromorphic
		functions, Laurent series, residues, contour integrals.
		\item \textbf{Algebraic}: rational functions $p(z)/q(z)$, projective
		coordinates, divisors, function fields.
		\item \textbf{Riemann surface theory}: compact Riemann surfaces,
		meromorphic maps $X\to\CP^1$, degree, genus.
	\end{itemize}
	
	Our main goals:
	
	\begin{enumerate}
		\item Show in detail:
		\[
		\M(\CP^1) \cong \C(z),
		\]
		i.e.\ every meromorphic function on the Riemann sphere is a rational
		function in one variable.
		
		\item Show:
		\[
		\M(X)\cong\C(z) \quad\Longleftrightarrow\quad X\cong\CP^1
		\]
		for a compact Riemann surface $X$.
	\end{enumerate}
	
	%%%%%%%%%%%%%%%%%%%%%%%%%%%%%%%%%%%%%%%%%%%%%%%%%%%%%%%%%%%%
	\section{Analytic / calculus side: $\M(\CP^1)=\C(z)$}
	
	\subsection{Identifying $\CP^1$ with the Riemann sphere}
	
	As a set,
	\[
	\CP^1 = \{[z_0:z_1]\neq [0:0]\}/\sim, \qquad
	[z_0:z_1]\sim[\lambda z_0:\lambda z_1],\ \lambda\neq 0.
	\]
	
	Analytically, we identify
	\[
	\CP^1 \cong \widehat{\C} := \C\cup\{\infty\}
	\]
	as follows:
	
	\begin{itemize}
		\item Affine chart $U_1 = \{[z_0:z_1]\mid z_1\neq 0\}$ with coordinate
		\[
		z = \frac{z_0}{z_1}: U_1\to\C.
		\]
		\item The point at infinity $[1:0]$ corresponds to the symbol $\infty$.
	\end{itemize}
	
	So we think of $\CP^1$ as the complex plane plus one extra point at infinity.
	
	\subsection{Meromorphic functions and meromorphic $1$-forms}
	
	A \emph{meromorphic function} on $\CP^1$ is a function
	\[
	f:\CP^1\to\CP^1
	\]
	that is holomorphic except possibly at isolated points where it can have
	poles (but NO essential singularities).
	
	On $\C\subset\CP^1$ we can write $f$ as an ordinary meromorphic function
	$f(z)$ (holomorphic except for isolated poles). At $\infty$ we use
	coordinate $w=1/z$; then
	\[
	F(w) := f\!\left(\frac{1}{w}\right)
	\]
	is meromorphic on a deleted neighborhood of $w=0$ and has a pole or removable
	singularity there.
	
	We attach to $f$ the meromorphic $1$-form
	\[
	\omega = f(z)\,dz.
	\]
	Integrals of $\omega$ around closed curves encode information about $f$
	(residues, principal parts).
	
	\subsection{Simple explicit example to see everything}
	
	Consider
	\[
	f(z) = \frac{1}{z^2(z-1)}
	\]
	as a meromorphic function on $\widehat{\C}=\C\cup\{\infty\}$.
	
	\subsubsection{Poles and orders (calculus style)}
	
	\paragraph{At $z=0$.}
	Clearly $z=0$ is a pole because the denominator has $z^2$. Write
	\[
	f(z) = \frac{1}{z^2(z-1)}.
	\]
	Near $z=0$, $(z-1)\approx -1$, so
	\[
	f(z) \sim \frac{1}{z^2\cdot(-1)} = -\frac{1}{z^2}.
	\]
	Thus $f$ has a pole of order $2$ at $z=0$.
	
	\paragraph{At $z=1$.}
	Set $u=z-1$. Then $z=u+1$, so
	\[
	f(z) = \frac{1}{(u+1)^2 u}.
	\]
	Near $u=0$, $(u+1)^2\approx 1$, so
	\[
	f(z) \sim \frac{1}{u} = \frac{1}{z-1}.
	\]
	So $z=1$ is a simple pole (order $1$).
	
	\paragraph{At $\infty$.}
	We analyze $z\to\infty$. As $|z|\to\infty$,
	\[
	f(z) = \frac{1}{z^2(z-1)} = \frac{1}{z^3(1-1/z)}.
	\]
	Expand $\frac{1}{1-1/z} = 1 + \frac{1}{z} + \frac{1}{z^2} + \dots$, so
	\[
	f(z) = \frac{1}{z^3}\Bigl(1 + \frac{1}{z} + \cdots\Bigr)
	= \frac{1}{z^3} + O\!\Bigl(\frac{1}{z^4}\Bigr).
	\]
	Thus $f(z)\to 0$ as $z\to\infty$, and in fact has a zero of order $3$ at
	$\infty$; equivalently, it's holomorphic at $\infty$ and $f(\infty)=0$.
	
	\subsubsection{Partial fraction decomposition and derivative}
	
	We try to write
	\[
	f(z) = \frac{1}{z^2(z-1)} = \frac{A}{z} + \frac{B}{z^2} + \frac{C}{z-1}.
	\]
	Multiply by $z^2(z-1)$:
	\[
	1 = A z(z-1) + B(z-1) + Cz^2.
	\]
	
	Compute:
	\[
	A z(z-1) = A(z^2 - z) = A z^2 - A z,
	\]
	\[
	B(z-1) = Bz - B,
	\]
	\[
	Cz^2 = Cz^2.
	\]
	So
	\[
	A z^2 - A z + Bz - B + Cz^2
	= (A+C)z^2 + (-A+B)z - B.
	\]
	We want this equal to $1$ for all $z$, meaning:
	\[
	(A+C)z^2 + (-A+B)z - B
	= 0\cdot z^2 + 0\cdot z + 1.
	\]
	So we must have
	\[
	A + C = 0,\quad -A + B = 0,\quad -B = 1.
	\]
	From $-B=1$ we get $B=-1$. Then $-A + (-1)=0\Rightarrow A=-1$. Then
	$A+C=0\Rightarrow C=1$.
	
	So
	\[
	f(z) =
	\frac{-1}{z} + \frac{-1}{z^2} + \frac{1}{z-1}.
	\]
	
	Now compute the derivative:
	\[
	f'(z)
	= \frac{d}{dz}\left(-\frac{1}{z} - \frac{1}{z^2} + \frac{1}{z-1}\right)
	= \frac{1}{z^2} + \frac{2}{z^3} - \frac{1}{(z-1)^2}.
	\]
	Then
	\[
	df = f'(z)\,dz
	= \left(\frac{1}{z^2} + \frac{2}{z^3} - \frac{1}{(z-1)^2}\right)\,dz.
	\]
	
	Poles of $df$:
	\begin{itemize}
		\item At $z=0$: order $3$ pole (from $2/z^3$ and $1/z^2$), residue $0$
		(no $1/z$ term).
		\item At $z=1$: order $2$ pole, residue $0$.
		\item At $\infty$: since $f$ is holomorphic at $\infty$, $df$ is also
		holomorphic there and residue $0$.
	\end{itemize}
	
	This matches the general fact: $df$ always has \emph{total residue $0$}.
	
	\subsection{General analytic proof that $\M(\CP^1)=\C(z)$}
	
	Now let $f$ be \emph{any} meromorphic function on $\widehat{\C}$.
	
	\subsubsection{Step 1: finitely many poles}
	
	Because $\CP^1$ is compact and poles are isolated, $f$ has only finitely many
	poles:
	\[
	\{a_1,\dots,a_N\} \subset \C\cup\{\infty\}.
	\]
	
	\subsubsection{Step 2: Laurent expansions at each pole}
	
	At each finite pole $a_j\in\C$, there is a small circle $\gamma_j$ around $a_j$
	enclosing no other poles, and a Laurent expansion
	\[
	f(z) = \sum_{n=-m_j}^{\infty} c_{j,n}(z-a_j)^n, \quad 0<|z-a_j|<\varepsilon,
	\]
	with $m_j\ge 1$. Coefficients are given by integrals:
	\[
	c_{j,n} = \frac{1}{2\pi i} \oint_{\gamma_j}
	\frac{f(\zeta)}{(\zeta - a_j)^{n+1}}\,d\zeta.
	\]
	
	The principal part at $a_j$ is
	\[
	\operatorname{PP}_{a_j}(f)(z)
	:= \sum_{n=-m_j}^{-1} c_{j,n}(z-a_j)^n.
	\]
	
	At $\infty$, in coordinate $w=1/z$:
	\[
	F(w) := f\!\left(\frac{1}{w}\right)
	= \sum_{n=-M}^{\infty} b_n w^n
	\]
	for some $M\ge 0$, and
	\[
	b_n = \frac{1}{2\pi i}\oint_{|\xi|=\rho}
	\frac{F(\xi)}{\xi^{n+1}}\,d\xi.
	\]
	The principal part at $\infty$ is
	\[
	\operatorname{PP}_\infty(f)(w)
	:= \sum_{n=-M}^{-1} b_n w^n.
	\]
	In terms of $z=1/w$, this is a polynomial $P(z)$.
	
	\subsubsection{Step 3: Build a rational function $R(z)$}
	
	Define
	\[
	R(z) := P(z) + \sum_{j=1}^N \operatorname{PP}_{a_j}(f)(z).
	\]
	
	Concretely,
	\[
	R(z) = \sum_{k=1}^M \tilde b_k z^k
	+ \sum_{j=1}^N\sum_{k=1}^{m_j} \frac{c_{j,-k}}{(z-a_j)^k}.
	\]
	
	It is clear that $R(z)$ is a rational function in $z$, i.e.\ belongs to
	$\C(z)$.
	
	By construction:
	\begin{itemize}
		\item At each finite pole $a_j$, the principal part of $R$ coincides with
		that of $f$.
		\item At $\infty$, the principal part of $R$ coincides with that of $f$.
	\end{itemize}
	
	\subsubsection{Step 4: The difference $g=f-R$ is holomorphic everywhere}
	
	Set
	\[
	g(z) := f(z) - R(z).
	\]
	At each finite pole $a_j$, the negative-power terms in the Laurent expansion
	cancel, so $g$ has no pole there (holomorphic at $a_j$).
	
	At $\infty$, using $w=1/z$, $f$ and $R$ have the same principal part in $w$,
	so $g$ has no negative powers in $w$ and is holomorphic at $w=0$
	(i.e.\ at $\infty$).
	
	Therefore $g$ is holomorphic on all of $\CP^1=\widehat{\C}$. A holomorphic
	function on a compact Riemann surface is constant (by the maximum modulus
	principle or Liouville), so $g(z)\equiv C$ for some $C\in\C$.
	
	Thus
	\[
	f(z) = R(z) + C.
	\]
	
	Since $R(z)\in\C(z)$, we conclude
	\[
	f(z)\in \C(z).
	\]
	
	\begin{center}
		\fbox{\(\displaystyle
			\M(\CP^1) = \C(z).
			\)}
	\end{center}
	
	This is the \emph{calculus} argument: it uses Laurent series, integrals for
	coefficients, residues, and Liouville.
	
	%%%%%%%%%%%%%%%%%%%%%%%%%%%%%%%%%%%%%%%%%%%%%%%%%%%%%%%%%%%%
	\section{Algebraic / projective viewpoint on $\M(\CP^1)$}
	
	Now we give a more algebraic description.
	
	\subsection{Affine chart and coordinate function $z$}
	
	On the chart
	\[
	U_1 = \{[z_0:z_1]\in\CP^1\mid z_1\neq 0\},
	\]
	we define
	\[
	z = \frac{z_0}{z_1}:U_1\to\C.
	\]
	
	This $z$ is a holomorphic function on $U_1$. On $\CP^1$ it extends
	\emph{meromorphically} with a single simple pole at the point $[1:0]$
	(i.e.\ $\infty$).
	
	The function $z$ generates the field $\C(z)$ of rational functions.
	
	\subsection{Rational functions as homogeneous maps}
	
	Let $R(z) = p(z)/q(z)$ with $p,q\in\C[z]$, $q\not\equiv 0$. Set
	\[
	m = \max\{\deg p,\deg q\}.
	\]
	Define homogeneous polynomials:
	\[
	P(z_0,z_1) = z_1^m p\!\left(\frac{z_0}{z_1}\right),\quad
	Q(z_0,z_1) = z_1^m q\!\left(\frac{z_0}{z_1}\right).
	\]
	Then $P,Q$ are homogeneous of degree $m$, and we define a map
	\[
	F_R:\CP^1\to\CP^1,\qquad
	F_R([z_0:z_1]) =
	\begin{cases}
		[P(z_0,z_1):Q(z_0,z_1)], & Q(z_0,z_1)\neq 0,\\[4pt]
		[1:0], & Q(z_0,z_1)=0.
	\end{cases}
	\]
	
	This is well-defined: scaling $(z_0,z_1)$ by $\lambda\neq 0$ scales both
	$P$ and $Q$ by $\lambda^m$, so the projective point $[P:Q]$ is the same.
	
	On the affine chart $U_1$, with $z=z_0/z_1$, we have
	\[
	F_R([z:1]) = [p(z):q(z)],
	\]
	and in the chart where $q(z)\neq 0$, this corresponds to
	\[
	\frac{p(z)}{q(z)} = R(z).
	\]
	
	So any rational function $R(z)$ yields a meromorphic map $F_R:\CP^1\to\CP^1$.
	
	\subsection{Conversely: maps $\CP^1\to\CP^1$ are rational}
	
	Conversely, any holomorphic map $F:\CP^1\to\CP^1$ between projective lines is
	given by homogeneous polynomials of the same degree:
	\[
	F([z_0:z_1]) = [P(z_0,z_1):Q(z_0,z_1)],
	\]
	where $P,Q$ are homogeneous of the same degree and have no common factor.
	Restricting to $U_1$ with $z=z_0/z_1$,
	\[
	F([z:1]) = [P(z,1):Q(z,1)];
	\]
	if $Q(z,1)\neq 0$ then in the affine chart we get
	\[
	F([z:1]) \mapsto \frac{P(z,1)}{Q(z,1)} \in \C,
	\]
	which is a rational function of $z$.
	
	Therefore, algebraically,
	\[
	\M(\CP^1)
	= \{\text{meromorphic maps }\CP^1\to\CP^1\}
	\cong \C(z).
	\]
	
	%%%%%%%%%%%%%%%%%%%%%%%%%%%%%%%%%%%%%%%%%%%%%%%%%%%%%%%%%%%%
	\section{General compact Riemann surface $X$ and $\M(X)$}
	
	Now consider an arbitrary compact Riemann surface $X$.
	
	\subsection{Non-constant meromorphic functions and maps $X\to\CP^1$}
	
	A non-constant meromorphic function $f$ on $X$ corresponds to a holomorphic map
	\[
	f:X\to\CP^1
	\]
	defined by
	\[
	f(p) =
	\begin{cases}
		[f(p):1], & f(p)\text{ finite},\\[4pt]
		[1:0], & f(p)=\infty.
	\end{cases}
	\]
	
	This map is \emph{finite-to-one}: for a generic point $w\in\CP^1$, the preimage
	$f^{-1}(w)$ consists of finitely many points, counted with multiplicity. The
	number of points in a generic fiber is the \emph{degree} of $f$, denoted
	$\degd(f)$.
	
	Locally, in coordinates, near a point $p\in X$, we can choose $z$ as a local
	coordinate on $X$ at $p$ and $\zeta$ as a local coordinate on $\CP^1$ at
	$f(p)$ so that
	\[
	\zeta = f(z) = z^k + (\text{higher order terms}),
	\]
	with $k\ge 1$. The integer $k$ is the local degree (ramification index) at $p$.
	Summing these local degrees over $f^{-1}(w)$ for generic $w$ gives $\degd(f)$.
	
	\subsection{Function field extension viewpoint}
	
	The map $f:X\to\CP^1$ induces a field embedding
	\[
	f^*:\M(\CP^1)\hookrightarrow \M(X),
	\]
	by pullback: if $R(z)\in\M(\CP^1)\cong\C(z)$, then
	\[
	f^*(R) := R\circ f \in \M(X).
	\]
	
	Thus we get an inclusion
	\[
	\C(z)\hookrightarrow \M(X).
	\]
	
	A deep theorem (from the theory of compact Riemann surfaces / algebraic
	curves) says:
	\[
	[\M(X):\C(z)] = \degd(f),
	\]
	i.e.\ the degree of the field extension equals the topological degree of the
	map $f$. Intuitively, each branch contributes one ``copy'' of $\C(z)$ in the
	extension.
	
	\subsection{If $\M(X)\cong\C(z)$, then $X\cong\CP^1$}
	
	Suppose we have \emph{as fields}:
	\[
	\M(X)\cong\C(z).
	\]
	
	This means that $\M(X)$ is a purely transcendental extension of $\C$ of
	transcendence degree $1$, with no nontrivial algebraic relations besides those
	already in $\C(z)$.
	
	Pick any non-constant meromorphic function $f\in\M(X)$. The map
	\[
	f:X\to\CP^1
	\]
	is non-constant, hence of some degree $d\ge 1$. The induced extension
	\[
	\C(z)\hookrightarrow \M(X)
	\]
	has degree $d$. But by assumption $\M(X)\cong\C(z)$, so as a field extension,
	$[\M(X):\C(z)]=1$. Therefore, $\degd(f)=1$.
	
	So we have a non-constant map $f:X\to\CP^1$ of degree $1$.
	
	\subsubsection*{Degree $1$ implies biholomorphism}
	
	We claim a holomorphic map $f:X\to\CP^1$ of degree $1$ between compact Riemann
	surfaces must be an isomorphism of Riemann surfaces.
	
	\begin{itemize}
		\item Since $f$ is non-constant holomorphic, it is open and its image is an
		open connected subset of $\CP^1$.
		\item Compactness of $X$ plus continuity of $f$ implies $f(X)$ is compact,
		hence closed in $\CP^1$.
		\item $\CP^1$ is connected, so the only nonempty closed and open subset is
		all of $\CP^1$. Thus $f$ is surjective.
		\item $\degd(f)=1$ means that for a generic point $w\in\CP^1$, the fiber
		$f^{-1}(w)$ consists of exactly one point (counted with multiplicity).
		Roughly speaking, this means $f$ is ``one-to-one almost everywhere''.
		\item One can show (using local behavior and that there is no branching if
		the total degree is $1$) that $f$ is globally one-to-one.
		\item A bijective holomorphic map between compact Riemann surfaces has a
		holomorphic inverse (since the inverse map is continuous, and by
		Riemann surface theory it is analytic). So $f$ is a biholomorphism.
	\end{itemize}
	
	Therefore $X\cong\CP^1$ as Riemann surfaces.
	
	\begin{center}
		\fbox{\(
			\M(X)\cong\C(z) \;\Longrightarrow\; X\cong\CP^1.
			\)}
	\end{center}
	
	\subsection{Conversely, if $X\cong\CP^1$, then $\M(X)\cong\C(z)$}
	
	Conversely, if we know $X\cong\CP^1$ (as Riemann surfaces), then by definition
	there is a biholomorphism $\varphi:X\to\CP^1$. Pullback of meromorphic
	functions along $\varphi$ gives a field isomorphism
	\[
	\M(\CP^1) \xrightarrow{\ \cong\ } \M(X),
	\]
	and we already know $\M(\CP^1)\cong\C(z)$. Hence
	\[
	\M(X) \cong \C(z).
	\]
	
	\subsection{Genus and the ``only genus 0 curve'' viewpoint}
	
	There is a more geometric / topological characterization.
	
	\begin{itemize}
		\item The \emph{genus} $g(X)$ is the number of ``holes'' of $X$ as a
		topological surface: $g(\CP^1)=0$, $g(\C/\Lambda)=1$, etc.
		\item Analytically, $g(X) = \dim_{\C} H^0(X,\Omega^1_X)$, the dimension of
		the space of holomorphic $1$-forms on $X$.
		\item One can show: $g(X)=0$ if and only if $X\cong\CP^1$.
		\item For such a genus-$0$ surface $X$, every meromorphic function behaves
		like a rational function in some coordinate, so $\M(X)\cong\C(z)$.
	\end{itemize}
	
	Thus we have an equivalence of three properties:
	
	\[
	X\cong\CP^1
	\quad\Longleftrightarrow\quad
	g(X)=0
	\quad\Longleftrightarrow\quad
	\M(X)\cong\C(z).
	\]
	
	%%%%%%%%%%%%%%%%%%%%%%%%%%%%%%%%%%%%%%%%%%%%%%%%%%%%%%%%%%%%
	\section{Final summary}
	
	\begin{itemize}
		\item \textbf{Calculus viewpoint:} On the Riemann sphere, any meromorphic
		function $f$ has finitely many poles. Around each pole we can write a
		Laurent series, whose principal part coefficients are given by integrals
		of the $1$-form $f(z)\,dz$. Using these principal parts, we build a
		rational function $R(z)$ with the same local behavior at all poles (finite
		and infinity). The difference $f-R$ is holomorphic everywhere on the
		compact sphere, hence constant. Therefore $f$ is rational, and
		\[
		\M(\CP^1) = \C(z).
		\]
		
		\item \textbf{Algebraic / projective viewpoint:} The affine coordinate
		$z=z_0/z_1$ on $\CP^1$ is a meromorphic function with a simple pole at
		infinity. Rational functions $R(z)=p(z)/q(z)$ can be expressed via
		homogeneous polynomials $P,Q$ on $\CP^1$, giving meromorphic maps
		$F_R:\CP^1\to\CP^1$. Conversely, any meromorphic map between projective
		lines arises this way. Thus $\M(\CP^1)\cong\C(z)$ as fields.
		
		\item \textbf{Characterization via function fields:} For any compact Riemann
		surface $X$, non-constant meromorphic functions $f:X\to\CP^1$ induce field
		embeddings $\C(z)\hookrightarrow\M(X)$ and finite extensions of fields.
		If $\M(X)\cong\C(z)$, the extension has degree $1$, so $f$ has degree $1$,
		hence is a biholomorphism $X\cong\CP^1$. Conversely, if $X\cong\CP^1$,
		then obviously $\M(X)\cong\C(z)$.
	\end{itemize}
	
	Thus both analytically and algebraically, \emph{the Riemann sphere is the
		unique compact Riemann surface whose function field is $\C(z)$}. 
	
	
	
	\newpage
	\section{Why we use $\CP^1$ (not $\C$) for meromorphic functions}
	
	In this section, $X$ denotes a compact Riemann surface.
	
	\subsection{Holomorphic maps $X\to\C$ are constant}
	
%	\begin{lemma}\label{lem:compact_to_C_constant}
		Let $X$ be a compact Riemann surface and $f:X\to\C$ a holomorphic map.
		Then $f$ is constant.
%	\end{lemma}
	
	\begin{proof}
		Since $f$ is continuous and $X$ is compact, the image $f(X)\subset\C$ is
		compact in $\C$. In particular, $f(X)$ is bounded.
		
		But a bounded holomorphic function on a connected open subset of $\C$ must
		be constant by Liouville's theorem. To connect this with $f$, proceed as
		follows.
		
		Let $p\in X$. Choose a local coordinate chart
		\[
		\varphi:U\subset X \longrightarrow V\subset\C,
		\]
		with $p\in U$ and $\varphi(p)=0$. In this chart, the restriction of $f$ to
		$U$ looks like a holomorphic function
		\[
		f\circ\varphi^{-1} : V \to \C
		\]
		which is bounded by the global boundedness of $f$. By Liouville's theorem
		on $V$ (considering extensions to entire functions or using the maximum
		modulus principle locally), this local holomorphic function must be constant
		on $V$.
		
		Since $X$ is connected and covered by such coordinate charts, we conclude
		$f$ is constant on $X$.
	\end{proof}
	
%	\begin{corollary}
		There is no nonconstant holomorphic map $f:\CP^1\to\C$.
%	\end{corollary}
	
	\begin{proof}
		The Riemann sphere $\CP^1$ is compact, and the previous lemma applies.
	\end{proof}
	
	\subsection{Meromorphic functions and maps to $\CP^1$}
	
	Recall that a point of $\CP^1$ can be written as $[z_0:z_1]$. We have the
	standard affine chart
	\[
	i:\C \hookrightarrow \CP^1,\qquad
	i(z) = [z:1],
	\]
	and the point at infinity
	\[
	\infty := [1:0]\in \CP^1.
	\]
	
%	\begin{definition}
		A \emph{meromorphic function} on $X$ is a holomorphic map
		\[
		F:X\to\CP^1.
		\]
		Points $p\in X$ with $F(p)\neq\infty$ are called \emph{finite values}, and
		points $p$ with $F(p)=\infty$ are called \emph{poles} of $F$.
%	\end{definition}
	
	On the open set $U:=F^{-1}(\CP^1\setminus\{\infty\})$, we can compose with
	the inverse of $i$ to obtain an honest holomorphic function
	\[
	f := i^{-1}\circ F : U \to \C,
	\]
	which is the usual local expression of a meromorphic function.
	
	\subsection{Why $F$ usually cannot factor via $i:\C\to\CP^1$}
	
	You might try to define meromorphic functions as compositions
	\[
	F = i\circ f,\qquad f:X\to\C,
	\]
	with $f$ holomorphic. We now show why this only produces \emph{constant}
	functions when $X$ is compact.
	
%	\begin{proposition}\label{prop:factor_through_C}
		Let $X$ be a compact Riemann surface and let
		\[
		F:X\to\CP^1
		\]
		be a nonconstant holomorphic map (so $F$ is a nonconstant meromorphic
		function on $X$). Then $F$ \emph{does not} factor through the inclusion
		$i:\C\hookrightarrow\CP^1$; i.e.\ there is no holomorphic
		$f:X\to\C$ such that
		\[
		F = i\circ f.
		\]
%	\end{proposition}
	
	\begin{proof}
		Suppose, for contradiction, that such an $f$ exists:
		\[
		F = i\circ f,\quad f:X\to\C\ \text{holomorphic}.
		\]
		Since $X$ is compact, Lemma~\ref{lem:compact_to_C_constant} implies that
		$f$ is constant, say $f\equiv c\in\C$.
		
		Then
		\[
		F(p) = i(f(p)) = i(c) = [c:1]
		\]
		for all $p\in X$, so $F$ is constant as a map $X\to\CP^1$. This contradicts
		the assumption that $F$ is nonconstant.
		
		Hence no nonconstant holomorphic map $F:X\to\CP^1$ can factor through
		$i:\C\hookrightarrow\CP^1$.
	\end{proof}
	
	In other words, if we tried to define
	\[
	\M(X) \stackrel{?}{=} \{\,i\circ f \mid f:X\to\C\ \text{holomorphic}\,\}
	\]
	for a compact Riemann surface $X$, then by the Lemma and Proposition above the
	right-hand side would consist \emph{only of constants}. This would lose all
	interesting meromorphic functions.
	
	This is why the correct, global definition of meromorphic function on a
	compact Riemann surface $X$ necessarily uses $\CP^1$ as the target.
	
	\subsection{Correct equivalence: meromorphic functions $=$ maps to $\CP^1$}
	
	We now record the standard equivalence in precise form.
	
%	\begin{proposition}\label{prop:mero_holo_CP1}
		Let $X$ be a Riemann surface. Then:
		\begin{enumerate}
			\item If $F:X\to\CP^1$ is holomorphic, then $F$ is a meromorphic
			function on $X$ in the usual sense (holomorphic except at isolated
			poles).
			\item Conversely, if $f$ is meromorphic on $X$ in the usual sense, there
			exists a unique holomorphic map $F:X\to\CP^1$ such that
			\[
			F(p) =
			\begin{cases}
				[f(p):1], & \text{$p$ not a pole of $f$},\\[4pt]
				[1:0], & \text{$p$ a pole of $f$}.
			\end{cases}
			\]
		\end{enumerate}
		Thus there is a natural one-to-one correspondence:
		\[
		\M(X)
		\;\cong\;
		\{\,F:X\to\CP^1 \mid F \text{ holomorphic}\,\}.
		\]
%	\end{proposition}
	
	\begin{proof}
		(Sketch of (1)) Let $F:X\to\CP^1$ be holomorphic. For a point $p\in X$ with
		$F(p)\neq[1:0]$ (finite value), choose the affine chart
		\[
		\phi_1:\CP^1\setminus\{[1:0]\}\to\C,\quad [z_0:z_1]\mapsto z_0/z_1,
		\]
		and set $f = \phi_1\circ F$ on a neighborhood of $p$. Then $f$ is holomorphic
		there. If $F(p)=[1:0]$ (infinite value), use the other chart
		\[
		\phi_0:\CP^1\setminus\{[0:0:1]\}\to\C,\quad [z_0:z_1]\mapsto z_1/z_0,
		\]
		and check that in this chart $F$ has a pole. Hence $F$ is meromorphic in the
		usual sense.
		
		(Sketch of (2)) Conversely, if $f$ is meromorphic on $X$, then on the open
		set where $f$ is finite, define
		\[
		F(p) = [f(p):1],
		\]
		and at poles set $F(p)=[1:0]$. Using local coordinates near a pole, one checks
		that this $F$ is holomorphic in a neighborhood of each point of $X$. This
		gives a holomorphic map $F:X\to\CP^1$. Uniqueness is clear from the defining
		formula.
	\end{proof}
	
	\noindent
	So the right way to think is:
	
	\begin{center}
		\emph{A meromorphic function on $X$ is a holomorphic map $X\to\CP^1$, not a
			map $X\to\C$ composed with the inclusion $\C\hookrightarrow\CP^1$.}
	\end{center}

\newpage
\section*{The Riemann sphere as a conic in $\CP^2$}

Consider $\CP^1$ as the Riemann sphere with affine coordinate $z$ on the chart
$[z:1]\in\CP^1\setminus\{\infty\}$. On this affine chart we have holomorphic
functions
\[
f(z) = z,\qquad g(z) = z^2,
\]
which satisfy the polynomial relation
\[
P(x,y) := y - x^2,\qquad P(f(z),g(z)) = g(z)-f(z)^2 = 0.
\]
Thus the map
\[
\phi:\C \longrightarrow \C^2,\qquad z\longmapsto (x,y)=(f(z),g(z))=(z,z^2)
\]
has image contained in the affine algebraic curve
\[
C_{\mathrm{aff}} := \{(x,y)\in\C^2 \mid y-x^2=0\},
\]
the parabola $y=x^2$. Conversely, every point on this parabola is of the form
$(x,y)=(t,t^2)$, so $\phi$ is a bijection $\C\simeq C_{\mathrm{aff}}$ of
complex manifolds.

\subsection*{Homogenization and the projective conic}

Let $[X:Y:Z]$ be homogeneous coordinates on $\CP^2$. On the affine chart
$Z\neq 0$ we set
\[
x = \frac{X}{Z},\qquad y = \frac{Y}{Z}.
\]
The affine equation $y-x^2=0$ becomes
\[
\frac{Y}{Z} - \left(\frac{X}{Z}\right)^2 = 0.
\]
Multiplying by $Z^2$ gives the homogeneous equation
\[
YZ - X^2 = 0.
\]
Thus the projective closure of the parabola is the conic
\[
C := \{[X:Y:Z]\in\CP^2 \mid YZ = X^2\}.
\]

(Equivalently, one often uses the isomorphic conic $XZ=Y^2$, obtained by
renaming coordinates; this is the standard form for the image of the
Veronese embedding below.)

\subsection*{The Veronese embedding}

We now consider $\CP^1$ with homogeneous coordinates $[u:v]$ and the degree--2
Veronese embedding
\[
\nu_2:\CP^1 \longrightarrow \CP^2,\qquad
[u:v] \longmapsto [X:Y:Z] = [u^2 : uv : v^2].
\]
A direct computation shows that the image of $\nu_2$ lies on the conic
\[
XZ = Y^2.
\]
Indeed, for $[X:Y:Z]=[u^2:uv:v^2]$ we have
\[
XZ = (u^2)(v^2) = u^2 v^2,\qquad Y^2 = (uv)^2 = u^2 v^2,
\]
so $XZ-Y^2=0$ is identically satisfied.

On the affine chart $Z\neq 0$ (set $Z=1$), the equation $XZ=Y^2$ becomes
$x = y^2$ with $x=X/Z$, $y=Y/Z$. This is (up to swapping $x$ and $y$) the
same parabola $y=x^2$ considered above.

One checks that $\nu_2$ is injective, and as every point of the conic $XZ=Y^2$
has the form $[u^2:uv:v^2]$ for some $[u:v]\in\CP^1$, the map $\nu_2$ induces
a biholomorphism
\[
\CP^1 \xrightarrow{\ \sim\ } C.
\]

In particular, the Riemann sphere $S^2\simeq\CP^1$ can be viewed equivalently
as the smooth projective conic $C\subset\CP^2$ given by $XZ=Y^2$, and the
coordinate $z$ on $\CP^1$ corresponds to the parameter that maps $z$ to the
point $(x,y)=(z^2,z)$ on the parabola in the affine chart.

	
\end{document}
