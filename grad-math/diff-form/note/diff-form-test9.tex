\documentclass[12pt]{article}
\usepackage{amsmath,amssymb}
\begin{document}
	
	\section*{Coordinate Charts on a Plane Curve and Its Tangent Line}
	
	Let \(C\subseteq\mathbb{R}^2\) be the graph of a smooth function \(f\colon\mathbb{R}\to\mathbb{R}\),
	\[
	C=\bigl\{(x,y)\in\mathbb{R}^2:y=f(x)\bigr\}.
	\]
	Fix a point \(p=(a,f(a))\in C\).  Its tangent vector is 
	\(\displaystyle\vec v=(1,f'(a))^T\), and 
	\[
	T_pC=\mathrm{span}\{\,(1,f'(a))^T\}\;\subset\;T_p\mathbb{R}^2\cong\mathbb{R}^2.
	\]
	
	\subsection*{1. Chart on the Curve \(C\)}
	Define the ambient coordinate projections
	\[
	x,y\;\colon\;\mathbb{R}^2\;\longrightarrow\;\mathbb{R},
	\qquad
	x(x,y)=x,\quad y(x,y)=y,
	\]
	and restrict them to \(C\):
	\[
	x|_C\colon C\to\mathbb{R}, 
	\quad 
	y|_C\colon C\to\mathbb{R}.
	\]
	These assemble into the smooth chart
	\[
	\Phi_C\;:\;C\;\longrightarrow\;\mathbb{R}^2,
	\qquad
	\Phi_C(p)
	=\bigl(x|_C(p),\,y|_C(p)\bigr)
	=(a,f(a)).
	\]
	\emph{Note:} \(\Phi_C\) records the \emph{location} of the point \(p\in C\subset\mathbb{R}^2\).
	
	\subsection*{2. Chart on the Tangent Line \(T_pC\)}
	On the ambient tangent plane \(T_p\mathbb{R}^2\cong\mathbb{R}^2\) we have the dual projections
	\[
	dx,\,dy\;\colon\;T_p\mathbb{R}^2\;\longrightarrow\;\mathbb{R},
	\qquad
	dx\bigl((v^1,v^2)^T\bigr)=v^1,\;
	dy\bigl((v^1,v^2)^T\bigr)=v^2.
	\]
	Restrict these functionals to the line \(T_pC\):
	\[
	dx\big|_{T_pC},\;dy\big|_{T_pC}
	\;\colon\;T_pC\;\longrightarrow\;\mathbb{R}.
	\]
	Stacking them gives the fiber‐chart
	\[
	\Psi_{T_pC}\;:\;T_pC\;\longrightarrow\;\mathbb{R}^2,
	\qquad
	\Psi_{T_pC}(\vec v)
	=\begin{pmatrix}
		dx(\vec v)\\[4pt]
		dy(\vec v)
	\end{pmatrix}
	=\begin{pmatrix}
		1\\[3pt]
		f'(a)
	\end{pmatrix}.
	\]
	\emph{Note:} \(\Psi_{T_pC}\) records the \emph{components} of the tangent vector in the
	ambient basis \(\{\partial_x,\partial_y\}\).
	
	\subsection*{3. Distinguishing Points vs.\ Vectors}
	\begin{itemize}
		\item A \emph{point} \(p=(a,f(a))\in C\) is an element of the set \(C\).  Its chart–coordinate
		\(\Phi_C(p)=(a,f(a))\in\mathbb{R}^2\) tells \emph{where} on the curve \(p\) lies.
		\item A \emph{tangent vector} \(\vec v\in T_pC\) is an element of the tangent space,
		encoding a \emph{direction and speed} at \(p\).  Its chart–coordinate
		\(\Psi_{T_pC}(\vec v)=(dx(\vec v),\,dy(\vec v))\in\mathbb{R}^2\) gives its
		components relative to \(\{\partial_x,\partial_y\}\).
	\end{itemize}
	
\end{document}
