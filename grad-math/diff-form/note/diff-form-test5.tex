\documentclass[12pt]{article}
\usepackage{amsmath,amssymb,amsthm, mathtools}
\usepackage{geometry}
\geometry{margin=1in}
\usepackage{hyperref}

\title{Lecture Notes: Coordinates and Differentials on a Plane Curve}
\author{}
\date{}

\newcommand{\R}{\mathbb{R}}
\renewcommand{\span}{\text{span}}
\begin{document}
	\maketitle
	
	\section{Setup: The Plane Curve and Its Tangent}
	
	Let \(f\colon\mathbb{R}\to\mathbb{R}\) be a \(C^1\)–function, and define the plane curve
	\[
	C \;=\;\bigl\{(x,y)\in\mathbb{R}^2 \mid y=f(x)\bigr\}\subseteq\mathbb{R}^2.
	\]
	Fix a point
	\[
	p=(a,f(a))\;\in\;C,
	\]
	and consider the standard parametrization
	\[
	\Phi\colon \mathbb{R}\longrightarrow C,
	\qquad
	\Phi(t) = \bigl(t,\;f(t)\bigr).
	\]
	At \(t=a\), the tangent (velocity) is
	\[
	\Phi'(a)
	=\begin{pmatrix}
		1\\
		f'(a)
	\end{pmatrix}
	=\vec v
	\;\in\;T_pC,
	\]
	so the tangent space is the one–dimensional subspace
	\[
	T_pC \;=\;\mathrm{span}\{\vec v\}
	=\mathrm{span}\Bigl\{\,(1,\;f'(a))\Bigr\}
	\;\subset\;T_p\mathbb{R}^2\cong\mathbb{R}^2.
	\]
	
	\section{Points vs.\ Tangent Vectors}
	
	\begin{itemize}
		\item A \emph{point} \(p\in C\) is an element of the set \(C\subseteq\mathbb{R}^2\).
		We denote it by unadorned parentheses, e.g.\ \(p=(a,f(a))\).
		\item A \emph{tangent vector} \(v\in T_pC\) is an element of the tangent space
		(a copy of \(\mathbb{R}^2\)) at the point \(p\).  We denote it in bold or arrow notation,
		e.g.\ \(\vec v=\bigl(1,f'(a)\bigr)^T\).
	\end{itemize}
	
	\section{Coordinates on the Curve \(C\)}
	
	Define the ambient coordinate functions
	\[
	x,y\;\colon\;\mathbb{R}^2\;\longrightarrow\;\mathbb{R},
	\qquad
	x(x,y)=x,
	\quad
	y(x,y)=y.
	\]
	Restricted to \(C\), these give a chart
	\[
	(C\subseteq\mathbb{R}^2)\;\longrightarrow\;\mathbb{R}^2,
	\qquad
	p\;\mapsto\;\bigl(x(p),\,y(p)\bigr)
	=\bigl(a,\;f(a)\bigr).
	\]
	Thus the \emph{point–coordinates} of \(p\) are \((x(p),y(p))\).
	
	\section{Coordinates on the Tangent Line \(T_pC\)}
	
	On the ambient tangent plane \(T_p\mathbb{R}^2\cong\mathbb{R}^2\), introduce the dual basis
	\(\{dx,dy\}\) defined by
	\[
	dx\bigl(e_1\bigr)=1,\;dx(e_2)=0,
	\quad
	dy(e_1)=0,\;dy(e_2)=1,
	\]
	where \(e_1=(1,0)\), \(e_2=(0,1)\).  Then for any \(v=(v^1,v^2)^T\),
	\[
	dx(v)=v^1,
	\quad
	dy(v)=v^2.
	\]
	Restricted to the tangent line \(T_pC=\span\{(1,f'(a))\}\), these give the fiber–chart
	\[
	T_pC\;\longrightarrow\;\mathbb{R}^2,
	\qquad
	\vec v\;\mapsto\;
	\begin{pmatrix}
		dx(\vec v)\\[4pt]
		dy(\vec v)
	\end{pmatrix}
	=
	\begin{pmatrix}
		1\\
		f'(a)
	\end{pmatrix}.
	\]
	Thus the \emph{vector–coordinates} of \(\vec v\) in the basis \(\{\partial_x,\partial_y\}\)
	are \(\bigl(dx(\vec v),\,dy(\vec v)\bigr)\).
	
	\section{Summary of Maps}
	
%\begin{align*}
%&\underbrace{C\subseteq\mathbb{R}^2}_{\text{curve}}
%\;\longrightarrow\;\mathbb{R}^2,
%&&
%p\;\longmapsto\;(x(p),y(p))=(a,f(a));\\
%&\underbrace{T_pC}_{\text{tangent line}}
%\;\longrightarrow\;\mathbb{R}^2,
%&&
%\vec v\;\longmapsto\;\bigl(dx(\vec v),\,dy(\vec v)\bigr)
%=(1,f'(a)).
%\end{align*}
\[
\begin{array}{ccccc}
&\underbrace{C\subseteq\mathbb{R}^2}_{\text{curve}}
\;\longrightarrow\;\mathbb{R}^2,
&&
p\;\longmapsto\;(x(p),y(p))=(a,f(a));\\
&\underbrace{T_pC}_{\text{tangent line}}
\;\longrightarrow\;\mathbb{R}^2,
&&
\vec v\;\longmapsto\;\bigl(dx(\vec v),\,dy(\vec v)\bigr)
=(1,f'(a)).
\end{array}
\]
	
\end{document}
