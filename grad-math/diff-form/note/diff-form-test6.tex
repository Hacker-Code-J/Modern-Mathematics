\documentclass[12pt]{article}
\usepackage{amsmath,amssymb,amsthm, mathtools}
\usepackage{geometry}
\geometry{margin=1in}
\usepackage{hyperref}

\newtheorem{theorem}{Theorem} % Theorem environment with section-wise numbering
\newtheorem*{theorem*}{Theorem} % Theorem environment with section-wise numbering
\newtheorem*{lemma*}{Lemma} % Theorem environment with section-wise numbering
\newtheorem*{proposition*}{Proposition} % Theorem environment with section-wise numbering
\newtheorem*{corollary*}{Corollary} % Theorem environment with section-wise numbering
\newtheorem{proposition}[theorem]{Proposition} % Theorem environment with section-wise numbering
\newtheorem{lemma}[theorem]{Lemma} % Lemma shares the counter with theorem
\newtheorem{corollary}[theorem]{Corollary} % Corollary shares the counter with theorem

\theoremstyle{definitionstyle}
\newtheorem{definition}{Definition} % Definition shares the counter with theorem
\newtheorem{example}{Example} % Example shares the counter with theorem
\newtheorem{exercise}{{Exercise}} % Example shares the counter with theorem
\newtheorem{remark}{Remark} % Remark shares the counter with theorem
\newtheorem*{note}{Note}

\title{Lecture Notes: Coordinates and Differentials on a Plane Curve}
\author{}
\date{}

\newcommand{\R}{\mathbb{R}}
\renewcommand{\span}{\text{span}}
\begin{document}
\maketitle
	
\section{Parametrized Curve and Coordinate Chart on \(C\)}
Let \(f\colon\mathbb{R}\to\mathbb{R}\) be a \(C^1\)–function.  Define the plane curve
\[
C \;=\;\bigl\{(x,y)\in\mathbb{R}^2 \;\big|\; y=f(x)\bigr\}.
\]
We introduce the standard coordinate functions
\[
x,y\colon\mathbb{R}^2\to\mathbb{R},
\qquad
x(x,y)=x,
\quad
y(x,y)=y,
\]
and restrict them to \(C\).  The resulting chart on \(C\) is
\[
\Phi_C\colon C\;\longrightarrow\;\mathbb{R}^2,
\qquad
\Phi_C\bigl(p\bigr)
=\bigl(x(p),\,y(p)\bigr).
\]
In particular, for
\[
p=(a,f(a))\in C,
\]
we have
\[
\Phi_C(p)=(a,f(a)).
\]

\section{Tangent Space and Fiber–Coordinates on \(T_pC\)}
Parametrize \(C\) by
\[
\gamma(t) = \bigl(t,f(t)\bigr).
\]
At \(t=a\) the velocity vector is
\[
\gamma'(a)
=\begin{pmatrix}
	1\\
	f'(a)
\end{pmatrix}
\;=\;\vec v
\;\in\;T_pC.
\]
Thus the tangent space is
\[
T_pC
=\mathrm{span}\{\vec v\}
=\mathrm{span}\Bigl\{\,(1,f'(a))\Bigr\}
\subset T_p\mathbb{R}^2\cong\mathbb{R}^2.
\]
On \(T_p\mathbb{R}^2\) we employ the dual basis \(\{dx,dy\}\) defined by
\[
dx(e_1)=1,\;dx(e_2)=0,
\quad
dy(e_1)=0,\;dy(e_2)=1,
\]
where \(e_1=(1,0)\), \(e_2=(0,1)\).  Each tangent vector \(v=(v^1,v^2)\) then satisfies
\[
dx(v)=v^1,
\quad
dy(v)=v^2.
\]
Restricted to \(T_pC\), we obtain the fiber–chart
\[
\Phi_{T_pC}\colon T_pC\;\longrightarrow\;\mathbb{R}^2,
\qquad
\Phi_{T_pC}(\vec v)
=\begin{pmatrix}
	dx(\vec v)\\[4pt]
	dy(\vec v)
\end{pmatrix}
=\begin{pmatrix}
	1\\
	f'(a)
\end{pmatrix}.
\]

\section{Fixed Line and Unit Direction}
Choose a nonzero vector
\[
w=(w_1,w_2)\in\mathbb{R}^2,
\qquad
\|w\|\neq0,
\]
and consider the line \(L=\mathrm{span}\{w\}\subset\mathbb{R}^2\).  Define the unit–direction
\[
\hat w
=\frac{w}{\|w\|}
=\Bigl(\tfrac{w_1}{\sqrt{w_1^2+w_2^2}},\,\tfrac{w_2}{\sqrt{w_1^2+w_2^2}}\Bigr),
\]
so that \(\|\hat w\|=1\).

\section{Definition of the Scalar–Projection 1–Form}
\begin{definition}
	The \emph{scalar–projection 1–form} onto the line \(L\) is the mapping
	\[
	\alpha\;\colon\;T\mathbb{R}^2
	\;\longrightarrow\;\mathbb{R},
	\qquad
	\alpha_p(v)
	=\bigl\langle \hat w,\;v\bigr\rangle,
	\]
	for each \(p\in\mathbb{R}^2\) and \(v\in T_p\mathbb{R}^2\).
\end{definition}

Writing \(v=(v^1,v^2)\) in the dual coordinates, one obtains
\[
\alpha_p(v)
=\hat w_1\,v^1 + \hat w_2\,v^2
=\hat w_1\,dx(v)+\hat w_2\,dy(v).
\]
Hence the global \(1\)–form is
\[
\boxed{
	\alpha
	=\hat w_1\,dx
	\;+\;
	\hat w_2\,dy
	\;=\;
	\frac{w_1}{\sqrt{w_1^2+w_2^2}}\;dx
	\;+\;
	\frac{w_2}{\sqrt{w_1^2+w_2^2}}\;dy.
}
\]

\section{Restriction to the Curve \(C\)}
Pulling back \(\alpha\) along the inclusion \(i\colon C\hookrightarrow\R^2\) yields
\[
i^*\alpha
=\hat w_1\,dx\bigl|_C
\;+\;\hat w_2\,dy\bigl|_C,
\]
which on \(T_pC\) evaluates by
\[
\bigl(i^*\alpha\bigr)_p(\tau\,(1,f'(a)))
=\tau\,\bigl(\hat w_1 + \hat w_2\,f'(a)\bigr),
\quad
\tau\in\mathbb{R}.
\]

\section*{Summary of Charts and Forms}
\[
\begin{aligned}
	&C\subseteq\mathbb{R}^2
	\;\xrightarrow{\;\Phi_C\;}
	\mathbb{R}^2,
	&&
	p\;\mapsto\;(x(p),y(p))
	=(a,f(a)),\\
	&T_pC
	\;\xrightarrow{\;\Phi_{T_pC}\;}
	\mathbb{R}^2,
	&&
	\vec v\;\mapsto\;(dx(\vec v),dy(\vec v))
	=(1,f'(a)),\\
	&\alpha\;=\;\hat w_1\,dx+\hat w_2\,dy,
	\quad
	i^*\alpha\in\Omega^1(C).
\end{aligned}
\]
	
\end{document}
