\documentclass[11pt,openany]{article}

\input{grad-math-preamble}
\usepackage{tcolorbox}
\tcbset{colback=white, arc=5pt}

\definecolor{axiomcolor}{HTML}{a88bfa}
\definecolor{defcolor}{RGB}{52, 152, 219}
\definecolor{procolor}{RGB}{241, 196, 15}
\definecolor{thmcolor}{RGB}{231, 76, 60}
\definecolor{lemcolor}{RGB}{155, 89, 182}
\definecolor{corcolor}{RGB}{46, 204, 113}
\definecolor{execolor}{RGB}{90, 128, 127}

% Define a new command for the custom tcolorbox
\newcommand{\axiombox}[2][]{%
	\begin{tcolorbox}[colframe=axiomcolor, title={\color{white}\bfseries #1}]
		#2
	\end{tcolorbox}
}

\newcommand{\defbox}[2][]{%
	\begin{tcolorbox}[colframe=defcolor, title={\color{white}\bfseries #1}]
		#2
	\end{tcolorbox}
}

\newcommand{\probox}[2][]{%
	\begin{tcolorbox}[colframe=procolor, title={\color{white}\bfseries #1}]
		#2
	\end{tcolorbox}
}

\newcommand{\thmbox}[2][]{%
	\begin{tcolorbox}[colframe=thmcolor, title={\color{white}\bfseries #1}]
		#2
	\end{tcolorbox}
}

\newcommand{\lembox}[2][]{%
	\begin{tcolorbox}[colframe=lemcolor, title={\color{white}\bfseries #1}]
		#2
	\end{tcolorbox}
}
\usepackage{amsthm}

% Define custom theorem styles
\newtheoremstyle{dotless} % Name of the style
{3pt} % Space above
{3pt} % Space below
{\itshape} % Body font
{} % Indent amount
{\bfseries} % Theorem head font
{} % Punctuation after theorem head
{2.5mm} % Space after theorem head
{} % Theorem head spec

\newtheoremstyle{definitionstyle} % Name of the style
{3pt} % Space above
{3pt} % Space below
{} % Body font
{} % Indent amount
{\bfseries} % Theorem head font
{.} % Punctuation after theorem head
{2.5mm} % Space after theorem head
{} % Theorem head spec

% Applying custom styles
%\theoremstyle{dotless}
\newtheorem{theorem}{Theorem} % Theorem environment with section-wise numbering
\newtheorem*{theorem*}{Theorem} % Theorem environment with section-wise numbering
\newtheorem*{lemma*}{Lemma} % Theorem environment with section-wise numbering
\newtheorem*{proposition*}{Proposition} % Theorem environment with section-wise numbering
\newtheorem*{corollary*}{Corollary} % Theorem environment with section-wise numbering
\newtheorem{proposition}[theorem]{Proposition} % Theorem environment with section-wise numbering
\newtheorem{lemma}[theorem]{Lemma} % Lemma shares the counter with theorem
\newtheorem{corollary}[theorem]{Corollary} % Corollary shares the counter with theorem

\theoremstyle{definitionstyle}
\newtheorem*{observation}{\textcolor{magenta}{Observation}}
\newtheorem*{illustration}{\textcolor{teal}{Illustration}}
\newtheorem*{torus}{{\color{red}T}{\color{orange}o}{\color{green!75!black}r}{\color{cyan}u}{\color{violet}s}}
\newtheorem{definition}{Definition} % Definition shares the counter with theorem
\newtheorem{example}{Example} % Example shares the counter with theorem
\newtheorem{exercise}{{Exercise}} % Example shares the counter with theorem
\newtheorem{remark}{Remark} % Remark shares the counter with theorem
\newtheorem*{note}{Note}
\newtheorem*{notation}{Notation}

\newtheorem*{axiom*}{Axiom}
\newtheorem*{definition*}{Definition} % Definition shares the counter with theorem
\newtheorem*{example*}{Example} % Example shares the counter with theorem
\newtheorem*{exercise*}{\textcolor{teal}{Exercise}} % Example shares the counter with theorem
\newtheorem*{remark*}{Remark} % Remark shares the counter with theorem


\usepackage{tikz}
\usepackage{tikz-cd}
\usetikzlibrary{shadows}
\usetikzlibrary{shapes.geometric, arrows.meta, positioning}
\newcommand{\ie}{\textnormal{i.e.}}
\newcommand{\rsa}{\mathsf{RSA}}
\newcommand{\rsacrt}{\mathsf{RSA}\textendash\mathsf{CRT}}
\newcommand{\inv}[1]{#1^{-1}}

%New Command
%\newcommand{\set}[1]{\left\{#1\right\}}
\newcommand{\N}{\mathbb{N}}
\newcommand{\Z}{\mathbb{Z}}
\newcommand{\Q}{\mathbb{Q}}
\newcommand{\R}{\mathbb{R}}
\newcommand{\cR}{\mathcal{R}}
\newcommand{\C}{\mathbb{C}}
\newcommand{\F}{\mathbb{F}}
\newcommand{\nbhd}{\mathcal{N}}
\newcommand{\Log}{\operatorname{Log}}
\newcommand{\Arg}{\operatorname{Arg}}
\newcommand{\pv}{\operatorname{P.V.}}

\newcommand{\of}[1]{\left( #1 \right)} 
%\newcommand{\abs}[1]{\left\lvert #1 \right\rvert}
%\newcommand{\norm}[1]{\left\| #1 \right\|}

\newcommand{\sol}{\textcolor{magenta}{\bf Sol}}
\newcommand{\conjugate}[1]{\overline{#1}}

\newcommand{\res}{\operatorname{res}}
\DeclareMathOperator*{\Res}{\operatorname{Res}}

%\renewcommand{\Re}{\operatorname{Re}}
%\renewcommand{\Im}{\operatorname{Im}}

\newcommand{\cyclic}[1]{\langle #1 \rangle}
\newcommand{\uniform}{\overset{\$}{\leftarrow}}
\newcommand{\xmark}{\textcolor{red}{\XSolidBrush}}
\newcommand{\vmark}{\textcolor{green!75!black}{\CheckmarkBold}}

\newcommand{\gen}[1]{\langle #1 \rangle}
\newcommand{\Gen}[1]{\left\langle #1 \right\rangle}

\newcommand{\img}[1]{\text{Img}(#1)}
\newcommand{\Img}[1]{\text{Img}\left(#1\right)}
\newcommand{\preimg}[1]{\text{Img}^{-1}(#1)}
\newcommand{\Preimg}[1]{\text{Img}^{-1}\left(#1\right)}

\newcommand{\relation}{\mathrel{\mathcal{R}}}
\newcommand{\injection}{\rightarrowtail}
\newcommand{\surjection}{\twoheadrightarrow}
\newcommand{\id}{\textnormal{id}}

\newcommand{\eqclass}[1]{\left[#1\right]}

% Define custom colors for O and X
\newcommand{\yes}{\textcolor{blue}{\bf \fullmoon}}
\newcommand{\no}{\textcolor{red}{\bf \texttimes}}

\DeclarePairedDelimiter\ceil{\lceil}{\rceil}
\DeclarePairedDelimiter\floor{\lfloor}{\rfloor}
%\renewcommand{\floor}[#1]{\lfloor #1\rfloor}
%\newcommand{\Floor}[#1]{\left\lfloor #1\right\rfloor}
%\newcommand{\ceil}[#1]{\lceil #1\rceil}
%\newcommand{\Ceil}[#1]{\left\lceil #1\right\rceil}

\newcommand{\topology}{\mathscr{T}}
\newcommand{\sequence}[1]{\langle #1\rangle}


\usepackage{mathtools}

\renewcommand{\vec}[1]{\mathbf{#1}}
\renewcommand{\Re}{\operatorname*{Re}}
\renewcommand{\Im}{\operatorname*{Im}}
\newcommand{\Mat}{\operatorname{Mat}}

\newcommand{\Sym}{\mathrm{Sym}}


%\newcommand{\F}{\mathbb{F}}
%\newcommand{\R}{\mathbb{R}}
%\newcommand{\C}{\mathbb{C}}
%\newcommand{\Span}{\operatorname{span}}
%\newcommand{\Mat}{\operatorname{Mat}}
\newcommand{\tr}{\operatorname{tr}}
\renewcommand{\d}{\mathrm{d}} % For the exterior derivative 'd'
\newcommand{\pderiv}[2]{\frac{\partial #1}{\partial #2}}
\newcommand{\spderiv}[3]{\frac{\partial^2 #1}{\partial #2\partial #3}}
\newcommand{\GL}{\operatorname{GL}}



%\newcommand{\basis}{def}

\setstretch{1.25}

\begin{document}
\pagenumbering{arabic}
\begin{center}
	\huge\textbf{Linear Algebra IV}\\
	\vspace{0.5em}
	\large{Ji, Yong-hyeon}\\
	\vspace{0.5em}
	\normalsize{\today}\\
\end{center}

\noindent 
We cover the following topics in this note.
\begin{itemize}
	\item Eigenvectors and Diagonalization.
\end{itemize}
\hrule\vspace{12pt}
%\tableofcontents
\vfill

\newpage
%\begin{observation}[Choosing a basis to simplify a linear map]
%Consider a finite-dimensional vector space $V$ over a field $\F$. Let $T:V\to V$ be a linear operator. Let $\basis=\set{\textbf{v}_1,\dots,\textbf{v}_n}$ be a basis of $V$ such that $[T]_{\basis}$ is a diagonal matrix: \[
%[T]_{\basis}=[T]_{\basis}^{\basis}=\begin{pmatrix}
%	{\color{blue}d_1} & 0 & \cdots & 0\\
%	0 & {\color{blue}d_1} & \cdots & 0 \\
%	\vdots & \vdots & \ddots & \vdots \\
%	0 & 0 & \cdots & {\color{blue}d_n} \\
%\end{pmatrix},\quad\ie,\quad T(\textbf{v}_i)=d_i\textbf{v}_i\;\text{with}\; 1\leq i\leq n.
%\]
%Then $T$ may be very complicated, but with respect to $[T]_{\basis}$ it looks nice. 
%%(``$T$ acts by scalars on the coordinate directions''). 
%\end{observation}

%\begin{example}[Shape operator; Kim, Namho]
%Let $S^2=\set{x\in\R^3:\norm{x}=1}$. Define the sphere by $f(x,y,z)=x^2+y^2+z^2=1$. Compute \[
%\nabla f=\left(\pderiv{f}{x},\pderiv{f}{y},\pderiv{f}{z}\right)=(2x,2y,2z).
%\] Pick any point $p=(x,y,z)$ on the surface. Then the outward unit normal is \[
%N:S^2\to S^2:p\mapsto N(p)=\frac{\nabla f}{\abs{\nabla f}}=\frac{2p}{\abs{2p}}=\frac{2p}{2}=p.
%\]
%\begin{center}
%\includegraphics[scale=1]{grad-math-tikz/shape_operator_1.pdf}
%\end{center}
%Since $N$ is a smooth map between manifolds, at each $p\in S^2$ it has a differential (pushforward) \[
%(dN)_p:T_pS^2\to T_{N(p)}S^2
%\] Take any smooth curve $\gamma:N_\varepsilon(0)\to S^2$ with \[
%\gamma(0)=p,\quad\gamma'(0)=\textbf{v}\in T_pS^2.
%\] Then \[
%(dN)_p(\textbf{v})=\frac{d}{dt}\left(N(\gamma(t))\right)\big|_{t=0}.
%\] We move on the surface in direction $\textbf{v}$, watch how the normal vector changes, and differentiate.
%
%The shape operator (Weingarten map) is defined pointwise by \[
%\fullfunction{S_p}{T_pS^2}{T_{N(p)}S^2}{\textbf{v}}{-(dN)_p(\textbf{v})}.
%\] Note that \[
%T_pS^2=\set{\textbf{w}\in R^3:\textbf{w}\cdot N(p)=0}=T_{N(p)}S^2.
%\] In summary, \[
%S^2\xrightarrow{N}S^2\quad\Longrightarrow\quad T_pS^2\xrightarrow{(dN)_p} T_{N(p)}S^2\simeq T_pS^2\quad\Longrightarrow\quad S_p=-(dN)_p.
%\]
%
%\newpage
%Let $U=\intoo{0,\pi}\times\intoo{0,2\pi}\subset\R^2$. Consider the standard spherical parametrization \[
%\fullfunction{X}{U}{S^2(\subset\R^3)}{(\theta,\phi)}{X(\theta,\phi)=\begin{pmatrix}
%		\sin\theta\cos\phi\\
%		\sin\theta\sin\phi\\
%		\cos\theta
%	\end{pmatrix}}.
%\] At $p=X(\theta,\phi)$, the tangent vectors are
%\[
%\pderiv{}{\theta}X(\theta,\phi)=
%\begin{pmatrix}
%	\cos\theta\cos\phi\\
%	\cos\theta\sin\phi\\
%	-\sin\theta
%\end{pmatrix},
%\qquad
%\pderiv{}{\phi}X(\theta,\phi)=
%\begin{pmatrix}
%	-\sin\theta\sin\phi\\
%	\sin\theta\cos\phi\\
%	0
%\end{pmatrix}.
%\]
%They span $T_pS^2$ for $0<\theta<\pi$.
%Note that \begin{enumerate}[(i)]
%	\item $\norm{\pderiv{}{\theta}X}=\sqrt{\cos^2\theta\cos^2\phi+\cos^2\theta\sin^2\phi+\sin^2\theta}=\sqrt{\cos^2\theta(\cos^2\phi+\sin^2\phi)+\sin^2\theta}=1$.
%	\item $\norm{\pderiv{}{\phi}X}=\sqrt{\sin^2\theta\sin^2\phi+\sin^2\theta\cos^2\phi}=\sqrt{\sin^2\theta(\sin^2\phi+\cos^2\phi)}=\abs{\sin\theta}=\sin\theta$.
%	\item $\pderiv{}{\theta}X\cdot\pderiv{}{\phi}X=(-\cos\theta\cos\phi\sin\theta\sin\phi)+(\cos\theta\sin\phi\sin\theta\cos\phi)+0=0$.
%\end{enumerate}
%Hence an \textbf{orthonormal basis} of $T_pS^2$ is
%\[
%e_1:=\pderiv{}{\theta}X,
%\qquad
%e_2:=\frac{1}{\sin\theta}\,\pderiv{}{\phi}X.
%\]
%\end{example}





\newpage
%\section*{Eigenvectors}
%\textbf{Def.}
%A nonzero vector $u\in V$ is called an \emph{eigenvector} of $T$ if
%\[
%T(u)=\lambda u \quad\text{for some }\lambda\in\F.
%\]
%$\lambda$ is called the \emph{eigenvalue} of $u$.
%
%\medskip
%\textbf{Remark.}
%If $u\neq 0$ and $T(u)=\lambda u$, then $T(\F u)\subseteq \F u$ (restriction map)
%\[
%T|_{\F u}:\F u \to \F u,
%\]
%and $T$ acts as scalar multiplication by $\lambda$ on the line $\F u$.
%
%\section*{Diagonalizable}
%\textbf{Def.}
%$T:V\to V$ is \emph{diagonalizable} if there exists a basis $\beta$ of $V$ such that
%$[T]_{\beta}$ is diagonal.
%Equivalently, there is a basis of $V$ consisting of eigenvectors of $T$.
%
%\medskip
%Example (matrices). Let $A\in \Mat_{n\times n}(\F)$ and consider
%\[
%L_A:\F^n\to \F^n, \qquad x\mapsto Ax.
%\]
%Suppose $L_A$ is diagonalizable. Then there is a basis $P$ of eigenvectors of $L_A$ such that
%\[
%[L_A]_P = D \quad\text{(a diagonal matrix)}.
%\]
%In matrix language this means
%\[
%P^{-1}AP = D \quad\Longleftrightarrow\quad A = PDP^{-1}.
%\]
%
%\medskip
%Consequently, $L_A$ is diagonalizable iff there exists an invertible $P$ such that $P^{-1}AP$
%is a diagonal matrix, i.e. $A$ is similar to a diagonal matrix.
%
%\medskip
%\textbf{Def.}
%A matrix $A\in \Mat_{n\times n}(\F)$ is \emph{diagonalizable} if $A$ is similar to a diagonal matrix,
%i.e. $\exists\, P\in \GL_n(\F)$ such that
%\[
%P^{-1}AP = D \quad\text{(diagonal)}.
%\]
%
%\section*{Example: rotation and dependence on the field}
%Let $A=R_{90}=\begin{pmatrix}0&-1\\ 1&0\end{pmatrix}$.
%
%\medskip
%\textbf{Case 1: $\F=\R$.}
%Then there is no nonzero vector $u$ such that $Au=\lambda u$ for some $\lambda\in\R$.
%Thus $R_{90}$ is not diagonalizable over $\R$.
%
%\medskip
%\textbf{Case 2: $\F=\C$.}
%Consider $L_A:\C^2\to\C^2$, $x\mapsto Ax$.
%We want $u\neq 0$ and $\lambda\in\C$ such that
%\[
%Au=\lambda u.
%\]
%Equivalently,
%\[
%(A-\lambda I)u=0.
%\]
%There exists a nonzero solution $u$ iff $A-\lambda I$ is not invertible, i.e.
%\[
%\det(A-\lambda I)=0.
%\]
%We call $\det(A-\lambda I)$ the \emph{characteristic polynomial} of $A$.
%
%\medskip
%More generally, for the rotation matrix
%\[
%A=
%\begin{pmatrix}
%	\cos\theta & -\sin\theta\\
%	\sin\theta & \cos\theta
%\end{pmatrix},
%\quad
%A-\lambda I=
%\begin{pmatrix}
%	\cos\theta-\lambda & -\sin\theta\\
%	\sin\theta & \cos\theta-\lambda
%\end{pmatrix},
%\]
%we compute
%\[
%\det(A-\lambda I)
%=(\cos\theta-\lambda)^2+\sin^2\theta
%=\lambda^2-2(\cos\theta)\lambda+1.
%\]
%Over $\C$ this has two distinct roots:
%\[
%\lambda_\pm = \cos\theta \pm i\sin\theta.
%\]
%So the eigenvalues are $\cos\theta\pm i\sin\theta$.
%
%\medskip
%For $\lambda_+=\cos\theta+i\sin\theta=e^{i\theta}$ one can take an eigenvector
%\[
%u_+=\begin{pmatrix}1\\ -i\end{pmatrix},
%\qquad
%Au_+=\lambda_+u_+.
%\]
%Similarly, for $\lambda_-=\cos\theta-i\sin\theta=e^{-i\theta}$ one can take
%\[
%u_-=\begin{pmatrix}1\\ i\end{pmatrix},
%\qquad
%Au_-=\lambda_-u_-.
%\]
%Hence over $\C$, the rotation matrix is diagonalizable.
%
%\medskip
%We saw that diagonalization indeed depends on the choice of $\F$.
%
%\section*{Characteristic polynomial}
%\textbf{Def.}
%Let $A\in \Mat_{n\times n}(\F)$. The \emph{characteristic polynomial} of $A$ is
%\[
%f_A(\lambda):=\det(A-\lambda I).
%\]
%(We often denote it by $\chi_A(\lambda)$.)
%
%\section*{Exercise}
%Show that if
%\[
%f_A(\lambda)=\det(A-\lambda I)=(-1)^n\lambda^n + c_{n-1}\lambda^{n-1}+\cdots + c_1\lambda + c_0,
%\]
%then
%\[
%c_0=\det(A),
%\qquad
%c_{n-1}=(-1)^{n-1}\tr(A).
%\]
%(Show this by induction; the $\lambda^{n-1}$ coefficient comes from the sum of diagonal entries.)




\newpage

%\section*{Motivation: }

%A key idea is that $T$ may look complicated in one basis but becomes simpler in a more suitable basis.
%In the best case, the matrix of $T$ becomes diagonal, so that $T$ acts by independent scalings in each coordinate direction.

%\section{Eigenvectors and eigenvalues}
%
%\begin{definition}[Eigenvector, eigenvalue]
%	A nonzero vector $v\in V$ is an \emph{eigenvector} of $T$ if there exists $\lambda\in\F$ such that
%	\[
%	T(v)=\lambda v.
%	\]
%	The scalar $\lambda$ is called the \emph{eigenvalue} corresponding to $v$:contentReference[oaicite:3]{index=3}.
%\end{definition}
%
%\begin{remark}[One-dimensional invariant subspace]
%	If $T(v)=\lambda v$ with $v\neq 0$, then the line $\Span(v)$ is $T$-invariant and
%	\[
%	T|_{\Span(v)}:\Span(v)\to \Span(v)
%	\]
%	acts as scalar multiplication by $\lambda$:contentReference[oaicite:4]{index=4}.
%\end{remark}
%
%\section{Diagonalizable operators and matrices}
%
%\begin{definition}[Diagonalizable operator]
%	We say $T:V\to V$ is \emph{diagonalizable} if there exists a basis of $V$ such that the matrix of $T$
%	with respect to that basis is diagonal.
%	Equivalently, $T$ is diagonalizable iff $V$ has a basis consisting of eigenvectors of $T$:contentReference[oaicite:5]{index=5}.
%\end{definition}
%
%\subsection*{Matrices and similarity}
%Let $A\in \Mat_{n\times n}(\F)$ and consider the linear map $L_A:\F^n\to\F^n$ given by $L_A(x)=Ax$.
%
%\begin{proposition}[Diagonalization and similarity]
%	A matrix $A$ is diagonalizable over $\F$ iff there exists an invertible matrix $P\in \mathrm{GL}_n(\F)$
%	and a diagonal matrix $D$ such that
%	\[
%	P^{-1}AP = D
%	\qquad\text{equivalently}\qquad
%	A = PDP^{-1}.
%	\]
%	In this case, the columns of $P$ may be taken to be a basis of eigenvectors of $A$:contentReference[oaicite:6]{index=6}.
%\end{proposition}
%
%\begin{remark}
%	If $A$ and $B$ are similar ($B=P^{-1}AP$), then they represent the same linear operator in two different bases.
%	Hence diagonalizability is a property of the linear operator (or similarity class), not of a specific matrix form:contentReference[oaicite:7]{index=7}.
%\end{remark}
%
%\section{Characteristic polynomial}
%Eigenvalues are found by solving the equation
%\[
%Av=\lambda v \quad (v\neq 0),
%\]
%which is equivalent to
%\[
%(A-\lambda I)v=0 \quad (v\neq 0).
%\]
%Thus $\lambda$ is an eigenvalue iff $A-\lambda I$ is not invertible, i.e.
%\[
%\det(A-\lambda I)=0.
%\]
%
%\begin{definition}[Characteristic polynomial]
%	For $A\in \Mat_{n\times n}(\F)$, the \emph{characteristic polynomial} of $A$ is
%	\[
%	\chi_A(\lambda) := \det(A-\lambda I).
%	\]
%	Its roots (in an extension field if necessary) are exactly the eigenvalues of $A$:contentReference[oaicite:8]{index=8}:contentReference[oaicite:9]{index=9}.
%\end{definition}
%
%\section{Example: rotations and dependence on the field}
%
%\begin{example}[Rotation by $90^\circ$]
%	Consider the matrix
%	\[
%	R=\begin{pmatrix}0&-1\\[2pt]1&0\end{pmatrix},
%	\]
%	which represents rotation by $90^\circ$ in $\R^2$.
%	Over $\R$, it has no real eigenvectors (no nonzero vector is mapped to a real scalar multiple of itself),
%	so it is not diagonalizable over $\R$.
%	Over $\C$, it has eigenvalues $\pm i$ and becomes diagonalizable over $\C$:contentReference[oaicite:10]{index=10}:contentReference[oaicite:11]{index=11}.
%\end{example}
%
%\begin{example}[Rotation by an angle $\theta$]
%	Let
%	\[
%	A=\begin{pmatrix}\cos\theta&-\sin\theta\\[2pt]\sin\theta&\cos\theta\end{pmatrix}.
%	\]
%	Then
%	\[
%	\chi_A(\lambda)=\det(A-\lambda I)
%	= \lambda^2 - 2(\cos\theta)\lambda + 1.
%	\]
%	Over $\C$, the eigenvalues are
%	\[
%	\lambda_{\pm}=\cos\theta \pm i\sin\theta = e^{\pm i\theta},
%	\]
%	so if $\theta\not\equiv 0,\pi \pmod{2\pi}$ then the roots are distinct and $A$ is diagonalizable over $\C$.
%	This illustrates that diagonalizability can depend on the choice of field $\F$:contentReference[oaicite:12]{index=12}.
%\end{example}
%
%\section{Exercise}
%\begin{exercise}
%	Let $A\in \Mat_{n\times n}(\F)$ and write
%	\[
%	\chi_A(\lambda)=\det(A-\lambda I)=(-1)^n\lambda^n + c_{n-1}\lambda^{n-1}+\cdots + c_1\lambda + c_0.
%	\]
%	Show that:
%	\begin{enumerate}
%		\item $c_0=\det(A)$,
%		\item $c_{n-1}=(-1)^{n-1}\tr(A)$.
%	\end{enumerate}
%	(Hint: prove by induction on $n$, using expansion of the determinant; the $\lambda^{n-1}$ coefficient comes from the sum of diagonal entries.):contentReference[oaicite:13]{index=13}
%\end{exercise}


\vfill
\begin{thebibliography}{9}
	\bibitem{linear_algebra_h}
	수학의 즐거움, Enjoying Math. ``수학 공부, 기초부터 대학원 수학까지, 31. 선형대수학 (h) 고유벡터와 행렬의 대각화 -1'' YouTube Video, 29:46. Published 
	November 06, 2019. URL: \url{https://www.youtube.com/watch?v=RSOxa1rI_Kk}.
%	\bibitem{linear_algebra_i}
%	수학의 즐거움, Enjoying Math. ``수학 공부, 기초부터 대학원 수학까지, 32. 선형대수학 (i) 고유치와 행렬의 대각화 -2'' YouTube Video, 30:21. Published 
%	November 08, 2019. URL: \url{https://www.youtube.com/watch?v=bjEuNw0FnPw}.
%	\bibitem{linear_algebra_j}
%	수학의 즐거움, Enjoying Math. ``수학 공부, 기초부터 대학원 수학까지, 33. 선형대수학 (j) 행렬의 대각화와 고유공간'' YouTube Video, 29:59. Published 
%	November 09, 2019. URL: \url{https://www.youtube.com/watch?v=AlTo9fqlSn8}.
\end{thebibliography}
\end{document}
