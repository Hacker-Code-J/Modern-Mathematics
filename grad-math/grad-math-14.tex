\documentclass[11pt,openany]{article}

\input{grad-math-preamble}
\usepackage{tcolorbox}
\tcbset{colback=white, arc=5pt}

\definecolor{axiomcolor}{HTML}{a88bfa}
\definecolor{defcolor}{RGB}{52, 152, 219}
\definecolor{procolor}{RGB}{241, 196, 15}
\definecolor{thmcolor}{RGB}{231, 76, 60}
\definecolor{lemcolor}{RGB}{155, 89, 182}
\definecolor{corcolor}{RGB}{46, 204, 113}
\definecolor{execolor}{RGB}{90, 128, 127}

% Define a new command for the custom tcolorbox
\newcommand{\axiombox}[2][]{%
	\begin{tcolorbox}[colframe=axiomcolor, title={\color{white}\bfseries #1}]
		#2
	\end{tcolorbox}
}

\newcommand{\defbox}[2][]{%
	\begin{tcolorbox}[colframe=defcolor, title={\color{white}\bfseries #1}]
		#2
	\end{tcolorbox}
}

\newcommand{\probox}[2][]{%
	\begin{tcolorbox}[colframe=procolor, title={\color{white}\bfseries #1}]
		#2
	\end{tcolorbox}
}

\newcommand{\thmbox}[2][]{%
	\begin{tcolorbox}[colframe=thmcolor, title={\color{white}\bfseries #1}]
		#2
	\end{tcolorbox}
}

\newcommand{\lembox}[2][]{%
	\begin{tcolorbox}[colframe=lemcolor, title={\color{white}\bfseries #1}]
		#2
	\end{tcolorbox}
}
\usepackage{amsthm}

% Define custom theorem styles
\newtheoremstyle{dotless} % Name of the style
{3pt} % Space above
{3pt} % Space below
{\itshape} % Body font
{} % Indent amount
{\bfseries} % Theorem head font
{} % Punctuation after theorem head
{2.5mm} % Space after theorem head
{} % Theorem head spec

\newtheoremstyle{definitionstyle} % Name of the style
{3pt} % Space above
{3pt} % Space below
{} % Body font
{} % Indent amount
{\bfseries} % Theorem head font
{.} % Punctuation after theorem head
{2.5mm} % Space after theorem head
{} % Theorem head spec

% Applying custom styles
%\theoremstyle{dotless}
\newtheorem{theorem}{Theorem} % Theorem environment with section-wise numbering
\newtheorem*{theorem*}{Theorem} % Theorem environment with section-wise numbering
\newtheorem*{lemma*}{Lemma} % Theorem environment with section-wise numbering
\newtheorem*{proposition*}{Proposition} % Theorem environment with section-wise numbering
\newtheorem*{corollary*}{Corollary} % Theorem environment with section-wise numbering
\newtheorem{proposition}[theorem]{Proposition} % Theorem environment with section-wise numbering
\newtheorem{lemma}[theorem]{Lemma} % Lemma shares the counter with theorem
\newtheorem{corollary}[theorem]{Corollary} % Corollary shares the counter with theorem

\theoremstyle{definitionstyle}
\newtheorem*{observation}{\textcolor{magenta}{Observation}}
\newtheorem*{illustration}{\textcolor{teal}{Illustration}}
\newtheorem*{torus}{{\color{red}T}{\color{orange}o}{\color{green!75!black}r}{\color{cyan}u}{\color{violet}s}}
\newtheorem{definition}{Definition} % Definition shares the counter with theorem
\newtheorem{example}{Example} % Example shares the counter with theorem
\newtheorem{exercise}{{Exercise}} % Example shares the counter with theorem
\newtheorem{remark}{Remark} % Remark shares the counter with theorem
\newtheorem*{note}{Note}
\newtheorem*{notation}{Notation}

\newtheorem*{axiom*}{Axiom}
\newtheorem*{definition*}{Definition} % Definition shares the counter with theorem
\newtheorem*{example*}{Example} % Example shares the counter with theorem
\newtheorem*{exercise*}{\textcolor{teal}{Exercise}} % Example shares the counter with theorem
\newtheorem*{remark*}{Remark} % Remark shares the counter with theorem


\usepackage{tikz}
\usepackage{tikz-cd}
\usetikzlibrary{shadows}
\usetikzlibrary{shapes.geometric, arrows.meta, positioning}
\newcommand{\ie}{\textnormal{i.e.}}
\newcommand{\rsa}{\mathsf{RSA}}
\newcommand{\rsacrt}{\mathsf{RSA}\textendash\mathsf{CRT}}
\newcommand{\inv}[1]{#1^{-1}}

%New Command
%\newcommand{\set}[1]{\left\{#1\right\}}
\newcommand{\N}{\mathbb{N}}
\newcommand{\Z}{\mathbb{Z}}
\newcommand{\Q}{\mathbb{Q}}
\newcommand{\R}{\mathbb{R}}
\newcommand{\cR}{\mathcal{R}}
\newcommand{\C}{\mathbb{C}}
\newcommand{\F}{\mathbb{F}}
\newcommand{\nbhd}{\mathcal{N}}
\newcommand{\Log}{\operatorname{Log}}
\newcommand{\Arg}{\operatorname{Arg}}
\newcommand{\pv}{\operatorname{P.V.}}

\newcommand{\of}[1]{\left( #1 \right)} 
%\newcommand{\abs}[1]{\left\lvert #1 \right\rvert}
%\newcommand{\norm}[1]{\left\| #1 \right\|}

\newcommand{\sol}{\textcolor{magenta}{\bf Sol}}
\newcommand{\conjugate}[1]{\overline{#1}}

\newcommand{\res}{\operatorname{res}}
\DeclareMathOperator*{\Res}{\operatorname{Res}}

%\renewcommand{\Re}{\operatorname{Re}}
%\renewcommand{\Im}{\operatorname{Im}}

\newcommand{\cyclic}[1]{\langle #1 \rangle}
\newcommand{\uniform}{\overset{\$}{\leftarrow}}
\newcommand{\xmark}{\textcolor{red}{\XSolidBrush}}
\newcommand{\vmark}{\textcolor{green!75!black}{\CheckmarkBold}}

\newcommand{\gen}[1]{\langle #1 \rangle}
\newcommand{\Gen}[1]{\left\langle #1 \right\rangle}

\newcommand{\img}[1]{\text{Img}(#1)}
\newcommand{\Img}[1]{\text{Img}\left(#1\right)}
\newcommand{\preimg}[1]{\text{Img}^{-1}(#1)}
\newcommand{\Preimg}[1]{\text{Img}^{-1}\left(#1\right)}

\newcommand{\relation}{\mathrel{\mathcal{R}}}
\newcommand{\injection}{\rightarrowtail}
\newcommand{\surjection}{\twoheadrightarrow}
\newcommand{\id}{\textnormal{id}}

\newcommand{\eqclass}[1]{\left[#1\right]}

% Define custom colors for O and X
\newcommand{\yes}{\textcolor{blue}{\bf \fullmoon}}
\newcommand{\no}{\textcolor{red}{\bf \texttimes}}

\DeclarePairedDelimiter\ceil{\lceil}{\rceil}
\DeclarePairedDelimiter\floor{\lfloor}{\rfloor}
%\renewcommand{\floor}[#1]{\lfloor #1\rfloor}
%\newcommand{\Floor}[#1]{\left\lfloor #1\right\rfloor}
%\newcommand{\ceil}[#1]{\lceil #1\rceil}
%\newcommand{\Ceil}[#1]{\left\lceil #1\right\rceil}

\newcommand{\topology}{\mathscr{T}}
\newcommand{\sequence}[1]{\langle #1\rangle}
\renewcommand{\vec}[1]{\mathbf{#1}}
\renewcommand{\Re}{\operatorname*{Re}}
\renewcommand{\Im}{\operatorname*{Im}}

\newcommand{\Sym}{\mathrm{Sym}}

\setstretch{1.25}

\begin{document}
\pagenumbering{arabic}
\begin{center}
	\huge\textbf{Abstract Algebra to Linear Algebra}\\
	\vspace{0.5em}
	\large{Ji, Yong-hyeon}\\
	\vspace{0.5em}
	\normalsize{\today}\\
\end{center}

\noindent 
We cover the following topics in this note.
\begin{itemize}
	\item Symmetric Group
\end{itemize}
\hrule\vspace{12pt}
\vfill

\newpage
\defbox[Symmetric group on a set]{\begin{definition*}
	Let $X$ be a set. The \textbf{symmetric group on $X$} is the group \[
	S_n:=\set{\sigma\in X^X:\sigma\; \text{is bijective}}
	\] whose elements are all bijections $\sigma: X \to X$ and whose group operation is composition of functions:
	\[
	(\sigma\circ\tau)(x) \coloneqq \sigma\bigl(\tau(x)\bigr)
	\quad\text{for all } x\in X.
	\]
	The identity element is the identity map $\mathrm{id}_X$, and the inverse of $\sigma$ is its inverse bijection $\sigma^{-1}$.
\end{definition*}}

%\begin{definition}[Finite case and $S_n$]
%	For a finite set $X$ with $|X|=n$, $\Sym(X)$ is (noncanonically) isomorphic to the \emph{symmetric group of degree $n$}, denoted $S_n$, which is the group of all permutations of the set $\{1,2,\dots,n\}$ under composition. In particular, $|S_n|=n!$.
%\end{definition}

\paragraph{Transpositions generate.}
$S_n$ is generated by transpositions (permutations that swap two elements and fix the rest), and in fact by the adjacent transpositions $s_i=(i\ i{+}1)$ for $i=1,\dots,n-1$.

\paragraph{Presentation.}
A standard presentation is
\[
S_n \cong \left\langle s_1,\dots,s_{n-1}\ \middle|\ 
\begin{aligned}
&s_i^2 = 1 && \text{for } i=1,\dots,n-1,\\
&s_is_j = s_js_i && \text{for } |i-j|\ge 2,\\
&s_is_{i+1}s_i = s_{i+1}s_is_{i+1} && \text{for } i=1,\dots,n-2
\end{aligned}
\right\rangle .
\]

\paragraph{Cycle notation.}
Elements of $S_n$ are often written in cycle notation; composition is right-to-left by convention.

\newpage
\probox[Cycle decomposition in $S_n$]{
\begin{proposition*}
Let $n\in\mathbb{N}$ and let $\sigma\in S_n$ act on the set $\{1,\dots,n\}$. Then
\begin{enumerate}
	\item There exist cycles $c_1,\dots,c_k$ with pairwise disjoint supports such that
	\[
	\sigma = c_1 c_2 \cdots c_k,
	\]
	where each $c_i$ is a cycle of length $\ge 2$; fixed points of $\sigma$ may be included as $1$-cycles or omitted.
	\item The cycles $c_1,\dots,c_k$ commute (because they are disjoint), and the multiset $\{c_1,\dots,c_k\}$ is uniquely determined by $\sigma$ up to reordering of the cycles and cyclic permutation (rotation) of the entries within each cycle.
\end{enumerate}
\end{proposition*}}

\begin{proof}
	Consider the action of the cyclic subgroup $\langle \sigma\rangle$ on $\{1,\dots,n\}$. This partitions $\{1,\dots,n\}$ into disjoint orbits. For each orbit $O=\{x,\sigma(x),\dots,\sigma^{m-1}(x)\}$ (with $m\ge 1$ minimal such that $\sigma^m(x)=x$), define the cycle
	\[
	c_O := (\,x\ \sigma(x)\ \sigma^2(x)\ \cdots\ \sigma^{m-1}(x)\,).
	\]
	If $m=1$ then $x$ is a fixed point and $c_O$ is a $1$-cycle. The supports of the $c_O$ are precisely the distinct orbits, hence pairwise disjoint, and thus the cycles commute.
	
	For any $y$ in an orbit $O$, we have $c_O(y)=\sigma(y)$ by construction, while for $y\notin O$, $c_O(y)=y$. Therefore the product over all orbits,
	\[
	\prod_{O} c_O,
	\]
	acts on each $y\in\{1,\dots,n\}$ exactly as $\sigma$ does; hence $\sigma=\prod_O c_O$. This proves existence.
	
	For uniqueness, suppose $\sigma=d_1\cdots d_r$ is another product of pairwise disjoint cycles (allowing $1$-cycles). Each $d_j$ permutes a subset that is a union of $\langle \sigma\rangle$-orbits, but since $d_j$ is a single cycle, that subset must be exactly one orbit of $\langle \sigma\rangle$. Thus every $d_j$ coincides with the corresponding $c_O$ up to cyclic rotation of its entries. Because disjoint cycles commute, the product is independent of the order of the cycles. Hence the decomposition is unique up to reordering of cycles and rotation within cycles.
\end{proof}



\vfill
\begin{thebibliography}{9}
	\bibitem{abstract_algebra_g}
	수학의 즐거움, Enjoying Math. ``수학 공부, 기초부터 대학원 수학까지, 27. 추상대수학에서 선형대수학으로 : 대칭군과 행렬식의 정의 symmetric group and def of determinant'' YouTube Video, 27:07. Published 
	October 29, 2019. URL: \url{https://www.youtube.com/watch?v=UIlC9ikSpNc&t=1026s}.
\end{thebibliography}
\end{document}
