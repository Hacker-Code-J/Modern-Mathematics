\documentclass[11pt,openany]{article}

\input{grad-math-preamble}
\usepackage{tcolorbox}
\tcbset{colback=white, arc=5pt}

\definecolor{axiomcolor}{HTML}{a88bfa}
\definecolor{defcolor}{RGB}{52, 152, 219}
\definecolor{procolor}{RGB}{241, 196, 15}
\definecolor{thmcolor}{RGB}{231, 76, 60}
\definecolor{lemcolor}{RGB}{155, 89, 182}
\definecolor{corcolor}{RGB}{46, 204, 113}
\definecolor{execolor}{RGB}{90, 128, 127}

% Define a new command for the custom tcolorbox
\newcommand{\axiombox}[2][]{%
	\begin{tcolorbox}[colframe=axiomcolor, title={\color{white}\bfseries #1}]
		#2
	\end{tcolorbox}
}

\newcommand{\defbox}[2][]{%
	\begin{tcolorbox}[colframe=defcolor, title={\color{white}\bfseries #1}]
		#2
	\end{tcolorbox}
}

\newcommand{\probox}[2][]{%
	\begin{tcolorbox}[colframe=procolor, title={\color{white}\bfseries #1}]
		#2
	\end{tcolorbox}
}

\newcommand{\thmbox}[2][]{%
	\begin{tcolorbox}[colframe=thmcolor, title={\color{white}\bfseries #1}]
		#2
	\end{tcolorbox}
}

\newcommand{\lembox}[2][]{%
	\begin{tcolorbox}[colframe=lemcolor, title={\color{white}\bfseries #1}]
		#2
	\end{tcolorbox}
}
\usepackage{amsthm}

% Define custom theorem styles
\newtheoremstyle{dotless} % Name of the style
{3pt} % Space above
{3pt} % Space below
{\itshape} % Body font
{} % Indent amount
{\bfseries} % Theorem head font
{} % Punctuation after theorem head
{2.5mm} % Space after theorem head
{} % Theorem head spec

\newtheoremstyle{definitionstyle} % Name of the style
{3pt} % Space above
{3pt} % Space below
{} % Body font
{} % Indent amount
{\bfseries} % Theorem head font
{.} % Punctuation after theorem head
{2.5mm} % Space after theorem head
{} % Theorem head spec

% Applying custom styles
%\theoremstyle{dotless}
\newtheorem{theorem}{Theorem} % Theorem environment with section-wise numbering
\newtheorem*{theorem*}{Theorem} % Theorem environment with section-wise numbering
\newtheorem*{lemma*}{Lemma} % Theorem environment with section-wise numbering
\newtheorem*{proposition*}{Proposition} % Theorem environment with section-wise numbering
\newtheorem*{corollary*}{Corollary} % Theorem environment with section-wise numbering
\newtheorem{proposition}[theorem]{Proposition} % Theorem environment with section-wise numbering
\newtheorem{lemma}[theorem]{Lemma} % Lemma shares the counter with theorem
\newtheorem{corollary}[theorem]{Corollary} % Corollary shares the counter with theorem

\theoremstyle{definitionstyle}
\newtheorem*{observation}{\textcolor{magenta}{Observation}}
\newtheorem*{illustration}{\textcolor{teal}{Illustration}}
\newtheorem*{torus}{{\color{red}T}{\color{orange}o}{\color{green!75!black}r}{\color{cyan}u}{\color{violet}s}}
\newtheorem{definition}{Definition} % Definition shares the counter with theorem
\newtheorem{example}{Example} % Example shares the counter with theorem
\newtheorem{exercise}{{Exercise}} % Example shares the counter with theorem
\newtheorem{remark}{Remark} % Remark shares the counter with theorem
\newtheorem*{note}{Note}
\newtheorem*{notation}{Notation}

\newtheorem*{axiom*}{Axiom}
\newtheorem*{definition*}{Definition} % Definition shares the counter with theorem
\newtheorem*{example*}{Example} % Example shares the counter with theorem
\newtheorem*{exercise*}{\textcolor{teal}{Exercise}} % Example shares the counter with theorem
\newtheorem*{remark*}{Remark} % Remark shares the counter with theorem


\usepackage{tikz}
\usepackage{tikz-cd}
\usetikzlibrary{shadows}
\usetikzlibrary{shapes.geometric, arrows.meta, positioning}
\newcommand{\ie}{\textnormal{i.e.}}
\newcommand{\rsa}{\mathsf{RSA}}
\newcommand{\rsacrt}{\mathsf{RSA}\textendash\mathsf{CRT}}
\newcommand{\inv}[1]{#1^{-1}}

%New Command
%\newcommand{\set}[1]{\left\{#1\right\}}
\newcommand{\N}{\mathbb{N}}
\newcommand{\Z}{\mathbb{Z}}
\newcommand{\Q}{\mathbb{Q}}
\newcommand{\R}{\mathbb{R}}
\newcommand{\cR}{\mathcal{R}}
\newcommand{\C}{\mathbb{C}}
\newcommand{\F}{\mathbb{F}}
\newcommand{\nbhd}{\mathcal{N}}
\newcommand{\Log}{\operatorname{Log}}
\newcommand{\Arg}{\operatorname{Arg}}
\newcommand{\pv}{\operatorname{P.V.}}

\newcommand{\of}[1]{\left( #1 \right)} 
%\newcommand{\abs}[1]{\left\lvert #1 \right\rvert}
%\newcommand{\norm}[1]{\left\| #1 \right\|}

\newcommand{\sol}{\textcolor{magenta}{\bf Sol}}
\newcommand{\conjugate}[1]{\overline{#1}}

\newcommand{\res}{\operatorname{res}}
\DeclareMathOperator*{\Res}{\operatorname{Res}}

%\renewcommand{\Re}{\operatorname{Re}}
%\renewcommand{\Im}{\operatorname{Im}}

\newcommand{\cyclic}[1]{\langle #1 \rangle}
\newcommand{\uniform}{\overset{\$}{\leftarrow}}
\newcommand{\xmark}{\textcolor{red}{\XSolidBrush}}
\newcommand{\vmark}{\textcolor{green!75!black}{\CheckmarkBold}}

\newcommand{\gen}[1]{\langle #1 \rangle}
\newcommand{\Gen}[1]{\left\langle #1 \right\rangle}

\newcommand{\img}[1]{\text{Img}(#1)}
\newcommand{\Img}[1]{\text{Img}\left(#1\right)}
\newcommand{\preimg}[1]{\text{Img}^{-1}(#1)}
\newcommand{\Preimg}[1]{\text{Img}^{-1}\left(#1\right)}

\newcommand{\relation}{\mathrel{\mathcal{R}}}
\newcommand{\injection}{\rightarrowtail}
\newcommand{\surjection}{\twoheadrightarrow}
\newcommand{\id}{\textnormal{id}}

\newcommand{\eqclass}[1]{\left[#1\right]}

% Define custom colors for O and X
\newcommand{\yes}{\textcolor{blue}{\bf \fullmoon}}
\newcommand{\no}{\textcolor{red}{\bf \texttimes}}

\DeclarePairedDelimiter\ceil{\lceil}{\rceil}
\DeclarePairedDelimiter\floor{\lfloor}{\rfloor}
%\renewcommand{\floor}[#1]{\lfloor #1\rfloor}
%\newcommand{\Floor}[#1]{\left\lfloor #1\right\rfloor}
%\newcommand{\ceil}[#1]{\lceil #1\rceil}
%\newcommand{\Ceil}[#1]{\left\lceil #1\right\rceil}

\newcommand{\topology}{\mathscr{T}}
\newcommand{\sequence}[1]{\langle #1\rangle}
\renewcommand{\vec}[1]{\mathbf{#1}}
\renewcommand{\Re}{\operatorname*{Re}}
\renewcommand{\Im}{\operatorname*{Im}}
\newcommand{\Mat}{\operatorname{Mat}}

\newcommand{\Sym}{\mathrm{Sym}}

\setstretch{1.25}

\begin{document}
\pagenumbering{arabic}
\begin{center}
	\huge\textbf{Linear Algebra III}\\
	\vspace{0.5em}
	\large{Ji, Yong-hyeon}\\
	\vspace{0.5em}
	\normalsize{\today}\\
\end{center}

\noindent 
We cover the following topics in this note.
\begin{itemize}
	\item Determinant and Inverse matrix
\end{itemize}
\hrule\vspace{12pt}
\tableofcontents
\vfill

\newpage
\section{Determinant}
\subsection{Matrix equation and Group structure}
Let $\F$ be a field and let $\Mat_{m\times n}(\F)$ denote the set
of $m\times n$ matrices with entries in $\F$. Given matrices \begin{align*}
A&=(a_{ij})_{1\le i\le m,\;1\le j\le n}=\begin{bmatrix}
	a_{11} & \cdots & a_{1n} \\
	\vdots & \ddots & \vdots \\
	a_{m1} & \cdots & a_{mn}
\end{bmatrix}\in\Mat_{m\times n}(\F),\\
X&=(x_{jk})_{1\le j\le n,\;1\le k\le \ell}=\begin{bmatrix}
	x_{11} & \cdots & x_{1\ell} \\
	\vdots & \ddots & \vdots \\
	x_{n1} & \cdots & x_{n\ell}
\end{bmatrix}\in\Mat_{n\times \ell}(\F),\\
B&=(b_{ik})_{1\le i\le m,\;1\le k\le \ell}=\begin{bmatrix}
	b_{11} & \cdots & b_{1\ell} \\
	\vdots & \ddots & \vdots \\
	b_{m1} & \cdots & b_{m\ell}
\end{bmatrix}\in\Mat_{m\times \ell}(\F),
\end{align*} the matrix equation \[
AX = B\quad\left(\Leftrightarrow\begin{bmatrix}
	a_{11} & \cdots & a_{1n} \\
	\vdots & \ddots & \vdots \\
	a_{m1} & \cdots & a_{mn}
\end{bmatrix}\begin{bmatrix}
x_{11} & \cdots & x_{1\ell} \\
\vdots & \ddots & \vdots \\
x_{n1} & \cdots & x_{n\ell}
\end{bmatrix}=\begin{bmatrix}
b_{11} & \cdots & b_{1\ell} \\
\vdots & \ddots & \vdots \\
b_{m1} & \cdots & b_{m\ell}
\end{bmatrix}\right)
\] means the entries satisfy \[
b_{ik} = \sum_{j=1}^n a_{ij}x_{jk},
\qquad 1\le i\le m,\ 1\le k\le\ell.
\] To solve $AX=B$ by ``dividing by $A$'' from the left, we need an inverse
matrix for $A$.  Thus we restrict to square matrices and look at the
multiplicative structure. Define
\[
G = \operatorname{GL}_n(\F) =
\left\{ A\in\Mat_{n\times n}(\F) :
\exists\,A^{-1}\text{s.t.}AA^{-1}=A^{-1}A=I_n \right\}.
\]
Then $G$ is a group under matrix multiplication.

\newpage
\subsection{Computing $A^{-1}$ by cofactors}
Let $A\in\Mat_{n\times n}(\F)$ and write\[
A=(a_{ij})_{1\le i,j\le n}=\begin{bmatrix}
	a_{11} & \cdots & a_{1n} \\
	\vdots & \ddots & \vdots \\
	a_{n1} & \cdots & a_{nn}
\end{bmatrix}.
\] To find $A^{-1}$ we look for an $n\times n$ matrix \[
X=(x_{kj})_{1\le k,j\le n}
\] such that \[
AX = \begin{bmatrix}
	a_{11} & \cdots & a_{1n} \\
	\vdots & \ddots & \vdots \\
	a_{n1} & \cdots & a_{nn}
\end{bmatrix}\begin{bmatrix}
x_{11} & \cdots & x_{1n} \\
\vdots & \ddots & \vdots \\
x_{n1} & \cdots & x_{nn}
\end{bmatrix}=\begin{bmatrix}
1 & \cdots & 0 \\
\vdots & \ddots & \vdots \\
0 & \cdots & 1
\end{bmatrix}=I_n.
\] In entries this means
\[
\delta_{ij} = \sum_{k=1}^n a_{ik}b_{kj},
\qquad 1\le i,j\le n,
\]
where $\delta_{ij}$ is the \emph{Kronecker delta}:
\[
\delta_{ij} =
\begin{cases}
	1, & i=j,\\[2pt]
	0, & i\neq j.
\end{cases}
\]

\medskip

Recall the Laplace expansion (expansion along the $j$-th column):
\[
\det A = \sum_{k=1}^n a_{kj}C_{kj},
\]
where
\[
C_{kj} = (-1)^{k+j}M_{kj}
\]
is the \emph{cofactor} of the entry $a_{kj}$, and $M_{kj}$ is the
determinant of the $(n-1)\times(n-1)$ minor obtained by deleting the
$k$-th row and $j$-th column of $A$.

Fix a column index $j$.  Set
\[
b_{kj} = \frac{C_{jk}}{\det A},\qquad k=1,\dots,n.
\]
Then
\[
\sum_{k=1}^n a_{ik}b_{kj}
= \frac{1}{\det A}\sum_{k=1}^n a_{ik}C_{jk}
=
\begin{cases}
	1,& i=j,\\
	0,& i\neq j.
\end{cases}
\]
Hence $AB=I_n$.

\medskip

Define the \emph{adjugate matrix} of $A$ by
\[
\operatorname{adj}(A) = (C_{ij})_{1\le i,j\le n}.
\]
In matrix form we have
\[
A^{-1} = \frac{1}{\det A}\,\operatorname{adj}(A)^{\mathsf T}.
\]
Therefore $A^{-1}$ exists only when $\det A\neq 0$.

\subsection{Determinant}

Let $A\in\Mat_{n\times n}(F)$, $A=(a_{ij})$.  The
\emph{determinant} of $A$ is
\[
\det A
=
\sum_{\sigma\in S_n}
\operatorname{sgn}(\sigma)\,
a_{1,\sigma(1)}
a_{2,\sigma(2)}
\cdots
a_{n,\sigma(n)},
\]
where $S_n$ is the symmetric group on $\{1,\dots,n\}$ and
$\operatorname{sgn}(\sigma)\in\{\pm1\}$ is the sign of the permutation
$\sigma$.

\paragraph{Example ($n=2$).}
For
\[
A =
\begin{pmatrix}
	a & b\\
	c & d
\end{pmatrix},
\]
we have
\[
\det A = ad-bc.
\]
Geometrically, if $A$ is regarded as having rows (or columns) the
vectors $(a,b)$ and $(c,d)$ in $\mathbb{R}^2$, then $\det A$ is the
signed area of the parallelogram spanned by these vectors.

\medskip

More generally, for $A\in\Mat_{n\times n}(\mathbb{R})$,
$\det A$ is the signed volume of the parallelotope spanned by the row
vectors of $A$.

\medskip

\noindent\textbf{Exercise.} Show that
\[
\det A = \det A^{\mathsf T}.
\]
Thus $\det A$ is also the signed volume spanned by the column vectors
of $A$.

\medskip

\noindent\textbf{Exercise.} For $A,B\in\Mat_{n\times n}(F)$,
prove that
\[
\det(AB) = (\det A)(\det B).
\]

\subsection*{Equivalent conditions for invertibility}

\begin{proposition}
	Let $A\in\Mat_{n\times n}(F)$.  The following conditions
	are equivalent:
	\begin{enumerate}
		\item $A$ is invertible, i.e.\ there exists $A^{-1}$ such that
		$AA^{-1}=A^{-1}A=I_n$.
		\item The homogeneous system $AX=0$ has only the trivial solution
		$X=0$.
		\item $A$ is row-equivalent to the identity matrix $I_n$.
		\item For every $B\in\Mat_{n\times \ell}(F)$ (any number
		of columns $\ell$), the system $AX=B$ has a solution $X$.
		\item $\det A\neq 0$.
		\item $\operatorname{rank}A = n$ (i.e.\ $A$ has full rank).
		\item The linear map
		\[
		T_A:F^n\to F^n,\qquad T_A(x)=Ax,
		\]
		is onto, i.e.\ $\operatorname{Im}T_A = F^n$ and hence
		$\dim\operatorname{Im}T_A = \operatorname{rank}A = n$.
		\item The row vectors of $A$ are linearly independent.
		\item The column vectors of $A$ are linearly independent.
	\end{enumerate}
\end{proposition}

\begin{proof}
	As one example, we show $(1)\Rightarrow(2)$.
	Suppose $AX=0$ and $A$ is invertible.  Then
	\[
	X = A^{-1}(AX) = A^{-1}0 = 0,
	\]
	so the only solution is $X=0$.
	The remaining implications are proved by standard arguments from linear
	algebra.
\end{proof}

\subsection*{Elementary row operations}

\begin{definition}
	Given any matrix, an \emph{elementary row operation} is one of:
	\begin{enumerate}
		\item Multiply a row by a nonzero scalar.
		\item Interchange two rows.
		\item Replace a row by itself plus a scalar multiple of another row.
	\end{enumerate}
\end{definition}

These operations are used to bring a matrix to row-echelon form and, in
the square case, to compute inverses.

\subsection*{Computing the inverse by row reduction}

To compute the inverse of an invertible matrix
$A\in\Mat_{n\times n}(F)$ by row operations, form the
augmented matrix
\[
\left[\,A \,\bigm|\, I_n\,\right]
\]
and apply elementary row operations until the left block becomes $I_n$:
\[
\left[\,A \,\bigm|\, I_n\,\right]
\;\sim\;
\left[\,I_n \,\bigm|\, A^{-1}\,\right].
\]
Then the right block is $A^{-1}$.

\newpage
\section{Coordinate Change Matrix and Similarity}
%\section*{Coordinate change matrix in \texorpdfstring{$\mathbb{R}^2$}{R2}}
%
%Let us consider $\mathbb{R}^2$.
%
%Take a basis
%\[
%E=\{e_1,e_2\}, \qquad
%e_1=(1,0),\; e_2=(0,1).
%\]
%
%Let us consider another basis
%\[
%F=\{f_1,f_2\},
%\]
%where $f_1,f_2\in\mathbb{R}^2$ are linearly independent
%vectors.
%
%Let $u\in\mathbb{R}^2$.
%
%\medskip
%
%\noindent
%Read $u$ in the basis $E$:
%\[
%u = u_1 e_1 + u_2 e_2, \qquad
%[u]_E =
%\begin{pmatrix}
%	u_1\\ u_2
%\end{pmatrix}.
%\]
%
%On the other hand we can read $u$ in the basis $F$:
%\[
%u = v_1 f_1 + v_2 f_2, \qquad
%[u]_F =
%\begin{pmatrix}
%	v_1\\ v_2
%\end{pmatrix}.
%\]
%
%Write each $f_j$ in the basis $E$:
%\[
%f_1 = a_{11} e_1 + a_{21} e_2,\qquad
%f_2 = a_{12} e_1 + a_{22} e_2.
%\]
%Equivalently,
%\[
%\begin{pmatrix} f_1 & f_2 \end{pmatrix}
%=
%\begin{pmatrix}
%	a_{11} & a_{12}\\
%	a_{21} & a_{22}
%\end{pmatrix}
%\begin{pmatrix}
%	e_1 & e_2
%\end{pmatrix}.
%\]
%
%Thus
%\[
%u = v_1 f_1 + v_2 f_2
%= v_1(a_{11}e_1+a_{21}e_2)
%+ v_2(a_{12}e_1+a_{22}e_2)
%= (a_{11}v_1+a_{12}v_2)e_1
%+ (a_{21}v_1+a_{22}v_2)e_2.
%\]
%Hence
%\[
%[u]_E
%=
%\begin{pmatrix}
%	a_{11} & a_{12}\\
%	a_{21} & a_{22}
%\end{pmatrix}
%[u]_F.
%\]
%
%We call the matrix
%\[
%P = \begin{pmatrix}
%	a_{11} & a_{12}\\
%	a_{21} & a_{22}
%\end{pmatrix}
%\]
%the \emph{coordinate change matrix from $F$ to $E$}.
%
%Thus we have observed that, given a vector $u$,
%\[
%[u]_E = P\,[u]_F.
%\]
%
%\section*{General case}
%
%Let $V$ be a finite-dimensional vector space over a field $\mathbb{F}$.
%For a basis $\beta = \{v_1,\dots,v_n\}$ of $V$ and $u\in V$, we write
%\[
%u = \sum_{i=1}^n a_i v_i
%\quad\Longrightarrow\quad
%[u]_\beta =
%\begin{pmatrix}
%	a_1\\ \vdots\\ a_n
%\end{pmatrix}\!.
%\]
%
%\begin{prop}\label{prop:change-coordinates}
%	Let $V$ be an $n$-dimensional vector space over $\mathbb{F}$ and
%	$\id_V:V\to V$ the identity map.
%	Choose bases $\beta$ and $\gamma$ for $V$.
%	Then for every $u\in V$,
%	\[
%	[u]_\gamma
%	=
%	[\id_V]_{\beta}^{\gamma}\,[u]_\beta,
%	\]
%	where $[\id_V]_{\beta}^{\gamma}$ is the coordinate change matrix
%	from $\beta$ to $\gamma$.
%\end{prop}
%
%\begin{proof}
%	Write
%	\[
%	\beta = \{v_1,\dots,v_n\},\qquad
%	\gamma = \{w_1,\dots,w_n\}.
%	\]
%	
%	Let $u\in V$ and write
%	\[
%	u = \sum_{i=1}^n a_i v_i
%	= \sum_{j=1}^n b_j w_j.
%	\]
%	
%	Each $w_j$ can be expressed in the basis $\beta$:
%	\[
%	w_j = \sum_{i=1}^n c_{ij} v_i, \qquad j=1,\dots,n.
%	\]
%	
%	Then
%	\[
%	u = \sum_{j=1}^n b_j w_j
%	= \sum_{j=1}^n b_j\Bigl(\sum_{i=1}^n c_{ij} v_i\Bigr)
%	= \sum_{i=1}^n
%	\Bigl(\sum_{j=1}^n c_{ij} b_j\Bigr) v_i.
%	\]
%	Comparing coefficients in the basis $\beta$ gives
%	\[
%	a_i = \sum_{j=1}^n c_{ij} b_j,\qquad i=1,\dots,n.
%	\]
%	
%	In matrix form this is
%	\[
%	\bm a = C \bm b,
%	\qquad
%	\bm a =
%	\begin{pmatrix}
%		a_1\\ \vdots\\ a_n
%	\end{pmatrix},\quad
%	\bm b =
%	\begin{pmatrix}
%		b_1\\ \vdots\\ b_n
%	\end{pmatrix},\quad
%	C = (c_{ij})_{1\le i,j\le n}.
%	\]
%	That is,
%	\[
%	[u]_\beta = C\,[u]_\gamma,
%	\]
%	so
%	\[
%	[u]_\gamma = C^{-1}[u]_\beta.
%	\]
%	By definition,
%	\[
%	[\id_V]_{\beta}^{\gamma} = C^{-1},
%	\]
%	whence
%	\[
%	[u]_\gamma = [\id_V]_{\beta}^{\gamma}\,[u]_\beta.
%	\]
%\end{proof}
%
%\subsection*{Matrix of a composition}
%
%\begin{lem}
%	Let $V,W,Z$ be finite-dimensional vector spaces over $\mathbb{F}$ with
%	bases
%	\[
%	\dim V = m,\ \beta=\{v_1,\dots,v_m\},\qquad
%	\dim W = n,\ \gamma=\{w_1,\dots,w_n\},\qquad
%	\dim Z = \ell,\ \delta=\{z_1,\dots,z_\ell\}.
%	\]
%	Let $T:V\to W$ and $S:W\to Z$ be linear transformations.
%	Then
%	\[
%	[S\circ T]_{\beta}^{\delta}
%	=
%	[S]_{\gamma}^{\delta}\,[T]_{\beta}^{\gamma}.
%	\]
%\end{lem}
%
%\section*{Change of basis and invertibility}
%
%\begin{prop}\label{prop:inverse-change}
%	Let $V$ be a finite-dimensional vector space over $\mathbb{F}$ with
%	$\dim V=n$, and let $\beta,\gamma$ be bases for $V$. Then
%	\begin{enumerate}
%		\item $[\id_V]_{\beta}^{\beta} = I_n$.
%		\item $[\id_V]_{\gamma}^{\beta} =
%		\bigl([\id_V]_{\beta}^{\gamma}\bigr)^{-1}$.
%	\end{enumerate}
%	In particular, every coordinate change matrix is invertible.
%	Moreover, every invertible matrix in $\Mat_n(\mathbb{F})$ is a
%	coordinate change matrix for $\mathbb{F}^n$.
%\end{prop}
%
%\begin{proof}
%	The first statement is clear.
%	
%	For (2), consider
%	\[
%	V \xrightarrow{\;\id_V\;} V \xrightarrow{\;\id_V\;} V.
%	\]
%	Apply the lemma to the composition $\id_V\circ\id_V=\id_V$ with
%	appropriate choices of bases.
%	
%	For the last statement, let $A\in\Mat_n(\mathbb{F})$ be invertible.
%	Think of $A$ as a linear map
%	\[
%	L_A:\mathbb{F}^n\to\mathbb{F}^n,\qquad x\mapsto Ax.
%	\]
%	Let $\varepsilon_n=\{e_1,\dots,e_n\}$ be the standard basis of
%	$\mathbb{F}^n$. The $j$-th column of $A$ is $A e_j$.
%	Let
%	\[
%	\beta = \{A e_1,\dots,A e_n\}.
%	\]
%	Then $A$ is invertible iff its columns are linearly independent, i.e.\
%	iff $\beta$ is a basis of $\mathbb{F}^n$.
%	
%	Now, with respect to the bases $\beta$ and $\varepsilon_n$, we have
%	\[
%	[\id_{\mathbb{F}^n}]_{\beta}^{\varepsilon_n} = A.
%	\]
%	Thus $A$ is a coordinate change matrix from $\beta$ to the standard
%	basis $\varepsilon_n$.
%\end{proof}
%
%\section*{Linear operators and similarity}
%
%Let $T:V\to V$ be a linear operator on an $n$-dimensional
%vector space $V$. Choose two bases $\beta$ and $\gamma$ of $V$. We then
%have the matrices $[T]_\beta$ and $[T]_\gamma$.
%
%\medskip
%
%\noindent\textbf{Question.}
%What is the relation between $[T]_\beta$ and $[T]_\gamma$?
%
%\medskip
%
%Consider the chain
%\[
%V \xrightarrow{\;\id_V\;} V
%\xrightarrow{\;T\;} V
%\xrightarrow{\;\id_V\;} V.
%\]
%More precisely,
%\[
%T_\gamma = \id_{\beta}^{\gamma}\circ T\circ\id_{\gamma}^{\beta},
%\]
%where $\id_{\beta}^{\gamma}$ denotes the identity map viewed as
%``change of basis'' from $\beta$-coordinates to $\gamma$-coordinates
%(and $\id_{\gamma}^{\beta}$ is its inverse).
%
%By the lemma,
%\[
%[T]_\gamma
%=
%[\id_V]_{\beta}^{\gamma}\,[T]_\beta\,[\id_V]_{\gamma}^{\beta}.
%\]
%If we set
%\[
%P = [\id_V]_{\gamma}^{\beta},
%\]
%then $P$ is invertible and
%\[
%[T]_\gamma = P^{-1}[T]_\beta P.
%\]
%We say that $[T]_\beta$ is \emph{similar} to $[T]_\gamma$.
%
%\section*{Similarity of matrices}
%
%\begin{defn}
%	Let $A,B\in\Mat_n(\mathbb{F})$.
%	We say that $A$ is \emph{similar} to $B$ if there exists an invertible
%	matrix $P\in\Mat_n(\mathbb{F})$ such that
%	\[
%	B = P^{-1} A P.
%	\]
%	We denote this relation by $B\sim A$.
%\end{defn}
%
%\begin{prop}\label{prop:similar-basis}
%	Let $A,B\in\Mat_n(\mathbb{F})$.
%	Then $B\sim A$ if and only if there exists a basis $\beta$ of
%	$\mathbb{F}^n$ such that
%	\[
%	[L_A]_\beta = B,
%	\]
%	where
%	\[
%	L_A:\mathbb{F}^n\to\mathbb{F}^n,\qquad x\mapsto Ax.
%	\]
%\end{prop}
%
%\begin{proof}
%	($\Rightarrow$)
%	Suppose $B = P^{-1} A P$ for some invertible
%	$P\in\Mat_n(\mathbb{F})$.
%	Let $\varepsilon_n=\{e_1,\dots,e_n\}$ be the standard basis of
%	$\mathbb{F}^n$, and let $p_j$ be the $j$-th column of $P$.
%	Set
%	\[
%	\beta = \{p_1,\dots,p_n\}.
%	\]
%	Then $\beta$ is a basis of $\mathbb{F}^n$, and by the previous
%	discussion
%	\[
%	P = [\id_{\mathbb{F}^n}]_{\beta}^{\varepsilon_n}.
%	\]
%	
%	By the formula for change of basis of the matrix of a linear map, we
%	obtain
%	\[
%	[L_A]_\beta
%	=
%	P^{-1}[L_A]_{\varepsilon_n} P
%	=
%	P^{-1} A P
%	=
%	B.
%	\]
%	
%	\medskip
%	
%	($\Leftarrow$)
%	Conversely, suppose there is a basis $\beta$ of $\mathbb{F}^n$ such
%	that $[L_A]_\beta = B$.
%	Let $P$ be the coordinate change matrix from $\beta$ to the standard
%	basis $\varepsilon_n$:
%	\[
%	P = [\id_{\mathbb{F}^n}]_{\beta}^{\varepsilon_n}.
%	\]
%	Then, again by the change-of-basis formula,
%	\[
%	B = [L_A]_\beta
%	= P^{-1}[L_A]_{\varepsilon_n} P
%	= P^{-1} A P.
%	\]
%	Thus $B\sim A$.
%\end{proof}


\vfill
\begin{thebibliography}{9}
	\bibitem{abstract_algebra_g}
	수학의 즐거움, Enjoying Math. ``수학 공부, 기초부터 대학원 수학까지, 27. 추상대수학에서 선형대수학으로 : 대칭군과 행렬식의 정의 symmetric group and def of determinant'' YouTube Video, 27:07. Published 
	October 29, 2019. URL: \url{https://www.youtube.com/watch?v=UIlC9ikSpNc&t=1026s}.
\end{thebibliography}
\end{document}
