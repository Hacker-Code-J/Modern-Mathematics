\documentclass[tikz,border=10pt]{standalone}
\usepackage{tikz-3dplot}
\usepackage{amsmath, amssymb}
\usetikzlibrary{arrows.meta, positioning, calc, decorations.pathreplacing, 3d}
\usetikzlibrary{matrix, fit, backgrounds, shapes}
\usepackage[T1]{fontenc}
\usepackage[utf8]{inputenc}
\usepackage{newpxtext,newpxmath}
\usepackage{sectsty}
\begin{document}
	% Set the viewing angle for the 3D plot
\tdplotsetmaincoords{70}{120}

\begin{tikzpicture}[
	tdplot_main_coords,
	font=\sffamily,
	>=Stealth
	]
	
	%% ==========================================================
	%% LEFT PANEL: GEOMETRIC VIEW (3D Rotation)
	%% ==========================================================
	
	% Title
%	\node[anchor=south, align=center] at (0,0,3.5) {\large \textbf{Geometric View}};
	
	% 1. Draw Axes (Back)
	\draw[dashed, gray] (0,0,0) -- (-2,0,0);
	\draw[dashed, gray] (0,0,0) -- (0,-2,0);
	
	% 2. Draw Subspace W (The Plane z=0)
	% We draw this before the vectors so vectors appear "on top"
	\fill[teal!10, opacity=0.8] (-2,-2,0) -- (3,-2,0) -- (3,3,0) -- (-2,3,0) -- cycle;
	\draw[teal!40, thick] (-2,-2,0) -- (3,-2,0) -- (3,3,0) -- (-2,3,0) -- cycle;
	\node[teal!60!black] at (2.5, -1.5, 0) {$W$};
	
	% 3. Draw Axes (Front)
	\draw[->, thick] (0,0,0) -- (3,0,0) node[anchor=north east]{$x$};
	\draw[->, thick] (0,0,0) -- (0,3,0) node[anchor=north west]{$y$};
	\draw[->, thick] (0,0,0) -- (0,0,3) node[anchor=south]{$z$};
	
	% 4. Vectors
	% Input Vector w (on x-axis for simplicity)
	\draw[->, thick, blue!80!black] (0,0,0) -- (2,0,0) node[midway, above left] {$w$};
	\coordinate (w) at (2,0,0);
	
	\coordinate (Tw) at (1, 1.732, 0);
	
	% Rotation Arc
	\draw[->, dashed, teal, thick, bend right=30] (w) to node[midway, below right, scale=0.8] {} (Tw);
	
	% Output Vector T(w) (Rotated by ~60 degrees)
	\draw[->, thick, teal!80!black] (0,0,0) -- (1, 1.732, 0) node[midway, below right=12pt] {$T(w)\in W$};
	
	% 5. The "Stability" Indicator
	 Shows that T(w) has NO vertical lift
	\draw[dotted, red, thick] (Tw) -- (1, 1.732, 1.5) node[above, scale=0.7, align=center] {};
	\node[red, scale=2, opacity=0.5] at (1, 1.732, 0.7) {$\times$};
	
	%% ==========================================================
	%% RIGHT PANEL: MATRIX VIEW
	%% ==========================================================
	
	% Switch to 2D scope by resetting coordinate transformation
	\begin{scope}[shift={(7cm, 1cm)}, tdplot_screen_coords]
		
%		\node[anchor=south] at (0, 2.5) {\large \textbf{Matrix View}};
		
		% The Matrix
		\matrix (M) [
		matrix of math nodes,
		nodes={minimum size=1cm, anchor=center},
		left delimiter={[},
		right delimiter={]},
		column sep=5pt,
		row sep=5pt
		] {
			\cos\theta & -\sin\theta & 0 \\
			\sin\theta & \cos\theta & 0 \\
			|[fill=red!10]| 0 & |[fill=red!10]| 0 & 1 \\
		};
		
		% Highlights
		% Block A (Rotation)
		\begin{scope}[on background layer]
			\node[fit=(M-1-1)(M-2-2), draw=blue!30, fill=blue!5, rounded corners, inner sep=2pt] (BlockA) {};
		\end{scope}
		
		% Zeros Highlight
%		\node[right=0.2cm of M-3-2, red, scale=0.8, align=left] (ZeroNote) {$\leftarrow$ No $z$-component};
		
		% Annotations
		
		% Input Basis (Columns)
		\node[above=0.1cm of M-1-1, scale=0.7, gray] {$e_1 \in W$};
		\node[above=0.1cm of M-1-2, scale=0.7, gray] {$e_2 \in W$};
		
		% Output Components (Rows)
		\node[left=0.3cm of M-1-1, scale=0.7, gray] {$x$};
		\node[left=0.3cm of M-2-1, scale=0.7, gray] {$y$};
		\node[left=0.3cm of M-3-1, scale=0.7, gray] {$z$};
		
%		 Connection Text
%					\node[below=1cm of M, align=center, width=6cm] {
%							\textbf{The Logic:}\\
%							Since $T(w)$ lies flat on the $xy$-plane,\\
%							the coefficient for $z$ (row 3) must be \textbf{0}.
%						};
	\end{scope}
\end{tikzpicture}
	
%\begin{tikzpicture}[
%	>=Stealth,
%	line cap=round,
%	line join=round,
%	font=\sffamily
%	]
%	
%	%% --- LEFT PANEL: GEOMETRIC VIEW (3D) ---
%	\begin{scope}[xshift=0cm, scale=1.2]
%		
%		% 2. Draw Subspace W (The Plane z=0)
%		% We simulate 3D perspective
%		\fill[blue!10, opacity=0.8] (-2, -1) -- (2, -1) -- (3, 1.5) -- (-1, 1.5) -- cycle;
%		\draw[blue!40, thick] (-2, -1) -- (2, -1) -- (3, 1.5) -- (-1, 1.5) -- cycle;
%		\node[blue!60!black] at (2.5, 1.2) {$W$};
%		\node[scale=1] at (1, 2.5) {$V$};
%		
%		% 3. Origin
%		\fill[black] (0,0) circle (1.5pt) node[below left] {$\mathbf{0}$};
%		
%		% 4. Vector w inside W
%		\draw[->, thick, blue!80!black] (0,0) -- (1.5, -0.2) node[midway, below] {$w$};
%		\coordinate (w_tip) at (1.5, -0.2);
%		
%		% 5. Vector T(w) inside W
%		\draw[->, thick, teal!80!black] (0,0) -- (0.5, 0.8) node[midway, left] {$T(w)$};
%		\coordinate (Tw_tip) at (0.5, 0.8);
%		
%		% 6. The Transformation Arrow
%		\draw[->, dashed, orange, thick, bend right=30] (w_tip) to node[auto, swap, scale=0.8] {$T$} (Tw_tip);
%		
%%		% 7. The "Forbidden" Mapping (What stability prevents)
%%		\draw[->, dashed, red, opacity=0.4] (0,0) -- (1.5, 2.5) node[right, red, opacity=1] {$\notin W$};
%%		\coordinate (bad_tip) at (1.5, 2.5);
%		
%%		% Cross out the bad mapping
%%		\draw[red, thick] ($(bad_tip)!0.5!(w_tip) + (-0.2, -0.2)$) -- ($(bad_tip)!0.5!(w_tip) + (0.2, 0.2)$);
%%		\draw[red, thick] ($(bad_tip)!0.5!(w_tip) + (-0.2, 0.2)$) -- ($(bad_tip)!0.5!(w_tip) + (0.2, -0.2)$);
%%		\node[red, scale=0.7, rotate=45] at ($(bad_tip)!0.5!(w_tip) + (0.4, 0)$) {Impossible};
%		
%		% 1. Draw Coordinate System (Background)
%		\draw[->] (0,-2,0) -- (0,2,0) node[above] {}; % Up represents V \ W direction
%		\draw[->] (-2,0,0) -- (3,0,0) node[above] {}; % Up represents V \ W direction
%		\draw[->] (0,0,3) -- (0,0,-5) node[above] {}; % Up represents V \ W direction
%	\end{scope}
%	
%	%% --- DIVIDER ---
%	\draw[gray!30, thick, dashed] (5, -4) -- (5, 4.5);
%	
%	%% --- RIGHT PANEL: ALGEBRAIC VIEW ---
%	\begin{scope}[xshift=6cm, yshift=-2.75cm]
%		
%		% Matrix Outline
%		\draw[thick] (0,0) rectangle (5,6);
%		
%		% Partition Lines
%		\draw[thick] (2,0) -- (2,6); % Vertical
%		\draw[thick] (0,2) -- (5,2); % Horizontal
%		
%		% Basis Labels (Top)
%		\node[text=blue!80!black] at (1, 6.3) {Basis of $W$};
%%		\node[scale=0.8] at (1, 4.6) {Input $w \in W$};
%		\node[text=gray] at (3.5, 6.3) {Basis of $V \setminus W$};
%		
%		% Basis Labels (Left)
%		\node[rotate=90, text=blue!80!black] at (-0.3, 4) {Coeffs in $W$};
%		\node[rotate=90, text=gray] at (-0.3, 1) {Coeffs in $V \setminus W$};
%		
%		% Blocks
%		\node[scale=1.5] at (1, 4) {$A$};
%		\node[scale=1.5] at (3.5, 4) {$*$};
%		\node[scale=1.5] at (3.5, 1) {$B$};
%		
%		% The Zero Block Highlight
%		\fill[red!5] (0.1, 0.1) rectangle (1.9, 1.9);
%		\node[scale=2, text=red] at (1, 1) {$\mathbf{0}$};
%		
%		% Explanation of Zero
%		\draw[<-, thick, red] (1, 0.8) -- (1, -0.5) node[below, align=center, text=black, scale=0.8] {
%			\textbf{CRITICAL:} \\
%			When input is in $W$, \\
%			output has \textbf{0} component \\
%			outside of $W$.
%		};
%		
%		% Connection to Logic
%		\node[anchor=west, scale=0.85, text width=5cm] at (5.5, 2) {
%			\textbf{Why?} \\
%			If $w \in W$, then $T(w) \in W$. \\
%			$T(w)$ is a linear combo of \\
%			\textcolor{blue}{blue basis vectors} only. \\
%			$\therefore$ Coeffs for the extension basis are 0.
%		};
%		
%	\end{scope}
%\end{tikzpicture}
\end{document}