\documentclass[tikz,border=6pt]{standalone}
\usepackage{amsmath,amssymb}
\usetikzlibrary{arrows.meta,calc,angles,quotes}

\begin{document}
	\begin{tikzpicture}[scale=4, line cap=round, line join=round, >=Latex]
		
		% ---------------------------
		% Concrete data
		% ---------------------------
		% Work in the xz-plane (y=0). Then S^2 ∩ {y=0} is the unit circle x^2+z^2=1.
		% Use p = (0,1) (north pole in this cross-section).
		% Use q = (sin eps, cos eps), where eps is a small concrete angle.
		\def\epsdeg{18} % choose a visible small step
		\pgfmathsetmacro{\sE}{sin(\epsdeg)}
		\pgfmathsetmacro{\cE}{cos(\epsdeg)}
		\pgfmathsetmacro{\epsrad}{\epsdeg*pi/180}
		
		% Points
		\coordinate (O) at (0,0);
		\coordinate (p) at (0,1);
		\coordinate (q) at (\sE,\cE);
		
		% Tangent vector v at p corresponds to (1,0) in this cross-section
		% (really (1,0,0) in R^3).
		\def\vlen{0.55}
		\coordinate (pv)  at ($(p)+( \vlen,0)$);
		\coordinate (pmv) at ($(p)+(-\vlen,0)$);
		
		% Extend normals a bit for visibility
		\def\nlen{0.55}
		\coordinate (NpEnd) at ($(p)+(0,\nlen)$);                 % N(p) points upward
		\coordinate (NqEnd) at ($(q)+(\nlen*\sE,\nlen*\cE)$);     % N(q) points radially
		
		% ---------------------------
		% Axes
		% ---------------------------
		\draw[->] (-1.15,0) -- (1.15,0) node[below] {$x$};
		\draw[->] (0,-0.15) -- (0,1.35) node[left] {$z$};
		
		% ---------------------------
		% Unit circle (cross-section of S^2)
		% ---------------------------
		\draw[thick] (0,0) circle (1);
		
		% Draw the arc of gamma from t=0 to t=eps
		\draw[red, very thick]
		(p) arc[start angle=90, end angle=90-\epsdeg, radius=1]
		node[pos=0.6, above right] {$\gamma(t)$};
		
		% Mark points
		\fill (p) circle (0.5pt) node[above left] {$p=\gamma(0)=(0,0,1)$};
		\fill (q) circle (0.5pt) node[below right] {$q=\gamma(\varepsilon)=(\sin\varepsilon,0,\cos\varepsilon)$};
		
		% ---------------------------
		% Tangent line at p (z=1 in 3D, here: horizontal line through (0,1))
		% ---------------------------
		\draw[gray!65] (-0.95,1) -- (0.95,1);
		\node[gray!65] at (0.88,1.06) {$T_pS^2$};
		
%		% ---------------------------
%		% Normal vectors N(p) and N(q)
%		% On S^2 with outward normal: N(x)=x.
%		% ---------------------------
%		\draw[very thick,->] (p) -- (NpEnd) node[above] {$N(p)=p$};
%		\draw[very thick,->] (q) -- (NqEnd) node[right] {$N(q)=q$};
		
%		% ---------------------------
%		% Tangent direction v and S_p(v) = -v (Weingarten convention S = -dN)
%		% ---------------------------
%		\draw[very thick,->] (p) -- (pv)  node[midway, above] {$v=\gamma'(0)=(1,0,0)$};
%		\draw[very thick,->] (p) -- (pmv) node[midway, above] {$S_p(v)=-v$};
		
%		% ---------------------------
%		% Secant chord q - p and difference quotient picture
%		% Since N = Id on S^2, N(q)-N(p)=q-p.
%		% ---------------------------
%		\draw[dashed, thick] (p) -- (q) node[midway, below right] {$q-p=N(q)-N(p)$};
		
%		% A small right-angle marker to emphasize that v is tangent at p
%		\draw pic["$\,$", draw=black, angle radius=6pt, angle eccentricity=1.3]
%		{right angle=($(p)+(0.25,0)$)--p--($(p)+(0,0.25)$)};
		
		% ---------------------------
		% Computation block (explicit limit)
		% ---------------------------
		\node[align=left] at (-1.10,-2.55) {%
			\(\displaystyle \gamma(t)=(\sin t,0,\cos t)\)\\
			\(\displaystyle \gamma'(t)=(\cos t,0,-\sin t)\Rightarrow \gamma'(0)=(1,0,0)=v\)\\[4pt]
			Outward normal on \(S^2\): \(\displaystyle N(x)=x\).\\
			So \(\displaystyle N(\gamma(t))=\gamma(t)\). Hence\\[2pt]
			\(\displaystyle \frac{N(\gamma(\varepsilon))-N(\gamma(0))}{\varepsilon}
			=\frac{\gamma(\varepsilon)-p}{\varepsilon}
			=\left(\frac{\sin\varepsilon}{\varepsilon},\,0,\,\frac{\cos\varepsilon-1}{\varepsilon}\right)\)\\[4pt]
			\(\displaystyle \xrightarrow[\varepsilon\to 0]{}(1,0,0)=v\Rightarrow (dN)_p(v)=v\).\\
			\(\displaystyle S_p(v)=-(dN)_p(v)=-v\Rightarrow S_p=-\mathrm{Id}_{T_pS^2}.\)
		};
		
%		% Title
%		\node[font=\bfseries] at (0,1.28) {Shape operator on \(S^2\) (clear cross-section \(y=0\))};
		
	\end{tikzpicture}
\end{document}
