\chapter{Preliminaries}
\section{Sets, Cartesian Products, and Relations}

\subsection{Sets and Ordered Pairs}

\subsection*{Set}

A \textbf{set} is a well-defined collection of distinct objects, called elements or members of the set. Sets are one of the most fundamental concepts in mathematics.

\defbox[Set]{
\begin{definition}
	A \textbf{set} is a well-defined collection of distinct objects, considered as an object in its own right. Sets are usually denoted by capital letters, and the elements are listed within curly braces.
\end{definition}
}
\begin{example}
For example:
\[
A = \{1, 2, 3\}
\]
This denotes a set \(A\) containing the elements 1, 2, and 3.
\end{example}
\vspace{12pt}
\begin{note}[Properties]
\ \begin{itemize}
	\item \textbf{No Repetition}: Each element in a set appears only once.
	\item \textbf{Order Irrelevance}: The order of elements in a set does not matter. For instance, \(\{1, 2, 3\} = \{3, 2, 1\}\).
	\item \textbf{Membership}: If an element \(a\) is in a set \(A\), we write \(a \in A\).
\end{itemize}
\end{note}
\vspace{12pt}
\begin{note}[Types of Sets]
\ \begin{itemize}
	\item \textbf{Finite and Infinite Sets}: A set with a finite number of elements is finite; otherwise, it is infinite.
	\item \textbf{Subset}: A set \(A\) is a subset of a set \(B\) if every element of \(A\) is also an element of \(B\), denoted \(A \subseteq B\).
	\item \textbf{Power Set}: The power set of \(A\) is the set of all subsets of \(A\), denoted \(\mathcal{P}(A)\).
\end{itemize}
\end{note}

\subsection*{Ordered Pair}

An \textbf{ordered pair} is a fundamental concept in mathematics used to combine two elements in a specific order. The notation for an ordered pair is \((a, b)\), where \(a\) is the first element and \(b\) is the second element.

\defbox[Ordered Pair]{
\begin{definition}
	An \textbf{ordered pair} \((a, b)\) is a collection of two elements where the order of the elements matters. This is in contrast to a set, where the order of elements does not matter.
\end{definition}
}
\begin{remark}
	\ \begin{itemize}
		\item The ordered pair \((a, b)\) is not the same as \((b, a)\) unless \(a = b\).
		\item Formally, the ordered pair \((a, b)\) can be defined using sets to ensure the distinction from unordered pairs. One common definition is:
		\[
		(a, b) = \{ \{a\}, \{a, b\} \}
		\]
		This definition ensures that:
		\[
		(a, b) = (c, d) \iff a = c \ \text{and} \ b = d
		\]
	\end{itemize}
\end{remark}
\vspace{12pt}
\begin{note}[Properties]
\begin{itemize}
	\item \textbf{Uniqueness}: Each ordered pair \((a, b)\) is unique if either \(a\) or \(b\) is unique.
	\item \textbf{Order}: The order of elements in an ordered pair is significant.
\end{itemize}
\end{note}

\subsection{Cartesian Product and Relation}

\subsection*{Cartesian Product}

The \textbf{Cartesian product} is a fundamental concept in set theory, used to define the set of all possible ordered pairs from two sets.

\defbox[Cartesian Product]{
Given two sets \( A \) and \( B \), the Cartesian product \( A \times B \) is defined as the set of all ordered pairs \((a, b)\) where \( a \in A \) and \( b \in B \). Formally,
\[
A \times B = \{ (a, b) \mid a \in A \ \text{and} \ b \in B \}
\]
}
\vspace{12pt}
\begin{note}[Properties]
\ \begin{itemize}
	\item \textbf{Order Matters}: The pair \((a, b)\) is different from the pair \((b, a)\) unless \(a = b\).
	\item \textbf{Empty Set}: If either \(A\) or \(B\) is the empty set \(\emptyset\), then \(A \times B\) is also empty:
	\[
	A \times \emptyset = \emptyset \quad \text{and} \quad \emptyset \times B = \emptyset
	\]
\end{itemize}
\end{note}
\begin{example}
\ \begin{enumerate}
	\item If \( A = \{1, 2\} \) and \( B = \{x, y\} \), then
	\[
	A \times B = \{ (1, x), (1, y), (2, x), (2, y) \}
	\]
	
	\item If \( A = \{a, b\} \) and \( B = \{1, 2, 3\} \), then
	\[
	A \times B = \{ (a, 1), (a, 2), (a, 3), (b, 1), (b, 2), (b, 3) \}
	\]
\end{enumerate}
\end{example}

\subsection*{Relation}

A \textbf{relation} generalizes the concept of Cartesian product to establish connections between elements of two sets.

\defbox[Relation]{
\begin{definition}
	A relation \( R \) from a set \( A \) to a set \( B \) is a subset of the Cartesian product \( A \times B \). Formally,
	\[
	R \subseteq A \times B
	\]
	This means that a relation \( R \) consists of ordered pairs \((a, b)\) where \(a \in A\) and \(b \in B\).
\end{definition}
}
\vspace{12pt}
\begin{note}[Properties of Relations]
\ \begin{itemize}
	\item \textbf{Domain and Range}:
	\begin{itemize}
		\item The \textbf{domain} of \( R \) is the set of all \( a \in A \) such that there exists \( b \in B \) with \((a, b) \in R\).
		\[
		\text{Domain}(R) = \{ a \in A \mid \exists b \in B, \ (a, b) \in R \}
		\]
		\item The \textbf{range} of \( R \) is the set of all \( b \in B \) such that there exists \( a \in A \) with \((a, b) \in R\).
		\[
		\text{Range}(R) = \{ b \in B \mid \exists a \in A, \ (a, b) \in R \}
		\]
	\end{itemize}
	
	\item \textbf{Inverse Relation}: The inverse of a relation \( R \), denoted \( R^{-1} \), is the set of all pairs \((b, a)\) such that \((a, b) \in R\):
	\[
	R^{-1} = \{ (b, a) \mid (a, b) \in R \}
	\]
	
	\item \textbf{Composition of Relations}: Given a relation \( R \) from \( A \) to \( B \) and a relation \( S \) from \( B \) to \( C \), the composition \( S \circ R \) is a relation from \( A \) to \( C \) defined by:
	\[
	S \circ R = \{ (a, c) \mid \exists b \in B, \ (a, b) \in R \ \text{and} \ (b, c) \in S \}
	\]
\end{itemize}
\end{note}

\vspace{12pt}
\begin{note}[Types of Relations]
\ \begin{itemize}
	\item \textbf{Binary Relation}: A relation involving two sets, as defined above.
	\item \textbf{Unary Relation}: A relation on a single set \( A \) is simply a subset of \( A \).
	\item \textbf{Ternary and Higher Relations}: Relations involving three or more sets, defined as subsets of the Cartesian product of those sets.
\end{itemize}
\end{note}

\begin{example}
	\ \begin{enumerate}
		\item If \( A = \{1, 2, 3\} \) and \( B = \{a, b\} \), a possible relation \( R \) from \( A \) to \( B \) could be:
		\[
		R = \{ (1, a), (2, b), (3, a) \}
		\]
		\begin{itemize}
			\item Domain: \(\{1, 2, 3\}\)
			\item Range: \(\{a, b\}\)
		\end{itemize}
		
		\item Consider the relation \( R \) on set \( A = \{1, 2, 3\} \) defined by:
		\[
		R = \{ (1, 2), (2, 3), (3, 1) \}
		\]
		\begin{itemize}
			\item Domain: \(\{1, 2, 3\}\)
			\item Range: \(\{1, 2, 3\}\)
			\item Inverse Relation: \( R^{-1} = \{ (2, 1), (3, 2), (1, 3) \} \)
		\end{itemize}
	\end{enumerate}
\end{example}

\section{Rational Number and Equivalence Class}

We define the equivalence relation \((a, b) \sim (c, d)\) on the set \(\mathbb{Z} \times \mathbb{Z}^*\) as:
\[
(a, b) \sim (c, d) \iff ad = bc
\]
\begin{proof}
To prove that \(\sim\) is an equivalence relation, we must show it is reflexive, symmetric, and transitive.
\begin{itemize}
	\item \textbf{Reflexive}: A relation \(\sim\) is reflexive if every element is related to itself.
	
	For any \((a, b) \in \mathbb{Z} \times \mathbb{Z}^*\), we need to show that \((a, b) \sim (a, b)\).
	\[
	(a, b) \sim (a, b) \iff ab = ba
	\]
	This is true because \(ab = ba\) holds for all integers \(a\) and \(b\).
	Thus, the relation is reflexive.
	\item \textbf{Symmetric}: A relation \(\sim\) is symmetric if whenever \((a, b) \sim (c, d)\), then \((c, d) \sim (a, b)\).
	
	Assume \((a, b) \sim (c, d)\). This means:
	\[
	ad = bc
	\]
	We need to show that \((c, d) \sim (a, b)\).
	\[
	(c, d) \sim (a, b) \iff cd = da
	\]
	Since \(ad = bc\), we have \(cd = da\) by the commutative property of multiplication.
	Thus, the relation is symmetric.
	\item \textbf{Transitive}: A relation \(\sim\) is transitive if whenever \((a, b) \sim (c, d)\) and \((c, d) \sim (e, f)\), then \((a, b) \sim (e, f)\).
	
	Assume \((a, b) \sim (c, d)\) and \((c, d) \sim (e, f)\). This means:
	\[
	ad = bc \quad \text{and} \quad cf = de
	\]
	We need to show that \((a, b) \sim (e, f)\).
	\[
	(a, b) \sim (e, f) \iff af = be
	\]
	From \(ad = bc\), we have \(d = \frac{bc}{a}\) (assuming \(a \neq 0\)). Substituting \(d\) into \(cf = de\):
	\[
	c f = \left(\frac{bc}{a}\right) e
	\]
	Multiplying both sides by \(a\):
	\[
	a c f = b c e
	\]
	Since \(c \neq 0\):
	\[
	a f = b e
	\]
	Thus, \((a, b) \sim (e, f)\), proving that the relation is transitive.	
\end{itemize}
\end{proof}

Since the relation \((a, b) \sim (c, d) \iff ad = bc\) is reflexive, symmetric, and transitive, it is an equivalence relation on \(\mathbb{Z} \times \mathbb{Z}^*\).


The equivalence relation \((a, b) \sim (c, d) \iff ad = bc\) naturally connects to the division of the set of pairs of integers into equivalence classes, where each equivalence class represents a unique rational number.

An equivalence class \([(a, b)]\) under this relation consists of all pairs \((c, d)\) such that \((a, b) \sim (c, d)\). This can be interpreted as:
\[
[(a, b)] = \{ (c, d) \mid ad = bc \}
\]

Each equivalence class \([(a, b)]\) corresponds to the rational number \(\frac{a}{b}\), and different pairs \((a, b)\) and \((c, d)\) represent the same rational number if and only if they belong to the same equivalence class.

\begin{example}
Consider \((1, 2)\), \((2, 4)\), \((3, 6)\), \((-1, -2)\), \((1, -2)\)
\begin{itemize}
	\item (Class of $(1,2)$)
\[
[1, 2] = \{ (c, d) \in \mathbb{Z} \times \mathbb{Z}^* \mid 1d = 2c \} = \{ (1, 2), (2, 4), (3, 6), (-1, -2), \ldots \}
\]
This class represents the rational number \(\frac{1}{2}\).
\item (Class of $(2,3)$)
\[
[2, 3] = \{ (c, d) \in \mathbb{Z} \times \mathbb{Z}^* \mid 2d = 3c \} = \{ (2, 3), (4, 6), (-2, -3), \ldots \}
\]
This class represents the rational number \(\frac{2}{3}\).
\item (Class of $(1,-2)$)
\[
[1, -2] = \{ (c, d) \in \mathbb{Z} \times \mathbb{Z}^* \mid 1d = -2c \} = \{ (1, -2), (2, -4), (-1, 2), \ldots \}
\]
This class represents the rational number \(\frac{1}{-2}\).
\end{itemize}
\end{example}

\begin{remark}[Properties of Partition]
\ \begin{itemize}
	\item (Disjoint) Each element of \(\mathbb{Z} \times \mathbb{Z}^*\) belongs to exactly one equivalence class. If \((a, b) \in [c, d]\), then \([a, b] = [c, d]\).
	\item (Exhaustive) The union of all equivalence classes covers the entire set \(\mathbb{Z} \times \mathbb{Z}^*\). Every pair \((a, b) \in \mathbb{Z} \times \mathbb{Z}^*\) is in some equivalence class.
\end{itemize}
\end{remark}

\section*{Bijection Between \(\mathbb{Z} \times \mathbb{Z}^*\) and \(\mathbb{N}\)}

To establish a bijection between \(\mathbb{Z} \times \mathbb{Z}^*\) and \(\mathbb{N}\), we construct a function that maps each pair \((a, b)\) in \(\mathbb{Z} \times \mathbb{Z}^*\) to a unique natural number.

\subsection*{Encoding Integers as Natural Numbers}

Define the encoding function \(e: \mathbb{Z} \to \mathbb{N}\) as follows:
\[
e(a) =
\begin{cases}
	2a & \text{if } a \geq 0 \\
	-2a - 1 & \text{if } a < 0
\end{cases}
\]

Define the encoding function \(e^*: \mathbb{Z}^* \to \mathbb{N}\) similarly:
\[
e^*(b) =
\begin{cases}
	2b & \text{if } b > 0 \\
	-2b - 1 & \text{if } b < 0
\end{cases}
\]

\subsection*{Pairing Function}

Define a pairing function \(\pi: \mathbb{N} \times \mathbb{N} \to \mathbb{N}\) by:
\[
\pi(n_1, n_2) = \frac{(n_1 + n_2)(n_1 + n_2 + 1)}{2} + n_2
\]

\subsection*{Mapping Function}

Define the mapping function \(f: \mathbb{Z} \times \mathbb{Z}^* \to \mathbb{N}\) by:
\[
f(a, b) = \pi(e(a), e^*(b))
\]

\subsection*{Bijection}

To show that \(f\) is a bijection, we need to prove that it is both injective and surjective.

\subsubsection*{Injectivity}

Assume \(f(a, b) = f(c, d)\). This implies:
\[
\pi(e(a), e^*(b)) = \pi(e(c), e^*(d))
\]
Since \(\pi\) is injective, we have:
\[
(e(a), e^*(b)) = (e(c), e^*(d))
\]
This implies \(e(a) = e(c)\) and \(e^*(b) = e^*(d)\), which in turn implies \(a = c\) and \(b = d\).

Thus, \(f\) is injective.

\subsubsection*{Surjectivity}

Let \(n \in \mathbb{N}\). We need to find \((a, b) \in \mathbb{Z} \times \mathbb{Z}^*\) such that \(f(a, b) = n\).

Since \(\pi\) is surjective, there exist \(n_1, n_2 \in \mathbb{N}\) such that:
\[
n = \pi(n_1, n_2)
\]
Using the inverse of \(e\) and \(e^*\), we can find \(a\) and \(b\) such that:
\[
e(a) = n_1 \quad \text{and} \quad e^*(b) = n_2
\]

Thus, \(f(a, b) = n\), and \(f\) is surjective.

\subsection*{Conclusion}

Since \(f\) is both injective and surjective, it is a bijection. Therefore, \(\mathbb{Z} \times \mathbb{Z}^*\) is in one-to-one correspondence with \(\mathbb{N}\).

\section*{Invalid Equivalence Classes and Proof Using Natural Numbers and Integers}

Consider the equivalence relation \((a, b) \sim (c, d)\) defined by:
\[
(a, b) \sim (c, d) \iff ad = bc
\]
where \((a, b), (c, d) \in \mathbb{Z} \times \mathbb{Z}^*\) and \(\mathbb{Z}^* = \mathbb{Z} \setminus \{0\}\).

\subsection*{Invalid Equivalence Classes}

\subsubsection*{Equivalence Classes Overview}

In mathematics, an equivalence class under a given equivalence relation is a subset formed by grouping all elements related to each other by that relation. For the relation \((a, b) \sim (c, d) \iff ad = bc\) on \(\mathbb{Z} \times \mathbb{Z}^*\), each equivalence class represents a unique rational number.

\subsubsection*{Valid Equivalence Classes}

Under the relation \((a, b) \sim (c, d) \iff ad = bc\) with \( b \neq 0 \) and \( d \neq 0 \), each equivalence class \([a, b]\) includes all pairs \((c, d)\) such that \( ad = bc \). Formally:
\[
[a, b] = \{(c, d) \in \mathbb{Z} \times \mathbb{Z}^* \mid ad = bc\}
\]
These equivalence classes correspond to unique relationships between pairs of integers.

\subsubsection*{Invalid Equivalence Classes with Zero Denominator}

When \( b = 0 \) or \( d = 0 \), the equivalence relation breaks down because:

\begin{itemize}
	\item \textbf{Undefined Products}: A pair \((a, 0)\) does not represent a valid mathematical entity since \(a \cdot 0\) is not meaningful in the context of this relation.
	\item \textbf{Equivalence Condition Breakdown}: If \( b = 0 \) or \( d = 0 \), the condition \( ad = bc \) can lead to contradictions or meaningless comparisons.
\end{itemize}

\subsection*{Proof: If \( b = 0 \) or \( d = 0 \), Then \((a, b)\) is Not Equivalent to \((c, d)\)}

\subsubsection*{Case 1: \( b = 0 \) and \( d \neq 0 \)}

Suppose \((a, 0) \sim (c, d)\). According to the equivalence relation:
\[
a \cdot d = 0 \cdot c \implies ad = 0
\]
For this to hold, at least one of \(a\) or \(d\) must be zero. Given that \(d \neq 0\), we must have \(a = 0\). Thus:
\[
(a, 0) \sim (0, d)
\]
This implies that \((a, 0)\) can only be equivalent to pairs of the form \((0, d)\).

\subsubsection*{Case 2: \( d = 0 \) and \( b \neq 0 \)}

Suppose \((a, b) \sim (c, 0)\). According to the equivalence relation:
\[
a \cdot 0 = b \cdot c \implies 0 = bc
\]
For this to hold, at least one of \(b\) or \(c\) must be zero. Given that \(b \neq 0\), we must have \(c = 0\). Thus:
\[
(a, b) \sim (0, 0)
\]
This implies that \((c, 0)\) can only be equivalent to pairs of the form \((a, 0)\).

\subsubsection*{Case 3: Both \( b = 0 \) and \( d = 0 \)}

Suppose \((a, 0) \sim (c, 0)\). According to the equivalence relation:
\[
a \cdot 0 = 0 \cdot c \implies 0 = 0
\]
This is trivially true, so pairs of the form \((a, 0)\) are all equivalent to each other, regardless of the value of \(a\) or \(c\).

\subsection*{Conclusion}

When \( b = 0 \) or \( d = 0 \), the equivalence relation \((a, b) \sim (c, d) \iff ad = bc\) does not define valid equivalence classes that can represent meaningful relationships between pairs of integers. These invalid equivalence classes do not correspond to well-defined mathematical entities because they involve undefined or meaningless products.
