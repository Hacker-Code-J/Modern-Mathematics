% !TEX TS-program = pdflatex
\documentclass[11pt]{article}

\usepackage[a4paper,margin=1in]{geometry}
\usepackage{amsmath,amssymb,amsfonts}
\usepackage{mathtools}
\usepackage{tikz-cd}
\usepackage{graphicx}
\usepackage[T1]{fontenc}
\usepackage[utf8]{inputenc}
\usepackage{microtype}
\usepackage{enumitem}
\usepackage{hyperref}

\hypersetup{
	colorlinks=true,
	linkcolor=blue,
	urlcolor=blue,
	citecolor=blue
}

\title{Reading Grothendieck's ``Riemann--Roch'' Doodle\\(Algebraic Geometry Notes)}
\author{}
\date{}

\begin{document}
	\maketitle
	
	\begin{figure}[ht]
		\centering
		% Put Grothentick-RR.webp next to this .tex file (or convert to PNG/PDF if your LaTeX can't include webp).
%		\includegraphics[width=\linewidth]{Grothentick-RR.webp}
		\caption{Grothendieck's doodle (scan/photo).}
	\end{figure}
	
	\tableofcontents
	\bigskip
	
	\section{What the doodle is about}
	Grothendieck is gesturing at the \emph{Grothendieck--Riemann--Roch theorem (GRR)}:
	a compatibility between
	\begin{itemize}[itemsep=2pt]
		\item pushforward in \emph{$K$-theory of coherent sheaves}, and
		\item pushforward in \emph{intersection theory} (Chow groups / cycles),
	\end{itemize}
	after applying a universal ``bridge'' built from the \emph{Chern character} and the
	\emph{Todd class}. The joke ``the latest craze: the diagram'' is that the correct statement
	is best expressed as a \emph{commuting square}, and setting up the frameworks in full generality
	is genuinely lengthy.
	
	\section{Geometric setting and hypotheses}
	
	\subsection{The morphism}
	Let $f\colon X\to Y$ be a morphism of schemes (or varieties) of finite type over a field.
	
	\subsection{Properness}
	Assume $f$ is \emph{proper}. This is the condition ensuring that pushforward of coherent sheaves
	preserves coherence and that Euler characteristics behave well. Properness is also the hypothesis
	under which Chow groups admit a natural pushforward.
	
	\subsection{Smoothness (recommended for a first pass)}
	For the cleanest first statement, assume $X$ and $Y$ are \emph{smooth} quasi-projective varieties
	(or smooth schemes of finite type). Smoothness ensures:
	\begin{itemize}[itemsep=2pt]
		\item the tangent bundles $T_X$ and $T_Y$ exist as vector bundles,
		\item Chern classes and Todd classes are straightforward to define, and
		\item $K$-theory of vector bundles and $K$-theory of coherent sheaves agree (see below).
	\end{itemize}
	
	\section{The two worlds GRR connects}
	
	\subsection{$K$-theory of coherent sheaves: $G_0$ and $K_0$}
	
	\paragraph{Two Grothendieck groups.}
	There are two closely related Grothendieck groups:
	\begin{itemize}[itemsep=2pt]
		\item $K_0(X)$: the Grothendieck group of \emph{vector bundles} (locally free sheaves) on $X$.
		\item $G_0(X)$: the Grothendieck group of \emph{coherent sheaves} on $X$.
	\end{itemize}
	
	\paragraph{Smooth case identification.}
	If $X$ is smooth (regular), every coherent sheaf admits a finite locally free resolution, and one has
	a canonical isomorphism $K_0(X)\cong G_0(X)$.
	
	\paragraph{Proper pushforward on $G_0$.}
	If $f\colon X\to Y$ is proper, there is a natural pushforward
	\[
	f_*\colon G_0(X)\to G_0(Y),\qquad
	[\mathcal{F}] \longmapsto \sum_{i\ge 0}(-1)^i\,[R^i f_*\mathcal{F}].
	\]
	This is the algebraic-geometric meaning of the doodle's upper horizontal arrow (often written $f_*$,
	and sometimes $f_!$ in ``wrong-way'' notation).
	
	\subsection{Chow groups and pushforward}
	
	\paragraph{Chow groups.}
	Let $A_k(X)$ denote the Chow group of $k$-dimensional cycles modulo rational equivalence, and set
	\[
	A_*(X) := \bigoplus_k A_k(X).
	\]
	Because GRR involves denominators (from Chern character and Todd class), we typically work with
	\[
	A_*(X)_{\mathbb{Q}} := A_*(X)\otimes_{\mathbb{Z}}\mathbb{Q}.
	\]
	
	\paragraph{Proper pushforward on Chow groups.}
	If $f$ is proper, there is a pushforward
	\[
	f_*\colon A_*(X)\to A_*(Y),
	\]
	defined on integral cycles by mapping a subvariety $V\subset X$ to
	$\deg(V/V')\cdot [V']$ if $f(V)=V'\subset Y$ and $\dim V'=\dim V$, and to $0$ otherwise.
	This is the lower horizontal arrow in the modern diagrammatic statement of GRR.
	
	\section{The bridge: Chern character and Todd class}
	
	\subsection{Chern character}
	When $X$ is smooth, the Chern character is a ring homomorphism
	\[
	\mathrm{ch}\colon K_0(X)\to A^*(X)_{\mathbb{Q}},
	\]
	where $A^*(X)$ is the Chow ring (codimension grading). It is additive on short exact sequences and
	multiplicative on tensor products. Concretely, if $E$ has Chern roots $x_i$, then
	\[
	\mathrm{ch}(E)=\sum_i e^{x_i}.
	\]
	
	\subsection{Todd class}
	For a vector bundle $E$ on $X$, the Todd class $\mathrm{Td}(E)\in A^*(X)_{\mathbb{Q}}$ is the multiplicative
	characteristic class defined (in terms of Chern roots $x_i$) by
	\[
	\mathrm{Td}(E)=\prod_i \frac{x_i}{1-e^{-x_i}}.
	\]
	For a smooth $X$, we write $\mathrm{Td}(T_X)$ for the Todd class of the tangent bundle.
	
	\section{Grothendieck--Riemann--Roch (GRR)}
	
	\subsection{The commuting-square formulation (smooth case)}
	Let $f\colon X\to Y$ be proper and assume $X$ and $Y$ are smooth. Define the ``Riemann--Roch transformation''
	\[
	\tau_X\colon K_0(X)\to A_*(X)_{\mathbb{Q}},\qquad
	\tau_X(\alpha):=\mathrm{ch}(\alpha)\cdot \mathrm{Td}(T_X)\cap [X],
	\]
	and similarly $\tau_Y$.
	
	\paragraph{The theorem (diagram form).}
	Then the following square commutes:
	\[
	\begin{tikzcd}[row sep=3.2em, column sep=4.4em]
		K_0(X) \arrow[r,"f_*"] \arrow[d,"\tau_X"'] &
		K_0(Y) \arrow[d,"\tau_Y"] \\
		A_*(X)_{\mathbb{Q}} \arrow[r,"f_*"'] &
		A_*(Y)_{\mathbb{Q}}
	\end{tikzcd}
	\]
	In words: translating a $K$-class to a cycle class via $\tau$ and then pushing forward
	equals pushing forward in $K$-theory first and translating afterwards.
	
	\subsection{Expanded identity}
	Equivalently, for every $\alpha\in K_0(X)$,
	\[
	\mathrm{ch}(f_*\alpha)\,\mathrm{Td}(T_Y)\cap [Y]
	\;=\;
	f_*\!\left(\mathrm{ch}(\alpha)\,\mathrm{Td}(T_X)\cap [X]\right).
	\]
	
	\subsection{A relative variant (optional)}
	One often rewrites GRR using the \emph{relative Todd class}
	\[
	\mathrm{Td}(T_f):=\frac{\mathrm{Td}(T_X)}{f^*\mathrm{Td}(T_Y)}\in A^*(X)_{\mathbb{Q}},
	\]
	so that
	\[
	\mathrm{ch}(f_*\alpha)
	=
	f_*\!\left(\mathrm{ch}(\alpha)\,\mathrm{Td}(T_f)\cap [X]\right)
	\qquad\text{(after identifying targets appropriately).}
	\]
	This is a convenient form when comparing to classical statements.
	
	\section{How GRR recovers classical Riemann--Roch}
	
	\subsection{Curves}
	Let $X$ be a smooth projective curve and let $f\colon X\to \mathrm{Spec}(k)$ be the structure morphism.
	For a line bundle $\mathcal{L}$, the pushforward $f_*[\mathcal{L}]$ in $K_0(k)\cong \mathbb{Z}$
	computes the Euler characteristic:
	\[
	f_*[\mathcal{L}] = \chi(X,\mathcal{L}) := \sum_i (-1)^i \dim_k H^i(X,\mathcal{L}).
	\]
	GRR expresses $\chi(X,\mathcal{L})$ as an intersection-theoretic number built from
	$\mathrm{ch}(\mathcal{L})$ and $\mathrm{Td}(T_X)$; evaluating gives
	\[
	\chi(X,\mathcal{L})=\deg(\mathcal{L}) + 1-g,
	\]
	the classical Riemann--Roch theorem for curves.
	
	\section{Mapping the doodle's symbols to modern notation}
	Grothendieck's doodle uses historically flavored notation (and some intentionally informal shorthand).
	A modern algebraic-geometry decoding is:
	\begin{itemize}[itemsep=4pt]
		\item $K'(X)$: read as $G_0(X)$ (coherent $K$-theory), or as $K_0(X)$ when $X$ is smooth.
		\item ``$\mathrm{Gr}$'': a hint that $K$-theory has natural filtrations (e.g.\ by codimension)
		whose associated graded relates to cycle groups; morally: ``linearize to Chow.''
		\item $\tau$: the Riemann--Roch natural transformation
		$\alpha\mapsto \mathrm{ch}(\alpha)\mathrm{Td}(T_X)\cap[X]$.
		\item $\mathrm{ch}$: the Chern character, explicitly present in the doodle.
		\item Tensoring with $\mathbb{Q}$: needed because $\mathrm{ch}$ and $\mathrm{Td}$ introduce denominators.
	\end{itemize}
	Thus, the punchline ``i.e.\ commutative!'' is precisely the GRR commuting square.
	
	\section{A minimal prerequisite roadmap}
	If you want to understand GRR efficiently, a good order is:
	\begin{enumerate}[itemsep=4pt]
		\item Coherent sheaves and derived pushforward $R f_*$.
		\item Grothendieck groups $G_0(X)$, $K_0(X)$; exact sequence relations.
		\item Chow groups $A_*(X)$ and proper pushforward of cycles.
		\item Chern classes; then Chern character $\mathrm{ch}$.
		\item Todd class $\mathrm{Td}$ and why rational coefficients appear.
		\item Statement and meaning of GRR; then compute special cases (curves, surfaces).
	\end{enumerate}
	
	\section{Mental model: why the theorem \emph{must} be a diagram}
	$K$-theory is the natural home for exact sequences and derived functors (like $R f_*$).
	Chow groups are the natural home for intersection products and characteristic classes.
	GRR says that, after translating $K$-classes to cycle classes via $\mathrm{ch}$ and $\mathrm{Td}$,
	pushforward becomes compatible:
	\[
	\tau_Y\circ f_* \;=\; f_* \circ \tau_X.
	\]
	In other words: the correct formulation is, inevitably, a commuting diagram.
	
	\section*{Optional appendix: a clean ``modern diagram'' block (copy-paste)}
%	\[
%	\begin{tikzcd}[row sep=3.2em, column sep=4.4em]
%		K_0(X) \arrow[r,"f_*"] \arrow[d,"\alpha \mapsto \mathrm{ch}(\alpha)\mathrm{Td}(T_X)\cap[X]"'] &
%		K_0(Y) \arrow[d,"\beta \mapsto \mathrm{ch}(\beta)\mathrm{Td}(T_Y)\cap[Y]"] \\
%		A_*(X)_{\mathbb{Q}} \arrow[r,"f_*"'] &
%		A_*(Y)_{\mathbb{Q}}
%	\end{tikzcd}
%	\]
	
\end{document}
