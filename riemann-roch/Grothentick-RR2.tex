% !TeX program = pdflatex
\documentclass[11pt]{article}

\usepackage[T1]{fontenc}
\usepackage[utf8]{inputenc}
\usepackage{lmodern}

\usepackage{geometry}
\geometry{margin=1in}

\usepackage{amsmath,amssymb}
\usepackage{tikz-cd}
\usepackage{graphicx}
\usepackage{caption}
\usepackage{microtype}

\title{Riemann--Roch Doodle (English Rendering)}
\author{}
\date{}

\begin{document}
	\maketitle
	
	\section*{1. The original image}
	% Put the image file next to this .tex file and name it: Grothentick-RR.webp
	% If your LaTeX does not support .webp, convert it to .png or .jpg and update the filename below.
%	\begin{figure}[ht]
%		\centering
%		\includegraphics[width=\linewidth]{Grothentick-RR.webp}
%		\caption{Original doodle image.}
%	\end{figure}
	
	\section*{2. English rendering (diagram + translated text)}
	
	\subsection*{2.1 ``The latest craze: the diagram''}
	\[
	\textbf{Riemann--Roch theorem: the latest craze --- the diagram:}
	\]
	
	\[
	\begin{tikzcd}[row sep=3.0em, column sep=4.0em]
		K'(X) \arrow[r,"f_!"] \arrow[d,"\tau"'] &
		K'(Y) \arrow[d,"\tau"] \\
		\operatorname{Gr}K'(X)\otimes\mathbb{Q} \arrow[r,"f_*"'] \arrow[d,"\mathrm{ch}"'] &
		\operatorname{Gr}K'(Y)\otimes\mathbb{Q} \arrow[d,"\mathrm{ch}"] \\
		H^*(X,\mathbb{Q}) \arrow[r,"f_*"'] &
		H^*(Y,\mathbb{Q})
	\end{tikzcd}
	\]
	\[
	\textit{i.e.\ commutative!}
	\]
	
	\subsection*{2.2 Translated paragraph (English)}
	\begin{quote}\small
		To give this statement about \(f\colon X\to Y\) even an approximate meaning,
		I had to abuse the audience's patience for almost two hours.
		In black and white (in Springer's \emph{Lecture Notes}) it probably runs to something like
		\(200\)–\(500\) pages.
		
		A striking example of how our drive for knowledge and discovery is increasingly getting lost
		in a life-detached logical delirium, while life itself is ``going to hell'' in a thousand ways
		--- and is threatened by \emph{irreversible destruction}.
		High time to change our course!
		
		\hfill --- Alexander Grothendieck
	\end{quote}
	
	\section*{Notes}
	\begin{itemize}
		\item The doodle’s symbols are rendered in a standard modern notation; the handwritten diagram is stylized and not always unambiguous at this resolution.
		\item If your LaTeX setup can’t include \texttt{.webp} images, convert the file to \texttt{.png} or \texttt{.jpg} and change the filename in the \texttt{\textbackslash includegraphics} line.
	\end{itemize}
	
\end{document}
