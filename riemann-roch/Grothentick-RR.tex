% !TEX TS-program = pdflatex
\documentclass[11pt]{article}

\usepackage[a4paper,margin=1in]{geometry}
\usepackage{amsmath,amssymb}
\usepackage{graphicx}
\usepackage{tikz-cd}     % commutative diagrams
\usepackage{microtype}
\usepackage[T1]{fontenc}
\usepackage[utf8]{inputenc}

\title{Grothendieck's Riemann--Roch Doodle (English Rendering)}
\author{}
\date{}

\begin{document}
	\maketitle
	
	\begin{figure}[ht]
		\centering
		% If you compile locally, put the image file next to this .tex file and use:
		% \includegraphics[width=\linewidth]{Grothentick-RR.webp}
		%
		% If you're compiling in this ChatGPT sandbox, the path would be:
		% \includegraphics[width=\linewidth]{/mnt/data/Grothentick-RR.webp}
%		\includegraphics[width=\linewidth]{Grothentick-RR.webp}
		\caption{Original doodle (scan/photo).}
	\end{figure}
	
	\section*{English transcription and diagram}
	
	\noindent\textbf{Riemann--Roch theorem: the latest craze --- the diagram.}
	
	\vspace{0.5em}
	
	\[
	\begin{tikzcd}[row sep=3.0em, column sep=4.0em]
		K'(X) \arrow[r,"f_!"] \arrow[d,"\tau"'] &
		K'(Y) \arrow[d,"\tau"] \\
		\operatorname{Gr}\,K'(X) \arrow[r,"f_*"'] \arrow[d,"\mathrm{ch}"'] &
		\operatorname{Gr}\,K'(Y) \arrow[d,"\mathrm{ch}"] \\
		\operatorname{Gr}\,H^*(X)\otimes\mathbb{Q} \arrow[r,"f_*"'] &
		\operatorname{Gr}\,H^*(Y)\otimes\mathbb{Q}
	\end{tikzcd}
	\]
	
	\noindent\emph{i.e.\ commutative!}
	
	\vspace{1em}
	
	\noindent
	To give this statement about $f\colon X\to Y$ even an approximate meaning,
	I had to abuse the audience's patience for almost two hours.
	In black and white (in Springer's \emph{Lecture Notes}) it probably runs to something like
	200--500 pages.
	
	\vspace{0.75em}
	
	\noindent
	A striking example of how our drive for knowledge and discovery is increasingly getting lost
	in a life-detached logical delirium, while life itself is ``going to hell'' in a thousand ways
	--- and is threatened by irreversible destruction.
	High time to change our course!
	
	\vspace{0.75em}
	
	\hfill --- Alexander Grothendieck
	
	\bigskip
	
	\section*{Notes (optional)}
	\begin{itemize}
		\item The diagram above is a clean \LaTeX{} rendering inspired by the doodle; the exact labels in the scan
		are stylized and partially ambiguous.
		\item If you want this to compile \emph{as-is}, place the image file \texttt{Grothentick-RR.webp}
		in the same folder as this \texttt{.tex} file.
		\item If your \LaTeX{} setup cannot include \texttt{.webp} images, convert it to PNG or PDF and update the filename.
	\end{itemize}
	
\end{document}
