% pdflatex -shell-escape software-analysis-main.tex
\documentclass[12pt,openany]{book}

\usepackage{kotex}
\usepackage{enumerate}
\usepackage{commath}
\usepackage{pifont} %http://ctan.org/pkg/pifont

\usepackage{stmaryrd} %llbracket
\usepackage{slantsc}
\usepackage{hyperref}
\usepackage{adjustbox}


% Packages for formatting
%\usepackage[margin=1in]{geometry}
%\usepackage{fancyhdr}
%\usepackage{graphicx}
%\usepackage{amsmath}
%\usepackage{amsthm}
%%\usepackage{algorithm2e,setspace}
%\usepackage{algpseudocode}
%%\usepackage{xcolor}
%\usepackage{amssymb}
\usepackage{amsthm}

% Define custom theorem styles
\newtheoremstyle{dotless} % Name of the style
{3pt} % Space above
{3pt} % Space below
{\itshape} % Body font
{} % Indent amount
{\bfseries} % Theorem head font
{} % Punctuation after theorem head
{2.5mm} % Space after theorem head
{} % Theorem head spec

\newtheoremstyle{definitionstyle} % Name of the style
{3pt} % Space above
{3pt} % Space below
{} % Body font
{} % Indent amount
{\bfseries} % Theorem head font
{.} % Punctuation after theorem head
{2.5mm} % Space after theorem head
{} % Theorem head spec

% Applying custom styles
\theoremstyle{dotless}
\newtheorem{theorem}{Theorem}[section] % Theorem environment with section-wise numbering
\newtheorem{lemma}[theorem]{Lemma} % Lemma shares the counter with theorem
\newtheorem{corollary}[theorem]{Corollary} % Corollary shares the counter with theorem

\theoremstyle{definitionstyle}
\newtheorem*{observation}{\textcolor{Magenta}{Observation}}
\newtheorem{definition}{Definition}[chapter] % Definition shares the counter with theorem
\newtheorem{example}{Example}[chapter] % Example shares the counter with theorem
\newtheorem{exercise}{Exercise}[chapter] % Example shares the counter with theorem
\newtheorem{remark}{Remark}[definition] % Remark shares the counter with theorem
\newtheorem*{note}{Note}

% Colors
\usepackage[dvipsnames,table]{xcolor}
\definecolor{titleblue}{RGB}{0,53,128}
\definecolor{chaptergray}{RGB}{140,140,140}
\definecolor{sectiongray}{RGB}{180,180,180}

\definecolor{thmcolor}{RGB}{231, 76, 60}
\definecolor{defcolor}{RGB}{52, 152, 219}
\definecolor{lemcolor}{RGB}{155, 89, 182}
\definecolor{corcolor}{RGB}{46, 204, 113}
\definecolor{procolor}{RGB}{241, 196, 15}
\definecolor{execolor}{RGB}{90, 128, 127}

% Fonts
\usepackage[T1]{fontenc}
\usepackage[utf8]{inputenc}
\usepackage{newpxtext,newpxmath}
\usepackage{sectsty}
\allsectionsfont{\sffamily\color{titleblue}\mdseries}

% Page layout
\usepackage{geometry}
\geometry{a4paper,left=.8in,right=.6in,top=.75in,bottom=1in,heightrounded}
\usepackage{fancyhdr}
\fancyhf{}
\fancyhead[LE,RO]{\thepage}
\fancyhead[LO]{\nouppercase{\rightmark}}
\fancyhead[RE]{\nouppercase{\leftmark}}
\renewcommand{\headrulewidth}{0.5pt}
\renewcommand{\footrulewidth}{0pt}

% Chapter formatting
\usepackage{titlesec}
\titleformat{\chapter}[display]
{\normalfont\sffamily\Huge\bfseries\color{titleblue}}{\chaptertitlename\ \thechapter}{20pt}{\Huge}
\titleformat{\section}
{\normalfont\sffamily\Large\bfseries\color{titleblue!100!gray}}{\thesection}{1em}{}
\titleformat{\subsection}
{\normalfont\sffamily\large\bfseries\color{titleblue!75!gray}}{\thesubsection}{1em}{}

% Table of contents formatting
\usepackage{tocloft}
\renewcommand{\cftchapfont}{\sffamily\color{titleblue}\bfseries}
\renewcommand{\cftsecfont}{\sffamily\color{titleblue!100!gray}}
\renewcommand{\cftsubsecfont}{\sffamily\color{titleblue!75!gray}}
\renewcommand{\cftchapleader}{\cftdotfill{\cftdotsep}}

\usepackage{tcolorbox}
\tcbset{colback=white, arc=5pt}

\definecolor{defcolor}{RGB}{52, 152, 219}
\definecolor{procolor}{RGB}{241, 196, 15}
\definecolor{thmcolor}{RGB}{231, 76, 60}
\definecolor{lemcolor}{RGB}{155, 89, 182}
\definecolor{corcolor}{RGB}{46, 204, 113}
\definecolor{execolor}{RGB}{90, 128, 127}

% Define a new command for the custom tcolorbox
\newcommand{\defbox}[2][]{%
	\begin{tcolorbox}[colframe=defcolor, title={\color{white}\bfseries #1}]
		#2
	\end{tcolorbox}
}

\newcommand{\probox}[2][]{%
	\begin{tcolorbox}[colframe=procolor, title={\color{white}\bfseries #1}]
		#2
	\end{tcolorbox}
}

\newcommand{\thmbox}[2][]{%
	\begin{tcolorbox}[colframe=thmcolor, title={\color{white}\bfseries #1}]
		#2
	\end{tcolorbox}
}
\usepackage{tikz}
\usepackage{tikz-cd}
\usetikzlibrary{shapes.geometric, arrows.meta, positioning}
\usetikzlibrary{calc}
\usepackage[ruled,linesnumbered]{algorithm2e}
\usepackage{setspace}
\usepackage{algpseudocode}
\SetKwComment{Comment}{/* }{ */}
\SetKw{Break}{break}
\SetKw{Downto}{downto}
\SetKwProg{Fn}{Function}{:}{end}
\SetKwProg{Procedure}{procedure}{:}{end}
\SetKwProg{Construct}{Construct}{:}{end}
\SetKwFunction{KeyGen}{KeyGen}

%Listing
\usepackage{listings} %Code
\renewcommand{\lstlistingname}{Code}%
\definecolor{keyword}{RGB}{255, 0, 0}
\definecolor{identifier}{RGB}{0, 0, 255}
\definecolor{comment}{RGB}{0, 128, 0}
\definecolor{string}{RGB}{163, 21, 21}

\lstdefinestyle{c}{
	language=C,
	basicstyle=\ttfamily\small,
	keywordstyle=\color{keyword},
	identifierstyle=\color{identifier},
	commentstyle=\color{comment}\itshape,
	stringstyle=\color{string},
	showstringspaces=false,
	%	numberstyle=\tiny\color{gray},
	%	numbersep=5pt,
	frame=single,
	tabsize=4,
	captionpos=b,
	breaklines=true,
	breakatwhitespace=true,
	%	numbers=left
}
\newcommand{\mathcolorbox}[2]{\colorbox{#1}{$\displaystyle #2$}}

\newcommand{\N}{\mathbb{N}}
\newcommand{\Z}{\mathbb{Z}}
\newcommand{\Q}{\mathbb{Q}}
\newcommand{\R}{\mathbb{R}}
\newcommand{\C}{\mathbb{C}}
\newcommand{\F}{\mathbb{F}}

\newcommand{\cP}{\mathcal{P}}
\newcommand{\cR}{\mathcal{R}}

\newcommand{\dom}{\mathsf{Dom}}
\newcommand{\cdm}{\mathsf{Cdm}}

\newcommand{\id}{\textnormal{id}}

\newcommand{\ie}{\textnormal i.e.}

\newcommand{\oracle}{\textsc{Oracle}}
\newcommand{\sub}[1]{\textsf{sub}\left(#1\right)}

\newcommand{\sol}{\textcolor{magenta}{\bf Sol}}
\newcommand{\xmark}{\ding{55}}%



\begin{document}
	
	\begin{titlepage}
    \centering
    
    \vspace*{1cm}
    
    \Huge\textsf{\textbf{Modern Mathematics}}
    
    \vspace{0.5cm}
    \LARGE\textsf{- A Journey from Concretization to Abstraction -}
    
    \vspace{1.5cm}
    \textbf{Ji, Yong-Hyeon}

    \vfill
    A document presented for\\
    the Modern Mathematics
    
    \vspace{0.8cm}
    {\large\textsf{Department of Information Security, Cryptology, and Mathematics}\par}
    {\large\textsf{College of Science and Technology}\par}
    {\large\textsf{Kookmin University}\par}
    \vspace{.25in}
    {\large \textsf{\today}\par}
    
%    \Large{Institution Name}\\
%    Date
\end{titlepage}


	\tableofcontents
	\newpage
	\chapter{Introduction}
	% Chapter I. Introduction

\section{Axiom}
	
	\newpage
	\chapter{Quadratic Formula and Peano Axiom}
	% Algebra: Quadratic Formula and Peano Axiom

\section{Quadratic Formula}
\begin{note}
	We want to find the roots of the quadratic equation: for $a\neq 0$,
	\[
	ax^2 + bx + c = 0.
	\]
	\begin{proof}[\sol]
		\begin{align*}
			ax^2 + bx + c = 0&\iff ax^2 + bx = -c & \\
			&\iff x^2 + \frac{b}{a}x = -\frac{c}{a} & \text{Divide every term by \(a\neq 0\)}\\
			&\iff x^2 + \frac{b}{a}x + \left(\frac{b}{2a}\right)^2 - \left(\frac{b}{2a}\right)^2 = -\frac{c}{a} & \text{Complete the square on the left side} \\
			&\iff \left(x + \frac{b}{2a}\right)^2 = \left(\frac{b}{2a}\right)^2 - \frac{c}{a} & \\
			&\iff x + \frac{b}{2a} = \pm \sqrt{\left(\frac{b}{2a}\right)^2 - \frac{c}{a}}
			& \text{Take the square root on both sides} \\
			&\iff x = -\frac{b}{2a} \pm \sqrt{\left(\frac{b}{2a}\right)^2 - \frac{c}{a}} & \text{Simplify to solve for \(x\)} \\
			&\iff x = -\frac{b}{2a} \pm \sqrt{\frac{b^2 - 4ac}{4a^2}} & \\
			&\iff x = \frac{-b \pm \sqrt{b^2 - 4ac}}{2a} & \text{Quadratic formula}
		\end{align*}
		This expression provides the solutions for \(x\) in the quadratic equation \(ax^2 + bx + c = 0\) ($a\neq 0$).
	\end{proof}
\end{note}

\newpage
\section{Peano Axiom and Natural Number}
\subsection{Peano Axiom and Successor Function}
The set of natural numbers, denoted by \(\mathbb{N}\), is defined by the following axioms:

\begin{enumerate}[\bf 1.]
	\item \textbf{\underline{Zero is a natural number}}: \(\mathcolorbox{yellow}{0 \in\N}\).
	
	There exists a natural number \(0\).\vspace{12pt}
	\item \textbf{\underline{Successor}}: \(\mathcolorbox{yellow}{n\in\N\implies S(n)\in\N}\).
	
	For every natural number \(n\), there exists a natural number \(S(n)\), called the successor of \(n\).
	\begin{enumerate}[(i)]
		\item (\textcolor{green!50!black}{$\boldsymbol{\checkmark}$})\begin{tikzcd}
			0 \arrow[r] & S(0) \arrow[r] & S(S(0)) \arrow[r] & \cdots
		\end{tikzcd}
		\item (\textcolor{red}{$\boldsymbol{\times}$})\begin{tikzcd}
			k\in\mathbb{N} \arrow[r] & S(k)=0 \arrow[r] & S0 \arrow[r] & SS0 \arrow[r] & \cdots
		\end{tikzcd}
		\item (\textcolor{red}{$\boldsymbol{\times}$}) \begin{center}\begin{tikzcd}
				0 \arrow[r] & S0 \arrow[r]    & SS0 \arrow[d]  \\
				& SSSS0 \arrow[u] & SSS0 \arrow[l]
			\end{tikzcd}
		\end{center}
		\item (\textcolor{red}{$\boldsymbol{\times}$}) \begin{center}\begin{tikzcd}
				0 \arrow[r] & S0 \arrow[r] & SS0 \arrow[r]            & SSS0 \\
				&              & S(k) \arrow[u]           &      \\
				&              & k\in\mathbb{N} \arrow[u] &     
			\end{tikzcd}
		\end{center}
		\item (\textcolor{red}{$\boldsymbol{\times}$}) \begin{center}\begin{tikzcd}
				0 \arrow[r] & S0 \arrow[r] & SS0 \arrow[r]  & SSS0 \arrow[r] & SSSS0 \\
				&              & k\in\mathbb{N} &                &      
			\end{tikzcd}
		\end{center}
	\end{enumerate}
	\vspace{12pt}
	\item \textbf{\underline{No natural number has 0 as its successor}}: \(\mathcolorbox{yellow}{n \in \mathbb{N}\implies \ S(n) \neq 0}\).
	
	There is no natural number whose successor is \(0\). (It solves 2-(ii))
	\vspace{12pt}
	\item \textbf{\underline{Distinctness}}: \(\mathcolorbox{yellow}{\forall m,n\in\N:[S(m)=S(n)\implies m=n]}\).Define addition to the set of natural numbers and define integers based on the concepts of identity and inverse. Also define rational numbers based on the multiplication of integers. In this way, derive the group structure and define the group. Give me the ratex code to be a professional mathematician.
	
	Distinct natural numbers have distinct successors. (It solves 2-(iii) and (iv))
	\vspace{12pt}
	\item \textbf{\underline{Induction}}: \(\mathcolorbox{yellow}{(0 \in M)\land(n \in M\Rightarrow S(n) \in M)\implies\N\subseteq M}\)
	
	If a set \(M\) of natural numbers contains \(0\) and is closed under the successor function (i.e., \(n \in M \implies S(n) \in M\)), then \(M\) contains all natural numbers. (It solves 2-(v))
\end{enumerate}

\begin{remark}[\textcolor{blue}{\bf Successor Function $\boldsymbol{S(n)}$}] \ \\
	The successor function $S(n)$ can be understood through these principles:
	\begin{enumerate}
		\item \textbf{Uniqueness and Existence}: For each natural number $n$, there exists a unique natural number $S(n)$. This means $S(n)$ is well-defined and there is no ambiguity about what the successor of $n$ is.
		\item \textbf{Construction of Natural Numbers}: The successor function constructs the sequence of natural numbers starting from 0. For example: \[
		S(0)=1,\quad S(1)=2,\quad S(2)=3,\quad\text{and so on}.
		\]
		Here, 1 is the successor of 0, 2 is the successor of 1, and so forth. Each natural number $n$ can be reached by repeatedly applying the successor function starting from 0.
		\item \textbf{Non-circularity} No natural number $n$ has $0$ as its successor. This avoids circular definitions and ensures a clear progression of numbers: \[
		\forall n\in\N:S(n)\neq 0.
		\]
		\item \textbf{Injectivity}:  The axiom $S(m)=S(n)\implies m=n$ ensures that the successor function is injective, meaning different numbers have different successors. This property is essential for maintaining the distinctness of natural numbers.
		\item \textbf{Basis of Induction}: The induction axiom relies on the successor function. It states that if a property holds for 0 and holds for $S(n)$ whenever it holds for $n$, then the property holds for all natural numbers. This principle is the foundation of mathematical induction.
	\end{enumerate}
	
	A  visual representation of the successor function can help understand its role: \[
	0\xrightarrow{S} 1\xrightarrow{S} 2\xrightarrow{S} 3\xrightarrow{S} 4\xrightarrow{S} \cdots
	\] Each arrow represents the application of the successor function, moving from one natural number to the next.
	
	In summary, the successor function $S(n)$ in Peano's axioms is a fundamental operation that:
	\begin{itemize}
		\item Provides a way to generate the next natural number from a given one.
		\item  Ensures the natural numbers are distinct and ordered.
		\item erves as the basis for defining natural numbers and performing induction.
	\end{itemize}
	These properties make the successor function an essential component in the foundation of arithmetic and number theory.
\end{remark}

\newpage
% Sec. Group Structure ========================================================================
\section{Group Structure}

% Sub. Addition and Multiplication on Natural Number ==========================================
\subsection{Addition and Multiplication on Natural Numbers}
\begin{observation}
	\ \begin{itemize}
		\item $1+1=2$
		\item $(-1)\times(-1)=1$
	\end{itemize}
\end{observation}

\defbox[Addition on Natural Numbers]{
Addition on the set of natural numbers \(\mathbb{N}\) is defined recursively:
\begin{itemize}
	\item \textbf{(Base Case)} \[
	n \in \mathbb{N}\implies \ 0 + n = n.
	\]
	\item \textbf{(Recursive Step)} \[
	m, n \in \mathbb{N}\implies S(m) + n = S(m + n).
	\]
\end{itemize}
}
\begin{remark}
	\begin{align*}
		1 &= S0	\\
		2 &= SS0 &= S^20 \\
		3 &= SSS0 &= S^30\\
		&\vdots \\
		n &= \underbrace{S\cdots S}_{n}0 &= S^n0
	\end{align*}
\end{remark}

\begin{example}
	Prove that $1+1=2$.
	\begin{proof}
		Consider $1=S(0)$. Then \begin{align*}
			1+1=S(0)+S(0)=S(S(0)+0)=S(S0)=2.
		\end{align*}
	\end{proof}
\end{example}

\defbox[Multiplication on Natural Numbers]{
	Multiplication on the set of natural numbers \(\mathbb{N}\) is defined recursively:
	\begin{itemize}
		\item \textbf{(Base Case)} \[
		n \in \mathbb{N}\implies \ 0 \cdot n = n.
		\]
		\item \textbf{(Recursive Step)} \[
		m, n \in \mathbb{N}\implies S(m) \cdot n = (m\cdot n) + n.
		\]
	\end{itemize}
}

\begin{example}
	Prove that $n\times 1 = n$ for all $n\in\N$.
	\begin{proof}
		Consider $n,1\in\N$, i.e., $n=S^n0$, $1=S0$. Then \begin{align*}
			n\times 1=S^n0\times S0&=S(S^{n-1}0)\times S0\\
			&= (S^{n-1}0\times S0)+S0 \\
			&= (S^{n-2}0\times S0) + (S0 + S0) \\ 
			&= (0\times S0) + (\underbrace{S0 + S0 + \cdots + S0}_n) \\
			&= 0 + n\\
			&= n.
		\end{align*}
	\end{proof}
\end{example}

\defbox[Construction of Integer]{
	The set of integers \(\mathbb{Z}\) includes identity and inverse elements.
	
	\begin{itemize}
		\item \textbf{Identity}: \(\forall a \in \mathbb{Z}, \ a + 0 = a\)
		\item \textbf{Inverses}: \(\forall n \in \mathbb{N}, \ \exists -n \in \mathbb{Z} \text{ such that } n + (-n) = 0\)
	\end{itemize}
	
	Formally, the set of integers \(\mathbb{Z}\) is:
	\begin{align*}
		\Z &= -\N\cup\set{0}\cup\N \\
		&= \set{-1,-2,-3,\dots}\cup\set{0}\cup\set{1,2,3,\dots} \\
		&= \set{\dots, -3, -2, -1, 0, 1, 2, 3, \dots}.
	\end{align*}
}

\begin{example}
	Prove that $(-1)\times (-1) = 1$.
	\begin{proof}
		\begin{align*}
			0 &= 0\times (-1) \\
			&= S(-1)\times (-1) \\
			&= ((-1)\times (-1)) + (-1).
		\end{align*} Thus, $(-1)\times (-1)=1$.
	\end{proof}
\end{example}

\subsection{Rational Number and Equivalence Relation}
\begin{observation}
	\ \begin{itemize}
		\item $\frac{1}{2}=0.5$
		\item $\frac{1}{2}=\frac{2}{4}=\cdots=\frac{1622660}{3245320}$
	\end{itemize}
\end{observation}


\defbox[Rational Numbers]{
A rational number $\Q$ is defined as an ordered pair of integers \((a, b)\) where \(a \in \mathbb{Z}\) and \(b \in \mathbb{Z} \setminus \{0\}\). This pair represents the fraction \(\frac{a}{b}\).
}

\begin{note}
	We introduce an equivalence relation on the set of pairs of integers:
	\[
	(a, b) \sim (c, d) \iff ad = bc
	\]
	This relation ensures that different pairs of integers representing the same rational number are considered equivalent.
		
	The set of rational numbers \(\mathbb{Q}\) is the set of equivalence classes of the pairs \((a, b)\):
	\[
	\mathbb{Q} = \left\{ \left. \frac{a}{b} \ \right| \ a \in \mathbb{Z}, b \in \mathbb{Z} \setminus \{0\}, (a, b) \sim (c, d) \iff ad = bc \right\}
	\]
\end{note}

\subsection{Groups}
\begin{observation}
\ \begin{center}
\begin{minipage}{.48\textwidth}
	\((\mathbb{Z}, +)\)
	\begin{itemize}
		\item \(\forall a, b \in \mathbb{Z}, \ a + b \in \mathbb{Z}\)
		\item \(\forall a, b, c \in \mathbb{Z}, \ (a + b) + c = a + (b + c)\)
		\item \(\exists 0 \in \mathbb{Z} \text{ such that } \forall a \in \mathbb{Z}, \ a + 0 = a\)
		\item \(\forall a \in \mathbb{Z}, \ \exists -a \in \mathbb{Z} \text{ such that } a + (-a) = 0\)
	\end{itemize}
\end{minipage}
\begin{minipage}{.48\textwidth}
	\((\mathbb{Q}^*, \cdot)\)
	\begin{itemize}
		\item \(\forall a, b \in \mathbb{Q}^*, \ a \cdot b \in \mathbb{Q}^*\)
		\item \(\forall a, b, c \in \mathbb{Q}^*, \ (a \cdot b) \cdot c = a \cdot (b \cdot c)\)
		\item \(\exists 1 \in \mathbb{Q}^* \text{ such that } \forall a \in \mathbb{Q}^*, \ a \cdot 1 = a\)
		\item \(\forall a \in \mathbb{Q}^*, \ \exists a^{-1} \in \mathbb{Q}^* \text{ such that } a \cdot a^{-1} = 1\)
	\end{itemize}
\end{minipage}
\end{center}
\end{observation}
\defbox[Group]{
\begin{definition}
A \textbf{group} is a set \( G \) equipped with a binary operation \(*:G\times G\to G\) that combines any two elements \( a \) and \( b \) to form another element denoted \( a * b \). The set and operation, \((G, *)\), must satisfy four fundamental properties known as the group axioms:

\begin{enumerate}
	\item \textbf{Closure}:
	\[
	a, b \in G\implies a * b \in G
	\]
	
	\item \textbf{Associativity}:
	\[
	a, b, c \in G\implies (a * b) \cdot c = a * (b * c)
	\]
	
	\item \textbf{Identity Element}:
	\[
	\exists e \in G:[a \in G\implies \ e * a = a = a * e]
	\]
	
	\item \textbf{Inverse Element}:
	\[
	a \in G\implies[\exists a^{-1} \in G : a * a^{-1} = e = a^{-1} * a]
	\]
\end{enumerate}
\end{definition}
}


	
	\newpage
	\chapter{Functions}
	\begin{observation}
\ \begin{center}
\begin{minipage}{.48\textwidth}
		\adjustbox{scale=.7, center}{
			\begin{tikzcd}
				&  &  & \text{Relation} \arrow[ddddd, "\text{Left-Total}", bend left] \arrow[ddddd, "\text{Many-to-one}"', bend right] &  &  &\\
				&  &  & &  &  &   \\
				&  &  & &  &  &  \\
				&  &  &  &  &  &  \\
				&  &  & &  &  &  \\
				&  &  & \text{Function} \arrow[rrrddd, "\text{Right-Total}"] \arrow[lllddd, "\text{One-to-Many}"']                             &  &  &  \\
				&  &  &  &  &  & \\
				&  &  & &  &  &   \\
				\text{Injection} \arrow[rrrdd] &  &  &   &  &  & \text{Surjection} \arrow[llldd] \\
				&  &  &  &  &  &  \\&  &  & \text{Bijection} &  &  &              
		\end{tikzcd}}
\end{minipage}\quad
\begin{minipage}{.48\textwidth}
		\adjustbox{scale=.7, center}{
			\begin{tikzcd}
				&  &  & f\subseteq S\times T \arrow[ddddd, "\displaystyle\frac{s\in S}{\exists t\in T\ \text{s.t.}\ s\mathrel{f}t}", bend left] \arrow[ddddd, "\displaystyle\frac{s\mathrel{f}t_1\quad s\mathrel{f}t_2}{t_1=t_2}"', bend right] &  &  &\\
				&  &  & &  &  &   \\
				&  &  & &  &  &  \\
				&  &  &  &  &  &  \\
				&  &  & &  &  &  \\
				&  &  & S\overset{f}{\to} T \arrow[rrrddd, "\displaystyle\frac{t\in T}{\exists s\in S\ \text{s.t.}\ t\mathrel{f}s}"] \arrow[lllddd, "\displaystyle\frac{s_1\mathrel{f}t\quad s_2\mathrel{f}t}{s_1=s_2}"']                             &  &  &  \\
				&  &  &  &  &  & \\
				&  &  & &  &  &   \\
				S\overset{f}{\rightarrowtail} T \arrow[rrrdd] &  &  &   &  &  & S\overset{f}{\twoheadrightarrow} T \arrow[llldd] \\
				&  &  &  &  &  &  \\&  &  & S\overset{f}{\leftrightarrow} T &  &  &              
		\end{tikzcd}}
\end{minipage}
\end{center} 

\begin{center}\adjustbox{scale=.7}{\begin{tikzpicture}
% Define circles
\draw[thick, fill=red!30, opacity=0.5] (0,0) circle (6cm); \node at (0, 6.25) {\large Relation};
\draw[thick, fill=blue!30, opacity=0.5] (0,0) circle (5cm); \node at (0, 5.25) {\large Function};
\draw[thick, fill=green!30, opacity=0.5] (1.5,0) circle (3.25cm); \node at (1.5, 3.5) {\large Surjection};
\draw[thick, fill=yellow!30, opacity=0.5] (-1.5,0) circle (3.25cm); \node at (-1.5, 3.5) {\large Injection};
%\draw[thick, fill=violet!30, opacity=0.5] (0,0) circle (1cm); 
\node at (0, 0) {\large Bijection};
\end{tikzpicture}}
\end{center}
\end{observation}

\section{Functions}
\defbox[Function]{
\begin{definition}
	Let $S$ and $T$ be sets. A \textbf{function} $f$ \textbf{from $\boldmath{S}$ to $\boldmath{T}$} is a relation on $S\times T$ satisfying as follows: \begin{enumerate}[(i)]
		\item (\textbf{Left-Total}\footnote{Every element of $S$ relates to some element of $T$.}) $\dom f=S$, \ie, \[
		s\in S\implies \exists t\in T:f(s)=t.
		\]
		\item (\textbf{Many-to-one}\footnote{Every element of $\dom{f}$ relates to no more than one element of its $\cdm{f}$.}) Let $s\in\dom{f}$ and $t_1,t_2\in\cdm{f}$. Then \[
		f(s)=t_1\land f(s)=t_2\implies s_1=s_2.
		\]
	\end{enumerate}
\end{definition}
}

\defbox[Domain, Codomain, and Range]{
\begin{definition}
\ \begin{itemize}
	\item \textbf{Domain:} The domain of a function \( f: A \to B \) is the set \( A \) of all possible input values for which the function is defined. Formally:
	\[
	\text{Domain}(f) = A
	\]
	\item \textbf{Co-domain:} The co-domain of a function \( f: A \to B \) is the set \( B \) which includes all potential output values. It is the target set for the function. Formally:
	\[
	\text{Co-domain}(f) = B
	\]
	\item \textbf{Range:} The range (or image) of a function \( f \) is the set of all actual output values produced by the function. It is a subset of the co-domain \( B \). Formally:
	\[
	\text{Range}(f) = f[A] = \{ f(a) \mid a \in A \} \subseteq B
	\]
\end{itemize}
\end{definition}
}
\begin{remark}
\ \\ \adjustbox{scale=.9, center}{
\begin{tikzpicture}
	% Draw the sets A and B
	\draw[thick] (-3,0) ellipse (2 and 3);
	\draw[thick] (3,0) ellipse (2 and 3);
	
	% Labels for sets
	\node at (-3, 3.25) {$\dom f$};
	\node at (3, 3.25) {$\cdm f$};
	
	% Draw the arrows representing the function
	\draw[-Stealth] (-2, 3.25) -- (2,3.25) node[midway, above] {$f$};
	\draw[thick, dotted] (-3,3) -- (3,1.5);
	\draw[thick, dotted] (-3,-3) -- (3,-1.5);
	
	\draw[fill] (-2.5,0) circle (.1);
	\draw[fill] (2.5,0) circle (.1);
	\draw[|-Stealth, line width=.5mm] (-2.25, 0) -- (2.25, 0);
	
	% Labels for elements in Domain
	\node at (-3, 0) {$a$};
	
	% Labels for elements in Co-domain
	\node at (3, 0) {$f(a)$};
	
	% Highlight the range
	\draw[thick, dotted] (3, 0) ellipse (1 and 1.5);
	
	% Label for the range
	\node at (3, 1.75) {$f[A]$};
\end{tikzpicture}}
\end{remark}

\section{Composition}
\defbox[Composition of Functions]{
\begin{definition}
Given two functions \( f \) and \( g \), where \( f: A \to B \) and \( g: B \to C \), the \textbf{composition} of \( g \) and \( f \), denoted by \( g \circ f \), is a function from \( A \) to \( C \) defined as follows:
\[
(g \circ f)(x) = g(f(x))
\]
for all \( x \in A \). That is, \[
\fullfunction{g\circ f}{A}{C}{a}{(g\circ f)(a)}
\]
\end{definition}}
\begin{remark}
\ \begin{itemize}
	\item \textbf{Functions}:
	\begin{itemize}
		\item Let \( f: B \to C \) be a function from set \( B \) to set \( C \).
		\item Let \( g: A \to B \) be a function from set \( A \) to set \( B \).
	\end{itemize}
	
	\item \textbf{Composition Definition}:
	\begin{itemize}
		\item The composition \( f \circ g \) is a function from \( A \) to \( C \).
		\item For each \( x \in A \), \( (f \circ g)(x) \) is defined as \( f(g(x)) \).
	\end{itemize}
	
	\item \textbf{Domain and Range}:
	\begin{itemize}
		\item The domain of the composite function \( f \circ g \) is \( A \).
		\item The range of the composite function \( f \circ g \) is a subset of \( C \).
	\end{itemize}
\end{itemize}
\end{remark}
\vspace{12pt}
\begin{remark}
Let \( G \) be a set of bijective functions from a set \( X \) to itself. Define the binary operation \(\circ\) to be the composition of functions. Then \( G \) under this operation is a group.
\begin{enumerate}
	\item \textbf{Closure}: If \( f, g \in G \), then \( f \circ g \in G \) because the composition of two bijective functions is bijective.
	
	\item \textbf{Associativity}: Function composition is associative. For any \( f, g, h \in G \),
	\[
	(f \circ g) \circ h = f \circ (g \circ h)
	\]
	\adjustbox{scale=.9, center}{
		\begin{tikzpicture}
			% Draw the sets A and B
			\draw[thick] (-6,0) ellipse (1.5 and 2);
			\draw[thick] (-2,0) ellipse (1.5 and 2);
			\draw[thick] (2,0) ellipse (1.5 and 2);
			\draw[thick] (6,0) ellipse (1.5 and 2);
			
			% Labels for sets
			\node at (-6, 2.25) {$A$};
			\node at (-2, 2.25) {$B$};
			\node at (2, 2.25) {$C$};
			\node at (6, 2.25) {$D$};
			
			% Draw the arrows representing the function
			\draw[-Stealth] (-5.5, 2.25) -- (-2.5, 2.25) node[midway, above] {$f$};
			\draw[-Stealth] (-1.5, 2.25) -- (1.5, 2.25) node[midway, above] {$g$};
			\draw[-Stealth] (2.5, 2.25) -- (5.5, 2.25) node[midway, above] {$h$};
			\draw[|-Stealth, line width=.5mm] (-5.5, 0) -- (-2.5, 0);
			\draw[|-Stealth, line width=.5mm] (-1.5, 0) -- (1.5, 0);
			\draw[|-Stealth, line width=.5mm] (2.5, 0) -- (5.5, 0);
			
			% Labels for elements in Domain
			\node[fill, circle, inner sep=0.05cm] at (-6,0) (A) {};
			\node[fill, circle, inner sep=0.05cm] at (-2,0) (B) {};
			\node[fill, circle, inner sep=0.05cm] at (2,0) (C) {};
			\node[fill, circle, inner sep=0.05cm] at (6,0) (D) {};
			\node (A2) at (-6, 0.5) {$a$};
			\node (B2) at (-2, 0.5) {$f(a)$};
			\node (C2) at (2, 0.5) {$g(f(a))$};
			\node (D2) at (6, 0.5) {$h(g(f(a)))$};
			
			\draw[-Stealth, bend right=25pt, shorten <= 5pt, shorten >= 5pt] (A) to node[midway,below] {$g\circ f$} (C);
			\draw[-Stealth, bend right=25pt, shorten <= 5pt, shorten >= 5pt] (A) to node[midway,below] {$h\circ(g\circ f)$} (D);
			
			\draw[-Stealth, bend left=25pt, shorten <= 5pt, shorten >= 5pt] (A) to node[midway,above] {$(h\circ g)\circ f$} (D);
			\draw[-Stealth, bend left=25pt, shorten <= 5pt, shorten >= 5pt] (B) to node[midway,above] {$h\circ g$} (D);
	\end{tikzpicture}}
	\item \textbf{Identity Element}: The identity function \( \text{id}_A: A \to A \), defined by \( \text{id}_A(a) = a \) for all \( a \in A \), is the identity element in \( G \). For any \( f \in G \),
	\[
	f \circ \text{id}_A = f = \text{id}_A \circ f
	\]
	\begin{itemize}
		\item[] $f\circ \id_A:$
		\item[] $\id_A\circ f:$
		\item[] $f:$
	\end{itemize}
	\item \textbf{Inverse Element}: For each \( f \in G \), its inverse \( f^{-1} \) exists and is also a bijection from \( A \) to \( A \). It satisfies
	\[
	f \circ f^{-1} = \text{id}_A = f^{-1} \circ f
	\]
\end{enumerate}
\end{remark}

\section{Symmetric Group}
\begin{exercise}[Symmetric Group $S_2$]
The \textbf{symmetric group} \( S_2 \) is the group of all permutations of a two-element set. For a set \( X = \{1, 2\} \), the symmetric group \( S_2 \) consists of all bijective functions (permutations) from \( X \) to itself.

There are exactly two permutations of the set \( S = \{1, 2\} \):
\begin{itemize}
	\item \textbf{Identity Permutation} \( \text{id} \):
	\[
	\text{id}(1) = 1, \quad \text{id}(2) = 2
	\]
	This permutation leaves every element in its original position.\\
	\ \\
	\adjustbox{scale=.9, center}{
		\begin{tikzpicture}
			% Draw the sets A and B
			\draw[thick] (-2,0) ellipse (1 and 2);
			\draw[thick] (2,0) ellipse (1 and 2);
			
			% Labels for sets
			\node at (-2, 2.25) {$S$};
			\node at (2, 2.25) {$S$};
			
%			% Draw the arrows representing the function
			\draw[-Stealth, thick] (-1.5, 2.25) -- (1.5,2.25) node[midway, above] {$\id$};
			
			\node (A) at (-2, .75) {$1$};
			\node (B) at (-2, -.75) {$2$};
			\draw[fill] (-1.75,.75) circle (.05);
			\draw[fill] (-1.75,-.75) circle (.05);
			
			\node (C) at (2, .75) {$1$};
			\node (D) at (2, -.75) {$2$};
			\draw[fill] (1.75,.75) circle (.05);
			\draw[fill] (1.75,-.75) circle (.05);
			
			\draw[-Stealth, thick] (-1.75, .75) -- (1.75, .75);
			\draw[-Stealth, thick] (-1.75, -.75) -- (1.75, -.75);
	\end{tikzpicture}}
	\item \textbf{Transposition} \( \sigma \):
	\[
	\sigma(1) = 2, \quad \sigma(2) = 1
	\]
	This permutation swaps the two elements.\\
	\ \\
	\adjustbox{scale=.9, center}{
		\begin{tikzpicture}
			% Draw the sets A and B
			\draw[thick] (-2,0) ellipse (1 and 2);
			\draw[thick] (2,0) ellipse (1 and 2);
			
			% Labels for sets
			\node at (-2, 2.25) {$S$};
			\node at (2, 2.25) {$S$};
			
			% Draw the arrows representing the function
			\draw[-Stealth, thick] (-1.5, 2.25) -- (1.5,2.25) node[midway, above] {$\sigma$};
			
			\node (A) at (-2, .75) {$1$};
			\node (B) at (-2, -.75) {$2$};
			\draw[fill] (-1.75,.75) circle (.05);
			\draw[fill] (-1.75,-.75) circle (.05);
			
			\node (C) at (2, .75) {$1$};
			\node (D) at (2, -.75) {$2$};
			\draw[fill] (1.75,.75) circle (.05);
			\draw[fill] (1.75,-.75) circle (.05);
			
			\draw[-Stealth, thick] (-1.75, .75) -- (1.75, -.75);
			\draw[-Stealth, thick] (-1.75, -.75) -- (1.75, .75);
	\end{tikzpicture}}
\end{itemize}

Therefore, the elements of \( S_2 \) can be written as:
\[
S_2 = \{ \text{id}, \sigma \}
\]
\end{exercise}

\vspace{12pt}
\begin{exercise}[Symmetric Group $S_3$]
	content...
\end{exercise}

\subsection*{Group Operation}

The group operation in \( S_2 \) is the composition of permutations. Given two permutations \( f \) and \( g \), their composition \( f \circ g \) is defined as:
\[
(f \circ g)(x) = f(g(x))
\]
for all \( x \in X \).

\subsection*{Group Table (Cayley Table)}

The Cayley table for \( S_2 \) describes the result of composing any two permutations:
\[
\begin{array}{c|cc}
	\circ & \text{id} & \sigma \\
	\hline
	\text{id} & \text{id} & \sigma \\
	\sigma & \sigma & \text{id} \\
\end{array}
\]

\subsection*{Group Axioms Verification}

\begin{itemize}
	\item \textbf{Closure}:
	\begin{itemize}
		\item The composition of any two elements in \( S_2 \) is also an element of \( S_2 \).
	\end{itemize}
	
	\item \textbf{Associativity}:
	\begin{itemize}
		\item Function composition is associative. For all \( f, g, h \in S_2 \),
		\[
		(f \circ g) \circ h = f \circ (g \circ h)
		\]
	\end{itemize}
	
	\item \textbf{Identity Element}:
	\begin{itemize}
		\item The identity permutation \(\text{id}\) acts as the identity element. For all \( f \in S_2 \),
		\[
		f \circ \text{id} = \text{id} \circ f = f
		\]
	\end{itemize}
	
	\item \textbf{Inverse Element}:
	\begin{itemize}
		\item Each element in \( S_2 \) has an inverse in \( S_2 \). Specifically,
		\[
		\text{id}^{-1} = \text{id}, \quad \sigma^{-1} = \sigma
		\]
	\end{itemize}
\end{itemize}
	
	\newpage
	\chapter{Group Homomorphism}
	% Group Homomorphism
\section{Exponentiation Function}

Consider the following groups:
\begin{itemize}
	\item The \textbf{additive group on integers} \((\mathbb{Z}, +)\):
	\begin{itemize}
		\item Set: \(\mathbb{Z}\)
		\item Operation: Addition (+)
		\item Identity Element: 0
		\item Inverses: For each \(a \in \mathbb{Z}\), the inverse is \(-a\).
	\end{itemize}
	
	\item The \textbf{multiplicative group on nonzero rational numbers} \((\mathbb{Q}^*, \cdot)\):
	\begin{itemize}
		\item Set: \(\mathbb{Q}^*\)
		\item Operation: Multiplication (\(\cdot\))
		\item Identity Element: 1
		\item Inverses: For each \(q \in \mathbb{Q}^*\), the inverse is \(q^{-1} = \frac{1}{q}\).
	\end{itemize}
\end{itemize}

We define the exponential function \( \exp : \mathbb{Z} \to \mathbb{Q}^* \) by:
\[
\exp(n) = 2^n \quad \text{for all} \ n \in \mathbb{Z}.
\]

\subsection*{Verification}

\paragraph{Homomorphism Property:}
\[
\exp(a + b) = 2^{a + b} = 2^a \cdot 2^b = \exp(a) \cdot \exp(b).
\]

\paragraph{Identity Element:}
\begin{itemize}
	\item In \((\mathbb{Z}, +)\), the identity element is 0.
	\item In \((\mathbb{Q}^*, \cdot)\), the identity element is 1.
\end{itemize}
\[
\exp(0) = 2^0 = 1.
\]

\paragraph{Inverses:}
\begin{itemize}
	\item For each \( n \in \mathbb{Z} \), the inverse of \( n \) in \(\mathbb{Z}\) is \(-n\).
	\item The inverse of \( \exp(n) = 2^n \) in \(\mathbb{Q}^*\) should be \( \exp(-n) = 2^{-n} \).
\end{itemize}
\[
\exp(-n) = 2^{-n} = \frac{1}{2^n} = (\exp(n))^{-1}.
\]

Thus, the exponential function \( \exp(n) = 2^n \) preserves the group structure between the additive group on integers \((\mathbb{Z}, +)\) and the multiplicative group on nonzero rational numbers \((\mathbb{Q}^*, \cdot)\).

\end{document}
	
	\newpage
	\chapter{Linear Algebra and Group}
	% Linear Algebra and Group
\begin{note}[General Definition of Vector Space]
The operations $+$ and $\cdot$ must satisfy the following properties for all $\mathbf{u}, \mathbf{v} \in V$ and $\alpha, \beta \in \mathbb{F}$:
\begin{enumerate}
	\item \textbf{Associativity of Addition}:
	\begin{equation*}
		(\mathbf{u} + \mathbf{v}) + \mathbf{w} = \mathbf{u} + (\mathbf{v} + \mathbf{w})
	\end{equation*}
	\item \textbf{Commutativity of Addition}:
	\begin{equation*}
		\mathbf{u} + \mathbf{v} = \mathbf{v} + \mathbf{u}
	\end{equation*}
	\item \textbf{Existence of Additive Identity}:
	\begin{equation*}
		\vec{u}\in V\implies\exists \mathbf{0} \in V:\mathbf{u} + \mathbf{0} = \mathbf{u}
	\end{equation*}
	\item \textbf{Existence of Additive Inverse}:
	\begin{equation*}
		\vec{u}\in V\implies \exists -\mathbf{u} \in V : \mathbf{u} + (-\mathbf{u}) = \mathbf{0}
	\end{equation*}
	\item \textbf{Distributivity of Scalar Multiplication over Vector Addition}:
	\begin{equation*}
		\alpha \cdot (\mathbf{u} + \mathbf{v}) = (\alpha \cdot \mathbf{u}) + (\alpha \cdot \mathbf{v})
	\end{equation*}
	\item \textbf{Distributivity of Scalar Multiplication over Field Addition}:
	\begin{equation*}
		(\alpha + \beta) \cdot \mathbf{u} = (\alpha \cdot \mathbf{u}) + (\beta \cdot \mathbf{u})
	\end{equation*}
	\item \textbf{Compatibility of Scalar Multiplication with Field Multiplication}:
	\begin{equation*}
		(\alpha \beta) \cdot \mathbf{u} = \alpha \cdot (\beta \cdot \mathbf{u})
	\end{equation*}
	\item \textbf{Identity Element of Scalar Multiplication}:
	\begin{equation*}
		1 \cdot \mathbf{u} = \mathbf{u}
	\end{equation*}
\end{enumerate}
\end{note}

\defbox[Linear Operation]{\begin{definition}
Let $V$ be a set over a field $\mathbb{F}$. We define the following linear operations on $V$:
\begin{enumerate}
	\item An \textbf{addition operation} 
	\begin{align*}
		+ : V \times V &\rightarrow V \\
		(\mathbf{u}, \mathbf{v}) &\mapsto \mathbf{u} + \mathbf{v}
	\end{align*}
	on $V$ such that $(V, +)$ is an abelian group.
		
	\item A \textbf{scalar multiplication operation} 
	\begin{align*}
		\cdot : \mathbb{F} \times V &\rightarrow V \\
		(\alpha, \mathbf{u}) &\mapsto \alpha \cdot \mathbf{u}
	\end{align*}
	on $V$. Here $0\cdot\vec{u}:=\vec{0}$ and $1\cdot\vec{u}:=\vec{u}$, \ie, \begin{align*}
		\cdot : \F \times V &\rightarrow V & \cdot : \F \times V &\rightarrow V \\
		(0, \mathbf{u}) &\mapsto \mathbf{0} & (1, \mathbf{u}) &\mapsto \mathbf{u}
	\end{align*}
\end{enumerate}
\end{definition}}
\begin{remark}
Consider \[
\cdot:\F\to[V\to V].
\] Then 
\begin{center}\adjustbox{scale=.9}{
\begin{tikzpicture}
	% Draw the sets A and B
	\draw[thick] (-4,0) ellipse (2 and 3);
	\draw[thick] ( 4,0) ellipse (2 and 3);
	
	% Labels for sets
	\node at (-4, 3.25) {$\F$};
	\node at ( 4, 3.25) {$V^V$};
	
	% Draw the arrows representing the function
	\draw[-Stealth, thick] (-3.5, 3.25) -- (3.5,3.25) node[midway, above] {$\cdot$};
	\draw[fill] (-4,.75) circle (.05) node[left] {$0=e_{(\F,+)}$};
	\draw[fill] (-4,-.75) circle (.05) node[left] {$1=e_{(\F,\times)}$};
	
	\draw[fill] (4,.75) circle (.05) node[right] {$Z_V$ (Zero Transformation)};
	\draw[fill] (4,-.75) circle (.05) node[right] {$I_V$ (Identity Transformation)};
	
	\draw[-Stealth, thick] (-4, .75) -- (4, .75);
	\draw[-Stealth, thick] (-4, -.75) -- (4, -.75);
\end{tikzpicture}}
\end{center}
\end{remark}
\begin{remark}
	Let $V$ be a set over a field $\F$. Assume that, for $\vec{x},\vec{y}\in V$ and $\alpha,\beta\in F$, \[
	\alpha\cdot\vec{x} + \beta\cdot\vec{y}\in V.
	\] Then \[\begin{array}{rcll}
		\alpha=1=\beta &\implies &\vec{x} + \vec{y} \in V &\cdots\cdots\text{(Additivity)} \\
		\beta=0 &\implies &\alpha\cdot \vec{x} \in V &\cdots\cdots\text{(Homogeneity)}
	\end{array}
	\]
\end{remark}

\defbox[Vector Space]{\begin{definition}
A \textbf{vector space} \((V,+,\cdot)\), simply \( V \), over a field \( F \) is a set \( V \) together with two operations:
\begin{enumerate}
	\item \textbf{Vector Addition:} \[
	\fullfunction{+}{V\times V}{V}{(\vec{u},\vec{v})}{\vec{u}+\vec{v}},
	\] such that \( (V, +) \) forms an \underline{abelian group}.
	\item \textbf{Scalar Multiplication:} \[
	\fullfunction{\cdot}{F\times V}{V}{(a,\vec{u})}{a\cdot\vec{u}},
	\] such that \( (V, \cdot) \) satisfies the following properties:
	\begin{enumerate}
		\item \textbf{Distributivity of Scalar Multiplication with Respect to Vector Addition:} \[
		a \cdot (\mathbf{u} + \mathbf{v}) = (a \cdot \mathbf{u}) + (a \cdot \mathbf{v}).
		\]
		\item \textbf{Distributivity of Scalar Multiplication with Respect to Field Addition:} \[
		(a + b) \cdot \mathbf{v} = (a \cdot \mathbf{v}) + (b \cdot \mathbf{v}).
		\]
		\item \textbf{Associativity of Scalar Multiplication:} \[
		a \cdot (b \cdot \mathbf{v}) = (a \cdot b) \cdot \mathbf{v}.
		\]
		\item \textbf{Multiplicative Identity:} \[
		\vec{v}\in V\implies 1 \cdot \mathbf{v} = \mathbf{v},
		\] where 1 is the multiplicative identity in \( F \).
	\end{enumerate}
\end{enumerate}
\end{definition}}

\defbox[Linear Transformation]{\begin{definition}
Let \( V \) and \( W \) be vector spaces over the same field \( F \). A function \( T : V \to W \) is called a \textbf{linear transformation} (or linear map) if for all \( \mathbf{u}, \mathbf{v} \in V \) and all scalars \( a \in F \), the following two conditions are satisfied:

\begin{enumerate}
	\item \textbf{Additivity:} \[
	T(\mathbf{u} + \mathbf{v}) = T(\mathbf{u}) + T(\mathbf{v}).
	\]
	\item \textbf{Homogeneity of Scalar Multiplication:} \[
	T(a \cdot \mathbf{u}) = a \cdot T(\mathbf{u}).
	\]
\end{enumerate}
That is, \( T \) preserves the operations of vector addition and scalar multiplication.
\end{definition}}
\begin{remark}
Given that \( T : V \to W \) is a linear transformation, the following properties hold:
\begin{enumerate}
	\item \( T(\mathbf{0}_V) = \mathbf{0}_W \), where \( \mathbf{0}_V \) and \( \mathbf{0}_W \) are the zero vectors in \( V \) and \( W \), respectively.
	\item \( T\left(\sum_{i=1}^n a_i \mathbf{u}_i\right) = \sum_{i=1}^n a_i T(\mathbf{u}_i) \) for any finite set of vectors \( \mathbf{u}_1, \mathbf{u}_2, \ldots, \mathbf{u}_n \in V \) and scalars \( a_1, a_2, \ldots, a_n \in F \).
\end{enumerate}
\end{remark}
\begin{remark}
\ \begin{center}
\begin{tikzpicture}[auto, node distance=2cm, thick, >=Stealth]
	% Vector Addition
	\node (O1) at (0,0) {};
	\node (U) at (2,1) {};
	\node (V) at (1,2) {};
	\node (U+V) at ($(U) + (V)$) {};
	
	\node (TO1) at (0,-5) {};
	\node (TU) at (2,-4) {};
	\node (TV) at (1,-3) {};
	\node (TU+TV) at (3,-2) {};
	
	% Vectors for Addition
	\draw[->] (O1.center) -- (U.center) node[midway, below right] {$\mathbf{u}$};
	\draw[->] (O1.center) -- (V.center) node[midway, above left] {$\mathbf{v}$};
	\draw[->, dashed] (U.center) -- (U+V.center) node[midway, above right] {};
	\draw[->, dashed] (V.center) -- (U+V.center) node[midway, below left] {};
	\draw[->, very thick, magenta] (O1.center) -- (U+V.center) node[midway, above] {};
	
	\draw[->] (TO1.center) -- (TU.center) node[midway, below right] {$T(\mathbf{u})$};
	\draw[->] (TO1.center) -- (TV.center) node[midway, above left] {$T(\mathbf{v})$};
	\draw[->, dashed] (TU.center) -- (TU+TV.center) node[midway, above right] {};
	\draw[->, dashed] (TV.center) -- (TU+TV.center) node[midway, below left] {};
	\draw[->, very thick, magenta] (TO1.center) -- (TU+TV.center) node[midway, above] {};
	
	% Points for Addition
	\fill (O1) circle (2pt) node[below left] {$\mathbf{O}$};
	\fill (U) circle (2pt) node[below right] {};
	\fill (V) circle (2pt) node[above left] {};
	\fill (U+V) circle (2pt) node[above right] {\textcolor{magenta}{$\mathbf{u} + \mathbf{v}$}};
	
	\fill (TO1) circle (2pt) node[below left] {$\mathbf{O}$};
	\fill (TU) circle (2pt) node[below right] {};
	\fill (TV) circle (2pt) node[above left] {};
	\fill (TU+TV) circle (2pt) node[above right] {\textcolor{magenta}{$T(\mathbf{u}) + T(\mathbf{v})$}};
	
	% Scalar Multiplication
	\node (O2) at (7,0) {};
	\node (U2) at (9,1) {};
	\node (AU2) at ($(O2)!2!(U2)$) {};
	
	\node (TO2) at (7,-5) {};
	\node (TU2) at (9,-4) {};
	\node (TAU2) at (11,-3) {};
	
	% Vectors for Multiplication
	\draw[->] (O2.center) -- (U2.center) node[midway, below right] {$\mathbf{u}$};
	\draw[->, very thick, magenta] (O2.center) -- (AU2.center) node[midway, above] {};
	
	\draw[->] (TO2.center) -- (TU2.center) node[midway, below right] {$T(\mathbf{u})$};
	\draw[->, very thick, magenta] (TO2.center) -- (TAU2.center) node[midway, above] {};
	
	% Points for Multiplication
	\fill (O2) circle (2pt) node[below left] {$\mathbf{O}$};
	\fill (U2) circle (2pt) node[below right] {};
	\fill (AU2) circle (2pt) node[above] {\textcolor{magenta}{$a \cdot \mathbf{u}$}};
	
	\fill (TO2) circle (2pt) node[below left] {$\mathbf{O}$};
	\fill (TU2) circle (2pt) node[below right] {};
	\fill (TAU2) circle (2pt) node[above] {\textcolor{magenta}{$a \cdot T(\mathbf{u})$}};
	
	% Mapping
	\draw[draw=black] (-1, 3.5) rectangle (6,-.5);
	\draw[draw=black] (-1, -1.25) rectangle (6,-5.75);
	\draw[draw=black] (6.25, 3.5) rectangle (13,-.5);
	\draw[draw=black] (6.25, -1.25) rectangle (13,-5.75);
%	\draw[|-Stealth, thick, shorten <= 10pt, shorten >= 5pt] (3, 3) to node[midway, right] {$T(\vec{u}+\vec{v})$} (3,-2);
%	\draw[|-Stealth, thick, shorten <= 10pt, shorten >= 5pt] (11, 2) to node[midway, right] {$T(a\cdot\vec{u})$} (11,-3);
\end{tikzpicture}
%\vspace{36pt}
%\begin{tikzpicture}[auto, node distance=2cm, thick, >=Stealth]
%	% Vector Addition
%	\node (O1) at (0,0) {};
%	\node (U) at (2,1) {};
%	\node (V) at (1,2) {};
%	\node (U+V) at ($(U) + (V)$) {};
%	
%	% Vectors for Addition
%	\draw[->] (O1.center) -- (U.center) node[midway, below right] {$T(\mathbf{u})$};
%	\draw[->] (O1.center) -- (V.center) node[midway, above left] {$T(\mathbf{v})$};
%	\draw[->, dashed] (U.center) -- (U+V.center) node[midway, above right] {};
%	\draw[->, dashed] (V.center) -- (U+V.center) node[midway, below left] {};
%	\draw[->, very thick, magenta] (O1.center) -- (U+V.center) node[midway, above] {};
%	
%	% Points for Addition
%	\fill (O1) circle (2pt) node[below left] {$\mathbf{O}$};
%	\fill (U) circle (2pt) node[below right] {};
%	\fill (V) circle (2pt) node[above left] {};
%	\fill (U+V) circle (2pt) node[above right] {\textcolor{magenta}{$T(\mathbf{u}) + T(\mathbf{v})$}};
%	
%	% Scalar Multiplication
%	\node (O2) at (7,0) {};
%	\node (U2) at (9,1) {};
%	\node (AU2) at ($(O2)!2!(U2)$) {};
%	
%	% Vectors for Multiplication
%	\draw[->] (O2.center) -- (U2.center) node[midway, below right] {$T(\mathbf{u})$};
%	\draw[->, very thick, magenta] (O2.center) -- (AU2.center) node[midway, above] {};
%	
%	% Points for Multiplication
%	\fill (O2) circle (2pt) node[below left] {$\mathbf{O}$};
%	\fill (U2) circle (2pt) node[below right] {};
%	\fill (AU2) circle (2pt) node[above] {\textcolor{magenta}{$a \cdot T(\mathbf{u})$}};
%\end{tikzpicture}
\end{center}
\end{remark}


	
	From the standpoint of solving equations, the algebraic structures we have discussed are interconnected in the following ways:
	
	\subsection*{Group}
	A \textbf{group} provides a foundation for solving equations involving a single operation. The group structure ensures that every element has an inverse, allowing us to "undo" operations and solve equations. For instance, in the group of integers \( (\mathbb{Z}, +) \), the equation \( x + a = b \) can be solved as \( x = b - a \).
	
	\subsection*{Ring}
	A \textbf{ring} extends the concept of a group by introducing a second operation, typically multiplication, in addition to addition. Rings allow us to solve more complex equations that involve both addition and multiplication. For example, solving polynomial equations \( f(x) = 0 \) where \( f(x) \) is a polynomial with coefficients in a ring \( R \).
	
	\subsection*{Field}
	A \textbf{field} is a ring with additional properties that make division possible (except by zero). This structure is crucial for solving linear equations and systems of linear equations. In a field \( F \), any linear equation \( ax = b \) (where \( a \neq 0 \)) can be solved as \( x = a^{-1}b \), where \( a^{-1} \) is the multiplicative inverse of \( a \).
	
	\subsection*{Vector Space}
	A \textbf{vector space} over a field \( F \) is a set of vectors that can be added together and scaled by elements of \( F \). Vector spaces provide a framework for solving linear systems of equations. Solutions to systems of linear equations can be understood as finding vectors \( \mathbf{x} \in V \) such that \( A\mathbf{x} = \mathbf{b} \), where \( A \) is a matrix and \( \mathbf{b} \) is a vector in \( V \).
	
	\subsection*{Module}
	A \textbf{module} is a generalization of vector spaces where the scalars come from a ring instead of a field. Modules allow us to solve equations in contexts where the coefficients are not from a field, such as systems of linear equations with integer coefficients. For example, solving \( A\mathbf{x} = \mathbf{b} \) where \( A \) is a matrix with entries from a ring \( R \) and \( \mathbf{x}, \mathbf{b} \in M \).
	
	\subsection*{Algebra}
	An \textbf{algebra} over a field \( F \) combines the structures of a vector space and a ring. Algebras provide a framework for solving polynomial equations and other equations involving both addition and multiplication of vectors. In an algebra \( A \), we can solve equations like \( x^2 + ax + b = 0 \) using techniques from both linear algebra and ring theory.
	
	\subsection*{Summary}
	The conceptual connection between these structures is rooted in the increasing complexity and capability they offer for solving equations:
	\begin{itemize}
		\item \textbf{Groups} provide a way to solve equations with a single operation.
		\item \textbf{Rings} introduce a second operation, allowing for more complex equations.
		\item \textbf{Fields} enable division, which is essential for solving linear equations.
		\item \textbf{Vector spaces} over fields extend these concepts to systems of linear equations.
		\item \textbf{Modules} generalize vector spaces to allow coefficients from rings.
		\item \textbf{Algebras} integrate vector space and ring structures, enabling the solution of polynomial and other complex equations.
	\end{itemize}
	
	\newpage
	\chapter*{LaTex Practice}
	\begin{itemize}
	\item \textbf{Block Size}: $n$ (number of bits in a block)
	\item \textbf{Key Size}: $k$ (number of bits in the key)
\end{itemize}
\begin{align*}
	E:\boxed{\set{0,1}^k\times\set{0,1}^n}\to\set{0,1}^n \\
	D:\boxed{\set{0,1}^k\times\set{0,1}^n}\to\set{0,1}^n
\end{align*}
\adjustbox{scale=.9, center}{
	\begin{tikzpicture}
		% Draw the sets A and B
		\draw[thick] (-6,0) ellipse (1.5 and 2);
		\draw[thick] (-2,0) ellipse (1.5 and 2);
		\draw[thick] (4,0) ellipse (1.5 and 2);
		\draw[draw=black] (-8, 4) rectangle (0,-4);
		% Labels for sets
		\node at (-6, 2.25) {$\mathcal{K}$};
		\node at (-4, 2.25) {$\times$};
		\node at (-2, 2.25) {$\mathcal{M}$};
		\node at (4, 2.25) {$\mathcal{C}$};
		
		% Draw the arrows representing the function
		\draw[-Stealth, thick] (0, 2.25) -- (3.5,2.25) node[midway, above] {$E$};
		
		\node (K) at (-6, .45) {$k$};
		\node (M) at (-2, .45) {$m$};
		\node (C) at (4, .45) {$c=E(k,m)$};
		\draw[fill] (-6,0) circle (.05);
		\draw[fill] (-2,0) circle (.05);
		\draw[fill] (4,0) circle (.05);
		
		\draw[|-Stealth, thick, bend right=25pt, shorten <= 5pt, shorten >= 5pt] (-6, 0) to node[midway,below] {} (4,0);
		\draw[|-Stealth, thick, bend right=25pt, shorten <= 5pt, shorten >= 5pt] (-2, 0) to node[midway,below] {} (4,0);
\end{tikzpicture}}
\begin{align*}
	E:\set{0,1}^k\to\boxed{\set{0,1}^n\to\set{0,1}^n} \\
	D:\set{0,1}^k\to\boxed{\set{0,1}^n\to\set{0,1}^n}
\end{align*}
\adjustbox{scale=.9, center}{
	\begin{tikzpicture}
		% Draw the sets A and B
		\draw[thick] (-6,0) ellipse (1.5 and 2);
		\draw[thick] (-2,0) ellipse (1.5 and 2);
		\draw[thick] (4,0) ellipse (1.5 and 2);
		\draw[draw=black] (-4, 4) rectangle (6,-4);
		% Labels for sets
		\node at (-6, 2.25) {$\mathcal{K}$};
		\node at (-2, 2.25) {$\mathcal{M}$};
		\node at (4, 2.25) {$\mathcal{C}$};
		
		% Draw the arrows representing the function
		\draw[-Stealth, thick] (-1.5, 2.25) -- (3.5,2.25) node[midway, above] {$E_k$};
		\draw[-Stealth, thick] (-5.5, 2.25) -- (-4,2.25) node[midway, above] {};
		
		\node (K) at (-6, .45) {$k$};
		\node (M) at (-2, .45) {$m$};
		\node (C) at (4, .45) {$c=E_k(m)$};
		\draw[fill] (-6,0) circle (.05);
		\draw[fill] (-2,0) circle (.05);
		\draw[fill] (4,0) circle (.05);
		
		\draw[|-Stealth, thick] (-5.5, 0) to node[midway,below] {} (-4,0);
		\draw[|-Stealth, thick] (-1.5, 0) to node[midway,below] {} (3.5,0);
\end{tikzpicture}}
	
	% Appendix
	\newpage
	\appendix
	\appendix
\chapter{Preliminaries}
\section{Sets, Cartesian Products, and Relations}

\subsection{Sets and Ordered Pairs}

\subsection*{Set}

A \textbf{set} is a well-defined collection of distinct objects, called elements or members of the set. Sets are one of the most fundamental concepts in mathematics.

\defbox[Set]{
\begin{definition}
	A \textbf{set} is a well-defined collection of distinct objects, considered as an object in its own right. Sets are usually denoted by capital letters, and the elements are listed within curly braces.
\end{definition}
}
\begin{example}
For example:
\[
A = \{1, 2, 3\}
\]
This denotes a set \(A\) containing the elements 1, 2, and 3.
\end{example}
\vspace{12pt}
\begin{note}[Properties]
\ \begin{itemize}
	\item \textbf{No Repetition}: Each element in a set appears only once.
	\item \textbf{Order Irrelevance}: The order of elements in a set does not matter. For instance, \(\{1, 2, 3\} = \{3, 2, 1\}\).
	\item \textbf{Membership}: If an element \(a\) is in a set \(A\), we write \(a \in A\).
\end{itemize}
\end{note}
\vspace{12pt}
\begin{note}[Types of Sets]
\ \begin{itemize}
	\item \textbf{Finite and Infinite Sets}: A set with a finite number of elements is finite; otherwise, it is infinite.
	\item \textbf{Subset}: A set \(A\) is a subset of a set \(B\) if every element of \(A\) is also an element of \(B\), denoted \(A \subseteq B\).
	\item \textbf{Power Set}: The power set of \(A\) is the set of all subsets of \(A\), denoted \(\mathcal{P}(A)\).
\end{itemize}
\end{note}

\subsection*{Ordered Pair}

An \textbf{ordered pair} is a fundamental concept in mathematics used to combine two elements in a specific order. The notation for an ordered pair is \((a, b)\), where \(a\) is the first element and \(b\) is the second element.

\defbox[Ordered Pair]{
\begin{definition}
	An \textbf{ordered pair} \((a, b)\) is a collection of two elements where the order of the elements matters. This is in contrast to a set, where the order of elements does not matter.
\end{definition}
}
\begin{remark}
	\ \begin{itemize}
		\item The ordered pair \((a, b)\) is not the same as \((b, a)\) unless \(a = b\).
		\item Formally, the ordered pair \((a, b)\) can be defined using sets to ensure the distinction from unordered pairs. One common definition is:
		\[
		(a, b) = \{ \{a\}, \{a, b\} \}
		\]
		This definition ensures that:
		\[
		(a, b) = (c, d) \iff a = c \ \text{and} \ b = d
		\]
	\end{itemize}
\end{remark}
\vspace{12pt}
\begin{note}[Properties]
\begin{itemize}
	\item \textbf{Uniqueness}: Each ordered pair \((a, b)\) is unique if either \(a\) or \(b\) is unique.
	\item \textbf{Order}: The order of elements in an ordered pair is significant.
\end{itemize}
\end{note}

\subsection{Cartesian Product and Relation}

\subsection*{Cartesian Product}

The \textbf{Cartesian product} is a fundamental concept in set theory, used to define the set of all possible ordered pairs from two sets.

\defbox[Cartesian Product]{
Given two sets \( A \) and \( B \), the Cartesian product \( A \times B \) is defined as the set of all ordered pairs \((a, b)\) where \( a \in A \) and \( b \in B \). Formally,
\[
A \times B = \{ (a, b) \mid a \in A \ \text{and} \ b \in B \}
\]
}
\vspace{12pt}
\begin{note}[Properties]
\ \begin{itemize}
	\item \textbf{Order Matters}: The pair \((a, b)\) is different from the pair \((b, a)\) unless \(a = b\).
	\item \textbf{Empty Set}: If either \(A\) or \(B\) is the empty set \(\emptyset\), then \(A \times B\) is also empty:
	\[
	A \times \emptyset = \emptyset \quad \text{and} \quad \emptyset \times B = \emptyset
	\]
\end{itemize}
\end{note}
\begin{example}
\ \begin{enumerate}
	\item If \( A = \{1, 2\} \) and \( B = \{x, y\} \), then
	\[
	A \times B = \{ (1, x), (1, y), (2, x), (2, y) \}
	\]
	
	\item If \( A = \{a, b\} \) and \( B = \{1, 2, 3\} \), then
	\[
	A \times B = \{ (a, 1), (a, 2), (a, 3), (b, 1), (b, 2), (b, 3) \}
	\]
\end{enumerate}
\end{example}

\subsection*{Relation}

A \textbf{relation} generalizes the concept of Cartesian product to establish connections between elements of two sets.

\defbox[Relation]{
\begin{definition}
	A relation \( R \) from a set \( A \) to a set \( B \) is a subset of the Cartesian product \( A \times B \). Formally,
	\[
	R \subseteq A \times B
	\]
	This means that a relation \( R \) consists of ordered pairs \((a, b)\) where \(a \in A\) and \(b \in B\).
\end{definition}
}
\vspace{12pt}
\begin{note}[Properties of Relations]
\ \begin{itemize}
	\item \textbf{Domain and Range}:
	\begin{itemize}
		\item The \textbf{domain} of \( R \) is the set of all \( a \in A \) such that there exists \( b \in B \) with \((a, b) \in R\).
		\[
		\text{Domain}(R) = \{ a \in A \mid \exists b \in B, \ (a, b) \in R \}
		\]
		\item The \textbf{range} of \( R \) is the set of all \( b \in B \) such that there exists \( a \in A \) with \((a, b) \in R\).
		\[
		\text{Range}(R) = \{ b \in B \mid \exists a \in A, \ (a, b) \in R \}
		\]
	\end{itemize}
	
	\item \textbf{Inverse Relation}: The inverse of a relation \( R \), denoted \( R^{-1} \), is the set of all pairs \((b, a)\) such that \((a, b) \in R\):
	\[
	R^{-1} = \{ (b, a) \mid (a, b) \in R \}
	\]
	
	\item \textbf{Composition of Relations}: Given a relation \( R \) from \( A \) to \( B \) and a relation \( S \) from \( B \) to \( C \), the composition \( S \circ R \) is a relation from \( A \) to \( C \) defined by:
	\[
	S \circ R = \{ (a, c) \mid \exists b \in B, \ (a, b) \in R \ \text{and} \ (b, c) \in S \}
	\]
\end{itemize}
\end{note}

\vspace{12pt}
\begin{note}[Types of Relations]
\ \begin{itemize}
	\item \textbf{Binary Relation}: A relation involving two sets, as defined above.
	\item \textbf{Unary Relation}: A relation on a single set \( A \) is simply a subset of \( A \).
	\item \textbf{Ternary and Higher Relations}: Relations involving three or more sets, defined as subsets of the Cartesian product of those sets.
\end{itemize}
\end{note}

\begin{example}
	\ \begin{enumerate}
		\item If \( A = \{1, 2, 3\} \) and \( B = \{a, b\} \), a possible relation \( R \) from \( A \) to \( B \) could be:
		\[
		R = \{ (1, a), (2, b), (3, a) \}
		\]
		\begin{itemize}
			\item Domain: \(\{1, 2, 3\}\)
			\item Range: \(\{a, b\}\)
		\end{itemize}
		
		\item Consider the relation \( R \) on set \( A = \{1, 2, 3\} \) defined by:
		\[
		R = \{ (1, 2), (2, 3), (3, 1) \}
		\]
		\begin{itemize}
			\item Domain: \(\{1, 2, 3\}\)
			\item Range: \(\{1, 2, 3\}\)
			\item Inverse Relation: \( R^{-1} = \{ (2, 1), (3, 2), (1, 3) \} \)
		\end{itemize}
	\end{enumerate}
\end{example}
	
\end{document}
