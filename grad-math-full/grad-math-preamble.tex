\usepackage{mathtools, commath}
% Packages for formatting
\usepackage[margin=1in]{geometry}
\usepackage{fancyhdr}
\usepackage{enumerate}
\usepackage{graphicx}
\usepackage{kotex}
\usepackage{arydshln} % Include this package
\usepackage{bbding}
\usepackage{amsmath}
\usepackage{amsthm}
\usepackage[dvipsnames,table]{xcolor}
\usepackage{amssymb, amsfonts}
\usepackage{wasysym}
\usepackage{footnote}
\usepackage{tablefootnote}
\usepackage{arydshln} % Include this package
\usepackage{adjustbox}
% Fonts
\usepackage[T1]{fontenc}
\usepackage[utf8]{inputenc}
\usepackage{newpxtext,newpxmath}
\usepackage{sectsty}

% Define colors
\definecolor{TealBlue1}{HTML}{0077c2}
\definecolor{TealBlue2}{HTML}{00a5e6}
\definecolor{TealBlue3}{HTML}{b3e0ff}
\definecolor{TealBlue4}{HTML}{00293c}
\definecolor{TealBlue5}{HTML}{e6f7ff}

\definecolor{thmcolor}{RGB}{231, 76, 60}
\definecolor{defcolor}{RGB}{52, 152, 219}
\definecolor{lemcolor}{RGB}{155, 89, 182}
\definecolor{corcolor}{RGB}{46, 204, 113}
\definecolor{procolor}{RGB}{241, 196, 15}

\usepackage{color,soul}
\usepackage{soul}
\newcommand{\mathcolorbox}[2]{\colorbox{#1}{$\displaystyle #2$}}
\usepackage{cancel}
\newcommand\crossout[3][black]{\renewcommand\CancelColor{\color{#1}}\cancelto{#2}{#3}}
\newcommand\ncrossout[2][black]{\renewcommand\CancelColor{\color{#1}}\cancel{#2}}

\usepackage{hyperref}
\usepackage{booktabs}

% Chapter formatting
\definecolor{titleTealBlue}{RGB}{0,53,128}
\usepackage{titlesec}
\titleformat{\section}
{\normalfont\sffamily\Large\bfseries\color{titleTealBlue!100!gray}}{\thesection}{1em}{}
\titleformat{\subsection}
{\normalfont\sffamily\large\bfseries\color{titleTealBlue!50!gray}}{\thesubsection}{1em}{}

%Tcolorbox
\usepackage[most]{tcolorbox}
\usepackage{multirow}
\usepackage{multicol}
\usepackage{blindtext}

\usepackage[linesnumbered,ruled]{algorithm2e}
\usepackage{algpseudocode}
\usepackage{setspace}
\SetKwComment{Comment}{/* }{ */}
\SetKwProg{Fn}{Function}{:}{end}
\SetKw{End}{end}
\SetKw{DownTo}{downto}

% Define a new environment for algorithms without line numbers
\newenvironment{algorithm2}[1][]{
	% Save the current state of the algorithm counter
	\newcounter{tempCounter}
	\setcounter{tempCounter}{\value{algocf}}
	% redefine the algorithm numbering (remove prefix)
	\renewcommand{\thealgocf}{}
	\begin{algorithm}
	}{
	\end{algorithm}
	% Restore the algorithm counter state
	\setcounter{algocf}{\value{tempCounter}}
}

\usepackage{adjustbox}
% Header and footer formatting
\pagestyle{fancy}
\fancyhead{}
\fancyhf{}
\rhead{\textcolor{TealBlue2}{\large\textbf{기대수(기초부터 대학원 수학까지 시리즈) 3기}}}%\rule{3cm}{0.4pt}}
\lhead{\textcolor{TealBlue2}{\large\textbf{수학의 즐거움, Enjoying Math}}}
% Define footer
%\newcommand{\footer}[1]{
%\begin{flushright}
%	\vspace{2em}
%	\includegraphics[width=2.5cm]{school_logo.jpg} \\
%	\vspace{1em}
%	\textcolor{TealBlue2}{\small\textbf{#1}}
%\end{flushright}
%}
%\rfoot{\large Department of Information Security, Cryptogrphy and Mathematics, Kookmin Uni.\includegraphics[height=1.5cm]{school_logo.jpg}}
%\fancyfoot{}
%\fancyfoot[L]{\it \textcolor{TealBlue}{\bfseries Think Globally, Act Locally}}
\fancyfoot[C]{-\thepage-}
