\documentclass[11pt,openany]{article}

\usepackage{mathtools, commath}
% Packages for formatting
\usepackage[margin=1in]{geometry}
\usepackage{fancyhdr}
\usepackage{enumerate}
\usepackage{graphicx}
\usepackage{kotex}
\usepackage{arydshln} % Include this package
\usepackage{bbding}
\usepackage{amsmath}
\usepackage{amsthm}
\usepackage[dvipsnames,table]{xcolor}
\usepackage{amssymb, amsfonts}
\usepackage{wasysym}
\usepackage{footnote}
\usepackage{tablefootnote}
\usepackage{arydshln} % Include this package
% Fonts
\usepackage[T1]{fontenc}
\usepackage[utf8]{inputenc}
\usepackage{newpxtext,newpxmath}
\usepackage{sectsty}

% Define colors
\definecolor{TealBlue1}{HTML}{0077c2}
\definecolor{TealBlue2}{HTML}{00a5e6}
\definecolor{TealBlue3}{HTML}{b3e0ff}
\definecolor{TealBlue4}{HTML}{00293c}
\definecolor{TealBlue5}{HTML}{e6f7ff}

\definecolor{thmcolor}{RGB}{231, 76, 60}
\definecolor{defcolor}{RGB}{52, 152, 219}
\definecolor{lemcolor}{RGB}{155, 89, 182}
\definecolor{corcolor}{RGB}{46, 204, 113}
\definecolor{procolor}{RGB}{241, 196, 15}

\usepackage{color,soul}
\usepackage{soul}
\newcommand{\mathcolorbox}[2]{\colorbox{#1}{$\displaystyle #2$}}
\usepackage{cancel}
\newcommand\crossout[3][black]{\renewcommand\CancelColor{\color{#1}}\cancelto{#2}{#3}}
\newcommand\ncrossout[2][black]{\renewcommand\CancelColor{\color{#1}}\cancel{#2}}

\usepackage{hyperref}
\usepackage{booktabs}

% Chapter formatting
\definecolor{titleTealBlue}{RGB}{0,53,128}
\usepackage{titlesec}
\titleformat{\section}
{\normalfont\sffamily\Large\bfseries\color{titleTealBlue!100!gray}}{\thesection}{1em}{}
\titleformat{\subsection}
{\normalfont\sffamily\large\bfseries\color{titleTealBlue!50!gray}}{\thesubsection}{1em}{}

%Tcolorbox
\usepackage[most]{tcolorbox}
\usepackage{multirow}
\usepackage{multicol}

\usepackage[linesnumbered,ruled]{algorithm2e}
\usepackage{algpseudocode}
\usepackage{setspace}
\SetKwComment{Comment}{/* }{ */}
\SetKwProg{Fn}{Function}{:}{end}
\SetKw{End}{end}
\SetKw{DownTo}{downto}

% Define a new environment for algorithms without line numbers
\newenvironment{algorithm2}[1][]{
	% Save the current state of the algorithm counter
	\newcounter{tempCounter}
	\setcounter{tempCounter}{\value{algocf}}
	% redefine the algorithm numbering (remove prefix)
	\renewcommand{\thealgocf}{}
	\begin{algorithm}
	}{
	\end{algorithm}
	% Restore the algorithm counter state
	\setcounter{algocf}{\value{tempCounter}}
}

\usepackage{adjustbox}
% Header and footer formatting
\pagestyle{fancy}
\fancyhead{}
\fancyhf{}
\rhead{\textcolor{TealBlue2}{\large\textbf{기대수(기초부터 대학원 수학까지 시리즈) 3기}}}%\rule{3cm}{0.4pt}}
\lhead{\textcolor{TealBlue2}{\large\textbf{수학의 즐거움, Enjoying Math}}}
% Define footer
%\newcommand{\footer}[1]{
%\begin{flushright}
%	\vspace{2em}
%	\includegraphics[width=2.5cm]{school_logo.jpg} \\
%	\vspace{1em}
%	\textcolor{TealBlue2}{\small\textbf{#1}}
%\end{flushright}
%}
%\rfoot{\large Department of Information Security, Cryptogrphy and Mathematics, Kookmin Uni.\includegraphics[height=1.5cm]{school_logo.jpg}}
\fancyfoot{}
\fancyfoot[C]{-\thepage-}

\usepackage{tcolorbox}
\tcbset{colback=white, arc=5pt}

\definecolor{axiomcolor}{HTML}{a88bfa}
\definecolor{defcolor}{RGB}{52, 152, 219}
\definecolor{procolor}{RGB}{241, 196, 15}
\definecolor{thmcolor}{RGB}{231, 76, 60}
\definecolor{lemcolor}{RGB}{155, 89, 182}
\definecolor{corcolor}{RGB}{46, 204, 113}
\definecolor{execolor}{RGB}{90, 128, 127}

% Define a new command for the custom tcolorbox
\newcommand{\axiombox}[2][]{%
	\begin{tcolorbox}[colframe=axiomcolor, title={\color{white}\bfseries #1}]
		#2
	\end{tcolorbox}
}

\newcommand{\defbox}[2][]{%
	\begin{tcolorbox}[colframe=defcolor, title={\color{white}\bfseries #1}]
		#2
	\end{tcolorbox}
}

\newcommand{\lembox}[2][]{%
	\begin{tcolorbox}[colframe=lemcolor, title={\color{white}\bfseries #1}]
		#2
	\end{tcolorbox}
}

\newcommand{\probox}[2][]{%
	\begin{tcolorbox}[colframe=procolor, title={\color{white}\bfseries #1}]
		#2
	\end{tcolorbox}
}

\newcommand{\thmbox}[2][]{%
	\begin{tcolorbox}[colframe=thmcolor, title={\color{white}\bfseries #1}]
		#2
	\end{tcolorbox}
}

\newcommand{\corbox}[2][]{%
	\begin{tcolorbox}[colframe=corcolor, title={\color{white}\bfseries #1}]
		#2
	\end{tcolorbox}
}



\usepackage{amsthm}

% Define custom theorem styles
\newtheoremstyle{dotless} % Name of the style
{3pt} % Space above
{3pt} % Space below
{\itshape} % Body font
{} % Indent amount
{\bfseries} % Theorem head font
{} % Punctuation after theorem head
{2.5mm} % Space after theorem head
{} % Theorem head spec

\newtheoremstyle{definitionstyle} % Name of the style
{3pt} % Space above
{3pt} % Space below
{} % Body font
{} % Indent amount
{\bfseries} % Theorem head font
{.} % Punctuation after theorem head
{2.5mm} % Space after theorem head
{} % Theorem head spec

% Applying custom styles
\theoremstyle{dotless}
\newtheorem{theorem}{Theorem} % Theorem environment with section-wise numbering
\newtheorem{proposition}[theorem]{Proposition} % Theorem environment with section-wise numbering
\newtheorem{lemma}[theorem]{Lemma} % Lemma shares the counter with theorem
\newtheorem{corollary}[theorem]{Corollary} % Corollary shares the counter with theorem

\theoremstyle{definitionstyle}
\newtheorem*{observation}{\textcolor{Magenta}{Observation}}
\newtheorem{definition}{Definition} % Definition shares the counter with theorem
\newtheorem{example}{Example} % Example shares the counter with theorem
\newtheorem{exercise}{Exercise} % Example shares the counter with theorem
\newtheorem{remark}{Remark} % Remark shares the counter with theorem
\newtheorem*{note}{Note}

\newtheorem*{definition*}{Definition} % Definition shares the counter with theorem
\newtheorem*{example*}{Example} % Example shares the counter with theorem
\newtheorem*{exercise*}{\textcolor{violet}{Exercise}} % Example shares the counter with theorem
\newtheorem*{remark*}{Remark} % Remark shares the counter with theorem


\usepackage{tikz}
\usepackage{tikz-cd}
\usepackage{tikz-3dplot}
\usepackage{pgfplots}
\pgfplotsset{compat=newest} % Adjust to your version of pgfplots
\def\Circlearrowleft{\ensuremath{%
		\rotatebox[origin=c]{180}{$\circlearrowleft$}}}
\def\Circlearrowright{\ensuremath{%
		\rotatebox[origin=c]{180}{$\circlearrowright$}}}
\def\CircleArrowleft{\ensuremath{%
		\reflectbox{\rotatebox[origin=c]{180}{$\circlearrowleft$}}}}
\def\CircleArrowright{\ensuremath{%
		\reflectbox{\rotatebox[origin=c]{180}{$\circlearrowright$}}}}
\usetikzlibrary{
	3d, % For 3D drawing
	angles,
	arrows,
	arrows.meta,
	backgrounds,
	bending,
	calc,
	decorations.pathmorphing,
	decorations.pathreplacing,
	decorations.markings,
	fit,
	matrix,
	patterns,
	patterns.meta,
	positioning,
	quotes,
	shadows,
	shapes,
	shapes.geometric,
	tikzmark
}
\tikzset{
	% single mid‐path arrow
	mid arrow/.style={
		decoration={
			markings,
			mark=at position 0.5 with {\arrow{Stealth[scale=1.2]}}
		},
		postaction={decorate},
	},
	% style for field arrows
	field arrow/.style={
		-{Stealth[scale=1.0]},
		thick,
		blue!70!black,
	},
}
\newcommand{\ie}{\textnormal{i.e.}}
\newcommand{\rsa}{\mathsf{RSA}}
\newcommand{\rsacrt}{\mathsf{RSA}\textendash\mathsf{CRT}}
\newcommand{\inv}[1]{#1^{-1}}

%New Command
%\newcommand{\set}[1]{\left\{#1\right\}}
\newcommand{\N}{\mathbb{N}}
\newcommand{\Z}{\mathbb{Z}}
\newcommand{\Q}{\mathbb{Q}}
\newcommand{\R}{\mathbb{R}}
\newcommand{\cR}{\mathcal{R}}
\newcommand{\C}{\mathbb{C}}
\newcommand{\F}{\mathbb{F}}
\newcommand{\nbhd}{\mathcal{N}}
\newcommand{\Log}{\operatorname{Log}}
\newcommand{\Arg}{\operatorname{Arg}}
\newcommand{\pv}{\operatorname{P.V.}}

\newcommand{\of}[1]{\left( #1 \right)} 
%\newcommand{\abs}[1]{\left\lvert #1 \right\rvert}
%\newcommand{\norm}[1]{\left\| #1 \right\|}

\newcommand{\sol}{\textcolor{magenta}{\bf Sol}}
\newcommand{\conjugate}[1]{\overline{#1}}

\newcommand{\res}{\operatorname{res}}
\DeclareMathOperator*{\Res}{\operatorname{Res}}

%\renewcommand{\Re}{\operatorname{Re}}
%\renewcommand{\Im}{\operatorname{Im}}

\newcommand{\cyclic}[1]{\langle #1 \rangle}
\newcommand{\uniform}{\overset{\$}{\leftarrow}}
\newcommand{\xmark}{\textcolor{red}{\XSolidBrush}}
\newcommand{\vmark}{\textcolor{green!75!black}{\CheckmarkBold}}

\newcommand{\gen}[1]{\langle #1 \rangle}
\newcommand{\Gen}[1]{\left\langle #1 \right\rangle}

\newcommand{\img}[1]{\text{Img}(#1)}
\newcommand{\Img}[1]{\text{Img}\left(#1\right)}
\newcommand{\preimg}[1]{\text{Img}^{-1}(#1)}
\newcommand{\Preimg}[1]{\text{Img}^{-1}\left(#1\right)}

\newcommand{\relation}{\mathrel{\mathcal{R}}}
\newcommand{\injection}{\rightarrowtail}
\newcommand{\surjection}{\twoheadrightarrow}
\newcommand{\id}{\textnormal{id}}

\newcommand{\eqclass}[1]{\left[#1\right]}

% Define custom colors for O and X
\newcommand{\yes}{\textcolor{blue}{\bf \fullmoon}}
\newcommand{\no}{\textcolor{red}{\bf \texttimes}}

\DeclarePairedDelimiter\ceil{\lceil}{\rceil}
\DeclarePairedDelimiter\floor{\lfloor}{\rfloor}
%\renewcommand{\floor}[#1]{\lfloor #1\rfloor}
%\newcommand{\Floor}[#1]{\left\lfloor #1\right\rfloor}
%\newcommand{\ceil}[#1]{\lceil #1\rceil}
%\newcommand{\Ceil}[#1]{\left\lceil #1\right\rceil}

\newcommand{\topology}{\mathscr{T}}
\newcommand{\sequence}[1]{\langle #1\rangle}
\renewcommand{\vec}[1]{\mathbf{#1}}
\setstretch{1.25}

%\usepackage{background}
%\backgroundsetup{
%	scale=3,
%	color=gray!20,
%	opacity=0.3,
%	angle=45,
%	contents={\Huge \sffamily Ji, Yong-hyeon}
%}
\begin{document}
\pagenumbering{arabic}
\begin{center}
	\huge\textbf{Linear Algebra II}\\
	\vspace{0.5em}
	\large{Ji, Yong-hyeon}\\
%	\large{\ttfamily \url{https://github.com/Hacker-Code-J}}\\
	\vspace{0.5em}
	\normalsize{\today}\\
\end{center}

\noindent 
We cover the following topics in this note.
\begin{itemize}
	\item Uniqueness of Representation with respect to a Basis; Coordinate
	\item Linear Transformation
	\item Vector Space Isomorphism (Linear Isomorphism)
	\item Classification of Vector Space (up to Isomorphism)
	\item \st{Matrix Representation of a Linear Transformation}
	\item TBA
\end{itemize}
\hrule\vspace{12pt}
%\tableofcontents
%\newpage
%\begin{tikzpicture}[scale=1, every node/.style={font=\small}]
%	
%	% Draw a faint grid for orientation (optional)
%	\draw[gray, very thin] (-1,-1) grid (7,6);
%	
%	% Draw coordinate axes
%	\draw[->] (-0.5,0) -- (7,0) node[right] {$x$};
%	\draw[->] (0,-0.5) -- (0,6) node[above] {$y$};
%	
%	% Define basis vectors (for example purposes, we choose b1=(3,1) and b2=(1,2.5))
%	\coordinate (b1) at (3,1);
%	\coordinate (b2) at (1,2.5);
%	
%	% Draw basis vectors starting at the origin
%	\draw[->, thick, blue] (0,0) -- (b1) node[midway, below right] {$\vec{b}_1$};
%	\draw[->, thick, green!70!black] (0,0) -- (b2) node[midway, above left] {$\vec{b}_2$};
%	
%	% Choose scalar coefficients for illustration
%	\def\alphaOne{1.5}
%	\def\alphaTwo{1.0}
%	
%	% Compute scaled basis vectors: alpha1*b1 and alpha2*b2
%	\coordinate (v1) at ($\alphaOne*(b1)$);   % alpha1 * b1
%	\coordinate (v2) at ($\alphaTwo*(b2)$);     % alpha2 * b2
%	
%	% Draw the scaled vectors with dashed style to indicate the components
%	\draw[->, dashed, blue] (0,0) -- (v1) node[midway, below] {$\alpha_1\vec{b}_1$};
%	\draw[->, dashed, green!70!black] (0,0) -- (v2) node[midway, left] {$\alpha_2\vec{b}_2$};
%	
%	% The sum vector v = alpha1*b1 + alpha2*b2
%	\coordinate (v) at ($ (v1) + (v2) $);
%	\draw[->, ultra thick, red] (0,0) -- (v) node[midway, above right] {$\vec{v}$};
%	
%	% Illustrate the vector addition: draw from v1 to v to indicate adding the second component
%	\draw[dotted] (v1) -- (v);
%	
%	% Optionally, add an annotation explaining the decomposition
%	\node[align=left, right] at (7,4) {Every vector \(\vec{v} \in V\)\\ has a unique representation\\ \(\vec{v}=\alpha_1\vec{b}_1+\alpha_2\vec{b}_2\).};
%	
%\end{tikzpicture}
\newpage
\probox[Uniqueness of Representation with respect to a Basis]{\begin{proposition*}\hypertarget{unique-wrt-basis}{
	Let \( V \) be a vector space over a field \( F \) and let $\dim V=n<\infty$. Let
	\[
	\basis=\set{\vec{b}_1, \vec{b}_2, \dots, \vec{b}_n }\subseteq V
	\]
	be a basis of \( V \). Then for every vector \( \vec{v} \in V \) there exists a unique scalars \(\alpha_1, \alpha_2, \dots, \alpha_n \in F\) such that
	\[
	\vec{v} = \alpha_1\vec{b}_1 + \alpha_2\vec{b}_2+\cdots+\alpha_n\vec{b}_n = \sum_{i=1}^{n} \alpha_i\, \vec{b}_i.
	\]}
\end{proposition*}}

\begin{proof}
	Suppose, for contradiction, that there exist two distinct representations of some vector \( \vec{v} \in V \) in terms of the basis \( \basis=\set{\vec{b}_1,\vec{b}_2,\dots,\vec{b}_n} \):
	\[
	\vec{v} = \sum_{i=1}^{n} \alpha_i\, \vec{b}_i \quad \text{and} \quad \vec{v} = \sum_{j=1}^{n} \beta_j\, \vec{b}_j,
	\]
	where \(\alpha_i, \beta_j \in F\) for all \( i, j \). Then \begin{align*}
		\sum_{i=1}^{n} \alpha_i\, \vec{b}_i - \sum_{j=1}^{n} \beta_j\, \vec{b}_j = \vec{0}\implies
		\sum_{i=1}^{n} (\alpha_i-\beta_i)\, \vec{b}_i= \vec{0}.
	\end{align*} Since a basis $\basis=\set{\vec{b}_1,\vec{b}_2,\dots,\vec{b}_n}$ is linearly independent, we have \[
	\alpha_i-\beta_i=0,\quad\ie,\quad \alpha_i=\beta_i
	\] for all $i=1,2,\dots,n$. 
	Therefore, the representation of any \( \vec{v} \in V \) as a finite linear combination of elements of the basis \( \basis \) is unique.
\end{proof}
\vfill
\defbox[Coordinate in a Finite-Dimensional Vector Space]{\begin{definition*}
Let \(V\) be a vector space over a field \(F\) with $\dim V=n<\infty$, and let  
\[
\basis = \{\vec{b}_i\}_{i=1}^n=\set{\vec{b}_1,\vec{b}_2,\dots,\vec{b}_n}
\]
be a basis of \(V\). The \hl{\textbf{coordinate of $\vec{v}\in V$ with respect to $\basis$}}, denoted by \([\vec{v}]_\basis\), is the \(n\)-tuple
\[
[\vec{v}]_\basis = (\alpha_1,\alpha_2,\dots,\alpha_n) \quad \text{where } \vec{v} = \alpha_1 \vec{b}_1 + \alpha_2 \vec{b}_2 + \cdots + \alpha_n \vec{b}_n.
\]
\end{definition*}}
{\color{gray!30}\begin{remark*}[Coordinate Function]
%In the general case (possibly infinite-dimensional), the coordinates are given by the unique function \( [\vec{v}]_\basis \) assigning to each basis vector \(\vec{b}\) the scalar coefficient in the unique representation of \(\vec{v}\).
Let \( V \) be a vector space over a field \( F \) and let $\basis = \{ \vec{b}_i \}_{i \in I}$
be a (Hamel) basis for \( V \). Then for every vector \( \vec{v} \in V \), there exists a unique function
\[
[\vec{v}]_\basis : \basis \to F
\]
with the finite set \( \{\, \vec{b} \in \basis : [\vec{v}]_\basis(\vec{b}) \neq 0 \,\} \) such that
\[
\vec{v} = \sum_{\vec{b} \in \basis} [\vec{v}]_\basis(\vec{b})\ \vec{b}.
\]
The function \([\vec{v}]_\basis\) is called the \textit{coordinates of \( \vec{v} \) with respect to the basis \( \basis \)}. In the finite-dimensional case where $\basis = \{\vec{b}_1, \vec{b}_2, \dots, \vec{b}_n\},$
the coordinate function \([\vec{v}]_\basis\) is naturally identified with the \( n \)-tuple
\[
[\vec{v}]_\basis = (\alpha_1, \alpha_2, \dots, \alpha_n) \quad \text{where} \quad \vec{v} = \alpha_1 \vec{b}_1 + \alpha_2 \vec{b}_2 + \cdots + \alpha_n \vec{b}_n.
\] 
Furthermore, the mapping \[
\Phi:V\to F^\basis,\quad \vec{v}\mapsto[\vec{v}]_\basis.
\] is a vector space isomorphism, which assigns to each $\vec{v}\in V$ its coordinate vector w.r.t. the basis $\basis$.
\end{remark*}}
\vfill
\defbox[Linear Transformation]{\begin{definition*}
	Let $V$ and $W$ be vector spaces over a field $F$. A function \[
	T:V\to W
	\] is called a \textbf{linear transformation} if for all vectors $\vec{v},\vec{w}\in V$ and for all scalars $\alpha,\beta\in F$, the following condition holds: \[
	T(\alpha\vec{v}+\beta\vec{w})=\alpha T(\vec{v})+\beta T(\vec{v}).
	\]
\end{definition*}}
\begin{remark*}
	Equivalently, a function $T:V\to W$ is linear if it satisfies \begin{enumerate}[(i)]
		\item (\textit{Additivity}) For all $\vec{v},\vec{w}\in V$, \[
		T(\vec{v}+\vec{w})=T(\vec{v})+T(\vec{w});
		\]
		\item (\textit{Homogeneity}) For all $\alpha\in F$ and $\vec{v}\in V$, \[
		T(\alpha\vec{v})=\alpha T(\vec{v}).
		\]
	\end{enumerate} 
\end{remark*}
\begin{remark*}
	This definition ensures $T$ preserves the vector space structure of $V$ in its image in $W$.
\end{remark*}

\newpage
\defbox[Vector Space Isomorphism]{\begin{definition*}
	Let $V$ and $W$ be vector spaces over a field $F$. A mapping \[
	T:V\to W
	\] is called a \textbf{vector space isomorphism} if the followings are satisfied: \begin{enumerate}[(i)]
		\item (\textit{Linearity}) For any vectors $\vec{v},\vec{w}\in V$ and any scalars $\alpha,\beta\in F$, \[
		T(\alpha\vec{v}+\beta\vec{w})=\alpha T(\vec{v})+\beta T(\vec{w}).
		\]
		\item (\textit{Bijectivity})
		\begin{itemize}
			\item (\textit{Injectivity}) $\forall \vec{v},\vec{w}\in V$,\ $T(\vec{v})=T(\vec{w})\implies\vec{v}=\vec{w}$;
			\item (\textit{Surjectivity}) $\forall \vec{w}\in W$,\ $\exists \vec{v}\in V$ such that $T(\vec{v})=\vec{w}$.
		\end{itemize}
	\end{enumerate} The bijectivity of $T$ guarantees the existence of an inverse mapping $T^{-1}:W\to V$, which satisfies \[
	(\forall \vec{v}\in V,\ T^{-1}(T(\vec{v}))=\vec{v}),\quad\text{and}\quad (\forall \vec{w}\in W,\ T(T^{-1}(\vec{w}))=\vec{w}).
	\]
\end{definition*}}
\begin{remark*}
	The inverse mapping $T^{-1}:W\to V$ is also a linear transformation.
	\begin{proof}
		Let $\vec{w}_1,\vec{w}_2\in W$ and let $\alpha,\beta\in F$. Since $T$ is bijective, for each $\vec{w}\in W$, there exists a unique $\vec{v}\in V$ such that $\vec{w}=T(\vec{v})$. Define \[
		\vec{v}_1=T^{-1}(\vec{w}_1)\in V\quad\text{and}\quad\vec{v}_2=T^{-1}(\vec{w}_2)\in V.
		\] Since $T$ is linear, we have \[
		T(\alpha\vec{v}_1+\beta\vec{v}_2)=\alpha T(\vec{v}_1)+\beta T(\vec{v}_2)=\alpha\vec{w}_1+\beta\vec{w}_2.
		\] Thus, \begin{align*}
			T^{-1}(\alpha\vec{w}_1+\beta\vec{w}_2)&=T^{-1}(T(\alpha\vec{v}_1+\beta\vec{v}_2))\\
			&=\alpha\vec{v}_1+\beta\vec{v}_2\\
			&=\alpha T^{-1}(\vec{w}_1) + \beta T^{-1}(\vec{w}_2). 
		\end{align*}
	\end{proof}
\end{remark*}
\begin{remark*}
	When a vector space isomorphism $T:V\to W$ exists, the vector spaces $V$ and $W$ are said to be \textbf{isomorphic}, denoted by $V\simeq W$.
\end{remark*}

\newpage
\probox{\begin{proposition*}\hypertarget{prop}{
	Let $V$ and $W$ be vector spaces over a field $F$ with $\dim V<\infty$ and $\dim W<\infty$. The following are equivalent: \begin{enumerate}[(1)]
		\item $\dim V=\dim W$
		\item There exists a vector space isomorphism $T$ from $V$ to $W$
	\end{enumerate}}
\end{proposition*}}
\begin{proof}
\begin{enumerate}[]
	\item $((2)\Rightarrow (1))$ Assume that there exists a vector space isomorphism $T:V\to W$. Let $\basis_V=\set{\vec{v}_1,\vec{v}_2,\dots,\vec{v}_n}$ be any basis of $V$. Consider the set \[
	\img{\basis_V}=T[\basis_V]=\set{T(\vec{v}):\vec{v}\in \basis_V}=\set{T(\vec{v}_1),T(\vec{v}_2),\dots,T(\vec{v}_n)}\subseteq W.
	\] We claim that $T[\basis_V]$ is a basis of $W$: \begin{itemize}
		\item (\textit{Linear Independence}) Suppose that for some finite scalars $\set{\alpha_i}_{i=1}^n\subseteq F$ we have \[
		\alpha_1T(\vec{v}_1)+\alpha_2T(\vec{v}_2)+\cdots+\alpha_nT(\vec{v}_n)=\vec{0}_W.
		\] By the linearity of $T$, we obtain $T(\alpha_1\vec{v}_1+\alpha_2\vec{v}_2+\cdots+\alpha_n\vec{v}_n)=\vec{0}_W$. Note that $T(\vec{0}_V)=T(0\cdot\vec{v})=0\cdot T(\vec{v})=\vec{0}_W$ for any $\vec{v}\in V$. Since $T$ is injective, it follows that \[
		\alpha_1\vec{v}_1+\alpha_2\vec{v}_2+\cdots+\alpha_n\vec{v}_n=\vec{0}_V.
		\] As $\basis_V=\vec{v_1,\vec{v}_2,\dots,\vec{v}_n}$ is a basis (and hence linearly independent), $\alpha_1=\alpha_2=\cdots=\alpha_n=0$. Thus, $T[\basis_V]$ is linearly independent.
		\item (\textit{Spanning Property}) Let $\vec{w}\in W$. Since $T$ is surjective, there exists $\vec{v}\in V$ such that \[
		T(\vec{v})=\vec{w}.
		\] By \hyperlink{unique-wrt-basis}{Uniqueness of Representation w.r.t. a Basis}, we know that there exists a unique scalars $\set{\alpha}_{i=1}^n\subseteq F$ such that \[
		\vec{v}=\alpha_1\vec{v}_1+\alpha_2\vec{v}_2+\cdots\alpha_n\vec{v}_n.
		\] Then \[
		\vec{w}=T(\vec{v})=T(\alpha_1\vec{v}_1+\alpha_2\vec{v}_2+\cdots\alpha_n\vec{v}_n)\overset{\text{linearity}}{=}\alpha_1 T(\vec{v}_1)+\alpha_2 T(\vec{v}_2) + \cdots +\alpha_n T(\vec{v}_n)\in \Span T[\basis_V].
		\] That is, $\vec{w}\in W$ is a linear combination of elements of $T[\basis_V]$. Therefore, $\Span T[\basis_V]=W$.
	\end{itemize} Since $
	\abs[1]{\basis_V}=\abs[1]{T[\basis_V]}=n$, thus, we have \[
	\dim V=\dim W.
	\]
	\item $((1)\Rightarrow (2))$  Conversely, assume that $\dim V=\dim W =:n.$ Consider bases \[
	\basis_V=\set{\vec{v}_1,\vec{v}_2,\dots,\vec{v}_n}\quad\text{and}\quad\basis_W=\set{\vec{w}_1,\vec{w}_2,\dots,\vec{w}_n}
	\] for $V$ and $W$, respectively. By \hyperlink{unique-wrt-basis}{Uniqueness of Representation w.r.t. a Basis}, for each vector $\vec{v}\in V$, there exists a unique finite scalars $\set{\alpha_i}_{i=1}^n\subseteq F$ such that \[
	\vec{v}=\alpha_1\vec{v}_1+\alpha_2\vec{v}_2+\cdots+\alpha_n\vec{v}_n.
	\] Define a mapping $T:V\to W$ given by \[
	T(\vec{v})=T\left(\sum_{i=1}^n\alpha_i\vec{v}_i\right):=\sum_{j=1}^n\alpha_j\vec{w}_j.
	\] We NTS that $T$ be a one-to-one and onto linear transformation: \begin{enumerate}[(i)]
		\item (\textit{Linearity}) Let $\vec{v},\vec{v}'\in V$ with $\vec{v}=\sum_{i=1}^n\alpha_i\vec{v}_i$ and $\vec{v}'=\sum_{j=1}^n\beta_j\vec{v}_j$. For any $\lambda,\mu\in F$, we have \begin{align*}
		\lambda\vec{v}+\mu\vec{v}'=\lambda\sum_{i=1}^{n}\alpha_i\vec{v}_i+\mu\sum_{j=1}^{n}\beta_j\vec{v}_j
		&=\lambda(\alpha_1\vec{v}_1+\alpha_2\vec{v}_2+\cdots+\alpha_n\vec{v}_n)+\mu(\beta_1\vec{v}_1+\beta_2\vec{v}_2+\cdots+\beta_n\vec{v}_n)\\
		&=(\lambda\alpha_1+\mu\beta_1)\vec{v}_1+(\lambda\alpha_2+\mu\beta_2)\vec{v}_2+\cdots+(\lambda\alpha_n+\mu\beta_n)\vec{v}_n\\
		&=\sum_{i=1}^{n}\gamma_k\vec{v}_k\quad\text{where}\quad\gamma_k=
		\lambda\alpha_k+\mu\beta_k.
		\end{align*} By definition of $T$, we have \[
		T(\lambda\vec{v}+\mu\vec{v}')=\sum_{i=1}^{n}\gamma_k\vec{w}_k=\lambda\sum_{i=1}^{n}\alpha_i\vec{w}_i+\mu\sum_{j=1}^{n}\beta_j\vec{w}_j=\lambda T(\vec{v})+\mu T(\vec{v}').
		\]\vspace{20pt}
		\item (\textit{Injectivity}) Let $\vec{v},\vec{v}'\in V$ with $\vec{v}=\sum_{i=1}^n\alpha_i\vec{v}_i$ and $\vec{v}'=\sum_{j=1}^n\beta_j\vec{v}_j$. Suppose $T(\vec{v})=T(\vec{v}')$. Then \[
		T(\vec{v})-T(\vec{v}')=\sum_{i=1}^{n}\gamma_k\vec{w}_k=\vec{0}_W,\quad\text{where}\ \gamma_k=\alpha_k-\beta_k.
		\] Since $\basis_W=\set{\vec{w}_1,\vec{w}_2,\dots,\vec{w}_n}$ is a basis of $W$, the linear independence of $\basis_W$ implies that\[
		\alpha_k=\beta_k
		\] for all $k=1,2,\dots,n$. Thus $\vec{v}=\vec{v}'$, and so $T$ is injective. 
		\newpage
		\item (\textit{Surjectivity}) Let $\vec{w}\in W$. Since $\basis_W=\set{\vec{w}_1,\vec{w}_2,\dots,\vec{w}_n}$ is a basis of $W$, there exists a unique finite scalars $\set{\alpha_i}_{i=1}^n\subseteq F$ such that \[
		\vec{w}=\alpha_1\vec{w}_1+\alpha_2\vec{w}_2+\cdots+\alpha_n\vec{w}_n.
		\] Define a vector \[
		\vec{v}:=\alpha_1\vec{v}_1+\alpha_2\vec{v}_2+\cdots+\alpha_n\vec{v}_n=\sum_{i=1}^n\alpha_i\vec{v}_i\in V.
		\] Then $T(\vec{v})=\sum_{i=1}^n\alpha_i\vec{w}_i=\vec{w}$. Thus, $T$ is surjective.
	\end{enumerate}
\end{enumerate}
\end{proof}
\thmbox[Classification of Vector Spaces up to Isomorphism]{\begin{theorem*}
	Let \[
	\mathcal{V}_F:=\set{V:\text{$V$ is a vector space over a field $F$}}.
	\] Define a relation $\sim$ on $\mathcal{V}_F$ by \[
	\forall V,W\in\mathcal{V}_F,\quad V\sim W\iff \exists T\in W^V\quad\text{such that}\quad\text{$T$ is a vector space isomorphism}.
	\] Then \begin{enumerate}[(1)]
		\item $\sim$ is an equivalence relation on $\mathcal{V}_F$;
		\item For any vector spaces $V,W\in\mathcal{V}_F$, $
		V\simeq W\iff \dim V=\dim W.$
	\end{enumerate}
	The isomorphism classes of vector spaces over $F$ are completely determined by their dimensions.
\end{theorem*}}
\begin{proof}
\ \begin{enumerate}[(1)]
	\item We NTS that the relation $\sim$ is reflexive, symmetric, and transitive: \begin{enumerate}[(i)]
		\item (\textit{Reflexivity})\ For each $V\in\mathcal{V}_F$, the identity map $\id_V:V\to V$ is a linear isomorphism, so $V\sim V$.
		\item (\textit{Symmetry})\ If $V\sim W$ via an isomorphism $T:V\to W$, then its inverse $T^{-1}:W\to V$ is also linear, implying $W\sim V$.
		\item (\textit{Transitivity})\ If $V\sim W$ via $T:V\to W$ and $W\sim U$ via $S:W\to U$, then the composition $S\circ T:V\to U$ is a linear isomorphism, so $V\sim U$.
	\end{enumerate}
	\item See \hyperlink{prop}{previous proposition}.
\end{enumerate}
\end{proof}
\corbox{\begin{corollary*}
	Let $V$ be a vector space over a field $F$ with $\dim V=n\in\N$, and let \[
	F^n=\set{(x_1,x_2,\dots,x_n):x_i\in F,\ 1\leq i\leq n}
	\] is the space of $n$-tuples over $F$ equipped with the usual operations of vector addition and scalar multiplication. . Then there exists a vector space isomorphism \[
	\Phi:V\to F^n,\quad \ie,\quad V\simeq F^n.
	\]
\end{corollary*}}
\begin{example*}
	Consider the vector space \[
	\text{Mat}_{n\times m}(\R)=\set{\begin{bmatrix}
			a_{11} & a_{12} & \cdots & a_{1m} \\
			a_{21} & a_{22} & \cdots & a_{2m} \\
			\vdots & \vdots & \ddots & \vdots \\
			a_{n1} & a_{n2} & \cdots & a_{nm}
		\end{bmatrix}: a_{ij}\in\R,\; 1\leq i\leq n,\; 1\leq j\leq m}
	\] which consists of all \( n \times m \) matrices with entries in \(\mathbb{R}\) (and where the vector space structure is defined over the field \(\mathbb{R}\)). Also, let \[
	\R^{nm}=\set{(x_1,x_2,\dots,x_{nm}):x_k\in \R,\; 1\leq k\leq nm}
	\] the vector space of \( nm \)-tuples of real numbers, with the usual coordinate-wise addition and scalar multiplication (again, over the field \(\mathbb{R}\)). Then there exists a vector space isomorphism \[
	\Phi : \operatorname{Mat}_{n \times m}(\mathbb{R}) \to \mathbb{R}^{nm},
	\] \ie, $\text{Mat}_{n\times m}(\R)\simeq \R^{nm}$.
%**Proof.**
%
%1. **Definition of the Mapping \(\Phi\).**
%
%For each matrix
%\[
%A = (a_{ij})_{1\le i\le n,\;1\le j\le m} \in \operatorname{Mat}_{n \times m}(\mathbb{R}),
%\]
%define the mapping
%\[
%\Phi(A) \coloneqq (a_{11},\, a_{12},\, \dots,\, a_{1m},\, a_{21},\, a_{22},\, \dots,\, a_{2m},\, \dots,\, a_{n1},\, a_{n2},\, \dots,\, a_{nm}) \in \mathbb{R}^{nm}.
%\]
%Here, the entries of \(A\) are arranged in a fixed order (e.g., row-major order).
%
%2. **Well-Definedness.**
%
%Since every matrix \(A \in \operatorname{Mat}_{n \times m}(\mathbb{R})\) has uniquely determined entries \(a_{ij} \in \mathbb{R}\), the ordered \(nm\)-tuple \(\Phi(A)\) is uniquely determined. Thus, \(\Phi\) is well-defined as a function from \(\operatorname{Mat}_{n \times m}(\mathbb{R})\) to \(\mathbb{R}^{nm}\).
%
%3. **Linearity of \(\Phi\).**
%
%We verify that \(\Phi\) preserves addition and scalar multiplication over \(\mathbb{R}\).
%
%- **Additivity:**  
%Let \(A = (a_{ij})\) and \(B = (b_{ij})\) be arbitrary matrices in \(\operatorname{Mat}_{n \times m}(\mathbb{R})\). Then
%\[
%A + B = (a_{ij} + b_{ij}),
%\]
%so
%\[
%\Phi(A+B) = (a_{11}+b_{11},\, a_{12}+b_{12},\, \dots,\, a_{nm}+b_{nm}).
%\]
%On the other hand,
%\[
%\Phi(A) + \Phi(B) = (a_{11},\, a_{12},\, \dots,\, a_{nm}) + (b_{11},\, b_{12},\, \dots,\, b_{nm}) = (a_{11}+b_{11},\, a_{12}+b_{12},\, \dots,\, a_{nm}+b_{nm}).
%\]
%Therefore,
%\[
%\Phi(A+B) = \Phi(A) + \Phi(B).
%\]
%
%- **Homogeneity:**  
%Let \(\lambda \in \mathbb{R}\) and \(A = (a_{ij})\). Then
%\[
%\lambda A = (\lambda a_{ij}),
%\]
%and thus
%\[
%\Phi(\lambda A) = (\lambda a_{11},\, \lambda a_{12},\, \dots,\, \lambda a_{nm}) = \lambda (a_{11},\, a_{12},\, \dots,\, a_{nm}) = \lambda\, \Phi(A).
%\]
%
%Consequently, \(\Phi\) is \(\mathbb{R}\)-linear:
%\[
%\forall\, A,B \in \operatorname{Mat}_{n \times m}(\mathbb{R}),\; \forall\, \lambda \in \mathbb{R},\quad \Phi(A+B) = \Phi(A) + \Phi(B) \quad \text{and} \quad \Phi(\lambda A) = \lambda\, \Phi(A).
%\]
%
%4. **Injectivity of \(\Phi\).**
%
%Assume \(A, B \in \operatorname{Mat}_{n \times m}(\mathbb{R})\) such that
%\[
%\Phi(A) = \Phi(B).
%\]
%Then,
%\[
%(a_{11},\, a_{12},\, \dots,\, a_{nm}) = (b_{11},\, b_{12},\, \dots,\, b_{nm}).
%\]
%This equality implies that
%\[
%a_{ij} = b_{ij} \quad \text{for all } 1 \le i \le n \text{ and } 1 \le j \le m.
%\]
%Hence, \(A = B\), which shows that \(\Phi\) is injective.
%
%5. **Surjectivity of \(\Phi\).**
%
%Let
%\[
%x = (x_1,x_2,\dots,x_{nm}) \in \mathbb{R}^{nm}
%\]
%be arbitrary. We need to find a matrix \(A \in \operatorname{Mat}_{n \times m}(\mathbb{R})\) such that \(\Phi(A) = x\). Define the matrix \(A = (a_{ij})\) by
%\[
%a_{ij} \coloneqq x_{(i-1)m + j}, \quad \text{for } 1 \le i \le n \text{ and } 1 \le j \le m.
%\]
%With this definition, the entries of \(A\) arranged in row-major order are exactly the coordinates of \(x\); that is,
%\[
%\Phi(A) = (x_1,x_2,\dots,x_{nm}) = x.
%\]
%Therefore, \(\Phi\) is surjective.
%
%6. **Conclusion.**
%
%Since \(\Phi : \operatorname{Mat}_{n \times m}(\mathbb{R}) \to \mathbb{R}^{nm}\) is a linear transformation that is both injective and surjective, it is an isomorphism of \(\mathbb{R}\)-vector spaces. Thus,
%\[
%\operatorname{Mat}_{n \times m}(\mathbb{R}) \cong \mathbb{R}^{nm}.
%\]
%
%\(\Box\)

\end{example*}
\vspace{20pt}
\begin{note}
We also denote the set of all $n\times m$ matrices with real entries, namely $\text{Mat}_{n\times m}(\R)$ by $\R^{n\times m}$.
\end{note}

\newpage
\defbox[Matrix Representation of a Linear Transformation]{\begin{definition*}
	TBA
\end{definition*}}		

\iffalse
\section*{Matrix Representation}
\begin{itemize}
	\item Let $V,W$ be vector space $V,S/F$ with $\dim V=n$ and $\dim W=m$.
	\item Let $T:V\to W$ be a linear transformation
	\item Fix a basis \[
	\basis_V=\set{\vec{v}_1,\vec{v}_2,\dots,\vec{v}_n}\subseteq V\quad\text{and}\quad\basis_W=\set{\vec{w}_1,\vec{w}_2,\dots,\vec{w}_m}\subseteq W.
	\]
	\item Then \begin{align*}
		T(\vec{v}_1)&=a_{11}\vec{w}_1+a_{21}\vec{w}_2+\cdots+a_{m1}\vec{w}_m\\
		T(\vec{v}_2)&=a_{12}\vec{w}_1+a_{22}\vec{w}_2+\cdots+a_{m2}\vec{w}_m\\
		\vdots &\\
		T(\vec{v}_k)&=a_{1k}\vec{w}_1+a_{2k}\vec{w}_2+\cdots+a_{mk}\vec{w}_m.
	\end{align*}
	\item \[
	\begin{bmatrix}
		\mid & & \mid \\
		T(\vec{v}_1)& \cdots & T(\vec{v}_n)\\
		\mid & & \mid
	\end{bmatrix}=\begin{bmatrix}
	a_{11} & a_{21} & \cdots & a_{1n}\\
	a_{21} & a_{22} & \cdots & a_{2n}\\
	\vdots & & &\vdots  \\
	a_{m1} & a_{m1} & \cdots & a_{mn}\\
\end{bmatrix}=:[T]_{\basis_V}^{\basis_W}
	\] is called the matrix representation of $\basis_W$ w.r.t. $\basis_V$.
	\item e.g. Given $n$-dim vector space $V$ over a field $F$. Let $\basis_V=\set{\vec{v}_1,\vec{v}_2,\dots,\vec{v}_n}$ be a basis of $V$. Define \[
	\fullfunction{\Phi}{V}{F^n}{\sum_{i=1}^{n}a_i\vec{v}_i}{\sum_{j=1}^{n}a_j\vec{e}_j},
	\] where $\mathcal{E}=\set{\vec{e}_1,\dots,\vec{e}_n}$ is the standard basis of $F^n$, \ie, $e_j=(0,\cdots,1,\cdots,0)$ Then \begin{align*}
	\Phi(\vec{v}_1)&=\Phi(1\cdot\vec{v}_1)=\vec{e}_1=1\vec{e}_1+0\vec{e}_2+\cdots+0\vec{e}_n\\
	\Phi(\vec{v}_2)&=\Phi(1\cdot\vec{v}_2)=\vec{e}_2=0\vec{e}_1+1\vec{e}_2+\cdots+0\vec{e}_n\\
	\vdots &\\
	\Phi(\vec{v}_n)&=\Phi(1\cdot\vec{v}_n)=\vec{e}_n=0\vec{e}_1+0\vec{e}_2+\cdots+1\vec{e}_n
	\end{align*} Then \[
\begin{bmatrix}
\Phi(\vec{v}_1)& \cdots & \Phi(\vec{v}_n)
\end{bmatrix}=\begin{bmatrix}
1 & 0 & \cdots & 0\\
0 & 1 & \cdots & 0\\
\vdots & & &\vdots  \\
0 & 0 & \cdots & 1\\
\end{bmatrix}=:Id_{n\times n}.
\]
\item e.g. \[
\fullfunction{T}{Mat_{2}(F)}{Mat_2(F)}{A}{A^T}
\] Then $T$ is a linear transformation. (since $T(A+B)=(A+B)^T=A^T+B^T=T(A)+T(B)$ and $T(kA)=(kA)^T=kA^T$). Choose a basis \[
\basis=\set{\begin{bmatrix}
	1 & 0 \\ 0 & 0
\end{bmatrix}, \begin{bmatrix}
0 & 1 \\ 0 & 0
\end{bmatrix}, \begin{bmatrix}
0 & 0 \\ 1 & 0
\end{bmatrix}, \begin{bmatrix}
0 & 0 \\ 0 & 1
\end{bmatrix}}
\] Then \[
[T]_\basis^\basis=\begin{bmatrix}
	1 & 0 & 0 & 0 \\
	0 & 0 & 1 & 0 \\
	0 & 1 & 0 & 0 \\
	0 & 0 & 0 & 1
\end{bmatrix}
\]
\item (Rmk1) The matrix representation depends on choices of basis of $V$ and $W$.
\item (Rmk2) Let $V,W$ vector spaces over a field $F$ with $\dim V=n$ and $\dim W=m$. Let $T:V\to W$.
\end{itemize}
\fi


\vfill
\begin{thebibliography}{9}
	\bibitem{linear_algebra_c}
	수학의 즐거움, Enjoying Math. ``수학 공부, 기초부터 대학원 수학까지, 16. 선형대수학 (c) 차원과 벡터공간의 분류'' YouTube Video, 29:08. Published 
	October 11, 2019. URL: \url{https://www.youtube.com/watch?v=rOKN645fRPs&t=399s}.
	\bibitem{linear_algebra_d}
	수학의 즐거움, Enjoying Math. ``수학 공부, 기초부터 대학원 수학까지, 17. 선형대수학 (d) 선형함수의 행렬 표현'' YouTube Video, 29:14. Published 
	October 12, 2019. URL: \url{https://www.youtube.com/watch?v=Fsy-9KW9-PA}.
\end{thebibliography}

%\newpage
%\appendix
%\section{Proof of Zorn's Lemma from Axiom of Choice}
%\thmbox{\begin{theorem*}\hypertarget{zorn}{}
%The following statements are equivalent: \begin{enumerate}
%	\item \textbf{Axiom of Choice (AC)}:\quad For every indexed family $\set{S_i}_{i\in I}$ of nonempty sets, there exists a choice function $f:I\to\bigcup_{i\in I}S_i$ such that $f(i)\in X_i$ for all $i\in I$.
%	\item \textbf{Zorn's Lemma}:\quad If $(P,\preceq)$ is a nonempty partially ordered set in which every chains has an upper bound in $P$, then $P$ contains at least one maximal element.
%\end{enumerate}
%\end{theorem*}}
%\begin{proof}
%\begin{itemize}
%	\item[($\Rightarrow$)] $(\textbf{AC}\implies \textbf{ZL})$ Assume that the Axiom of Choice holds.
%	\begin{enumerate}
%		\item \textbf{Definition of Poset}
%		
%		Let $(P,\preceq)$ be a nonempty partially ordered set with the property that every chain in $P$ has an upper bound in  $P$.
%		
%		\item \textbf{Construction of an Extending Function}
%		
%		Define the family $\set{\mathcal{C}_i}_{i\in I}$ of chains in $P$. For any chain $\mathcal{C}_i$ that is not maximal with respect to inclusion (\ie, $\exists $), AC guarantees that we can elect an element \[
%		f(C)
%		\] 
%	\end{enumerate}
%	\item[($\Leftarrow$)] $(\textbf{ZL}\implies \textbf{AC})$ 
%\end{itemize}
%\end{proof}
\end{document}
