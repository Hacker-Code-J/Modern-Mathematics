\documentclass[11pt,openany]{article}

\usepackage{mathtools, commath}
% Packages for formatting
\usepackage[margin=1in]{geometry}
\usepackage{fancyhdr}
\usepackage{enumerate}
\usepackage{graphicx}
\usepackage{kotex}
\usepackage{arydshln} % Include this package
\usepackage{bbding}
\usepackage{amsmath}
\usepackage{amsthm}
\usepackage[dvipsnames,table]{xcolor}
\usepackage{amssymb, amsfonts}
\usepackage{wasysym}
\usepackage{footnote}
\usepackage{tablefootnote}
\usepackage{arydshln} % Include this package
% Fonts
\usepackage[T1]{fontenc}
\usepackage[utf8]{inputenc}
\usepackage{newpxtext,newpxmath}
\usepackage{sectsty}

% Define colors
\definecolor{TealBlue1}{HTML}{0077c2}
\definecolor{TealBlue2}{HTML}{00a5e6}
\definecolor{TealBlue3}{HTML}{b3e0ff}
\definecolor{TealBlue4}{HTML}{00293c}
\definecolor{TealBlue5}{HTML}{e6f7ff}

\definecolor{thmcolor}{RGB}{231, 76, 60}
\definecolor{defcolor}{RGB}{52, 152, 219}
\definecolor{lemcolor}{RGB}{155, 89, 182}
\definecolor{corcolor}{RGB}{46, 204, 113}
\definecolor{procolor}{RGB}{241, 196, 15}

\usepackage{color,soul}
\usepackage{soul}
\newcommand{\mathcolorbox}[2]{\colorbox{#1}{$\displaystyle #2$}}
\usepackage{cancel}
\newcommand\crossout[3][black]{\renewcommand\CancelColor{\color{#1}}\cancelto{#2}{#3}}
\newcommand\ncrossout[2][black]{\renewcommand\CancelColor{\color{#1}}\cancel{#2}}

\usepackage{hyperref}
\usepackage{booktabs}

% Chapter formatting
\definecolor{titleTealBlue}{RGB}{0,53,128}
\usepackage{titlesec}
\titleformat{\section}
{\normalfont\sffamily\Large\bfseries\color{titleTealBlue!100!gray}}{\thesection}{1em}{}
\titleformat{\subsection}
{\normalfont\sffamily\large\bfseries\color{titleTealBlue!50!gray}}{\thesubsection}{1em}{}

%Tcolorbox
\usepackage[most]{tcolorbox}
\usepackage{multirow}
\usepackage{multicol}

\usepackage[linesnumbered,ruled]{algorithm2e}
\usepackage{algpseudocode}
\usepackage{setspace}
\SetKwComment{Comment}{/* }{ */}
\SetKwProg{Fn}{Function}{:}{end}
\SetKw{End}{end}
\SetKw{DownTo}{downto}

% Define a new environment for algorithms without line numbers
\newenvironment{algorithm2}[1][]{
	% Save the current state of the algorithm counter
	\newcounter{tempCounter}
	\setcounter{tempCounter}{\value{algocf}}
	% redefine the algorithm numbering (remove prefix)
	\renewcommand{\thealgocf}{}
	\begin{algorithm}
	}{
	\end{algorithm}
	% Restore the algorithm counter state
	\setcounter{algocf}{\value{tempCounter}}
}

\usepackage{adjustbox}
% Header and footer formatting
\pagestyle{fancy}
\fancyhead{}
\fancyhf{}
\rhead{\textcolor{TealBlue2}{\large\textbf{기대수(기초부터 대학원 수학까지 시리즈) 3기}}}%\rule{3cm}{0.4pt}}
\lhead{\textcolor{TealBlue2}{\large\textbf{수학의 즐거움, Enjoying Math}}}
% Define footer
%\newcommand{\footer}[1]{
%\begin{flushright}
%	\vspace{2em}
%	\includegraphics[width=2.5cm]{school_logo.jpg} \\
%	\vspace{1em}
%	\textcolor{TealBlue2}{\small\textbf{#1}}
%\end{flushright}
%}
%\rfoot{\large Department of Information Security, Cryptogrphy and Mathematics, Kookmin Uni.\includegraphics[height=1.5cm]{school_logo.jpg}}
\fancyfoot{}
\fancyfoot[C]{-\thepage-}

\usepackage{tcolorbox}
\tcbset{colback=white, arc=5pt}

\definecolor{axiomcolor}{HTML}{a88bfa}
\definecolor{defcolor}{RGB}{52, 152, 219}
\definecolor{procolor}{RGB}{241, 196, 15}
\definecolor{thmcolor}{RGB}{231, 76, 60}
\definecolor{lemcolor}{RGB}{155, 89, 182}
\definecolor{corcolor}{RGB}{46, 204, 113}
\definecolor{execolor}{RGB}{90, 128, 127}

% Define a new command for the custom tcolorbox
\newcommand{\axiombox}[2][]{%
	\begin{tcolorbox}[colframe=axiomcolor, title={\color{white}\bfseries #1}]
		#2
	\end{tcolorbox}
}

\newcommand{\defbox}[2][]{%
	\begin{tcolorbox}[colframe=defcolor, title={\color{white}\bfseries #1}]
		#2
	\end{tcolorbox}
}

\newcommand{\lembox}[2][]{%
	\begin{tcolorbox}[colframe=lemcolor, title={\color{white}\bfseries #1}]
		#2
	\end{tcolorbox}
}

\newcommand{\probox}[2][]{%
	\begin{tcolorbox}[colframe=procolor, title={\color{white}\bfseries #1}]
		#2
	\end{tcolorbox}
}

\newcommand{\thmbox}[2][]{%
	\begin{tcolorbox}[colframe=thmcolor, title={\color{white}\bfseries #1}]
		#2
	\end{tcolorbox}
}

\newcommand{\corbox}[2][]{%
	\begin{tcolorbox}[colframe=corcolor, title={\color{white}\bfseries #1}]
		#2
	\end{tcolorbox}
}



\usepackage{amsthm}

% Define custom theorem styles
\newtheoremstyle{dotless} % Name of the style
{3pt} % Space above
{3pt} % Space below
{\itshape} % Body font
{} % Indent amount
{\bfseries} % Theorem head font
{} % Punctuation after theorem head
{2.5mm} % Space after theorem head
{} % Theorem head spec

\newtheoremstyle{definitionstyle} % Name of the style
{3pt} % Space above
{3pt} % Space below
{} % Body font
{} % Indent amount
{\bfseries} % Theorem head font
{.} % Punctuation after theorem head
{2.5mm} % Space after theorem head
{} % Theorem head spec

% Applying custom styles
\theoremstyle{dotless}
\newtheorem{theorem}{Theorem} % Theorem environment with section-wise numbering
\newtheorem{proposition}[theorem]{Proposition} % Theorem environment with section-wise numbering
\newtheorem{lemma}[theorem]{Lemma} % Lemma shares the counter with theorem
\newtheorem{corollary}[theorem]{Corollary} % Corollary shares the counter with theorem

\theoremstyle{definitionstyle}
\newtheorem*{observation}{\textcolor{Magenta}{Observation}}
\newtheorem{definition}{Definition} % Definition shares the counter with theorem
\newtheorem{example}{Example} % Example shares the counter with theorem
\newtheorem{exercise}{Exercise} % Example shares the counter with theorem
\newtheorem{remark}{Remark} % Remark shares the counter with theorem
\newtheorem*{note}{Note}

\newtheorem*{definition*}{Definition} % Definition shares the counter with theorem
\newtheorem*{example*}{Example} % Example shares the counter with theorem
\newtheorem*{exercise*}{\textcolor{violet}{Exercise}} % Example shares the counter with theorem
\newtheorem*{remark*}{Remark} % Remark shares the counter with theorem


\usepackage{tikz}
\usepackage{tikz-cd}
\usepackage{tikz-3dplot}
\usepackage{pgfplots}
\pgfplotsset{compat=newest} % Adjust to your version of pgfplots
\def\Circlearrowleft{\ensuremath{%
		\rotatebox[origin=c]{180}{$\circlearrowleft$}}}
\def\Circlearrowright{\ensuremath{%
		\rotatebox[origin=c]{180}{$\circlearrowright$}}}
\def\CircleArrowleft{\ensuremath{%
		\reflectbox{\rotatebox[origin=c]{180}{$\circlearrowleft$}}}}
\def\CircleArrowright{\ensuremath{%
		\reflectbox{\rotatebox[origin=c]{180}{$\circlearrowright$}}}}
\usetikzlibrary{
	3d, % For 3D drawing
	angles,
	arrows,
	arrows.meta,
	backgrounds,
	bending,
	calc,
	decorations.pathmorphing,
	decorations.pathreplacing,
	decorations.markings,
	fit,
	matrix,
	patterns,
	patterns.meta,
	positioning,
	quotes,
	shadows,
	shapes,
	shapes.geometric,
	tikzmark
}
\tikzset{
	% single mid‐path arrow
	mid arrow/.style={
		decoration={
			markings,
			mark=at position 0.5 with {\arrow{Stealth[scale=1.2]}}
		},
		postaction={decorate},
	},
	% style for field arrows
	field arrow/.style={
		-{Stealth[scale=1.0]},
		thick,
		blue!70!black,
	},
}
\newcommand{\ie}{\textnormal{i.e.}}
\newcommand{\rsa}{\mathsf{RSA}}
\newcommand{\rsacrt}{\mathsf{RSA}\textendash\mathsf{CRT}}
\newcommand{\inv}[1]{#1^{-1}}

%New Command
%\newcommand{\set}[1]{\left\{#1\right\}}
\newcommand{\N}{\mathbb{N}}
\newcommand{\Z}{\mathbb{Z}}
\newcommand{\Q}{\mathbb{Q}}
\newcommand{\R}{\mathbb{R}}
\newcommand{\cR}{\mathcal{R}}
\newcommand{\C}{\mathbb{C}}
\newcommand{\F}{\mathbb{F}}
\newcommand{\nbhd}{\mathcal{N}}
\newcommand{\Log}{\operatorname{Log}}
\newcommand{\Arg}{\operatorname{Arg}}
\newcommand{\pv}{\operatorname{P.V.}}

\newcommand{\of}[1]{\left( #1 \right)} 
%\newcommand{\abs}[1]{\left\lvert #1 \right\rvert}
%\newcommand{\norm}[1]{\left\| #1 \right\|}

\newcommand{\sol}{\textcolor{magenta}{\bf Sol}}
\newcommand{\conjugate}[1]{\overline{#1}}

\newcommand{\res}{\operatorname{res}}
\DeclareMathOperator*{\Res}{\operatorname{Res}}

%\renewcommand{\Re}{\operatorname{Re}}
%\renewcommand{\Im}{\operatorname{Im}}

\newcommand{\cyclic}[1]{\langle #1 \rangle}
\newcommand{\uniform}{\overset{\$}{\leftarrow}}
\newcommand{\xmark}{\textcolor{red}{\XSolidBrush}}
\newcommand{\vmark}{\textcolor{green!75!black}{\CheckmarkBold}}

\newcommand{\gen}[1]{\langle #1 \rangle}
\newcommand{\Gen}[1]{\left\langle #1 \right\rangle}

\newcommand{\img}[1]{\text{Img}(#1)}
\newcommand{\Img}[1]{\text{Img}\left(#1\right)}
\newcommand{\preimg}[1]{\text{Img}^{-1}(#1)}
\newcommand{\Preimg}[1]{\text{Img}^{-1}\left(#1\right)}

\newcommand{\relation}{\mathrel{\mathcal{R}}}
\newcommand{\injection}{\rightarrowtail}
\newcommand{\surjection}{\twoheadrightarrow}
\newcommand{\id}{\textnormal{id}}

\newcommand{\eqclass}[1]{\left[#1\right]}

% Define custom colors for O and X
\newcommand{\yes}{\textcolor{blue}{\bf \fullmoon}}
\newcommand{\no}{\textcolor{red}{\bf \texttimes}}

\DeclarePairedDelimiter\ceil{\lceil}{\rceil}
\DeclarePairedDelimiter\floor{\lfloor}{\rfloor}
%\renewcommand{\floor}[#1]{\lfloor #1\rfloor}
%\newcommand{\Floor}[#1]{\left\lfloor #1\right\rfloor}
%\newcommand{\ceil}[#1]{\lceil #1\rceil}
%\newcommand{\Ceil}[#1]{\left\lceil #1\right\rceil}

\newcommand{\topology}{\mathscr{T}}
\newcommand{\sequence}[1]{\langle #1\rangle}

\setstretch{1.25}
\begin{document}
\pagenumbering{arabic}
\begin{center}
	\huge\textbf{Advanced Calculus III}\\
	\vspace{0.5em}
	\large{Ji, Yong-hyeon}\\
%	\large{\ttfamily \url{https://github.com/Hacker-Code-J}}\\
	\vspace{0.5em}
	\normalsize{\today}\\
\end{center}

\noindent 
We cover the following topics in this note.
\begin{itemize}
	\item Limit of a Function
	\item TBA
\end{itemize}
\hrule\vspace{12pt}

\newpage

\defbox[Limit Point (Metric Space)]{\begin{definition*}
	Let $(X,d)$ be a metric space. Let $S\subseteq X$ and $\alpha\in X$. A point $p\in X$ is a \textbf{limit point} of $S$ if and only if 
	\[
	\forall\varepsilon>0,\ B_{\varepsilon}(\alpha)\cap (S\setminus\set{p})\neq\varnothing.
	\] That is, \[
	\forall\varepsilon>0,\ \set{x\in S:0< d(x,p)<\varepsilon}\neq\varnothing.
	\]
\end{definition*}}
\begin{remark}
	Note that $\alpha$ does not have to be an element of $A$ to be a limit point.
\end{remark}
\begin{note}
	Let $(X,\tau)$ be a topological space. For a subset $S\subseteq X$. A point $p\in X$ is a limit point of $S$ if and only if \[
	\forall U\in\tau\ \text{with}\ p\in U,\ U\cap (S\setminus\set{p})\neq\varnothing.
	\]
\end{note}
%\begin{center}
%\begin{tikzpicture}[scale=3]
%	% Draw the coordinate axes
%%	\draw[->] (-2, 0) -- (4, 0) node[right] {x};
%%	\draw[->] (0, -2) -- (0, 4) node[above] {y};
%	% Define the set of points
%	\foreach \x/\y in {0.5/1, 1.5/0.5, 2/1.5, 2.5/2.2, 1.8/2.5, 1.2/2.3, 0.8/1.8} {
%		\fill[blue] (\x, \y) circle[radius=2pt];
%	}
%	% Highlight the limit point
%	\fill[red] (2, 1) circle[radius=2.5pt] node[below right] {$\alpha$};
%	% Draw the open ball
%	\draw[green!60!black, thick, dashed] (2, 1) circle (1);
%	\node[green!60!black] at (3, 2) {$B_\varepsilon(\alpha)$};
%	% Annotate nearby points in the open ball
%	\foreach \x/\y in {1.5/0.5, 2/1.5, 2.5/2.2} {
%		\draw[->, orange, thick] (\x, \y) -- (2, 1);
%	}
%	
%	% Explanation of other points
%	\node[blue] at (0.5, 0.5) {Other points in the set};
%	\node[red] at (2.8, 0.8) {Limit point $p$};
%\end{tikzpicture}
%\end{center}
\defbox[$\star$ Limit of a Function ($\epsilon-\delta$) $\star$]{\begin{definition*}
	Let $f:X\to\R$ be a function defined on a subset $X$ of a topological space, and let $p\in X$ be a limit point of $X$. We say that $L\in\R$ is the \textbf{limit of the function $f$ as $x$ approaches $p$} if
	\[
	\boxed{\forall\varepsilon>0,\ \exists\delta>0\ \text{such that}\ \forall x\in X,\ 0<\abs[0]{x-p}<\delta\implies\abs[0]{f(x)-L}<\varepsilon}.
	\] We write \[
	\boxed{\lim\limits_{x\to p} f(x)=L}.
	\]
\end{definition*}}
\begin{remark}
\[
\lim\limits_{x\to p} f(x)\neq L\iff \exists\varepsilon>0:[\forall\delta>0:\exists x\in X: 0<\abs[0]{x-p}<\delta\ \text{but}\ \abs[0]{f(x)-L}>0].
\]
\end{remark}

\defbox[Continuity of a Function]{\begin{definition*}
	Let $f:X\to\R$ be a function defined on a subset $X$ of a topological space, and let $p\in X$. The function $f$ is said to be $f$ is \textbf{continuous at $p$} if and only if \[
	\lim\limits_{x\to p}f(x)=f(p).
	\] That is, \[
	\forall\varepsilon>0,\ \exists\delta>0\ \text{such that}\ 0<\abs{x-p}<\delta\implies|f(x)-f(p)|<\varepsilon.
	\]
\end{definition*}}
\begin{remark}[Continuity of a Set]
	The function $f$ is continuous on subset $S\subseteq X$ if it it continuous at every point $p\in S$.
\end{remark}
\begin{remark}[Continuity in a Topological Space]
	Let $(X,\tau_X)$ and $(Y,\tau_Y)$ are topological spaces. $f:X\to Y$ is \textbf{continuous} if and only if $$U_Y\in\tau_Y\implies f^{-1}[U_Y]\in\tau_X,$$ where $f^{-1}[U_Y]=\set{x\in X:f(x)\in U_Y}$ is the preimage of $U_Y$ under $f$.
\end{remark}

\begin{note}[Subsequence]
	Let $\set{a_n}$ be a sequence of real numbers, and let $n_1<n_2<\cdots<n_k<\cdots$ be a strictly increasing of natural numbers. Then $\set{a_{n_k}}$ is called \textbf{subsequence} of $\set{a_n}$.
\end{note}
\probox{\begin{proposition*}
A sequence ${a_n}$ of real numbers converges to $L\in\R$ if and only if any
subsequence $\set{a_{n_k}}$ of $\set{a_n}$ converges to $L\in\R$. Formally, \[
\lim\limits_{n\to\infty}a_n=L\iff  \lim\limits_{k\to\infty} a_{n_k}=L.
\]
\end{proposition*}}

\thmbox[Sandwitch Theorem; Squeeze Theorem]{\begin{theorem*}
	
\end{theorem*}}

\begin{note}
	TBA
\end{note}
\vfill

\begin{thebibliography}{9}
	\bibitem{advanced_calc_e}
	수학의 즐거움, Enjoying Math. ``수학 공부, 기초부터 대학원 수학까지, 10. 해석학 개론 (e) 엡실론-델타와 수열의 수렴성'' YouTube Video, 25:57. Published 
	September 29, 2019. URL: \url{https://www.youtube.com/watch?v=2Ml3G_Duffk&t=899s}.
\end{thebibliography}
%\newpage
%\appendix
%\section{Complement of Family}
%\begin{note}
%	\[
%	\left(\bigcup_{i\in\Lambda}E_i\right)^C=
%	\bigcap_{i\in\Lambda}\left(E_i\right)^C
%	\]
%	\begin{proof}
%		content...
%	\end{proof}
%\end{note}
\end{document}
