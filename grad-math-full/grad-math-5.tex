\documentclass[11pt,openany]{article}

\input{grad-math-preamble}
\usepackage{tcolorbox}
\tcbset{colback=white, arc=5pt}

\definecolor{axiomcolor}{HTML}{a88bfa}
\definecolor{defcolor}{RGB}{52, 152, 219}
\definecolor{procolor}{RGB}{241, 196, 15}
\definecolor{thmcolor}{RGB}{231, 76, 60}
\definecolor{lemcolor}{RGB}{155, 89, 182}
\definecolor{corcolor}{RGB}{46, 204, 113}
\definecolor{execolor}{RGB}{90, 128, 127}

% Define a new command for the custom tcolorbox
\newcommand{\axiombox}[2][]{%
	\begin{tcolorbox}[colframe=axiomcolor, title={\color{white}\bfseries #1}]
		#2
	\end{tcolorbox}
}

\newcommand{\defbox}[2][]{%
	\begin{tcolorbox}[colframe=defcolor, title={\color{white}\bfseries #1}]
		#2
	\end{tcolorbox}
}

\newcommand{\probox}[2][]{%
	\begin{tcolorbox}[colframe=procolor, title={\color{white}\bfseries #1}]
		#2
	\end{tcolorbox}
}

\newcommand{\thmbox}[2][]{%
	\begin{tcolorbox}[colframe=thmcolor, title={\color{white}\bfseries #1}]
		#2
	\end{tcolorbox}
}

\newcommand{\lembox}[2][]{%
	\begin{tcolorbox}[colframe=lemcolor, title={\color{white}\bfseries #1}]
		#2
	\end{tcolorbox}
}
\usepackage{amsthm}

% Define custom theorem styles
\newtheoremstyle{dotless} % Name of the style
{3pt} % Space above
{3pt} % Space below
{\itshape} % Body font
{} % Indent amount
{\bfseries} % Theorem head font
{} % Punctuation after theorem head
{2.5mm} % Space after theorem head
{} % Theorem head spec

\newtheoremstyle{definitionstyle} % Name of the style
{3pt} % Space above
{3pt} % Space below
{} % Body font
{} % Indent amount
{\bfseries} % Theorem head font
{.} % Punctuation after theorem head
{2.5mm} % Space after theorem head
{} % Theorem head spec

% Applying custom styles
%\theoremstyle{dotless}
\newtheorem{theorem}{Theorem} % Theorem environment with section-wise numbering
\newtheorem*{theorem*}{Theorem} % Theorem environment with section-wise numbering
\newtheorem*{lemma*}{Lemma} % Theorem environment with section-wise numbering
\newtheorem*{proposition*}{Proposition} % Theorem environment with section-wise numbering
\newtheorem*{corollary*}{Corollary} % Theorem environment with section-wise numbering
\newtheorem{proposition}[theorem]{Proposition} % Theorem environment with section-wise numbering
\newtheorem{lemma}[theorem]{Lemma} % Lemma shares the counter with theorem
\newtheorem{corollary}[theorem]{Corollary} % Corollary shares the counter with theorem

\theoremstyle{definitionstyle}
\newtheorem*{observation}{\textcolor{magenta}{Observation}}
\newtheorem*{illustration}{\textcolor{teal}{Illustration}}
\newtheorem*{torus}{{\color{red}T}{\color{orange}o}{\color{green!75!black}r}{\color{cyan}u}{\color{violet}s}}
\newtheorem{definition}{Definition} % Definition shares the counter with theorem
\newtheorem{example}{Example} % Example shares the counter with theorem
\newtheorem{exercise}{{Exercise}} % Example shares the counter with theorem
\newtheorem{remark}{Remark} % Remark shares the counter with theorem
\newtheorem*{note}{Note}
\newtheorem*{notation}{Notation}

\newtheorem*{axiom*}{Axiom}
\newtheorem*{definition*}{Definition} % Definition shares the counter with theorem
\newtheorem*{example*}{Example} % Example shares the counter with theorem
\newtheorem*{exercise*}{\textcolor{teal}{Exercise}} % Example shares the counter with theorem
\newtheorem*{remark*}{Remark} % Remark shares the counter with theorem


\usepackage{tikz}
\usepackage{tikz-cd}
\usetikzlibrary{shadows}
\usetikzlibrary{shapes.geometric, arrows.meta, positioning}
\newcommand{\ie}{\textnormal{i.e.}}
\newcommand{\rsa}{\mathsf{RSA}}
\newcommand{\rsacrt}{\mathsf{RSA}\textendash\mathsf{CRT}}
\newcommand{\inv}[1]{#1^{-1}}

%New Command
%\newcommand{\set}[1]{\left\{#1\right\}}
\newcommand{\N}{\mathbb{N}}
\newcommand{\Z}{\mathbb{Z}}
\newcommand{\Q}{\mathbb{Q}}
\newcommand{\R}{\mathbb{R}}
\newcommand{\cR}{\mathcal{R}}
\newcommand{\C}{\mathbb{C}}
\newcommand{\F}{\mathbb{F}}
\newcommand{\nbhd}{\mathcal{N}}
\newcommand{\Log}{\operatorname{Log}}
\newcommand{\Arg}{\operatorname{Arg}}
\newcommand{\pv}{\operatorname{P.V.}}

\newcommand{\of}[1]{\left( #1 \right)} 
%\newcommand{\abs}[1]{\left\lvert #1 \right\rvert}
%\newcommand{\norm}[1]{\left\| #1 \right\|}

\newcommand{\sol}{\textcolor{magenta}{\bf Sol}}
\newcommand{\conjugate}[1]{\overline{#1}}

\newcommand{\res}{\operatorname{res}}
\DeclareMathOperator*{\Res}{\operatorname{Res}}

%\renewcommand{\Re}{\operatorname{Re}}
%\renewcommand{\Im}{\operatorname{Im}}

\newcommand{\cyclic}[1]{\langle #1 \rangle}
\newcommand{\uniform}{\overset{\$}{\leftarrow}}
\newcommand{\xmark}{\textcolor{red}{\XSolidBrush}}
\newcommand{\vmark}{\textcolor{green!75!black}{\CheckmarkBold}}

\newcommand{\gen}[1]{\langle #1 \rangle}
\newcommand{\Gen}[1]{\left\langle #1 \right\rangle}

\newcommand{\img}[1]{\text{Img}(#1)}
\newcommand{\Img}[1]{\text{Img}\left(#1\right)}
\newcommand{\preimg}[1]{\text{Img}^{-1}(#1)}
\newcommand{\Preimg}[1]{\text{Img}^{-1}\left(#1\right)}

\newcommand{\relation}{\mathrel{\mathcal{R}}}
\newcommand{\injection}{\rightarrowtail}
\newcommand{\surjection}{\twoheadrightarrow}
\newcommand{\id}{\textnormal{id}}

\newcommand{\eqclass}[1]{\left[#1\right]}

% Define custom colors for O and X
\newcommand{\yes}{\textcolor{blue}{\bf \fullmoon}}
\newcommand{\no}{\textcolor{red}{\bf \texttimes}}

\DeclarePairedDelimiter\ceil{\lceil}{\rceil}
\DeclarePairedDelimiter\floor{\lfloor}{\rfloor}
%\renewcommand{\floor}[#1]{\lfloor #1\rfloor}
%\newcommand{\Floor}[#1]{\left\lfloor #1\right\rfloor}
%\newcommand{\ceil}[#1]{\lceil #1\rceil}
%\newcommand{\Ceil}[#1]{\left\lceil #1\right\rceil}

\newcommand{\topology}{\mathscr{T}}
\newcommand{\sequence}[1]{\langle #1\rangle}

\setstretch{1.25}
\begin{document}
\pagenumbering{arabic}
\begin{center}
	\huge\textbf{Topology I}\\
	\vspace{0.5em}
	\large{Ji, Yong-hyeon}\\
	\vspace{0.5em}
	\normalsize{\today}\\
\end{center}

\noindent 
We cover the following topics in this note.
\begin{itemize}
	\item Topology
	
\end{itemize}
\hrule\vspace{12pt}
%Preliminaries:
%\begin{itemize}
%	\item Boundedness, Supremum and Infimum
%	\item Least Upper Bound Property (Completeness Axiom)
%	\item Well-Ordering Principle and Mathematical Induction
%	\item Archimedean Property
%\end{itemize}
%\hrule\vspace{12pt}
\begin{center}
\begin{minipage}{.45\textwidth}\centering
\includegraphics[scale=1.2]{../tikz/grad-math-tikz-pdf/sphere.pdf}
\end{minipage}
\begin{minipage}{.45\textwidth}\centering
\includegraphics[scale=.7]{../tikz/grad-math-tikz-pdf/torus.pdf}
\end{minipage}
\end{center}



\vfill


\defbox[Topology]{\begin{definition*}
Let $S$ be a non-empty set. A \textbf{topology} on $S$ is a subset \[
\topology=\set{E:E\subseteq S}\subseteq 2^S
\] that satisfies the open set axioms: \begin{itemize}
	\item[(O1)] $\varnothing$ and $S$ are elements of $\topology$: $\set{\varnothing, S}\subseteq\topology$.
	\item[(O2)\footnote{$\topology$ is closed under \textit{arbitrary} unions}] 	The union of an arbitrary subset of $\topology$
	is an element of $\topology$: \[
	\set{E_\alpha}_{\alpha\in\Lambda}\subseteq\topology\implies\bigcup_{\alpha\in\Lambda}E_\alpha\in\topology.
	\]
	\item[(O3)\footnote{$\topology$ is closed under \textit{finite} intersection}] The intersection of any finite subset of $\topology$
	is an element of $\topology$: \[
	\set{E_i}_{i=1}^n\subseteq\topology\implies\bigcap_{i=1}^n E_i\in\topology.
	\]
\end{itemize}
\end{definition*}}
\begin{remark*}
	By mathematical induction, we have \[
	\text{O3}\iff [\set{E_1,E_2}\subseteq\topology\Rightarrow E_1\cap E_2\in\topology].
	\]
\end{remark*}

%\begin{example}
%	Consider a set $S=\set{a,b,c,d,e}$. \begin{center}
%\begin{tikzpicture}[scale=.75]
%%	\def \x{3}
%%	\def \y{3}
%%	\draw[very thin,color=gray!15,step=.5] (-\x,-\y) grid (\x,\y);
%%	
%%	\foreach \i in {-\x,...,-2,-1,1,2,...,\x}
%%	\draw[gray] (\i,.1)--(\i,-.1) node[below] {$\i$};%x-axis
%%	\foreach \i in {-\y,-1,1,1,\y}
%%	\draw[gray] (.1,\i)--(-.1,\i) node[left] {$\i$};%y-axis
%	
%	\draw[rounded corners=8pt] (0, 3) -- (-1.5, 2.5) -- (-3, 1) -- (-2, .9) -- (-1.9, 0) -- (-2.5, -1) -- (-3, -2.5) -- (-2, -2) -- (0,-3) -- (2, -1.5) -- (3,-.5) -- (3,.5) -- (2, 1) -- (1.5, 1.5) -- (1.75, 2.5) -- cycle;
%	
%	\filldraw (0,1.5) circle (2pt) node[above] {$a$};
%	\filldraw (-1,.5) circle (2pt) node[above] {$c$};
%	\filldraw (1,0) circle (2pt) node[above] {$d$};
%	\filldraw (-.5,-2) circle (2pt) node[above] {$b$};
%	\filldraw (.5,-1) circle (2pt) node[above] {$e$};
%\end{tikzpicture}
%	\end{center}
%\begin{center}
%\begin{minipage}{.32\textwidth}
%\begin{tikzpicture}[scale=.75]
%%	\def \x{3}
%%	\def \y{3}
%%	\draw[very thin,color=gray!15,step=.5] (-\x,-\y) grid (\x,\y);
%%	
%%	\foreach \i in {-\x,...,-2,-1,1,2,...,\x}
%%	\draw[gray] (\i,.1)--(\i,-.1) node[below] {$\i$};%x-axis
%%	\foreach \i in {-\y,-1,1,1,\y}
%%	\draw[gray] (.1,\i)--(-.1,\i) node[left] {$\i$};%y-axis
%	
%	\draw[rounded corners=8pt] (0, 3) -- (-1.5, 2.5) -- (-3, 1) -- (-2, .9) -- (-1.9, 0) -- (-2.5, -1) -- (-3, -2.5) -- (-2, -2) -- (0,-3) -- (2, -1.5) -- (3,-.5) -- (3,.5) -- (2, 1) -- (1.5, 1.5) -- (1.75, 2.5) -- cycle;
%	
%	\filldraw (0,1.5) circle (2pt) node[above] {$a$};
%	\filldraw (-1,.5) circle (2pt) node[above] {$c$};
%	\filldraw (.5,0) circle (2pt) node[above] {$d$};
%	\filldraw (-.5,-1.5) circle (2pt) node[above] {$b$};
%	\filldraw (1,-1.5) circle (2pt) node[above] {$e$};
%	
%	\draw[line width=.5mm, opacity=.5, magenta] (0, 1.5) circle (10pt);
%	\draw[line width=.5mm, opacity=.5, orange, rotate=-20, xshift=-.15cm, yshift=.15cm] (0,0) ellipse (1.25cm and .5cm);
%	\draw[line width=.5mm, opacity=.5, green!50!black, yshift=.5cm] (0, 0) circle (40pt);
%	\draw[line width=.5mm, opacity=.5, cyan=-20, rotate=-20, xshift=0cm, yshift=-.5cm] (0,0) ellipse (1.75cm and 1.5cm);
%\end{tikzpicture}
%\end{minipage}
%\end{center}
%\end{example}

\begin{example}[Cofinite Topology]
	Let $S$ be a set. Define a subset $\topology_C\subseteq 2^S$ by \[
	\topology_C:=\set{T\subseteq S:T^C\subseteq S\ \text{is a finite set}}\cup\set{\varnothing}
	\] We claim that $\topology_C$ be a topology on $S$:
	\begin{enumerate}[(i)]
		\item Clearly $\subseteq\in\topology_C$. Since $S^C=\varnothing$ and $\varnothing$ is finite, $S\in\topology$.
		\item Let $\set{E_\alpha}_{\alpha\in\Lambda}\subseteq\topology_C$. Then \[
		\of{\bigcup_{\alpha\in\Lambda} E_\alpha}^C=\bigcap_{\alpha\in\Lambda}E_\alpha^C
		\] and so
		\item 
	\end{enumerate}
\end{example}

\defbox[Topological Space]{\begin{definition*}
	Let $S$ be a set. Let $\topology$ be a topology on $S$. Then the ordered pair $(S,\topology)$ is called a \textbf{topological space}.
\end{definition*}}
\defbox[Open Set]{\begin{definition*}
	Let $(S,\topology)$ be a topological space. $E\subseteq S$ is an \textbf{open set}, or \textbf{open} (in $S$) iff $E\in\topology$.
\end{definition*}}
\begin{remark*}
A subset $\topology\subseteq 2^S$ is a topology on $S$ if and only if \begin{enumerate}
	\item[(i)] $\varnothing$ and $S$ are open;
	\item[(ii)] Let $\set{E_\alpha}_{\alpha\in\Lambda}\subseteq\topology$. Then $\displaystyle\bigcup_{\alpha\in\Lambda}E_\alpha$ is open.
	\item[(iii)] Let $\set{E_i}_{i=1}^n\subseteq\topology$. Then $\displaystyle\bigcap_{i=1}^n E_i$ is open.
\end{enumerate}
\end{remark*}
\newpage

\begin{tikzpicture}[thick] 
	\node[ellipse,draw,minimum width=2cm,minimum height=4cm,label=below:$X$] (X) {};
	\node[ellipse,draw,minimum width=2cm,minimum height=4cm,right=3cm of X,label=below:$Y$]
	(Y){};
	\node[dashed,draw] (f) at (X.center) {$f^{-1}(U)$};
	\node[circle,dashed,draw,minimum size=1cm] (U) at (Y.center) {$U$};
	\draw[-latex] (X) -- (Y) node[midway,above]{$f$};
	\draw[-latex] (U) to[bend left] (f);
\end{tikzpicture}
\vfill
\begin{thebibliography}{9}
	\bibitem{advanced_topo_a}
	수학의 즐거움, Enjoying Math. ``수학 공부, 기초부터 대학원 수학까지, 8. 위상수학 (a) 위상공간의 정의.'' YouTube Video, 41:25. Published 
	September 27, 2019. URL: \url{https://www.youtube.com/watch?v=q8BtXIFzo2Q}.
%	\bibitem{advanced_calc_d}
%	수학의 즐거움, Enjoying Math. ``수학 공부, 기초부터 대학원 수학까지, 7. 해석학 개론 (d) 극한 정리'' YouTube Video, 26:46. Published 
%	September 26, 2019. URL: \url{https://www.youtube.com/watch?v=1TRD34QbIaw}.
\end{thebibliography}
%\newpage
\appendix
\section{Complement of Family}
\begin{note}
	\[
	\left(\bigcup_{i\in\Lambda}E_i\right)^C=
	\bigcap_{i\in\Lambda}\left(E_i\right)^C
	\]
	\begin{proof}
		content...
	\end{proof}
\end{note}
\end{document}
