\documentclass[11pt,openany]{article}

\usepackage{mathtools, commath}
% Packages for formatting
\usepackage[margin=1in]{geometry}
\usepackage{fancyhdr}
\usepackage{enumerate}
\usepackage{graphicx}
\usepackage{kotex}
\usepackage{arydshln} % Include this package
\usepackage{bbding}
\usepackage{amsmath}
\usepackage{amsthm}
\usepackage[dvipsnames,table]{xcolor}
\usepackage{amssymb, amsfonts}
\usepackage{wasysym}
\usepackage{footnote}
\usepackage{tablefootnote}
\usepackage{arydshln} % Include this package
% Fonts
\usepackage[T1]{fontenc}
\usepackage[utf8]{inputenc}
\usepackage{newpxtext,newpxmath}
\usepackage{sectsty}

% Define colors
\definecolor{TealBlue1}{HTML}{0077c2}
\definecolor{TealBlue2}{HTML}{00a5e6}
\definecolor{TealBlue3}{HTML}{b3e0ff}
\definecolor{TealBlue4}{HTML}{00293c}
\definecolor{TealBlue5}{HTML}{e6f7ff}

\definecolor{thmcolor}{RGB}{231, 76, 60}
\definecolor{defcolor}{RGB}{52, 152, 219}
\definecolor{lemcolor}{RGB}{155, 89, 182}
\definecolor{corcolor}{RGB}{46, 204, 113}
\definecolor{procolor}{RGB}{241, 196, 15}

\usepackage{color,soul}
\usepackage{soul}
\newcommand{\mathcolorbox}[2]{\colorbox{#1}{$\displaystyle #2$}}
\usepackage{cancel}
\newcommand\crossout[3][black]{\renewcommand\CancelColor{\color{#1}}\cancelto{#2}{#3}}
\newcommand\ncrossout[2][black]{\renewcommand\CancelColor{\color{#1}}\cancel{#2}}

\usepackage{hyperref}
\usepackage{booktabs}

% Chapter formatting
\definecolor{titleTealBlue}{RGB}{0,53,128}
\usepackage{titlesec}
\titleformat{\section}
{\normalfont\sffamily\Large\bfseries\color{titleTealBlue!100!gray}}{\thesection}{1em}{}
\titleformat{\subsection}
{\normalfont\sffamily\large\bfseries\color{titleTealBlue!50!gray}}{\thesubsection}{1em}{}

%Tcolorbox
\usepackage[most]{tcolorbox}
\usepackage{multirow}
\usepackage{multicol}

\usepackage[linesnumbered,ruled]{algorithm2e}
\usepackage{algpseudocode}
\usepackage{setspace}
\SetKwComment{Comment}{/* }{ */}
\SetKwProg{Fn}{Function}{:}{end}
\SetKw{End}{end}
\SetKw{DownTo}{downto}

% Define a new environment for algorithms without line numbers
\newenvironment{algorithm2}[1][]{
	% Save the current state of the algorithm counter
	\newcounter{tempCounter}
	\setcounter{tempCounter}{\value{algocf}}
	% redefine the algorithm numbering (remove prefix)
	\renewcommand{\thealgocf}{}
	\begin{algorithm}
	}{
	\end{algorithm}
	% Restore the algorithm counter state
	\setcounter{algocf}{\value{tempCounter}}
}

\usepackage{adjustbox}
% Header and footer formatting
\pagestyle{fancy}
\fancyhead{}
\fancyhf{}
\rhead{\textcolor{TealBlue2}{\large\textbf{기대수(기초부터 대학원 수학까지 시리즈) 3기}}}%\rule{3cm}{0.4pt}}
\lhead{\textcolor{TealBlue2}{\large\textbf{수학의 즐거움, Enjoying Math}}}
% Define footer
%\newcommand{\footer}[1]{
%\begin{flushright}
%	\vspace{2em}
%	\includegraphics[width=2.5cm]{school_logo.jpg} \\
%	\vspace{1em}
%	\textcolor{TealBlue2}{\small\textbf{#1}}
%\end{flushright}
%}
%\rfoot{\large Department of Information Security, Cryptogrphy and Mathematics, Kookmin Uni.\includegraphics[height=1.5cm]{school_logo.jpg}}
\fancyfoot{}
\fancyfoot[C]{-\thepage-}

\usepackage{tcolorbox}
\tcbset{colback=white, arc=5pt}

\definecolor{axiomcolor}{HTML}{a88bfa}
\definecolor{defcolor}{RGB}{52, 152, 219}
\definecolor{procolor}{RGB}{241, 196, 15}
\definecolor{thmcolor}{RGB}{231, 76, 60}
\definecolor{lemcolor}{RGB}{155, 89, 182}
\definecolor{corcolor}{RGB}{46, 204, 113}
\definecolor{execolor}{RGB}{90, 128, 127}

% Define a new command for the custom tcolorbox
\newcommand{\axiombox}[2][]{%
	\begin{tcolorbox}[colframe=axiomcolor, title={\color{white}\bfseries #1}]
		#2
	\end{tcolorbox}
}

\newcommand{\defbox}[2][]{%
	\begin{tcolorbox}[colframe=defcolor, title={\color{white}\bfseries #1}]
		#2
	\end{tcolorbox}
}

\newcommand{\lembox}[2][]{%
	\begin{tcolorbox}[colframe=lemcolor, title={\color{white}\bfseries #1}]
		#2
	\end{tcolorbox}
}

\newcommand{\probox}[2][]{%
	\begin{tcolorbox}[colframe=procolor, title={\color{white}\bfseries #1}]
		#2
	\end{tcolorbox}
}

\newcommand{\thmbox}[2][]{%
	\begin{tcolorbox}[colframe=thmcolor, title={\color{white}\bfseries #1}]
		#2
	\end{tcolorbox}
}

\newcommand{\corbox}[2][]{%
	\begin{tcolorbox}[colframe=corcolor, title={\color{white}\bfseries #1}]
		#2
	\end{tcolorbox}
}



\usepackage{amsthm}

% Define custom theorem styles
\newtheoremstyle{dotless} % Name of the style
{3pt} % Space above
{3pt} % Space below
{\itshape} % Body font
{} % Indent amount
{\bfseries} % Theorem head font
{} % Punctuation after theorem head
{2.5mm} % Space after theorem head
{} % Theorem head spec

\newtheoremstyle{definitionstyle} % Name of the style
{3pt} % Space above
{3pt} % Space below
{} % Body font
{} % Indent amount
{\bfseries} % Theorem head font
{.} % Punctuation after theorem head
{2.5mm} % Space after theorem head
{} % Theorem head spec

% Applying custom styles
\theoremstyle{dotless}
\newtheorem{theorem}{Theorem} % Theorem environment with section-wise numbering
\newtheorem{proposition}[theorem]{Proposition} % Theorem environment with section-wise numbering
\newtheorem{lemma}[theorem]{Lemma} % Lemma shares the counter with theorem
\newtheorem{corollary}[theorem]{Corollary} % Corollary shares the counter with theorem

\theoremstyle{definitionstyle}
\newtheorem*{observation}{\textcolor{Magenta}{Observation}}
\newtheorem{definition}{Definition} % Definition shares the counter with theorem
\newtheorem{example}{Example} % Example shares the counter with theorem
\newtheorem{exercise}{Exercise} % Example shares the counter with theorem
\newtheorem{remark}{Remark} % Remark shares the counter with theorem
\newtheorem*{note}{Note}

\newtheorem*{definition*}{Definition} % Definition shares the counter with theorem
\newtheorem*{example*}{Example} % Example shares the counter with theorem
\newtheorem*{exercise*}{\textcolor{violet}{Exercise}} % Example shares the counter with theorem
\newtheorem*{remark*}{Remark} % Remark shares the counter with theorem


\usepackage{tikz}
\usepackage{tikz-cd}
\usepackage{tikz-3dplot}
\usepackage{pgfplots}
\pgfplotsset{compat=newest} % Adjust to your version of pgfplots
\def\Circlearrowleft{\ensuremath{%
		\rotatebox[origin=c]{180}{$\circlearrowleft$}}}
\def\Circlearrowright{\ensuremath{%
		\rotatebox[origin=c]{180}{$\circlearrowright$}}}
\def\CircleArrowleft{\ensuremath{%
		\reflectbox{\rotatebox[origin=c]{180}{$\circlearrowleft$}}}}
\def\CircleArrowright{\ensuremath{%
		\reflectbox{\rotatebox[origin=c]{180}{$\circlearrowright$}}}}
\usetikzlibrary{
	3d, % For 3D drawing
	angles,
	arrows,
	arrows.meta,
	backgrounds,
	bending,
	calc,
	decorations.pathmorphing,
	decorations.pathreplacing,
	decorations.markings,
	fit,
	matrix,
	patterns,
	patterns.meta,
	positioning,
	quotes,
	shadows,
	shapes,
	shapes.geometric,
	tikzmark
}
\tikzset{
	% single mid‐path arrow
	mid arrow/.style={
		decoration={
			markings,
			mark=at position 0.5 with {\arrow{Stealth[scale=1.2]}}
		},
		postaction={decorate},
	},
	% style for field arrows
	field arrow/.style={
		-{Stealth[scale=1.0]},
		thick,
		blue!70!black,
	},
}
\newcommand{\ie}{\textnormal{i.e.}}
\newcommand{\rsa}{\mathsf{RSA}}
\newcommand{\rsacrt}{\mathsf{RSA}\textendash\mathsf{CRT}}
\newcommand{\inv}[1]{#1^{-1}}

%New Command
%\newcommand{\set}[1]{\left\{#1\right\}}
\newcommand{\N}{\mathbb{N}}
\newcommand{\Z}{\mathbb{Z}}
\newcommand{\Q}{\mathbb{Q}}
\newcommand{\R}{\mathbb{R}}
\newcommand{\cR}{\mathcal{R}}
\newcommand{\C}{\mathbb{C}}
\newcommand{\F}{\mathbb{F}}
\newcommand{\nbhd}{\mathcal{N}}
\newcommand{\Log}{\operatorname{Log}}
\newcommand{\Arg}{\operatorname{Arg}}
\newcommand{\pv}{\operatorname{P.V.}}

\newcommand{\of}[1]{\left( #1 \right)} 
%\newcommand{\abs}[1]{\left\lvert #1 \right\rvert}
%\newcommand{\norm}[1]{\left\| #1 \right\|}

\newcommand{\sol}{\textcolor{magenta}{\bf Sol}}
\newcommand{\conjugate}[1]{\overline{#1}}

\newcommand{\res}{\operatorname{res}}
\DeclareMathOperator*{\Res}{\operatorname{Res}}

%\renewcommand{\Re}{\operatorname{Re}}
%\renewcommand{\Im}{\operatorname{Im}}

\newcommand{\cyclic}[1]{\langle #1 \rangle}
\newcommand{\uniform}{\overset{\$}{\leftarrow}}
\newcommand{\xmark}{\textcolor{red}{\XSolidBrush}}
\newcommand{\vmark}{\textcolor{green!75!black}{\CheckmarkBold}}

\newcommand{\gen}[1]{\langle #1 \rangle}
\newcommand{\Gen}[1]{\left\langle #1 \right\rangle}

\newcommand{\img}[1]{\text{Img}(#1)}
\newcommand{\Img}[1]{\text{Img}\left(#1\right)}
\newcommand{\preimg}[1]{\text{Img}^{-1}(#1)}
\newcommand{\Preimg}[1]{\text{Img}^{-1}\left(#1\right)}

\newcommand{\relation}{\mathrel{\mathcal{R}}}
\newcommand{\injection}{\rightarrowtail}
\newcommand{\surjection}{\twoheadrightarrow}
\newcommand{\id}{\textnormal{id}}

\newcommand{\eqclass}[1]{\left[#1\right]}

% Define custom colors for O and X
\newcommand{\yes}{\textcolor{blue}{\bf \fullmoon}}
\newcommand{\no}{\textcolor{red}{\bf \texttimes}}

\DeclarePairedDelimiter\ceil{\lceil}{\rceil}
\DeclarePairedDelimiter\floor{\lfloor}{\rfloor}
%\renewcommand{\floor}[#1]{\lfloor #1\rfloor}
%\newcommand{\Floor}[#1]{\left\lfloor #1\right\rfloor}
%\newcommand{\ceil}[#1]{\lceil #1\rceil}
%\newcommand{\Ceil}[#1]{\left\lceil #1\right\rceil}

\newcommand{\topology}{\mathscr{T}}
\newcommand{\sequence}[1]{\langle #1\rangle}
\renewcommand{\vec}[1]{\mathbf{#1}}
\renewcommand{\Re}{\operatorname*{Re}}
\renewcommand{\Im}{\operatorname*{Im}}
\setstretch{1.25}

%\usepackage{background}
%\backgroundsetup{
%	scale=3,
%	color=gray!20,
%	opacity=0.3,
%	angle=45,
%	contents={\Huge \sffamily Ji, Yong-hyeon}
%}
\begin{document}
\pagenumbering{arabic}
\begin{center}
	\huge\textbf{Abstract Algebra II}\\
	\vspace{0.5em}
	\large{Ji, Yong-hyeon}\\
%	\large{\ttfamily \url{https://github.com/Hacker-Code-J}}\\
	\vspace{0.5em}
	\normalsize{\today}\\
\end{center}

\noindent 
We cover the following topics in this note.
\begin{itemize}
	\item Group Action
	\item Cayley Theorem
	\item Normal Subgroups
	\item TBA
\end{itemize}
\hrule\vspace{12pt}
\tableofcontents


\def\acts{\curvearrowright}
\newpage
\defbox[Group Action]{\begin{definition*}
Let \( (G,\ast) \) be a group and let \( X\neq\varnothing \). A \textbf{(left) group action} of \( G \) on \( X \) is a function \[
\cdot : G \times X \to X,\quad (g, x) \mapsto g \cdot x
\]
satisfying the followings: for all \( g, h \in G \) and all \( x \in X \),
\begin{enumerate}[(i)]
	\item (Identity)\; \( e \cdot x = x \), where \( e \in G \) is the identity element of \( G \);
	\item (Compatibility)\; \( (g\ast h) \cdot x = g \cdot (h \cdot x) \).
\end{enumerate}
The pair \( (X, \cdot) \) (or simply \( X \) ) is then called a \( G \)-set.
\end{definition*}}
\begin{note}[Notation]
If a group \( G \) acts on a set \( X \), one commonly writes: $G \curvearrowright X$.
\end{note}
\begin{remark*}
	A right group action of \( G \) on \( X \) is a function $
	\cdot : X \times G \to X,\quad (x, g) \mapsto x \cdot g$ 
	satisfying: \begin{enumerate}[(i)]
		\item \( x \cdot e = x \) for all \( x \in X \);
		\item \( (x \cdot g) \cdot h = x \cdot (gh) \) for all \( g, h \in G \), \( x \in X \).
	\end{enumerate}
\end{remark*}
\begin{example*}[Scalar Multiplication on a Vector Space]
	Let \( \mathbb{F} \) be a field, and let \( X = \mathbb{F}^n \) be the \( n \)-dimensional vector space over \( \mathbb{F} \). Consider the multiplicative group of nonzero scalars in \( \mathbb{F} \): \[
	G = (\F^\times, \times),\quad\text{where}\; \F^\times=\mathbb{F} \setminus \{0\}.
	\] We define an action \( G \curvearrowright X \) by scalar multiplication:
	\[
	\fullfunction{\cdot}{\F^\times\times\F^n}{\F^n}{(\lambda,\vec{v})}{\lambda\cdot\vec{v}},
	\] where the product \( \lambda\cdot \vec{v} \) is defined componentwise. Then \begin{enumerate}[(i)]
		\item \( 1 \cdot \vec{v} = \vec{v} \) for all \( \vec{v} \in \mathbb{F}^n \).
		\item \( (\lambda\mu) \cdot \vec{v} = \lambda \cdot (\mu \cdot \vec{v}) \) for all \( \lambda, \mu \in \mathbb{F}^\times \), \( \vec{v} \in \mathbb{F}^n \).
	\end{enumerate}
	\begin{center}
	\begin{tikzpicture}[scale=1]
		\coordinate (O) at (0,0);
		\coordinate (v) at (2,1);
		\coordinate (lv) at (3,1.5); % assuming \lambda=1.5
		\draw[->] (-0.5,0) -- (4,0) node[right] {$x$};
		\draw[->] (0,-0.5) -- (0,3) node[above] {$y$};
		\draw[->, thick, blue]  (O) -- (v);
		\fill[blue] (v)  circle (2pt) node[above right] {$\vec v = (v_1,v_2)$};
		\draw[dashed, -Latex] (5,2) to node[above] {$\F^\times\acts\F^2$} (9,2) ;
		\begin{scope}[xshift=10cm]
			\coordinate (O) at (0,0);
			\coordinate (v) at (2,1);
			\coordinate (lv) at (3,1.5); % assuming \lambda=1.5
			\draw[->] (-0.5,0) -- (4,0) node[right] {$x$};
			\draw[->] (0,-0.5) -- (0,3) node[above] {$y$};
			\draw[->, thick, red]   (O) -- (lv);
			\fill[red] (lv) circle (2pt) node[above right] {$\lambda\vec v=(\lambda v_1,\lambda v_2)$};
		\end{scope}
	\end{tikzpicture}
	\end{center}
\end{example*}

\newpage
\begin{example*}[Conjugation Action on the Group Itself]
Let \( G \) be any group, and consider \( X = G \). Define an action of \( G \) on itself by conjugation: \[
G \curvearrowright G, \quad (g, x) \mapsto g \cdot x := g\ast x\ast g^{-1}.
\] Then \begin{enumerate}[(i)]
	\item \( e \cdot x = e\ast x\ast e^{-1} = x \) for all \( x \in G \).
	\item Note that \begin{align*}
		(g\ast h) \cdot x &= (g\ast h)\ast x\ast (g\ast h)^{-1} \\
		&= (g\ast h)\ast x\ast (h^{-1}\ast g^{-1})\\
		&= g\ast (h\ast x\ast h^{-1})\ast g^{-1}\\
		&= g\ast (h\cdot x)\ast g^{-1}\\
		&=g \cdot (h \cdot x).
	\end{align*} Thus, this is a left group action.
%	Its **orbits** are the **conjugacy classes** in \( G \), and the **stabilizer** of \( x \in G \) is the **centralizer**: \[
%	\operatorname{Stab}_G(x) = \{ g \in G \mid gxg^{-1} = x \} = C_G(x).
%	\]
%	The kernel of this action is the **center** \( Z(G) = \{ g \in G \mid gx = xg \text{ for all } x \in G \} \).
\end{enumerate}
\end{example*}
\begin{example*}[Trivial \( G \)-Set]
Let \( G \) be any group and define the set \( X = \{x\} \), a singleton. Define the action
\[
G \curvearrowright X, \quad (g,x)\mapsto g \cdot x := x \quad \text{for all } g \in G.
\] This is the \textbf{trivial action}, where every group element acts as the identity on \( X \):
\vfill
\begin{enumerate}[(i)]
	\item \( e \cdot x = x \).
	\item \( (g\ast h) \cdot x = x = g \cdot (h \cdot x) \).
\end{enumerate}
%This is the unique action of \( G \) on a singleton, and it appears naturally when defining constant \( G \)-sets or fixed points. 
%in more general actions.
\end{example*}
\vfill
\begin{example*}[Action on Coset Space \( G/H \)]
Let $(G,\ast)$ be a group, and let \( H \leq G \). Let \( X = G/H \) be the set of left cosets of \( H \) in \( G \), \ie,
\[
X = G/H = \{ gH \mid g \in G \}.
\] Define an action \[
G \curvearrowright G/H, \quad (g, aH) \mapsto (ga)H.
\] This is well-defined because if \( a_1H = a_2H \), then \( a_1^{-1}a_2 \in H \), so: $
g a_1 H = g a_2 H.$. Since \begin{enumerate}[(i)]
	\item \( e \cdot aH = aH \);
	\item \( (gh) \cdot aH = g \cdot (h \cdot aH) \),
\end{enumerate}
this is a \textbf{transitive action}.
%and the stabilizer of the coset \( eH \in G/H \) is \( H \) itself. 
%This action reflects the **quotient structure** of \( G \) by \( H \), and is fundamental in the construction of **homogeneous spaces** and **induced representations**.
\end{example*}
\newpage
\probox[Group Elements Act as Permutations]{\begin{proposition*}
Let \( G \) be a group action on a set \( X \) via a left action \( G \curvearrowright X \), given by \( (g, x) \mapsto g \cdot x \). Then for each \( g \in G \), the map
\[
\sigma_g : X \to X, \quad x \mapsto g \cdot x
\]
is one-to-one and onto. {\color{gray!50} That is, \( \sigma_g \in \operatorname{Sym}(X) \), the group of all permutations of \( X \).}
\end{proposition*}}
\begin{proof}
	TBA
	%To show that \( \sigma_g \) is bijective, observe that its inverse is \( \sigma_{g^{-1}} \), since for all \( x \in X \):
	%
	%\[
	%\sigma_{g^{-1}}(\sigma_g(x)) = g^{-1} \cdot (g \cdot x) = (g^{-1}g) \cdot x = e \cdot x = x,
	%\]
	%\[
	%\sigma_g(\sigma_{g^{-1}}(x)) = g \cdot (g^{-1} \cdot x) = (gg^{-1}) \cdot x = e \cdot x = x.
	%\]
	%
	%Hence, \( \sigma_g \) is invertible with inverse \( \sigma_{g^{-1}} \), and thus \( \sigma_g \) is bijective.
\end{proof}

\thmbox[Group Actions Induce Permutation Representations]{\begin{theorem*}
Let \( G \) be a group action on a set \( X \) via a left group action $G \curvearrowright X, \quad (g, x) \mapsto g \cdot x$. For each \( g \in G \), define the bijection \( \sigma_g : X \to X \) by $
\sigma_g(x) := g \cdot x$. Then the map \[
\phi: G \to \operatorname{Sym}(X), \quad g\mapsto \sigma_g,
\] is a \textbf{group homomorphism} from \( G \) to the symmetric group \( \operatorname{Sym}(X) \).
In other words, for all \( g, h \in G \), \[
\phi(g\ast h) = \sigma_{g\ast h}=\sigma_g\circ\sigma_h= \phi(g) \circ \phi(h).
\]
\end{theorem*}}
\begin{remark*}
	A group action $G\acts X$ is equivalent to a group homomorphism $G\to \operatorname{Sym}(X)$, i.e., a \textbf{permutation representation} of $G$.
\end{remark*}
\begin{proof}
	TBA
%### **Proof.**
%
%Let \( g, h \in G \), and let \( x \in X \). By the definition of \( \sigma_g \) and the group action axioms,
%
%\[
%\sigma_{gh}(x) = (gh) \cdot x = g \cdot (h \cdot x) = \sigma_g(\sigma_h(x)) = (\sigma_g \circ \sigma_h)(x).
%\]
%
%Since this equality holds for all \( x \in X \), it follows that \( \sigma_{gh} = \sigma_g \circ \sigma_h \). Therefore, \( \varphi \) is a homomorphism.
\end{proof}

\thmbox[Cayley Theorem]{\begin{theorem*} 
Let \(G\) be a group. Consider the action of \(G\) on itself by left multiplication.  For each \(g\in G\), define  
\[
\sigma_g \;:\; G \;\longrightarrow\; G,\qquad x\;\mapsto\;g\cdot x.
\]  
Then the map  
\[
\phi \;:\; G \;\longrightarrow\;\operatorname{Sym}(G),\qquad g\;\mapsto\;\sigma_g
\]  
is an injective group homomorphism (group monomorphism).  In particular,  \[
\phi(G)\simeq G
\quad\text{and}\quad
\varphi(G)\le\operatorname{Sym}(G).
\]
\end{theorem*}}
\begin{proof}
%#### Proof sketch.  
%1. **Homomorphism.**  For all \(g,h\in G\) and \(x\in G\),  
%\[
%\varphi(gh)(x)
%= \sigma_{gh}(x)
%= (gh)\,x
%= g\,(h\,x)
%= \sigma_g\bigl(\sigma_h(x)\bigr)
%= \bigl(\sigma_g\circ\sigma_h\bigr)(x),
%\]  
%hence \(\varphi(gh)=\sigma_g\circ\sigma_h=\varphi(g)\varphi(h)\).
%
%2. **Injectivity.**  If \(\varphi(g)=\mathrm{id}_G\), then for every \(x\in G\),  
%\[
%g\,x = \sigma_g(x) = x,
%\]  
%so in particular \(g\cdot e = e\) forces \(g=e\).  Thus \(\ker\varphi=\{e\}\) and \(\varphi\) is injective.
\end{proof}

\newpage
\section*{Normal Subgroups}
\probox[Existence of the Quotient Group]{\begin{proposition*}
	Let \((G,\ast)\) be a group and let \(H\le G\) be a subgroup.  Define a binary operation $\boxast$ on the set of left cosets \(G/H\) by  \[
	(g\ast H)\boxast (g'\ast H) \;=\;(g\ast g')\ast H
	\] where $g,g'\in G$.
	Then this operation is well-defined (and makes \(G/H\) into a group) if and only if \[
	g\ast h\ast g^{-1}\;\in\;H.
	\]  for all $g\in G$, $h\in H.$
\end{proposition*}}
\begin{proof}
\begin{itemize}
	\item[($\Rightarrow$)] Let $g\in G$ and $h\in H$. Then \begin{align*}
		h\ast g^{-1}\in H\implies g\ast H=g(h)
	\end{align*}
	\item[($\Leftarrow$)]
\end{itemize}
\end{proof}

\defbox[Normal Subgroup]{\begin{definition*}
	
\end{definition*}}

\vfill
\begin{thebibliography}{9}
	\bibitem{abstract_algebra_d}
	수학의 즐거움, Enjoying Math. ``수학 공부, 기초부터 대학원 수학까지, 23. 추상대수학 (d) 군 작용과 케일리-정리 group action and Cayley theorem'' YouTube Video, 29:20. Published 
	October 23, 2019. URL: \url{https://www.youtube.com/watch?v=5SQfrH83HfA&t=1040s}.
	\bibitem{abstract_algebra_e}
	수학의 즐거움, Enjoying Math. ``수학 공부, 기초부터 대학원 수학까지, 24. 추상대수학 (e) 정규부분군의 정의 def of normal subgroups'' YouTube Video, 23:00. Published 
	October 25, 2019. URL: \url{https://www.youtube.com/watch?v=3UJILZr4CNo}.
\end{thebibliography}

\newpage

\appendix
\section*{Appendices}
\defbox[Orbit and Stabilizer]{\begin{definition*}
Let \( G \) be a group acting on a set \( X \) via a (left) group action: \[
G \curvearrowright X, \quad (g, x) \mapsto g \cdot x.
\] Let $x\in X$.
\begin{enumerate}[(1)]
	\item The \textbf{orbit} of \( x \) under the action of \( G \) is defined by
	\[
	\operatorname{Orb}_G(x) := G \cdot x = \{ g \cdot x \mid g \in G \}\subseteq X.
	\]
	This is the set of all elements of \( X \) to which \( x \) can be moved by the action of $g\in G$. 
	\item The \textbf{stabilizer subgroup} of an \( x \in X \), also called the \textbf{isotropy subgroup} or \textbf{fixer}, is defined by
	\[
	\operatorname{Stab}_G(x) := \{ g \in G \mid g \cdot x = x \}.
	\]
	This is a subgroup of \( G \) consisting of all elements that fix \( x \) under the action.
\end{enumerate}
\end{definition*}}
\begin{remark*}
	\ \begin{itemize}
		\item The orbits form a \emph{partition} of \( X \); that is, \( X \) is the disjoint union of its orbits under the action of \( G \).
	\end{itemize}
%### **Key Properties**
%
%- \( \operatorname{Stab}_G(x) \leq G \) is a subgroup of \( G \).
%- The **orbit-stabilizer theorem** states that if \( G \) is a finite group, then:
%\[
%|\operatorname{Orb}_G(x)| = [G : \operatorname{Stab}_G(x)],
%\]
%where the right-hand side is the index of \( \operatorname{Stab}_G(x) \) in \( G \).
\end{remark*}

\end{document}
