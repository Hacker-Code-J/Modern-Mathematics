\documentclass[11pt,openany]{article}

\input{grad-math-preamble}
\usepackage{tcolorbox}
\tcbset{colback=white, arc=5pt}

\definecolor{axiomcolor}{HTML}{a88bfa}
\definecolor{defcolor}{RGB}{52, 152, 219}
\definecolor{procolor}{RGB}{241, 196, 15}
\definecolor{thmcolor}{RGB}{231, 76, 60}
\definecolor{lemcolor}{RGB}{155, 89, 182}
\definecolor{corcolor}{RGB}{46, 204, 113}
\definecolor{execolor}{RGB}{90, 128, 127}

% Define a new command for the custom tcolorbox
\newcommand{\axiombox}[2][]{%
	\begin{tcolorbox}[colframe=axiomcolor, title={\color{white}\bfseries #1}]
		#2
	\end{tcolorbox}
}

\newcommand{\defbox}[2][]{%
	\begin{tcolorbox}[colframe=defcolor, title={\color{white}\bfseries #1}]
		#2
	\end{tcolorbox}
}

\newcommand{\probox}[2][]{%
	\begin{tcolorbox}[colframe=procolor, title={\color{white}\bfseries #1}]
		#2
	\end{tcolorbox}
}

\newcommand{\thmbox}[2][]{%
	\begin{tcolorbox}[colframe=thmcolor, title={\color{white}\bfseries #1}]
		#2
	\end{tcolorbox}
}

\newcommand{\lembox}[2][]{%
	\begin{tcolorbox}[colframe=lemcolor, title={\color{white}\bfseries #1}]
		#2
	\end{tcolorbox}
}
\usepackage{amsthm}

% Define custom theorem styles
\newtheoremstyle{dotless} % Name of the style
{3pt} % Space above
{3pt} % Space below
{\itshape} % Body font
{} % Indent amount
{\bfseries} % Theorem head font
{} % Punctuation after theorem head
{2.5mm} % Space after theorem head
{} % Theorem head spec

\newtheoremstyle{definitionstyle} % Name of the style
{3pt} % Space above
{3pt} % Space below
{} % Body font
{} % Indent amount
{\bfseries} % Theorem head font
{.} % Punctuation after theorem head
{2.5mm} % Space after theorem head
{} % Theorem head spec

% Applying custom styles
%\theoremstyle{dotless}
\newtheorem{theorem}{Theorem} % Theorem environment with section-wise numbering
\newtheorem*{theorem*}{Theorem} % Theorem environment with section-wise numbering
\newtheorem*{lemma*}{Lemma} % Theorem environment with section-wise numbering
\newtheorem*{proposition*}{Proposition} % Theorem environment with section-wise numbering
\newtheorem*{corollary*}{Corollary} % Theorem environment with section-wise numbering
\newtheorem{proposition}[theorem]{Proposition} % Theorem environment with section-wise numbering
\newtheorem{lemma}[theorem]{Lemma} % Lemma shares the counter with theorem
\newtheorem{corollary}[theorem]{Corollary} % Corollary shares the counter with theorem

\theoremstyle{definitionstyle}
\newtheorem*{observation}{\textcolor{magenta}{Observation}}
\newtheorem*{illustration}{\textcolor{teal}{Illustration}}
\newtheorem*{torus}{{\color{red}T}{\color{orange}o}{\color{green!75!black}r}{\color{cyan}u}{\color{violet}s}}
\newtheorem{definition}{Definition} % Definition shares the counter with theorem
\newtheorem{example}{Example} % Example shares the counter with theorem
\newtheorem{exercise}{{Exercise}} % Example shares the counter with theorem
\newtheorem{remark}{Remark} % Remark shares the counter with theorem
\newtheorem*{note}{Note}
\newtheorem*{notation}{Notation}

\newtheorem*{axiom*}{Axiom}
\newtheorem*{definition*}{Definition} % Definition shares the counter with theorem
\newtheorem*{example*}{Example} % Example shares the counter with theorem
\newtheorem*{exercise*}{\textcolor{teal}{Exercise}} % Example shares the counter with theorem
\newtheorem*{remark*}{Remark} % Remark shares the counter with theorem


\usepackage{tikz}
\usepackage{tikz-cd}
\usetikzlibrary{shadows}
\usetikzlibrary{shapes.geometric, arrows.meta, positioning}
\newcommand{\ie}{\textnormal{i.e.}}
\newcommand{\rsa}{\mathsf{RSA}}
\newcommand{\rsacrt}{\mathsf{RSA}\textendash\mathsf{CRT}}
\newcommand{\inv}[1]{#1^{-1}}

%New Command
%\newcommand{\set}[1]{\left\{#1\right\}}
\newcommand{\N}{\mathbb{N}}
\newcommand{\Z}{\mathbb{Z}}
\newcommand{\Q}{\mathbb{Q}}
\newcommand{\R}{\mathbb{R}}
\newcommand{\cR}{\mathcal{R}}
\newcommand{\C}{\mathbb{C}}
\newcommand{\F}{\mathbb{F}}
\newcommand{\nbhd}{\mathcal{N}}
\newcommand{\Log}{\operatorname{Log}}
\newcommand{\Arg}{\operatorname{Arg}}
\newcommand{\pv}{\operatorname{P.V.}}

\newcommand{\of}[1]{\left( #1 \right)} 
%\newcommand{\abs}[1]{\left\lvert #1 \right\rvert}
%\newcommand{\norm}[1]{\left\| #1 \right\|}

\newcommand{\sol}{\textcolor{magenta}{\bf Sol}}
\newcommand{\conjugate}[1]{\overline{#1}}

\newcommand{\res}{\operatorname{res}}
\DeclareMathOperator*{\Res}{\operatorname{Res}}

%\renewcommand{\Re}{\operatorname{Re}}
%\renewcommand{\Im}{\operatorname{Im}}

\newcommand{\cyclic}[1]{\langle #1 \rangle}
\newcommand{\uniform}{\overset{\$}{\leftarrow}}
\newcommand{\xmark}{\textcolor{red}{\XSolidBrush}}
\newcommand{\vmark}{\textcolor{green!75!black}{\CheckmarkBold}}

\newcommand{\gen}[1]{\langle #1 \rangle}
\newcommand{\Gen}[1]{\left\langle #1 \right\rangle}

\newcommand{\img}[1]{\text{Img}(#1)}
\newcommand{\Img}[1]{\text{Img}\left(#1\right)}
\newcommand{\preimg}[1]{\text{Img}^{-1}(#1)}
\newcommand{\Preimg}[1]{\text{Img}^{-1}\left(#1\right)}

\newcommand{\relation}{\mathrel{\mathcal{R}}}
\newcommand{\injection}{\rightarrowtail}
\newcommand{\surjection}{\twoheadrightarrow}
\newcommand{\id}{\textnormal{id}}

\newcommand{\eqclass}[1]{\left[#1\right]}

% Define custom colors for O and X
\newcommand{\yes}{\textcolor{blue}{\bf \fullmoon}}
\newcommand{\no}{\textcolor{red}{\bf \texttimes}}

\DeclarePairedDelimiter\ceil{\lceil}{\rceil}
\DeclarePairedDelimiter\floor{\lfloor}{\rfloor}
%\renewcommand{\floor}[#1]{\lfloor #1\rfloor}
%\newcommand{\Floor}[#1]{\left\lfloor #1\right\rfloor}
%\newcommand{\ceil}[#1]{\lceil #1\rceil}
%\newcommand{\Ceil}[#1]{\left\lceil #1\right\rceil}

\newcommand{\topology}{\mathscr{T}}
\newcommand{\sequence}[1]{\langle #1\rangle}
\renewcommand{\vec}[1]{\mathbf{#1}}
\renewcommand{\Re}{\operatorname*{Re}}
\renewcommand{\Im}{\operatorname*{Im}}
\setstretch{1.25}

%\usepackage{background}
%\backgroundsetup{
%	scale=3,
%	color=gray!20,
%	opacity=0.3,
%	angle=45,
%	contents={\Huge \sffamily Ji, Yong-hyeon}
%}
\begin{document}
\pagenumbering{arabic}
\begin{center}
	\huge\textbf{Abstract Algebra I}\\
	\vspace{0.5em}
	\large{Ji, Yong-hyeon}\\
%	\large{\ttfamily \url{https://github.com/Hacker-Code-J}}\\
	\vspace{0.5em}
	\normalsize{\today}\\
\end{center}

\noindent 
We cover the following topics in this note.
\begin{itemize}
	\item Cyclic Group
	\item Classification of Cyclic Group
	\item Order of an Element
	\item TBA
\end{itemize}
\hrule\vspace{12pt}
%\tableofcontents
\newpage

%We develop the concept of cyclic groups and establish their complete classification.

\begin{note}
Let \( (G, \ast) \) be a group with identity element \( e \). Recall that the axioms of a group require:
\begin{enumerate}[(G1)]
	\item[(G0)] $\forall\, x, y \in G,  x \ast y\in G$;
	\item $\forall\, x, y, z \in G,\;  (x \ast y) \ast z = x \ast (y\ast z)$;
	\item $\exists\, e \in G,\; \text{s.t.}\; \forall\, x \in G,\; e \cdot x = x \cdot e = x$;
	\item $\forall\, x \in G,\; \exists\, x^{-1} \in G\; \text{s.t.}\; x \cdot x^{-1} = x^{-1} \cdot x = e$.
\end{enumerate}
%Consider \((\mathbb{Z},+)\) as the additive group of integers. For \( a \in G \) and \( n \in \mathbb{Z} \), the notation
%\[
%a^n \coloneqq
%\begin{cases}
%	\underbrace{a \ast a \ast \cdots \ast a}_{n \text{ times}} &: n > 0, \\
%	e &: n = 0, \\
%	(a^{-1})^{-n}&: n < 0,
%\end{cases}
%\]
%defines the \( n \)-th power of \( a \).
\end{note}
\defbox[Cyclic Group]{\begin{definition*}
	A group $G$ is said to be \textbf{cyclic} if and only if \[
	\exists\, a \in G \;\text{such that}\; \Bigl[\, \forall\, g \in G,\; \exists\, n \in \mathbb{Z} \text{ with } g = a^n \Bigr].
	\] The element \( a \) is called a \textbf{generator} of \( G \).
\end{definition*}}
\begin{remark*}
	The notation \(a^n\) (or $na$) is understood in the group-theoretic sense, \[
	a^n \coloneqq
	\begin{cases}
		\underbrace{a \ast a \ast \cdots \ast a}_{n \text{ factors}} &: n > 0, \\
		e_G &: n = 0, \\
		(a^{-1})^{-n}&: n < 0,
	\end{cases}\quad\quad\text{or}\quad\quad	
na \coloneqq
\begin{cases}
\underbrace{a \ast a \ast \cdots \ast a}_{n \text{ factors}} &: n > 0, \\
e_G &: n = 0, \\
(-n)(-a)&: n < 0.
\end{cases}
\]
%	where for \(n \ge 0\), 
%	\[
%	a^n = \underbrace{a \ast a\ast \cdots \ast a}_{n\text{ factors}},
%	\]
%	and for \(n < 0\), 
%	\[
%	a^n = (a^{-1})^{-n}.
%	\]
We wish to show that for all \(m, n\in\Z\), \[
g^{m+n} = g^m \ast g^n.
\]
\textbf{Case 1.} (\(m,n \ge 0\)); We prove by induction on \(n\) that \(g^{m+n} = g^m \cdot g^n\) for any fixed \(m \ge 0\).
\begin{enumerate}[(i)]
	\item Basic Step: $n=0$. Since \(g^0 = e\), we have: \[
	g^{m+0} = g^m = g^m \ast e = g^m \ast g^0.
	\]
	\item Inductive Step: Assume that for some \(n \ge 0\), the statement holds; that is, $g^{m+n} = g^m \ast g^n$. Observe that \[
	g^{m+(n+1)} = g^{(m+n)+1} = g^{m+n} \ast g.
	\]
	By the induction hypothesis, \[
	g^{m+n} \ast g = \bigl( g^m \ast g^n \bigr) \ast g.
	\]
	By the associativity of the group operation, we can regroup the factors: \[
	\bigl( g^m \ast g^n \bigr) \ast g = g^m \ast (g^n \ast g) = g^m \ast g^{n+1}.
	\]
\end{enumerate}
\textbf{Case 2.} \(m,n \le 0\).\\
Let \(m = -p\) and \(n = -q\) with \(p,q \ge 0\). Then,
\[
g^{m+n} = g^{-p - q} = \bigl( g^{p+q} \bigr)^{-1}.
\]
From Case 1, we know that
\[
g^{p+q} = g^p \cdot g^q.
\]
Taking the inverse of both sides and using the group property \((xy)^{-1} = y^{-1} x^{-1}\), we have:
\[
\bigl( g^{p+q} \bigr)^{-1} = (g^p \cdot g^q)^{-1} = (g^q)^{-1} \cdot (g^p)^{-1}.
\]
By definition, \((g^q)^{-1} = g^{-q}\) and \((g^p)^{-1} = g^{-p}\). Thus,
\[
g^{m+n} = g^{-q} \cdot g^{-p}.
\]
Since addition in the integers is commutative, we note that
\[
g^m \cdot g^n = g^{-p} \cdot g^{-q} = g^{-q} \cdot g^{-p},
\]
which implies \(g^{m+n} = g^m \cdot g^n\) when \(m,n \le 0\).

\textbf{Case 3.} Mixed Signs.\\
Without loss of generality, assume \(m \ge 0\) and \(n < 0\). Write \(n = -q\) with \(q \ge 0\). We wish to prove:
\[
g^{m - q} = g^m \cdot g^{-q}.
\]
There are two subcases:

\underline{Subcase 3a:} \(m \ge q\).\\
Then \(m - q \ge 0\) and from the definition of the inverse and Case 1, note that:
\[
g^m = g^{(m - q) + q} = g^{m-q} \cdot g^q.
\]
Multiplying on the right by \((g^q)^{-1}\) (which is \(g^{-q}\)) yields:
\[
g^{m-q} = g^m \cdot (g^q)^{-1} = g^m \cdot g^{-q}.
\]

\underline{Subcase 3b:} \(m < q\).\\
Then \(m - q < 0\) and we write
\[
g^{m-q} = \bigl( g^{q-m} \bigr)^{-1}.
\]
By Case 1,
\[
g^q = g^m \cdot g^{q-m}.
\]
Taking inverses gives:
\[
g^{-q} = \bigl( g^m \cdot g^{q-m} \bigr)^{-1} = \bigl( g^{q-m} \bigr)^{-1} \cdot (g^m)^{-1} = g^{m-q} \cdot g^{-m}.
\]
Then multiplying on the right by \(g^m\) yields:
\[
g^{m-q} = g^{-q} \cdot g^m.
\]
In a group, although the operation need not be commutative, the definitions are set up so that the exponent law remains consistent. (A detailed handling of this subcase can be achieved by rephrasing the argument in terms of the identity \(g^{q-m} = (g^{-m} \cdot g^q)^{-1}\); the result is analogous.)

Thus, in all cases, we conclude that for every \(m,n \in \mathbb{Z}\),
\[
g^{m+n} = g^m \cdot g^n.
\]
\end{remark*}

\newpage
\thmbox[The Classification for Cyclic Groups]{\begin{theorem*}
Let \((G,\ast)\) be a cyclic group. Then
\[
(G,\ast) \simeq
\begin{cases}
	(\mathbb{Z},+) & \text{if } G \text{ is infinite,} \\
	(\mathbb{Z}/n\mathbb{Z},+_n) & \text{if } G \text{ is finite of order } n.
\end{cases}
\] In other words, every cyclic group $G$ is isomorphic to either $\mathbb{Z}$ or $\mathbb{Z}/n\mathbb{Z}$ for some $n \in \mathbb{N}$.	
\end{theorem*}}
\begin{proof}
%	The structure of cyclic groups is completely determined by the order of any generator. 
	Let \( a \in G \) be a generator of the cyclic group \(G\). \\
	\begin{center}
	\ttfamily ------------------------------------------------------------------------------------ Multiplicative Version ------------------------------------------------------------------------------------
	\end{center}
	(Case I) ($G$ is infinite) Assume that $G$ is infinite.
		Define the mapping \[
	\varphi : (\mathbb{Z}, +) \to (G, \ast),\quad n\mapsto \varphi(n) = \underbrace{a\ast\cdots\ast a}_{n\; \text{times}}=:a^n.
	\] We claim that $\varphi$ is bijective homomorphism: \begin{enumerate}[(i)]
		\item (Homomorphism)\;  Let $a,b\in\Z$. Then, we have \[
		\varphi(a+b)=g^{a+b}=\underbrace{g\ast\cdots\ast g}_{a+b\; \text{times}}=\underbrace{(g\ast\cdots\ast g)}_{a\; \text{times}}\;\ast\;\underbrace{(g\ast\cdots\ast g)}_{b\; \text{times}}=g^a\ast g^b = \varphi(a)\ast\varphi(b).
		\]
		\item (Surjectivity)\; By definition of a cyclic group, every element \(h\in G\) is of the form \(h=g^k\) for some \(k\in \mathbb{Z}\). Hence, \[
		\forall h\in G,\; \exists k\in\Z\; \text{s.t.}\; \varphi(k)=g^k = h.
		\] Therefore, \(\varphi\) is surjective.
		\item (Injectivity)\; Suppose \(\varphi(k)=\varphi(\ell)\) for some \(k,\ell\in \mathbb{Z}\). Then \begin{align*}
			g^k=g^\ell&\implies \underbrace{g\ast\cdots\ast g}_{k\; \text{times}}=\underbrace{g\ast\cdots\ast g}_{\ell\; \text{times}}\\
			&\implies \underbrace{g\ast\cdots\ast g}_{k\; \text{times}}\ast \underbrace{g^{-1}\ast\cdots\ast g^{-1}}_{\ell\; \text{times}}=e_G \\
			&\implies g^{k-\ell}=e_G\\
			&\implies k-\ell=0\\
			&\implies k=\ell.
		\end{align*}
	Let \(G\) be a group and \(g\in G\). Suppose that for some integers \(k\) and \(l\), we have
	\[
	g^k = g^l.
	\]
	Since every element in a group has an inverse, we can multiply both sides of the equation by \(g^{-l}\) (the inverse of \(g^l\)). That is,
	\[
	g^k \cdot g^{-l} = g^l \cdot g^{-l}.
	\]
	By the definition of an inverse, we know that \(g^l \cdot g^{-l} = e\), the identity element in \(G\). Also, by the laws of exponents in groups,
	\[
	g^k \cdot g^{-l} = g^{k-l}.
	\]
	Thus, we obtain
	\[
	g^{k-l} = e.
	\]
	
		Hence, \(\varphi\) is injective.
	\end{enumerate}
	Thus, \(\varphi\) is a bijective homomorphism, \ie, $(G,\ast) \simeq (\mathbb{Z},+)$.\ \\
	\ \\
	\noindent
	(Case II)\; (\(G\) is Finite of Order \(n\))

	Now assume that \(G\) is finite, say, \(|G|=n\). Then by the definition of a cyclic group of finite order, there exists a minimal positive integer \(n\) such that
\[
g^n = e_G.
\]
We now show that for any \(k,\ell\in \mathbb{Z}\), \[
g^k = g^\ell \quad \text{if and only if} \quad k\equiv \ell \ (\text{mod } n).
\] \begin{itemize}
	\item[($\Rightarrow$)] Let \(g^k = g^\ell\). Then
	\[
	g^{k-\ell}=e_G.
	\]
	By the minimality of \(n\), it must be that \(n\) divides \(k-\ell\); that is,
	\[
	k-\ell = tn \quad \text{for some } t\in \mathbb{Z},
	\]
	which precisely means \(k\equiv \ell \ (\text{mod } n)\).
	\item[($\Leftarrow$)] Conversely, let \(k\equiv \ell \pmod{n}\). Then \[
	\exists t\in\Z\quad \text{such that}\quad k = \ell + tn.
	\]
	Hence, \begin{align*}
	g^k = g^{\ell + tn} = g^\ell \ast (g^n)^t &= g^\ell \ast e_G^t\\
	&=\underbrace{g\ast\cdots\ast g}_{\ell\; \text{times}}\; \ast\; \underbrace{e_G\ast \cdots\ast e_G}_{t\; \text{times}}\\
	&=g^\ell.
	\end{align*}
	\[
	g^k = g^{\ell + tn} = g^\ell \ast (g^n)^t = g^\ell \ast e_G^t = g^\ell.
	\]
\end{itemize}
Thus, the relation \(g^k = g^\ell\) holds if and only if \(k\) and \(\ell\) are congruent modulo \(n\).
Define the mapping\[
\psi: \mathbb{Z}/n\mathbb{Z} \to G,\quad [k]\mapsto \psi([k]):=g^k,
\] where \([k]\) denotes the equivalence class of \(k\) modulo \(n\), that is, $[k]=\set{\ell\in\Z:\ell\equiv k \pmod{n} }$.
We NTS that \(\psi\) is a bijective homomorphism:
\begin{enumerate}[(i)]
	\item (Homomorphism)\; Let \([k],[\ell]\in \mathbb{Z}/n\mathbb{Z}\). Then
	\[
	\psi([k]+[\ell]) = \psi([k+\ell]) = g^{k+\ell} = g^k \ast g^\ell = \psi([k]) \ast \psi([\ell]).
	\]
	\item (Surjectivity)\; Every element \(h\in G\) is of the form \(h=g^k\) for some \(k\in \mathbb{Z}\), and so \(h=\psi([k])\).  That is, \[
	\forall h\in G,\; \exists [k]\in\Z/n\Z\; \text{s.t.}\; \psi([k])=g^k = h.
	\] Therefore, \(\psi\) is surjective.
	\item (Injectivity)\; Suppose \(\psi([k])=\psi([\ell])\) for some $[k],[\ell]\in\Z/n\Z$. That is, \(g^k=g^\ell\). Then \(k\equiv \ell\ (\text{mod } n)\), and so $[k]=[l]$.
\end{enumerate}
\begin{center}
	\ttfamily --------------------------------------------------------------------------------------------- Additive Version ---------------------------------------------------------------------------------------------
\end{center}
(Case I) ($G$ is infinite) Assume that $G$ is infinite.
Define the mapping \[
\varphi : (\mathbb{Z}, +) \to (G, \ast),\quad n\mapsto \varphi(n) = \underbrace{a\ast\cdots\ast a}_{n\; \text{times}}=: na.
\] We claim that $\varphi$ is bijective homomorphism: \begin{enumerate}[(i)]
	\item (Homomorphism)\;  Let $a,b\in\Z$. Then, we have \[
	\varphi(a+b)=g^{a+b}=\underbrace{g\ast\cdots\ast g}_{a+b\; \text{times}}=\underbrace{(g\ast\cdots\ast g)}_{a\; \text{times}}\;\ast\;\underbrace{(g\ast\cdots\ast g)}_{b\; \text{times}}=ag\ast bg = \varphi(a)\ast\varphi(b).
	\]
	\item (Surjectivity)\; By definition of a cyclic group, every element \(h\in G\) is of the form \(h=kg\) for some \(k\in \mathbb{Z}\). Hence, \[
	\forall h\in G,\; \exists k\in\Z\; \text{s.t.}\; \varphi(k)=kg = h.
	\] Therefore, \(\varphi\) is surjective.
	\item (Injectivity)\; Suppose \(\varphi(k)=\varphi(l)\) for some \(k,l\in \mathbb{Z}\). Then \begin{align*}
		kg=lg&\implies (k-l)g=e_G\\
		&\implies k-l=0\\
		&\implies k=l.
	\end{align*}
	Hence, \(\varphi\) is injective.
\end{enumerate}
Thus, \(\varphi\) is a bijective homomorphism, \ie, $(G,\ast) \simeq (\mathbb{Z},+)$.
\newpage\noindent
(Case II)\; (\(G\) is Finite of Order \(n\))

Now assume that \(G\) is finite, say, \(|G|=n\). Then by the definition of a cyclic group of finite order, there exists a minimal positive integer \(n\) such that
\[
g^n = e_G.
\]
We now show that for any \(k,\ell\in \mathbb{Z}\), \[
g^k = g^\ell \quad \text{if and only if} \quad k\equiv \ell \ (\text{mod } n).
\] \begin{itemize}
	\item[($\Rightarrow$)] Let \(g^k = g^\ell\). Then
	\[
	g^{k-\ell}=e_G.
	\]
	By the minimality of \(n\), it must be that \(n\) divides \(k-\ell\); that is,
	\[
	k-\ell = tn \quad \text{for some } t\in \mathbb{Z},
	\]
	which precisely means \(k\equiv \ell \ (\text{mod } n)\).
	\item[($\Leftarrow$)] Conversely, let \(k\equiv \ell \pmod{n}\). Then \[
	\exists t\in\Z\quad \text{such that}\quad k = \ell + tn.
	\]
	Hence, \begin{align*}
		g^k = g^{\ell + tn} = g^\ell \ast (g^n)^t &= g^\ell \ast e_G^t\\
		&=\underbrace{g\ast\cdots\ast g}_{\ell\; \text{times}}\; \ast\; \underbrace{e_G\ast \cdots\ast e_G}_{t\; \text{times}}\\
		&=g^\ell.
	\end{align*}
	\[
	g^k = g^{\ell + tn} = g^\ell \ast (g^n)^t = g^\ell \ast e_G^t = g^\ell.
	\]
\end{itemize}
Thus, the relation \(g^k = g^\ell\) holds if and only if \(k\) and \(\ell\) are congruent modulo \(n\).
Define the mapping\[
\psi: \mathbb{Z}/n\mathbb{Z} \to G,\quad [k]\mapsto \psi([k]):=g^k,
\] where \([k]\) denotes the equivalence class of \(k\) modulo \(n\), that is, $[k]=\set{\ell\in\Z:\ell\equiv k \pmod{n} }$.
We NTS that \(\psi\) is a bijective homomorphism:
\begin{enumerate}[(i)]
	\item (Homomorphism)\; Let \([k],[\ell]\in \mathbb{Z}/n\mathbb{Z}\). Then
	\[
	\psi([k]+[\ell]) = \psi([k+\ell]) = g^{k+\ell} = g^k \ast g^\ell = \psi([k]) \ast \psi([\ell]).
	\]
	\item (Surjectivity)\; Every element \(h\in G\) is of the form \(h=g^k\) for some \(k\in \mathbb{Z}\), and so \(h=\psi([k])\).  That is, \[
	\forall h\in G,\; \exists [k]\in\Z/n\Z\; \text{s.t.}\; \psi([k])=g^k = h.
	\] Therefore, \(\psi\) is surjective.
	\item (Injectivity)\; Suppose \(\psi([k])=\psi([\ell])\) for some $[k],[\ell]\in\Z/n\Z$. That is, \(g^k=g^\ell\). Then \(k\equiv \ell\ (\text{mod } n)\), and so $[k]=[l]$.
\end{enumerate}
Hence, we conclude that:
\[
(G,\ast) \simeq
\begin{cases}
	(\mathbb{Z},+) & \text{if } G \text{ is infinite,} \\
	(\mathbb{Z}/n\mathbb{Z},+_n) & \text{if } G \text{ is finite of order } n.
\end{cases}
\]
\end{proof}

\newpage

\newpage
\probox{\begin{proposition*}
	The subgroup of cyclic group is also cyclic. 
\end{proposition*}}

\thmbox{\begin{theorem*}
	Let $(G,\ast)$ be a group and let $H\leq G$ be a cyclic subgroup of $G$. Then $H$ is abelian.
\end{theorem*}}
\begin{proof}
	Since $H$ is cyclic, $\exists g\in G$ such that \[
	H=\cyclic{g}=\set{g^k:k\in\Z},\quad\text{where}\quad g^k:=\begin{cases}
		\underbrace{g\ast\cdots\ast g}_{k\; \text{factors}} & :k>0 \\
		0 &: k=0 \\
		(g^{-1})^{-k} &: k < 0
	\end{cases}.
	\] Let $h_1,h_2$ be arbitrary. Then $\exists m,n\in\Z$ such that \[
	h_1=g^m\quad\text{and}\quad h_2=g^n.
	\] Thus, \[
	h_1\ast h_2=g^n\ast g^m=g^{n+m}=g^{m+n}=g^m\ast g^n=h_2\ast h_1.
	\]
\end{proof}

\newpage
\defbox[Order of an Element]{\begin{definition*}
	Let $(G,\ast)$ be a group with identity element $e\in G$. For any $g\in G$, we define the set \[
	\set{n\in\N:g^n=e}\subseteq\N,
	\] where for each $n\in\N$, $g^n$ is the $n$-fold $\ast$-operation of $g$ with itself (and by convention, $g^0:=e$). Then the \hl{\textbf{order of $g$}}, denoted by $\ord(g)$, is defined by \[
	\ord(g):=\begin{cases}
		\min\set{n\in\N:g^n=e}&:\varnothing\neq\set{n\in\N:g^n=e}\\
		\infty &:\varnothing=\set{n\in\N:g^n=e}
	\end{cases}
	\] That is, if \hl{there exists at least one positive integer $n\in\Z^+$ such that 
	$g^n$, then $\ord(g)$ is the smallest such $n$}; otherwise, we say that 
	$g$ has infinite order and write $\ord(g)=\infty$.
\end{definition*}}
\begin{remark*}[Specialization to Cyclic Groups.]
If \((G, \ast)\) is a cyclic group, then \(\exists g \in G \) (called a generator) such that \[
\cyclic{g}: = \{ g^k : k \in \mathbb{Z} \} = G.
\] \begin{itemize}
	\item If \( G \) is infinite, then no positive integer \( n \) satisfies \( g^n = e \), so \[
	\set{n\in\N:g^n=e}=\varnothing\quad\text{and}\quad\text{and consequently $\operatorname{ord}(g) = \infty$.}
	\]
	\item If \( G \) is finite of order \( n \), then by \emph{Lagrange’s Theorem}\footnote{\; $\abs{\cyclic{g}}=\ord(g)$ divides $\abs{G}=n$.} the unique smallest positive integer \( n \) for which \( g^n = e \) must divide \(|G|\), and in the case where \( g \) is a generator, \(\ord(g) = n = \abs{G}\).
\end{itemize}
\end{remark*}
\vfill
\begin{remark*}
	Let $x\in G$ be an element of a cyclic group $G$ with finite order $n=\ord(x)$. Then \[
	x^m=e\iff n\mid m\quad\text{for any $m\in\Z$}.
	\]
%	with finite order $n=\ord(x)$, so that, $x^n=e$ and no positive integer $k$ smaller than $n$ satisfies $x^k=e$. Note that for any integer $m\in\Z$ \[
%	x^m=e\iff n\mid m.
%	\]
	\begin{itemize}
		\item[($\Rightarrow$)] By the Division Algorithm, $\exists! q,r$ s.t. $m=nq+r$ and $0\leq r<n$. Then \[
		x^m=x^{nq+r}=x^{nq}\ast x^r=(x^n)^q\ast x^r=e^q\ast x^r=x^r.
		\] Since $x^m=e$, we have \[
		x^r=e\quad\text{with}\quad 0\leq r<n.
		\] However, by definition of $n=\ord(x)$, $r$ must be 0. Thus, $m=nq$, \ie, $n\mid m$.
		\item[($\Leftarrow$)] $n\mid m\implies \exists q\in\Z: m=nq\implies x^m=x^{nq}=(x^{n})^q=e^q=e$.
	\end{itemize}
\end{remark*}

\newpage
\thmbox[Lagrange’s Theorem]{\begin{theorem*}
Let $G$ be a finite group and let $H\leq G$ be a subgroup. Then \[
\abs{H}\quad\text{divides}\quad\abs{G}.
\]
\end{theorem*}}
\begin{proof}
TBA
\end{proof}
\begin{remark*}
	Let $G$ be a finite group and let $H\leq G$ be a subgroup. Then by Lagrange's Theorem, \[
	\exists a\in \Z\;\text{such that}\; \abs{G}= a\cdot\abs{H}.
	\]The \textbf{index} of $H$ in $G$, denoted by $[G:H]$, is defined by \[
	[G:H]=\frac{\abs{G}}{\abs{H}}.
	\] In other words, $\abs{G}=[G:H]\cdot\abs{H}.$
\end{remark*}

\thmbox[Division Algorithm]{\begin{theorem*}
	
\end{theorem*}}

\lembox{\begin{lemma*}
	Let $G$ be a cyclic group and let $x\in G$ with $\ord(x)=n\in\N$. Then, for each $a\in\Z$, \[
	\boxed{\ord(x^a)=\frac{n}{\gcd(n,a)}}.
	\]
\end{lemma*}}
\begin{proof}
	By definition of order of an element $x^a$, we know that \[
	\ord(x^a):=\min\set{k\in\N:(x^a)^k=e}.
	\] Since $x^{ak}=e\iff n\mid ak$, we may write \[
	\ord(x^a):=\min\set{k\in\N:(x^a)^k=e}.
	\] Let $d:=\gcd(n,a)\neq 0$. Then $d\mid n$ and $d\mid a$, and so \[
	\exists k_n, k_a\in\Z\; \text{such that}\; n=dk_n\; \text{and}\; a=dk_a,\quad\quad\text{with}\quad \gcd(k_n,k_a)=\gcd\left(\frac{n}{d},\frac{a}{d}\right)=1.
	\] Since \[
	n\mid ak\implies dk_n\mid (dk_a)k\implies k_n\mid k_ak\overset{\gcd(k_n,k_a)=1}{\implies} k_n\mid k\overset{\text{Euclid's Lemma}}{\implies} k_n=k,
	\] we obtain \[
	\ord(x^a)=k_n=\frac{n}{d}=\frac{n}{\gcd(n,a)}.
	\]
\end{proof}
\begin{remark*}
	By above lemma, we obtain \[
	\boxed{\ord(x)=\ord(x^a)\gcd(\ord(x),a)}.
	\]
\end{remark*}

\thmbox[The Converse of Lagrange’s Theorem for Finite Cyclic Groups]{\begin{theorem*}
	Let $G$ be a finite cyclic group with $\abs{G}=n$. Then for each $d\in\N$ with $d\mid n$, \[
	\exists! H\leq G\;\text{such that}\; \abs{H}=d.
	\]
\end{theorem*}}
\begin{proof}
	Since $G$ is cyclic, $\exists x\in G$ such that \[
	G=\cyclic{x}=\set[1]{x^k:k\in\Z},
	\] and by the definition of order, \[
	x^n=e\quad\text{and\quad} n=\ord(x)=\min\set[1]{k\in\N:x^k=e}.
	\] Let $d\in\N$ be a divisor of $n$; that is $d\mid n$, and so \[
	\exists m\in\N\;\text{such that}\; n=dm.
	\]
	\paragraph{(Existence)} Define the element \[
	y:=x^{m}=x^{\frac{n}{d}}\in G
	\] We claim that the subgroup generated by $y$, $H:=\cyclic{y}$, has order $d$; that is $\ord(y)=d$. Note that \[
	H=\cyclic{y}=\set[1]{y^k:k\in\Z}=\set[1]{(x^{m})^k:k\in\Z}.
	\] Here, let $k$ be the smallest positive integer $k$ such that $y^k=e$. Then \begin{align*}
		y^k=e &\implies x^{mk}=e\\
		&\implies n\mid mk\quad\because \ord(x)=n\\
		&\implies dm\mid mk \\
		&\implies d\mid k\quad\because m\neq 0.
	\end{align*} Since $y^d=(x^m)^{d}=x^{md}=x^n=e$ and $k$ is the \emph{smallest} positive integer with this property, $k=d$. Thus, \[
	\ord(y)=k=d.
	\]
	\paragraph{(Uniqueness)} Let $K\leq G$ is any subgroup with $\abs{K}=d$. Then \[
	K=\cyclic{x^m}\quad\text{with}\quad m=\frac{n}{d}.
	\] Since $x$ is a generator of $G$, $\exists k\in \Z$ such that \[
	y=x^k\quad\text{and}\quad\ord(x^k)=d.
	\] Then \[
	\ord(x^k)=\frac{n}{\gcd(n,k)}=d.
	\] Thus, the subgroup of order $d$ in $G$ is unique.
\end{proof}

\begin{remark*}[The Converse of Lagrange’s Theorem for Finite Abelian Groups]
Let $G$ be a finite abelian group and let $d\in\N$ with $d\mid \abs{G}$. Then \[
\exists H\leq G\quad\text{such that}\quad \abs{H}=d.
\] 
\end{remark*}

\corbox{\begin{corollary*}
	Let $p\in\Z$ be a prime. Then $\Z/p\Z$ has no proper subgroup except $\set{e}$.
\end{corollary*}}
\begin{proof}
	TBA.
\end{proof}

\corbox[Euler-Phi Function]{\begin{corollary*}
Define a mapping \[
\fullfunction{\varphi}{\Z}{\Z}{n}{\varphi(n):=\#\set{k\in\Z:k=0,1,2,\dots,n-1\; \text{such that}\; \gcd(n,k)=1}}
\] is called the \textbf{Euler-phi function}.
\end{corollary*}}
\begin{remark*}
	$\varphi(n)$ is the number of generators of $\Z/n\Z$.
\end{remark*}

\probox[Properties of Euler-Phi Function]{\begin{proposition*}
Let $p\in\Z$ be a prime, and let $k,m,n\in\Z$. Then \begin{enumerate}[(1)]
	\item $\varphi(p^k)=p^k-p^{k-1}$.
	\item $\varphi(mn)=\varphi(m)\varphi(n)$.
\end{enumerate}
\end{proposition*}}

\vfill
\begin{thebibliography}{9}
	\bibitem{abstract_algebra_a}
	수학의 즐거움, Enjoying Math. ``수학 공부, 기초부터 대학원 수학까지, 20. 추상대수학 (a) 순환군의 분류 Classification of cyclic group'' YouTube Video, 22:01. Published 
	October 18, 2019. URL: \url{https://www.youtube.com/watch?v=1yQ52OSB_Cc&t=708s}.
%	\bibitem{linear_to_abstract_b}
%	수학의 즐거움, Enjoying Math. ``수학 공부, 기초부터 대학원 수학까지, 19. 선형대수학에서 추상대수학으로 (b) 대수적 구조를 보존하는 함수 algebraic homomorphisms'' YouTube Video, 25:21. Published 
%	October 16, 2019. URL: \url{https://www.youtube.com/watch?v=9TtGaY5COlg&t=187s}.
\end{thebibliography}

\newpage
\appendix
\section{Number Theory}

\subsection{Divisibility}
\defbox[Divisibility]{\begin{definition*}
	Let $a,b\in\Z$ with $a\neq 0$. Then $a$ \textbf{divides} $b$ if \[
	\exists c\in\Z\quad\text{such that}\quad b=ac.
	\] Then $a$ is \emph{divisor} or \emph{factor} of $b$ and $b$ is \emph{multiple} of $a$.
\end{definition*}}
\begin{remark*}
	We write $a\mid b$ if $a$ divides $b$, and $a\nmid b$ otherwise. 
%	That is, \begin{align*}
%		a\mid b&\iff\exists c\in\Z\;\text{such that}\; b=ac\\
%		a\nmid b&\iff\forall c\in\Z,\; b\neq ac.
%	\end{align*}
\end{remark*}
\vfill
\probox[]{\begin{proposition*}
	Let $a,b,c\in \Z$. \begin{enumerate}[(1)]
		\item $a\mid b$ and $b\mid c\implies a\mid c$.
		\item Let $c\neq 0$. Then $ca\mid cb\implies a\mid b$.
	\end{enumerate}
\end{proposition*}}
\begin{proof}
	Let $a,b,c\in\Z$. \begin{enumerate}[(1)]
		\item Let $a\mid b$ and $b\mid c$. Then $\exists \lambda, \mu\in\Z\;\text{s.t.}\; a\lambda=b\;\text{and}\; b\mu=c$. Hence $c=b\mu=(a\lambda)\mu=a(\lambda\mu)$, and so $a\mid c$.
		\item Let $ca\mid cb$ with $c\neq 0$. Then $\exists \lambda\in\Z$ s.t. $cb=ca\lambda$. Hence $b=a\lambda$, and so $a\mid b$.
	\end{enumerate}
\end{proof}
\vfill
\probox[]{\begin{proposition*}
	Let $a,b,c\in \Z$. For any $m,n\in\Z$, \[
	c\mid a\;\text{and}\; c\mid b\implies c\mid (ma+nb).
	\]
\end{proposition*}}
\begin{proof}
	Let $m.n\in\Z$, and let $a\mid b$ and $b\mid c$. Then \[
	\exists e, f\in\Z\;\text{such that}\; a=ce\;\text{and}\; b=cf.
	\] Hence \[
	ma+nb=m(ce)+n(cf)=c(me+nf),
	\] and so $c\mid (ma+nb)$.
\end{proof}
\newpage
\thmbox[Euclid's Lemma]{\begin{theorem*}
	Let \( p \in \mathbb{Z} \) be a prime number and let \( a,b \in \mathbb{Z} \). Then
	\[
	p \mid ab \;\Longrightarrow\; p \mid a\;\text{or}\; p \mid b.
	\]
\end{theorem*}}
\begin{proof}
Let $p\in\mathbb{Z}$ prime and $a,b\in\Z$. Assume that $p\mid ab$. We NTS that \[
p\mid a \quad \text{or} \quad p\mid b.
\] Suppose that \[
p\nmid a.
\]
Since \( p \) is prime and \( p \nmid a \), we know that \[
\gcd(p,a) = 1.
\]
By Bézout's Identity, \( \exists u,v\in\mathbb{Z} \) such that $
up + va = 1$. Then \begin{align*}
	u p + va = 1 &\implies b\bigl(up + va\bigr) = b\\
	&\implies upb+vab=b\\
	&\implies p\mid b\quad\because p\mid ab\; \text{and so}\; p\mid (upb+vab)
\end{align*}
Hence, under the assumption \( p\nmid a \), we obtain \[
p\mid b.
\] Therefore, combining the two cases, we conclude:
\[
p\mid ab \;\Longrightarrow\; p\mid a\;\text{or}\;p\mid b.
\]
\end{proof}
\newpage
\subsection{Modular Arithmetic}
\defbox[Congruence (Number Theory)]{\begin{definition*}
	Let $n$ be a positive integer ($n\in\Z^+$). Two integers 
	$a$ and  $b$ are said to be \textbf{congruent modulo $n$}, written as
	\[
	a\equiv b\pmod{n},
	\] if and only if \[
	n\mid a-b,\quad\ie,\quad \exists k\in\ Z\; \text{s.t.}\; a-b=kn.
	\]
\end{definition*}}
\begin{remark*}[Modulo Operation]
According to the \textbf{division algorithm}, for any integer 
$a$ and any positive integer $n$, there exist unique integers 
$q$ (the quotient) and $r$ (the remainder) such that \[
a=qn+r\quad\text{with}\quad 0\leq r< n.
\] When we express this using the floor function and the mod operation, we identify: \[
q=\floor*{\frac{a}{n}}\quad\text{and}\quad r= a\bmod n.
\] Thus,  we can rewrite the division algorithm as: \[
a=n\floor*{\frac{a}{n}}+(a\bmod n).
\] Thus, we have \[
a\bmod n:=\begin{cases}
	\displaystyle a-n\floor*{\frac{a}{n}} &: n\neq 0\\
	\displaystyle 0 &: n= 0.
\end{cases}
\] Note that \[
a\equiv b\pmod{n}\iff a\bmod n=b\bmod n.
\]
\end{remark*}

\subsection{Greatest Common Divisors}
\defbox[Greatest Common Divisor; GCD]{\begin{definition*}
	Let $a,b\in \Z$. An nonnegative integer $d\in\Z_{\geq 0}$ is called a \textbf{greatest common divisor (gcd)} of $a$ and $b$, denoted by $d=\gcd(a,b)$, if it satisfies the following two conditions: \begin{enumerate}[(i)]
		\item (Divisibility)\; $d\mid a$ and $d\mid b$.
		\item (Maximality)\; For any $c\in Z$, \[
		c\mid a\; \text{and}\; c\mid b\implies c\mid d.
		\]
	\end{enumerate}
\end{definition*}}

\probox{\begin{proposition*}
	Let $a,b,c\in \Z$. \begin{enumerate}[(1)]
		\item $\gcd(a+cb,b)=\gcd(a,b)$.
		\item $\displaystyle\gcd(a,b)=d\implies \gcd\left(\frac{a}{d},\frac{a}{d}\right)=1$.
	\end{enumerate}
\end{proposition*}}

\thmbox[Bezout's Identity]{\begin{theorem*}
	Let $a,b\in\Z$. Then \[
	\exists m,n\in\Z\quad\text{such that}\quad \gcd(a,b)=ma+mb.
	\]
\end{theorem*}}
\begin{remark*}
	Note that there are infinitely many such $m$ and $n$.
\end{remark*}

\corbox{\begin{corollary*}
	Let $a,b\in\Z$. \[
	\gcd(a,b)=1\implies \exists m,n\in\Z\;\ \text{such that}\; ma+nb=1.
	\]
\end{corollary*}}

\end{document}
