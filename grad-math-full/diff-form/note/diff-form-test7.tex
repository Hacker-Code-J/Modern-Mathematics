\documentclass[12pt]{article}
\usepackage{amsmath,amssymb,amsthm, mathtools}
\usepackage{geometry}
\geometry{margin=1in}
\usepackage{hyperref}

\newtheorem{theorem}{Theorem} % Theorem environment with section-wise numbering
\newtheorem*{theorem*}{Theorem} % Theorem environment with section-wise numbering
\newtheorem*{lemma*}{Lemma} % Theorem environment with section-wise numbering
\newtheorem*{proposition*}{Proposition} % Theorem environment with section-wise numbering
\newtheorem*{corollary*}{Corollary} % Theorem environment with section-wise numbering
\newtheorem{proposition}[theorem]{Proposition} % Theorem environment with section-wise numbering
\newtheorem{lemma}[theorem]{Lemma} % Lemma shares the counter with theorem
\newtheorem{corollary}[theorem]{Corollary} % Corollary shares the counter with theorem

\theoremstyle{definitionstyle}
\newtheorem{definition}{Definition} % Definition shares the counter with theorem
\newtheorem{example}{Example} % Example shares the counter with theorem
\newtheorem{exercise}{{Exercise}} % Example shares the counter with theorem
\newtheorem{remark}{Remark} % Remark shares the counter with theorem
\newtheorem*{note}{Note}

\title{Lecture Notes: Coordinates and Differentials on a Plane Curve}
\author{}
\date{}

\newcommand{\R}{\mathbb{R}}
\renewcommand{\span}{\text{span}}
\begin{document}
\maketitle
\section*{Lecture Note: 1-Form as Scalar Projection onto a Fixed Direction}

\subsection*{1. Curve and Its Tangent Line}
Let 
\[
C \;=\;\bigl\{(x,y)\in\R^2\colon y=f(x)\bigr\}
\]
be a smooth curve.  Fix a point 
\[
p=(a,f(a))\in C
\]
and write the tangent direction at \(p\) as
\[
\vec v
=\begin{pmatrix}1\\ f'(a)\end{pmatrix}\in T_p\R^2.
\]
Then the one–dimensional tangent space is
\[
T_pC \;=\;\span\{\vec v\}
\;\subset\;T_p\R^2.
\]

\subsection*{2. Coordinate System on \(C\)}
On the ambient plane \(\R^2\) we have the standard projections
\[
x,y\;:\;\R^2\longrightarrow\R,
\qquad
x(x,y)=x,
\quad
y(x,y)=y.
\]
Restricting to \(C\) yields two functions
\[
x\big|_C\,:\;C\to\R,
\quad
y\big|_C\,:\;C\to\R.
\]
Define the chart
\[
\Phi_C\;:\;C\;\longrightarrow\;\R^2,
\qquad
\Phi_C(p)=\bigl(x|_C(p),\,y|_C(p)\bigr)=(a,f(a)).
\]

\subsection*{3. Coordinate Projections on \(T_pC\)}
At each \(p\in C\), the ambient tangent plane is
\[
T_p\R^2
=\span\bigl\{\partial_x|_p,\;\partial_y|_p\bigr\}
\;\cong\;\R^2.
\]
Its dual coordinates are
\[
dx,\,dy\;:\;T_p\R^2\to\R,
\qquad
dx\bigl((v^1,v^2)^T\bigr)=v^1,
\quad
dy\bigl((v^1,v^2)^T\bigr)=v^2.
\]
Restrict these to the line \(T_pC=\span\{(1,f'(a))^T\}\) to obtain
\[
dx\big|_{T_pC},\;dy\big|_{T_pC}
\;:\;
T_pC\;\longrightarrow\;\R.
\]
Stacking gives the fiber–chart
\[
\Phi_{T_pC}\;:\;T_pC\;\longrightarrow\;\R^2,
\qquad
\Phi_{T_pC}(v)
=\begin{pmatrix}
	dx(v)\\[4pt]
	dy(v)
\end{pmatrix}.
\]
Concretely, if \(v=t\,(1,f'(a))^T\in T_pC\), then
\[
dx(v)=t,
\quad
dy(v)=t\,f'(a),
\quad
\Phi_{T_pC}(v)
=\begin{pmatrix}t\\[3pt]t\,f'(a)\end{pmatrix}.
\]

\subsection*{4. The 1-Form of Scalar Projection}
Fix a unit direction
\[
\mathbf{u}=(\cos\theta,\sin\theta)\;\in\;\R^2.
\]
Define a differential \(1\)-form
\[
\omega\;\in\;\Omega^1(C)
\]
by declaring its action on each tangent vector \(v\in T_pC\subset T_p\R^2\) to be the scalar
projection onto \(\mathbf{u}\):
\[
\forall\,v\in T_pC:\qquad
\omega_p(v)
=\langle \mathbf{u},v\rangle
=\cos\theta\;dx(v)
+\sin\theta\;dy(v).
\]
Linearity in \(v\) and smooth dependence on \(p\) show that \(\omega\) is indeed a smooth
section of the cotangent bundle \(T^*C\).  In particular, for the canonical basis vector
\(\vec v=(1,f'(a))^T\),
\[
\omega_p(\vec v)
=\cos\theta\cdot 1
+\sin\theta\cdot f'(a).
\]

\vspace{1ex}\noindent
\textbf{Abstract Formulation.}  A \emph{differential \(1\)-form} on \(C\) is by definition
a section \(\omega\in\Gamma(T^*C)\).  The above \(\omega\) arises from the ambient
inner–product by pulling back the functional \(v\mapsto\langle\mathbf{u},v\rangle\)
along the inclusion \(T_pC\hookrightarrow T_p\R^2\), thereby encoding the scalar
projection onto the fixed direction \(\mathbf{u}\) at each point of the curve.

	
\end{document}
