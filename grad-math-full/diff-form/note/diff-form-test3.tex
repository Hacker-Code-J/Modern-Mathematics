\documentclass[12pt]{article}
\usepackage{amsmath,amssymb,amsthm}
\usepackage{geometry}
\geometry{margin=1in}
\usepackage{hyperref}
%\usepackage{bm}

\numberwithin{equation}{section}

\newtheorem{theorem}{Theorem} % Theorem environment with section-wise numbering
\newtheorem*{theorem*}{Theorem} % Theorem environment with section-wise numbering
\newtheorem*{lemma*}{Lemma} % Theorem environment with section-wise numbering
\newtheorem*{proposition*}{Proposition} % Theorem environment with section-wise numbering
\newtheorem*{corollary*}{Corollary} % Theorem environment with section-wise numbering
\newtheorem{proposition}[theorem]{Proposition} % Theorem environment with section-wise numbering
\newtheorem{lemma}[theorem]{Lemma} % Lemma shares the counter with theorem
\newtheorem{corollary}[theorem]{Corollary} % Corollary shares the counter with theorem

\theoremstyle{definitionstyle}
\newtheorem{definition}{Definition} % Definition shares the counter with theorem
\newtheorem{example}{Example} % Example shares the counter with theorem
\newtheorem{exercise}{{Exercise}} % Example shares the counter with theorem
\newtheorem{remark}{Remark} % Remark shares the counter with theorem
\newtheorem*{note}{Note}

\title{Lecture Note: 1-Form as Scalar Projection onto a Line in \(\R^2\)}
\author{}
\date{}
\newcommand{\R}{\mathbb{R}}
\begin{document}
	\maketitle
	
	\section{Ambient Space and Tangent Spaces}
	Let
	\[
	\R^2 \;=\;\{\, (x,y) : x,y\in\R\}
	\]
	be equipped with the standard basis
	\[
	\mathbf{e}_1=(1,0),
	\quad
	\mathbf{e}_2=(0,1).
	\]
	At any point \(p\in\R^2\), the tangent space is canonically
	\[
	T_p\R^2 \;=\;\mathrm{span}\{\mathbf{e}_1,\mathbf{e}_2\}\;\cong\;\R^2.
	\]
	We write a general tangent vector at \(p\) as
	\[
	v \;=\; v^1\,\mathbf{e}_1 \;+\; v^2\,\mathbf{e}_2,
	\qquad
	v^1,v^2\in\R.
	\]
	
	\section{Definition of the Line and Its Unit Direction}
	Fix a nonzero vector
	\[
	\mathbf{w} \;=\;(w_1,w_2)\;\in\;\R^2,
	\qquad
	\mathbf{w}\neq(0,0).
	\]
	This determines the line
	\[
	L \;=\;\mathrm{span}\{\mathbf{w}\}
	\;\subset\;\R^2.
	\]
	Define the unit–direction vector
	\[
	\hat{\mathbf{w}}
	\;=\;
	\frac{\mathbf{w}}{\|\mathbf{w}\|}
	\;=\;
	\frac{1}{\sqrt{w_1^2+w_2^2}}\,(w_1,w_2),
	\]
	so that \(\|\hat{\mathbf{w}}\|=1\).
	
	\section{Scalar Projection Functional}
	\begin{definition}
		The \emph{scalar projection onto the line \(L\)} is the map
		\[
		\alpha\colon T_p\R^2\;\longrightarrow\;\R
		\]
		given by
		\[
		\alpha(v)
		\;=\;
		\bigl\langle \hat{\mathbf{w}},\,v\bigr\rangle
		\;=\;
		\hat w_1\,v^1 \;+\;\hat w_2\,v^2.
		\]
	\end{definition}
	
	\begin{lemma}
		\(\alpha\) is a linear functional on \(T_p\R^2\).  Hence it defines a differential
		\(1\)--form on \(\R^2\).
	\end{lemma}
	
	\begin{proof}
		For \(v,v'\in T_p\R^2\) and \(\lambda\in\R\),
		\[
		\alpha(v+v')
		=\bigl\langle\hat{\mathbf{w}},\,v+v'\bigr\rangle
		=\bigl\langle\hat{\mathbf{w}},\,v\bigr\rangle
		+\bigl\langle\hat{\mathbf{w}},\,v'\bigr\rangle
		=\alpha(v)+\alpha(v'),
		\]
		\[
		\alpha(\lambda\,v)
		=\bigl\langle\hat{\mathbf{w}},\,\lambda\,v\bigr\rangle
		=\lambda\,\bigl\langle\hat{\mathbf{w}},\,v\bigr\rangle
		=\lambda\,\alpha(v).
		\]
		Thus \(\alpha\in (T_p\R^2)^\ast\).
	\end{proof}
	
	\section{Expression in the Dual Basis}
	Recall that the dual basis \(\{dx,dy\}\) of \(\{\mathbf{e}_1,\mathbf{e}_2\}\) is defined by
	\[
	dx(\mathbf{e}_1)=1,\quad dx(\mathbf{e}_2)=0,
	\qquad
	dy(\mathbf{e}_1)=0,\quad dy(\mathbf{e}_2)=1.
	\]
	Writing \(v=v^1\mathbf{e}_1+v^2\mathbf{e}_2\), one has
	\[
	dx(v)=v^1,
	\quad
	dy(v)=v^2.
	\]
	Therefore
	\[
	\alpha(v)
	=\hat w_1\,v^1+\hat w_2\,v^2
	=\hat w_1\,dx(v)+\hat w_2\,dy(v),
	\]
	and as a \(1\)--form,
	\[
	\boxed{
		\alpha
		=\hat w_1\,dx \;+\;\hat w_2\,dy
		\;=\;
		\frac{w_1}{\sqrt{w_1^2+w_2^2}}\,dx
		\;+\;
		\frac{w_2}{\sqrt{w_1^2+w_2^2}}\,dy.
	}
	\]
	
	\section*{Interpretation}
	\begin{itemize}
		\item \(x,y\colon\R^2\to\R\) are the coordinate functions associated to the ambient basis \((\mathbf e_1,\mathbf e_2)\).
		\item \(dx,dy\colon T_p\R^2\to\R\) are the dual coordinate functions on each tangent space.
		\item The 1-form \(\alpha\) is precisely the functional which takes any tangent vector \(v\) and returns its signed scalar projection onto the fixed direction \(\hat{\mathbf w}\).
	\end{itemize}
	
\end{document}
