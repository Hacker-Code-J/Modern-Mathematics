\documentclass[12pt]{article}
\usepackage{amsmath,amssymb,amsthm}
\usepackage{geometry}
\geometry{margin=1in}
\usepackage{hyperref}
\newtheorem{theorem}{Theorem} % Theorem environment with section-wise numbering
\newtheorem*{theorem*}{Theorem} % Theorem environment with section-wise numbering
\newtheorem*{lemma*}{Lemma} % Theorem environment with section-wise numbering
\newtheorem*{proposition*}{Proposition} % Theorem environment with section-wise numbering
\newtheorem*{corollary*}{Corollary} % Theorem environment with section-wise numbering
\newtheorem{proposition}[theorem]{Proposition} % Theorem environment with section-wise numbering
\newtheorem{lemma}[theorem]{Lemma} % Lemma shares the counter with theorem
\newtheorem{corollary}[theorem]{Corollary} % Corollary shares the counter with theorem

\theoremstyle{definitionstyle}
\newtheorem{definition}{Definition} % Definition shares the counter with theorem
\newtheorem{example}{Example} % Example shares the counter with theorem
\newtheorem{exercise}{{Exercise}} % Example shares the counter with theorem
\newtheorem{remark}{Remark} % Remark shares the counter with theorem
\newtheorem*{note}{Note}
\title{Lecture Note: A 1-Form as Scalar Projection onto a Fixed Line}
\author{}
\date{}

\newcommand{\R}{\mathbb{R}}
\renewcommand{\span}{\mathrm{span}}
\begin{document}
	\maketitle
	
	\section{The Curve and Its Tangent Spaces}
	Let \(f\colon\mathbb{R}\to\mathbb{R}\) be a \(C^1\)‐function and set
	\[
	C \;=\;\bigl\{(x,y)\in\mathbb{R}^2\mid y=f(x)\bigr\}.
	\]
	Fix \(a\in\mathbb{R}\) and let
	\[
	p=(a,f(a))\in C.
	\]
	The inclusion \(C\hookrightarrow\mathbb{R}^2\) may be written as
	\[
	C\;\longrightarrow\;\mathbb{R}^2,
	\qquad
	p\;\mapsto\;\bigl(x(p),\,y(p)\bigr),
	\]
	where \(x,y\colon\mathbb{R}^2\to\mathbb{R}\) are the standard coordinate functions,
	\(\;x(x,y)=x,\;y(x,y)=y.\)
	
	The velocity of the parametrization \(t\mapsto(t,f(t))\) at \(t=a\) is
	\[
	\Phi'(a) \;=\;\bigl(1,\;f'(a)\bigr)\;\in\;T_pC.
	\]
	Thus
	\[
	T_pC \;=\;\span\bigl\{(1,f'(a))\bigr\}
	\;\subset\;T_p\mathbb{R}^2\cong\mathbb{R}^2.
	\]
	Each tangent vector \(v\in T_pC\) is uniquely
	\[
	v \;=\;\tau\,(1,f'(a)),
	\qquad
	\tau\in\mathbb{R}.
	\]
	We record the induced coordinate map on the fiber:
	\[
	T_pC\;\longrightarrow\;\mathbb{R}^2,\qquad
	v\;\mapsto\;\begin{pmatrix}
		dx(v)\\[6pt]
		dy(v)
	\end{pmatrix}.
	\]
	Since \(dx,dy\) are the dual basis on \(T_p\R^2\cong\R^2\), one has
	\[
	dx\bigl(1,f'(a)\bigr)=1,
	\quad
	dy\bigl(1,f'(a)\bigr)=f'(a).
	\]
	
	\section{A Fixed Line in the Plane}
	Choose a nonzero vector
	\[
	w=(w_1,w_2)\in\mathbb{R}^2,
	\quad
	\|w\|\neq0,
	\]
	and let \(L=\span\{w\}\subset\R^2\).  Define the unit direction
	\[
	\hat w=\frac{w}{\|w\|},
	\quad
	\|\hat w\|=1.
	\]
	
	\section{Definition of the 1-Form}
	\begin{definition}
		The \emph{scalar–projection 1-form} onto the line \(L\) is
		\[
		\alpha\;\colon\;T\R^2\;\longrightarrow\;\R,
		\qquad
		\alpha_p(v)=\bigl\langle\hat w\,,\,v\bigr\rangle,
		\]
		for each \(p\in\R^2\) and \(v\in T_p\R^2\).
	\end{definition}
	
	Since
	\(\hat w=(\hat w_1,\hat w_2)\) and \(v=(v^1,v^2)\), one has
	\[
	\alpha_p(v)
	=\hat w_1\,v^1 + \hat w_2\,v^2
	=\hat w_1\,dx(v)+\hat w_2\,dy(v).
	\]
	Hence, in the usual notation for 1-forms,
	\[
	\boxed{
		\alpha
		=\hat w_1\,dx\;+\;\hat w_2\,dy
		\;=\;
		\frac{w_1}{\sqrt{w_1^2+w_2^2}}\,dx
		\;+\;
		\frac{w_2}{\sqrt{w_1^2+w_2^2}}\,dy.
	}
	\]
	
	\section{Restriction to the Curve}
	Pulled back along the inclusion \(i\colon C\hookrightarrow\R^2\), the 1-form
	\(\alpha\) restricts to a well‐defined \(1\)-form on \(C\):
	\[
	i^*\alpha
	=\hat w_1\,dx\big|_C \;+\;\hat w_2\,dy\big|_C,
	\]
	which on \(T_pC\) evaluates by
	\[
	\bigl(i^*\alpha\bigr)_p\bigl(\tau(1,f'(a))\bigr)
	=\tau\,\bigl(\hat w_1 + \hat w_2\,f'(a)\bigr).
	\]
	
	\section*{Summary}
	\begin{itemize}
		\item The map \(p\mapsto(x(p),y(p))\) simply records the two ambient coordinates
		of each point in \(C\subset\R^2\).
		\item The map \(v\mapsto(dx(v),dy(v))\) records the two components of any tangent
		vector \(v\in T_p\R^2\).
		\item The 1-form \(\alpha=\hat w_1\,dx+\hat w_2\,dy\) is precisely the functional
		that takes each tangent vector and returns its scalar component in the fixed
		direction \(\hat w\).
	\end{itemize}
	
\end{document}
