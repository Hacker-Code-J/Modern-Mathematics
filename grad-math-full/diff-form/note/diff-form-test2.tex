\documentclass[12pt]{article}
\usepackage{amsmath,amssymb,amsthm}
\usepackage{geometry}
\geometry{margin=1in}
\usepackage{hyperref}

\numberwithin{equation}{section}

\title{Lecture Note: Coordinates and Differentials on a Plane Curve}
\author{}
\date{}

\begin{document}
	\maketitle
	
	\section{Setup}
	Let \(f\colon \mathbb{R}\to\mathbb{R}\) be a \(C^1\)--function and define the embedded curve
	\[
	C \;=\;\bigl\{(x,y)\in\mathbb{R}^2 \mid y = f(x)\bigr\}.
	\]
	Fix \(a\in\mathbb{R}\) and set
	\[
	p \;=\;(a,\;f(a)) \;\in\; C.
	\]
	
	\section{Coordinate System on \(C\)}
	\begin{enumerate}
		\item Define the global parametrization
		\[
		\Phi\colon \mathbb{R}\;\longrightarrow\;C,\quad
		\Phi(t)=\bigl(t,f(t)\bigr).
		\]
		\item Its inverse is
		\[
		\Phi^{-1}\colon C\;\longrightarrow\;\mathbb{R},
		\quad
		(x,y)\mapsto x.
		\]
		\item Thus \(t\) is a \emph{coordinate on \(C\)}, and every \(p\in C\) is uniquely
		\(p=\Phi(t)\).
	\end{enumerate}
	
	\section{Coordinate System on \(T_pC\)}
	\begin{enumerate}
		\item Differentiate \(\Phi\) at \(t=a\):
		\[
		d\Phi_{\,a}(1)
		=\frac{d}{dt}(t,f(t))\bigg|_{t=a}
		=(1,\;f'(a))
		=:\vec v\in T_pC.
		\]
		\item By definition,
		\[
		T_pC
		=\mathrm{span}\{\,(1,f'(a))\}
		\subset T_{p}\mathbb{R}^2.
		\]
		\item Every \(v\in T_pC\) writes \(v=\tau\,(1,f'(a))\) for a unique
		\(\tau\in\mathbb{R}\).  Hence \(\tau\) is a \emph{fiber coordinate} on \(T_pC\).
	\end{enumerate}
	
	\section{The Functions \(x,y\colon C\to\mathbb{R}\)}
	\begin{align*}
		x\;&=\;\pi_1\bigl|_C\colon C\to\mathbb{R},\quad (x,y)\mapsto x,\\
		y\;&=\;\pi_2\bigl|_C\colon C\to\mathbb{R},\quad (x,y)\mapsto y.
	\end{align*}
	Restricted to \(\Phi(t)\), these give
	\[
	x\bigl(\Phi(t)\bigr)=t,
	\quad
	y\bigl(\Phi(t)\bigr)=f(t).
	\]
	
	\section{The Differentials \(dx,dy\colon T_pC\to\mathbb{R}\)}
	\begin{enumerate}
		\item In the ambient \(\mathbb{R}^2\), the differentials act by
		\[
		dx(v_1,v_2)=v_1,
		\quad
		dy(v_1,v_2)=v_2.
		\]
		\item On the generator \(\vec v=(1,f'(a))\in T_pC\),
		\[
		dx(\vec v)=1,
		\quad
		dy(\vec v)=f'(a).
		\]
		\item Hence \(dx,dy\) extract the \(x\)– and \(y\)–components of any vector in \(T_pC\).
	\end{enumerate}
	
	\section*{Abstract Graduate–Level Synthesis}
	Let \(M=C\) be the \(1\)--dimensional submanifold of \(\mathbb{R}^2\) defined by \(y=f(x)\).
	The chart \(\Phi\colon\mathbb{R}\to M\) endows \(M\) with the coordinate \(t\), and its
	differential \(d\Phi\) trivializes the tangent bundle,
	\[
	d\Phi\colon T\mathbb{R}\cong\mathbb{R}\;\longrightarrow\;TM,
	\quad
	\tau\mapsto \tau\,\Phi'(t).
	\]
	Dually, the ambient projections restrict to
	\[
	dx,\,dy\colon TM\;\longrightarrow\;\mathbb{R},
	\]
	providing a coframe.  Thus \((t)\) on \(M\) and \((dx,dy)\) on \(TM\) constitute
	local frames that distinguish base points from tangent vectors, a paradigm
	that generalizes to all smooth manifolds.
\end{document}
