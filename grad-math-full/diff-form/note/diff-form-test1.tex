\documentclass[12pt]{article}
\usepackage{amsmath,amssymb,amsthm}
\usepackage{geometry}
\geometry{margin=1in}
\usepackage{hyperref}

\numberwithin{equation}{section}

\newtheorem{theorem}{Theorem} % Theorem environment with section-wise numbering
\newtheorem*{theorem*}{Theorem} % Theorem environment with section-wise numbering
\newtheorem*{lemma*}{Lemma} % Theorem environment with section-wise numbering
\newtheorem*{proposition*}{Proposition} % Theorem environment with section-wise numbering
\newtheorem*{corollary*}{Corollary} % Theorem environment with section-wise numbering
\newtheorem{proposition}[theorem]{Proposition} % Theorem environment with section-wise numbering
\newtheorem{lemma}[theorem]{Lemma} % Lemma shares the counter with theorem
\newtheorem{corollary}[theorem]{Corollary} % Corollary shares the counter with theorem

\theoremstyle{definitionstyle}
\newtheorem{definition}{Definition} % Definition shares the counter with theorem
\newtheorem{example}{Example} % Example shares the counter with theorem
\newtheorem{exercise}{{Exercise}} % Example shares the counter with theorem
\newtheorem{remark}{Remark} % Remark shares the counter with theorem
\newtheorem*{note}{Note}

\title{Coordinate and Tangent‐Space Charts for a Plane Curve}
\author{}
\date{}
\newcommand{\R}{\mathbb{R}}
\begin{document}
	\maketitle
	
	\section{The Curve as a $1$–Dimensional Submanifold}
	
	\begin{definition}
		Let $f\colon \R \to \R$ be a $C^1$–function.  Its \emph{graph}
		\[
		C \;=\;\bigl\{(x,y)\in\R^2 \,\bigm|\, y \;=\; f(x)\bigr\}
		\]
		is a $1$–dimensional embedded submanifold of $\R^2\,$.
	\end{definition}
	
	\noindent Fix $a\in\R$ and set 
	\[
	p \;=\;(a,f(a)) \;\in\; C.
	\]
	
	\section{Global Coordinate Chart on \(\mathbf{C}\)}
	
	\subsection*{6.1. Chart via a Parametrization}
	
	Define
	\[
	\Phi\;:\;\underbrace{\R}_{\displaystyle U}\;\longrightarrow\;C,
	\qquad
	\Phi(t)\;=\;\bigl(t,\,f(t)\bigr).
	\]
	\begin{proposition}
		$\Phi$ is a diffeomorphism from $U=\R$ onto $C$, with inverse
		\[
		\Phi^{-1}\;:\;C\;\longrightarrow\;\R,
		\qquad
		(x,y)\;\longmapsto\;x.
		\]
		Hence $t\in\R$ is a global coordinate on $C$, and every point $p\in C$ admits the unique representation $p=\Phi(t)$.
	\end{proposition}
	
	\begin{proof}
		$\Phi$ is $C^1$ with Jacobian determinant $1\neq0$, hence a local diffeo; injectivity and surjectivity are immediate from its formula.  The inverse $\Phi^{-1}(x,y)=x$ is $C^1$.
	\end{proof}
	
	\subsection*{6.2.\ Ambient Coordinate Restriction}
	
	Equivalently, let
	\[
	\pi_i\;:\;\R^2\;\longrightarrow\;\R,\quad
	\pi_1(x,y)=x,\;\pi_2(x,y)=y.
	\]
	Then the pair
	\[
	(x,y)\bigl|_C
	\;=\;
	(\pi_1,\pi_2)\bigl|_C
	\;:\;
	C
	\;\longrightarrow\;
	\R^2
	\]
	serves as the inclusion of $C$ into its ambient coordinate system, with
	\[
	(x,y)\bigl(t,f(t)\bigr)\;=\;(t,f(t)).
	\]
	
	\section{Coordinate Chart on the Tangent Spaces}
	
	\subsection*{7.1.\ The Tangent Bundle and Its Trivialization}
	
	The tangent bundle of $C$ may be presented via the push–forward
	\[
	d\Phi\;:\;
	TU\;=\;U\times\underbrace{\R}_{\text{model fiber}}
	\;\longrightarrow\;TC\subset T\R^2,
	\]
	where
	\[
	d\Phi_{\,t}(\dot t)
	\;=\;
	\frac{d}{dt}\Bigl(t,f(t)\Bigr)\Big|_{t}\,\dot t
	\;=\;
	\dot t\;\bigl(1,f'(t)\bigr)
	\;\in\;T_{\,\Phi(t)}C.
	\]
	Since $d\Phi_{\,t}$ is a vector‐space isomorphism $\R\to T_{\Phi(t)}C$, this gives a trivialization
	\[
	TC \;\cong\; U\times\R,
	\]
	and in particular, at $t=a$,
	\[
	d\Phi_{\,a}\;:\;
	\R\;\xrightarrow{\;\sim\;}\;T_{p}C,
	\qquad
	\tau\;\longmapsto\;\tau\,(1,f'(a)).
	\]
	Thus \(\tau\in\R\) is a \emph{local coordinate} on the fibre \(T_pC\).
	
	\subsection*{7.2.\ Cotangent‐Bundle Coordinates}
	
	Dually, the ambient projections $\pi_i$ induce
	\[
	d\pi_i\;:\;
	T_{p}\R^2\;\longrightarrow\;\R,
	\qquad
	(v_1,v_2)\;\longmapsto\;v_i.
	\]
	Restricting to the line $T_pC=\mathrm{span}\{\,(1,f'(a))\}\subset T_p\R^2$ yields a map
	\[
	(d\pi_1,d\pi_2)\bigl|_{T_pC}
	\;:\;
	T_pC
	\;\longrightarrow\;
	\underbrace{\R^2}_{\displaystyle \text{model cotangent fiber}},
	\]
	\[
	v\;\longmapsto\;\bigl(dx(v),\,dy(v)\bigr).
	\]
	Again this is injective onto the one‐dimensional subspace $\{(\tau,\;f'(a)\,\tau)\}\subset\R^2$, and projecting to the first factor recovers the scalar coordinate~$\tau$.
	
	\section{Coordinate Functions on \(\mathbf{C}\)}
	
	\[
	x\;=\;\pi_1\bigl|_C\;:\;C\longrightarrow\R,
	\qquad
	y\;=\;\pi_2\bigl|_C\;:\;C\longrightarrow\R.
	\]
	Explicitly,
	\[
	x\bigl(\Phi(t)\bigr)=t,
	\quad
	y\bigl(\Phi(t)\bigr)=f(t),
	\]
	and at $p=\Phi(a)$, $x(p)=a$ and $y(p)=f(a)$.
	
	\section{Differentials on the Tangent Space}
	
	\[
	dx\;=\;d\pi_1\bigl|_{T_p\R^2}\;:\;T_p\R^2\to\R,
	\quad
	dy\;=\;d\pi_2\bigl|_{T_p\R^2}\;:\;T_p\R^2\to\R,
	\]
	\[
	dx(v_1,v_2)=v_1,\quad dy(v_1,v_2)=v_2.
	\]
	Restricted to $T_pC=\mathrm{span}\{(1,f'(a))\}$ one has
	\[
	dx\bigl(1,f'(a)\bigr)=1,
	\qquad
	dy\bigl(1,f'(a)\bigr)=f'(a).
	\]
	In particular $dy/dx\big|_p=f'(a)$ as expected.
	
\end{document}
