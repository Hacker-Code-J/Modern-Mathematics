\documentclass[11pt,openany]{article}

\input{grad-math-preamble}
\usepackage{tcolorbox}
\tcbset{colback=white, arc=5pt}

\definecolor{axiomcolor}{HTML}{a88bfa}
\definecolor{defcolor}{RGB}{52, 152, 219}
\definecolor{procolor}{RGB}{241, 196, 15}
\definecolor{thmcolor}{RGB}{231, 76, 60}
\definecolor{lemcolor}{RGB}{155, 89, 182}
\definecolor{corcolor}{RGB}{46, 204, 113}
\definecolor{execolor}{RGB}{90, 128, 127}

% Define a new command for the custom tcolorbox
\newcommand{\axiombox}[2][]{%
	\begin{tcolorbox}[colframe=axiomcolor, title={\color{white}\bfseries #1}]
		#2
	\end{tcolorbox}
}

\newcommand{\defbox}[2][]{%
	\begin{tcolorbox}[colframe=defcolor, title={\color{white}\bfseries #1}]
		#2
	\end{tcolorbox}
}

\newcommand{\probox}[2][]{%
	\begin{tcolorbox}[colframe=procolor, title={\color{white}\bfseries #1}]
		#2
	\end{tcolorbox}
}

\newcommand{\thmbox}[2][]{%
	\begin{tcolorbox}[colframe=thmcolor, title={\color{white}\bfseries #1}]
		#2
	\end{tcolorbox}
}

\newcommand{\lembox}[2][]{%
	\begin{tcolorbox}[colframe=lemcolor, title={\color{white}\bfseries #1}]
		#2
	\end{tcolorbox}
}
\usepackage{amsthm}

% Define custom theorem styles
\newtheoremstyle{dotless} % Name of the style
{3pt} % Space above
{3pt} % Space below
{\itshape} % Body font
{} % Indent amount
{\bfseries} % Theorem head font
{} % Punctuation after theorem head
{2.5mm} % Space after theorem head
{} % Theorem head spec

\newtheoremstyle{definitionstyle} % Name of the style
{3pt} % Space above
{3pt} % Space below
{} % Body font
{} % Indent amount
{\bfseries} % Theorem head font
{.} % Punctuation after theorem head
{2.5mm} % Space after theorem head
{} % Theorem head spec

% Applying custom styles
%\theoremstyle{dotless}
\newtheorem{theorem}{Theorem} % Theorem environment with section-wise numbering
\newtheorem*{theorem*}{Theorem} % Theorem environment with section-wise numbering
\newtheorem*{lemma*}{Lemma} % Theorem environment with section-wise numbering
\newtheorem*{proposition*}{Proposition} % Theorem environment with section-wise numbering
\newtheorem*{corollary*}{Corollary} % Theorem environment with section-wise numbering
\newtheorem{proposition}[theorem]{Proposition} % Theorem environment with section-wise numbering
\newtheorem{lemma}[theorem]{Lemma} % Lemma shares the counter with theorem
\newtheorem{corollary}[theorem]{Corollary} % Corollary shares the counter with theorem

\theoremstyle{definitionstyle}
\newtheorem*{observation}{\textcolor{magenta}{Observation}}
\newtheorem*{illustration}{\textcolor{teal}{Illustration}}
\newtheorem*{torus}{{\color{red}T}{\color{orange}o}{\color{green!75!black}r}{\color{cyan}u}{\color{violet}s}}
\newtheorem{definition}{Definition} % Definition shares the counter with theorem
\newtheorem{example}{Example} % Example shares the counter with theorem
\newtheorem{exercise}{{Exercise}} % Example shares the counter with theorem
\newtheorem{remark}{Remark} % Remark shares the counter with theorem
\newtheorem*{note}{Note}
\newtheorem*{notation}{Notation}

\newtheorem*{axiom*}{Axiom}
\newtheorem*{definition*}{Definition} % Definition shares the counter with theorem
\newtheorem*{example*}{Example} % Example shares the counter with theorem
\newtheorem*{exercise*}{\textcolor{teal}{Exercise}} % Example shares the counter with theorem
\newtheorem*{remark*}{Remark} % Remark shares the counter with theorem


\usepackage{tikz}
\usepackage{tikz-cd}
\usetikzlibrary{shadows}
\usetikzlibrary{shapes.geometric, arrows.meta, positioning}
\newcommand{\ie}{\textnormal{i.e.}}
\newcommand{\rsa}{\mathsf{RSA}}
\newcommand{\rsacrt}{\mathsf{RSA}\textendash\mathsf{CRT}}
\newcommand{\inv}[1]{#1^{-1}}

%New Command
%\newcommand{\set}[1]{\left\{#1\right\}}
\newcommand{\N}{\mathbb{N}}
\newcommand{\Z}{\mathbb{Z}}
\newcommand{\Q}{\mathbb{Q}}
\newcommand{\R}{\mathbb{R}}
\newcommand{\cR}{\mathcal{R}}
\newcommand{\C}{\mathbb{C}}
\newcommand{\F}{\mathbb{F}}
\newcommand{\nbhd}{\mathcal{N}}
\newcommand{\Log}{\operatorname{Log}}
\newcommand{\Arg}{\operatorname{Arg}}
\newcommand{\pv}{\operatorname{P.V.}}

\newcommand{\of}[1]{\left( #1 \right)} 
%\newcommand{\abs}[1]{\left\lvert #1 \right\rvert}
%\newcommand{\norm}[1]{\left\| #1 \right\|}

\newcommand{\sol}{\textcolor{magenta}{\bf Sol}}
\newcommand{\conjugate}[1]{\overline{#1}}

\newcommand{\res}{\operatorname{res}}
\DeclareMathOperator*{\Res}{\operatorname{Res}}

%\renewcommand{\Re}{\operatorname{Re}}
%\renewcommand{\Im}{\operatorname{Im}}

\newcommand{\cyclic}[1]{\langle #1 \rangle}
\newcommand{\uniform}{\overset{\$}{\leftarrow}}
\newcommand{\xmark}{\textcolor{red}{\XSolidBrush}}
\newcommand{\vmark}{\textcolor{green!75!black}{\CheckmarkBold}}

\newcommand{\gen}[1]{\langle #1 \rangle}
\newcommand{\Gen}[1]{\left\langle #1 \right\rangle}

\newcommand{\img}[1]{\text{Img}(#1)}
\newcommand{\Img}[1]{\text{Img}\left(#1\right)}
\newcommand{\preimg}[1]{\text{Img}^{-1}(#1)}
\newcommand{\Preimg}[1]{\text{Img}^{-1}\left(#1\right)}

\newcommand{\relation}{\mathrel{\mathcal{R}}}
\newcommand{\injection}{\rightarrowtail}
\newcommand{\surjection}{\twoheadrightarrow}
\newcommand{\id}{\textnormal{id}}

\newcommand{\eqclass}[1]{\left[#1\right]}

% Define custom colors for O and X
\newcommand{\yes}{\textcolor{blue}{\bf \fullmoon}}
\newcommand{\no}{\textcolor{red}{\bf \texttimes}}

\DeclarePairedDelimiter\ceil{\lceil}{\rceil}
\DeclarePairedDelimiter\floor{\lfloor}{\rfloor}
%\renewcommand{\floor}[#1]{\lfloor #1\rfloor}
%\newcommand{\Floor}[#1]{\left\lfloor #1\right\rfloor}
%\newcommand{\ceil}[#1]{\lceil #1\rceil}
%\newcommand{\Ceil}[#1]{\left\lceil #1\right\rceil}

\newcommand{\topology}{\mathscr{T}}
\newcommand{\sequence}[1]{\langle #1\rangle}
\renewcommand{\vec}[1]{\mathbf{#1}}
\setstretch{1.25}

%\usepackage{background}
%\backgroundsetup{
%	scale=3,
%	color=gray!20,
%	opacity=0.3,
%	angle=45,
%	contents={\Huge \sffamily Ji, Yong-hyeon}
%}
\begin{document}
\pagenumbering{arabic}
\begin{center}
	\huge\textbf{Linear Algebra to Abstract Algebra}\\
	\vspace{0.5em}
	\large{Ji, Yong-hyeon}\\
%	\large{\ttfamily \url{https://github.com/Hacker-Code-J}}\\
	\vspace{0.5em}
	\normalsize{\today}\\
\end{center}

\noindent 
We cover the following topics in this note.
\begin{itemize}
	\item Subspace; Span
	\item Subgroup
	\item TBA
\end{itemize}
\hrule\vspace{12pt}
%\tableofcontents
%\newpage

\newpage
\begin{note}[span]
Let $V$ be a vector space over a field $\F$, and let $S\subseteq V$. Recall that, for $n\in\N$, 
\begin{align*}
\Span(S):&=\set{\lambda_1\vec{v}_1+\lambda_2\vec{v}_2+\cdots+\lambda_n\vec{v}_n\mid \lambda_i\in\F,\; \vec{v}_i\in S\; \text{for all}\; i=1,2,\dots,n}\\
&=\set{\sum_{i=1}^n\lambda_i\vec{v}_i\; \bigg|\; \lambda_i\in\F,\; \vec{v}_i\in S\; \text{for all}\; 1\leq i\leq n}.
\end{align*}
\end{note}\vfill
\defbox[(Vector) Subspace]{\begin{definition*}
	Let $V$ be a vector space over a field $\F$, and let $U\subseteq V$. We write $
	\boxed{U\leq V}$ if $V$ is a \textbf{(vector) subspace} of $V$. That is, $U\leq V$ if and only if $U$ satisfy the following conditions: \begin{enumerate}[(i)]
		\item $\vec{0}_V\in U$;
		\item $\forall \vec{u},\tilde{\vec{u}}\in U$,\; $\vec{u}+\tilde{\vec{u}}\in U$;
		\item $\forall \vec{u}\in U$,\; $\forall \lambda\in\F$,\; $\lambda\vec{u}\in U$.
	\end{enumerate}
\end{definition*}}
\begin{remark*}
If $S\subseteq V$, then $\Span(S)\leq V$.
\begin{proof}
We must verify that $\Span(S)$ satisfies the three defining properties of a subspace of $V$: \begin{enumerate}[(i)]
	\item If $S=\varnothing$, by convention we define $\Span(\varnothing):=\set{\vec{0}_V}$. Let $S\neq\varnothing$. Choose any $\vec{v}\in S(\subseteq V)$ and take $n=1$ with the scalar $\lambda_1=0\in\F$. Then $\vec{0}_V=0\cdot\vec{v}\in\Span(S)$.
	\item Let $\vec{u},\tilde{\vec{u}}\in\Span(S)$, say, \[
	\vec{u}=\sum_{i=1}^{n}\lambda_i\vec{v}_i\quad\text{and}\quad\tilde{\vec{u}}=\sum_{j=1}^{m}\mu_j\tilde{\vec{v}}_j,
	\] where $n,m\in\N$, $\lambda_i,\mu_j\in\F$, and $\vec{v}_i,\tilde{\vec{v}}_j\in S$ for all indices $i,j$. Then \[
	\vec{u}+\tilde{\vec{u}}=\sum_{i=1}^{n}\lambda_i\vec{v}_i+\sum_{j=1}^{m}\mu_j\tilde{\vec{v}}_j=\overbrace{\underbrace{\lambda_1\vec{v}_1+\lambda_2\vec{v}_2+\cdots+\lambda_n\vec{v}_n}_{\text{$n$ terms}}+\underbrace{\mu_1\tilde{\vec{v}}_1+\mu_2\tilde{\vec{v}}_2+\cdots+\mu_m\tilde{\vec{v}}_m}_{\text{$m$ terms}}}^{\text{$n+m$ terms}}\in\Span(S).
	\]
	\item Let $\alpha\in\F$. Let $\vec{u}\in\Span(S)$, say, $\displaystyle\vec{u}=\sum_{i=1}^{n}\lambda_i\vec{v}_i,$  where $n\in\N$, $\lambda_i,\in\F$, and $\vec{v}_i\in S$ for each $1\leq i\leq n$. Then \[
	\alpha\vec{u}=\alpha\left(\sum_{i=1}^{n}\lambda_i\vec{v}_i\right)=\sum_{i=1}^{n}(\alpha\lambda_i)\vec{v}_i\overset{}{\in}\Span(S).
	\] since $\alpha\lambda_i\in\F$ for all $i=1,2,\dots,n$.
\end{enumerate}
\end{proof}
\end{remark*}

\newpage
\probox{\begin{proposition*}
Let $V$ be a vector space over a field $\F$, and let 
$S\subseteq V$. Then \begin{enumerate}[(1)]
	\item $S\subseteq\Span(S)\subseteq V$.
	\item If $U\leq V$ is any subspace of $V$ such that $S\subseteq U$, then $\Span(S)\subseteq U$.
\end{enumerate}
\end{proposition*}}
\begin{proof}
\ \begin{enumerate}[(1)]
	\item Let \( \vec{s} \in S \). Then, choosing \( n = 1 \) and \(\lambda_1 = 1 \in \F\), we have
		$\vec{s} = 1 \cdot \vec{s} \in \Span(S).$ Each element \( \vec{s} \in \Span(S) \) is of the form
		\[
		\vec{s} = \sum_{i=1}^n \lambda_i \vec{v}_i,
		\]
		where \( \vec{v}_i \in S \subseteq V \) and \( \lambda_i \in \F \). Since \( V \) is a vector space and is closed under finite linear combinations, it follows that $
		\vec{s} \in V.$
	\item Let \( U \le V \) and $S\subseteq U$. Let \( \vec{s} \in \Span(S) \). Then, there exist \( n \in \N \), scalars \( \lambda_1,\lambda_2, \dots, \lambda_n \in \F \), and vectors \( \vec{v}_1, \vec{v}_2 \dots, \vec{v}_n \in S\subseteq V \) such that
	\[
	\vec{s} = \sum_{i=1}^n \lambda_i \vec{v}_i\in\Span(S).
	\] Since \begin{itemize}
		\item $S\subseteq U$, \ie, $\vec{v}_i\in S\subseteq U$ for each $i=1,2,\dots,n$, and
		\item $U\leq V$, \ie, $\vec{u}+\tilde{\vec{u}}\in U$ and $\alpha\vec{u}\in U$ for any $\vec{u},\tilde{\vec{u}}\in U$,\; $\alpha\in\F$,
	\end{itemize} it follows that \[
\forall i\in\set{1,2,\dots,n},\; \lambda_i\vec{v}_i\in U\quad\text{and}\quad\vec{s}=\sum_{i=1}^n \lambda_i \vec{v}_i\in U.
	\]
\end{enumerate}
\end{proof}
\newpage
\probox{\begin{proposition*}
Let $V$ be a vector space over a field $\F$, and let $S\subseteq V$. Let $\mathcal{U}:=\set{U\leq V:S\subseteq U}.$ Then \[
\Span(S)=\bigcap_{U\in\mathcal{U}}U.
\] In other words, $\Span(S)$ is the smallest subspace of $V$ containing $S$.
\end{proposition*}}
\begin{proof}
We want to show that $\Span(S)=\bigcap_{U\in\mathcal{U}}U$.
\begin{itemize}
	\item[$(\subseteq)$] Let $\vec{u}\in\Span(S)$. By definition, there exists $n\in\N$, scalars $\lambda_1,\lambda_2,\dots,\lambda_n\in\F$, and vectors $\vec{v}_1,\vec{v}_2,\dots,\vec{v}_n\in S$ such that \[
	\vec{u}=\sum_{i=1}^n\lambda_i\vec{v}_i.
	\] Let $U\in\mathcal{U}$ be arbitrary. Since $S\subseteq U$ and $U\leq V$, it is closed under finite linear combinations: \[
	\sum_{i=1}^n\lambda_i\vec{v}_i\in U.
	\] Since $\forall U\in\mathcal{U},\; \vec{u}\in U\Leftrightarrow \vec{u}\in\bigcap_{U\in\mathcal{U}} U$, we obtain \[
	\vec{u}=\sum_{i=1}^n\lambda_i\vec{v}_i\in\bigcap_{U\in\mathcal{U}}U.
	\]
	\item[$(\supseteq)$] Since $S\subseteq\Span(S)$ and $\Span(S)\leq V$, we know $\Span(S)\in\mathcal{U}$. Let $\vec{u}\in\bigcap_{U\in\mathcal{U}}U$. Then \[
	\vec{u}\in\bigcap_{U\in\mathcal{U}}U\iff \forall U\in\mathcal{U},\; \vec{u}\in U\implies \vec{u}\in\Span(S).
	\]
\end{itemize} Hence, we conclude that $\Span(S)=\bigcap_{U\in\mathcal{U}}U$.
\end{proof}
\vfill
\defbox[Subgroup]{\begin{definition*}
	Let $G$ be a group. Let $H\subseteq G$. We say that $H$ is a \textbf{subgroup} of $G$, denoted by $H\leq G$, if and only if $H$ is itself a group \textcolor{gray!50}{(with the operation inherited from G)}.
\end{definition*}}
\begin{example*}
\ \begin{itemize}
	\item $(\Q,+)\leq(\R,+)$.
	\item $(\Q^\times,\times)\leq(\R^\times,\times)$.
\end{itemize}
\end{example*}

\probox[Subgroup Test]{\begin{proposition*}
Let $G$ be a group, and let $H\subseteq G$ with $H\neq\varnothing$. \begin{enumerate}[(1)]
	\item (2-step Test) \[
	H\leq G\iff \left(x,y\in H\implies xy\in H,\; x^{-1}\in H\right)
	\]
	\item (1-step Test) \[
	H\leq G\iff \left(x,y\in H\implies xy^{-1}\in H\right)
	\]
\end{enumerate}
\end{proposition*}}
\begin{proof}
We want to show that \[
\underbrace{H\leq G}_{\text{(a)}}\iff \underbrace{\left(x,y\in H\implies xy\in H,\; x^{-1}\in H\right)}_{\text{(b)}}\iff
\underbrace{\left(x,y\in H\implies xy^{-1}\in H\right)}_{\text{(c)}}
\] \vfill
\begin{itemize}
	\item[] $\left(\text{(a)}\Rightarrow\text{(b)}\right)$\; Let $H\leq G$. Let $x,y\in H$. Since every subgroup is closed under the group operation and taking inverses, we have \[
	xy\in H\quad\text{and}\quad x^{-1}\in H.
	\]\vfill
	\item[] $\left(\text{(b)}\Rightarrow\text{(c)}\right)$\; Let $x,y\in H$. Suppose that $xy\in H$ and $x^{-1}\in H.$ Clearly, $xy^{-1}\in H$.\vfill
	\item[] $\left(\text{(c)}\Rightarrow\text{(a)}\right)$\; Let $x,y\in H$. Suppose that \[
	xy^{-1}\in H.
	\] Since $H\neq\varnothing$, $\exists a\in H$, and so \[
	aa^{-1}\in H\implies e\in H.
	\] Since $x\in H$ and $e\in H$, we have \[
	ex^{-1}\in H\implies x^{-1}\in H.
	\] Then, since $x,y\in H$ and $y^{-1}\in H$, we obtain \[
	x(y^{-1})^{-1}\in H\implies xy\in H,
	\] \ie, $H$ is closed under binary operation on $G$.
\end{itemize}
\end{proof}

\newpage
\defbox[Subgroup Generated by $S$]{\begin{definition*}
	Let \( G \) be a group, and let \( S \subseteq G \).  
	The \textbf{subgroup of \( G \) generated by \( S \)}, denoted by
	$
	\langle S \rangle,$
	is defined as:
	\[
	\langle S \rangle \coloneqq \bigcap \{ H \le G : S \subseteq H \}=\bigcap_{S\subseteq H\leq G}H.
	\]
\end{definition*}}
\begin{exercise*}
Let $G$ be a group, and let $S\subseteq G$. Show that \(\langle S \rangle\) is the unique smallest subgroup of \( G \) containing \( S \).
\end{exercise*}
\begin{proof}[\sol]
	TBA
\end{proof}
\begin{exercise*}
Let $G$ be a group, and let $S\subseteq G$. Let $H_i\leq G$ for each $i\in I$. Show that \[
\bigcap_{i\in I} H_i\leq G.
\]
\end{exercise*}
\begin{proof}[\sol]
TBA
\end{proof}

\newpage
\probox[]{\begin{proposition*}
Let $(G,+)$ be an abelian group with identity $0_G$, and let $x,y\in G$. Then \begin{enumerate}[(1)]
	\item $\gen{x}=\set{nx:n\in\Z}$
	\item $\gen{x,y}=\set{nx+my:n,m\in\Z}$
\end{enumerate}
\end{proposition*}}
%\begin{remark*}
%	The notation $nx$ is defined to represent the $n$-fold sum of $x$ with itself when $n\geq 0$. We define the function \[
%	\fullfunction{f}{\textcolor{blue}{\Z}\times \textcolor{red}{G}}{\textcolor{red}{G}}{(n,x)}{nx}
%	\] by the following recursive rules: \begin{enumerate}
%		\item Base Case: $0x\coloneq0_G$.
%		\item Positive Integer: For $n\in\Z$ with $n\geq 1$, \[
%		nx\coloneq x+\underbrace{x+x+\cdots+x}_{\text{$n+(-1)=n-1$ times}},\quad\ie,\;\text{by induction}\;,\quad nx=x+(n-1)x.
%		\]
%		\item Negative Integer: For $n\in\Z$ with $n<0$, define \[
%		nx\coloneq \textcolor{red}{-}((\textcolor{blue}{-}n)x)
%		\]
%	\end{enumerate}
%	Here, for any $n\in Z$ with $n>0$, we have $(-n)x=x+(-n-1)x=2x+(-n-2)x= -(nx)$.
%\end{remark*}
\begin{proof}
	TBA
%\begin{enumerate}[(1)]
%	\item Define
%	\[
%	H_1 \coloneqq \{\, nx : n\in\mathbb{Z}\,\}.
%	\] We NTS that \(H_1\) is a subgroup of \(G\) and that \(H_1\) is the smallest subgroup containing \(x\).
%	\begin{enumerate}
%		\item (\(H_1\) is a subgroup of \(G\))
%		
%		Since $0\in\Z$ and $0x=0_G$, we have $0_G\in H_1$, \ie, $H_1\neq\varnothing$ Let $a,b\in H_1$. Then there exists $n,m\in\Z$ such that \[
%		a=nx\quad\text{and}\quad b=mx.
%		\] Then \[
%		a+(-b)=a+(-(mx)),
%		\] and since $n+(-m)\in\Z$, we have $a+(-b)\in H_1$.
%	\end{enumerate}
%	\item 
%\end{enumerate}
%
%(a) *\(H_1\) is a subgroup of \(G\):*  
%We verify the subgroup criteria:
%
%- **Nonemptiness:**  
%Since \(0\in\mathbb{Z}\) and \(0\cdot x = 0_G\), it follows that
%\[
%0_G\in H_1.
%\]
%
%- **Closure under Addition:**  
%Let \(n,m\in\mathbb{Z}\). Then
%\[
%n\cdot x + m\cdot x = (n+m)\cdot x \in H_1,
%\]
%because \(n+m\in\mathbb{Z}\).
%
%- **Closure under Inverses:**  
%For \(n\in\mathbb{Z}\), the additive inverse of \(n\cdot x\) is given by
%\[
%-(n\cdot x) = (-n)\cdot x \in H_1,
%\]
%since \(-n\in\mathbb{Z}\).
%
%Thus, \(H_1\le G\).
%
%(b) *\(H_1\) is the smallest subgroup containing \(x\):*  
%Let \(K\le G\) be any subgroup with \(x\in K\). Then, by the closure properties of \(K\),
%\[
%(\forall\, n\in\mathbb{Z})\quad n\cdot x \in K.
%\]
%Hence,
%\[
%H_1\subseteq K.
%\]
%By the definition of the subgroup generated by \(x\),
%\[
%\langle x\rangle = \bigcap\{ K\le G : x\in K\},
%\]
%it follows that
%\[
%\langle x\rangle \subseteq H_1.
%\]
%Conversely, since \(x\in H_1\) and \(H_1\) is a subgroup, we have
%\[
%\langle x\rangle \supseteq H_1.
%\]
%Therefore,
%\[
%\langle x\rangle = H_1 = \{\, n\cdot x : n\in\mathbb{Z}\,\}.
%\]
%
%2. **(Subgroup Generated by \(x\) and \(y\))**
%
%Define
%\[
%H_2 \coloneqq \{\, n\cdot x + m\cdot y : n,m\in\mathbb{Z}\,\}.
%\]
%
%We show that \(H_2\) is a subgroup of \(G\) and that \(H_2\) is the smallest subgroup containing both \(x\) and \(y\).
%
%(a) *\(H_2\) is a subgroup of \(G\):*  
%We check the subgroup criteria:
%
%- **Nonemptiness:**  
%With \(n=m=0\) we have
%\[
%0\cdot x + 0\cdot y = 0_G \in H_2.
%\]
%
%- **Closure under Addition:**  
%Let \(u,v\in H_2\) so that
%\[
%u = n\cdot x + m\cdot y \quad\text{and}\quad v = n'\cdot x + m'\cdot y,
%\]
%with \(n,m,n',m'\in\mathbb{Z}\). Then,
%\[
%u+v = (n+n')\cdot x + (m+m')\cdot y \in H_2,
%\]
%since \(n+n',\, m+m'\in\mathbb{Z}\).
%
%- **Closure under Inverses:**  
%For \(u = n\cdot x + m\cdot y \in H_2\),
%\[
%-u = (-n)\cdot x + (-m)\cdot y \in H_2,
%\]
%because \(-n,-m\in\mathbb{Z}\).
%
%Hence, \(H_2\le G\).
%
%(b) *\(H_2\) is the smallest subgroup containing \(x\) and \(y\):*  
%Let \(K\le G\) be any subgroup with \(x,y\in K\). Then by closure under the group operation,
%\[
%(\forall\, n,m\in\mathbb{Z})\quad n\cdot x + m\cdot y \in K.
%\]
%Therefore,
%\[
%H_2 \subseteq K.
%\]
%Since
%\[
%\langle x,y\rangle = \bigcap\{ K\le G : \{x,y\}\subseteq K\},
%\]
%it follows that
%\[
%\langle x,y\rangle \subseteq H_2.
%\]
%Conversely, as \(x,y\in H_2\) and \(H_2\) is a subgroup, we deduce that
%\[
%\langle x,y\rangle \supseteq H_2.
%\]
%Hence,
%\[
%\langle x,y\rangle = H_2 = \{\, n\cdot x + m\cdot y : n,m\in\mathbb{Z}\,\}.
%\]
%
%---
%
%**Conclusion.**  
%We have shown that if \((G,+)\) is an abelian group and \(x,y\in G\), then
%\[
%\langle x\rangle = \{\, n\cdot x : n\in\mathbb{Z}\,\} \quad\text{and}\quad \langle x,y\rangle = \{\, n\cdot x + m\cdot y : n,m\in\mathbb{Z}\,\}.
%\]
%This completes the formal proof.
%
%\[
%\quad\square
%\]
\end{proof}
%\newpage
%

\vfill
\begin{thebibliography}{9}
	\bibitem{linear_to_abstract_a}
	수학의 즐거움, Enjoying Math. ``수학 공부, 기초부터 대학원 수학까지, 18. 선형대수학에서 추상대수학으로 (a) 선형결합의 추상화'' YouTube Video, 24:25. Published 
	October 15, 2019. URL: \url{https://www.youtube.com/watch?v=zg63xXZYNM8&t=598s}.
	\bibitem{linear_to_abstract_b}
	수학의 즐거움, Enjoying Math. ``수학 공부, 기초부터 대학원 수학까지, 19. 선형대수학에서 추상대수학으로 (b) 대수적 구조를 보존하는 함수 algebraic homomorphisms'' YouTube Video, 25:21. Published 
	October 16, 2019. URL: \url{https://www.youtube.com/watch?v=9TtGaY5COlg&t=187s}.
\end{thebibliography}

\end{document}
