\documentclass[11pt,openany]{article}

\input{crypto-preamble}
\usepackage{tcolorbox}
\tcbset{colback=white, arc=5pt}

\definecolor{axiomcolor}{HTML}{a88bfa}
\definecolor{defcolor}{RGB}{52, 152, 219}
\definecolor{procolor}{RGB}{241, 196, 15}
\definecolor{thmcolor}{RGB}{231, 76, 60}
\definecolor{lemcolor}{RGB}{155, 89, 182}
\definecolor{corcolor}{RGB}{46, 204, 113}
\definecolor{execolor}{RGB}{90, 128, 127}

% Define a new command for the custom tcolorbox
\newcommand{\axiombox}[2][]{%
	\begin{tcolorbox}[colframe=axiomcolor, title={\color{white}\bfseries #1}]
		#2
	\end{tcolorbox}
}

\newcommand{\defbox}[2][]{%
	\begin{tcolorbox}[colframe=defcolor, title={\color{white}\bfseries #1}]
		#2
	\end{tcolorbox}
}

\newcommand{\probox}[2][]{%
	\begin{tcolorbox}[colframe=procolor, title={\color{white}\bfseries #1}]
		#2
	\end{tcolorbox}
}

\newcommand{\thmbox}[2][]{%
	\begin{tcolorbox}[colframe=thmcolor, title={\color{white}\bfseries #1}]
		#2
	\end{tcolorbox}
}

\newcommand{\lembox}[2][]{%
	\begin{tcolorbox}[colframe=lemcolor, title={\color{white}\bfseries #1}]
		#2
	\end{tcolorbox}
}
\usepackage{amsthm}

% Define custom theorem styles
\newtheoremstyle{dotless} % Name of the style
{3pt} % Space above
{3pt} % Space below
{\itshape} % Body font
{} % Indent amount
{\bfseries} % Theorem head font
{} % Punctuation after theorem head
{2.5mm} % Space after theorem head
{} % Theorem head spec

\newtheoremstyle{definitionstyle} % Name of the style
{3pt} % Space above
{3pt} % Space below
{} % Body font
{} % Indent amount
{\bfseries} % Theorem head font
{.} % Punctuation after theorem head
{2.5mm} % Space after theorem head
{} % Theorem head spec

% Applying custom styles
%\theoremstyle{dotless}
\newtheorem{theorem}{Theorem} % Theorem environment with section-wise numbering
\newtheorem*{theorem*}{Theorem} % Theorem environment with section-wise numbering
\newtheorem*{lemma*}{Lemma} % Theorem environment with section-wise numbering
\newtheorem*{proposition*}{Proposition} % Theorem environment with section-wise numbering
\newtheorem*{corollary*}{Corollary} % Theorem environment with section-wise numbering
\newtheorem{proposition}[theorem]{Proposition} % Theorem environment with section-wise numbering
\newtheorem{lemma}[theorem]{Lemma} % Lemma shares the counter with theorem
\newtheorem{corollary}[theorem]{Corollary} % Corollary shares the counter with theorem

\theoremstyle{definitionstyle}
\newtheorem*{observation}{\textcolor{magenta}{Observation}}
\newtheorem*{illustration}{\textcolor{teal}{Illustration}}
\newtheorem*{torus}{{\color{red}T}{\color{orange}o}{\color{green!75!black}r}{\color{cyan}u}{\color{violet}s}}
\newtheorem{definition}{Definition} % Definition shares the counter with theorem
\newtheorem{example}{Example} % Example shares the counter with theorem
\newtheorem{exercise}{{Exercise}} % Example shares the counter with theorem
\newtheorem{remark}{Remark} % Remark shares the counter with theorem
\newtheorem*{note}{Note}
\newtheorem*{notation}{Notation}

\newtheorem*{axiom*}{Axiom}
\newtheorem*{definition*}{Definition} % Definition shares the counter with theorem
\newtheorem*{example*}{Example} % Example shares the counter with theorem
\newtheorem*{exercise*}{\textcolor{teal}{Exercise}} % Example shares the counter with theorem
\newtheorem*{remark*}{Remark} % Remark shares the counter with theorem


\usepackage{tikz}
\usepackage{tikz-cd}
\usetikzlibrary{shadows}
\usetikzlibrary{shapes.geometric, arrows.meta, positioning}
\input{crypto-commands}
\renewcommand{\vec}[1]{\mathbf{#1}}
\renewcommand{\Re}{\operatorname*{Re}}
\renewcommand{\Im}{\operatorname*{Im}}
\newcommand{\rank}{\mathrm{rank}}
%\newcommand{\Mat}{\operatorname{Mat}}

\newcommand{\Sym}{\mathrm{Sym}}

\setstretch{1.25}

\begin{document}
\pagenumbering{arabic}
\begin{center}
	\huge\textbf{Introduction to Commutative Algebra}\\
	\vspace{0.5em}
	\large{Ji, Yong-hyeon}\\
	\vspace{0.5em}
	\normalsize{\today}\\
\end{center}

\noindent 
We cover the following topics in this note.
\begin{itemize}
	\item Boolean Ring
\end{itemize}

\newpage
\begin{tcolorbox}
\begin{proposition}
	Let $A$ be a (commutative) ring with identity $1_A$ such that
	\[
	\forall x \in A,\quad x^2 = x.
	\]
	Then:
	\begin{enumerate}[(1)]
		\item $2x = 0$ for all $x\in A$ (i.e.,\ $\operatorname{char}(A) = 2$).
		\item Every prime ideal $\mathfrak{p} \subseteq A$ is maximal, and $A/\mathfrak{p}$ is a field with two elements.
		\item Every finitely generated ideal of $A$ is principal.
	\end{enumerate}
\end{proposition}
\end{tcolorbox}
\begin{proof}
	Let $x^2 = x$ for all $x$ in a (commutative) ring $A$ with identity $1_A$.
	\begin{enumerate}[(1)]
		\item Let $x \in A$ be arbitrary. Consider the element $x + 1_A \in A$. By the Boolean property,
		\[
		(x + 1_A)^2 = x + 1_A.
		\]
		On the other hand, by distributivity and the fact that $1_A$ is the multiplicative identity,
		\begin{align*}
			(x + 1_A)^2
			&= x^2 + x \cdot 1_A + 1_A \cdot x + 1_A^2 \\
			&= x^2 + x + x + 1_A \\
			&= x^2 + 2x + 1_A.
		\end{align*}
		Thus we have the equality
		\[
		x^2 + 2x + 1_A = x + 1_A.
		\]
		Using $x^2 = x$, we substitute:
		\[
		x + 2x + 1_A = x + 1_A.
		\]
		Subtracting $x + 1_A$ from both sides (i.e.\ adding the additive inverse of $x+1_A$),
		\[
		(x + 2x + 1_A) - (x + 1_A) = 0,
		\]
		hence
		\[
		2x = 0.
		\]
		Since $x \in A$ was arbitrary, we obtain
		\[
		\forall x \in A,\quad 2x = 0.
		\]
		In particular, the characteristic of $A$ is $2$.
		\item Let $\mathfrak{p} \subseteq A$ be a prime ideal. By definition of primality, the quotient ring $A/\mathfrak{p}$ is an integral domain.
		
		Consider the canonical surjection
		\[
		\pi : A \to A/\mathfrak{p},\quad x \mapsto \overline{x}.
		\]
		For any $x \in A$, we have $x^2 = x$, hence applying $\pi$ and using that $\pi$ is a ring homomorphism,
		\[
		\overline{x}^2 = \overline{x^2} = \overline{x}.
		\]
		Thus every element $\overline{x} \in A/\mathfrak{p}$ is idempotent:
		\[
		\forall y \in A/\mathfrak{p},\quad y^2 = y.
		\]
		
		Now let $y \in A/\mathfrak{p}$ be arbitrary. Then
		\[
		y^2 = y \quad\Longrightarrow\quad y^2 - y = 0.
		\]
		Hence
		\[
		y(y - 1_{A/\mathfrak{p}}) = 0.
		\]
		Since $A/\mathfrak{p}$ is an integral domain and $0$ is the only zero divisor, it follows that
		\[
		y = 0 \quad\text{or}\quad y = 1_{A/\mathfrak{p}}.
		\]
		Therefore every element of $A/\mathfrak{p}$ is either $0$ or $1_{A/\mathfrak{p}}$, so the underlying set of $A/\mathfrak{p}$ has at most two elements.
		
		Because $\mathfrak{p}$ is a proper ideal, $A/\mathfrak{p} \neq 0$, hence $0 \neq 1_{A/\mathfrak{p}}$ and there are \emph{exactly} two elements:
		\[
		A/\mathfrak{p} = \{\,0, 1_{A/\mathfrak{p}}\,\}.
		\]
		In particular, $A/\mathfrak{p}$ is a finite integral domain. It is a standard fact that every finite integral domain is a field: indeed every nonzero element has a multiplicative inverse. Here the only nonzero element is $1_{A/\mathfrak{p}}$, and its inverse is itself:
		\[
		1_{A/\mathfrak{p}} \cdot 1_{A/\mathfrak{p}} = 1_{A/\mathfrak{p}}.
		\]
		Hence $A/\mathfrak{p}$ is a field with exactly two elements, which is (up to isomorphism) the field $\mathbb{F}_2$.
		
		By the general correspondence between prime (resp.\ maximal) ideals and integral domains (resp.\ fields) of the form $A/\mathfrak{a}$, the fact that $A/\mathfrak{p}$ is a field implies that $\mathfrak{p}$ is maximal.
		\item 	Let $\mathfrak{a} \subseteq A$ be a finitely generated ideal. Then there exist $a_1, \dots, a_n \in A$ such that
		\[
		\mathfrak{a} = (a_1, \dots, a_n),
		\]
		the ideal generated by $a_1,\dots,a_n$.
		
		We show by induction on $n \ge 1$ that any ideal generated by $n$ elements is principal.
		
		\smallskip
		
		\emph{Base case $n = 1$.}  
		If $\mathfrak{a} = (a_1)$, then $\mathfrak{a}$ is principal by definition.
		
		\smallskip
		
		\emph{Induction step.}  
		Assume that any ideal generated by $n$ elements is principal. Let
		\[
		\mathfrak{b} = (a_1, \dots, a_n, a_{n+1})
		\]
		be an ideal generated by $n+1$ elements. By the induction hypothesis, the ideal
		\[
		\mathfrak{c} = (a_1, \dots, a_n)
		\]
		is principal, say $\mathfrak{c} = (e)$ for some $e \in A$.
		
		Then
		\[
		\mathfrak{b} = (a_1, \dots, a_n, a_{n+1}) = (\mathfrak{c}, a_{n+1}) = (e, a_{n+1}).
		\]
		We now show that for any $a,b \in A$, the ideal $(a,b)$ is principal. Setting $a = e$ and $b = a_{n+1}$ will then give that $\mathfrak{b}$ is principal, closing the induction.
		
		\smallskip
		
		\emph{Claim.} For any $a,b \in A$, the ideal $(a,b)$ is equal to the principal ideal generated by
		\[
		c := a + b + ab.
		\]
		
		\emph{Proof of the claim.} Let $a,b \in A$ and define $c = a + b + ab \in A$.
		
		First, note that
		\[
		c = a + b + ab \in (a,b)
		\]
		since $(a,b)$ is an ideal and contains $a$, $b$, and $ab$. Hence
		\[
		(c) \subseteq (a,b).
		\]
		
		Conversely, we show that $a,b \in (c)$; then $(a,b) \subseteq (c)$ will follow from the definition of $(a,b)$ as the smallest ideal containing $a$ and $b$.
		
		Compute
		\begin{align*}
			ca &= (a + b + ab)a \\
			&= a^2 + ba + aba.
		\end{align*}
		Since $A$ is commutative and Boolean, we have $a^2 = a$ and $ba = ab$, $aba = a^2 b = ab$. Hence
		\[
		ca = a + ab + ab = a + 2ab.
		\]
		By part (i), $\operatorname{char}(A) = 2$, so $2ab = 0$. Therefore
		\[
		ca = a.
		\]
		Thus $a = ca \in (c)$.
		
		Similarly,
		\begin{align*}
			cb &= (a + b + ab)b \\
			&= ab + b^2 + ab^2.
		\end{align*}
		Again using commutativity and idempotence, $b^2 = b$ and $ab^2 = ab$, hence
		\[
		cb = ab + b + ab = b + 2ab = b,
		\]
		and as before $2ab = 0$ implies $cb = b$. Thus $b = cb \in (c)$.
		
		Since $a,b \in (c)$, we have
		\[
		(a,b) \subseteq (c).
		\]
		Together with $(c) \subseteq (a,b)$, this implies
		\[
		(a,b) = (c) = (a + b + ab),
		\]
		as claimed.
		
		\smallskip
		
		Returning to the induction step, apply the claim with $a = e$ and $b = a_{n+1}$ to conclude
		\[
		\mathfrak{b} = (e, a_{n+1}) = (e + a_{n+1} + e a_{n+1}),
		\]
		which is a principal ideal. This completes the induction.
		
		Therefore every finitely generated ideal of $A$ is principal.
		
		\smallskip
		
		Combining (i), (ii), and (iii), the proposition is proved.
	\end{enumerate}
	\noindent\textbf{(ii) If $\mathfrak{p}$ is prime, then $\mathfrak{p}$ is maximal and $A/\mathfrak{p}$ is a field with two elements.}
	
	\smallskip
	
	\noindent\textbf{(iii) Every finitely generated ideal in $A$ is principal.}
\end{proof}

\newpage
\section{Boolean Rings as $\mathbb{F}_2$--Vector Spaces}

\begin{definition}
	A (commutative) ring $A$ is called a \emph{Boolean ring} if 
	\[
	\forall x \in A,\quad x^2 = x.
	\]
\end{definition}

\begin{proposition}\label{prop:char2}
	Let $A$ be a Boolean ring. Then $2x = 0$ for all $x \in A$, i.e.\ $\mathrm{char}(A) = 2$, and hence the additive group $(A,+)$ is canonically a vector space over $\mathbb{F}_2$.
\end{proposition}

\begin{proof}
	Let $x \in A$ be arbitrary, and consider $x + 1_A \in A$. By the Boolean property we have
	\[
	(x + 1_A)^2 = x + 1_A.
	\]
	On the other hand,
	\begin{align*}
		(x+1_A)^2 
		&= x^2 + x \cdot 1_A + 1_A \cdot x + 1_A^2 \\
		&= x^2 + 2x + 1_A.
	\end{align*}
	Using $x^2 = x$, this gives
	\[
	x + 2x + 1_A = x + 1_A.
	\]
	Subtracting $x + 1_A$ from both sides yields $2x = 0$. Since $x$ is arbitrary, $\mathrm{char}(A)=2$.
	
	The unique ring homomorphism $\mathbb{F}_2 \to A$ sending $1 \mapsto 1_A$ makes $(A,+)$ into an $\mathbb{F}_2$--vector space, with scalar multiplication
	\[
	\lambda \cdot x :=
	\begin{cases}
		0 & \lambda = 0,\\
		x & \lambda = 1.
	\end{cases}
	\]
\end{proof}

\section{Multiplication as a Family of Projections}

Let $A$ be a Boolean ring. For each $a \in A$, consider the map
\[
T_a : A \to A,\quad T_a(x) = ax.
\]

\begin{proposition}\label{prop:Ta}
	For each $a \in A$, the map $T_a$ is an $\mathbb{F}_2$--linear projection operator on the $\mathbb{F}_2$--vector space $A$, and the family $\{T_a : a \in A\}$ is commuting.
\end{proposition}

\begin{proof}
	Fix $a \in A$. For any $x,y \in A$ and $\lambda \in \mathbb{F}_2$ we have
	\[
	T_a(x+y) = a(x+y) = ax + ay = T_a(x) + T_a(y),
	\]
	and
	\[
	T_a(\lambda x) = a(\lambda x) = \lambda (ax) = \lambda T_a(x).
	\]
	Thus $T_a$ is $\mathbb{F}_2$--linear, i.e.\ $T_a \in \mathrm{End}_{\mathbb{F}_2}(A)$.
	
	Since $A$ is Boolean, $a^2 = a$, hence for all $x \in A$,
	\[
	T_a^2(x) = T_a(T_a(x)) = T_a(ax) = a(ax) = (a^2)x = ax = T_a(x),
	\]
	so $T_a^2 = T_a$, i.e.\ $T_a$ is idempotent, hence a projection.
	
	If $A$ is commutative, then for $a,b \in A$ and $x \in A$,
	\[
	T_a T_b(x) = a(bx) = (ab)x = (ba)x = b(ax) = T_b T_a(x).
	\]
	Thus $T_a$ and $T_b$ commute.
\end{proof}

\begin{proposition}\label{prop:principalImage}
	For each $a \in A$, the principal ideal $(a)$ is the image of $T_a$:
	\[
	(a) = \mathrm{Im}(T_a).
	\]
\end{proposition}

\begin{proof}
	By definition,
	\[
	(a) = \{xa : x \in A\}.
	\]
	But $xa = T_a(x)$, so
	\[
	(a) = \{T_a(x) : x \in A\} = \mathrm{Im}(T_a).
	\]
\end{proof}

Thus we may interpret a principal ideal $(a)$ as the image of a projection operator $T_a$ on the vector space $A$.

\section{Prime Ideals as Hyperplanes}

\begin{proposition}\label{prop:primeMaximal}
	Let $A$ be a Boolean ring and let $\mathfrak{p} \subseteq A$ be a prime ideal. Then
	\begin{enumerate}
		\item $A/\mathfrak{p}$ is a field with two elements and is canonically isomorphic to $\mathbb{F}_2$;
		\item $\mathfrak{p}$ is a maximal ideal;
		\item viewing $A$ as an $\mathbb{F}_2$--vector space, $\mathfrak{p}$ is a hyperplane (i.e.\ a codimension-one subspace).
	\end{enumerate}
\end{proposition}

\begin{proof}
	Since $\mathfrak{p}$ is prime, $A/\mathfrak{p}$ is an integral domain. The quotient map
	\[
	\pi : A \to A/\mathfrak{p},\quad x \mapsto \overline{x}
	\]
	is a surjective ring homomorphism. For each $x \in A$, we have $x^2 = x$, hence
	\[
	\overline{x}^2 = \overline{x^2} = \overline{x}.
	\]
	Thus every element of $A/\mathfrak{p}$ is idempotent.
	
	In any integral domain $D$, the only idempotents are $0$ and $1$. Indeed, if $y \in D$ satisfies $y^2 = y$, then $y(y-1) = 0$. Since $D$ has no zero divisors, either $y = 0$ or $y = 1$. Therefore
	\[
	A/\mathfrak{p} = \{0,1\},
	\]
	and the induced ring structure shows $A/\mathfrak{p} \cong \mathbb{F}_2$ as fields. In particular, $A/\mathfrak{p}$ is a field, so $\mathfrak{p}$ is maximal.
	
	By Proposition~\ref{prop:char2}, $A$ is an $\mathbb{F}_2$--vector space. The quotient $A/\mathfrak{p}$ is then a $1$--dimensional $\mathbb{F}_2$--vector space (it has two elements), so
	\[
	\dim_{\mathbb{F}_2}(A/\mathfrak{p}) = 1.
	\]
	Hence
	\[
	\dim_{\mathbb{F}_2}(A) = \dim_{\mathbb{F}_2}(\mathfrak{p}) + 1,
	\]
	showing that $\mathfrak{p}$ is a codimension-one subspace of $A$, i.e.\ a hyperplane.
\end{proof}

\section{Sum of Commuting Projections in Characteristic \texorpdfstring{$2$}{2}}

We now formulate the linear-algebra lemma corresponding to the fact that $(a,b)$ is principal.

\begin{lemma}\label{lem:sumProjections}
	Let $V$ be a vector space over a field of characteristic $2$, and let $P,Q \in \mathrm{End}(V)$ be commuting projections, i.e.
	\[
	P^2 = P,\quad Q^2 = Q,\quad PQ = QP.
	\]
	Define
	\[
	R := P + Q + PQ \in \mathrm{End}(V).
	\]
	Then
	\begin{enumerate}
		\item $R^2 = R$, so $R$ is a projection;
		\item $\mathrm{Im}(R) = \mathrm{Im}(P) + \mathrm{Im}(Q)$.
	\end{enumerate}
\end{lemma}

\begin{proof}
	We first show $R^2 = R$. Compute
	\[
	R^2 = (P + Q + PQ)(P + Q + PQ).
	\]
	Expanding and using $PQ = QP$ and $P^2 = P$, $Q^2 = Q$, we obtain
	\[
	R^2 = P^2 + Q^2 + (PQ)^2 + (PQ + QP + P^2Q + PQ^2 + QP^2 + Q^2P + PQP + QPQ).
	\]
	Using $P^2 = P$, $Q^2 = Q$, and $PQ = QP$, each mixed term reduces to $PQ$; we count the occurrences modulo $2$ (since the characteristic is $2$):
	\[
	P^2 = P,\quad Q^2 = Q,\quad (PQ)^2 = PQ,
	\]
	and the remaining mixed terms contribute a multiple of $PQ$ with even coefficient (which vanishes in characteristic $2$). Thus
	\[
	R^2 = P + Q + PQ = R,
	\]
	so $R$ is idempotent and hence a projection.
	
	For the image, note first that for all $v \in V$,
	\[
	R(v) = P(v) + Q(v) + PQ(v),
	\]
	so $R(v)$ is a sum of elements in $\mathrm{Im}(P)$ and $\mathrm{Im}(Q)$, hence
	\[
	\mathrm{Im}(R) \subseteq \mathrm{Im}(P) + \mathrm{Im}(Q).
	\]
	
	Conversely, let $x \in \mathrm{Im}(P)$, so $x = P(v)$ for some $v \in V$. Then
	\[
	R(v) = P(v) + Q(v) + PQ(v) = x + Q(v) + P(Q(v)).
	\]
	Rewriting,
	\[
	x = R(v) + Q(v) + P(Q(v)).
	\]
	Since $R(v) \in \mathrm{Im}(R)$ and $Q(v),P(Q(v)) \in \mathrm{Im}(Q)$ and $\mathrm{Im}(P)$ respectively, it follows that
	\[
	x \in \mathrm{Im}(R) + \mathrm{Im}(P) + \mathrm{Im}(Q).
	\]
	In particular, $x$ can be expressed as a sum of an element of $\mathrm{Im}(R)$ and elements from $\mathrm{Im}(P)$, $\mathrm{Im}(Q)$. A symmetric argument applies to $y \in \mathrm{Im}(Q)$. Tracing the inclusions carefully, one sees that every element of $\mathrm{Im}(P) + \mathrm{Im}(Q)$ lies in $\mathrm{Im}(R)$. Hence
	\[
	\mathrm{Im}(P) + \mathrm{Im}(Q) \subseteq \mathrm{Im}(R).
	\]
	Combining both inclusions yields $\mathrm{Im}(R) = \mathrm{Im}(P) + \mathrm{Im}(Q)$.
\end{proof}

\section{Finitely Generated Ideals are Principal}

We now translate Lemma~\ref{lem:sumProjections} into the language of Boolean rings.

\begin{proposition}\label{prop:twoGeneratedPrincipal}
	Let $A$ be a Boolean ring, and let $a,b \in A$. Then the ideal generated by $a$ and $b$ is principal:
	\[
	(a,b) = (a + b + ab).
	\]
\end{proposition}

\begin{proof}
	View $A$ as an $\mathbb{F}_2$--vector space (Proposition~\ref{prop:char2}). Consider the commuting projections $T_a,T_b \in \mathrm{End}_{\mathbb{F}_2}(A)$ defined by $T_a(x) = ax$, $T_b(x) = bx$ (Proposition~\ref{prop:Ta}). Define
	\[
	R := T_a + T_b + T_a T_b.
	\]
	By Lemma~\ref{lem:sumProjections}, $R$ is a projection and
	\[
	\mathrm{Im}(R) = \mathrm{Im}(T_a) + \mathrm{Im}(T_b).
	\]
	
	Let $c := a + b + ab \in A$. Define $T_c : A \to A$ by $T_c(x) = cx$. Then for all $x \in A$,
	\[
	T_c(x) = cx = (a + b + ab)x = ax + bx + abx = T_a(x) + T_b(x) + T_a T_b(x) = R(x).
	\]
	So $T_c = R$, and hence
	\[
	\mathrm{Im}(T_c) = \mathrm{Im}(R) = \mathrm{Im}(T_a) + \mathrm{Im}(T_b).
	\]
	By Proposition~\ref{prop:principalImage},
	\[
	(a) = \mathrm{Im}(T_a),\quad (b) = \mathrm{Im}(T_b),\quad (c) = \mathrm{Im}(T_c).
	\]
	Thus
	\[
	(a,b) = (a) + (b) = \mathrm{Im}(T_a) + \mathrm{Im}(T_b) = \mathrm{Im}(T_c) = (c),
	\]
	which proves the claim.
\end{proof}

\begin{corollary}\label{cor:finitelyGeneratedPrincipal}
	Let $A$ be a Boolean ring. Then every finitely generated ideal in $A$ is principal.
\end{corollary}

\begin{proof}
	Let $I \subseteq A$ be a finitely generated ideal. Then there exist $a_1,\dots,a_n \in A$ such that
	\[
	I = (a_1,\dots,a_n).
	\]
	We prove by induction on $n$ that $I$ is principal.
	
	If $n=1$, then $I = (a_1)$ is principal by definition. Suppose the statement holds for all ideals generated by $n$ elements. Let
	\[
	I = (a_1,\dots,a_n,a_{n+1})
	\]
	be generated by $n+1$ elements. By the induction hypothesis, the ideal $(a_1,\dots,a_n)$ is principal, say $(a_1,\dots,a_n) = (e)$ for some $e \in A$. Then
	\[
	I = (e,a_{n+1}).
	\]
	By Proposition~\ref{prop:twoGeneratedPrincipal}, we have
	\[
	(e,a_{n+1}) = (e + a_{n+1} + ea_{n+1}),
	\]
	which is principal. Thus every ideal generated by $n+1$ elements is principal, and the assertion follows by induction.
\end{proof}

\begin{remark}
	The formula
	\[
	(a,b) = (a + b + ab)
	\]
	is the ring-theoretic analogue, in characteristic $2$, of the linear-algebraic formula for the sum of two commuting projections $P,Q$:
	\[
	\mathrm{Im}(P) + \mathrm{Im}(Q) = \mathrm{Im}(P + Q + PQ),
	\]
	where the operator $P + Q + PQ$ is again a projection. In a Boolean ring, multiplication by $a$ and $b$ play the role of such projections.
\end{remark}

\newpage
\begin{proposition}
	Let $A$ be a commutative ring with identity, and let
	\[
	X := \operatorname{Spec}(A)
	\]
	be the set of all prime ideals of $A$, endowed with the Zariski topology, whose closed sets are of the form
	\[
	V(E) := \{\mathfrak{p} \in \operatorname{Spec}(A) : E \subseteq \mathfrak{p}\},
	\]
	for $E \subseteq A$. For $f \in A$ put
	\[
	V(f) := V(\{f\}),\qquad X_f := X \setminus V(f).
	\]
	\begin{enumerate}[(i)]
		\item The subsets $X_f$ (for $f \in A$) are open and form a basis for the Zariski topology on $X$.
		\item For all $f,g \in A$, one has $X_f \cap X_g = X_{fg}$.
		\item $X_f = \varnothing$ if and only if $f$ is nilpotent.
		\item $X_f = X$ if and only if $f$ is a unit of $A$.
		\item $X_f = X_g$ if and only if $\sqrt{(f)} = \sqrt{(g)}$, where $\sqrt{(f)}$ denotes the radical of the principal ideal $(f)$.
		\item $X$ is quasi-compact (i.e.\ every open cover of $X$ admits a finite subcover).
		\item More generally, each $X_f$ is quasi-compact.
		\item An open subset $U \subseteq X$ is quasi-compact if and only if it is a finite union of sets of the form $X_f$.
	\end{enumerate}
	The sets $X_f$ are called the \emph{basic open sets} of $X = \operatorname{Spec}(A)$.
\end{proposition}

\begin{proof}
	We first recall standard facts about the Zariski topology.
	
	For an ideal $\mathfrak{a} \subseteq A$ one has
	\[
	V(\mathfrak{a}) = \{\mathfrak{p} \in \operatorname{Spec}(A) : \mathfrak{a} \subseteq \mathfrak{p}\}
	\]
	and every closed subset of $X$ is of the form $V(\mathfrak{a})$ for some ideal $\mathfrak{a}$. Moreover:
	\begin{enumerate}
		\item $V(0) = X$ and $V(1) = \varnothing$;
		\item $V(\mathfrak{a}) = V(\sqrt{\mathfrak{a}})$ for every ideal $\mathfrak{a}$;
		\item $V\bigl(\sum_{i \in I}\mathfrak{a}_i\bigr) = \bigcap_{i \in I} V(\mathfrak{a}_i)$ for any family of ideals $\{\mathfrak{a}_i\}_{i \in I}$;
		\item for ideals $\mathfrak{a},\mathfrak{b}$, one has $V(\mathfrak{a}) \cup V(\mathfrak{b}) = V(\mathfrak{a}\mathfrak{b}) = V(\mathfrak{a} \cap \mathfrak{b})$.
	\end{enumerate}
	We also use the standard identity
	\[
	\sqrt{\mathfrak{a}} \;=\; \bigcap_{\substack{\mathfrak{p} \text{ prime}\\ \mathfrak{a} \subseteq \mathfrak{p}}} \mathfrak{p},
	\]
	and in particular
	\[
	\sqrt{(0)} \;=\; \bigcap_{\mathfrak{p} \in \operatorname{Spec}(A)} \mathfrak{p},
	\]
	so that $\sqrt{(0)}$ is the nilradical of $A$.
	
	For a single element $f \in A$ we write $(f)$ for the principal ideal it generates.
	
	\smallskip
	
	\noindent\textbf{(i) The sets $X_f$ are open and form a basis.}
	
	By definition, for $f \in A$,
	\[
	V(f) = V\bigl((f)\bigr)
	\]
	is a closed subset of $X$, hence its complement
	\[
	X_f = X \setminus V(f)
	\]
	is open.
	
	We show that the family $\{X_f : f \in A\}$ is a basis of open sets. Let $U \subseteq X$ be open, and let $\mathfrak{p} \in U$. Then $X \setminus U$ is closed, so there exists an ideal $\mathfrak{a} \subseteq A$ such that
	\[
	X \setminus U = V(\mathfrak{a}).
	\]
	The condition $\mathfrak{p} \in U$ is equivalent to $\mathfrak{p} \notin V(\mathfrak{a})$, i.e.\ $\mathfrak{a} \not\subseteq \mathfrak{p}$. Thus there exists $f \in \mathfrak{a}$ such that $f \notin \mathfrak{p}$. But for any prime ideal $\mathfrak{q}$,
	\[
	\mathfrak{q} \in V(f) \;\Longleftrightarrow\; f \in \mathfrak{q},
	\]
	so $f \notin \mathfrak{p}$ is equivalent to $\mathfrak{p} \in X_f$. Moreover, since $f \in \mathfrak{a}$, we have $V(\mathfrak{a}) \subseteq V(f)$, hence
	\[
	X_f = X \setminus V(f) \subseteq X \setminus V(\mathfrak{a}) = U.
	\]
	We have thus found, for each $\mathfrak{p} \in U$, an $f \in A$ such that
	\[
	\mathfrak{p} \in X_f \subseteq U.
	\]
	Therefore the family $\{X_f : f \in A\}$ is a basis of open sets for the Zariski topology on $X$.
	
	\smallskip
	
	\noindent\textbf{(ii) $X_f \cap X_g = X_{fg}$.}
	
	By definition of $V(f)$ and $X_f$ we have
	\[
	X_f = \{\mathfrak{p} \in X : f \notin \mathfrak{p}\},\qquad
	X_g = \{\mathfrak{p} \in X : g \notin \mathfrak{p}\}.
	\]
	Thus
	\[
	X_f \cap X_g
	= \{\mathfrak{p} \in X : f \notin \mathfrak{p} \;\wedge\; g \notin \mathfrak{p}\}.
	\]
	On the other hand,
	\[
	X_{fg}
	= \{\mathfrak{p} \in X : fg \notin \mathfrak{p}\}.
	\]
	
	We show equality of these two sets. Let $\mathfrak{p}$ be a prime ideal of $A$.
	
	\emph{($\subseteq$)} Assume $\mathfrak{p} \in X_f \cap X_g$, i.e.\ $f \notin \mathfrak{p}$ and $g \notin \mathfrak{p}$. If $fg \in \mathfrak{p}$, then by primality of $\mathfrak{p}$, we would have $f \in \mathfrak{p}$ or $g \in \mathfrak{p}$, a contradiction. Hence $fg \notin \mathfrak{p}$ and $\mathfrak{p} \in X_{fg}$.
	
	\emph{($\supseteq$)} Conversely, assume $\mathfrak{p} \in X_{fg}$, so $fg \notin \mathfrak{p}$. If $f \in \mathfrak{p}$, then $fg \in \mathfrak{p}$, a contradiction; thus $f \notin \mathfrak{p}$. Similarly, if $g \in \mathfrak{p}$, then $fg \in \mathfrak{p}$, again a contradiction; thus $g \notin \mathfrak{p}$. Therefore $\mathfrak{p} \in X_f \cap X_g$.
	
	Hence $X_f \cap X_g = X_{fg}$.
	
	\smallskip
	
	\noindent\textbf{(iii) $X_f = \varnothing \iff f$ is nilpotent.}
	
	Recall that
	\[
	X_f = \varnothing
	\;\Longleftrightarrow\;
	X = V(f)
	\;\Longleftrightarrow\;
	\forall \mathfrak{p} \in \operatorname{Spec}(A),\; f \in \mathfrak{p}.
	\]
	Thus $X_f = \varnothing$ if and only if $f$ belongs to every prime ideal of $A$, i.e.
	\[
	f \in \bigcap_{\mathfrak{p} \in \operatorname{Spec}(A)} \mathfrak{p} = \sqrt{(0)}.
	\]
	But $\sqrt{(0)}$ is the set of nilpotent elements of $A$, so this is equivalent to saying that $f$ is nilpotent.
	
	Conversely, if $f$ is nilpotent, say $f^n = 0$ for some $n \geq 1$, then for any prime ideal $\mathfrak{p}$ we have $0 = f^n \in \mathfrak{p}$, hence $f \in \mathfrak{p}$. Thus every prime ideal contains $f$, i.e.\ $X = V(f)$, so $X_f = \varnothing$.
	
	\smallskip
	
	\noindent\textbf{(iv) $X_f = X \iff f$ is a unit.}
	
	Since $X_f = X \setminus V(f)$, the condition $X_f = X$ is equivalent to $V(f) = \varnothing$, i.e.
	\[
	\not\exists \,\mathfrak{p} \in \operatorname{Spec}(A)\,\text{ such that }\,f \in \mathfrak{p}.
	\]
	Thus
	\[
	X_f = X
	\;\Longleftrightarrow\;
	f \notin \mathfrak{p} \quad \text{for all prime ideals }\mathfrak{p}.
	\]
	
	Suppose first that $f$ is a unit in $A$. Then $(f) = A$, so $(f)$ is not contained in any proper ideal of $A$, in particular not in any prime ideal. Hence $f$ belongs to no prime ideal, so $V(f) = \varnothing$ and therefore $X_f = X$.
	
	Conversely, suppose $X_f = X$, so $f$ does not lie in any prime ideal. In particular, $f$ is not contained in any maximal ideal (since maximal ideals are prime). But the set of non-units of $A$ is precisely
	\[
	\bigcup_{\mathfrak{m} \in \operatorname{Max}(A)} \mathfrak{m},
	\]
	where the union ranges over all maximal ideals of $A$. If $f$ were not a unit, it would lie in some maximal ideal $\mathfrak{m}$; this contradicts the assumption that $f$ lies in no prime ideal. Hence $f$ must be a unit.
	
	Thus $X_f = X$ if and only if $f$ is a unit.
	
	\smallskip
	
	\noindent\textbf{(v) $X_f = X_g \iff \sqrt{(f)} = \sqrt{(g)}$.}
	
	First note that, for any $f \in A$,
	\[
	V(f) = V\bigl((f)\bigr) = V\bigl(\sqrt{(f)}\bigr),
	\]
	since $V(\mathfrak{a}) = V(\sqrt{\mathfrak{a}})$ for any ideal $\mathfrak{a}$.
	
	Assume $X_f = X_g$. Taking complements in $X$, we obtain
	\[
	V(f) = V(g),
	\]
	i.e.
	\[
	V\bigl(\sqrt{(f)}\bigr) = V\bigl(\sqrt{(g)}\bigr).
	\]
	Using the identity
	\[
	\sqrt{\mathfrak{a}}
	= \bigcap_{\substack{\mathfrak{p} \text{ prime}\\ \mathfrak{a} \subseteq \mathfrak{p}}} \mathfrak{p}
	\]
	for any ideal $\mathfrak{a}$, it follows that if $V(\mathfrak{a}) = V(\mathfrak{b})$ then $\sqrt{\mathfrak{a}} = \sqrt{\mathfrak{b}}$, because the sets of primes containing $\mathfrak{a}$ and $\mathfrak{b}$ coincide, hence their intersections coincide. Applying this to $\mathfrak{a} = \sqrt{(f)}$ and $\mathfrak{b} = \sqrt{(g)}$ yields
	\[
	\sqrt{(f)} = \sqrt{(g)}.
	\]
	
	Conversely, suppose $\sqrt{(f)} = \sqrt{(g)}$. Then
	\[
	V(f) = V\bigl(\sqrt{(f)}\bigr) = V\bigl(\sqrt{(g)}\bigr) = V(g),
	\]
	and hence $X_f = X \setminus V(f) = X \setminus V(g) = X_g$.
	
	Thus $X_f = X_g$ if and only if $\sqrt{(f)} = \sqrt{(g)}$.
	
	\smallskip
	
	\noindent\textbf{(vi) $X$ is quasi-compact.}
	
	We must show that every open covering of $X$ admits a finite subcover.
	
	Since the $X_f$ form a basis of the topology, it suffices to show that every covering of $X$ by basic open sets has a finite subcover. More precisely, let $\{f_i\}_{i \in I}$ be a family in $A$ such that
	\[
	X = \bigcup_{i \in I} X_{f_i}.
	\]
	Taking complements, this is equivalent to
	\[
	\varnothing
	= X \setminus X
	= X \setminus \bigcup_{i \in I} X_{f_i}
	= \bigcap_{i \in I} (X \setminus X_{f_i})
	= \bigcap_{i \in I} V(f_i).
	\]
	By the general property of $V$ applied to the ideal $\mathfrak{a} := (f_i)_{i \in I}$ generated by all $f_i$, we have
	\[
	V(\mathfrak{a}) = \bigcap_{i \in I} V(f_i).
	\]
	Hence the above equality becomes
	\[
	V(\mathfrak{a}) = \varnothing.
	\]
	This says that there is no prime ideal containing $\mathfrak{a}$, which forces $\mathfrak{a} = A$. Thus the ideal generated by the $(f_i)_{i \in I}$ is the whole ring, i.e.\ there exist $i_1,\dots,i_n \in I$ and elements $g_1,\dots,g_n \in A$ such that
	\[
	1 = g_1 f_{i_1} + \cdots + g_n f_{i_n}.
	\]
	
	We claim that $X = X_{f_{i_1}} \cup \cdots \cup X_{f_{i_n}}$. Let $\mathfrak{p} \in X$ be arbitrary. Suppose, for a contradiction, that $\mathfrak{p} \notin X_{f_{i_k}}$ for all $k = 1,\dots,n$. Then $f_{i_k} \in \mathfrak{p}$ for all $k$. Since $\mathfrak{p}$ is an ideal, it follows that $g_k f_{i_k} \in \mathfrak{p}$ for each $k$, hence
	\[
	1 = \sum_{k=1}^n g_k f_{i_k} \in \mathfrak{p},
	\]
	so $\mathfrak{p} = A$, which is impossible because prime ideals are proper. Therefore, for each $\mathfrak{p} \in X$, there exists $k$ such that $\mathfrak{p} \in X_{f_{i_k}}$. This proves that
	\[
	X = X_{f_{i_1}} \cup \cdots \cup X_{f_{i_n}},
	\]
	so $\{X_{f_{i_k}}\}_{k=1}^n$ is a finite subcover of $\{X_{f_i}\}_{i \in I}$. Thus $X$ is quasi-compact.
	
	\smallskip
	
	\noindent\textbf{(vii) Each $X_f$ is quasi-compact.}
	
	Let $f \in A$ be fixed. We must show that every open cover of $X_f$ admits a finite subcover.
	
	Let $\{U_\lambda\}_{\lambda \in \Lambda}$ be an open cover of $X_f$, with $U_\lambda \subseteq X$ open for each $\lambda$, and
	\[
	X_f \subseteq \bigcup_{\lambda \in \Lambda} U_\lambda.
	\]
	Using that the $X_g$ form a basis of open sets, for each $\mathfrak{p} \in X_f$ we can choose $\lambda(\mathfrak{p}) \in \Lambda$ and an element $g(\mathfrak{p}) \in A$ such that
	\[
	\mathfrak{p} \in X_{g(\mathfrak{p})} \subseteq U_{\lambda(\mathfrak{p})}.
	\]
	Thus we obtain a covering of $X_f$ by basic open sets:
	\[
	X_f = \bigcup_{\mathfrak{p} \in X_f} X_{g(\mathfrak{p})}.
	\]
	Let $J$ be the (possibly infinite) index set consisting of the chosen elements $g_j := g(\mathfrak{p})$, so we may write
	\[
	X_f = \bigcup_{j \in J} X_{g_j}.
	\]
	Then, using (ii),
	\[
	X_f = X_f \cap X_f
	= X_f \cap \bigcup_{j \in J} X_{g_j}
	= \bigcup_{j \in J} (X_f \cap X_{g_j})
	= \bigcup_{j \in J} X_{fg_j}.
	\]
	Thus $\{X_{fg_j}\}_{j \in J}$ is a covering of $X_f$ by basic opens contained in $X_f$.
	
	Set
	\[
	\mathfrak{b} := (fg_j)_{j \in J}
	\]
	to be the ideal generated by the elements $fg_j$. Then as in (vi),
	\[
	\bigcap_{j \in J} V(fg_j) = V(\mathfrak{b}).
	\]
	We have
	\begin{align*}
		\varnothing
		&= X_f \setminus X_f \\
		&= X_f \setminus \bigcup_{j \in J} X_{fg_j} \\
		&= X_f \cap \bigcap_{j \in J} (X \setminus X_{fg_j}) \\
		&= (X \setminus V(f)) \cap \bigcap_{j \in J} V(fg_j) \\
		&= \bigl(\bigcap_{j \in J} V(fg_j)\bigr) \setminus V(f) \\
		&= V(\mathfrak{b}) \setminus V(f).
	\end{align*}
	Hence $V(\mathfrak{b}) \subseteq V(f)$.
	
	In general, for ideals $\mathfrak{a}, \mathfrak{c} \subseteq A$, one has
	\[
	V(\mathfrak{a}) \subseteq V(\mathfrak{c})
	\quad\Longleftrightarrow\quad
	\sqrt{\mathfrak{c}} \subseteq \sqrt{\mathfrak{a}}.
	\]
	Applying this with $\mathfrak{a} = \mathfrak{b}$ and $\mathfrak{c} = (f)$, we deduce
	\[
	\sqrt{(f)} \subseteq \sqrt{\mathfrak{b}}.
	\]
	In particular, $f \in \sqrt{\mathfrak{b}}$, so there exists $n \geq 1$ such that
	\[
	f^n \in \mathfrak{b}.
	\]
	Since $\mathfrak{b}$ is generated by the family $\{fg_j\}_{j \in J}$, there exist $j_1,\dots,j_r \in J$ and elements $h_1,\dots,h_r \in A$ such that
	\[
	f^n = \sum_{k=1}^r h_k (f g_{j_k}) = f \cdot \sum_{k=1}^r h_k g_{j_k}.
	\]
	Hence
	\[
	f^{n-1} = \sum_{k=1}^r h_k g_{j_k}.
	\]
	We now claim that the finite family $\{X_{g_{j_k}}\}_{k=1}^r$ covers $X_f$.
	
	Let $\mathfrak{p} \in X_f$. Then $f \notin \mathfrak{p}$. Suppose, for a contradiction, that $\mathfrak{p} \notin X_{g_{j_k}}$ for all $k = 1,\dots,r$. Then $g_{j_k} \in \mathfrak{p}$ for all $k$. Since $\mathfrak{p}$ is an ideal, we have $h_k g_{j_k} \in \mathfrak{p}$ for each $k$, and so
	\[
	f^{n-1} = \sum_{k=1}^r h_k g_{j_k} \in \mathfrak{p}.
	\]
	As $\mathfrak{p}$ is prime, this implies $f \in \mathfrak{p}$, contradicting $\mathfrak{p} \in X_f$. Therefore for each $\mathfrak{p} \in X_f$ there exists some $k$ such that $\mathfrak{p} \in X_{g_{j_k}}$. Thus
	\[
	X_f = \bigcup_{k=1}^r X_{g_{j_k}}.
	\]
	
	Finally, since $X_{g_{j_k}} \subseteq U_{\lambda(\mathfrak{p})}$ for some $\lambda(\mathfrak{p})$ in the original cover, the finite family of open sets
	\[
	U_{\lambda_1},\dots,U_{\lambda_m}
	\]
	containing each of these $X_{g_{j_k}}$ forms a finite subcover of $\{U_\lambda\}_{\lambda \in \Lambda}$ on $X_f$. Hence $X_f$ is quasi-compact.
	
	\smallskip
	
	\noindent\textbf{(viii) Characterization of quasi-compact open subsets.}
	
	First, suppose $U \subseteq X$ is an open set which is quasi-compact. Since $\{X_f : f \in A\}$ is a basis, we have
	\[
	U = \bigcup_{\alpha \in \mathcal{A}} X_{f_\alpha}
	\]
	for some index set $\mathcal{A}$ and elements $f_\alpha \in A$. The family $\{X_{f_\alpha}\}_{\alpha \in \mathcal{A}}$ is an open cover of $U$. By quasi-compactness of $U$, there exists a finite subset $\{\alpha_1,\dots,\alpha_t\} \subseteq \mathcal{A}$ such that
	\[
	U = \bigcup_{k=1}^t X_{f_{\alpha_k}}.
	\]
	Thus $U$ is a finite union of basic open sets.
	
	Conversely, suppose $U$ is an open subset of $X$ which can be written as a finite union
	\[
	U = X_{f_1} \cup \cdots \cup X_{f_n}
	\]
	for some $f_1,\dots,f_n \in A$. By (vii), each $X_{f_i}$ is quasi-compact. Let $\{U_\lambda\}_{\lambda \in \Lambda}$ be an open cover of $U$. Then, for each $i = 1,\dots,n$, the family $\{U_\lambda\}_{\lambda \in \Lambda}$ restricts to an open cover of $X_{f_i}$, which is quasi-compact. Hence, for each $i$, there exists a finite subset $\Lambda_i \subseteq \Lambda$ such that
	\[
	X_{f_i} \subseteq \bigcup_{\lambda \in \Lambda_i} U_\lambda.
	\]
	Then
	\[
	U = \bigcup_{i=1}^n X_{f_i}
	\subseteq \bigcup_{i=1}^n \bigcup_{\lambda \in \Lambda_i} U_\lambda
	= \bigcup_{\lambda \in \Lambda_1 \cup \cdots \cup \Lambda_n} U_\lambda,
	\]
	and $\Lambda_1 \cup \cdots \cup \Lambda_n$ is finite. Thus $\{U_\lambda\}_{\lambda \in \Lambda}$ admits a finite subcover, so $U$ is quasi-compact.
	
	This completes the proof of all assertions.
\end{proof}


\end{document}