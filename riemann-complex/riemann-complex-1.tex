\documentclass[11pt,openany]{article}

\usepackage{mathtools, commath}
% Packages for formatting
\usepackage[margin=1in]{geometry}
\usepackage{fancyhdr}
\usepackage{enumerate}
\usepackage{graphicx}
\usepackage{kotex}
\usepackage{arydshln} % Include this package
\usepackage{bbding}
\usepackage{amsmath}
\usepackage{amsthm}
\usepackage[dvipsnames,table]{xcolor}
\usepackage{amssymb, amsfonts}
\usepackage{wasysym}
\usepackage{footnote}
\usepackage{tablefootnote}
\usepackage{arydshln} % Include this package

% Fonts
\usepackage[T1]{fontenc}
\usepackage[utf8]{inputenc}
\usepackage{newpxtext,newpxmath}
\usepackage{sectsty}

% Define colors
\definecolor{TealBlue1}{HTML}{0077c2}
\definecolor{TealBlue2}{HTML}{00a5e6}
\definecolor{TealBlue3}{HTML}{b3e0ff}
\definecolor{TealBlue4}{HTML}{00293c}
\definecolor{TealBlue5}{HTML}{e6f7ff}

\definecolor{thmcolor}{RGB}{231, 76, 60}
\definecolor{defcolor}{RGB}{52, 152, 219}
\definecolor{lemcolor}{RGB}{155, 89, 182}
\definecolor{corcolor}{RGB}{46, 204, 113}
\definecolor{procolor}{RGB}{241, 196, 15}

\usepackage{color,soul}
\usepackage{soul}
\newcommand{\mathcolorbox}[2]{\colorbox{#1}{$\displaystyle #2$}}
\usepackage{cancel}
\newcommand\crossout[3][black]{\renewcommand\CancelColor{\color{#1}}\cancelto{#2}{#3}}
\newcommand\ncrossout[2][black]{\renewcommand\CancelColor{\color{#1}}\cancel{#2}}

\usepackage{hyperref}
\usepackage{booktabs}

% Chapter formatting
\definecolor{titleTealBlue}{RGB}{0,53,128}
\usepackage{titlesec}
\titleformat{\section}
{\normalfont\sffamily\Large\bfseries\color{titleTealBlue!100!gray}}{\thesection}{1em}{}
\titleformat{\subsection}
{\normalfont\sffamily\large\bfseries\color{titleTealBlue!50!gray}}{\thesubsection}{1em}{}

%Tcolorbox
\usepackage[most]{tcolorbox}
\usepackage{multirow}
\usepackage{multicol}
\usepackage{blindtext}

\usepackage[linesnumbered,ruled]{algorithm2e}
\usepackage{algpseudocode}
\usepackage{setspace}
\SetKwComment{Comment}{/* }{ */}
\SetKwProg{Fn}{Function}{:}{end}
\SetKw{End}{end}
\SetKw{DownTo}{downto}

% Define a new environment for algorithms without line numbers
\newenvironment{algorithm2}[1][]{
	% Save the current state of the algorithm counter
	\newcounter{tempCounter}
	\setcounter{tempCounter}{\value{algocf}}
	% redefine the algorithm numbering (remove prefix)
	\renewcommand{\thealgocf}{}
	\begin{algorithm}
	}{
	\end{algorithm}
	% Restore the algorithm counter state
	\setcounter{algocf}{\value{tempCounter}}
}

\usepackage{adjustbox}
% Header and footer formatting
\pagestyle{fancy}
\fancyhead{}
\fancyhf{}
\rhead{\textcolor{TealBlue2}{\large\textbf{리만의 복소해석을 토대로 얻는 내 수학적 시야 (2기)}}}%\rule{3cm}{0.4pt}}
\lhead{\textcolor{TealBlue2}{\large\textbf{수학의 즐거움, Enjoying Math}}}
% Define footer
%\newcommand{\footer}[1]{
%\begin{flushright}
%	\vspace{2em}
%	\includegraphics[width=2.5cm]{school_logo.jpg} \\
%	\vspace{1em}
%	\textcolor{TealBlue2}{\small\textbf{#1}}
%\end{flushright}
%}
%\rfoot{\large Department of Information Security, Cryptogrphy and Mathematics, Kookmin Uni.\includegraphics[height=1.5cm]{school_logo.jpg}}
\fancyfoot{}
\fancyfoot[C]{-\thepage-}

\usepackage{tcolorbox}
\tcbset{colback=white, arc=5pt}

\definecolor{axiomcolor}{HTML}{a88bfa}
\definecolor{defcolor}{RGB}{52, 152, 219}
\definecolor{procolor}{RGB}{241, 196, 15}
\definecolor{thmcolor}{RGB}{231, 76, 60}
\definecolor{lemcolor}{RGB}{155, 89, 182}
\definecolor{corcolor}{RGB}{46, 204, 113}
\definecolor{execolor}{RGB}{90, 128, 127}

% Define a new command for the custom tcolorbox
\newcommand{\axiombox}[2][]{%
	\begin{tcolorbox}[colframe=axiomcolor, title={\color{white}\bfseries #1}]
		#2
	\end{tcolorbox}
}

\newcommand{\defbox}[2][]{%
	\begin{tcolorbox}[colframe=defcolor, title={\color{white}\bfseries #1}]
		#2
	\end{tcolorbox}
}

\newcommand{\probox}[2][]{%
	\begin{tcolorbox}[colframe=procolor, title={\color{white}\bfseries #1}]
		#2
	\end{tcolorbox}
}

\newcommand{\thmbox}[2][]{%
	\begin{tcolorbox}[colframe=thmcolor, title={\color{white}\bfseries #1}]
		#2
	\end{tcolorbox}
}

\newcommand{\lembox}[2][]{%
	\begin{tcolorbox}[colframe=lemcolor, title={\color{white}\bfseries #1}]
		#2
	\end{tcolorbox}
}
\usepackage{amsthm}

% Define custom theorem styles
\newtheoremstyle{dotless} % Name of the style
{3pt} % Space above
{3pt} % Space below
{\itshape} % Body font
{} % Indent amount
{\bfseries} % Theorem head font
{} % Punctuation after theorem head
{2.5mm} % Space after theorem head
{} % Theorem head spec

\newtheoremstyle{definitionstyle} % Name of the style
{3pt} % Space above
{3pt} % Space below
{} % Body font
{} % Indent amount
{\bfseries} % Theorem head font
{.} % Punctuation after theorem head
{2.5mm} % Space after theorem head
{} % Theorem head spec

% Applying custom styles
%\theoremstyle{dotless}
\newtheorem{theorem}{Theorem} % Theorem environment with section-wise numbering
\newtheorem*{theorem*}{Theorem} % Theorem environment with section-wise numbering
\newtheorem*{lemma*}{Lemma} % Theorem environment with section-wise numbering
\newtheorem*{proposition*}{Proposition} % Theorem environment with section-wise numbering
\newtheorem*{corollary*}{Corollary} % Theorem environment with section-wise numbering
\newtheorem{proposition}[theorem]{Proposition} % Theorem environment with section-wise numbering
\newtheorem{lemma}[theorem]{Lemma} % Lemma shares the counter with theorem
\newtheorem{corollary}[theorem]{Corollary} % Corollary shares the counter with theorem

\theoremstyle{definitionstyle}
\newtheorem*{observation}{\textcolor{magenta}{Observation}}
\newtheorem*{illustration}{\textcolor{teal}{Illustration}}
\newtheorem*{torus}{{\color{red}T}{\color{orange}o}{\color{green!75!black}r}{\color{cyan}u}{\color{violet}s}}
\newtheorem{definition}{Definition} % Definition shares the counter with theorem
\newtheorem{example}{Example} % Example shares the counter with theorem
\newtheorem{exercise}{{Exercise}} % Example shares the counter with theorem
\newtheorem{remark}{Remark} % Remark shares the counter with theorem
\newtheorem*{note}{Note}
\newtheorem*{notation}{Notation}

\newtheorem*{axiom*}{Axiom}
\newtheorem*{definition*}{Definition} % Definition shares the counter with theorem
\newtheorem*{example*}{Example} % Example shares the counter with theorem
\newtheorem*{exercise*}{\textcolor{teal}{Exercise}} % Example shares the counter with theorem
\newtheorem*{remark*}{Remark} % Remark shares the counter with theorem


\usepackage{tikz}
\usepackage{tikz-cd}
\usetikzlibrary{shadows}
\usetikzlibrary{shapes.geometric, arrows.meta, positioning}
\input{riemann-complex-commands}
\renewcommand{\vec}[1]{\mathbf{#1}}
\setstretch{1.25}

%\usepackage{background}
%\backgroundsetup{
%	scale=3,
%	color=gray!20,
%	opacity=0.3,
%	angle=45,
%	contents={\Huge \sffamily Ji, Yong-hyeon}
%}
\begin{document}
\pagenumbering{arabic}
\begin{center}
	\huge\textbf{Line Integral I}\\
	\vspace{0.5em}
	\large{Ji, Yong-hyeon}\\
%	\large{\ttfamily \url{https://github.com/Hacker-Code-J}}\\
	\vspace{0.5em}
	\normalsize{\today}\\
\end{center}

\noindent 
We cover the following topics in this note.
\begin{itemize}
	\item Unique representation as a finite linear combination of the elements of Basis.
\end{itemize}
\hrule\vspace{12pt}
%\tableofcontents
%\newpage

\probox{\begin{proposition*}
	Let $V$ be a vector space over a field $F$, and let $\dim V<\infty$, say, $\dim V=n$.
	Fix a basis $\basis$. Then every vector $\vec{v}\in V$ has a unique expression of linear combination by $\basis$.
\end{proposition*}}
\begin{proof}
	Let $\basis=\set{\vec{v}_1,\vec{v}_2,\dots,\vec{v}_n}$ be a basis of $V$. Take any $\vec{v}\in V(=\Span\basis)$. Then \[
	\exists a_1,a_2,\dots,a_n\in F\quad\text{such that}\quad a_1\vec{v}_1+\cdots+a_n\vec{v}_n=\vec{v}.
	\] Suppose that $\exists b_1,b_2,\dots, b_n\in F$ such that $\sum_{i=1}^{n}b_i\vec{v}_i=\vec{v}$. Then \begin{align*}
		a_1\vec{v}_1+a_2\vec{v}_2+\cdots a_n\vec{v}_n&=b_1\vec{v}_1+b_2\vec{v}_2+\cdots+b_n\vec{v}_n,\\
		(a_1-b_1)\vec{v}_1+(a_2-b_2)\vec{v}_2+\cdots (a_n-b_n)\vec{v}_n&=\vec{0},
	\end{align*} and so $a_i=b_i$ for all $i=1,2,\dots,n$ since a basis $\basis=\set{\vec{v}_1,\vec{v}_2,\cdots,\vec{v}_n}$ is linearly independent.
\end{proof}

\defbox[Coordinate]{\begin{definition*}
	Write \[
	[\vec{v}]_{\basis}=(a_1,a_2,\dots,a_n).
	\] is called the coordinate of $\vec{v}$ with respect to $\basis$.
\end{definition*}}

\defbox[Linear Transformation]{\begin{definition*}
	We say $\Phi:V\to W$ is a linear transformation if $\Phi$ preserves a linearity, \ie, \[
	\Phi(a\cdot \vec{v}+b\cdot \vec{w})=a\cdot\Phi(\vec{v})+b\cdot\Phi(\vec{w})
	\] for any $a,b\in F$ and $\vec{v},\vec{w}\in V$. Here, if $\Phi:W\to V$ is also a linear transformation the we say $\Phi$ is the vector-space isomorphism.
\end{definition*}}

\probox{\begin{definition*}
	finite-dimensional vector space $V,W/F$. Then \[
	\dim V=\dim W\iff\exists\text{a vector-space isomorphism}\ \Phi:V\to W,\ie, V\simeq W.
	\]
\end{definition*}}
\begin{proof}
\begin{itemize}
	\item[($\Rightarrow$)] Suppose that $\dim V=\dim W=n\in\N$. Take basis $\basis_V=\set{\vec{v}_1,\vec{v}_2,\dots,\vec{v}_n}$ and $\basis_W=\set{\vec{w}_1,\vec{w}_2,\dots,\vec{w}_n}$ of $W$. Define \[
	\fullfunction{\Phi}{V}{W}{\vec{v}}{a_1\vec{w}_1+\cdots+a_n\vec{w}_n}.
	\] We claim that $\Phi$ is one-to-one and onto linear transformation.
	\item[($\Leftarrow$)] Suppose there exists $\Phi:V\to W$. Take any basis $\basis_V=\set{\vec{v}_1,\vec{v}_2,\dots,\vec{v}_n}$ of $V$. Define \[
	\basis_W:=\set{\Phi(\vec{v}_1),\Phi(\vec{v}_2),\dots,\Phi(\vec{v}_n)}.
	\]  We claim that $\basis_W$ be a basis of $W$: \begin{enumerate}[]
		\item (Linearly Independent) Suppose that $a_1\Phi(\vec{v}_1)+\cdots a_n\Phi(\vec{v}_n)=0$. Since $\Phi$ is a linear transformation, we have \[
		\Phi(a_1\vec{v}_1+\cdots+a_n\vec{v}_n)=0.
		\] Since $\Phi$ is one-to-one, and $0=\Phi(0)$, $a_1\vec{v}_1+a_n\vec{v}_n=0$, $a_1=a_n=0$ since $\basis_V$ is a basis.
		\item (Spanning Property) Take $\vec{w}\in W$. Since $\Phi$ is onto, $\exists \vec{v}\in V$ s.t. $\Phi(\vec{v})=\vec{w}$. Since $\basis_V$ is a basis, $\exists!a_1,a_2,\dots,a_n$ s.t. $a_1\vec{v}_1+\cdots+a_n\vec{v_n}=\vec{v}1$, and so \[
		\vec{w}=\Phi(a_1\vec{v}_1+\cdots+a_n\vec{v}_n)=a_1\Phi(\vec{v}_1)+\cdots+a_n\Phi(\vec{v}_n)
		\]
	\end{enumerate}
\end{itemize}
\end{proof}

\vfill
\begin{thebibliography}{9}
	\bibitem{linear_algebra_c}
	수학의 즐거움, Enjoying Math. ``수학 공부, 기초부터 대학원 수학까지, 16. 선형대수학 (c) 차원과 벡터공간의 분류'' YouTube Video, 29:08. Published 
	October 11, 2019. URL: \url{https://www.youtube.com/watch?v=rOKN645fRPs&t=399s}.
	\bibitem{linear_algebra_d}
	수학의 즐거움, Enjoying Math. ``수학 공부, 기초부터 대학원 수학까지, 17. 선형대수학 (d) 선형함수의 행렬 표현'' YouTube Video, 29:14. Published 
	October 12, 2019. URL: \url{https://www.youtube.com/watch?v=Fsy-9KW9-PA}.
\end{thebibliography}

%\newpage
%\appendix
%\section{Proof of Zorn's Lemma from Axiom of Choice}
%\thmbox{\begin{theorem*}\hypertarget{zorn}{}
%The following statements are equivalent: \begin{enumerate}
%	\item \textbf{Axiom of Choice (AC)}:\quad For every indexed family $\set{S_i}_{i\in I}$ of nonempty sets, there exists a choice function $f:I\to\bigcup_{i\in I}S_i$ such that $f(i)\in X_i$ for all $i\in I$.
%	\item \textbf{Zorn's Lemma}:\quad If $(P,\preceq)$ is a nonempty partially ordered set in which every chains has an upper bound in $P$, then $P$ contains at least one maximal element.
%\end{enumerate}
%\end{theorem*}}
%\begin{proof}
%\begin{itemize}
%	\item[($\Rightarrow$)] $(\textbf{AC}\implies \textbf{ZL})$ Assume that the Axiom of Choice holds.
%	\begin{enumerate}
%		\item \textbf{Definition of Poset}
%		
%		Let $(P,\preceq)$ be a nonempty partially ordered set with the property that every chain in $P$ has an upper bound in  $P$.
%		
%		\item \textbf{Construction of an Extending Function}
%		
%		Define the family $\set{\mathcal{C}_i}_{i\in I}$ of chains in $P$. For any chain $\mathcal{C}_i$ that is not maximal with respect to inclusion (\ie, $\exists $), AC guarantees that we can elect an element \[
%		f(C)
%		\] 
%	\end{enumerate}
%	\item[($\Leftarrow$)] $(\textbf{ZL}\implies \textbf{AC})$ 
%\end{itemize}
%\end{proof}
\end{document}
