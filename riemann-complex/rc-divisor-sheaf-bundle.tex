\documentclass[11pt,openany]{article}

\usepackage{mathtools, commath}
% Packages for formatting
\usepackage[margin=1in]{geometry}
\usepackage{fancyhdr}
\usepackage{enumerate}
\usepackage{graphicx}
\usepackage{kotex}
\usepackage{arydshln} % Include this package
\usepackage{bbding}
\usepackage{amsmath}
\usepackage{amsthm}
\usepackage[dvipsnames,table]{xcolor}
\usepackage{amssymb, amsfonts}
\usepackage{wasysym}
\usepackage{footnote}
\usepackage{tablefootnote}
\usepackage{arydshln} % Include this package

% Fonts
\usepackage[T1]{fontenc}
\usepackage[utf8]{inputenc}
\usepackage{newpxtext,newpxmath}
\usepackage{sectsty}

% Define colors
\definecolor{TealBlue1}{HTML}{0077c2}
\definecolor{TealBlue2}{HTML}{00a5e6}
\definecolor{TealBlue3}{HTML}{b3e0ff}
\definecolor{TealBlue4}{HTML}{00293c}
\definecolor{TealBlue5}{HTML}{e6f7ff}

\definecolor{thmcolor}{RGB}{231, 76, 60}
\definecolor{defcolor}{RGB}{52, 152, 219}
\definecolor{lemcolor}{RGB}{155, 89, 182}
\definecolor{corcolor}{RGB}{46, 204, 113}
\definecolor{procolor}{RGB}{241, 196, 15}

\usepackage{color,soul}
\usepackage{soul}
\newcommand{\mathcolorbox}[2]{\colorbox{#1}{$\displaystyle #2$}}
\usepackage{cancel}
\newcommand\crossout[3][black]{\renewcommand\CancelColor{\color{#1}}\cancelto{#2}{#3}}
\newcommand\ncrossout[2][black]{\renewcommand\CancelColor{\color{#1}}\cancel{#2}}

\usepackage{hyperref}
\usepackage{booktabs}

% Chapter formatting
\definecolor{titleTealBlue}{RGB}{0,53,128}
\usepackage{titlesec}
\titleformat{\section}
{\normalfont\sffamily\Large\bfseries\color{titleTealBlue!100!gray}}{\thesection}{1em}{}
\titleformat{\subsection}
{\normalfont\sffamily\large\bfseries\color{titleTealBlue!50!gray}}{\thesubsection}{1em}{}

%Tcolorbox
\usepackage[most]{tcolorbox}
\usepackage{multirow}
\usepackage{multicol}
\usepackage{blindtext}

\usepackage[linesnumbered,ruled]{algorithm2e}
\usepackage{algpseudocode}
\usepackage{setspace}
\SetKwComment{Comment}{/* }{ */}
\SetKwProg{Fn}{Function}{:}{end}
\SetKw{End}{end}
\SetKw{DownTo}{downto}

% Define a new environment for algorithms without line numbers
\newenvironment{algorithm2}[1][]{
	% Save the current state of the algorithm counter
	\newcounter{tempCounter}
	\setcounter{tempCounter}{\value{algocf}}
	% redefine the algorithm numbering (remove prefix)
	\renewcommand{\thealgocf}{}
	\begin{algorithm}
	}{
	\end{algorithm}
	% Restore the algorithm counter state
	\setcounter{algocf}{\value{tempCounter}}
}

\usepackage{adjustbox}
% Header and footer formatting
\pagestyle{fancy}
\fancyhead{}
\fancyhf{}
\rhead{\textcolor{TealBlue2}{\large\textbf{리만의 복소해석을 토대로 얻는 내 수학적 시야 (2기)}}}%\rule{3cm}{0.4pt}}
\lhead{\textcolor{TealBlue2}{\large\textbf{수학의 즐거움, Enjoying Math}}}
% Define footer
%\newcommand{\footer}[1]{
%\begin{flushright}
%	\vspace{2em}
%	\includegraphics[width=2.5cm]{school_logo.jpg} \\
%	\vspace{1em}
%	\textcolor{TealBlue2}{\small\textbf{#1}}
%\end{flushright}
%}
%\rfoot{\large Department of Information Security, Cryptogrphy and Mathematics, Kookmin Uni.\includegraphics[height=1.5cm]{school_logo.jpg}}
\fancyfoot{}
\fancyfoot[C]{-\thepage-}

\usepackage{animate}
% Load the PDF and grab its total pages into \NumPages:
\newcount\NumPagesA
\pdfximage{../riemann-tikz/secant_line_gif.pdf}% loads the PDF
\NumPagesA=\pdflastximagepages
\usepackage{tcolorbox}
\tcbset{colback=white, arc=5pt}

\definecolor{axiomcolor}{HTML}{a88bfa}
\definecolor{defcolor}{RGB}{52, 152, 219}
\definecolor{procolor}{RGB}{241, 196, 15}
\definecolor{thmcolor}{RGB}{231, 76, 60}
\definecolor{lemcolor}{RGB}{155, 89, 182}
\definecolor{corcolor}{RGB}{46, 204, 113}
\definecolor{execolor}{RGB}{90, 128, 127}

% Define a new command for the custom tcolorbox
\newcommand{\axiombox}[2][]{%
	\begin{tcolorbox}[colframe=axiomcolor, title={\color{white}\bfseries #1}]
		#2
	\end{tcolorbox}
}

\newcommand{\defbox}[2][]{%
	\begin{tcolorbox}[colframe=defcolor, title={\color{white}\bfseries #1}]
		#2
	\end{tcolorbox}
}

\newcommand{\probox}[2][]{%
	\begin{tcolorbox}[colframe=procolor, title={\color{white}\bfseries #1}]
		#2
	\end{tcolorbox}
}

\newcommand{\thmbox}[2][]{%
	\begin{tcolorbox}[colframe=thmcolor, title={\color{white}\bfseries #1}]
		#2
	\end{tcolorbox}
}

\newcommand{\lembox}[2][]{%
	\begin{tcolorbox}[colframe=lemcolor, title={\color{white}\bfseries #1}]
		#2
	\end{tcolorbox}
}
\usepackage{amsthm}

% Define custom theorem styles
\newtheoremstyle{dotless} % Name of the style
{3pt} % Space above
{3pt} % Space below
{\itshape} % Body font
{} % Indent amount
{\bfseries} % Theorem head font
{} % Punctuation after theorem head
{2.5mm} % Space after theorem head
{} % Theorem head spec

\newtheoremstyle{definitionstyle} % Name of the style
{3pt} % Space above
{3pt} % Space below
{} % Body font
{} % Indent amount
{\bfseries} % Theorem head font
{.} % Punctuation after theorem head
{2.5mm} % Space after theorem head
{} % Theorem head spec

% Applying custom styles
%\theoremstyle{dotless}
\newtheorem{theorem}{Theorem} % Theorem environment with section-wise numbering
\newtheorem*{theorem*}{Theorem} % Theorem environment with section-wise numbering
\newtheorem*{lemma*}{Lemma} % Theorem environment with section-wise numbering
\newtheorem*{proposition*}{Proposition} % Theorem environment with section-wise numbering
\newtheorem*{corollary*}{Corollary} % Theorem environment with section-wise numbering
\newtheorem{proposition}[theorem]{Proposition} % Theorem environment with section-wise numbering
\newtheorem{lemma}[theorem]{Lemma} % Lemma shares the counter with theorem
\newtheorem{corollary}[theorem]{Corollary} % Corollary shares the counter with theorem

\theoremstyle{definitionstyle}
\newtheorem*{observation}{\textcolor{magenta}{Observation}}
\newtheorem*{illustration}{\textcolor{teal}{Illustration}}
\newtheorem*{torus}{{\color{red}T}{\color{orange}o}{\color{green!75!black}r}{\color{cyan}u}{\color{violet}s}}
\newtheorem{definition}{Definition} % Definition shares the counter with theorem
\newtheorem{example}{Example} % Example shares the counter with theorem
\newtheorem{exercise}{{Exercise}} % Example shares the counter with theorem
\newtheorem{remark}{Remark} % Remark shares the counter with theorem
\newtheorem*{note}{Note}
\newtheorem*{notation}{Notation}

\newtheorem*{axiom*}{Axiom}
\newtheorem*{definition*}{Definition} % Definition shares the counter with theorem
\newtheorem*{example*}{Example} % Example shares the counter with theorem
\newtheorem*{exercise*}{\textcolor{teal}{Exercise}} % Example shares the counter with theorem
\newtheorem*{remark*}{Remark} % Remark shares the counter with theorem


\usepackage{tikz}
\usepackage{tikz-cd}
\usetikzlibrary{shadows}
\usetikzlibrary{shapes.geometric, arrows.meta, positioning}
\input{riemann-complex-commands}
\renewcommand{\vec}[1]{\mathbf{#1}}
\renewcommand{\emph}[1]{\textbf{#1}}
\renewcommand{\d}{\mathrm{d}} % For the exterior derivative 'd'
\newcommand{\pderiv}[2]{\frac{\partial #1}{\partial #2}}
\newcommand{\spderiv}[3]{\frac{\partial^2 #1}{\partial #2\partial #3}}
\newcommand{\vect}[1]{\begin{bmatrix} #1 \end{bmatrix}}
\newcommand{\ord}{\operatorname{ord}}

\newcommand{\circulationsquare}[1]{
	\draw[thick, gray] #1 rectangle ++(1,1);
	\begin{scope}[decoration={markings, mark=at position 0.5 with {\arrow{>}}}]
		\draw[postaction={decorate}, blue] #1 -- ++(1,0);
		\draw[postaction={decorate}, blue] ++(1,0) -- ++(0,1);
		\draw[postaction={decorate}, blue] ++(0,1) -- ++(-1,0);
		\draw[postaction={decorate}, blue] ++(-1,0) -- cycle;
	\end{scope}
}

\usepackage{esvect}
\usepackage{physics}

\setstretch{1.25}

\begin{document}
\pagenumbering{arabic}
\begin{center}
	\huge\textbf{Cartier Divisors, Invertible Sheaves, and Holomorphic Line Bundles}\\
%	\Large - HW1 -\\
	\vspace{0.5em}
	\large{Ji, Yong-hyeon}\\
%	\large{\ttfamily \url{https://github.com/Hacker-Code-J}}\\
	\vspace{0.5em}
	\normalsize{\today}\\
\end{center} 
\noindent 
We cover the following topics in this note.
\begin{itemize}
	\item Cartier Divisors
	\item Invertible Sheaves
	\item Holomorphic Line Bundles
\end{itemize}
%\begin{center}
%	\begin{tabular*}{\textwidth}{@{\extracolsep{\fill}} l c l}
%		\hline
%		\textbf{Vector Calculus (in $\R^2$ or $\R^3$)} & & \textbf{Differential Forms} \\
%		\hline
%		Vector Field $\vec F$ & $\iff$ & 1-form $\omega$ \\
%		Conservative Vector Field ($\vec F = \nabla f$) & $\iff$ & Exact 1-form ($\omega = \d f$) \\
%		Irrotational Vector Field ($\nabla \times \vec F = \mathbf{0}$) & $\iff$ & Closed 1-form ($\d\omega = 0$) \\
%		\hline
%	\end{tabular*}
%\end{center}
%
%\vspace{1em}
%
%\begin{framed}
%	\noindent\textbf{The Fundamental Implication:}
%	\begin{itemize}
%		\item \textbf{Conservative $\implies$ Irrotational},\; i.e.,\; \textbf{Exact $\implies$ Closed}\\ (This is always true.)
%		\item \textbf{Irrotational $\implies$ Conservative},\; i.e,\; \textbf{Closed $\implies$ Exact}\\ (This is only true on ``nice'' domains, e.g., simply connected ones.)
%	\end{itemize}
%\end{framed}


\tableofcontents
\newpage

\section{Cartier Divisors $D=\set{(U_\alpha, f_\alpha)}$}
% Cartier divisors as "winding-form integrals": a compact narrative

\paragraph{Slogan.}
\emph{A Cartier divisor is a coherent way to package local winding-form data
	(i.e.\ local orders) so it is independent of choices of local equations.}

\paragraph{Why Riemann surfaces and meromorphic functions?}
Let $X$ be a Riemann surface.
A holomorphic coordinate change $w=w(z)$ satisfies $w'(p)\neq 0$, so
vanishing multiplicity is coordinate-invariant.
Restricting to \emph{meromorphic} functions excludes essential singularities
(e.g.\ $e^{1/z}$), ensuring the order at a point is a \emph{finite integer}
(zeros and poles have finite order).

\paragraph{Order of a meromorphic function.}
Let $f\in \mathcal M_X(U)$ be nonzero and $p\in U$.
Choose a local coordinate $z$ with $z(p)=0$. Then there exists a unique $k\in\mathbb Z$
and a holomorphic $g$ on a neighborhood of $p$ with $g(p)\neq 0$ such that
\[
f(z)=z^k g(z).
\]
Define
\[
\ord_p(f):=k.
\]
Equivalently (winding form / logarithmic derivative), for a small positively oriented
loop $\gamma$ around $p$,
\[
\ord_p(f)=\frac{1}{2\pi i}\int_{\gamma}\frac{df}{f}.
\]

\paragraph{Units do not change order.}
If $h\in \mathcal O_X^*(V)$ is holomorphic and nowhere vanishing near $p$, then
$\ord_p(h)=0$ and hence
\[
\ord_p(hf)=\ord_p(h)+\ord_p(f)=\ord_p(f).
\]
Analytically: on a sufficiently small simply connected neighborhood $W\ni p$,
there exists a holomorphic logarithm $H$ with $e^{H}=h$, so
\[
\frac{dh}{h}=dH
\quad\Rightarrow\quad
\int_{\gamma}\frac{dh}{h}=0
\]
for every closed loop $\gamma\subset W$ (``FTC for holomorphic primitives'').

\paragraph{Weil divisors (recording integer data at points).}
A (Weil) divisor on $X$ is a finite formal sum
\[
D=\sum_{p\in X} n_p\,p,\qquad n_p\in\mathbb Z,\ \ n_p=0\text{ for all but finitely many }p.
\]
For a nonzero meromorphic function $f$, its \emph{principal divisor} is
\[
(f):=\sum_{p\in X}\ord_p(f)\,p.
\]

\paragraph{Cartier divisors (local equations with overlap coherence).}
A Cartier divisor is given by an open cover $\{U_\alpha\}$ of $X$ and meromorphic functions
$f_\alpha\in \mathcal M_X^*(U_\alpha)$ such that on overlaps
$U_{\alpha\beta}:=U_\alpha\cap U_\beta$ one has
\[
\frac{f_\alpha}{f_\beta}\in \mathcal O_X^*(U_{\alpha\beta}).
\]
This condition \emph{is exactly} what guarantees that the local orders glue:
for any $p\in U_{\alpha\beta}$,
\[
\ord_p(f_\alpha)=\ord_p\!\Bigl(\frac{f_\alpha}{f_\beta}\cdot f_\beta\Bigr)
=\ord_p\!\Bigl(\frac{f_\alpha}{f_\beta}\Bigr)+\ord_p(f_\beta)=\ord_p(f_\beta),
\]
since $\frac{f_\alpha}{f_\beta}$ is a unit.

\paragraph{Winding-form viewpoint on the overlap condition.}
On $U_{\alpha\beta}$,
\[
\frac{df_\alpha}{f_\alpha}-\frac{df_\beta}{f_\beta}
=\frac{d(f_\alpha/f_\beta)}{(f_\alpha/f_\beta)}.
\]
If $f_\alpha/f_\beta$ is a holomorphic unit, then locally it has a holomorphic logarithm,
so the right-hand side is exact; therefore its integral around a small loop is $0$.
Hence the winding integrals (and thus the orders) computed from $df_\alpha/f_\alpha$
and $df_\beta/f_\beta$ agree.

\paragraph{Conclusion.}
A Cartier divisor can be viewed as a collection of \emph{local winding/ordering data}
presented by meromorphic ``local equations'' $\{f_\alpha\}$, with the overlap rule
$f_\alpha/f_\beta\in\mathcal O_X^*$ ensuring the induced integer $\ord_p$ is
well-defined (choice-independent) at every point.

\newpage
\section{Invertible Sheaf $\mathcal{O}_X(D)$}
% Cartier divisor -> invertible sheaf O_X(D)

Let $X$ be a Riemann surface. A Cartier divisor $D$ is given by an open cover
$\{U_\alpha\}$ and $f_\alpha\in \mathcal M_X^*(U_\alpha)$ such that on overlaps
$U_{\alpha\beta}:=U_\alpha\cap U_\beta$ one has
\[
\frac{f_\alpha}{f_\beta}\in \mathcal O_X^*(U_{\alpha\beta}).
\]

\paragraph{Definition.}
Define a subsheaf $\mathcal O_X(D)\subset \mathcal M_X$ by declaring
\[
\mathcal O_X(D)\big|_{U_\alpha} := f_\alpha^{-1}\,\mathcal O_X\big|_{U_\alpha}.
\]
Equivalently, for $V\subset U_\alpha$,
\[
\mathcal O_X(D)(V)=\{\, s\in \mathcal M_X(V)\ :\ f_\alpha s\in \mathcal O_X(V)\,\}.
\]

\paragraph{Gluing on overlaps.}
On $U_{\alpha\beta}$,
\[
f_\alpha^{-1}\mathcal O_X
=
f_\beta^{-1}\left(\frac{f_\beta}{f_\alpha}\right)\mathcal O_X
=
f_\beta^{-1}\mathcal O_X,
\]
since $\frac{f_\beta}{f_\alpha}\in \mathcal O_X^*(U_{\alpha\beta})$. Hence
$\mathcal O_X(D)$ is well-defined globally.

\paragraph{Invertibility.}
Multiplication by $f_\alpha$ yields an isomorphism
\[
\mathcal O_X(D)\big|_{U_\alpha} \xrightarrow{\ \cdot f_\alpha\ } \mathcal O_X\big|_{U_\alpha},
\]
so $\mathcal O_X(D)$ is locally free of rank $1$, i.e.\ an invertible sheaf.

\paragraph{Transition functions.}
If $e_\alpha:=f_\alpha^{-1}$ is a local generator of $\mathcal O_X(D)$ on $U_\alpha$, then
on $U_{\alpha\beta}$,
\[
e_\alpha=\left(\frac{f_\beta}{f_\alpha}\right)e_\beta,
\]
so the transition functions are $g_{\alpha\beta}:=\frac{f_\beta}{f_\alpha}\in\mathcal O_X^*(U_{\alpha\beta})$.

\paragraph{Logarithmic differentials (winding-form view).}
On $U_{\alpha\beta}$,
\[
\frac{df_\alpha}{f_\alpha}-\frac{df_\beta}{f_\beta}
=
\frac{d(f_\alpha/f_\beta)}{(f_\alpha/f_\beta)}.
\]
Since $f_\alpha/f_\beta\in \mathcal O_X^*$, locally it admits a holomorphic logarithm,
so the right-hand side is exact and its integral over small loops is $0$.


\newpage
% Cartier divisors <-> invertible sheaves (Picard group) on a Riemann surface

\section{Cartier divisors and invertible sheaves: the precise dictionary}

Let $X$ be a Riemann surface.
Write $\mathcal O_X$ for the sheaf of holomorphic functions,
$\mathcal O_X^*$ for nowhere-vanishing holomorphic functions (units),
$\mathcal M_X$ for meromorphic functions, and $\mathcal M_X^*$ for nonzero meromorphic functions.

\subsection*{1) Cartier divisors as a quotient sheaf}
A \emph{Cartier divisor} can be defined as a global section of the quotient sheaf
\[
\mathcal M_X^*/\mathcal O_X^*.
\]
Concretely, it is given by a cover $\{U_\alpha\}$ and $f_\alpha\in \mathcal M_X^*(U_\alpha)$ such that
\[
\frac{f_\alpha}{f_\beta}\in \mathcal O_X^*(U_{\alpha\beta})
\quad\text{on }U_{\alpha\beta}:=U_\alpha\cap U_\beta,
\]
and two such collections $\{f_\alpha\}$ and $\{f'_\alpha\}$ define the same Cartier divisor iff
$f'_\alpha/f_\alpha\in \mathcal O_X^*(U_\alpha)$ for all $\alpha$.

\subsection*{2) From a Cartier divisor $D$ to the invertible sheaf $\mathcal O_X(D)$}
Given Cartier data $\{(U_\alpha,f_\alpha)\}$, define
\[
\mathcal O_X(D)\big|_{U_\alpha}:= f_\alpha^{-1}\,\mathcal O_X\big|_{U_\alpha}\subset \mathcal M_X\big|_{U_\alpha}.
\]
Equivalently, for $V\subset U_\alpha$,
\[
\mathcal O_X(D)(V)=\{\, s\in \mathcal M_X(V)\;:\; f_\alpha s\in \mathcal O_X(V)\,\}.
\]
On overlaps,
\[
f_\alpha^{-1}\mathcal O_X
=
f_\beta^{-1}\left(\frac{f_\beta}{f_\alpha}\right)\mathcal O_X
=
f_\beta^{-1}\mathcal O_X
\quad\text{since }\frac{f_\beta}{f_\alpha}\in \mathcal O_X^*,
\]
so the local definitions glue. Moreover multiplication by $f_\alpha$ gives an isomorphism
\[
\mathcal O_X(D)\big|_{U_\alpha}\xrightarrow{\ \cdot f_\alpha\ }\mathcal O_X\big|_{U_\alpha},
\]
hence $\mathcal O_X(D)$ is locally free of rank $1$, i.e.\ an invertible sheaf.

\subsection*{3) From an invertible sheaf to a Cartier divisor}
Let $\mathcal L$ be an invertible sheaf on $X$.
Choose a cover $\{U_\alpha\}$ and local frames $e_\alpha$ identifying
\[
\mathcal L|_{U_\alpha}\cong \mathcal O_X|_{U_\alpha}.
\]
On overlaps $U_{\alpha\beta}$ there are transition functions
\[
e_\alpha = g_{\alpha\beta}\, e_\beta,\qquad g_{\alpha\beta}\in \mathcal O_X^*(U_{\alpha\beta}),
\]
and $\{g_{\alpha\beta}\}$ is a \v{C}ech $1$-cocycle with values in $\mathcal O_X^*$.
Now pick meromorphic trivializations $s_\alpha$ of $\mathcal L$ on $U_\alpha$ (nonzero meromorphic sections);
write $s_\alpha = f_\alpha e_\alpha$ with $f_\alpha\in \mathcal M_X^*(U_\alpha)$.
Then on overlaps,
\[
\frac{f_\alpha}{f_\beta} = g_{\alpha\beta}\in \mathcal O_X^*(U_{\alpha\beta}),
\]
so $\{f_\alpha\}$ defines a Cartier divisor. (Intuitively: $\mathcal L$ becomes meromorphically trivial, and the
failure of holomorphic triviality is measured by the divisor.)

\subsection*{4) Principal divisors correspond to trivial line bundles}
If $D=(f)$ is principal (coming from a global $f\in \mathcal M_X^*(X)$), take $f_\alpha=f|_{U_\alpha}$.
Then
\[
\mathcal O_X((f)) = f^{-1}\mathcal O_X \cong \mathcal O_X
\]
via multiplication by $f$. Thus principal divisors map to the trivial bundle.

\subsection*{5) Picard group and the exact-sequence explanation}
The \emph{Picard group} is the group of isomorphism classes of invertible sheaves:
\[
\mathrm{Pic}(X):=\{\text{invertible sheaves on }X\}/\cong,
\]
with tensor product as the group law.

There is a short exact sequence of sheaves of abelian groups
\[
1\longrightarrow \mathcal O_X^*\longrightarrow \mathcal M_X^*
\longrightarrow \mathcal M_X^*/\mathcal O_X^*\longrightarrow 1.
\]
Taking cohomology yields a connecting homomorphism
\[
\delta:\ H^0\!\left(X,\mathcal M_X^*/\mathcal O_X^*\right)\longrightarrow H^1(X,\mathcal O_X^*).
\]
Here
\[
H^0\!\left(X,\mathcal M_X^*/\mathcal O_X^*\right)\cong \{\text{Cartier divisors on }X\},
\qquad
H^1(X,\mathcal O_X^*)\cong \mathrm{Pic}(X).
\]
Under these identifications, $\delta(D)$ is exactly the isomorphism class of the invertible sheaf $\mathcal O_X(D)$.

Moreover, $\ker(\delta)$ consists of divisors that come from global meromorphic functions (principal divisors),
so one gets the fundamental identification
\[
\mathrm{CaCl}(X):=\frac{\{\text{Cartier divisors}\}}{\{\text{principal Cartier divisors}\}}
\ \cong\ \mathrm{Pic}(X),
\qquad
[D]\longmapsto [\mathcal O_X(D)].
\]

\subsection*{6) How this matches your winding-form intuition}
On overlaps $U_{\alpha\beta}$, put $u_{\alpha\beta}:=f_\alpha/f_\beta\in \mathcal O_X^*$.
Then
\[
\frac{df_\alpha}{f_\alpha}-\frac{df_\beta}{f_\beta}=\frac{du_{\alpha\beta}}{u_{\alpha\beta}}.
\]
Locally (on a small simply connected set) $u_{\alpha\beta}=e^{h_{\alpha\beta}}$ for a holomorphic $h_{\alpha\beta}$, so
\[
\frac{du_{\alpha\beta}}{u_{\alpha\beta}}=dh_{\alpha\beta}
\]
is exact. Hence small-loop integrals of $\frac{df_\alpha}{f_\alpha}$ agree across overlaps,
so the local winding/order data glue. The same units $u_{\alpha\beta}$ are also exactly the
transition functions of the line bundle $\mathcal O_X(D)$:
\[
e_\alpha = u_{\beta\alpha}\, e_\beta
\quad\text{with}\quad u_{\beta\alpha}=f_\beta/f_\alpha\in \mathcal O_X^*.
\]

% End.

\newpage
\subsection*{Sheaves and the definition of $\mathcal O_X(D)$}

\paragraph{Definition (presheaf and sheaf).}
Let $X$ be a topological space. A \emph{presheaf} $\mathcal F$ (of sets / abelian groups / rings / modules)
assigns to each open set $U\subset X$ an object $\mathcal F(U)$ and to each inclusion $V\subset U$
a restriction map $\rho_{UV}:\mathcal F(U)\to \mathcal F(V)$, $s\mapsto s|_V$, such that
\[
s|_U=s,\qquad (s|_V)|_W=s|_W\quad (W\subset V\subset U).
\]
A presheaf is a \emph{sheaf} if for every open cover $U=\bigcup_i U_i$ the following hold:

\smallskip
\noindent\textbf{(S1) Locality.}
If $s,t\in \mathcal F(U)$ and $s|_{U_i}=t|_{U_i}$ for all $i$, then $s=t$.

\noindent\textbf{(S2) Gluing.}
If $s_i\in \mathcal F(U_i)$ satisfy $s_i|_{U_i\cap U_j}=s_j|_{U_i\cap U_j}$ for all $i,j$,
then there exists a unique $s\in \mathcal F(U)$ such that $s|_{U_i}=s_i$ for all $i$.

\paragraph{Setup.}
Let $X$ be a Riemann surface. Let $\mathcal O_X$ be the sheaf of holomorphic functions and
$\mathcal M_X$ the sheaf of meromorphic functions.
A Cartier divisor $D$ is given by an open cover $\{U_\alpha\}$ and $f_\alpha\in \mathcal M_X^*(U_\alpha)$
such that on overlaps $U_{\alpha\beta}:=U_\alpha\cap U_\beta$,
\[
\frac{f_\alpha}{f_\beta}\in \mathcal O_X^*(U_{\alpha\beta}).
\]

\paragraph{Definition (the sheaf $\mathcal O_X(D)$).}
Define a presheaf $\mathcal O_X(D)\subset \mathcal M_X$ by
\[
\mathcal O_X(D)(V)
:=
\Bigl\{\, s\in \mathcal M_X(V)\ :\ (f_\alpha s)\big|_{V\cap U_\alpha}\in \mathcal O_X(V\cap U_\alpha)\ \forall \alpha \,\Bigr\}
\]
for every open $V\subset X$. Restriction maps are the restrictions in $\mathcal M_X$:
$s\mapsto s|_W$ for $W\subset V$.

\paragraph{Lemma (well-defined on overlaps).}
On $V\cap U_{\alpha\beta}$, the condition $(f_\alpha s)\in\mathcal O_X$ is equivalent to
$(f_\beta s)\in\mathcal O_X$.

\paragraph{Proof.}
On $V\cap U_{\alpha\beta}$,
\[
f_\alpha s = \left(\frac{f_\alpha}{f_\beta}\right)(f_\beta s).
\]
Since $\frac{f_\alpha}{f_\beta}\in \mathcal O_X^*(U_{\alpha\beta})$, multiplication by this factor
preserves holomorphicity, hence $f_\alpha s$ is holomorphic iff $f_\beta s$ is holomorphic.
\qed

\paragraph{Proposition (sheaf property).}
The presheaf $\mathcal O_X(D)$ is a sheaf (indeed, an $\mathcal O_X$-module subsheaf of $\mathcal M_X$).

\paragraph{Proof.}
Let $V=\bigcup_i V_i$ be an open cover.

\smallskip
\noindent\emph{Locality (S1).}
If $s,t\in\mathcal O_X(D)(V)$ and $s|_{V_i}=t|_{V_i}$ for all $i$, then $s=t$ in $\mathcal M_X(V)$
because $\mathcal M_X$ is a sheaf. Hence $s=t$ in $\mathcal O_X(D)(V)$.

\smallskip
\noindent\emph{Gluing (S2).}
Let $s_i\in\mathcal O_X(D)(V_i)$ satisfy $s_i=s_j$ on $V_i\cap V_j$.
Since $\mathcal M_X$ is a sheaf, there exists a unique $s\in\mathcal M_X(V)$ with $s|_{V_i}=s_i$.
We claim $s\in\mathcal O_X(D)(V)$.

Fix $\alpha$. On each $V_i\cap U_\alpha$,
\[
(f_\alpha s)|_{V_i\cap U_\alpha}=f_\alpha(s|_{V_i\cap U_\alpha})=f_\alpha(s_i|_{V_i\cap U_\alpha}),
\]
and the right-hand side lies in $\mathcal O_X(V_i\cap U_\alpha)$ because $s_i\in\mathcal O_X(D)(V_i)$.
Moreover these holomorphic functions agree on overlaps because the $s_i$ agree.
Since $\mathcal O_X$ is a sheaf, they glue to a holomorphic function on $V\cap U_\alpha$.
Thus $(f_\alpha s)|_{V\cap U_\alpha}\in\mathcal O_X(V\cap U_\alpha)$ for all $\alpha$, i.e.\ $s\in\mathcal O_X(D)(V)$.

Uniqueness follows from uniqueness in $\mathcal M_X$.
Therefore $\mathcal O_X(D)$ satisfies (S1) and (S2), hence is a sheaf.
\qed

\newpage
%=========================================================
% Sheaf definition + O_X(D) on CP^1 (two-chart explicit)
%=========================================================

\subsection*{Sheaves and $\mathcal O_{\mathbf{CP}^1}(D)$ on $\mathbf{CP}^1$}

\paragraph{Definition (sheaf).}
Let $X$ be a topological space. A \emph{presheaf} $\mathcal F$ (of sets / abelian groups / rings / modules)
assigns to every open set $U\subset X$ an object $\mathcal F(U)$ and to every inclusion $V\subset U$
a restriction map
\[
\mathrm{res}_{U,V}:\mathcal F(U)\to \mathcal F(V),\qquad s\mapsto s|_V,
\]
such that $\mathrm{res}_{U,U}=\mathrm{id}$ and $\mathrm{res}_{V,W}\circ \mathrm{res}_{U,V}=\mathrm{res}_{U,W}$
for $W\subset V\subset U$.
A presheaf is a \emph{sheaf} if for every open cover $U=\bigcup_i U_i$:

\smallskip
\noindent\textbf{(S1) Locality.}
If $s,t\in \mathcal F(U)$ and $s|_{U_i}=t|_{U_i}$ for all $i$, then $s=t$.

\noindent\textbf{(S2) Gluing.}
If $s_i\in \mathcal F(U_i)$ satisfy $s_i|_{U_i\cap U_j}=s_j|_{U_i\cap U_j}$ for all $i,j$,
then there exists a unique $s\in \mathcal F(U)$ such that $s|_{U_i}=s_i$ for all $i$.

\paragraph{The standard cover of $\mathbf{CP}^1$.}
Let $X=\mathbf{CP}^1$ with the standard affine cover
\[
U_0=\{[z_0:z_1]\in\mathbf{CP}^1:\ z_0\neq 0\},\qquad
U_\infty=\{[z_0:z_1]\in\mathbf{CP}^1:\ z_1\neq 0\}.
\]
On $U_0$ use the coordinate $z=z_1/z_0\in\mathbf C$, and on $U_\infty$ use $w=z_0/z_1\in\mathbf C$.
On the overlap $U_0\cap U_\infty$ one has $w=1/z$ and $U_0\cap U_\infty\simeq \mathbf C^*$.

\paragraph{Setup (Cartier divisor).}
Let $D$ be a Cartier divisor on $X$ given by meromorphic functions
\[
f_0\in \mathcal M_X^*(U_0),\qquad f_\infty\in \mathcal M_X^*(U_\infty),
\]
such that on the overlap
\[
\frac{f_0}{f_\infty}\in \mathcal O_X^*(U_0\cap U_\infty).
\]

\paragraph{Definition (the presheaf $\mathcal O_X(D)$).}
For each open set $V\subset X$, define
\[
\mathcal O_X(D)(V)
:=
\Bigl\{
s\in \mathcal M_X(V):
(f_0 s)\big|_{V\cap U_0}\in\mathcal O_X(V\cap U_0),\ 
(f_\infty s)\big|_{V\cap U_\infty}\in\mathcal O_X(V\cap U_\infty)
\Bigr\}.
\]
Restriction maps are the restrictions in $\mathcal M_X$: for $W\subset V$,
$\mathrm{res}_{V,W}(s)=s|_W$.

\paragraph{Lemma (overlap consistency).}
On $V\cap U_0\cap U_\infty$, the condition $(f_0 s)\in\mathcal O_X$ is equivalent to
$(f_\infty s)\in\mathcal O_X$.

\paragraph{Proof.}
On $U_0\cap U_\infty$ we have $f_0 = u\, f_\infty$ where
$u:=f_0/f_\infty\in \mathcal O_X^*(U_0\cap U_\infty)$. Hence
\[
f_0 s = u\,(f_\infty s).
\]
Since $u$ is holomorphic and nowhere vanishing, multiplication by $u$ preserves holomorphicity.
Therefore $f_0 s$ is holomorphic iff $f_\infty s$ is holomorphic. \qed

\paragraph{Proposition (sheaf property on $\mathbf{CP}^1$).}
The presheaf $\mathcal O_X(D)$ is a sheaf (indeed, an $\mathcal O_X$-module subsheaf of $\mathcal M_X$).

\paragraph{Proof.}
Let $V=\bigcup_i V_i$ be an open cover.

\smallskip
\noindent\emph{Locality (S1).}
If $s,t\in\mathcal O_X(D)(V)$ and $s|_{V_i}=t|_{V_i}$ for all $i$, then $s=t$ in $\mathcal M_X(V)$
because $\mathcal M_X$ is a sheaf. Hence $s=t$ in $\mathcal O_X(D)(V)$.

\smallskip
\noindent\emph{Gluing (S2).}
Let $s_i\in\mathcal O_X(D)(V_i)$ satisfy $s_i=s_j$ on $V_i\cap V_j$.
Since $\mathcal M_X$ is a sheaf, there exists a unique $s\in\mathcal M_X(V)$ with $s|_{V_i}=s_i$.
We show $s\in\mathcal O_X(D)(V)$.

On $V\cap U_0$, the open cover $\{V_i\cap U_0\}_i$ satisfies
\[
(f_0 s)|_{V_i\cap U_0}=f_0(s|_{V_i\cap U_0})=f_0(s_i|_{V_i\cap U_0})\in \mathcal O_X(V_i\cap U_0),
\]
because $s_i\in\mathcal O_X(D)(V_i)$. These holomorphic functions agree on overlaps
(since the $s_i$ agree), hence glue (by the sheaf property of $\mathcal O_X$) to a holomorphic function
on $V\cap U_0$. Thus $(f_0 s)|_{V\cap U_0}\in \mathcal O_X(V\cap U_0)$.
The same argument on $V\cap U_\infty$ shows $(f_\infty s)|_{V\cap U_\infty}\in \mathcal O_X(V\cap U_\infty)$.
Therefore $s\in\mathcal O_X(D)(V)$.

Uniqueness is inherited from uniqueness in $\mathcal M_X$. \qed

\paragraph{Example: $D=n[\infty]$ gives $\mathcal O(n)$.}
Let $D=n[\infty]$. Take $f_0=1$ on $U_0$ and $f_\infty=w^n$ on $U_\infty$.
On $U_0\cap U_\infty$,
\[
\frac{f_0}{f_\infty}=\frac{1}{w^n}=z^n\in \mathcal O_X^*(\mathbf C^*),
\]
so this is Cartier data.
Then $s\in \mathcal O(n)(X)$ iff $s$ is holomorphic on $U_0\simeq \mathbf C$ and $w^n s$ is holomorphic
at $w=0$ (i.e.\ at $\infty$), hence $s$ is a polynomial in $z$ of degree $\le n$.
Consequently,
\[
H^0(\mathbf{CP}^1,\mathcal O(n))\cong \langle 1,z,\dots,z^n\rangle.
\]

\newpage
%=========================================================
% Why we consider f_\alpha s in the definition of O_X(D)
% (local frame + order/winding interpretation + CP^1 example)
%=========================================================

\subsection*{Why the condition $f_\alpha s\in \mathcal O_X$?}

\paragraph{Setup.}
Let $X$ be a Riemann surface and let $D$ be a Cartier divisor represented by
$\{(U_\alpha,f_\alpha)\}$ with $f_\alpha\in\mathcal M_X^*(U_\alpha)$ and
$f_\alpha/f_\beta\in\mathcal O_X^*(U_{\alpha\beta})$.

\paragraph{1) Local-frame (line bundle) viewpoint.}
On $U_\alpha$ define the local generator (frame)
\[
e_\alpha := f_\alpha^{-1}.
\]
The intended meaning of $\mathcal O_X(D)$ is: ``sections are holomorphic multiples of the local frame.''
Thus a section $s$ on $V\subset U_\alpha$ should be of the form
\[
s = h\cdot e_\alpha = h\cdot f_\alpha^{-1}
\quad\text{for some } h\in \mathcal O_X(V).
\]
Rearranging gives
\[
f_\alpha s = h\in\mathcal O_X(V).
\]
Hence the defining condition
\[
s\in\mathcal O_X(D)(V)\iff f_\alpha s\in\mathcal O_X(V)
\]
simply says: ``$s$ has holomorphic coefficient in the chosen local frame.''

\paragraph{2) Order / winding-form viewpoint.}
For a nonzero meromorphic function $g$ and a point $p$, let $\gamma_p$ be a small positively oriented loop.
Then
\[
\ord_p(g)=\frac{1}{2\pi i}\int_{\gamma_p}\frac{dg}{g}.
\]
Using $\dfrac{d(f_\alpha s)}{f_\alpha s}=\dfrac{df_\alpha}{f_\alpha}+\dfrac{ds}{s}$, we get additivity:
\[
\ord_p(f_\alpha s)=\ord_p(f_\alpha)+\ord_p(s).
\]
Now $f_\alpha s$ is holomorphic near $p$ iff $\ord_p(f_\alpha s)\ge 0$, so
\[
f_\alpha s\ \text{holomorphic near }p
\quad\Longleftrightarrow\quad
\ord_p(s)\ge -\ord_p(f_\alpha).
\]
Thus the condition $f_\alpha s\in\mathcal O_X$ is exactly the statement that the pole order of $s$
is bounded by the divisor $D$.

\paragraph{3) Independence of chart (overlap consistency).}
On $U_{\alpha\beta}$, write $f_\alpha=u_{\alpha\beta}f_\beta$ with $u_{\alpha\beta}\in\mathcal O_X^*(U_{\alpha\beta})$.
Then
\[
f_\alpha s\in\mathcal O_X
\iff
u_{\alpha\beta}(f_\beta s)\in\mathcal O_X
\iff
f_\beta s\in\mathcal O_X,
\]
since multiplication by a holomorphic unit preserves holomorphicity (and contributes zero winding/order).

\paragraph{Example on $\mathbf{CP}^1$: $D=n[\infty]$ and $\mathcal O(n)$.}
Let $X=\mathbf{CP}^1$ with charts $U_0\simeq \mathbf C$ (coordinate $z$) and $U_\infty\simeq \mathbf C$ (coordinate $w=1/z$).
For $D=n[\infty]$, take
\[
f_0=1\ \text{on }U_0,\qquad f_\infty=w^n\ \text{on }U_\infty.
\]
Then $s\in \mathcal O(n)(X)$ iff
\[
s \ \text{is holomorphic on }U_0\simeq\mathbf C,
\qquad
w^n s \ \text{is holomorphic at }w=0 \ (\text{i.e. at }\infty).
\]
Equivalently, $\ord_\infty(s)\ge -n$, so $s$ has a pole at $\infty$ of order at most $n$.
This forces $s$ to be a polynomial in $z$ of degree $\le n$, hence
\[
H^0(\mathbf{CP}^1,\mathcal O(n))\cong \langle 1,z,\dots,z^n\rangle.
\]

\end{document}
