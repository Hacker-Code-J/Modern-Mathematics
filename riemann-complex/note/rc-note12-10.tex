\documentclass[12pt]{article}
\usepackage{amsmath, amssymb, amsthm}
\usepackage{physics}
\usepackage{enumitem}
\usepackage{geometry}
\geometry{margin=1in}

\title{Why \(\partial Q/\partial x - \partial P/\partial y\) is the Curl in 2D}
\author{}
\date{}

\begin{document}
	\maketitle
	
	\section*{Local circulation density from a small rectangle}
	Let \(\vec F=\langle P(x,y),Q(x,y)\rangle\) be \(C^1\). Consider a small, axis-aligned rectangle
	centered at \((x_0,y_0)\) with side lengths \(\Delta x,\Delta y\). Its counterclockwise circulation is
	\[
	\oint_{\partial R}\vec F\cdot d\vec r
	=\int_{\text{right}}Q\,dy+\int_{\text{top}}P\,dx+\int_{\text{left}}Q\,dy+\int_{\text{bottom}}P\,dx.
	\]
	Taylor expand \(P,Q\) to first order along each edge and keep first-order terms. One obtains
	\[
	\oint_{\partial R}\vec F\cdot d\vec r
	\approx \big(Q_x(x_0,y_0)-P_y(x_0,y_0)\big)\,\Delta x\,\Delta y.
	\]
	Dividing by the area \(\Delta A=\Delta x\,\Delta y\) and shrinking the rectangle,
	\[
	\lim_{\Delta A\to 0}\frac{1}{\Delta A}\oint_{\partial R}\vec F\cdot d\vec r
	= Q_x(x_0,y_0)-P_y(x_0,y_0).
	\]
	Thus the scalar \(Q_x-P_y\) is the \emph{circulation per unit area}---the 2D curl.
	
	\section*{Green's theorem (global circulation)}
	If \(C=\partial D\) is a positively oriented simple closed curve enclosing a region \(D\),
	Green's theorem states
	\[
	\oint_{C}\vec F\cdot d\vec r
	=\iint_{D}\Big(\frac{\partial Q}{\partial x}-\frac{\partial P}{\partial y}\Big)\,dA.
	\]
	So the total circulation equals the area integral of the local circulation density.
	
	\section*{Example 1: Rigid rotation and angular velocity}
	Consider the rigid rotation field with angular speed \(\omega\):
	\[
	\vec F(x,y)=\langle -\omega y,\ \omega x\rangle.
	\]
	Then
	\[
	\frac{\partial Q}{\partial x}=\omega,\qquad \frac{\partial P}{\partial y}=-\omega
	\quad\Rightarrow\quad
	\operatorname{curl}\vec F = Q_x - P_y = 2\omega.
	\]
	This shows curl equals twice the angular velocity. For a circle of radius \(R\),
	parametrize \(r(t)=(R\cos t,R\sin t)\), \(dr=(-R\sin t,R\cos t)\,dt\). Then
	\[
	\oint \vec F\cdot d\vec r
	=\int_0^{2\pi}\omega R^2\,dt
	=2\pi\omega R^2.
	\]
	Meanwhile, \(\iint_{D}(2\omega)\,dA=2\omega \cdot \pi R^2=2\pi\omega R^2\), agreeing with Green's theorem.
	
	\section*{Example 2: Curl-free but not conservative (topology matters)}
	On \(\mathbb{R}^2\setminus\{(0,0)\}\), define
	\[
	\vec F(x,y)=\Big\langle -\frac{y}{x^2+y^2},\ \frac{x}{x^2+y^2}\Big\rangle.
	\]
	A direct calculation shows \(Q_x-P_y=0\) wherever defined (curl-free). However, the circulation
	around the unit circle is
	\[
	\oint \vec F\cdot d\vec r=2\pi\neq 0.
	\]
	Hence there is no global potential function; the puncture creates a topological obstruction.
	This illustrates that \(\operatorname{curl}\vec F=0\) captures \emph{local} rotation, while global
	circulation can persist in domains with holes.
	
	\section*{Summary checklist}
	\begin{itemize}[nosep]
		\item \(Q_x-P_y\) is the infinitesimal (per-area) circulation density.
		\item Green's theorem sums local curl to give total circulation.
		\item Rigid rotation: \(\operatorname{curl}=2\omega\) (twice angular velocity).
		\item Curl \(=0\) can still have nonzero loop integrals if the domain has holes.
	\end{itemize}
	
\end{document}
