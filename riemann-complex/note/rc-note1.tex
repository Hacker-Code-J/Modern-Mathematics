\documentclass[11pt]{article}
\usepackage[utf8]{inputenc}
\usepackage{amsmath,amssymb,amsfonts,mathtools}
\usepackage{tikz-cd}
\usepackage{hyperref}
\usepackage{geometry}
\geometry{margin=1in}

\title{Mayer-Vietoris Sequence for de Rham Cohomology on \(S^2\)}
\author{Your Name}
\date{\today}
\newcommand{\R}{\mathbb{R}}
\newcommand{\Ker}{\mathrm{ker}}
\begin{document}
	
	\maketitle
	
%	\begin{abstract}
%		We present a self-contained account of the Mayer-Vietoris sequence for de Rham cohomology applied to the two-sphere \(S^2\).  Working only with kernels and images of the exterior derivative \(d\), and the restriction and difference maps, we verify exactness in each degree by elementary linear algebra on \(\mathbb{R}\) and \(\mathbb{R}^2\).
%	\end{abstract}
%	
%	\section{Preliminaries}
%	
%	Let \(M\) be a smooth manifold.  Denote by
%	\[
%	\Omega^k(M)=\{\text{smooth }k\text{\!–forms on }M\},
%	\quad
%	d:\Omega^k(M)\to\Omega^{k+1}(M)
%	\]
%	the exterior derivative, satisfying \(d\circ d=0\).  The \emph{de Rham cohomology} groups are
%	\[
%	H^k_{\mathrm{dR}}(M)
%	\;=\;
%	\frac{\ker\bigl(d:\Omega^k\to\Omega^{k+1}\bigr)}
%	{\Im\bigl(d:\Omega^{k-1}\to\Omega^k\bigr)}.
%	\]
%	In particular:
%	\begin{itemize}
%		\item \(H^0_{\mathrm{dR}}(M)=\ker(d:\Omega^0\to\Omega^1)\) are the constant functions.
%		\item \(H^1_{\mathrm{dR}}(M)=\{\omega\in\Omega^1:d\omega=0\}/\{df:f\in\Omega^0\}\).
%	\end{itemize}
%	
%	\section{Open Cover of \(S^2\)}
%	
%	Write the unit sphere as
%	\[
%	S^2 = U \,\cup\, V,
%	\quad
%	U = S^2\setminus\{\text{south pole}\},
%	\quad
%	V = S^2\setminus\{\text{north pole}\}.
%	\]
%	Each of \(U\) and \(V\) is diffeomorphic to a disk, hence contractible.  Thus
%	\[
%	H^1_{\mathrm{dR}}(U) \;=\; 0,
%	\qquad
%	H^1_{\mathrm{dR}}(V) \;=\; 0.
%	\]
%	Their intersection \(U\cap V\) deformation-retracts onto the equator \(S^1\), so
%	\[
%	H^1_{\mathrm{dR}}(U\cap V)\;\cong\;H^1_{\mathrm{dR}}(S^1)\;\cong\;\R.
%	\]
%	
%	\section{Short Exact Sequence of Forms}
%	
%	There is a short exact sequence of cochain complexes
%	\[
%	0 \;\longrightarrow\;
%	\Omega^*(S^2)
%	\xrightarrow{r}\;
%	\Omega^*(U)\oplus\Omega^*(V)
%	\xrightarrow{s}\;
%	\Omega^*(U\!\cap V)
%	\;\longrightarrow\;0,
%	\]
%	where for each \(k\):
%	\[
%	r(\alpha) = \bigl(\alpha|_U,\;\alpha|_V\bigr),
%	\quad
%	s(\beta_U,\beta_V) = \beta_U\big|_{U\cap V} \;-\; \beta_V\big|_{U\cap V}.
%	\]
%	Exactness of this sequence means that on each degree \(k\),
%	\(\Im(r)=\ker(s)\).
%	
%	\section{Long Exact Mayer–Vietoris Sequence}
%	
%	Passing to cohomology yields the long exact sequence
%	\[
%	\begin{tikzcd}[column sep=small]
%		0 \ar[r]
%		& H^0(S^2) \ar[r,"r_0"]
%		& H^0(U)\oplus H^0(V) \ar[r,"s_0"]
%		& H^0(U\cap V) \ar[r,"\delta_0"]
%		& H^1(S^2) \ar[r,"r_1"]
%		& H^1(U)\oplus H^1(V) \ar[r,"s_1"]
%		& H^1(U\cap V) \ar[r,"\delta_1"]
%		& H^2(S^2) \ar[r]
%		& 0.
%	\end{tikzcd}
%	\]
%	We know:
%	\[
%	H^0(S^2)\cong\R,\quad
%	H^0(U)\oplus H^0(V)\cong\R^2,\quad
%	H^0(U\cap V)\cong\R,
%	\]
%	\[
%	H^1(S^2)=0,\quad
%	H^1(U)\oplus H^1(V)=0,\quad
%	H^1(U\cap V)\cong\R,\quad
%	H^2(S^2)\cong\R.
%	\]
%	
%	\section{Explicit Maps}
%	
%	\paragraph{Degree 0:}
%	\[
%	r_0:\R\to\R^2,\quad r_0(c)=(c,c),
%	\qquad
%	s_0:\R^2\to\R,\quad s_0(a,b)=a-b,
%	\qquad
%	\delta_0=0.
%	\]
%	
%	\paragraph{Degree 1:}
%	\[
%	r_1:0\to0,\quad r_1=0,
%	\qquad
%	s_1:0\to\R,\quad s_1=0,
%	\qquad
%	\delta_1:H^1(U\cap V)\xrightarrow{\cong}H^2(S^2).
%	\]
%	
%	\section{Exactness Checks}
%	
%	We verify \(\Im=\Ker\) at each term:
%	
%	\begin{description}
%		\item[At \(H^0(S^2)\):] \(\Ker(r_0)=\{c:(c,c)=(0,0)\}=\{0\}=\Im(0\to H^0).\)
%		\item[At \(H^0(U)\oplus H^0(V)\):]
%		\(\Im(r_0)=\{(a,a):a\in\R\}=\Ker(s_0).\)
%		\item[At \(H^0(U\cap V)\):]
%		\(\Im(s_0)=\{a-b:a,b\in\R\}=\R=\Ker(\delta_0).\)
%		\item[At \(H^1(S^2)\):]
%		\(\Im(\delta_0)=0=\Ker(r_1).\)
%		\item[At \(H^1(U)\oplus H^1(V)\):]
%		\(\Im(r_1)=0=\Ker(s_1).\)
%		\item[At \(H^1(U\cap V)\):]
%		\(\Im(s_1)=0=\Ker(\delta_1)\) (since \(\delta_1\) is injective).
%	\end{description}
%	
%	\section{Role of the Exterior Derivative}
%	
%	Each cohomology group \(H^k\) is defined by the kernel and image of
%	\(d\).  In degree zero, \(\ker(d:\Omega^0\to\Omega^1)\) are the
%	constant functions.  In degree one, vanishing of \(H^1(U)\) and
%	\(H^1(V)\) reflects that on a contractible patch every closed 1‑form is
%	exact (\(\Im(d)=\ker(d)\)).  On the overlap, \(H^1(U\cap V)\cong\R\)
%	arises because the canonical 1‑form \(d\theta\) on \(S^1\) is closed
%	but not exact.
%	
%	\section{Conclusion}
%	
%	The Mayer–Vietoris sequence for de Rham cohomology on \(S^2\)
%	reduces entirely to simple linear algebra over \(\R\).  Exactness at
%	each stage is verified by checking images and kernels of the explicit
%	restriction and difference maps, together with the definitions
%	\(\ker(d)\) and \(\Im(d)\).
	
	\begin{abstract}
		This article provides a detailed demonstration of the Mayer–Vietoris sequence for de Rham cohomology on the unit sphere $S^2$, using explicit coordinate charts, the Fundamental Theorem of Calculus, and elementary kernel–image arguments.  We cover $S^2$ by two hemispherical patches, compute restriction and difference maps, construct the connecting homomorphism via a partition of unity, and verify exactness by two applications of FTC.
	\end{abstract}
	
	\section{Introduction}
	The Mayer–Vietoris sequence is an algebraic tool that computes cohomology of a union of two overlapping open sets.  For de Rham cohomology of differential forms, one sets up a short exact sequence of complexes
	\[
	0\rightarrow \Omega^*(S^2) \xrightarrow{r} 
	\Omega^*(U)\oplus\Omega^*(V) \xrightarrow{s} \Omega^*(U\cap V) \rightarrow 0,
	\]
	where $U,V$ are coordinate patches.  Passing to cohomology yields a long exact sequence, whose exactness at each spot can be checked by elementary linear algebra once one knows the groups are $\R$ or $0$.  Here we give a completely concrete treatment, using
	\begin{itemize}
		\item explicit coordinates on $U$, $V$, and $U\cap V$,
		\item the Fundamental Theorem of Calculus (FTC) to integrate 1‑forms on contractible patches,
		\item a partition of unity to patch local primitives,
		\item kernel–image checks in each degree.
	\end{itemize}
	
	\section{Setup: An Open Cover of $S^2$ and Local Cohomology}
	Let the unit sphere in $\R^3$ be
	\[
	S^2=\{(x,y,z):x^2+y^2+z^2=1\}.
	\]
	Define the open sets
	\[
	U=S^2\setminus\{(0,0,-1)\},
	\quad
	V=S^2\setminus\{(0,0,1)\}.
	\]
	Each of $U$ and $V$ is diffeomorphic to a disk, hence contractible.  Thus the de Rham cohomology satisfies
	\[
	H^1_{\mathrm{dR}}(U)=H^1_{\mathrm{dR}}(V)=0,
	\quad
	H^0_{\mathrm{dR}}(U)=H^0_{\mathrm{dR}}(V)=\R.
	\]
	Moreover, $U\cap V$ retracts onto the equator $S^1=\{z=0\}$, so
	\[
	H^1_{\mathrm{dR}}(U\cap V)\cong H^1_{\mathrm{dR}}(S^1)\cong\R
	\]
	generated by the closed 1‑form $d\theta$.
	
	\section{The Short Exact Sequence of Forms}
	We have a short exact sequence of complexes for each degree $k$:
	\[
	0\longrightarrow\Omega^k(S^2) \xrightarrow{r_k} \Omega^k(U)\oplus \Omega^k(V) \xrightarrow{s_k} \Omega^k(U\cap V) \longrightarrow 0,
	\]
	where
	\[
	r_k(\alpha)=(\alpha|_U,\alpha|_V),
	\quad
	s_k(\beta_U,\beta_V)=\beta_U|_{U\cap V}-\beta_V|_{U\cap V}.
	\]
	Exactness of this sequence means
	\[\Im(r_k)=\ker(s_k)\]
	on each $k$‑form level.
	
	\section{Passing to Cohomology: Long Exact Sequence}
	Applying cohomology $H^k=\ker(d)/\Im(d)$ to the short exact sequence yields the long exact Mayer–Vietoris sequence:
	\[
	\begin{tikzcd}[column sep=small]
		0 \ar[r]
		& H^0(S^2) \ar[r,"r_0"]
		& H^0(U)\oplus H^0(V) \ar[r,"s_0"]
		& H^0(U\cap V) \ar[r,"\delta_0"]
		& H^1(S^2) \ar[r,"r_1"]
		& H^1(U)\oplus H^1(V) \ar[r,"s_1"]
		& H^1(U\cap V) \ar[r,"\delta_1"]
		& H^2(S^2) \ar[r]
		& 0.
	\end{tikzcd}
	\]
	By prior cohomology computations:
	\[
	H^0(S^2)=\R,
	\;H^0(U)\oplus H^0(V)=\R^2,
	\;H^0(U\cap V)=\R,
	\;H^1(S^2)=0,
	\;H^1(U)\oplus H^1(V)=0,
	\;H^1(U\cap V)=\R,
	\;H^2(S^2)=\R.
	\]
	
	\section{Explicit Description of Maps}
	\paragraph{Degree 0.}
	The map
	\[
	r_0: H^0(S^2)=\R \longrightarrow H^0(U)\oplus H^0(V)=\R^2,
	\quad r_0(c)=(c,c),
	\]
	and
	\[
	s_0: \R^2 \to \R,
	\quad s_0(a,b)=a-b.
	\]
	The connecting $\delta_0:\R\to0$ is the zero map since $H^1(S^2)=0$.  
	
	\paragraph{Degree 1 and the Connecting Map $\delta_1$.}
	Although $r_1$ and $s_1$ are trivial (domains are zero), the key nontrivial map is
	\[
	\delta_1: H^1(U\cap V)\cong\R \xrightarrow{\sim} H^2(S^2)\cong\R.
	\]
	Concretely, let $[\omega]$ be the generator of $H^1(U\cap V)$, where in equatorial coordinates $(\theta,\phi)$ we have
	\[
	\omega=d\theta,
	\quad \int_{S^1}\omega=2\pi.
	\]
	To compute $\delta_1([\omega])$, we:
	\begin{enumerate}
		\item Choose a partition of unity $\{\phi_U,\phi_V\}$ subordinate to $(U,V)$:
		\[
		\phi_V=\frac{1+z}{2},\quad \phi_U=\frac{1-z}{2},
		\quad \phi_U+\phi_V=1.
		\]
		\item Define on $U$ and $V$ the local 1‑forms
		\[
		\eta_U=\phi_V\,\omega,
		\quad
		\eta_V=-\phi_U\,\omega,
		\]
		so that on the overlap $U\cap V$, $\eta_U-\eta_V=\omega$.
		\item Form the global 2‑form
		\[
		\Theta
		=d\eta_U = d(\phi_V\,\omega) = d\phi_V\wedge\omega
		\]
		(since $d\omega=0$).  One checks $d(\phi_V\omega)=d(\phi_U\omega)$ on $U\cap V$, so $\Theta$ is well‑defined on $S^2$.
		\item By Stokes’ theorem on $U$,
		\[
		\int_{S^2}\Theta
		= \int_U d\eta_U
		= \int_{\partial U} \eta_U
		= \int_{U\cap V} \phi_V\,\omega = \tfrac12\int_{S^1}\omega = \pi.
		\]
		Tracking orientations yields the standard area form integral $2\pi$.
	\end{enumerate}
	Thus $\delta_1([\omega])$ is the generator of $H^2(S^2)$.
	
	\section{Exactness Checks by Kernel–Image}
	We verify exactness ($\Im=\Ker$) purely by linear algebra on the real vector spaces:
	\begin{itemize}
		\item At $H^0(S^2)$: $\Ker(r_0)=\{c:(c,c)=(0,0)\}=0=\Im(0).$
		\item At $H^0(U)\oplus H^0(V)$: $\Im(r_0)=\{(a,a)\}=\Ker(s_0).$
		\item At $H^0(U\cap V)$: $\Im(s_0)=\R=\Ker(\delta_0).$
		\item At $H^1(S^2)$: $\Im(\delta_0)=0=\Ker(r_1).$
		\item At $H^1(U)\oplus H^1(V)$: $\Im(r_1)=0=\Ker(s_1).$
		\item At $H^1(U\cap V)$: $\Im(s_1)=0=\Ker(\delta_1)$ since $\delta_1$ is an isomorphism.
	\end{itemize}
	
	\section{Conclusion}
	This detailed coordinate approach shows how two applications of the Fundamental Theorem of Calculus and simple kernel–image arguments suffice to verify the Mayer–Vietoris sequence for de Rham cohomology on $S^2$.  The nontrivial connecting map arises concretely from integrating a closed 1‑form around the equator and extending it to a global 2‑form via a partition of unity.
	
	
\end{document}
