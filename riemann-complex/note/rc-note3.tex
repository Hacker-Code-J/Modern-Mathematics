\documentclass[11pt]{article}
\usepackage[utf8]{inputenc}
\usepackage[T1]{fontenc}
\usepackage{lmodern}
\usepackage{amsmath,amssymb}
\usepackage{geometry}
\geometry{margin=1in}

% Title and author information
\title{A Calculus-First, Step-by-Step Lecture on the Mayer--Vietoris Sequence for $\Omega^1$ on $S^2$}
\author{Your Name}
\date{\today}

\begin{document}
	\maketitle
	
	\begin{abstract}
		We develop the Mayer--Vietoris argument for de~Rham $1$-forms on the unit sphere $S^2$ using only first-year calculus (single-variable and double integrals) and basic vector calculus.  In particular:
		\begin{itemize}
			\item Review of antiderivatives: $\displaystyle\int\sin t\,dt$, $\displaystyle\int\cos t\,dt$.
			\item Gradient theorem (line integral around the unit circle equals $2\pi$).
			\item Surface integral of curl via iterated integrals (value $27/4$).
			\item Detailed Mayer--Vietoris: gluing local antiderivatives on overlapping hemispheres using two FTC steps in $\phi$ and $\theta$.
		\end{itemize}
	\end{abstract}
	
	\section{1. Fundamental Theorem of Calculus (FTC) Revisited}
	Let $F(t)$ be differentiable with continuous derivative on $[a,b]$.  The FTC states:
	\[
	\int_{a}^{b}F'(t)\,dt = F(b)-F(a).
	\]
	Thus any continuous $f(t)$ admits an antiderivative $F(t)$ with $F'(t)=f(t)$, giving
	\[
	\int f(t)\,dt = F(t) + C.
	\]
	
	\subsection*{Examples}
	\begin{enumerate}
		\item $f(t)=\sin t$.  Since $\frac{d}{dt}(-\cos t)=\sin t$, we have by FTC
		\[
		\int_{a}^{b}\sin t\,dt = [-\cos t]_{a}^{b} = -\cos b + \cos a,
		\]
		and thus
		\[
		\int \sin t\,dt = -\cos t + C,
		\quad d(-\cos t) = \sin t\,dt.
		\]
		\item $f(t)=\cos t$.  Since $\frac{d}{dt}(\sin t)=\cos t$, we have
		\[
		\int_{a}^{b}\cos t\,dt = [\sin t]_{a}^{b} = \sin b - \sin a,
		\]
		so
		\[
		\int \cos t\,dt = \sin t + C,
		\quad d(\sin t) = \cos t\,dt.
		\]
	\end{enumerate}
	
	\section{2. Line Integrals and the Gradient Theorem}
	Given a vector field $\mathbf{F}(x,y)=(P(x,y),Q(x,y))$ and a parameterized curve $C: \mathbf{r}(t)=(x(t),y(t))$, $t\in[\alpha,\beta]$, the line integral is
	\[
	\int_{C}\mathbf{F}\cdot d\mathbf{r} = \int_{\alpha}^{\beta}\bigl(P(x(t),y(t))x'(t) + Q(x(t),y(t))y'(t)\bigr)\,dt.
	\]
	If $\mathbf{F}=\nabla F$ is a gradient field, then
	\[
	\int_{C}\nabla F\cdot d\mathbf{r} = F\bigl(\mathbf{r}(\beta)\bigr) - F\bigl(\mathbf{r}(\alpha)\bigr).
	\]
	
	\subsection*{Example: Circulation around the Unit Circle}
	Let $C$ be the unit circle parameterized by
	\[
	\mathbf{r}(t) = (\cos t,\sin t), \quad t\in[0,2\pi],
	\]
	and consider the field
	\[
	\mathbf{F}(x,y) = \Bigl(-\tfrac{y}{x^2+y^2},\;\tfrac{x}{x^2+y^2}\Bigr).
	\]
	On $C$, we have
	\[
	x'(t)=-\sin t, \quad y'(t)=\cos t, \quad P=-\sin t, \quad Q=\cos t.
	\]
	Hence the integrand simplifies to
	\[
	P\,x' + Q\,y' = (-\sin t)(-\sin t)+(\cos t)(\cos t)=\sin^2t+\cos^2t=1.
	\]
	By the FTC,
	\[
	\oint_{C}\mathbf{F}\cdot d\mathbf{r} = \int_{0}^{2\pi}1\,dt = 2\pi.
	\]
	This demonstrates that $\mathbf{F}$ is closed but not exact on $\mathbb{R}^2\setminus\{0\}$.
	
	\section{3. Surface Integral of Curl (Stokes' Theorem)}
	Stokes' theorem states
	\[
	\iint_{S}(\nabla\times\mathbf{F})\cdot d\mathbf{S} = \oint_{\partial S}\mathbf{F}\cdot d\mathbf{r}.
	\]
	We compute the left side directly by iterated integrals.
	
	\subsection*{Example: Constant Curl on a Square}
	Let $S$ be the unit square in the $z=0$ plane, $(u,v)\in[0,1]^2$, and suppose
	\[
	(\nabla\times\mathbf{F})\cdot\mathbf{n} = 27\,u\,v,
	\quad \mathbf{n}=(0,0,1),
	\]
	so $dS=du\,dv$.  Then
	\[
	\iint_{S}(\nabla\times\mathbf{F})\cdot d\mathbf{S} = \int_{0}^{1}\int_{0}^{1}27\,u\,v\,du\,dv.
	\]
	Compute step by step using FTC:
	\begin{align*}
		\int_{0}^{1}u\,du &= \Bigl[\tfrac{u^2}{2}\Bigr]_{0}^{1}=\tfrac12,\\%
		\int_{0}^{1}v\,dv &= \tfrac12.
	\end{align*}
	Thus
	\[
	27\times\tfrac12\times\tfrac12 = \frac{27}{4}.
	\]
	
	\section{4. Mayer--Vietoris for $\Omega^1$ on $S^2$}
	Cover $S^2$ by two open hemispheres:
	\[
	U = \{(x,y,z)\in S^2 : z>0\}, \quad V = \{(x,y,z)\in S^2 : z<0\}.
	\]
	Their overlap $U\cap V$ contains the equator $S^1$.
	
	The de~Rham complex begins
	\[
	0 \longrightarrow \Omega^0(S^2) \xrightarrow{d} \Omega^1(S^2) \xrightarrow{d} \Omega^2(S^2) \longrightarrow \cdots.
	\]
	Restriction gives a short exact sequence in degree~1:
	\[
	0 \longrightarrow \Omega^1(S^2) \xrightarrow{r_1}
	\Omega^1(U)\oplus\Omega^1(V) \xrightarrow{s_1} \Omega^1(U\cap V) \longrightarrow 0.
	\]
	Exactness at $\Omega^1(U)\oplus\Omega^1(V)$ means
	\[
	\mathrm{Im}(r_1)=\ker(s_1)
	\quad\Longleftrightarrow\quad
	(\beta_U,\beta_V)\text{ glue to a global }\alpha
	\iff \beta_U|_{U\cap V}=\beta_V|_{U\cap V}.
	\]
	
	\subsection*{Step-by-Step Gluing via FTC}
	\paragraph{1.~Spherical Coordinates.}
	On each hemisphere use coordinates $(\phi,\theta)$:
	\[
	x=\sin\phi\cos\theta,\quad y=\sin\phi\sin\theta,\quad z=\cos\phi,
	\]
	with $\phi\in(0,\pi)$, $\theta\in[0,2\pi)$.
	
	\paragraph{2.~Local Potential on $U$ (FTC in $\phi$).}
	Write
	\[
	\beta_U = P_U(\phi,\theta)\,d\phi + Q_U(\phi,\theta)\,d\theta.
	\]
	For fixed $\theta$, define
	\[
	F_U(\phi,\theta)=\int_{\phi_0}^{\phi}P_U(s,\theta)\,ds.
	\]
	By FTC, $\partial_{\phi}F_U = P_U$, so
	\[
	dF_U = P_U\,d\phi + \frac{\partial F_U}{\partial \theta}\,d\theta.
	\]
	
	\paragraph{3.~Overlap and Difference.}
	Similarly define $F_V(\phi,\theta)$ on $V$.  On $U\cap V$, $\beta_U=\beta_V$ implies
	\[
	F_U-F_V=H(\theta)
	\]
	for some function $H$ of $\theta$ alone.
	
	\paragraph{4.~FTC in $\theta$ Forces $H$ Constant.}
	Differentiate:
	\[
	H'(\theta)=\partial_{\theta}F_U-\partial_{\theta}F_V=Q_U-Q_V=0
	\]
	so $H$ is constant.  By adding this constant to $F_V$, we achieve $F_U=F_V$ on the overlap.
	
	\paragraph{5.~Gluing to a Global Potential.}
	Define
	\[
	F(p)=\begin{cases}F_U(p),&p\in U,\\F_V(p),&p\in V,\end{cases}
	\]
	which is smooth and satisfies $dF|_U=\beta_U$, $dF|_V=\beta_V$.  Hence $(\beta_U,\beta_V)\in\mathrm{Im}(r_1)$.
	
	This completes the Mayer--Vietoris gluing using only two one-dimensional FTC applications, fully accessible to first-year calculus students.
	
\end{document}