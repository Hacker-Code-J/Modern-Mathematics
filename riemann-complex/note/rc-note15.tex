\documentclass[11pt]{article}

\usepackage{amsmath, amssymb, amsthm}
\usepackage{geometry}
\usepackage{hyperref}
\usepackage{mathrsfs}

\geometry{a4paper, margin=1in}

\title{Classifying Complex Elliptic Curves via the Period Map}
\author{Joy of Mathematics}
\date{}

\newtheorem{definition}{Definition}[section]
\newtheorem{theorem}[definition]{Theorem}
\newtheorem{proposition}[definition]{Proposition}
\newtheorem{remark}[definition]{Remark}

\begin{document}
	
	\maketitle
	
	\tableofcontents
	
	\section{Complex Elliptic Curves and Complex Tori}
	
	The goal of this note is to define the periods of a complex elliptic curve as explicit integrals and, using these, to describe the period map into the upper half-plane $\mathbb{H}$ with concrete computations.
	
	\begin{definition}[Complex Elliptic Curve]\label{def:elliptic-curve}
		A \emph{complex elliptic curve} (or elliptic curve over $\mathbb{C}$) is a pair $(E, 0)$ satisfying:
		\begin{enumerate}
			\item $E$ is a one-dimensional complex analytic manifold (that is, a compact Riemann surface), and
			\item $E$ is equipped with the structure of a complex Lie group, with distinguished identity element $0 \in E$.
		\end{enumerate}
		Usually one also regards $E$ as a smooth projective algebraic curve over $\mathbb{C}$.
	\end{definition}
	
	A classical fact in the theory of elliptic curves is that every complex elliptic curve is (analytically) isomorphic to a complex torus $\mathbb{C}/\Lambda$ for a suitable lattice $\Lambda \subset \mathbb{C}$.
	
	\begin{theorem}[Classification of Complex Elliptic Curves]\label{thm:classification}
		For any complex elliptic curve $E$ there exists a lattice
		\[
		\Lambda = \mathbb{Z}\omega_1 \oplus \mathbb{Z}\omega_2 \subset \mathbb{C}
		\]
		such that $E$ is analytically isomorphic to the complex torus
		\[
		E \cong \mathbb{C}/\Lambda.
		\]
		Conversely, for any lattice $\Lambda \subset \mathbb{C}$ of rank $2$ (i.e.\ $\omega_1, \omega_2 \in \mathbb{C}$ are $\mathbb{R}$-linearly independent), the complex torus $\mathbb{C}/\Lambda$ carries a natural structure of an elliptic curve.
	\end{theorem}
	
	Here $\omega_1, \omega_2 \in \mathbb{C}$ are $\mathbb{R}$-linearly independent so that $\Lambda$ is a discrete subgroup of rank $2$ in $\mathbb{C}$.
	
	
	\section{Holomorphic 1-Forms and the Definition of Periods}
	
	To define periods, we use holomorphic $1$-forms on an elliptic curve.
	
	\begin{proposition}
		Let $E$ be a complex elliptic curve. Then there exists a nonzero holomorphic $1$-form $\omega$ on $E$. Moreover, the space
		\[
		H^0(E, \Omega_E^1)
		\]
		of holomorphic $1$-forms on $E$ is a one-dimensional complex vector space.
	\end{proposition}
	
	In particular, any nonzero holomorphic $1$-form $\omega$ spans this space, and every other holomorphic $1$-form is a complex scalar multiple of $\omega$.
	
	Next we recall that, topologically, a complex elliptic curve is a real two-dimensional torus $T^2$. Thus its first (singular) homology group with integer coefficients is
	\[
	H_1(E, \mathbb{Z}) \cong \mathbb{Z}^2.
	\]
	
	\begin{definition}[Marking]\label{def:marking}
		A \emph{marking} of a complex elliptic curve $E$ is the choice of a basis
		\[
		\alpha, \beta \in H_1(E, \mathbb{Z})
		\]
		of the first homology group together with an orientation convention such that the intersection number satisfies
		\[
		\langle \alpha, \beta \rangle = 1.
		\]
		Such a basis $(\alpha, \beta)$ is also called a \emph{symplectic basis} of $H_1(E, \mathbb{Z})$.
	\end{definition}
	
	\begin{definition}[Periods]\label{def:periods}
		Let $\omega$ be a nonzero holomorphic $1$-form on $E$, and let $(\alpha, \beta)$ be a marking as above. Define the \emph{periods} of $\omega$ by
		\[
		\tau_1 := \int_{\alpha} \omega, \qquad
		\tau_2 := \int_{\beta} \omega.
		\]
		These are complex numbers, and the $\mathbb{Z}$-span
		\[
		\Lambda = \langle \tau_1, \tau_2 \rangle_{\mathbb{Z}}
		\]
		is a lattice in $\mathbb{C}$.
	\end{definition}
	
	A crucial fact is that $\tau_1$ and $\tau_2$ are $\mathbb{R}$-linearly independent. In particular, the ratio
	\[
	\frac{\tau_2}{\tau_1}
	\]
	is not real; in fact, it lies in the upper half-plane
	\[
	\mathbb{H} = \{ \tau \in \mathbb{C} \mid \mathrm{Im}\,\tau > 0 \}.
	\]
	We now verify this by a standard computation.
	
	
	\section{Why the Period Lies in the Upper Half-Plane $\mathbb{H}$: Detailed Computation}
	
	Let $\omega$ be a nonzero holomorphic $1$-form on $E$. Locally we can write
	\[
	\omega = f(z)\,dz
	\]
	for some holomorphic function $f$ in a local coordinate $z$. Consider the $2$-form
	\[
	i\,\omega \wedge \overline{\omega}.
	\]
	Since $E$ is a compact Riemann surface, this provides a positive volume form on $E$, and hence
	\[
	\int_E i\,\omega \wedge \overline{\omega} > 0.
	\]
	This integral is essentially the area of $E$ with respect to the metric induced by $\omega$.
	
	On the other hand, we can compute the same integral in terms of the periods of $\omega$ and the marking $(\alpha, \beta)$. This is given by a special case of the Riemann bilinear relations.
	
	\begin{proposition}\label{prop:riemann-bilinear}
		Let $(\alpha, \beta)$ be a marking on $E$, and let $\tau_1, \tau_2$ be the periods
		\[
		\tau_1 = \int_\alpha \omega, \qquad
		\tau_2 = \int_\beta \omega.
		\]
		Then
		\[
		\int_E i\,\omega \wedge \overline{\omega}
		= 2\,\mathrm{Im}(\tau_1 \overline{\tau_2}).
		\]
		In particular,
		\[
		\mathrm{Im}\left( \frac{\tau_2}{\tau_1} \right) > 0.
		\]
	\end{proposition}
	
	\begin{proof}[Idea of the proof]
		The general Riemann bilinear relation for a compact Riemann surface of genus $g$ says that for holomorphic $1$-forms $\omega, \eta$ one has
		\[
		\int_E \omega \wedge \eta
		=
		\sum_{j=1}^g 
		\left(
		\int_{\alpha_j} \omega \int_{\beta_j} \eta
		-
		\int_{\beta_j} \omega \int_{\alpha_j} \eta
		\right),
		\]
		where $(\alpha_j, \beta_j)_{j=1}^g$ is a symplectic basis of $H_1(E, \mathbb{Z})$.
		
		In our case, the genus is $g=1$, so we choose $\alpha_1 = \alpha$ and $\beta_1 = \beta$. Taking $\eta = \overline{\omega}$ and using that $\overline{\int_\gamma \omega} = \int_\gamma \overline{\omega}$, we obtain
		\[
		\int_E \omega \wedge \overline{\omega}
		=
		\left(
		\int_\alpha \omega \int_\beta \overline{\omega}
		-
		\int_\beta \omega \int_\alpha \overline{\omega}
		\right)
		=
		\tau_1 \overline{\tau_2} - \tau_2 \overline{\tau_1}
		=
		2i\,\mathrm{Im}(\tau_1 \overline{\tau_2}).
		\]
		Multiplying by $i$ yields
		\[
		\int_E i\,\omega \wedge \overline{\omega}
		= 2\,\mathrm{Im}(\tau_1 \overline{\tau_2}) > 0.
		\]
		Thus $\mathrm{Im}(\tau_1 \overline{\tau_2}) > 0$, and
		\[
		\mathrm{Im}\left(\frac{\tau_2}{\tau_1}\right)
		=
		\frac{\mathrm{Im}(\tau_1 \overline{\tau_2})}{|\tau_1|^2}
		> 0.
		\]
		Hence the ratio
		\[
		\tau := \frac{\tau_2}{\tau_1}
		\]
		lies in the upper half-plane $\mathbb{H}$.
	\end{proof}
	
	We call this complex number $\tau$ (depending on the data $(E,\omega,\alpha,\beta)$) the \emph{period} of the elliptic curve (more precisely, the period with respect to the given marking and $1$-form).
	
	
	\section{Normalization and the Complex Torus $\mathbb{C}/(\mathbb{Z} + \mathbb{Z}\tau)$}
	
	From the discussion above, $\tau_1$ is a nonzero complex number. We may thus \emph{normalize} the holomorphic $1$-form by dividing by $\tau_1$.
	
	Define a new holomorphic $1$-form
	\[
	\omega' := \frac{\omega}{\tau_1}.
	\]
	Then the periods of $\omega'$ with respect to the same marking $(\alpha,\beta)$ satisfy
	\[
	\int_\alpha \omega' = 1, \qquad
	\int_\beta \omega' = \tau,
	\]
	where
	\[
	\tau = \frac{\tau_2}{\tau_1} \in \mathbb{H}.
	\]
	
	\begin{definition}[Normalized Period and Standard Lattice]\label{def:normalized-period}
		Choose a holomorphic $1$-form $\omega'$ on $E$ such that
		\[
		\int_{\alpha} \omega' = 1.
		\]
		Let
		\[
		\tau := \int_{\beta} \omega' \in \mathbb{H}.
		\]
		We call this $\tau$ the \emph{normalized period} of $E$ (relative to the marking), and we define the associated standard lattice
		\[
		\Lambda_{\tau} = \mathbb{Z} \cdot 1 \oplus \mathbb{Z} \cdot \tau \subset \mathbb{C}.
		\]
	\end{definition}
	
	A classical theorem then states that the elliptic curve $E$ (with the marking and normalized $1$-form) is analytically isomorphic to the complex torus $\mathbb{C}/\Lambda_{\tau}$.
	
	More precisely, for the $\tau$ obtained from $(E,\omega',\alpha,\beta)$ as in Definition \ref{def:normalized-period}, there is an analytic isomorphism
	\[
	E \cong \mathbb{C}/\Lambda_{\tau}
	\]
	such that under the quotient map $\pi : \mathbb{C} \to \mathbb{C}/\Lambda_{\tau} \cong E$, the form $\omega'$ corresponds to the standard $1$-form $dz$ on $\mathbb{C}$, i.e.
	\[
	\pi^* \omega' = dz.
	\]
	
	\begin{remark}
		The normalization condition $\int_{\alpha} \omega' = 1$ fixes the scale of the $1$-form and provides a standard reference. Under this normalization, the entire information of the marked elliptic curve is encoded in the single complex number $\tau \in \mathbb{H}$.
	\end{remark}
	
	
	\section{Definition of the Period Map}
	
	We now define the moduli space of \emph{marked} complex elliptic curves with a nonzero holomorphic $1$-form.
	
	\begin{definition}[Moduli of Marked Elliptic Curves]
		Let $\mathcal{E}$ be the set of isomorphism classes of data
		\[
		(E, \omega, \alpha, \beta),
		\]
		where:
		\begin{itemize}
			\item $E$ is a complex elliptic curve,
			\item $0 \neq \omega \in H^0(E, \Omega_E^1)$ is a nonzero holomorphic $1$-form on $E$,
			\item $(\alpha, \beta)$ is a symplectic basis of $H_1(E,\mathbb{Z})$.
		\end{itemize}
		Two such tuples $(E_1,\omega_1,\alpha_1,\beta_1)$ and $(E_2,\omega_2,\alpha_2,\beta_2)$ are considered equivalent if there exists a biholomorphism
		\[
		f : E_1 \longrightarrow E_2
		\]
		such that
		\[
		f^* \omega_2 = \omega_1,\quad
		f_* (\alpha_1) = \alpha_2,\quad
		f_* (\beta_1) = \beta_2.
		\]
	\end{definition}
	
	Given such a tuple $(E,\omega,\alpha,\beta)$, we can define its period as the normalized ratio
	\[
	\tau = \frac{\displaystyle\int_{\beta} \omega}{\displaystyle\int_{\alpha} \omega} \in \mathbb{H}.
	\]
	
	\begin{definition}[Period Map]\label{def:period-map}
		The \emph{period map} is the map
		\[
		\mathrm{Per} : \mathcal{E} \longrightarrow \mathbb{H}
		\]
		defined by
		\[
		\mathrm{Per}(E,\omega,\alpha,\beta)
		= 
		\frac{\displaystyle\int_{\beta} \omega}{\displaystyle\int_{\alpha} \omega}.
		\]
	\end{definition}
	
	\begin{theorem}[Biholomorphism of the Period Map]\label{thm:Per-biholo}
		The period map
		\[
		\mathrm{Per} : \mathcal{E} \longrightarrow \mathbb{H}
		\]
		is a biholomorphism. In particular, the moduli space $\mathcal{E}$ of marked elliptic curves with a holomorphic $1$-form is analytically isomorphic to the upper half-plane $\mathbb{H}$.
	\end{theorem}
	
	\begin{proof}[Idea of the proof]
		For each $\tau \in \mathbb{H}$, consider the standard complex torus
		\[
		E_{\tau} := \mathbb{C}/(\mathbb{Z} + \mathbb{Z}\tau).
		\]
		On $\mathbb{C}$ we have the standard $1$-form $dz$, which descends to a holomorphic $1$-form $\omega_{\tau}$ on $E_{\tau}$. Take the standard symplectic basis $(\alpha,\beta)$ of $H_1(E_{\tau},\mathbb{Z})$ given by:
		\begin{itemize}
			\item $\alpha$: the cycle corresponding to the direction of $1 \in \mathbb{C}$,
			\item $\beta$: the cycle corresponding to the direction of $\tau \in \mathbb{C}$.
		\end{itemize}
		Then by construction,
		\[
		\int_{\alpha} \omega_{\tau} = 1, \qquad
		\int_{\beta} \omega_{\tau} = \tau,
		\]
		so that
		\[
		\mathrm{Per}(E_{\tau}, \omega_{\tau}, \alpha, \beta) = \tau.
		\]
		This shows that $\mathrm{Per}$ is surjective.
		
		Conversely, given an arbitrary $(E,\omega,\alpha,\beta)$, we can normalize $\omega$ so that $\int_{\alpha} \omega = 1$. The corresponding normalized period is
		\[
		\tau = \int_{\beta} \omega \in \mathbb{H}.
		\]
		A classical result states that $E$ is analytically isomorphic to $E_{\tau}$ in such a way that $\omega$ corresponds to $\omega_{\tau}$ up to scaling, and the marking $(\alpha,\beta)$ corresponds to the standard marking on $E_{\tau}$. This shows that $\mathrm{Per}$ is injective. The dependence on $(E,\omega,\alpha,\beta)$ is analytic, so $\mathrm{Per}$ is a biholomorphism.
	\end{proof}
	
	
	\section{\texorpdfstring{$\mathrm{SL}_2(\mathbb{Z})$}{SL(2,Z)}-Action and the Unmarked Moduli Space}
	
	So far, we have kept track of a marking $(\alpha,\beta)$. If we ``forget'' the marking and keep only the underlying elliptic curve $E$, then there is an additional freedom: we may choose a different symplectic basis of $H_1(E,\mathbb{Z})$.
	
	Changing the symplectic basis $(\alpha,\beta)$ is equivalent to acting by an element of the group $\mathrm{SL}_2(\mathbb{Z})$. Concretely, let
	\[
	\gamma =
	\begin{pmatrix}
		a & b \\
		c & d
	\end{pmatrix}
	\in \mathrm{SL}_2(\mathbb{Z}),
	\]
	and define a new basis $(\alpha',\beta')$ by
	\[
	\begin{pmatrix}
		\alpha' \\
		\beta'
	\end{pmatrix}
	=
	\begin{pmatrix}
		a & b \\
		c & d
	\end{pmatrix}
	\begin{pmatrix}
		\alpha \\
		\beta
	\end{pmatrix}.
	\]
	Then $(\alpha',\beta')$ is again a symplectic basis of $H_1(E,\mathbb{Z})$.
	
	Let $\omega$ be a holomorphic $1$-form on $E$, and let
	\[
	\tau_1 = \int_{\alpha} \omega, \quad \tau_2 = \int_{\beta} \omega
	\]
	be its periods. The periods with respect to $(\alpha',\beta')$ are then
	\[
	\int_{\alpha'} \omega = a \tau_1 + b \tau_2, \qquad
	\int_{\beta'} \omega = c \tau_1 + d \tau_2.
	\]
	If we normalize $\omega'$ so that $\int_{\alpha'} \omega' = 1$, the new normalized period is
	\[
	\tau'
	=
	\frac{\displaystyle\int_{\beta'} \omega'}{\displaystyle\int_{\alpha'} \omega'}
	=
	\frac{c\tau_1 + d\tau_2}{a\tau_1 + b\tau_2}
	=
	\frac{c + d\tau}{a + b\tau},
	\]
	where $\tau = \tau_2/\tau_1$ is the previous normalized period.
	
	Thus we obtain the familiar fractional linear (Möbius) transformation
	\[
	\tau' = \gamma \cdot \tau := \frac{a\tau + b}{c\tau + d}.
	\]
	
	\begin{remark}
		Changing the marking $(\alpha,\beta)$ corresponds exactly to the action of $\mathrm{SL}_2(\mathbb{Z})$ on the upper half-plane $\mathbb{H}$ via fractional linear transformations
		\[
		\tau \longmapsto \frac{a\tau + b}{c\tau + d}.
		\]
		Hence, when we forget the marking and consider elliptic curves up to isomorphism \emph{without} additional data, the moduli space is
		\[
		\{\text{complex elliptic curves up to isomorphism}\}
		\;\;\cong\;\;
		\mathbb{H}/\mathrm{SL}_2(\mathbb{Z}),
		\]
		which is the classical moduli space of elliptic curves.
	\end{remark}
	
	
	\section{Weierstrass Equation and a Concrete Example of Period Integrals}
	
	We conclude with a standard example of how the period is realized as a concrete integral for an elliptic curve in Weierstrass form.
	
	Consider the elliptic curve given by the Weierstrass equation
	\[
	E: \quad y^2 = 4x^3 - g_2 x - g_3,
	\]
	where $g_2, g_3 \in \mathbb{C}$ are constants such that the cubic polynomial on the right-hand side has distinct roots (so that $E$ is nonsingular).
	
	Let $\wp(z)$ be the Weierstrass $\wp$-function associated to a lattice $\Lambda \subset \mathbb{C}$, and let $\wp'(z)$ denote its derivative. Then the corresponding elliptic curve $\mathbb{C}/\Lambda$ can be parametrized by
	\[
	x = \wp(z), \qquad y = \wp'(z),
	\]
	and this parametrization satisfies the Weierstrass equation
	\[
	\wp'(z)^2 = 4\wp(z)^3 - g_2 \wp(z) - g_3.
	\]
	
	On this curve, consider the $1$-form
	\[
	\omega = \frac{dx}{y}.
	\]
	
	\begin{proposition}\label{prop:weierstrass-period}
		With $x = \wp(z)$ and $y = \wp'(z)$ as above, we have
		\[
		\omega = \frac{dx}{y} = dz.
		\]
		Therefore, the periods of $\omega$ along the basic cycles of $\mathbb{C}/\Lambda$ coincide exactly with the generators of the lattice $\Lambda$.
	\end{proposition}
	
	\begin{proof}
		Since $y = \wp'(z) = \dfrac{d}{dz}\wp(z)$, we have
		\[
		dx = d(\wp(z)) = \wp'(z)\,dz.
		\]
		Substituting into $\omega = \dfrac{dx}{y}$, we get
		\[
		\omega
		=
		\frac{\wp'(z)\,dz}{\wp'(z)} = dz.
		\]
		Thus the pullback of $\omega$ to $\mathbb{C}$ via $z \mapsto (\wp(z),\wp'(z))$ is just $dz$. If $\alpha, \beta$ are the basic cycles corresponding to generators $\omega_1,\omega_2$ of the lattice $\Lambda$, then
		\[
		\int_{\alpha} \omega = \int_{\alpha} dz = \omega_1,
		\qquad
		\int_{\beta} \omega = \int_{\beta} dz = \omega_2.
		\]
		Hence the periods of $\omega$ are exactly the generators of $\Lambda$.
	\end{proof}
	
	This example explicitly shows how the period lattice of the holomorphic $1$-form $\omega = \dfrac{dx}{y}$ is identified with the lattice $\Lambda$ defining the complex torus $\mathbb{C}/\Lambda$. It illustrates concretely the relationship between periods and the complex torus structure of an elliptic curve.
	
\end{document}
