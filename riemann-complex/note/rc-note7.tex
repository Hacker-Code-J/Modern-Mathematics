\documentclass[12pt, letterpaper]{article}
\usepackage{amsmath, amssymb, amsfonts, amsthm}
\usepackage[margin=1in]{geometry}
\usepackage{xcolor}
\usepackage{hyperref}

\definecolor{darkblue}{rgb}{0.0, 0.0, 0.55}
\definecolor{darkred}{rgb}{0.55, 0.0, 0.0}

\hypersetup{
	colorlinks=true,
	linkcolor=darkblue,
	citecolor=darkred,
	urlcolor=darkblue,
}

\theoremstyle{definition}
\newtheorem{defn}{Definition}[section]
\newtheorem{exmp}{Example}[section]
\newtheorem{idea}{Core Correspondence}[section]

\title{\bfseries A Bridge Between Worlds: \\ \large From Gradients and Jacobians to Differential Forms}
\author{A Unified View of Vector Calculus}
\date{\today}
\newcommand{\R}{\mathbb{R}}
\begin{document}
	\maketitle
	
	\begin{abstract}
		This lecture provides a detailed mathematical bridge between the familiar concepts of multivariable calculus—the gradient of a scalar function and the Jacobian matrix of a vector function—and the powerful, modern language of differential forms. We will formally establish the precise relationship between these objects, demonstrating that they are different representations of the same underlying mathematical structures. The goal is to understand the correspondences:
		\[
		\underbrace{f}_{\Omega^0}
		\;\xrightarrow{d}\;
		\underbrace{df}_{\Omega^1}
		\;\longleftrightarrow\;
		\underbrace{\nabla f}_{\substack{\text{gradient}\\\text{vector field}}}
		\qquad \text{and} \qquad
		\underbrace{\mathbf F}_{(\Omega^0)^m}
		\;\xrightarrow{d}\;
		\underbrace{d\mathbf F}_{\Omega^1\otimes\R^m}
		\;\longleftrightarrow\;
		\underbrace{D\mathbf F}_{\substack{\text{Jacobian}\\\text{matrix}}}
		\]
	\end{abstract}
	
	\section{Part 1: The Scalar Case -- From Functions to Gradients}
	
	We begin by establishing the fundamental link between a scalar function, its differential, and its gradient vector field.
	
	\subsection{Functions as 0-Forms}
	
	In the language of differential geometry, a smooth (infinitely differentiable) scalar function is called a \textbf{0-form}.
	
	\begin{defn}[0-Form]
		Let $U$ be an open subset of $\mathbb{R}^n$. The space of smooth functions $f: U \to \mathbb{R}$ is denoted by $\Omega^0(U)$. Each such function $f$ is a 0-form.
	\end{defn}
	
	\begin{exmp}
		Consider the function $f(x, y, z) = x^2 \sin(y) + e^z$ on $\mathbb{R}^3$. This is a 0-form, so we can write $f \in \Omega^0(\mathbb{R}^3)$.
	\end{exmp}
	
	\subsection{The Exterior Derivative and 1-Forms}
	The universal operator for differentiation in this context is the \textbf{exterior derivative}, $d$. When applied to a 0-form, it produces the function's total differential.
	
	\begin{defn}[Exterior Derivative on 0-Forms]
		For a 0-form $f \in \Omega^0(\mathbb{R}^n)$, its exterior derivative $df$ is the 1-form defined by:
		\begin{equation*}
			df = \frac{\partial f}{\partial x_1} dx_1 + \frac{\partial f}{\partial x_2} dx_2 + \cdots + \frac{\partial f}{\partial x_n} dx_n = \sum_{i=1}^n \frac{\partial f}{\partial x_i} dx_i
		\end{equation*}
		The resulting object, $df$, is an element of the space of 1-forms, denoted $\Omega^1(\mathbb{R}^n)$.
	\end{defn}
	
	\begin{exmp}
		For our 0-form $f(x, y, z) = x^2 \sin(y) + e^z$, its exterior derivative is the 1-form:
		\begin{equation*}
			df = (2x \sin y) \, dx + (x^2 \cos y) \, dy + (e^z) \, dz
		\end{equation*}
	\end{exmp}
	
	\subsection{The Correspondence with the Gradient Vector Field}
	Now, we connect this to the familiar gradient from vector calculus.
	
	\begin{defn}[Gradient Vector Field]
		For a scalar function $f \in C^\infty(\mathbb{R}^n)$, its gradient, $\nabla f$, is the vector field whose components are the partial derivatives of $f$:
		\begin{equation*}
			\nabla f = \left\langle \frac{\partial f}{\partial x_1}, \frac{\partial f}{\partial x_2}, \dots, \frac{\partial f}{\partial x_n} \right\rangle
		\end{equation*}
	\end{defn}
	
	By comparing the definitions, the relationship becomes clear.
	
	\begin{idea}[The $df \longleftrightarrow \nabla f$ Correspondence]
		The 1-form $df$ and the gradient vector field $\nabla f$ are two different representations of the same first-derivative information of the function $f$.
		\begin{equation*}
			df = \underbrace{\frac{\partial f}{\partial x_1}}_{(\nabla f)_1} dx_1 + \underbrace{\frac{\partial f}{\partial x_2}}_{(\nabla f)_2} dx_2 + \cdots + \underbrace{\frac{\partial f}{\partial x_n}}_{(\nabla f)_n} dx_n
		\end{equation*}
		The components of the gradient vector field are precisely the coefficient functions of the 1-form $df$. This establishes a canonical isomorphism between the space of gradient vector fields and the space of exact 1-forms.
	\end{idea}
	
	\section{Part 2: The Vector Case -- From Mappings to Jacobians}
	
	Next, we generalize this correspondence to functions that map between higher-dimensional spaces, i.e., vector fields or vector-valued functions.
	
	\subsection{Vector Fields as Tuples of 0-Forms}
	A smooth map from $\mathbb{R}^n$ to $\mathbb{R}^m$ is a vector field. It can be viewed as a column vector (or tuple) of $m$ scalar functions, where each function is a 0-form.
	
	\begin{defn}[Vector-Valued Function]
		A smooth function $\mathbf{F}: \mathbb{R}^n \to \mathbb{R}^m$ is given by $m$ component functions, $F_1, F_2, \dots, F_m$, where each $F_i: \mathbb{R}^n \to \mathbb{R}$ is a 0-form in $\Omega^0(\mathbb{R}^n)$. We can write:
		\begin{equation*}
			\mathbf{F}(\mathbf{x}) = \begin{pmatrix} F_1(\mathbf{x}) \\ F_2(\mathbf{x}) \\ \vdots \\ F_m(\mathbf{x}) \end{pmatrix}
		\end{equation*}
		We denote the space of such maps as $(C^\infty(\mathbb{R}^n))^m$ or, more suggestively, $(\Omega^0)^m$.
	\end{defn}
	
	\subsection{Applying the Exterior Derivative Component-wise}
	We apply the exterior derivative $d$ to the vector field $\mathbf{F}$ by applying it to each of its component 0-forms individually.
	
	\begin{defn}[Exterior Derivative on Vector Fields]
		For a vector field $\mathbf{F} = (F_1, \dots, F_m)$, its exterior derivative $d\mathbf{F}$ is the vector of 1-forms:
		\begin{equation*}
			d\mathbf{F} = \begin{pmatrix} dF_1 \\ dF_2 \\ \vdots \\ dF_m \end{pmatrix}
		\end{equation*}
		Each $dF_i$ is a 1-form in $\Omega^1(\mathbb{R}^n)$. The resulting object $d\mathbf{F}$ is an element of the space $\Omega^1(\mathbb{R}^n) \otimes \mathbb{R}^m$.
	\end{defn}
	
	\begin{exmp}
		Let $\mathbf{F}: \mathbb{R}^2 \to \mathbb{R}^3$ be defined by $\mathbf{F}(x,y) = \begin{pmatrix} x^2y \\ x-y^2 \end{pmatrix}$.
		The component functions are $F_1(x,y) = x^2y$ and $F_2(x,y) = x-y^2$. Applying $d$:
		\begin{align*}
			dF_1 &= \frac{\partial F_1}{\partial x} dx + \frac{\partial F_1}{\partial y} dy = 2xy \, dx + x^2 \, dy \\
			dF_2 &= \frac{\partial F_2}{\partial x} dx + \frac{\partial F_2}{\partial y} dy = 1 \, dx - 2y \, dy
		\end{align*}
		So, the derivative of the vector field is the vector of 1-forms:
		\begin{equation*}
			d\mathbf{F} = \begin{pmatrix} 2xy \, dx + x^2 \, dy \\ dx - 2y \, dy \end{pmatrix}
		\end{equation*}
	\end{exmp}
	
	\subsection{The Correspondence with the Jacobian Matrix}
	The Jacobian matrix provides a compact and powerful way to organize the coefficients of the 1-forms in $d\mathbf{F}$.
	
	\begin{defn}[Jacobian Matrix]
		For a map $\mathbf{F}: \mathbb{R}^n \to \mathbb{R}^m$, its Jacobian matrix $D\mathbf{F}$ (or $J_\mathbf{F}$) is the $m \times n$ matrix of first-order partial derivatives, where the entry in the $i$-th row and $j$-th column is $\frac{\partial F_i}{\partial x_j}$.
		\begin{equation*}
			D\mathbf{F} = \begin{pmatrix}
				\frac{\partial F_1}{\partial x_1} & \frac{\partial F_1}{\partial x_2} & \cdots & \frac{\partial F_1}{\partial x_n} \\
				\frac{\partial F_2}{\partial x_1} & \frac{\partial F_2}{\partial x_2} & \cdots & \frac{\partial F_2}{\partial x_n} \\
				\vdots & \vdots & \ddots & \vdots \\
				\frac{\partial F_m}{\partial x_1} & \frac{\partial F_m}{\partial x_2} & \cdots & \frac{\partial F_m}{\partial x_n}
			\end{pmatrix}
		\end{equation*}
	\end{defn}
	
	The connection between $d\mathbf{F}$ and $D\mathbf{F}$ is revealed when we write $d\mathbf{F}$ using matrix multiplication.
	
	\begin{idea}[The $d\mathbf{F} \longleftrightarrow D\mathbf{F}$ Correspondence]
		The vector of 1-forms $d\mathbf{F}$ can be expressed as the product of the Jacobian matrix $D\mathbf{F}$ and the column vector of basis differentials $d\mathbf{x}$.
		\begin{align*}
			d\mathbf{F} = \begin{pmatrix} dF_1 \\ \vdots \\ dF_m \end{pmatrix}
			&= \begin{pmatrix} \sum_{j=1}^n \frac{\partial F_1}{\partial x_j} dx_j \\ \vdots \\ \sum_{j=1}^n \frac{\partial F_m}{\partial x_j} dx_j \end{pmatrix} \\
			&= \underbrace{
				\begin{pmatrix}
					\frac{\partial F_1}{\partial x_1} & \cdots & \frac{\partial F_1}{\partial x_n} \\
					\vdots & \ddots & \vdots \\
					\frac{\partial F_m}{\partial x_1} & \cdots & \frac{\partial F_m}{\partial x_n}
				\end{pmatrix}
			}_{\text{Jacobian Matrix } D\mathbf{F}}
			\underbrace{
				\begin{pmatrix} dx_1 \\ \vdots \\ dx_n \end{pmatrix}
			}_{\text{Vector of Differentials } d\mathbf{x}}
		\end{align*}
		This shows that the Jacobian matrix $D\mathbf{F}$ is the matrix representation of the linear map that takes an infinitesimal displacement vector $d\mathbf{x}$ in the domain to a corresponding infinitesimal vector $d\mathbf{F}$ in the codomain. The Jacobian matrix is the ``best linear approximation'' to the function $\mathbf{F}$ at a point, and the language of forms makes this relationship explicit.
	\end{idea}
	
	\section{Conclusion}
	We have established two fundamental correspondences that translate the language of vector calculus into the language of differential forms.
	\begin{enumerate}
		\item The \textbf{gradient vector field $\nabla f$} of a scalar function is the vector representation whose components are the coefficients of the \textbf{1-form $df$}.
		\item The \textbf{Jacobian matrix $D\mathbf{F}$} of a vector function is the matrix representation of the linear transformation described by the \textbf{vector of 1-forms $d\mathbf{F}$}.
	\end{enumerate}
	This framework is powerful because it is independent of coordinates and generalizes to any dimension. It forms the foundation for unifying all of vector calculus (including curl and divergence) under the single, elegant umbrella of the exterior calculus of differential forms.
	
\end{document}
