\documentclass[11pt]{article}
\usepackage[margin=1in]{geometry}
\usepackage{amsmath,amssymb,amsthm,mathtools}
\usepackage{mathrsfs}

\title{Why Nonconstant Holomorphic Functions Don't Exist on Compact Riemann Surfaces\\
	\large A proof via the winding (logarithmic) form}
\author{}
\date{}

\theoremstyle{definition}
\newtheorem{definition}{Definition}
\theoremstyle{plain}
\newtheorem{theorem}{Theorem}
\newtheorem{lemma}{Lemma}
\newtheorem{proposition}{Proposition}
\theoremstyle{remark}
\newtheorem*{remark}{Remark}

\newcommand{\C}{\mathbb{C}}

\begin{document}
	\maketitle
	
	\section*{0. Setup and the winding form}
	Let \(X\) be a compact Riemann surface (connected, without boundary). 
	If \(f:X\to\C\) is holomorphic and \emph{nonvanishing} on some open set, we may define on that open set the \emph{logarithmic derivative} (winding form)
	\[
	\omega_f \ :=\ d\log f \ =\ \frac{f'}{f}\,dz,
	\]
	where \(z\) is any local holomorphic coordinate. This is a well-defined meromorphic \(1\)-form on \(X\) with the following basic property:
	
	\begin{lemma}[Local argument principle]
		If \(p\in X\) is a zero of \(f\) of order \(m\ge 1\) (so locally \(f(z)=u(z)(z-p)^m\) with \(u(p)\neq 0\)), then
		\[
		\operatorname{Res}_p\!\left(\frac{f'}{f}\,dz\right)=m.
		\]
		If \(p\) is a pole of order \(n\ge 1\) (so \(f(z)=u(z)(z-p)^{-n}\) with \(u(p)\neq 0\)), then
		\[
		\operatorname{Res}_p\!\left(\frac{f'}{f}\,dz\right)=-\,n.
		\]
	\end{lemma}
	
	\begin{proof}
		Write \(f(z)=u(z)(z-p)^m\) with \(u\) holomorphic and nowhere zero; then
		\[
		\frac{f'}{f}=\frac{u'}{u}+\frac{m}{z-p},
		\]
		so the residue is \(m\). The pole case is identical with \(m=-n\).
	\end{proof}
	
	\section*{1. Residue theorem on a compact surface}
	On any compact Riemann surface, the sum of residues of a meromorphic \(1\)-form is zero:
	
	\begin{proposition}[Residue theorem, global form]\label{prop:residue}
		If \(\eta\) is a meromorphic \(1\)-form on a compact Riemann surface \(X\), then
		\[
		\sum_{p\in X}\operatorname{Res}_p(\eta)=0.
		\]
	\end{proposition}
	
	\begin{proof}[Idea]
		Cover \(X\) by coordinate discs avoiding poles and small punctured discs around each pole; apply Stokes' theorem on the complement of the punctures and use that small positively oriented circles around the punctures bound with negative orientation; the boundary integrals are \(2\pi i\) times the residues and sum to \(0\).
	\end{proof}
	
	\section*{2. The winding-form proof of Liouville on compact surfaces}
	\begin{theorem}\label{thm:main}
		If \(X\) is compact and \(f:X\to\C\) is holomorphic, then \(f\) is constant.
	\end{theorem}
	
	\begin{proof}[Proof via the winding (logarithmic) form]
		Consider \(\omega_f = \dfrac{f'}{f}\,dz\), which is meromorphic on \(X\). Since \(f\) is holomorphic on \(X\), it has \emph{no poles}; thus, by the local argument principle, the only possible residues of \(\omega_f\) occur at the zeros of \(f\), and each such residue equals the zero's multiplicity (a nonnegative integer).
		
		By the residue theorem (Proposition~\ref{prop:residue}),
		\[
		\sum_{p\in X}\operatorname{Res}_p(\omega_f)=0.
		\]
		But each term is \(\ge 0\). Hence every term must be \(0\), so \(f\) has \emph{no zeros}. Therefore \(\omega_f\) is in fact holomorphic everywhere on \(X\) (no poles anywhere).
		
		Now take real parts:
		\[
		d\log|f| \ =\ \Re\!\left(\frac{f'}{f}\,dz\right).
		\]
		Since \(\omega_f\) is holomorphic, its real part is a \emph{closed} \(1\)-form; locally it is the differential of the harmonic function \(\log|f|\). Globally, \(\log|f|\) is thus a harmonic function on the \emph{compact} surface \(X\). A harmonic function on a compact manifold attains its max/min and hence is constant. Therefore \(|f|\) is constant on \(X\).
		
		Finally, if \(|f|\equiv c>0\) is constant and \(f\) is holomorphic, then \(f\overline{f}=c^2\). Differentiating in \(z\),
		\[
		0=\partial_z(f\overline{f})=f'\,\overline{f},
		\]
		so \(f'\equiv 0\) and \(f\) is constant.
	\end{proof}
	
	\begin{remark}[Equivalent phrasing via the argument principle]
		Equivalently, on a compact surface, the divisor of a meromorphic function has degree \(0\):
		\[
		\sum_{p}\operatorname{ord}_p(f)=0.
		\]
		When \(f\) is holomorphic (no poles), this forces \(\operatorname{ord}_p(f)=0\) for all \(p\), i.e. no zeros. Then \(\log|f|\) is globally harmonic and must be constant, hence \(f\) is constant.
	\end{remark}
	
	\section*{3. Two standard one-line proofs (for comparison)}
	\paragraph{(A) Maximum modulus principle.}
	A continuous image of a compact space is compact, so \(|f|\) attains a maximum on \(X\). A holomorphic function with an interior maximum is constant. Hence \(f\) is constant.
	
	\paragraph{(B) Open mapping + compactness.}
	If \(f\) is nonconstant holomorphic, it is open; thus \(f(X)\) is an open subset of \(\C\). But \(X\) is compact, so \(f(X)\) is compact. The only subset of \(\C\) that is both open and compact is empty; contradiction. Hence \(f\) is constant.
	
	\section*{4. Intuition: winding numbers and \(\frac{dz}{z}\)}
	On \(\C^\times\), the \emph{winding form} \(\dfrac{dz}{z}\) satisfies
	\[
	\frac{1}{2\pi i}\int_\gamma \frac{dz}{z} \in \mathbb{Z},
	\]
	the winding number of \(\gamma\) about \(0\).
	For a holomorphic \(f\), \(\dfrac{f'}{f}dz = d\log f\) pulls this counting to the domain:
	\[
	\frac{1}{2\pi i}\int_{\partial U} \frac{f'}{f}dz
	=\sum_{p\in U}\operatorname{ord}_p(f)-\sum_{p\in U}\operatorname{ord}_p(\text{poles of }f),
	\]
	the \emph{argument principle}. On a compact surface with \(U=X\), the boundary term vanishes and the global count must be \(0\). This forces holomorphic \(f\) to have neither zeros nor poles; then \(\log|f|\) is harmonic and constant.
	
	\section*{5. Tiny exercise set}
	\begin{itemize}
		\item Show directly that if \(f\) is holomorphic and nowhere zero on a compact Riemann surface, then \(\log|f|\) is harmonic and hence constant; conclude \(f\) is constant.
		\item Let \(\omega\) be a meromorphic \(1\)-form on a compact Riemann surface. Prove
		\(\sum_p \operatorname{Res}_p(\omega)=0\) using Stokes' theorem.
		\item (Argument principle) For a domain \(U\Subset X\) with smooth boundary avoiding zeros/poles of a meromorphic \(f\), prove
		\[
		\frac{1}{2\pi i}\int_{\partial U} \frac{f'}{f}dz
		=\sum_{p\in U}\operatorname{ord}_p(f)-\sum_{p\in U}\operatorname{ord}_p(1/f).
		\]
	\end{itemize}
	
\end{document}
