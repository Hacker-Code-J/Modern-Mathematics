\documentclass[12pt, a4paper]{article}

% PACKAGES for math, formatting, and colors
\usepackage{amsmath}
\usepackage{amssymb}
\usepackage[margin=1in]{geometry}
\usepackage{xcolor}
\usepackage{graphicx}

% Custom command for partial derivatives
\newcommand{\pderiv}[2]{\frac{\partial #1}{\partial #2}}
\renewcommand{\d}{\mathrm{d}} % For a straight 'd' in differentials
\newcommand{\R}{\mathbb{R}} % For the set of real numbers
\renewcommand{\vec}[1]{\mathbf{#1}} % For vectors

\title{From Conservative Fields to Exact Forms: \\ A Gentle Introduction}
\author{Notes for a Vector Calculus Student}
\date{\today}

\begin{document}
	
	\maketitle
	
	You've already mastered the most important idea: when a vector field $\vec{F}$ is the gradient of a potential function $f$ (i.e., $\vec{F} = \nabla f$), line integrals become incredibly simple. The language of \textbf{differential forms} rephrases these ideas in a way that reveals a deeper geometric story, especially when dealing with tricky domains.
	
	\section{The Dictionary: Translating to the New Language}
	
	Think of this as a direct translation guide. The concepts are identical, just the notation is different.
	
	\begin{center}
		\begin{tabular}{l|l}
			\hline
			\textbf{Vector Calculus (Your Current Language)} & \textbf{Differential Forms (The New Language)} \\
			\hline \hline
			Vector Field $\vec{F} = \langle P, Q \rangle$ & \textbf{1-Form} $\omega = P\,\d x + Q\,\d y$ \\
			\textbf{Conservative} Field ($\vec{F} = \nabla f$) & \textbf{Exact} Form ($\omega = \d f$) \\
			\textbf{Curl-Free} Field ($\pderiv{Q}{x} = \pderiv{P}{y}$) & \textbf{Closed} Form ($\d\omega = 0$) \\
			\hline
		\end{tabular}
	\end{center}
	
	\begin{itemize}
		\item An \textbf{exact} form is one that is the ``total differential'' of a function, which is the same as a vector field being the gradient of a potential.
		\item A \textbf{closed} form is one whose own ``next derivative'' is zero. This new derivative, called the \textbf{exterior derivative} $\d$, turns out to be the same as the curl test you already know. For a 1-form $\omega = P\,\d x + Q\,\d y$, its derivative is $\d\omega = (\pderiv{Q}{x} - \pderiv{P}{y})\,\d x \wedge \d y$. So, ``closed'' just means the part in the parenthesis is zero.
	\end{itemize}
	
	\section{The Big Idea: The Role of ``Holes''}
	
	In vector calculus, you learned that a curl-free field is conservative, usually with a footnote that this works on ``nice'' domains. Differential forms make this idea crystal clear. Here is the fundamental rule:
	
	\begin{enumerate}
		\item \textbf{Exact $\implies$ Closed (Always True)}: If a form has a potential function, its ``curl'' will always be zero. (This is equivalent to the vector identity $\nabla \times (\nabla f) = \vec{0}$).
		\item \textbf{Closed $\implies$ Exact (Only on ``Nice'' Domains)}: If a form is ``curl-free,'' it is guaranteed to have a potential function \textit{only if the domain has no holes}. A domain with no holes is called \textbf{simply connected}.
	\end{enumerate}
	
	The most interesting things happen when the domain has a hole.
	
	\section{The Classic Example: The Winding Form}
	
	Let's investigate a vector field on the plane with the origin punched out, $\R^2 \setminus \{(0,0)\}$. This domain has a ``hole.''
	\[
	\vec{F}(x,y) = \left\langle \frac{-y}{x^2+y^2}, \frac{x}{x^2+y^2} \right\rangle
	\]
	In the language of forms, this is:
	\[
	\omega = \frac{-y}{x^2+y^2}\,\d x + \frac{x}{x^2+y^2}\,\d y
	\]
	
	\subsection{Step 1: Is it closed (curl-free)?}
	Let's run the local test. Here, $P = \frac{-y}{x^2+y^2}$ and $Q = \frac{x}{x^2+y^2}$.
	\begin{align*}
		\pderiv{P}{y} &= \frac{(-1)(x^2+y^2) - (-y)(2y)}{(x^2+y^2)^2} = \frac{-x^2-y^2+2y^2}{(x^2+y^2)^2} = \frac{y^2-x^2}{(x^2+y^2)^2} \\
		\pderiv{Q}{x} &= \frac{(1)(x^2+y^2) - (x)(2x)}{(x^2+y^2)^2} = \frac{x^2+y^2-2x^2}{(x^2+y^2)^2} = \frac{y^2-x^2}{(x^2+y^2)^2}
	\end{align*}
	The mixed partials are equal! So, the form is \textbf{closed} (the vector field is \textbf{curl-free}). At every single point, there is no local ``swirl.''
	
	\subsection{Step 2: Is it exact (conservative)?}
	You know that if a field is conservative, its integral around \textbf{any closed loop must be zero}. Let's test this by integrating around the unit circle, $\gamma(t) = (\cos t, \sin t)$ for $t \in [0, 2\pi]$.
	\begin{itemize}
		\item $x = \cos t \implies \d x = -\sin t\,\d t$
		\item $y = \sin t \implies \d y = \cos t\,\d t$
		\item $x^2+y^2 = \cos^2 t + \sin^2 t = 1$
	\end{itemize}
	Now, substitute this into the line integral $\oint_\gamma \omega$:
	\begin{align*}
		\oint_\gamma \frac{-y}{x^2+y^2}\,\d x + \frac{x}{x^2+y^2}\,\d y &= \int_0^{2\pi} \left( \frac{-\sin t}{1}(-\sin t\,\d t) + \frac{\cos t}{1}(\cos t\,\d t) \right) \\
		&= \int_0^{2\pi} (\sin^2 t + \cos^2 t)\,\d t \\
		&= \int_0^{2\pi} 1\,\d t = 2\pi
	\end{align*}
	
	\subsection{The Punchline}
	The integral is $2\pi$, which is \textbf{not zero}.
	\vskip 1em
	\textcolor{red}{\textbf{Conclusion:}} Even though the field is curl-free everywhere (it's \textbf{closed}), the line integral around a closed loop is non-zero. This proves the field \underline{cannot} be conservative (it's \textbf{not exact}).
	\vskip 1em
	The failure is caused entirely by the \textbf{hole at the origin}. You can't define a single potential function $f(x,y)$ that works everywhere on this punctured domain. This field is secretly tracking the angle $\theta$, and $\omega$ is its differential, $\d\theta$. But the angle itself isn't a well-defined function globally---after one loop, it increases by $2\pi$. The hole allows for a ``global'' circulation that the local, curl-free test cannot detect.
	
\end{document}