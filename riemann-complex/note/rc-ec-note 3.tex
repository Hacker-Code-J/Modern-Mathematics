\documentclass[11pt]{article}

% ---------- Packages ----------
\usepackage[a4paper,margin=1in]{geometry}
\usepackage{amsmath,amssymb,amsthm,mathtools}
\usepackage{physics} % for \dd, \partial, etc.
\usepackage{microtype}
\usepackage{enumitem}
\setlist{nosep}
\usepackage{tikz}
\usetikzlibrary{arrows.meta,calc,angles,quotes,decorations.pathreplacing}
\usepackage{pgfplots}
\pgfplotsset{compat=1.18}

% ---------- Macros ----------
\newcommand{\C}{\mathbb{C}}
\newcommand{\R}{\mathbb{R}}
\newcommand{\ii}{\mathrm{i}}
\newcommand{\dz}{\,\mathrm{d}z}
\newcommand{\dzbar}{\,\mathrm{d}\bar z}
\newcommand{\dx}{\,\mathrm{d}x}
\newcommand{\dy}{\,\mathrm{d}y}
\newcommand{\darg}{\,\mathrm{d}\arg}
\newcommand{\dlog}{\,\mathrm{d}\log}

\newtheorem{definition}{Definition}
\newtheorem{remark}{Remark}
\newtheorem{example}{Example}
\newtheorem{exercise}{Exercise}

% ---------- Title ----------
\title{\Large A Gentle Note on \(dz=dx+\ii\,dy\), Dual Frames, and the Winding Form \(\displaystyle \frac{\dz}{z}\)}
\author{ }
\date{ }

\begin{document}
	\maketitle
	
	\section*{Goal and mindset}
	We view \(\C\cong\R^2\) with coordinates \(x,y\) and complex coordinate \(z=x+\ii y\). One-forms act on vector fields:
	\[
	dx(\partial_x)=1,\quad dx(\partial_y)=0,\qquad
	dy(\partial_x)=0,\quad dy(\partial_y)=1.
	\]
	You can think of \(dx\) as “measuring the \(x\)-component of motion” and ignoring motion tangent to \(x\)-level sets; similarly for \(dy\).
	This note packages them into the \emph{complex} 1-form
	\[
	dz = dx + \ii\,dy,
	\]
	explains what \(dz\) does to vectors, and shows why holomorphic differentials are multiples of \(dz\). We finish with the geometric decomposition of the winding form \(\dz/z\) into radial and angular parts.
	
	\section{What \(dz\) does to a vector}
	Let \(v=a\,\partial_x + b\,\partial_y\). Then
	\[
	dz(v) = dx(v) + \ii\,dy(v) = a + \ii b.
	\]
	Thus \(dz\) turns the real vector \((a,b)\) into the complex number \(a+\ii b\): its modulus is the speed, its argument is the direction angle.
	
%	\begin{center}
%		\begin{tikzpicture}[scale=1.05]
%			% axes
%			\draw[->] (-0.3,0) -- (3.2,0) node[below right] {$x$};
%			\draw[->] (0,-0.3) -- (0,2.2) node[left] {$y$};
%			% vector v
%			\coordinate (O) at (0,0);
%			\coordinate (V) at (2.6,1.5);
%			\draw[very thick, -{Latex[width=3mm]}] (O) -- (V) node[midway, above] {$v=a\partial_x+b\partial_y$};
%			% projection markers
%			\draw[dashed] (V) -- (2.6,0) node[below] {$a$};
%			\draw[dashed] (V) -- (0,1.5) node[left] {$b$};
%			% angle
%			\pic[draw, ->, "$\theta$", angle eccentricity=1.4, angle radius=1cm] {angle = (1,0)--(0,0)--(2.6,1.5)};
%			\node at (4.1,1.2) {$dz(v)=a+\ii b=|v|e^{\ii\theta}$};
%		\end{tikzpicture}
%		
%		\vspace{0.5em}
%		\small Fig.~1. \(dz\) returns a complex number encoding magnitude and direction of \(v\).
%	\end{center}
	
	\section{Complex frames \(\partial_z,\partial_{\bar z}\) and the dual coframe \(dz,d\bar z\)}
	Define
	\[
	\partial_z := \tfrac12\bigl(\partial_x - \ii\,\partial_y\bigr),\qquad
	\partial_{\bar z} := \tfrac12\bigl(\partial_x + \ii\,\partial_y\bigr),
	\]
	and
	\[
	dz:=dx+\ii\,dy,\qquad d\bar z:=dx-\ii\,dy.
	\]
	Then \(\{dz,d\bar z\}\) is dual to \(\{\partial_z,\partial_{\bar z}\}\):
	\[
	dz(\partial_z)=1,\quad dz(\partial_{\bar z})=0,\qquad
	d\bar z(\partial_{\bar z})=1,\quad d\bar z(\partial_z)=0.
	\]
	So \(dz\) detects motion in the \emph{holomorphic} direction \(\partial_z\) and kills the anti-holomorphic direction \(\partial_{\bar z}\).
	
	\begin{example}[How holomorphicity appears]
		For a complex-valued \(f(x,y)\),
		\[
		df = f_x\,dx + f_y\,dy = f_z\,dz + f_{\bar z}\,d\bar z,\qquad
		f_z = \tfrac12(f_x- \ii f_y),\quad f_{\bar z}=\tfrac12(f_x+\ii f_y).
		\]
		Holomorphicity is exactly the condition \(f_{\bar z}=0\), i.e.\ \(df\) is a multiple of \(dz\) only. This is the \(1\)-form version of the Cauchy--Riemann equations.
	\end{example}
	
	\section{Level-set intuition, revisited}
	Because \(dz\) is complex-valued, talking about a single “level set” is less natural. Instead:
	\[
	\Re(dz)=dx \quad\text{(vertical lines \(x=\text{const}\))},\qquad
	\Im(dz)=dy \quad\text{(horizontal lines \(y=\text{const}\))}.
	\]
	The pair \((dx,dy)\) forms two orthogonal foliations; \(dz\) packages both and carries orientation via its complex phase.
	
%	\begin{center}
%		\begin{tikzpicture}[scale=0.7]
%			% Grid of level sets
%			\foreach \x in {-3,-2,...,3} \draw[lightgray] (\x,-3) -- (\x,3);
%			\foreach \y in {-3,-2,...,3} \draw[lightgray] (-3,\y) -- (3,\y);
%			\draw[->] (-3.2,0) -- (3.4,0) node[right] {$x$};
%			\draw[->] (0,-3.2) -- (0,3.4) node[above] {$y$};
%			\node at (2.7,2.8) {\small level sets of $dx$ and $dy$};
%		\end{tikzpicture}
%		
%		\vspace{0.25em}
%		\small Fig.~2. \(dx\): vertical leaves, \(dy\): horizontal leaves; \(dz\) encodes both.
%	\end{center}
	
	\section{Integrating \(dz\)}
	For a path \(\gamma:[a,b]\to\C\), \(\gamma(t)=x(t)+\ii y(t)\),
	\[
	\int_\gamma dz=\int_\gamma (dx+\ii\,dy)=\bigl[x(t)+\ii y(t)\bigr]_{t=a}^{t=b}=z(\gamma(b))-z(\gamma(a)).
	\]
	Thus \(dz\) is exact with potential \(z\). This is why on \(\C\) the “flat holomorphic \(1\)-form” is just \(dz\).
	
	\section{The winding form \(\displaystyle \frac{\dz}{z}\)}
	Away from \(z=0\), write \(z=re^{\ii\theta}\) (\(r>0\), \(\theta=\arg z\)). Then
	\[
	\frac{\dz}{z} = \frac{d(re^{\ii\theta})}{re^{\ii\theta}}
	= \frac{dr}{r} + \ii\, d\theta
	= d(\log r) + \ii\, d\arg z.
	\]
	\(\Re(\dz/z)=d(\log r)\) measures \emph{radial} change; \(\Im(\dz/z)=d\theta\) measures \emph{angular} change (winding).
	
%	\begin{center}
%		\begin{tikzpicture}[scale=1.05]
%			% axes
%			\draw[->] (-3.2,0) -- (3.2,0) node[right] {$x$};
%			\draw[->] (0,-3.2) -- (0,3.2) node[above] {$y$};
%			% circle
%			\draw[thick] (0,0) circle (2.5);
%			\node at (2.9,0.2) {\small $r=\text{const}$};
%			% radial line and angle
%			\draw[thick, -{Latex[length=3mm]}] (0,0) -- (40:2.5) node[pos=0.6, above right] {\small $dr$};
%			\pic[draw, ->, "$\theta$", angle eccentricity=1.4, angle radius=1cm] {angle = (1,0)--(0,0)--(40:1)};
%			% arc indicating dtheta
%			\draw[thick, -{Latex[length=2mm]}] (40:2.5) arc (40:55:2.5);
%			\node at (2.0,2.2) {\small $d\theta$};
%			\node at (-2.5,-2.5) {$\displaystyle \frac{\dz}{z} = \underbrace{\frac{dr}{r}}_{\text{radial}} + \ii\,\underbrace{d\theta}_{\text{angular}}$};
%		\end{tikzpicture}
%		
%		\vspace{0.25em}
%		\small Fig.~3. Decomposing \(\dz/z\) into radial and angular pieces.
%	\end{center}
	
	\begin{example}[Winding number]
		If \(\gamma\) is a closed loop avoiding \(0\),
		\[
		\frac{1}{2\pi\ii}\oint_\gamma \frac{\dz}{z} = \text{winding number of }\gamma\text{ around }0\in\mathbb{Z}.
		\]
		Indeed \(\oint d(\log r)=0\) (since \(\log r\) is single-valued on \(\C^\times\)), and \(\oint d\theta = 2\pi\,(\text{winding})\).
	\end{example}
	
	\section{Mini-computations and sanity checks}
	
	\begin{example}[Action on basis vectors]
		\(dz(\partial_x)=1\), \(dz(\partial_y)=\ii\). For \(v=\partial_x+\partial_y\), \(dz(v)=1+\ii\).
	\end{example}
	
	\begin{example}[Directional derivative of a holomorphic function]
		If \(f\) is holomorphic, \(df=f_z\,dz\) with \(f_z=\partial f/\partial z\). For \(v=a\,\partial_x+b\,\partial_y\),
		\[
		df(v) = f_z\,dz(v) = f_z\,(a+\ii b).
		\]
		So the real \(2\text{D}\) directional derivative is the complex derivative times the complex number representing \(v\).
	\end{example}
	
	\begin{example}[Line integrals]
		Let \(\gamma(t)=t\) on \([0,1]\). Then \(\int_\gamma dz = 1\).
		Let \(\gamma(t)=e^{\ii t}\) on \([0,2\pi]\). Then \(\displaystyle \oint \frac{\dz}{z} = 2\pi\ii\).
	\end{example}
	
	\section{Exercises (with short solutions at the end)}
	\begin{exercise}
		Show that \(dz(\partial_{\bar z})=0\) and \(d\bar z(\partial_z)=0\).
	\end{exercise}
	
	\begin{exercise}
		Write \(df\) in the \((dz,d\bar z)\)-basis and show that \(f\) holomorphic \(\iff f_{\bar z}=0\).
	\end{exercise}
	
	\begin{exercise}
		Let \(\gamma(t)=re^{\ii t}\) with constant \(r>0\), \(t\in[t_0,t_1]\). Compute \(\int_\gamma \dz/z\).
	\end{exercise}
	
	\begin{exercise}
		Prove that for \(\gamma\) closed avoiding \(0\), \(\frac{1}{2\pi\ii}\oint_\gamma\dz/z\in\mathbb{Z}\).
	\end{exercise}
	
	\subsection*{Sketch solutions}
	\begin{enumerate}[label=\arabic*)]
		\item Use definitions: \(\partial_{\bar z}=\tfrac12(\partial_x+\ii\partial_y)\) and \(dz=dx+\ii dy\).
		\item Compute \(df=f_x\,dx+f_y\,dy\) and rewrite as \(f_z\,dz+f_{\bar z}\,d\bar z\).
		\item \(\dz/z = \ii\,dt\) along \(\gamma\) (since \(dr=0\)), so the integral is \(\ii(t_1-t_0)\).
		\item Approximate \(\gamma\) by a polygon and use the argument principle, or apply homotopy invariance of \(\oint \dz/z\) on \(\C^\times\) together with the circle case.
	\end{enumerate}
	
	\section*{Where this connects next}
	On a complex torus \(X=\C/\Lambda\) the (nowhere-vanishing) holomorphic \(1\)-form is the pushdown of \(dz\). Choosing cycles \(a,b\) gives periods \(\int_a dz=1\), \(\int_b dz=\tau\).
	From \(dz\) one builds the Weierstrass \(\wp\)-function and the elliptic curve \(y^2=4x^3-g_2x-g_3\). (See the companion note: constructing \(\wp\) from \(dt=\omega\).)
	
\end{document}
