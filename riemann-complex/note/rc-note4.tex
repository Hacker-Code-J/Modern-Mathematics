\documentclass[12pt]{article}
\usepackage{amsmath,amssymb,amsfonts}
\usepackage{geometry}
\geometry{margin=1in}
\newcommand{\R}{\mathbb{R}}
\begin{document}
	
	\title{From Calculus to Mayer–Vietoris in de Rham Cohomology of \(S^2\)}
	\author{Lecture Notes}
	\date{}
	\maketitle
	
	\tableofcontents
	
	\section{Calculus Preliminaries}
	
	\subsection{Fundamental Theorem of Calculus}
	For any \(a,b\in\mathbb{R}\),
	\[
	\int_a^b \cos t\,dt = \sin b - \sin a,
	\]
	since \(\tfrac{d}{dt}(\sin t)=\cos t\).  Equivalently,
	\[
	d(\sin t)=\cos t\,dt,
	\quad
	d(-\cos t)=\sin t\,dt,
	\]
	so
	\[
	\int_a^b \cos t\,dt = \int_a^b d(\sin t) = \sin b - \sin a,
	\quad
	\int_a^b \sin t\,dt = -\cos b + \cos a.
	\]
	
	\subsection{Fundamental Theorem for Gradients (Line Integrals)}
	Let 
	\[
	\mathbf{F}(x,y)
	=\Bigl(-\frac{y}{x^2+y^2},\;\frac{x}{x^2+y^2}\Bigr),
	\]
	and let \(C\) be the unit circle \(x^2+y^2=1\) oriented counterclockwise.  Then
	\[
	\oint_C \mathbf{F}\cdot d\mathbf{r}
	= \oint_C \Bigl(-\tfrac{y}{r^2},\,\tfrac{x}{r^2}\Bigr)\!\cdot\!(dx,dy)
	= 2\pi.
	\]
	Although \(\mathbf{F}\) is not a gradient globally, one computes via parameterization 
	\(\mathbf{r}(t)=(\cos t,\sin t)\), \(t\in[0,2\pi]\):
	\[
	dx=-\sin t\,dt,\quad dy=\cos t\,dt,\quad r^2=1,
	\]
	so
	\[
	\oint_C\mathbf{F}\cdot d\mathbf{r}
	=\int_0^{2\pi}\Bigl(-\sin t,\cos t\Bigr)\!\cdot\!\bigl(-\sin t,\cos t\bigr)\,dt
	=\int_0^{2\pi}(\sin^2t+\cos^2t)\,dt
	=2\pi.
	\]
	
	\subsection{Surface Integrals (Flux) -- Stokes' Theorem}
	Let \(D=[0,1]^2\) in the \(uv\)-plane, and parameterize a surface \(S\) by
	\(\mathbf{T}(u,v)\).  Suppose
	\[
	\mathbf{F}(u,v)=(27\,u\,v,0,0),
	\]
	so that the \(2\)-form is
	\(\beta=27\,u\,v\,du\wedge dv\).  Then
	\[
	\iint_S \mathbf{F}\cdot d\mathbf{S}
	=\iint_D 27\,u\,v\,du\,dv
	=27\!\int_0^1\!\int_0^1 u\,v\,du\,dv
	=27\Bigl(\tfrac12\Bigr)\Bigl(\tfrac12\Bigr)
	=\frac{27}{4}.
	\]
	
	\section{Differential Forms on \(\R^n\)}
	
%	\begin{definition}
	\textbf{Definition.}
		The space of \emph{smooth functions} is
		\[
		C^\infty(\R^n)
		=\bigl\{f:\R^n\to\R\mid f\text{ has continuous derivatives of all orders}\bigr\}.
		\]
		A \emph{smooth \(1\)-form} on \(\R^n\) is
		\[
		\omega
		= f_1(x)\,dx^1 + \cdots + f_n(x)\,dx^n,
		\quad f_i\in C^\infty(\R^n),
		\]
		and the collection of all of them is denoted \(\Omega^1(\R^n)\).
%	\end{definition}
	
	The \emph{exterior derivative} \(d:C^\infty(\R^n)\to\Omega^1(\R^n)\) is
	\[
	d\bigl(f(x)\bigr)
	=\sum_{i=1}^n \frac{\partial f}{\partial x^i}\,dx^i.
	\]
	
	\section{de Rham Cohomology of the 2–Sphere \(S^2\)}
	
	Cover \(S^2\subset\R^3\) by two charts:
	\[
	U = S^2\setminus\{\text{north pole}\}, 
	\quad
	V = S^2\setminus\{\text{south pole}\}.
	\]
	Each of \(U\) and \(V\) is diffeomorphic to \(\R^2\).  Their intersection
	\(U\cap V\) is diffeomorphic to \(\R\times S^1\).
	
	\subsection{Mayer–Vietoris Sequence in de Rham Theory}
	
	For the cover \(\{U,V\}\) of \(S^2\), the Mayer–Vietoris sequence in de Rham cohomology begins
	\[
	0 \;\longrightarrow\; \Omega^0(S^2)
	\;\xrightarrow{\;\phi\;}
	\Omega^0(U)\oplus\Omega^0(V)
	\;\xrightarrow{\;\psi\;}
	\Omega^0(U\cap V)
	\;\xrightarrow{\;\delta\;}
	\Omega^1(S^2)
	\;\xrightarrow{\;\phi'\;}
	\Omega^1(U)\oplus\Omega^1(V)
	\;\rightarrow\;\cdots
	\]
	where
	\[
	\phi(f) = (f|_U,\;f|_V),
	\quad
	\psi(g_U,g_V) = g_U|_{U\cap V} - g_V|_{U\cap V}.
	\]
	
	\subsection{Computations}
	
	\paragraph{Degree 0.}
	\[
	\Omega^0(S^2)=C^\infty(S^2),
	\quad
	\Omega^0(U)\cong C^\infty(\R^2),
	\quad
	\Omega^0(U\cap V)\cong C^\infty(\R\times S^1).
	\]
	Exactness at \(\Omega^0(S^2)\) and \(\Omega^0(U)\oplus\Omega^0(V)\) shows
	\[
	H^0_{\mathrm{dR}}(S^2)\cong\R
	\]
	(smooth functions constant on connected \(S^2\)).
	
	\paragraph{Connecting Map \(\delta\).}
	Given \(h\in\Omega^0(U\cap V)\), one constructs a \(1\)-form on \(S^2\) whose difference of restrictions equals \(dh\).  Explicitly this uses a partition of unity, but on first-year level one may think of \(\delta(h)\) as “gluing data” producing a closed \(1\)-form on \(S^2\).
	
	\paragraph{Degree 1.}
	Exactness at \(\Omega^1(S^2)\) and \(\Omega^1(U)\oplus\Omega^1(V)\) shows
	\[
	H^1_{\mathrm{dR}}(S^2)=0,
	\]
	since every closed \(1\)-form on \(S^2\) is exact (no “holes” in \(S^2\)).
	
	\section*{Conclusion}
	We have built step by step from standard calculus—FTC, line integrals, surface integrals—to the language of differential forms, and used the Mayer–Vietoris sequence to compute
	\[
	H^0_{\mathrm{dR}}(S^2)\cong\R,
	\quad
	H^1_{\mathrm{dR}}(S^2)=0.
	\]
\end{document}