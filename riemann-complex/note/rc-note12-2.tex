\documentclass[11pt]{article}

% Preamble: Packages and Commands
\usepackage[a4paper, margin=1in]{geometry}
\usepackage{amsmath, amssymb, amsthm}
\usepackage[colorlinks=true, urlcolor=blue, linkcolor=blue, citecolor=red]{hyperref}

% Custom theorem-like environments for structure
\theoremstyle{definition}
\newtheorem{definition}{Definition}[section]
\newtheorem{example}{Example}[section]
\newtheorem{remark}{Remark}[section]
\theoremstyle{plain}
\newtheorem{theorem}{Theorem}[section]
\newtheorem{prop}[theorem]{Proposition}
\newtheorem{lemma}[theorem]{Lemma}
\newtheorem{corollary}[theorem]{Corollary}

% Custom commands for mathematical notation
\newcommand{\R}{\mathbb{R}}
\newcommand{\dd}{\mathrm{d}} % For the exterior derivative 'd'
\newcommand{\vect}[1]{\mathbf{#1}} % For vectors/vector fields

\title{From Vector Calculus to Differential Forms: \\ A Dictionary for Conservative and Curl-Free Fields}
\author{Gemini}
\date{\today}

\begin{document}
	
	\maketitle
	
	\begin{abstract}
		This lecture note establishes a rigorous correspondence between fundamental concepts in vector calculus on $\R^2$ and $\R^3$ and the language of differential forms. We explore the relationship between conservative vector fields and exact 1-forms, and between curl-free vector fields and closed 1-forms. The critical role of the domain's topology, particularly simple connectedness, is highlighted through the Poincaré Lemma.
	\end{abstract}
	
	\section{The Core Dictionary}
	
	The translation between vector calculus and differential forms provides both computational power and deeper theoretical insight. The primary correspondences are as follows:
	
	\begin{center}
		\begin{tabular}{lcl}
			\textbf{Vector Calculus} & $\iff$ & \textbf{Differential Forms} \\
			\hline
			Vector field $\vect{F}$ on $\R^n$ ($n=2,3$) & $\iff$ & 1-form $\omega$ \\
			Conservative field: $\vect{F} = \nabla f$ & $\iff$ & Exact 1-form: $\omega = \dd f$ \\
			Curl-free field: $\nabla \times \vect{F} = \vect{0}$ & $\iff$ & Closed 1-form: $\dd\omega = 0$ \\
		\end{tabular}
	\end{center}
	
	This dictionary is governed by a fundamental logical implication structure.
	
	\begin{prop}
		For any smooth 1-form $\omega$ on an open domain $U \subseteq \R^n$:
		\[
		\omega \text{ is exact} \implies \omega \text{ is closed.}
		\]
		The converse holds if the domain $U$ is simply connected (e.g., star-shaped, convex, or all of $\R^n$).
	\end{prop}
	
	\section{Definitions and Interpretations}
	
	\subsection{Conservative Fields and Exact Forms}
	
	\begin{definition}
		A vector field $\vect{F}: U \to \R^n$ on an open set $U \subseteq \R^n$ is \textbf{conservative} if there exists a scalar function $f: U \to \R$, called a scalar potential, such that $\vect{F} = \nabla f$.
	\end{definition}
	
	\begin{definition}
		A 1-form $\omega$ on an open set $U \subseteq \R^n$ is \textbf{exact} if there exists a smooth function (a 0-form) $f: U \to \R$ such that $\omega = \dd f$.
	\end{definition}
	
	In coordinates, if $\vect{F} = \langle F_1, \dots, F_n \rangle$, its corresponding 1-form is $\omega = \sum_{i=1}^n F_i \, \dd x_i$. The condition $\vect{F} = \nabla f = \langle \frac{\partial f}{\partial x_1}, \dots, \frac{\partial f}{\partial x_n} \rangle$ is identical to $\omega = \dd f = \sum_{i=1}^n \frac{\partial f}{\partial x_i} \, \dd x_i$.
	
	\begin{remark}[Physical Significance]
		The primary utility of a conservative field is the path-independence of its line integral, a consequence of the Fundamental Theorem for Line Integrals:
		\[
		\int_{\gamma} \vect{F} \cdot \dd\vect{r} = \int_{\gamma} \nabla f \cdot \dd\vect{r} = f(\gamma(b)) - f(\gamma(a)).
		\]
		This implies that for any closed loop $\gamma$, $\oint_\gamma \vect{F} \cdot \dd\vect{r} = 0$.
	\end{remark}
	
	\subsection{Curl-Free Fields and Closed Forms}
	
	\begin{definition}
		A vector field $\vect{F} = \langle P, Q, R \rangle$ on $U \subseteq \R^3$ is \textbf{curl-free} (or irrotational) if its curl is the zero vector: $\nabla \times \vect{F} = \vect{0}$.
	\end{definition}
	
	\begin{definition}
		A differential form $\omega$ is \textbf{closed} if its exterior derivative is zero: $\dd\omega = 0$.
	\end{definition}
	
	For a 1-form $\omega = P\,\dd x + Q\,\dd y + R\,\dd z$ in $\R^3$, its exterior derivative is the 2-form:
	\[
	\dd\omega = (\partial_y R - \partial_z Q)\,\dd y \wedge \dd z + (\partial_z P - \partial_x R)\,\dd z \wedge \dd x + (\partial_x Q - \partial_y P)\,\dd x \wedge \dd y.
	\]
	The coefficients of the basis 2-forms are precisely the components of $\nabla \times \vect{F}$ where $\vect{F} = \langle P, Q, R \rangle$. Thus, $\dd\omega = 0$ is equivalent to $\nabla \times \vect{F} = \vect{0}$.
	
	\begin{remark}
		In 2D, for $\vect{F} = \langle P, Q \rangle$ and $\omega = P\,\dd x + Q\,\dd y$, the condition $\dd\omega = 0$ simplifies to $(\partial_x Q - \partial_y P)\,\dd x \wedge \dd y = 0$, which is the familiar scalar curl condition $\partial_x Q = \partial_y P$.
	\end{remark}
	
	\section{Practical Tests for Exactness}
	
	Given a 1-form $\omega = P\,\dd x + Q\,\dd y$ on a domain $U \subseteq \R^2$.
	
	\begin{enumerate}
		\item \textbf{Local Test (Mixed Partials):} Check if $\omega$ is closed. This involves testing the equality of mixed partial derivatives:
		\[
		\dd\omega = 0 \iff \frac{\partial Q}{\partial x} = \frac{\partial P}{\partial y}.
		\]
		If the domain $U$ is simply connected, the Poincaré Lemma guarantees that closedness implies exactness.
		
		\item \textbf{Global Test (Path Independence):} The form $\omega$ is exact if and only if its integral over any closed loop $\gamma \subset U$ is zero: $\oint_\gamma \omega = 0$. This is equivalent to stating that the integral $\int_\gamma \omega$ depends only on the start and end points of the path $\gamma$.
		
		\item \textbf{Constructive Test (Potential Recovery):} To find a potential function $f(x,y)$ such that $\dd f = \omega$, one can proceed by integration.
		\begin{enumerate}
			\item Integrate $P = \frac{\partial f}{\partial x}$ with respect to $x$:
			\[ f(x,y) = \int P(x,y) \, \dd x + h(y), \]
			where $h(y)$ is an unknown function of $y$.
			\item Differentiate this expression for $f$ with respect to $y$ and set it equal to $Q$:
			\[ \frac{\partial f}{\partial y} = \frac{\partial}{\partial y} \left( \int P(x,y) \, \dd x \right) + h'(y) = Q(x,y). \]
			\item Solve for $h'(y)$ and integrate to find $h(y)$. The potential $f$ is unique up to an additive constant.
		\end{enumerate}
	\end{enumerate}
	
	\section{The Role of Domain Topology}
	
	The distinction between closed and exact forms is entirely topological.
	
	\begin{definition}
		A path-connected set $U \subseteq \R^n$ is \textbf{simply connected} if every simple closed curve in $U$ can be continuously shrunk to a point within $U$. Intuitively, $U$ has no "holes."
	\end{definition}
	
	\begin{itemize}
		\item \textbf{Simply Connected Domains:} $\R^n$, convex sets, star-shaped domains.
		\item \textbf{Non-Simply Connected Domains:} The punctured plane $\R^2 \setminus \{0\}$, an annulus, $\R^3$ minus a line (e.g., the $z$-axis).
	\end{itemize}
	
	\begin{lemma}[Poincaré Lemma]
		On a simply connected domain $U \subseteq \R^n$, every closed form is exact.
	\end{lemma}
	
	\section{Illustrative Examples}
	
	\begin{example}[Exact on $\R^2$]
		Consider the 1-form $\omega = (2xy+ye^{xy})\,\dd x + (x^2+xe^{xy})\,\dd y$ on $\R^2$.
		We identify $P = 2xy+ye^{xy}$ and $Q = x^2+xe^{xy}$.
		\begin{itemize}
			\item \textbf{Closedness:} We check the mixed partials:
			\begin{align*}
				\frac{\partial Q}{\partial x} &= 2x + (e^{xy} + xye^{xy}) = 2x + e^{xy}(1+xy) \\
				\frac{\partial P}{\partial y} &= 2x + (e^{xy} + yxe^{xy}) = 2x + e^{xy}(1+xy)
			\end{align*}
			Since $\frac{\partial Q}{\partial x} = \frac{\partial P}{\partial y}$, the form is closed.
			\item \textbf{Exactness:} Since the domain $\R^2$ is simply connected, $\omega$ must be exact. We can construct the potential $f(x,y) = x^2y + e^{xy}$, for which $\dd f = \omega$.
		\end{itemize}
	\end{example}
	
	\begin{example}[Closed but Not Exact on $\R^2 \setminus \{0\}$]
		Consider the 1-form $\omega = \frac{-y}{x^2+y^2}\,\dd x + \frac{x}{x^2+y^2}\,\dd y$ on the punctured plane $U = \R^2 \setminus \{(0,0)\}$.
		\begin{itemize}
			\item \textbf{Closedness:} A direct calculation shows $\frac{\partial Q}{\partial x} = \frac{\partial P}{\partial y} = \frac{y^2-x^2}{(x^2+y^2)^2}$, so $\omega$ is closed.
			\item \textbf{Non-Exactness:} We test the integral around a closed loop that encloses the origin. Let $\gamma$ be the unit circle, parameterized by $\vect{r}(t) = (\cos t, \sin t)$ for $t \in [0, 2\pi]$.
			\begin{align*}
				\oint_\gamma \omega &= \int_0^{2\pi} \left( \frac{-\sin t}{\cos^2 t + \sin^2 t}(-\sin t) + \frac{\cos t}{\cos^2 t + \sin^2 t}(\cos t) \right) \, \dd t \\
				&= \int_0^{2\pi} (\sin^2 t + \cos^2 t) \, \dd t = \int_0^{2\pi} 1 \, \dd t = 2\pi.
			\end{align*}
			Since the integral over a closed loop is non-zero, $\omega$ cannot be exact on $U$. The "hole" at the origin is a topological obstruction.
		\end{itemize}
		\begin{remark}
			In polar coordinates $(r, \theta)$, this form is simply $\omega = \dd\theta$. This makes it clear that it is locally exact, but the potential function $\theta$ is not globally single-valued on $\R^2 \setminus \{0\}$, preventing global exactness.
		\end{remark}
	\end{example}
	
	\section{The Fundamental Identity: \texorpdfstring{$d^2=0$}{d-squared is zero}}
	
	The fact that exact forms are always closed is a direct consequence of a fundamental property of the exterior derivative.
	
	\begin{theorem}
		For any smooth $k$-form $\alpha$, the exterior derivative of its exterior derivative is zero: $\dd(\dd\alpha) = 0$. This is often written as $\dd^2 = 0$.
	\end{theorem}
	
	\begin{proof}[Proof for 0-forms]
		If $\omega$ is an exact 1-form, then $\omega = \dd f$ for some 0-form (function) $f$. Applying the exterior derivative again yields:
		\[
		\dd\omega = \dd(\dd f).
		\]
		In coordinates on $\R^n$, $\dd f = \sum_i \frac{\partial f}{\partial x_i} \dd x_i$. Then
		\[
		\dd(\dd f) = \sum_{j,i} \frac{\partial^2 f}{\partial x_j \partial x_i} \dd x_j \wedge \dd x_i.
		\]
		By the symmetry of mixed partial derivatives ($\frac{\partial^2 f}{\partial x_j \partial x_i} = \frac{\partial^2 f}{\partial x_i \partial x_j}$) and the anti-symmetry of the wedge product ($\dd x_j \wedge \dd x_i = - \dd x_i \wedge \dd x_j$), the terms in the sum cancel pairwise. For example, the term involving $\dd x_1 \wedge \dd x_2$ is
		\[
		\left( \frac{\partial^2 f}{\partial x_1 \partial x_2} - \frac{\partial^2 f}{\partial x_2 \partial x_1} \right) \dd x_1 \wedge \dd x_2 = 0.
		\]
		Hence, $\dd\omega = 0$. In vector calculus terms, this identity is equivalent to $\nabla \times (\nabla f) = \vect{0}$ (the curl of a gradient is always zero).
	\end{proof}
	
	\section{Exercises}
	
	\begin{enumerate}
		\item Let $\omega = (3x^2y + ye^{xy})\,\dd x + (x^3 + xe^{xy})\,\dd y$ on $\R^2$. Show that $\omega$ is closed and find a scalar potential $f$ such that $\omega = \dd f$. \\
		\textit{Answer sketch: Check $\partial_x Q = \partial_y P = 3x^2 + e^{xy}(1+xy)$. The potential is $f(x,y) = x^3y + e^{xy} + C$.}
		
		\item Let $\vect{F}(x,y) = \langle -y, x \rangle$ on $\R^2$. Determine if this field is conservative. \\
		\textit{Answer sketch: The corresponding 1-form is $\omega = -y\,\dd x + x\,\dd y$. Then $\dd\omega = (\partial_x(x) - \partial_y(-y))\,\dd x \wedge \dd y = 2\,\dd x \wedge \dd y \neq 0$. The form is not closed, and therefore not exact (conservative).}
		
		\item Consider the vector field $\vect{F}(x,y,z) = \langle yz, xz, xy \rangle$ on $\R^3$.
		\begin{enumerate}
			\item Show that $\vect{F}$ is curl-free.
			\item Since the domain $\R^3$ is simply connected, find a scalar potential $f$ such that $\vect{F} = \nabla f$.
		\end{enumerate}
		\textit{Answer sketch: (a) $\nabla \times \vect{F} = \vect{0}$. (b) The potential is $f(x,y,z) = xyz + C$.}
	\end{enumerate}
	
\end{document}