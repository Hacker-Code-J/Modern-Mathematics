\documentclass[12pt]{article}
\usepackage{amsmath,amssymb,amsthm}
\usepackage{geometry}
\geometry{margin=1in}
\theoremstyle{definitionstyle}
\newtheorem{definition}{Definition} % Definition shares the counter with theorem
\newtheorem{example}{Example} % Example shares the counter with theorem
\newtheorem{remark}{Remark} % Remark shares the counter with theorem
\newcommand{\R}{\mathbb{R}}

\begin{document}
\title{Lecture Notes: Differentials, Gradients, and Jacobians via Integrals}
\author{}
\date{}
\maketitle

\tableofcontents

\section*{Overview}

We explore the chain of operators and their inverses (integrals or anti‐derivatives) in three contexts:
\[
\underbrace{f}_{\Omega^0}
\;\xrightarrow{d}\;
\underbrace{df}_{\Omega^1}
\;\overset{\nabla}{\longleftrightarrow}\;
\underbrace{\nabla f}_{\substack{\text{gradient}\\\text{vector field}}}
\quad
\longrightarrow
\quad
\underbrace{\mathbf F}_{(\Omega^0)^m}
\;\xrightarrow{d}\;
\underbrace{d\mathbf F}_{\Omega^1\otimes\R^m}
\;\overset{D}{\longleftrightarrow}\;
\underbrace{D\mathbf F}_{\substack{\text{Jacobian}\\\text{matrix}}}.
\]
Here each double‐arrow labeled \(I\) below denotes the appropriate integral inverse of \(d\).

\bigskip

\section{From Functions to 1‐Forms}

\subsection{0‐Forms and the Exterior Derivative}

\begin{definition}
	\(\Omega^0(\R^n)=C^\infty(\R^n)\) is the space of smooth real‐valued functions (0‐forms).  The exterior derivative
	\[
	d:\Omega^0(\R^n)\;\longrightarrow\;\Omega^1(\R^n)
	\]
	is defined by
	\[
	d f
	=\sum_{i=1}^n \frac{\partial f}{\partial x_i}(x)\,dx^i.
	\]
\end{definition}

\subsection{Integral Inverse on 1‐Forms}

On a simply‐connected region \(U\subset\R^n\), the inverse of \(d\) on exact 1‐forms is given by choosing a base point \(x_0\in U\) and integrating along any path:
\[
I_{x_0}:\{\,\omega=df\}\;\longrightarrow\;\Omega^0(U),
\qquad
I_{x_0}(\omega)(x)
=\int_{x_0}^x \omega.
\]
By the Fundamental Theorem of Calculus,
\[
I_{x_0}(df)(x)
= \int_{x_0}^x df
= f(x) - f(x_0).
\]

\bigskip

\section{Gradient as Metric Dual}

Equipping \(\R^n\) with the Euclidean metric identifies each \(1\)-form
\(\omega=\sum_i g_i\,dx^i\)
with the vector field
\(\sum_i g_i\,\partial/\partial x_i\).

\begin{definition}
	The \emph{gradient} of \(f\in\Omega^0(\R^n)\) is the vector field
	\[
	\nabla f(x)
	=\begin{pmatrix}
		\partial_{x_1}f(x)\\
		\partial_{x_2}f(x)\\
		\vdots\\
		\partial_{x_n}f(x)
	\end{pmatrix}.
	\]
\end{definition}

\begin{remark}
	One recovers \(f\) (up to constant) by integrating its gradient along any curve from \(x_0\) to \(x\):
	\[
	f(x)-f(x_0)
	=\int_{x_0}^x \nabla f\cdot d\mathbf r.
	\]
\end{remark}

\bigskip

\section{Vector Fields and Their Differentials}

\subsection{Vector Fields as \((\Omega^0)^m\)}

A smooth vector field
\(\mathbf F:\R^n\to\R^m\)
is an \(m\)-tuple of scalar fields:
\[
\mathbf F(x)
=\begin{pmatrix}F_1(x)\\F_2(x)\\\vdots\\F_m(x)\end{pmatrix}
\;\in\;(\Omega^0(\R^n))^m.
\]

\subsection{Exterior Derivative on Vector Fields}

Apply \(d\) componentwise:
\[
d\mathbf F
=\begin{pmatrix}dF_1\\dF_2\\\vdots\\dF_m\end{pmatrix}
\;\in\;\Omega^1(\R^n)\otimes\R^m,
\]
where
\(\displaystyle dF_i=\sum_{j=1}^n\frac{\partial F_i}{\partial x_j}(x)\,dx^j.\)

\subsection{Integral Inverse: Line Integrals of 1‐Forms}

If each \(dF_i\) is exact on \(U\), define
\[
I_{x_0}(d\mathbf F)
=\begin{pmatrix}
	\int_{x_0}^x dF_1\\
	\int_{x_0}^x dF_2\\
	\vdots\\
	\int_{x_0}^x dF_m
\end{pmatrix}
=
\begin{pmatrix}
	F_1(x)-F_1(x_0)\\
	F_2(x)-F_2(x_0)\\
	\vdots\\
	F_m(x)-F_m(x_0)
\end{pmatrix}.
\]

\bigskip

\section{Jacobian Matrix}

\subsection{Identification with 1‐Forms}

Choosing the basis \(\{dx^1,\dots,dx^n\}\) identifies \(d\mathbf F\) with the \(m\times n\) Jacobi an matrix:
\[
D\mathbf F(x)
=\begin{pmatrix}
	\partial_{x_1}F_1(x) & \cdots & \partial_{x_n}F_1(x)\\
	\vdots & \ddots & \vdots\\
	\partial_{x_1}F_m(x) & \cdots & \partial_{x_n}F_m(x)
\end{pmatrix}.
\]
This matrix gives the linear approximation:
\[
\mathbf F(x+h)
= \mathbf F(x) + D\mathbf F(x)\,h + o(\|h\|).
\]

\subsection{Recovering \(\mathbf F\) via Component Integrals}

If \(D\mathbf F\) is integrable, then
\[
\mathbf F(x)
=\mathbf F(x_0)
+\int_{x_0}^x D\mathbf F(x')\,dx',
\]
with the integral taken componentwise.

\bigskip

\section*{Summary Diagram}

\[
\underbrace{f}_{\Omega^0}
\;\xrightarrow{d}\;
\underbrace{df}_{\Omega^1}
\;\overset{I}{\longleftrightarrow}\;
\underbrace{\nabla f}_{\substack{\text{gradient}\\\text{vector field}}}
\quad
\longrightarrow
\quad
\underbrace{\mathbf F}_{(\Omega^0)^m}
\;\xrightarrow{d}\;
\underbrace{d\mathbf F}_{\Omega^1\otimes\R^m}
\;\overset{I}{\longleftrightarrow}\;
\underbrace{D\mathbf F}_{\substack{\text{Jacobian}\\\text{matrix}}}.
\]

Here \(d\) is the exterior derivative and \(I\) denotes the corresponding integral inverse.

\end{document}