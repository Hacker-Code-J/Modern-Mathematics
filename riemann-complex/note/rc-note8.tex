\documentclass[12pt, letterpaper]{article}
\usepackage{amsmath, amssymb, amsfonts, amsthm}
\usepackage[margin=1in]{geometry}
\usepackage{xcolor}
\usepackage{hyperref}
\usepackage{graphicx}

\definecolor{darkblue}{rgb}{0.0, 0.0, 0.55}
\definecolor{darkred}{rgb}{0.55, 0.0, 0.0}

\hypersetup{
	colorlinks=true,
	linkcolor=darkblue,
	citecolor=darkred,
	urlcolor=darkblue,
}

\theoremstyle{definition}
\newtheorem{defn}{Definition}[section]
\newtheorem{exmp}{Example}[section]
\newtheorem{idea}{Core Correspondence}[section]

\title{\bfseries A Bridge Between Worlds: \\ \large From Gradients and Jacobians to Differential Forms \\ \Large (With Penetrating Examples)}
\author{A Unified View of Vector Calculus}
\date{\today}
\newcommand{\R}{\mathbb{R}}

\begin{document}
	\maketitle
	
	\begin{abstract}
		This lecture provides a detailed mathematical bridge between the familiar concepts of multivariable calculus—the gradient of a scalar function and the Jacobian matrix of a vector function—and the powerful, modern language of differential forms. We will move beyond simple polynomial examples to explore cases with deep geometric and physical meaning, demonstrating that these different formalisms are representations of the same underlying structures. We will establish the correspondences:
		\[
		\underbrace{f}_{\Omega^0}
		\;\xrightarrow{d}\;
		\underbrace{df}_{\Omega^1}
		\;\longleftrightarrow\;
		\underbrace{\nabla f}_{\substack{\text{gradient}\\\text{vector field}}}
		\qquad \text{and} \qquad
		\underbrace{\mathbf F}_{(\Omega^0)^m}
		\;\xrightarrow{d}\;
		\underbrace{d\mathbf F}_{\Omega^1\otimes\R^m}
		\;\longleftrightarrow\;
		\underbrace{D\mathbf F}_{\substack{\text{Jacobian}\\\text{matrix}}}
		\]
	\end{abstract}
	
	\section{Part 1: The Scalar Case -- Gradients and 1-Forms}
	
	\subsection{The Correspondence}
	A smooth scalar function $f$ is a \textbf{0-form}. Its exterior derivative $df$ is a \textbf{1-form} that captures the function's total change. This 1-form is the direct counterpart to the gradient vector field $\nabla f$.
	
	\begin{idea}[The $df \longleftrightarrow \nabla f$ Correspondence]
		The components of the gradient vector field $\nabla f$ are precisely the coefficient functions of the 1-form $df$.
		\begin{equation*}
			\nabla f = \left\langle \frac{\partial f}{\partial x_1}, \dots, \frac{\partial f}{\partial x_n} \right\rangle
			\quad\longleftrightarrow\quad
			df = \frac{\partial f}{\partial x_1} dx_1 + \cdots + \frac{\partial f}{\partial x_n} dx_n
		\end{equation*}
	\end{idea}
	
	\subsection{Penetrating Example 1: The Geometry of a Hill}
	Let's model a hill centered at the origin with a Gaussian function. This function represents the altitude at any point $(x,y)$.
	
	\begin{itemize}
		\item \textbf{The 0-Form (Altitude):} $f(x, y) = e^{-(x^2+y^2)}$. This is a smooth function assigning a height to each point in the plane.
		
		\item \textbf{The Gradient Vector Field (Direction of Steepest Ascent):} The gradient $\nabla f$ is a vector field that, at any point, points in the direction you would have to walk to go uphill most steeply.
		\begin{align*}
			\nabla f &= \left\langle \frac{\partial}{\partial x}\left(e^{-(x^2+y^2)}\right), \frac{\partial}{\partial y}\left(e^{-(x^2+y^2)}\right) \right\rangle \\
			&= \left\langle -2x e^{-(x^2+y^2)}, -2y e^{-(x^2+y^2)} \right\rangle \\
			&= -2e^{-(x^2+y^2)} \langle x, y \rangle
		\end{align*}
		\textbf{Intuition:} The vector $\langle x, y \rangle$ points radially away from the origin. The negative sign means $\nabla f$ points radially \textit{towards} the origin. This makes perfect sense: to go uphill, you must walk towards the peak at $(0,0)$.
		
		\item \textbf{The 1-Form (Work Done / Change in Altitude):} The 1-form $df$ tells us the infinitesimal change in altitude for an infinitesimal step $d\vec{r} = \langle dx, dy \rangle$. It answers the question: "If I move a tiny bit, how much does my height change?"
		\begin{equation*}
			df = -2x e^{-(x^2+y^2)} \, dx - 2y e^{-(x^2+y^2)} \, dy
		\end{equation*}
		Notice the components of $\nabla f$ are the coefficients of $df$. The relationship is explicit. The dot product from calculus, $\nabla f \cdot d\vec{r}$, is precisely how a 1-form "acts on" a vector. Let's test this. Consider a path along a contour line, a circle of radius $R$. This can be parameterized by $\vec{r}(t) = \langle R\cos t, R\sin t \rangle$. The velocity vector is $\vec{v} = \langle -R\sin t, R\cos t \rangle$. The change in height along this path should be zero.
		\begin{align*}
			df(\vec{v}) &= \left(-2x e^{-R^2}\right)(-R\sin t) + \left(-2y e^{-R^2}\right)(R\cos t) \\
			&= \left(-2R\cos t \, e^{-R^2}\right)(-R\sin t) + \left(-2R\sin t \, e^{-R^2}\right)(R\cos t) \\
			&= 2R^2 e^{-R^2} (\cos t \sin t - \sin t \cos t) = 0
		\end{align*}
		The 1-form correctly reports zero change in altitude when moving along a path of constant height.
	\end{itemize}
	
	\section{Part 2: The Vector Case -- Jacobians and Vectors of 1-Forms}
	
	\subsection{The Correspondence}
	A map $\mathbf{F}: \mathbb{R}^n \to \mathbb{R}^m$ is a vector of $m$ 0-forms. Its derivative $d\mathbf{F}$ is a vector of $m$ 1-forms. This object corresponds directly to the Jacobian matrix $D\mathbf{F}$.
	
	\begin{idea}[The $d\mathbf{F} \longleftrightarrow D\mathbf{F}$ Correspondence]
		The vector of 1-forms $d\mathbf{F}$ is the product of the Jacobian matrix $D\mathbf{F}$ and the column vector of basis differentials $d\mathbf{x}$.
		\begin{equation*}
			d\mathbf{F} = (D\mathbf{F}) \, d\mathbf{x}
		\end{equation*}
	\end{idea}
	
	\subsection{Penetrating Example 2: Polar Coordinate Transformation}
	This is the canonical example of a change of coordinates, a fundamental operation in physics and engineering. It maps the polar grid $(r, \theta)$ to the Cartesian grid $(x, y)$.
	
	\begin{itemize}
		\item \textbf{The Vector-Valued Function:} $\mathbf{F}: \mathbb{R}^2 \to \mathbb{R}^2$ maps from the $(r, \theta)$-plane to the $(x, y)$-plane.
		\begin{equation*}
			\mathbf{F}(r, \theta) = \begin{pmatrix} x(r, \theta) \\ y(r, \theta) \end{pmatrix} = \begin{pmatrix} r\cos\theta \\ r\sin\theta \end{pmatrix}
		\end{equation*}
		The components $F_1 = r\cos\theta$ and $F_2 = r\sin\theta$ are our 0-forms.
		
		\item \textbf{The Jacobian Matrix (The Local Scaling and Rotation):} The Jacobian matrix tells us how an infinitesimal rectangle in polar coordinates gets transformed into a parallelogram in Cartesian coordinates.
		\begin{align*}
			D\mathbf{F} = \begin{pmatrix} \frac{\partial x}{\partial r} & \frac{\partial x}{\partial \theta} \\ \frac{\partial y}{\partial r} & \frac{\partial y}{\partial \theta} \end{pmatrix}
			= \begin{pmatrix} \cos\theta & -r\sin\theta \\ \sin\theta & r\cos\theta \end{pmatrix}
		\end{align*}
		\textbf{Intuition:} The determinant of this matrix, $\det(D\mathbf{F}) = r\cos^2\theta - (-r\sin^2\theta) = r$, is the area distortion factor used when changing variables in double integrals: $dx\,dy = r\,dr\,d\theta$.
		
		\item \textbf{The Vector of 1-Forms (The Transformation of Differentials):} We apply $d$ to each component of $\mathbf{F}$ to see how the output differentials ($dx, dy$) relate to the input differentials ($dr, d\theta$).
		\begin{align*}
			dx &= d(r\cos\theta) = \frac{\partial}{\partial r}(r\cos\theta)dr + \frac{\partial}{\partial \theta}(r\cos\theta)d\theta = \cos\theta \, dr - r\sin\theta \, d\theta \\
			dy &= d(r\sin\theta) = \frac{\partial}{\partial r}(r\sin\theta)dr + \frac{\partial}{\partial \theta}(r\sin\theta)d\theta = \sin\theta \, dr + r\cos\theta \, d\theta
		\end{align*}
		So, the vector of 1-forms is $d\mathbf{F} = \begin{pmatrix} dx \\ dy \end{pmatrix} = \begin{pmatrix} \cos\theta \, dr - r\sin\theta \, d\theta \\ \sin\theta \, dr + r\cos\theta \, d\theta \end{pmatrix}$.
		
		Now we verify the correspondence:
		\begin{align*}
			(D\mathbf{F}) \, d\mathbf{x} &= \begin{pmatrix} \cos\theta & -r\sin\theta \\ \sin\theta & r\cos\theta \end{pmatrix} \begin{pmatrix} dr \\ d\theta \end{pmatrix} \\
			&= \begin{pmatrix} (\cos\theta)(dr) + (-r\sin\theta)(d\theta) \\ (\sin\theta)(dr) + (r\cos\theta)(d\theta) \end{pmatrix} = \begin{pmatrix} dx \\ dy \end{pmatrix} = d\mathbf{F}
		\end{align*}
		The relationship is perfect. The Jacobian matrix is the matrix that represents the linear map $d\mathbf{F}$.
	\end{itemize}
	
	\subsection{Penetrating Example 3: A Rotational Fluid Flow}
	Consider a fluid rotating like a solid disk around the $z$-axis with constant angular velocity. The velocity at any point $(x,y,z)$ is given by a vector field.
	
	\begin{itemize}
		\item \textbf{The Vector Field:} $\mathbf{V}(x, y, z) = \langle -y, x, 0 \rangle$. This is a map from $\mathbb{R}^3 \to \mathbb{R}^3$.
		
		\item \textbf{The Jacobian Matrix (The Velocity Gradient Tensor):} In fluid dynamics, this matrix is called the velocity gradient tensor. It describes how the velocity changes from one point to a nearby point, capturing local stretching and rotation (shear).
		\begin{equation*}
			D\mathbf{V} = \begin{pmatrix}
				\frac{\partial V_x}{\partial x} & \frac{\partial V_x}{\partial y} & \frac{\partial V_x}{\partial z} \\
				\frac{\partial V_y}{\partial x} & \frac{\partial V_y}{\partial y} & \frac{\partial V_y}{\partial z} \\
				\frac{\partial V_z}{\partial x} & \frac{\partial V_z}{\partial y} & \frac{\partial V_z}{\partial z}
			\end{pmatrix}
			= \begin{pmatrix}
				0 & -1 & 0 \\
				1 & 0 & 0 \\
				0 & 0 & 0
			\end{pmatrix}
		\end{equation*}
		\textbf{Intuition:} This constant matrix tells us the flow is a uniform shear. The off-diagonal terms $(D\mathbf{V})_{21} = 1$ and $(D\mathbf{V})_{12} = -1$ are responsible for the rotation. The "vorticity" of the fluid is related to the anti-symmetric part of this matrix, which connects directly to the curl. The curl is $\nabla \times \mathbf{V} = \langle 0, 0, \frac{\partial V_y}{\partial x} - \frac{\partial V_x}{\partial y} \rangle = \langle 0, 0, 1 - (-1) \rangle = \langle 0, 0, 2 \rangle$. The non-zero curl is a direct consequence of the asymmetric Jacobian.
		
		\item \textbf{The Vector of 1-Forms:}
		\begin{equation*}
			d\mathbf{V} = \begin{pmatrix} d(-y) \\ d(x) \\ d(0) \end{pmatrix} = \begin{pmatrix} -dy \\ dx \\ 0 \end{pmatrix}
		\end{equation*}
		Let's check the correspondence again:
		\begin{equation*}
			(D\mathbf{V}) \, d\mathbf{x} = \begin{pmatrix} 0 & -1 & 0 \\ 1 & 0 & 0 \\ 0 & 0 & 0 \end{pmatrix} \begin{pmatrix} dx \\ dy \\ dz \end{pmatrix} = \begin{pmatrix} -dy \\ dx \\ 0 \end{pmatrix} = d\mathbf{V}
		\end{equation*}
		The correspondence holds perfectly and reveals the underlying structure of the fluid's motion.
	\end{itemize}
	
	\section{Conclusion}
	Through these examples, we see that the language of differential forms is not merely an abstract reformulation of vector calculus; it is a clarifying and unifying framework.
	\begin{enumerate}
		\item The correspondence between $\nabla f$ and $df$ shows that a gradient is the vector representation of the 1-form that measures infinitesimal change.
		\item The correspondence between the Jacobian $D\mathbf{F}$ and the vector of 1-forms $d\mathbf{F}$ reveals that the Jacobian is the matrix representation of the derivative map, which transforms infinitesimal changes in the domain to infinitesimal changes in the codomain.
	\end{enumerate}
	This perspective is essential for generalizing calculus to curved spaces (manifolds) and for understanding deeper theories in physics and geometry, such as electromagnetism and general relativity.
	
\end{document}
