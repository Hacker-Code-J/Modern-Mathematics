\documentclass[11pt,openany]{article}

\usepackage{mathtools, commath}
% Packages for formatting
\usepackage[margin=1in]{geometry}
\usepackage{fancyhdr}
\usepackage{enumerate}
\usepackage{graphicx}
\usepackage{kotex}
\usepackage{arydshln} % Include this package
\usepackage{bbding}
\usepackage{amsmath}
\usepackage{amsthm}
\usepackage[dvipsnames,table]{xcolor}
\usepackage{amssymb, amsfonts}
\usepackage{wasysym}
\usepackage{footnote}
\usepackage{tablefootnote}
\usepackage{arydshln} % Include this package

% Fonts
\usepackage[T1]{fontenc}
\usepackage[utf8]{inputenc}
\usepackage{newpxtext,newpxmath}
\usepackage{sectsty}

% Define colors
\definecolor{TealBlue1}{HTML}{0077c2}
\definecolor{TealBlue2}{HTML}{00a5e6}
\definecolor{TealBlue3}{HTML}{b3e0ff}
\definecolor{TealBlue4}{HTML}{00293c}
\definecolor{TealBlue5}{HTML}{e6f7ff}

\definecolor{thmcolor}{RGB}{231, 76, 60}
\definecolor{defcolor}{RGB}{52, 152, 219}
\definecolor{lemcolor}{RGB}{155, 89, 182}
\definecolor{corcolor}{RGB}{46, 204, 113}
\definecolor{procolor}{RGB}{241, 196, 15}

\usepackage{color,soul}
\usepackage{soul}
\newcommand{\mathcolorbox}[2]{\colorbox{#1}{$\displaystyle #2$}}
\usepackage{cancel}
\newcommand\crossout[3][black]{\renewcommand\CancelColor{\color{#1}}\cancelto{#2}{#3}}
\newcommand\ncrossout[2][black]{\renewcommand\CancelColor{\color{#1}}\cancel{#2}}

\usepackage{hyperref}
\usepackage{booktabs}

% Chapter formatting
\definecolor{titleTealBlue}{RGB}{0,53,128}
\usepackage{titlesec}
\titleformat{\section}
{\normalfont\sffamily\Large\bfseries\color{titleTealBlue!100!gray}}{\thesection}{1em}{}
\titleformat{\subsection}
{\normalfont\sffamily\large\bfseries\color{titleTealBlue!50!gray}}{\thesubsection}{1em}{}

%Tcolorbox
\usepackage[most]{tcolorbox}
\usepackage{multirow}
\usepackage{multicol}
\usepackage{blindtext}

\usepackage[linesnumbered,ruled]{algorithm2e}
\usepackage{algpseudocode}
\usepackage{setspace}
\SetKwComment{Comment}{/* }{ */}
\SetKwProg{Fn}{Function}{:}{end}
\SetKw{End}{end}
\SetKw{DownTo}{downto}

% Define a new environment for algorithms without line numbers
\newenvironment{algorithm2}[1][]{
	% Save the current state of the algorithm counter
	\newcounter{tempCounter}
	\setcounter{tempCounter}{\value{algocf}}
	% redefine the algorithm numbering (remove prefix)
	\renewcommand{\thealgocf}{}
	\begin{algorithm}
	}{
	\end{algorithm}
	% Restore the algorithm counter state
	\setcounter{algocf}{\value{tempCounter}}
}

\usepackage{adjustbox}
% Header and footer formatting
\pagestyle{fancy}
\fancyhead{}
\fancyhf{}
\rhead{\textcolor{TealBlue2}{\large\textbf{리만의 복소해석을 토대로 얻는 내 수학적 시야 (2기)}}}%\rule{3cm}{0.4pt}}
\lhead{\textcolor{TealBlue2}{\large\textbf{수학의 즐거움, Enjoying Math}}}
% Define footer
%\newcommand{\footer}[1]{
%\begin{flushright}
%	\vspace{2em}
%	\includegraphics[width=2.5cm]{school_logo.jpg} \\
%	\vspace{1em}
%	\textcolor{TealBlue2}{\small\textbf{#1}}
%\end{flushright}
%}
%\rfoot{\large Department of Information Security, Cryptogrphy and Mathematics, Kookmin Uni.\includegraphics[height=1.5cm]{school_logo.jpg}}
\fancyfoot{}
\fancyfoot[C]{-\thepage-}

\usepackage{animate}
% Load the PDF and grab its total pages into \NumPages:
\newcount\NumPagesA
\pdfximage{../riemann-tikz/secant_line_gif.pdf}% loads the PDF
\NumPagesA=\pdflastximagepages
\usepackage{tcolorbox}
\tcbset{colback=white, arc=5pt}

\definecolor{axiomcolor}{HTML}{a88bfa}
\definecolor{defcolor}{RGB}{52, 152, 219}
\definecolor{procolor}{RGB}{241, 196, 15}
\definecolor{thmcolor}{RGB}{231, 76, 60}
\definecolor{lemcolor}{RGB}{155, 89, 182}
\definecolor{corcolor}{RGB}{46, 204, 113}
\definecolor{execolor}{RGB}{90, 128, 127}

% Define a new command for the custom tcolorbox
\newcommand{\axiombox}[2][]{%
	\begin{tcolorbox}[colframe=axiomcolor, title={\color{white}\bfseries #1}]
		#2
	\end{tcolorbox}
}

\newcommand{\defbox}[2][]{%
	\begin{tcolorbox}[colframe=defcolor, title={\color{white}\bfseries #1}]
		#2
	\end{tcolorbox}
}

\newcommand{\probox}[2][]{%
	\begin{tcolorbox}[colframe=procolor, title={\color{white}\bfseries #1}]
		#2
	\end{tcolorbox}
}

\newcommand{\thmbox}[2][]{%
	\begin{tcolorbox}[colframe=thmcolor, title={\color{white}\bfseries #1}]
		#2
	\end{tcolorbox}
}

\newcommand{\lembox}[2][]{%
	\begin{tcolorbox}[colframe=lemcolor, title={\color{white}\bfseries #1}]
		#2
	\end{tcolorbox}
}
\usepackage{amsthm}

% Define custom theorem styles
\newtheoremstyle{dotless} % Name of the style
{3pt} % Space above
{3pt} % Space below
{\itshape} % Body font
{} % Indent amount
{\bfseries} % Theorem head font
{} % Punctuation after theorem head
{2.5mm} % Space after theorem head
{} % Theorem head spec

\newtheoremstyle{definitionstyle} % Name of the style
{3pt} % Space above
{3pt} % Space below
{} % Body font
{} % Indent amount
{\bfseries} % Theorem head font
{.} % Punctuation after theorem head
{2.5mm} % Space after theorem head
{} % Theorem head spec

% Applying custom styles
%\theoremstyle{dotless}
\newtheorem{theorem}{Theorem} % Theorem environment with section-wise numbering
\newtheorem*{theorem*}{Theorem} % Theorem environment with section-wise numbering
\newtheorem*{lemma*}{Lemma} % Theorem environment with section-wise numbering
\newtheorem*{proposition*}{Proposition} % Theorem environment with section-wise numbering
\newtheorem*{corollary*}{Corollary} % Theorem environment with section-wise numbering
\newtheorem{proposition}[theorem]{Proposition} % Theorem environment with section-wise numbering
\newtheorem{lemma}[theorem]{Lemma} % Lemma shares the counter with theorem
\newtheorem{corollary}[theorem]{Corollary} % Corollary shares the counter with theorem

\theoremstyle{definitionstyle}
\newtheorem*{observation}{\textcolor{magenta}{Observation}}
\newtheorem*{illustration}{\textcolor{teal}{Illustration}}
\newtheorem*{torus}{{\color{red}T}{\color{orange}o}{\color{green!75!black}r}{\color{cyan}u}{\color{violet}s}}
\newtheorem{definition}{Definition} % Definition shares the counter with theorem
\newtheorem{example}{Example} % Example shares the counter with theorem
\newtheorem{exercise}{{Exercise}} % Example shares the counter with theorem
\newtheorem{remark}{Remark} % Remark shares the counter with theorem
\newtheorem*{note}{Note}
\newtheorem*{notation}{Notation}

\newtheorem*{axiom*}{Axiom}
\newtheorem*{definition*}{Definition} % Definition shares the counter with theorem
\newtheorem*{example*}{Example} % Example shares the counter with theorem
\newtheorem*{exercise*}{\textcolor{teal}{Exercise}} % Example shares the counter with theorem
\newtheorem*{remark*}{Remark} % Remark shares the counter with theorem


\usepackage{tikz}
\usepackage{tikz-cd}
\usetikzlibrary{shadows}
\usetikzlibrary{shapes.geometric, arrows.meta, positioning}
\input{riemann-complex-commands}
\renewcommand{\vec}[1]{\mathbf{#1}}
\renewcommand{\emph}[1]{\textbf{#1}}
\renewcommand{\d}{\mathrm{d}} % For the exterior derivative 'd'
\newcommand{\pderiv}[2]{\frac{\partial #1}{\partial #2}}
\newcommand{\spderiv}[3]{\frac{\partial^2 #1}{\partial #2\partial #3}}
\newcommand{\vect}[1]{\begin{bmatrix} #1 \end{bmatrix}}

\newcommand{\HH}{\mathbb{H}}
\newcommand{\PP}{\mathbb{P}}
\newcommand{\OO}{\mathcal{O}}
\newcommand{\E}{\mathcal{E}}
%\newcommand{\Res}{\operatorname{Res}}
\newcommand{\Aut}{\operatorname{Aut}}
\newcommand{\ord}{\operatorname{ord}}
\newcommand{\im}{\operatorname{Im}}
\newcommand{\re}{\operatorname{Re}}
\newcommand{\divisor}{\operatorname{div}}
\newcommand{\de}{\,\mathrm{d}}
\newcommand{\la}{\langle}
\newcommand{\ra}{\rangle}
\newcommand{\D}[2]{D(#1,#2)}                % disk D(a,R)
\newcommand{\Ann}[3]{A(#1;#2,#3)}           % annulus A(a;R1,R2)


\newcommand{\circulationsquare}[1]{
\draw[thick, gray] #1 rectangle ++(1,1);
\begin{scope}[decoration={markings, mark=at position 0.5 with {\arrow{>}}}]
	\draw[postaction={decorate}, blue] #1 -- ++(1,0);
	\draw[postaction={decorate}, blue] ++(1,0) -- ++(0,1);
	\draw[postaction={decorate}, blue] ++(0,1) -- ++(-1,0);
	\draw[postaction={decorate}, blue] ++(-1,0) -- cycle;
\end{scope}
}

\usepackage{esvect}
\usepackage{physics}

\setstretch{1.25}

\begin{document}
\pagenumbering{arabic}
\begin{center}
	%	\huge\textbf{Riemann; Complex Analysis}\\
	\huge\textbf{Elliptic Curves, Elliptic Integrals, and \\ the Complex Torus}\\
	%	\Large - HW1 -\\
	\vspace{0.5em}
	\large{Ji, Yong-hyeon}\\
	%	\large{\ttfamily \url{https://github.com/Hacker-Code-J}}\\
	\vspace{0.5em}
	\normalsize{\today}\\
\end{center} 
\noindent 
We cover the following topics in this note.
\begin{itemize}
	\item Vector calculus (conservative fields, irrotational field)
	\item Differential forms (exact forms, closed forms)
\end{itemize}
%In this note, we build a bridge between the familiar concepts of vector calculus (conservative fields, curl) and the language of differential forms (exact forms, closed forms)
%\hrule\vspace{12pt}
\begin{center}
	\begin{tabular*}{\textwidth}{@{\extracolsep{\fill}} l c l}
		\hline
		\textbf{Vector Calculus (in $\R^2$ or $\R^3$)} & & \textbf{Differential Forms} \\
		\hline
		Vector Field $\vec F$ & $\iff$ & 1-form $\omega$ \\
		Conservative Vector Field ($\vec F = \nabla f$) & $\iff$ & Exact 1-form ($\omega = \d f$) \\
		Irrotational Vector Field ($\nabla \times \vec F = \mathbf{0}$) & $\iff$ & Closed 1-form ($\d\omega = 0$) \\
		\hline
	\end{tabular*}
\end{center}

\vspace{1em}

%\begin{framed}
%	\noindent\textbf{The Fundamental Implication:}
%	\begin{itemize}
%		\item \textbf{Conservative $\implies$ Irrotational},\; i.e.,\; \textbf{Exact $\implies$ Closed}\\ (This is always true.)
%		\item \textbf{Irrotational $\implies$ Conservative},\; i.e,\; \textbf{Closed $\implies$ Exact}\\ (This is only true on ``nice'' domains, e.g., simply connected ones.)
%	\end{itemize}
%\end{framed}


\tableofcontents
\newpage
\section{Setup}
Fix a lattice \begin{align*}
\Lambda&=\Z\omega_1+\Z\omega_2=\set{m\omega_1+n\omega_2:m,n\in \Z,\; \omega_1,\omega_2\in\C^\times} \subset\C
\end{align*} with \(\omega_1,\omega_2\) are \(\R\)-linearly independent, i.e., span a rank-2 lattice.
\begin{figure}[h!]\centering
\includegraphics[scale=.8]{tikz-illustrations/lattice.pdf}
\caption{The lattice $\Lambda=\mathbb{Z}\,\omega_1+\mathbb{Z}\,\omega_2\subset\mathbb{C}$.}\label{fig:lattice-lambda}
\end{figure}
\begin{remark}
A \textbf{lattice} in $\C$ must be rank-2 over $\Z$, i.e., $\set{\omega_1,\omega_2}$ is $\R$-linearly independent. Note that \begin{itemize}
	\item $\set{\omega_1,\omega_2}$ is $\R$-linearly dependent $\iff\displaystyle\frac{\omega_2}{\omega_1}\in\R$.
	\item $\set{\omega_1,\omega_2}$ is $\R$-linearly independent $\iff\displaystyle\frac{\omega_2}{\omega_1}\notin\R$.
	\item $\displaystyle\frac{\omega_2}{\omega_1}\notin\R\iff\Im\left(\frac{\omega_2}{\omega_1}\right)\neq 0$.
	\item Let $\omega_1=u_1+iu_2\in\C^\times$, $\omega_2=v_1+iv_2\in\C^\times$, $\vec{u}=(u_1,u_2)\in(\R^\times)^2$ and $\vec{v}=(v_1,v_2)\in(\R^\times)^2$. Then \begin{align*}
	\frac{\omega_2}{\omega_1}&=\frac{v_1+iv_2}{u_1+iu_2}\\
	&=\frac{v_1+iv_2}{u_1+iu_2}\cdot\frac{(u_1-iu_2)}{(u_1-iu_2)}=\frac{u_1v_1+u_2v_2+i(u_1v_2-u_2v_1)}{u_1^2+u_2^2} =\frac{1}{\norm{\vec{u}}^2}\left(\vec{u}\cdot\vec{v} +i\det(\vec{u},\vec{v})\right)
	\end{align*} Thus, we have $\displaystyle\Re(\omega_2/\omega_1)=\frac{\vec{u}\cdot\vec{v}}{\norm{\vec{u}}^2}$ and $\displaystyle\Im(\omega_2/\omega_1)=\frac{1}{\norm{\vec{u}}}\det\begin{pmatrix}
	\Re(\omega_1) & \Im(\omega_1) \\
	\Re(\omega_2) & \Im(\omega_2)
\end{pmatrix}$.
\end{itemize}
\end{remark}

\newpage\noindent
The complex torus is \(X=\C/\Lambda\); write the class of \(z\in\C\) as \([z]\in X\).


\newpage
\subsection*{Holomorphic on a Disk vs.\ Holomorphic on an Annulus}

\paragraph{Domains.}
\[
\text{Disk: } \D{a}{R}=\{\,z\in\C:|z-a|<R\,\},\qquad
\text{Annulus: } \Ann{a}{R_1}{R_2}=\{\,z\in\C:R_1<|z-a|<R_2\,\}.
\]

\paragraph{Canonical expansions.}
\begin{itemize}
	\item \textbf{Disk (Taylor).} If $f$ is holomorphic on $\D{a}{R}$, then
	\[
	f(z)=\sum_{n=0}^{\infty}\frac{f^{(n)}(a)}{n!}(z-a)^n,\qquad |z-a|<R.
	\]
	\item \textbf{Annulus (Laurent).} If $f$ is holomorphic on $\Ann{a}{R_1}{R_2}$, then
	\[
	f(z)=\sum_{n=-\infty}^{\infty}c_n(z-a)^n,\qquad R_1<|z-a|<R_2,
	\]
	with coefficients given by the same Cauchy integral formula
	\[
	\boxed{\,c_n=\frac{1}{2\pi i}\oint_{|\zeta-a|=r}\frac{f(\zeta)}{(\zeta-a)^{n+1}}\,d\zeta\,,\quad
		R_1<r<R_2,\ n\in\mathbb{Z}\,}
	\]
	(In particular, $c_{-1}=\operatorname{Res}(f,a)$.)
\end{itemize}

\paragraph{Comparison table.}
\begin{table}[h]
	\centering
	\begin{tabular}{p{0.19\linewidth}|p{0.28\linewidth}|p{0.30\linewidth}|p{0.17\linewidth}}
		\textbf{Region} & \textbf{Expansion type} & \textbf{Coefficient formula} & \textbf{Negative powers} \\ \hline
		Disk $\D{a}{R}$ &
		Taylor series $\displaystyle \sum_{n\ge0}\frac{f^{(n)}(a)}{n!}(z-a)^n$ &
		$\displaystyle \frac{f^{(n)}(a)}{n!}=\frac{1}{2\pi i}\oint\frac{f(\zeta)}{(\zeta-a)^{n+1}}\,d\zeta$ &
		None (holomorphic at $a$) \\
		Annulus $\Ann{a}{R_1}{R_2}$ &
		Laurent series $\displaystyle \sum_{n\in\mathbb{Z}} c_n (z-a)^n$ &
		$\displaystyle c_n=\frac{1}{2\pi i}\oint\frac{f(\zeta)}{(\zeta-a)^{n+1}}\,d\zeta$ &
		Allowed; principal part $\sum_{n\le-1} c_n (z-a)^n$
	\end{tabular}
	\caption{Taylor (disk) vs.\ Laurent (annulus).}
\end{table}

\paragraph{Radii of convergence.}
\begin{itemize}
	\item \textbf{Disk:} $R=$ distance from $a$ to the nearest singularity of $f$.
	\item \textbf{Annulus:} inner $R_1=$ distance from $a$ to the nearest singularity \emph{inside}, outer $R_2=$ distance to the nearest singularity \emph{outside}.
\end{itemize}

\paragraph{Classification at the center $a$ (when a punctured neighborhood exists).}
\begin{itemize}
	\item Principal part $=0$ $\Rightarrow$ removable singularity (extends holomorphically; Taylor only).
	\item Finite principal part $\Rightarrow$ pole (order = largest $-n$ with $c_n\neq0$).
	\item Infinite principal part $\Rightarrow$ essential singularity.
\end{itemize}

\paragraph{Examples.}
\begin{align*}
	\frac{1}{1-z} &= \sum_{n\ge0} z^n \quad (|z|<1),\\[2mm]
	\frac{1}{z(z-1)} &= \frac{1}{z}-\frac{1}{z-1}
	= \begin{cases}
		\displaystyle \frac{1}{z}+1+z+z^2+\cdots,& 0<|z|<1,\\[1mm]
		\displaystyle \frac{1}{z}+\frac{1}{z^2}+\frac{1}{z^3}+\cdots,& |z|>1~.
	\end{cases}
\end{align*}

\paragraph{One-line takeaway.}
\emph{Holomorphic on a disk $\Rightarrow$ Taylor (no negative powers). Holomorphic on an annulus $\Rightarrow$ Laurent (negative powers allowed; $c_{-1}$ is the residue).}
% --- End snippet ---

\newpage
\begin{center}
\begin{tabular}{|p{4.5cm}|p{4.75cm}|p{4.75 cm}|}
	\hline
	& $\operatorname{ord}_a f$ & $\operatorname{Res}(f,a)$ \\
	\hline
	Basic idea
	& Measures how $f$ vanishes or blows up at $a$
	& Coefficient of $(z-a)^{-1}$ in the Laurent expansion of $f$ at $a$ \\
	\hline
	Defined by
	& $f(z) = (z-a)^m g(z)$ with $g(a)\neq 0 \;\Rightarrow\; \operatorname{ord}_a f = m$
	& If $f(z) = \sum_{n=-\infty}^\infty c_n (z-a)^n$, then $\operatorname{Res}(f,a) = c_{-1}$ \\
	\hline
	Possible values
	& Integer ($\dots,-2,-1,0,1,2,\dots$)
	& Complex number \\
	\hline
	Interpretation of sign
	& $>0$: zero of order $m$; $=0$: $f(a)\neq 0$; $<0$: pole of order $-m$
	& Nonzero $\iff$ the Laurent series has a $(z-a)^{-1}$ term \\
	\hline
	When $f$ is holomorphic and $f(a)\neq 0$
	& $\operatorname{ord}_a f = 0$
	& $\operatorname{Res}(f,a) = 0$ \\
	\hline
	When $f$ has a zero of order $m\ge 1$ at $a$
	& $\operatorname{ord}_a f = m$
	& $\operatorname{Res}(f,a) = 0$ (no negative powers) \\
	\hline
	When $f$ has a pole of order $k\ge 1$ at $a$
	& $\operatorname{ord}_a f = -k$
	& May be $0$ or nonzero, depends on $(z-a)^{-1}$ term \\
	\hline
	Key relation
	& $\operatorname{ord}_a f = \operatorname{Res}\!\left(\dfrac{f'}{f}, a\right)$
	& $\operatorname{Res}\!\left(\dfrac{f'}{f}, a\right) = \operatorname{ord}_a f$ \\
	\hline
	Main use
	& Divisors, counting zeros/poles, argument principle
	& Contour integration, residue theorem \\
	\hline
\end{tabular}
\end{center}

\begin{center}
\begin{table}[h]
	\centering
	\begin{tabular}{c|c|l}
		\(\operatorname{ord}_a f\) & \(\operatorname{Res}(f,a)\) & Description \\ \hline
		\(> 0\)     & \(0\)                      & \(f\) has a zero at \(a\) \\
		\(= 0\)     & \(0\)                      & \(f\) holomorphic and nonzero at \(a\) \\
		\(-1\)      & \(\neq 0\) in general      & simple pole; residue is coeff.\ of \((z-a)^{-1}\) \\
		\(< -1\)    & \(0\)                      & pole of order \(\ge 2\); residue vanishes
	\end{tabular}
	\caption{Relation between the order of \(f\) at \(a\) and the residue \(\operatorname{Res}(f,a)\).}
\end{table}
\end{center}

\begin{itemize}
	\item A \emph{1-form} on $\C$ is something we can integrate along curves; the basic ones look like $\omega=f(z)\,dz$.
	\item If $f$ is holomorphic (complex differentiable), then $\omega$ is a \emph{holomorphic 1-form}.
	\item Given a path $\gamma:[a,b]\to\mathbb C$, we compute
	\[
	\int_\gamma \omega \;=\; \int_a^b f(\gamma(t))\,\gamma'(t)\,dt.
	\]
\end{itemize}

\newpage
\begin{theorem}[No nonconstant holomorphic functions on compact Riemann surfaces]
	Let $X$ be a compact, connected Riemann surface. If $f:X\to\mathbb{C}$ is holomorphic, then $f$ is constant.
\end{theorem}

\begin{proof}
	Let $Z(f)=\{p\in X:\, f(p)=0\}$ be the (finite) set of zeros of $f$, each with multiplicity
	$\operatorname{ord}_p(f)\in\mathbb{Z}_{\ge 0}$. Consider the (logarithmic / winding) $1$-form
	\[
	\omega \;:=\; \frac{1}{2\pi i}\,\frac{df}{f}
	\qquad\text{on }X\setminus Z(f).
	\]
	This form is closed on $X\setminus Z(f)$ (indeed $d\omega=\frac{1}{2\pi i}d\!\left(\frac{df}{f}\right)=0$ there)
	and has integer periods around small loops encircling the points of $Z(f)$, as recorded below.
	
	\begin{lemma}[Winding / argument principle in local form]
		For $p\in Z(f)$ and a sufficiently small positively oriented circle $\gamma_p$ around $p$
		(in a local coordinate $z$ with $z(p)=0$), one has
		\[
		\int_{\gamma_p}\omega \;=\; \operatorname{ord}_p(f).
		\]
	\end{lemma}
	
	\begin{proof}[Proof of the lemma]
		In a local coordinate $z$ at $p$ we can write $f(z)=z^{m}g(z)$ with $m=\operatorname{ord}_p(f)$ and $g$ holomorphic and nonvanishing near $0$.
		Then
		\[
		\frac{df}{f} \;=\; \frac{m\,dz}{z} \;+\; \frac{dg}{g}.
		\]
		Since $\frac{dg}{g}$ is holomorphic, its integral over a small circle vanishes by Cauchy's theorem, while
		$\int_{\gamma_p}\frac{dz}{z}=2\pi i$. Therefore
		\[
		\int_{\gamma_p}\omega \;=\; \frac{1}{2\pi i}\int_{\gamma_p}\frac{df}{f}
		\;=\; \frac{1}{2\pi i}\Big(m\int_{\gamma_p}\frac{dz}{z}\Big) \;=\; m.
		\]
	\end{proof}
	
	Now remove from $X$ a disjoint union of small discs $D_p$ centered at each $p\in Z(f)$, and set
	\[
	X_\varepsilon \;:=\; X\setminus \bigcup_{p\in Z(f)} D_p.
	\]
	Then $X_\varepsilon$ is a compact manifold \emph{with} boundary
	\(
	\partial X_\varepsilon = \bigsqcup_{p\in Z(f)} (-\partial D_p),
	\)
	where $-\partial D_p$ denotes the boundary circle with the induced (outward) orientation of $X_\varepsilon$.
	Since $d\omega=0$ on $X_\varepsilon$, the generalized Stokes theorem gives
	\[
	0 \;=\; \int_{X_\varepsilon} d\omega \;=\; \int_{\partial X_\varepsilon} \omega
	\;=\; \sum_{p\in Z(f)} \int_{-\partial D_p}\omega
	\;=\; - \sum_{p\in Z(f)} \int_{\partial D_p}\omega
	\;=\; - \sum_{p\in Z(f)} \operatorname{ord}_p(f).
	\]
	Hence
	\[
	\sum_{p\in Z(f)} \operatorname{ord}_p(f) \;=\; 0.
	\]
	Because each $\operatorname{ord}_p(f)\ge 0$, it follows that $\operatorname{ord}_p(f)=0$ for all $p$, i.e.\ $f$ has no zeros on $X$.
	
	Consequently $u:=\log|f|$ is a globally defined harmonic function on $X$ (in local coordinates,
	$\Delta u=0$ wherever $f\neq 0$, and we just proved $f$ never vanishes). On a compact, connected
	Riemann surface every harmonic function is constant (by the maximum principle, or by integrating $|\nabla u|^2$).
	Therefore $|f|$ is constant on $X$.
	
	Finally, a holomorphic map whose modulus is constant must itself be constant: otherwise its image would be
	an open subset of the circle $\{w\in\mathbb{C}:|w|=\text{const}\}$, contradicting the open mapping theorem.
	Hence $f$ is constant.
\end{proof}

\bigskip

\noindent\textbf{Remark (Meromorphic variant).}
If $f$ were merely meromorphic on $X$, the same argument (with the same $\omega$) yields the residue/divisor
identity
\[
\sum_{p\in X} \operatorname{ord}_p(f) \;=\; 0,
\]
i.e.\ the divisor of a meromorphic function has degree zero. The holomorphic case corresponds to ``no poles'',
forcing the sum of zeros to vanish and hence no zeros at all.


\begin{theorem}[Degree of a meromorphic function’s divisor is zero]
	Let $X$ be a compact, connected Riemann surface and let $f\not\equiv 0$ be meromorphic on $X$.
	Then
	\[
	\sum_{p\in X}\operatorname{ord}_p(f)=0,
	\]
	i.e.\ the sum of the orders of zeros (positive) and poles (negative) of $f$ is $0$.
\end{theorem}

\begin{proof}
	Let $S=\{p\in X:\text{$p$ is a zero or pole of $f$}\}$, which is finite by compactness.
	On $X\setminus S$ consider the \emph{logarithmic} 1\!--form
	\[
	\omega \;=\; \frac{1}{2\pi i}\,\frac{df}{f}.
	\]
	Since $\frac{df}{f}$ is holomorphic on $X\setminus S$, we have $d\omega=0$ there.
	
	Choose pairwise disjoint coordinate discs $\{D_p\}_{p\in S}$ centered at the points of $S$, small
	enough so that $f$ has no zeros or poles on any $\partial D_p$, and set
	\[
	X_\varepsilon \;:=\; X \setminus \bigcup_{p\in S} D_p .
	\]
	Then $X_\varepsilon$ is a compact surface with boundary
	$\partial X_\varepsilon \;=\; \bigsqcup_{p\in S} (-\partial D_p)$,
	where $-\partial D_p$ denotes the boundary circle with the induced (outward) orientation of $X_\varepsilon$.
	
	By the generalized Stokes theorem,
	\[
	0 \;=\; \int_{X_\varepsilon} d\omega
	\;=\; \int_{\partial X_\varepsilon} \omega
	\;=\; \sum_{p\in S}\int_{-\partial D_p}\omega
	\;=\; -\sum_{p\in S}\int_{\partial D_p}\omega.
	\tag{1}
	\]
	
	It remains to compute the boundary integrals. Fix $p\in S$ and take a local coordinate $z$ with $z(p)=0$.
	Write
	\[
	f(z)=z^m g(z), \qquad m=\operatorname{ord}_p(f)\in\mathbb{Z}, \quad g \text{ holomorphic and } g(0)\neq 0.
	\]
	Then
	\[
	\frac{df}{f}=\frac{m\,dz}{z}+\frac{dg}{g},
	\]
	where $\frac{dg}{g}$ is holomorphic near $0$. Hence for a small positively oriented circle
	$\gamma_p=\partial D_p$ we have, by Cauchy,
	\[
	\int_{\gamma_p}\frac{dg}{g}=0 \quad\text{and}\quad
	\int_{\gamma_p}\frac{dz}{z}=2\pi i,
	\]
	so
	\[
	\int_{\partial D_p}\omega
	=\frac{1}{2\pi i}\int_{\partial D_p}\frac{df}{f}
	=\frac{1}{2\pi i}\bigg(m\int_{\partial D_p}\frac{dz}{z}\bigg)
	= m
	=\operatorname{ord}_p(f).
	\tag{2}
	\]
	Substituting \((2)\) into \((1)\) gives
	\[
	0 \;=\; -\sum_{p\in S}\operatorname{ord}_p(f),
	\]
	hence \(\sum_{p\in X}\operatorname{ord}_p(f)=0\).
	
	\smallskip
	\emph{Residue viewpoint.}
	Equivalently, $\frac{df}{f}$ has at worst simple poles at $S$ with
	$\operatorname{Res}_p\!\left(\frac{df}{f}\right)=\operatorname{ord}_p(f)$.
	Since $X$ is compact,
	\(\sum_{p\in X}\operatorname{Res}_p\!\left(\frac{df}{f}\right)=0\),
	yielding the same conclusion after dividing by \(2\pi i\).
\end{proof}

\newpage

\begin{theorem}[Holomorphic functions on compact Riemann surfaces are constant]
	Let $X$ be a compact, connected Riemann surface and let $f:X\to\mathbb{C}$ be holomorphic. Then $f$ is constant.
\end{theorem}

We give a detailed proof split into transparent steps. Throughout, $f\not\equiv 0$ unless explicitly stated (if $f\equiv 0$ the theorem is trivial).

\subsection*{Step 1: The sum of orders of zeros equals $0$}
\begin{claim}\label{claim:sumzero}
	If $f$ is holomorphic on $X$, then the set of zeros $Z(f)$ is finite and
	\[
	\sum_{p\in Z(f)} \operatorname{ord}_p(f) \;=\; 0.
	\]
\end{claim}

\begin{proof}[Proof of finiteness]
	Zeros of a holomorphic function on a Riemann surface are isolated. Since $X$ is compact, an infinite set of isolated points would have an accumulation point, contradicting isolatedness. Hence $Z(f)$ is finite.
\end{proof}

\begin{proof}[Proof of the identity]
	Let $S=Z(f)$ (there are no poles because $f$ is holomorphic). On $X\setminus S$ define the \emph{logarithmic form}
	\[
	\omega:=\frac{1}{2\pi i}\frac{df}{f}.
	\]
	This $1$-form is smooth and \emph{closed} on $X\setminus S$ because $f$ is holomorphic and nonvanishing there:
	$d\omega = \frac{1}{2\pi i}d\big(\frac{df}{f}\big)=0$ on $X\setminus S$.
	Choose pairwise disjoint coordinate discs $\{D_p\}_{p\in S}$ so that $f$ has no zeros on the boundary circles $\gamma_p:=\partial D_p$, and set
	\[
	X_\varepsilon:=X\setminus \bigcup_{p\in S}D_p.
	\]
	Then $X_\varepsilon$ is a compact manifold \emph{with} boundary
	\(\partial X_\varepsilon=\bigsqcup_{p\in S}(-\gamma_p)\), where the minus sign denotes the outward orientation induced from $X_\varepsilon$ (i.e.\ the geometric boundary circles are traversed negatively relative to the positive orientation of the small discs).
	
	By the generalized Stokes theorem:
	\[
	0=\int_{X_\varepsilon} d\omega \;=\; \int_{\partial X_\varepsilon}\omega
	\;=\; \sum_{p\in S}\int_{-\gamma_p}\omega
	\;=\; -\sum_{p\in S}\int_{\gamma_p}\omega.
	\tag{$\ast$}
	\]
	Thus it remains to compute each boundary integral. Fix $p\in S$ and take a local coordinate $z$ with $z(p)=0$. By Weierstrass preparation,
	\(
	f(z)=z^m g(z)
	\)
	with $m=\operatorname{ord}_p(f)\in\mathbb{Z}_{\ge 0}$ and $g$ holomorphic, $g(0)\neq 0$. Then
	\[
	\frac{df}{f}=\frac{m\,dz}{z}+\frac{dg}{g}.
	\]
	The $1$-form $\frac{dg}{g}$ is holomorphic in a neighborhood of $0$, so $\int_{\gamma_p}\frac{dg}{g}=0$ by Cauchy's theorem. Also $\int_{\gamma_p}\frac{dz}{z}=2\pi i$. Therefore
	\[
	\int_{\gamma_p}\omega=\frac{1}{2\pi i}\int_{\gamma_p}\frac{df}{f}
	=\frac{1}{2\pi i}\cdot m\int_{\gamma_p}\frac{dz}{z}=m=\operatorname{ord}_p(f).
	\]
	Plugging this into $(\ast)$ yields $0=-\sum_{p\in S}\operatorname{ord}_p(f)$, i.e.\ $\sum_{p\in Z(f)}\operatorname{ord}_p(f)=0$.
\end{proof}

\subsection*{Step 2: Nonnegativity forces no zeros}
\begin{claim}\label{claim:nozeros}
	For holomorphic $f$, each $\operatorname{ord}_p(f)\ge 0$. Hence \(\sum_{p\in Z(f)}\operatorname{ord}_p(f)=0\)
	implies $\operatorname{ord}_p(f)=0$ for all $p$, so $Z(f)=\varnothing$.
\end{claim}

\begin{proof}
	By definition, for a holomorphic function the order at a zero is the nonnegative integer $m$ such that $f(z)=z^m g(z)$ with $g(0)\neq 0$. Thus all summands are $\ge 0$. A finite sum of nonnegative integers can be $0$ only if each is $0$.
\end{proof}

\begin{remark}[Why we only summed over zeros]
	In the meromorphic case, the residue/Stokes computation gives
	\(\sum_{p\in X}\operatorname{ord}_p(f)=0\), where pole orders are negative. In the holomorphic case there are no poles, so the sum is over zeros only, and nonnegativity forces triviality.
\end{remark}

\subsection*{Step 3: Global harmonicity of \(u=\log|f|\)}
\begin{claim}\label{claim:harmonic}
	If $f$ is holomorphic and \emph{nowhere vanishing} on $X$, then $u:=\log|f|$ is a well-defined global \emph{harmonic} real-valued function on $X$.
\end{claim}

\begin{proof}
	\textbf{(Well-definedness)} Since $f$ never vanishes, $|f|>0$ and $\log|f|$ is a single-valued continuous function globally (no branch issue, because we take the real logarithm of the positive function $|f|$).
	
	\smallskip
	\textbf{(Harmonicity is local.)} A function is harmonic if and only if it is (real) harmonic in every coordinate chart; this is local. So fix a coordinate $z=x+iy$ on a simply connected chart $U\Subset X$. Because $f$ is holomorphic and never zero on $U$, there exists a holomorphic logarithm $g$ on $U$ with $e^{g}=f$ (define $g$ by integrating $f'/f$, or pick a holomorphic branch of $\log f$). Write $g=\varphi+i\psi$ with real-valued $\varphi,\psi$. Then
	\[
	|f|=|e^g|=e^{\Re g}=e^{\varphi}\quad\Longrightarrow\quad u=\log|f|=\varphi.
	\]
	But the real part of a holomorphic function is harmonic: $\Delta \varphi=0$ in the Euclidean Laplacian $\Delta=\partial_x^2+\partial_y^2$. Hence $u$ is harmonic on each such chart; therefore $u$ is harmonic globally on $X$.
	
	\smallskip
	\textbf{(Equivalent differential-form computation)} In a complex coordinate $z$, write $\partial=\frac{\partial}{\partial z}\,dz$ and $\bar\partial=\frac{\partial}{\partial\bar z}\,d\bar z$. Locally choose a holomorphic branch $\log f$. Then
	\[
	\partial\bar\partial \log|f|
	=\frac12\,\partial\bar\partial\big(\log f+\log\bar f\big)
	=\tfrac12\big(\underbrace{\partial\bar\partial\log f}_{=0}+\underbrace{\partial\bar\partial\log\bar f}_{=0}\big)=0,
	\]
	because $\bar\partial\log f=0$ and $\partial\log\bar f=0$. Since the Laplacian is a (nonzero) scalar multiple of $\partial\bar\partial$ in complex notation, this is another way to see $\Delta u=0$.
\end{proof}

\subsection*{Step 4: Harmonic functions on compact Riemann surfaces are constant}
\begin{proposition}\label{prop:harmonic-constant}
	If $X$ is compact and connected, any harmonic function $u:X\to\mathbb{R}$ is constant.
\end{proposition}

\begin{proof}[Two standard proofs]
	\textbf{(Maximum principle)} A harmonic function satisfies the strong maximum principle: it cannot achieve a nonconstant local maximum or minimum in the interior. By compactness, $u$ attains a global max and min on $X$; by the maximum principle both are attained on every neighborhood, forcing $u$ to be constant.
	
	\smallskip
	\textbf{(Energy identity)} Equivalently, integrate by parts. In a conformal coordinate $z=x+iy$,
	\[
	\int_X |\nabla u|^2\,dA
	= -\int_X u\,\Delta u\,dA \;+\; \underbrace{\int_{\partial X} u\,\partial_\nu u\,ds}_{=0}
	= 0,
	\]
	since $\Delta u=0$ and $\partial X=\varnothing$. Thus $|\nabla u|\equiv 0$, hence $u$ is constant on each chart, and by connectedness on $X$.
\end{proof}

\subsection*{Step 5: From constant modulus to constant function}
\begin{claim}\label{claim:modulus-constant-implies-constant}
	If $f$ is holomorphic on a connected Riemann surface and $|f|$ is constant, then $f$ is constant.
\end{claim}

\begin{proof}[Two ways]
	\textbf{(Open mapping theorem)} A nonconstant holomorphic map is open, so $f(X)$ would be an open subset of $\{w:|w|=\mathrm{const}\}$, a circle, which has empty interior—a contradiction.
	
	\smallskip
	\textbf{(Differential identity)} Let $F=|f|^2=f\bar f$. In a local coordinate $z$,
	\(
	\partial F = (\partial f)\,\bar f = f'(z)\,\overline{f(z)}\,dz.
	\)
	If $F$ is constant then $\partial F=0$, so $f'(z)\,\overline{f(z)}=0$. On the locus where $f\neq 0$ we get $f'(z)=0$. But the set $\{f=0\}$ is discrete for holomorphic $f$, hence $f'$ vanishes on a dense open set and by analyticity $f'\equiv 0$ on $X$. Therefore $f$ is constant on each connected component, hence on $X$.
\end{proof}

\subsection*{Putting it all together}
From Claim~\ref{claim:sumzero} and Claim~\ref{claim:nozeros}, a holomorphic $f$ on compact $X$ has \emph{no zeros} (unless $f\equiv 0$). Therefore $u=\log|f|$ is globally well-defined and harmonic by Claim~\ref{claim:harmonic}. By Proposition~\ref{prop:harmonic-constant}, $u$ is constant; hence $|f|$ is constant. Finally, by Claim~\ref{claim:modulus-constant-implies-constant}, $f$ is constant.

\begin{remark}[What changes for meromorphic $f$?]
	If $f$ is meromorphic, the same Stokes argument gives
	\(
	\sum_{p\in X}\operatorname{ord}_p(f)=0,
	\)
	with negative contributions from poles; this is the statement that the divisor of a meromorphic function has degree $0$. The holomorphic case is the special case “no poles,” forcing the sum of zero-orders to vanish and hence to be trivial termwise.
\end{remark}

\newpage
\begin{theorem}[Harmonic $\Rightarrow$ constant on compact Riemann surfaces]
	Let $X$ be a compact, connected Riemann surface (thus a compact oriented $2$-manifold with a conformal structure).
	If $u\in C^\infty(X,\mathbb{R})$ is harmonic, then $u$ is constant.
\end{theorem}

\begin{proof}[Proof via Hodge star and generalized Stokes]
	Equip $X$ with any Riemannian metric compatible with the complex structure (e.g.\ from a local conformal coordinate).
	Let $d$ be the exterior derivative and $*$ the Hodge star. The codifferential is $\delta:= - * d *$ on $1$-forms,
	and the Laplacian on functions is $\Delta u = \delta(du) = - * d * du$.
	Harmonicity means $\Delta u = 0$, equivalently $d * du = 0$.
	
	Consider the $1$-form $u\, * du$ on $X$.
	By the Leibniz rule,
	\[
	d(u * du) \;=\; du \wedge * du \;+\; u\, d * du.
	\]
	Since $u$ is harmonic, $d*du=0$, hence
	\[
	d(u * du) \;=\; du \wedge * du.
	\]
	Integrate over $X$ and apply the generalized Stokes theorem (no boundary since $X$ is compact):
	\[
	\int_X du \wedge * du \;=\; \int_X d(u * du) \;=\; \int_{\partial X} u * du \;=\; 0.
	\]
	But pointwise $du \wedge * du = |du|^2 \, d\mathrm{vol}$, hence
	\[
	\int_X |du|^2 \, d\mathrm{vol} \;=\; 0.
	\]
	Therefore $|du|^2 \equiv 0$ on $X$, so $du \equiv 0$. Since $X$ is connected, $u$ is constant.
\end{proof}

\bigskip

\begin{theorem}[Equivalent $\partial\bar\partial$-proof]
	In complex notation, let $\partial,\bar\partial$ be the Dolbeault operators and fix the standard normalization
	$\Delta u = 4 \, \partial\bar\partial u$ (as a $(1,1)$-form after identifying functions with $(0,0)$-forms).
	If $u$ is harmonic (so $\partial\bar\partial u=0$), then $u$ is constant.
\end{theorem}

\begin{proof}
	On a Riemann surface,
	\[
	i\,\partial u \wedge \bar\partial u
	\;=\;
	\frac{i}{2}\,\partial\bar\partial(u^2)
	\;-\; i\,u\,\partial\bar\partial u.
	\]
	Integrate over $X$:
	\[
	\int_X i\,\partial u \wedge \bar\partial u
	=
	\frac{i}{2}\int_X \partial\bar\partial(u^2)
	\;-\; i\int_X u\,\partial\bar\partial u.
	\]
	The first term vanishes by Stokes (exact form on a compact manifold), and the second vanishes by harmonicity
	($\partial\bar\partial u=0$). Hence
	\[
	\int_X i\,\partial u \wedge \bar\partial u \;=\; 0.
	\]
	Pointwise, $i\,\partial u \wedge \bar\partial u$ is a nonnegative $(1,1)$-form equal to
	$\tfrac{1}{2}|\nabla u|^2\, d\mathrm{vol}$ (up to a harmless normalization); thus it vanishes identically.
	Therefore $\partial u \equiv 0$, so $u$ is (anti)holomorphic and real-valued, hence constant.
\end{proof}

\newpage
\begin{theorem}[Harmonic $\Rightarrow$ constant on compact Riemann surfaces, no Hodge star]
	Let $X$ be a compact, connected Riemann surface. If $u\in C^\infty(X,\mathbb{R})$ is harmonic, then $u$ is constant.
\end{theorem}

\subsection*{Proof 1 (Dolbeault $\partial\bar\partial$ + Stokes)}
On a Riemann surface, harmonicity is equivalent to $\partial\bar\partial u=0$ (the complex Laplacian).
Compute the $(1,1)$-form identity
\[
i\,\partial u\wedge \bar\partial u
\;=\;
\frac{i}{2}\,\partial\bar\partial(u^2)\;-\; i\,u\,\partial\bar\partial u.
\]
Integrate over $X$ and apply the generalized Stokes theorem:
\[
\int_X i\,\partial u\wedge \bar\partial u
=
\frac{i}{2}\int_X \partial\bar\partial(u^2)
\;-\; i\int_X u\,\partial\bar\partial u
=0-0=0,
\]
since $X$ has no boundary and $\partial\bar\partial u=0$.

Locally, in a coordinate $z=x+iy$, one checks
\[
i\,\partial u\wedge \bar\partial u
=\frac12\left( u_x^2+u_y^2\right)\,dx\wedge dy,
\]
so the integrand is pointwise nonnegative. Hence the integral vanishes only if
$u_x=u_y\equiv 0$ everywhere; thus $u$ is constant on each component, and $X$ is connected.

\subsection*{Proof 2 (Real-variable $1$-form trick + Stokes)}
Choose a local real coordinate chart $(x,y)$ compatible with the complex structure (possible locally everywhere).
Set
\[
\alpha \;:=\; -u_y\,dx + u_x\,dy
\quad\text{(a $90^\circ$ rotation of the gradient as a $1$-form).}
\]
A direct computation gives the two identities
\[
du \wedge \alpha
=\big(u_x\,dx+u_y\,dy\big)\wedge\big(-u_y\,dx+u_x\,dy\big)
=\big(u_x^2+u_y^2\big)\,dx\wedge dy,
\]
\[
d\alpha
= -u_{yx}\,dx\wedge dx - u_{yy}\,dy\wedge dx + u_{xx}\,dx\wedge dy + u_{xy}\,dy\wedge dy
=(u_{xx}+u_{yy})\,dx\wedge dy
=(\Delta u)\,dx\wedge dy.
\]
Now apply Leibniz to the $1$-form $u\,\alpha$:
\[
d(u\,\alpha)=du\wedge \alpha + u\,d\alpha
=\big(|\nabla u|^2 + u\,\Delta u\big)\,dx\wedge dy.
\]
If $u$ is harmonic, $\Delta u=0$, so $d(u\,\alpha)=|\nabla u|^2\,dx\wedge dy$.
Integrate over the compact surface $X$ and use Stokes (no boundary):
\[
\int_X |\nabla u|^2\,dx\wedge dy
=\int_X d(u\,\alpha)
=\int_{\partial X} u\,\alpha
=0.
\]
Hence $|\nabla u|^2\equiv 0$, so $du\equiv 0$ and $u$ is constant on the connected $X$.

\newpage
\begin{theorem}[Harmonic $\Rightarrow$ constant on a compact Riemann surface, no Hodge star]
	Let $X$ be a compact, connected Riemann surface. If $u\in C^\infty(X,\mathbb{R})$ is harmonic, then $u$ is constant.
\end{theorem}

\paragraph{What ``harmonic'' means here.}
A Riemann surface carries a conformal (angle-preserving) atlas. In any \emph{conformal} local coordinate
$(x,y)$ on $U\subset X$, harmonicity means the \emph{flat} Laplacian vanishes:
\[
\Delta u \;:=\; u_{xx}+u_{yy} \;=\; 0 \quad \text{on } U.
\]
(Equivalently, with any conformal metric $\lambda(x,y)(dx^2+dy^2)$, the Laplace--Beltrami operator is
$\Delta_g u=\lambda^{-1}(u_{xx}+u_{yy})$, so $\Delta_g u=0 \iff u_{xx}+u_{yy}=0$.)

\medskip

\paragraph{Key local calculation (purely real-variable).}
On any conformal chart $(x,y)$, define the $1$-form
\[
\alpha \;:=\; -\,u_y\,dx \;+\; u_x\,dy .
\]
A direct computation (product rule + $dx\wedge dx=0=dy\wedge dy$) gives the identities
\begin{align*}
	du \wedge \alpha
	&= (u_x\,dx+u_y\,dy)\wedge(-u_y\,dx+u_x\,dy)
	\;=\; (u_x^2+u_y^2)\,dx\wedge dy, \\
	d\alpha
	&= -u_{yx}\,dx\wedge dx - u_{yy}\,dy\wedge dx
	+ u_{xx}\,dx\wedge dy + u_{xy}\,dy\wedge dy \\
	&= (u_{xx}+u_{yy})\,dx\wedge dy
	\;=\; (\Delta u)\,dx\wedge dy .
\end{align*}
Therefore, for the $1$-form $u\,\alpha$ we have
\[
d(u\,\alpha) \;=\; du\wedge\alpha \;+\; u\,d\alpha
\;=\; \big(u_x^2+u_y^2 + u\,\Delta u\big)\,dx\wedge dy .
\tag{$\ast$}
\]

\paragraph{Conformal invariance of the right-hand side.}
If $(u,v)$ is another conformal chart with $(u,v)=(u(x,y),v(x,y))$ coming from a holomorphic change of
coordinates, then
\[
du\wedge dv \;=\; J\,dx\wedge dy, \qquad
u_u^2+u_v^2 \;=\; J^{-1}\,(u_x^2+u_y^2),
\]
where $J>0$ is the Jacobian factor (for a conformal change, $J=|h'(z)|^2$).
Hence
\[
\big(u_u^2+u_v^2\big)\,du\wedge dv \;=\; (u_x^2+u_y^2)\,dx\wedge dy,
\]
and similarly $(\Delta u)\,dx\wedge dy=(\Delta u)\,du\wedge dv / J$ transforms the same way. Thus the
$2$-form on the right-hand side of $(\ast)$ is independent of the chosen conformal chart. Consequently, the
following $2$-form is \emph{globally} well-defined on $X$:
\[
\Omega \;:=\; \big(u_x^2+u_y^2 + u\,\Delta u\big)\,dA,
\]
where in each chart $dA$ denotes the oriented area form ($dx\wedge dy$).

\paragraph{Exactness and Stokes.}
By $(\ast)$, on each chart $\Omega = d(u\,\alpha)$. Hence $\Omega$ is \emph{locally exact}.
By the generalized Stokes theorem on manifolds without boundary, the integral of a globally defined $2$-form
that is locally exact is $0$:
\[
\int_X \Omega \;=\; 0.
\]
(One way to see this concretely: choose a finite conformal atlas $\{U_j\}$ with a partition of unity
$\{\rho_j\}$, and set $\Theta:=\sum_j \rho_j\,u\,\alpha_j$, where $\alpha_j$ is the above form in $U_j$.
A brief check shows $d\Theta=\Omega$, so $\int_X\Omega=\int_X d\Theta=\int_{\partial X}\Theta=0$.)

\paragraph{Conclude for harmonic $u$.}
If $u$ is harmonic (\(\Delta u=0\) in each conformal chart), then
\[
\Omega \;=\; (u_x^2+u_y^2)\,dA \;\;\ge 0 \quad\text{pointwise,}
\]
and
\[
0 \;=\; \int_X \Omega \;=\; \int_X (u_x^2+u_y^2)\,dA .
\]
Thus $u_x=u_y\equiv 0$ everywhere. Hence $du\equiv 0$, so $u$ is constant on each connected component; since
$X$ is connected, $u$ is constant.

\hfill$\square$

\bigskip

\noindent\textbf{Remarks.}
\begin{itemize}
	\item This proof uses only: (i) conformal local coordinates on a Riemann surface,
	(ii) the product rule for $d$, and (iii) the generalized Stokes theorem. No Hodge star, no
	$\partial,\bar\partial$.
	\item The same argument works on any compact oriented surface endowed with a conformal structure (equivalently, a Riemannian metric up to smooth positive scaling) because harmonicity and the $2$-form
	$(u_x^2+u_y^2)\,dA$ are conformally invariant as used above.
\end{itemize}


\newpage
\section{Complex tori as Riemann surfaces}
\begin{definition}
	A \emph{lattice} in $\C$ is a discrete rank-$2$ subgroup
	\[
	\Lambda = \Z \omega_1 \oplus \Z \omega_2, \qquad \omega_1,\omega_2 \in \C \text{ linearly independent over }\R.
	\]
	The associated \emph{complex torus} is the quotient
	\[
	E_\Lambda \coloneqq \C/\Lambda.
	\]
\end{definition}

\begin{proposition}
	The quotient $E_\Lambda$ carries a natural structure of a complex $1$-dimensional manifold (a compact Riemann surface). The projection $\pi:\C\to E_\Lambda$ is a holomorphic universal covering map.
\end{proposition}

\begin{proof}[Sketch]
	The action of $\Lambda$ on $\C$ by translations is properly discontinuous and free. Local charts descend from $\C$, and compactness follows from the parallelogram fundamental domain. The universal covering property is standard.
\end{proof}

\begin{remark}[Linear entire lifts]
	Let $F:E_{\Lambda}\to E_{\Lambda'}$ be holomorphic. Any lift $\widetilde F:\C\to\C$ with $\pi'\circ \widetilde F = F\circ \pi$ is entire and $\Lambda$--equivariant:
	\[
	\widetilde F(z+\lambda)=\widetilde F(z)+\lambda'_\lambda\qquad (\lambda\in\Lambda,\ \lambda'_\lambda\in \Lambda').
	\]
	By Liouville/Weierstrass, an entire function with at most linear growth is affine. Equivariance forces $\widetilde F(z)=az+b$ with $a\in\C$ and $a\Lambda\subseteq \Lambda'$. Replacing $F$ by a translate we may assume $b=0$.
\end{remark}

\begin{proposition}[Biholomorphisms of tori]
	Two complex tori $E_\Lambda$ and $E_{\Lambda'}$ are biholomorphic iff there exists $a\in\C^\times$ with $a\Lambda=\Lambda'$.
\end{proposition}

\begin{proof}
	If $F:E_{\Lambda}\to E_{\Lambda'}$ is biholomorphic, pick a lift $\widetilde F(z)=az$ with $a\neq 0$ and $a\Lambda\subseteq \Lambda'$. Applying the argument to $F^{-1}$ shows $a^{-1}\Lambda'\subseteq\Lambda$, hence equality.
\end{proof}

\section{The moduli of complex tori}
Choose a lattice basis $(\omega_1,\omega_2)$ and set $\tau=\omega_2/\omega_1$ with $\im\tau>0$. Scaling by $\omega_1$ identifies $\Lambda=\omega_1(\Z+\Z\tau)$, so any torus is $E_\tau \coloneqq \C/(\Z+\Z\tau)$ with $\tau\in\HH$.

\begin{proposition}
	$E_\tau \cong E_{\tau'}$ iff $\tau'=\dfrac{a\tau+b}{c\tau+d}$ for some $\begin{psmallmatrix}a&b\\ c&d\end{psmallmatrix}\in \mathrm{SL}_2(\Z)$.
\end{proposition}

\begin{proof}
	Changing the basis of the lattice by an integer matrix corresponds to the indicated fractional linear transformation of~$\tau$.
\end{proof}

Thus the moduli of complex tori is $\mathrm{SL}_2(\Z)\backslash \HH$, concretely represented by the classical fundamental domain
\[
\mathcal{D}=\bigl\{\tau\in\HH : |\re\tau|\le \tfrac12,\ |\tau|\ge 1 \bigr\},
\]
with suitable edge identifications.

\section{Elliptic curves as plane cubics}
\begin{definition}
	An \emph{elliptic curve} over $\C$ is a pair $(E, O)$ where $E$ is a smooth projective curve of genus $1$ and $O\in E(\C)$ is a distinguished point.
\end{definition}

\begin{theorem}[Plane cubic model]
	Every elliptic curve over $\C$ is isomorphic to a smooth plane cubic in Weierstrass form
	\[
	y^2 = 4x^3 - g_2 x - g_3,
	\]
	for some $g_2,g_3\in\C$ with discriminant $\Delta = g_2^3 - 27 g_3^2 \ne 0$.
\end{theorem}

\begin{proof}[Idea via the holomorphic implicit function theorem]
	Consider a cubic $F(x,y,z)=0$ in $\PP^2$ with a nonsingular point $O=[0:1:0]$. In an affine chart ($z=1$) the curve is locally given by $f(x,y)=0$ with suitable partial derivative nonvanishing at $O$, so by the complex IFT one solves locally for a holomorphic parameter. Global projectivity and genus computation yield a smooth genus-$1$ curve, which can be transformed to Weierstrass form by linear changes of variables.
\end{proof}

\section{Meromorphic functions and Riemann--Roch facts}
Let $M$ be a compact Riemann surface of genus $g$. Write $\mathbb{C}(M)$ for its field of meromorphic functions.

\begin{theorem}[Riemann--Roch (special cases)]
	If $f\in\mathbb{C}(M)$ is nonconstant, the sum of orders of its poles equals the sum of orders of its zeros. In particular, on $M$ any meromorphic differential $\omega$ has total order $2g-2$. Moreover, given a point $p\in M$,
	\begin{itemize}
		\item there exists $f$ with a pole at $p$ of order at most $g+1$;
		\item if $g\ge 2$, for all but finitely many $p$ (the Weierstrass points) there exists $f$ \emph{with a single pole at $p$} of exact order $g+1$.
	\end{itemize}
\end{theorem}

\begin{remark}
	For an elliptic curve ($g=1$), a nonzero holomorphic $1$-form has no zeros; its divisor has degree $0$. This matches the translation-invariant form on a complex torus.
\end{remark}

\section{The Weierstrass \texorpdfstring{$\wp$}{wp}-function}
Fix a lattice $\Lambda\subset\C$. The Weierstrass $\wp$-function is
\[
\wp(z) = \frac{1}{z^2} + \sum_{\omega\in\Lambda\setminus\{0\}}
\left(\frac{1}{(z-\omega)^2}-\frac{1}{\omega^2}\right),\qquad
\wp'(z) = -2\sum_{\omega\in\Lambda} \frac{1}{(z-\omega)^3}.
\]
It is even, $\wp(-z)=\wp(z)$, while $\wp'$ is odd.

\begin{proposition}[Periodic meromorphicity]
	$\wp$ and $\wp'$ are $\Lambda$-periodic meromorphic functions on $\C$; they descend to meromorphic functions on $E_\Lambda$. The function $\wp$ has a unique (double) pole at $z\equiv 0\ (\mathrm{mod}\ \Lambda)$ and no other poles.
\end{proposition}

\begin{lemma}[Residues]
	The differential $\wp(z)\,dz$ is a globally defined meromorphic $1$-form on $E_\Lambda$ with vanishing total residue; in particular $\Res_0(\wp\,dz)=0$.
\end{lemma}

\begin{proposition}[Differential equation]
	Set the Eisenstein series
	\[
	g_2 = 60\!\!\sum_{\omega\in\Lambda\setminus\{0\}} \frac{1}{\omega^4},
	\qquad
	g_3 = 140\!\!\sum_{\omega\in\Lambda\setminus\{0\}} \frac{1}{\omega^6}.
	\]
	Then
	\[
	\bigl(\wp'(z)\bigr)^2 = 4\bigl(\wp(z)\bigr)^3 - g_2\,\wp(z) - g_3,
	\]
	and the right-hand side has nonzero discriminant precisely when $E_\Lambda$ is nonsingular.
\end{proposition}

\begin{proof}[Idea]
	One shows that $\wp$ and $\wp'$ generate the field of meromorphic functions on $E_\Lambda$ with pole at $0$; their minimal algebraic relation must be cubic in $\wp$ and quadratic in $\wp'$. Coefficient identification via Laurent expansions near $0$ yields the stated equation with $g_2,g_3$ as above.
\end{proof}

\subsection{Half-periods and zeros of \texorpdfstring{$\wp'$}{wp'}}
Let the three \emph{half-periods} be
\[
\omega_1/2,\qquad \omega_2/2,\qquad (\omega_1+\omega_2)/2 \quad \text{mod }\Lambda.
\]
\begin{proposition}
	The zeros of $\wp'$ are exactly the three half-period classes and they are simple. Denote
	\[
	e_1=\wp\!\left(\tfrac{\omega_1}{2}\right),\quad
	e_2=\wp\!\left(\tfrac{\omega_2}{2}\right),\quad
	e_3=\wp\!\left(\tfrac{\omega_1+\omega_2}{2}\right),
	\]
	then $e_1+e_2+e_3=0$ and
	\[
	(\wp')^2 = 4(\wp-e_1)(\wp-e_2)(\wp-e_3).
	\]
\end{proposition}

\section{Uniformization: torus \texorpdfstring{$\longleftrightarrow$}{<->} cubic}
Define
\[
\Phi:\C/\Lambda \longrightarrow E \subset \PP^2,\qquad
z \longmapsto \bigl[X:Y:Z\bigr]=\bigl(1:\wp(z):\tfrac{1}{2}\wp'(z)\bigr),
\]
which in affine coordinates is $(x,y)=\bigl(\wp(z),\tfrac{1}{2}\wp'(z)\bigr)$.

\begin{theorem}
	$\Phi$ is a biholomorphism from the complex torus $E_\Lambda$ onto the smooth plane cubic
	\[
	y^2 = 4x^3 - g_2 x - g_3,
	\]
	sending $z\equiv 0$ to the point at infinity $O=[0:1:0]$. Thus every complex elliptic curve is analytically isomorphic to a complex torus.
\end{theorem}

\begin{proof}[Sketch]
	Local holomorphicity and nondegeneracy follow from the inverse function theorem away from the zeros of $\wp'$ and from the cubic relation. The map is $\Lambda$-periodic, hence well-defined on the quotient; it is bijective with holomorphic inverse given by an elliptic integral.
\end{proof}

\begin{remark}[Elliptic integrals]
	Writing $x=\wp(z)$ and $y=\tfrac{1}{2}\wp'(z)$,
	\[
	dz = \frac{dx}{\sqrt{4x^3 - g_2 x - g_3}}.
	\]
	Hence the inverse uniformization is given by an elliptic integral
	\[
	z = \int^x \frac{du}{\sqrt{4u^3 - g_2 u - g_3}},
	\]
	with branch cuts chosen to make the integral single-valued on $E$.
\end{remark}

\section{Automorphisms and special \texorpdfstring{$\tau$}{tau}}
The automorphism group of $E_\tau$ contains translations and the group of linear automorphisms preserving $\Lambda$. For generic $\tau$ this group is $\{\pm 1\}$; at special moduli (square and hexagonal lattices) it is larger:
\[
\tau=i\ (\text{square})\ \Rightarrow\ \Aut(E_\tau)\cong \mu_4,\qquad
\tau=e^{\pi i/3}\ (\text{hexagonal})\ \Rightarrow\ \Aut(E_\tau)\cong \mu_6.
\]
These correspond to extra symmetries of the cubic (e.g.\ $j=1728$ and $j=0$).

\section{Moduli and $j$-invariant (brief)}
The coarse moduli of complex elliptic curves is parametrized by the modular $j$-invariant
\[
j(\tau)=1728\,\frac{g_2(\tau)^3}{\Delta(\tau)},\qquad \Delta(\tau)=g_2(\tau)^3-27g_3(\tau)^2,
\]
constant on $\mathrm{SL}_2(\Z)$-orbits. The fundamental domain $\mathcal D$ gives a concrete set of representatives for isomorphism classes.

\section{Appendix: analytic ingredients}
\subsection*{Complex implicit function theorem (CIFT)}
Let $U\subset\C^2$ open and $F:U\to\C$ holomorphic. If $F(z_0,w_0)=0$ and $\frac{\partial F}{\partial w}(z_0,w_0)\neq 0$, then in a neighborhood of $z_0$ there is a holomorphic function $w=w(z)$ with $F(z,w(z))=0$. This produces holomorphic charts on plane curves and ensures nonsingularity when the gradient does not vanish.

\subsection*{Riemann–Roch snapshots}
For a divisor $D$ on a compact Riemann surface $M$,
\[
\ell(D)-\ell(K-D)=\deg D + 1 - g,\qquad K\ \text{canonical divisor}.
\]
Specializing to $g=1$ implies $\deg K=0$ and $\ell(D)=\deg D$ for effective $D$ of degree $\ge 1$, enabling the description of the function field of $E$ via poles at~$O$.

\bigskip
\noindent\textbf{Summary.}
\begin{itemize}
	\item Complex tori $\C/\Lambda$ are precisely (analytic) elliptic curves.
	\item The moduli is $\mathrm{SL}_2(\Z)\backslash\HH$, with fundamental domain $\mathcal D$.
	\item The Weierstrass functions $\wp,\wp'$ uniformize the cubic $y^2=4x^3-g_2x-g_3$; the zeros of $\wp'$ are the half-periods and yield the roots $e_1,e_2,e_3$.
	\item Elliptic integrals invert the uniformization map.
\end{itemize}




\newpage
\section{Elliptic Curves, Elliptic Integrals, and the Complex Torus}
% ============================================================

In this section, we describe how elliptic curves arise naturally from
the complex torus \(\mathbb{C}/\Lambda\), where \(\Lambda = \mathbb{Z} + \mathbb{Z}\tau\) for
\(\Im\tau>0\).
We show how meromorphic functions on \(\mathbb{C}/\Lambda\) can be expressed
using the Weierstrass \(\wp\)-function and its derivative,
how the Weierstrass equation defines an algebraic curve in \(\mathbb{CP}^2\),
and how the elliptic integral connects the complex torus and this projective curve.

% ------------------------------------------------------------
\subsection{The Weierstrass \(\wp\)-function and its differential equation}

Let \(\Lambda = \mathbb{Z} + \mathbb{Z}\tau\) be a lattice in \(\mathbb{C}\), with
\(\Im\tau>0\).
Define the Weierstrass \(\wp\)-function by the absolutely convergent series
\[
\wp(z;\Lambda)
=
\frac{1}{z^2}
+\!\!\sum_{\substack{\omega\in\Lambda\\ \omega\neq0}}
\left(
\frac{1}{(z-\omega)^2}-\frac{1}{\omega^2}
\right),
\qquad z\in\mathbb{C}.
\]
Then \(\wp\) is a \emph{doubly periodic meromorphic function} on \(\mathbb{C}\)
with periods \(1\) and \(\tau\),
and thus descends to a meromorphic function on the torus
\(\mathbb{C}/\Lambda\).

The derivative
\(\wp'(z) = \dfrac{d}{dz}\wp(z)\)
is also \(\Lambda\)-periodic but \emph{odd}, satisfying
\(\wp'(-z)=-\wp'(z)\), while \(\wp\) itself is even:
\(\wp(-z)=\wp(z)\).

\begin{theorem}[Weierstrass differential equation]
	The functions \(\wp(z)\) and \(\wp'(z)\) satisfy the cubic relation
	\[
	(\wp'(z))^2 = 4\wp(z)^3 - g_2\,\wp(z) - g_3,
	\]
	where the invariants
	\[
	g_2 = 60\sum_{\omega\in\Lambda\setminus\{0\}}\frac{1}{\omega^4}, \qquad
	g_3 = 140\sum_{\omega\in\Lambda\setminus\{0\}}\frac{1}{\omega^6}
	\]
	depend only on the lattice \(\Lambda\).
\end{theorem}

This polynomial relation shows that
the function field of the torus satisfies
\[
\mathcal{M}(\mathbb{C}/\Lambda) = \mathbb{C}(\wp,\wp').
\]
Thus the complex torus can be viewed as a compact Riemann surface
of genus one, whose field of meromorphic functions is generated
by \(\wp\) and \(\wp'\).

% ------------------------------------------------------------
\subsection{Zeros of \(\wp'\) and the half-periods}

For a lattice \(\Lambda = \mathbb{Z} + \mathbb{Z}\tau\), the zeros of \(\wp'\)
are exactly the \emph{half-periods}:
\[
\frac{1}{2},\quad \frac{\tau}{2},\quad \frac{1+\tau}{2},
\]
and these zeros are all simple.

\begin{proof}[Sketch]
	Since \(\wp'\) is odd and periodic,
	\(\wp'(1-z)=-\wp'(z)\),
	so evaluating at \(z=\tfrac{1}{2}\) gives \(\wp'(\tfrac{1}{2})=0\).
	Similarly, using \(\wp'(z+\tau)=-\wp'(z)\),
	we find \(\wp'(\tfrac{\tau}{2})=\wp'(\tfrac{1+\tau}{2})=0\).
\end{proof}

The corresponding values of \(\wp\) at these points,
\[
e_1 = \wp\!\left(\tfrac{1}{2}\right),\quad
e_2 = \wp\!\left(\tfrac{\tau}{2}\right),\quad
e_3 = \wp\!\left(\tfrac{1+\tau}{2}\right),
\]
are the three roots of the cubic polynomial
\(4x^3 - g_2 x - g_3 = 0\).

% ------------------------------------------------------------
\subsection{Elliptic integrals and inversion}

From the Weierstrass equation
\[
(\wp'(z))^2 = 4(\wp(z)-e_1)(\wp(z)-e_2)(\wp(z)-e_3),
\]
we may regard \(x=\wp(z)\) and \(y=\wp'(z)\),
so that
\[
\frac{dy}{dz} = \frac{d\wp'(z)}{dz} = 6\wp(z)^2 - \frac{1}{2}g_2.
\]
By the inverse function theorem, away from zeros of \(\wp'\),
we can invert \(\wp\) locally:
\[
z = \wp^{-1}(x)
=
\int_{\infty}^{x}
\frac{du}{\sqrt{4u^3 - g_2u - g_3}}.
\]
Thus the coordinate \(z\) on the torus is expressed as
an \emph{elliptic integral of the first kind}.
If \(a,b\in\mathbb{R}\) and \(e_1>e_2>e_3\in\mathbb{R}\),
restricting to real values yields the classical real elliptic integral
\[
\int_{a}^{b}\frac{du}{\sqrt{4u^3 - g_2u - g_3}}.
\]
This parallels the trigonometric case
\[
\int_{a}^{b}\frac{dx}{\sqrt{1-x^2}} = \sin^{-1}(b)-\sin^{-1}(a),
\]
with \(\sin(x)\) being a single-periodic function,
while \(\wp(z)\) is doubly periodic.

% ------------------------------------------------------------
\subsection{Embedding the torus into projective space}

Define a holomorphic map
\[
\begin{aligned}
	F: \mathbb{C} &\longrightarrow \mathbb{CP}^2,\\
	z &\longmapsto [1 : \wp(z) : \wp'(z)].
\end{aligned}
\]
Since \(\wp\) and \(\wp'\) are doubly periodic,
\(F\) descends to a well-defined holomorphic map
\[
F: \mathbb{C}/\Lambda \longrightarrow \mathbb{CP}^2.
\]
Using the Weierstrass equation,
we see that the image \(F(\mathbb{C}/\Lambda)\)
is contained in the projective cubic curve
\[
E = \{[Z_0:Z_1:Z_2]\in\mathbb{CP}^2 \mid
Z_2^2Z_0 = 4Z_1^3 - g_2Z_1Z_0^2 - g_3Z_0^3\}.
\]
This defines a nonsingular projective cubic curve,
called an \emph{elliptic curve}.

\begin{proposition}
	The map \(F:\mathbb{C}/\Lambda \to E\subset\mathbb{CP}^2\)
	is a biholomorphism.
\end{proposition}

\begin{proof}[Idea]
	\(F\) is well-defined and holomorphic.
	Locally, it is one-to-one except at lattice points where \(\wp'\) vanishes.
	The inverse is given by
	\[
	z = \int_{\infty}^{x} \frac{du}{\sqrt{4u^3 - g_2u - g_3}},
	\]
	which is holomorphic on \(E\) minus the point at infinity.
	Since both \(\mathbb{C}/\Lambda\) and \(E\) are compact Riemann surfaces of genus \(1\),
	\(F\) must be biholomorphic.
\end{proof}

Thus every complex torus \(\mathbb{C}/\Lambda\) can be realized as a nonsingular cubic
curve in projective space.

% ------------------------------------------------------------
\subsection{Elliptic curves as complex tori}

Conversely, every smooth projective cubic curve
\[
E = \{[Z_0:Z_1:Z_2]\in\mathbb{CP}^2 \mid
Z_2^2Z_0 = 4Z_1^3 - g_2Z_1Z_0^2 - g_3Z_0^3\}
\]
is a compact Riemann surface of genus \(1\).
Its holomorphic differential
\[
\omega = \frac{dX}{Y}, \qquad (X,Y)=\Big(\frac{Z_1}{Z_0},\frac{Z_2}{Z_0}\Big),
\]
is nowhere vanishing, and integrating over a basis of
\(H_1(E,\mathbb{Z})\) gives the two fundamental periods
\(\omega_1,\omega_2\).
The corresponding period lattice
\(\Lambda = \mathbb{Z}\omega_1 + \mathbb{Z}\omega_2\)
yields an isomorphism of Riemann surfaces
\[
E \simeq \mathbb{C}/\Lambda.
\]
Hence we have the equivalence:
\[
\boxed{
	\text{Elliptic curve} \quad \Longleftrightarrow \quad
	\text{Complex torus } \mathbb{C}/\Lambda.
}
\]

% ------------------------------------------------------------
\subsection{Moduli and lattice equivalence}

Two lattices \(\Lambda_1=\mathbb{Z}+\mathbb{Z}\tau_1\)
and \(\Lambda_2=\mathbb{Z}+\mathbb{Z}\tau_2\)
give rise to biholomorphic tori
\(\mathbb{C}/\Lambda_1 \cong \mathbb{C}/\Lambda_2\)
if and only if there exists a matrix
\(\begin{pmatrix} a & b \\ c & d \end{pmatrix}\in SL(2,\mathbb{Z})\)
such that
\[
\tau_2 = \frac{a\tau_1 + b}{c\tau_1 + d}.
\]
The moduli space of complex tori is therefore represented by
the fundamental domain of the modular group \(SL(2,\mathbb{Z})\)
acting on the upper half-plane
\[
\mathbb{H} = \{\tau\in\mathbb{C}\mid \Im\tau>0\}.
\]
\begin{theorem}[Moduli interpretation]
	The moduli space of complex tori (or elliptic curves over \(\mathbb{C}\))
	is given by
	\[
	\mathcal{M}_1(\mathbb{C})
	\simeq SL(2,\mathbb{Z})\backslash\mathbb{H}.
	\]
\end{theorem}

% ============================================================
\section{Galois Group and Riemann Surface}
% ============================================================

In this section we explain the Galois–theoretic viewpoint on holomorphic maps
between compact Riemann surfaces. The key dictionary identifies holomorphic
coverings with finite extensions of function fields and deck transformations with
field automorphisms. We end by recalling the uniformization that realizes every
compact Riemann surface as a quotient of a simply connected model.

% ------------------------------------------------------------
\subsection{ Function fields and pullback}

For a Riemann surface \(X\), let \(\mathcal{M}(X)\) denote its field of meromorphic functions.
A nonconstant holomorphic map \(\pi:Y\to X\) induces a field embedding
\[
\pi^{*}:\ \mathcal{M}(X)\hookrightarrow \mathcal{M}(Y),\qquad f\longmapsto f\circ\pi .
\]
We identify \(\mathcal{M}(X)\) with \(\pi^{*}\mathcal{M}(X)\subset\mathcal{M}(Y)\).

\paragraph{Symmetric functions.}
Assume first that \(\pi\) is an unramified covering of degree \(d\). For any evenly covered
neighborhood \(U\subset X\) with sheets \(V_{1},\dots,V_{d}\) and local inverses
\(\tau_{j}:U\to V_{j}\), and any \(f\in\mathcal{M}(Y)\), set \(f_{j}=f\circ\tau_{j}\in\mathcal{M}(U)\).
Then
\[
\prod_{j=1}^{d}(T-f_{j})=T^{d}+c_{1}T^{d-1}+\cdots+c_{d}\in\mathcal{M}(U)[T]
\]
has coefficients \(c_{i}\) that glue on \(X\) (they are elementary symmetric polynomials in the \(f_{j}\)).
If \(\pi\) is branched, the same \(c_{i}\) are defined on \(X\setminus\) (branch values) and extend
meromorphically across the branch values.

\begin{theorem}[Minimal polynomial]
	Let \(\pi:Y\to X\) have degree \(d\) and \(f\in\mathcal{M}(Y)\). If \(c_{1},\dots,c_{d}\in\mathcal{M}(X)\) are the
	signed symmetric functions attached to \(f\), then in \(\mathcal{M}(Y)\)
	\[
	f^{d}+(\pi^{*}c_{1})f^{\,d-1}+\cdots+(\pi^{*}c_{d-1})f+\pi^{*}c_{d}=0.
	\]
	Hence \([\mathcal{M}(Y):\mathcal{M}(X)]\le d\); in fact equality holds.
\end{theorem}

Thus \(\mathcal{M}(Y)/\mathcal{M}(X)\) is a finite algebraic extension whose degree equals \(\deg\pi\).

% ------------------------------------------------------------
\subsection{ Galois coverings and Galois extensions}

\begin{definition}[Galois/normal covering]
	A covering map \(p:Y\to X\) is \emph{Galois} if the deck transformation group
	\(\operatorname{Deck}(Y/X)\) acts transitively on each fiber (equivalently,
	\(p_{*}\pi_{1}(Y)\trianglelefteq \pi_{1}(X)\)).
	For a branched map \(F:Y\to X\), we call \(F\) Galois if the restriction over the
	complement of the branch values is a Galois covering.
\end{definition}

\begin{theorem}[Deck transformations vs.\ field automorphisms]
	Let \(K=\mathcal{M}(X)\) and \(L=\mathcal{M}(Y)\) for a finite holomorphic map \(\pi:Y\to X\).
	Composition with deck transformations gives a group isomorphism
	\[
	\operatorname{Deck}(Y/X)\ \xrightarrow{\ \sim\ }\ \operatorname{Aut}(L/K),\qquad
	\sigma\longmapsto (f\mapsto f\circ\sigma^{-1}).
	\]
	Consequently, \(\pi\) is Galois if and only if the field extension \(L/K\) is Galois.
\end{theorem}

\begin{example}[Hyperelliptic double cover]
	Let \(E:\ y^{2}=x^{3}-x\) and \(\pi:E\to\mathbb{P}^{1}\), \((x,y)\mapsto x\). Then
	\(\mathcal{M}(E)=\mathbb{C}(x)[y]/(y^{2}-(x^{3}-x))\) is quadratic over \(\mathbb{C}(x)\).
	The nontrivial deck transformation \((x,y)\mapsto(x,-y)\) corresponds to the unique
	nontrivial automorphism of \(\mathcal{M}(E)/\mathbb{C}(x)\). Hence \(\pi\) is a Galois (normal)
	cover of degree \(2\).
\end{example}

\begin{definition}[Function field of one variable]
	A \emph{function field in one variable over \(\mathbb{C}\)} is a finite extension of \(\mathbb{C}(z)\).
\end{definition}

\begin{proposition}[Equivalence of categories]
	The assignment
	\[
	X\longmapsto \mathcal{M}(X),\qquad
	(\pi:Y\to X)\longmapsto \pi^{*}:\mathcal{M}(X)\hookrightarrow \mathcal{M}(Y)
	\]
	defines an equivalence between the category of compact, connected Riemann surfaces
	(with holomorphic maps) and the category of function fields in one variable over
	\(\mathbb{C}\) (with field monomorphisms). Under this equivalence,
	\(\operatorname{Deck}(Y/X)\cong \operatorname{Aut}(\mathcal{M}(Y)/\mathcal{M}(X))\).
\end{proposition}

% ------------------------------------------------------------
\subsection{ Uniformization and the Galois picture}

\begin{theorem}[Simply connected models]
	Every simply connected Riemann surface is isomorphic to exactly one of
	\(\mathbb{P}^{1}\), \(\mathbb{C}\), or the unit disc \(\mathbb{D}\).
	Thus any Riemann surface \(X\) has a universal cover \(\widetilde{X}\) of this form and
	\(X\cong G\backslash \widetilde{X}\) for a discrete group \(G\) of automorphisms of \(\widetilde{X}\).
\end{theorem}

\begin{theorem}[Uniformization of compact Riemann surfaces]
	Let \(X\) be compact and connected.
	\begin{enumerate}
		\item If \(g(X)=0\), then \(X\cong \mathbb{P}^{1}\).
		\item If \(g(X)=1\), then \(X\cong \mathbb{C}/\Lambda\) for a full lattice \(\Lambda\subset\mathbb{C}\).
		\item If \(g(X)\ge 2\), then \(X\cong \Gamma\backslash\mathbb{H}\) for a torsion-free, discrete
		\(\Gamma\le \operatorname{PSL}_{2}(\mathbb{R})\).
	\end{enumerate}
\end{theorem}

\paragraph{\(\operatorname{PSL}_{2}(\mathbb{R})\) as isometries of \(\mathbb{H}\).}
Equip \(\mathbb{H}=\{x+iy\in\mathbb{C}\mid y>0\}\) with the hyperbolic metric
\[
ds^{2}=\frac{|dz|^{2}}{(\operatorname{Im}z)^{2}}=\frac{dx^{2}+dy^{2}}{y^{2}}.
\]
Then fractional linear transformations \(z\mapsto \dfrac{az+b}{cz+d}\) with
\(\begin{psmallmatrix}a&b\\c&d\end{psmallmatrix}\in \operatorname{PSL}_{2}(\mathbb{R})\)
act by orientation-preserving isometries and give the full isometry group.
Hence any compact surface of genus \(\ge2\) is realized as a quotient
\(\Gamma\backslash\mathbb{H}\), and its holomorphic maps correspond to inclusions of
Fuchsian groups \(\Gamma\), equivalently to inclusions of the associated function fields.

\medskip
\noindent
\textbf{Takeaway.}
Holomorphic coverings \(Y\to X\) are simultaneously:
\begin{center}
	\(\displaystyle
	\text{(topology)}\quad \text{deck transformation actions}\ \Longleftrightarrow\
	\text{(algebra)}\quad \operatorname{Aut}\big(\mathcal{M}(Y)/\mathcal{M}(X)\big)\ \Longleftrightarrow\
	\text{(geometry)}\quad \Gamma\subset\Gamma'\le\operatorname{Aut}(\widetilde{X}).
	\)
\end{center}
This Galois dictionary is the backbone for later discussion of Belyi maps and dessins.


\end{document}
