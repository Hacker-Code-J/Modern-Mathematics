%%%%%%%%%%%%%%%%%%%%%%%%%%%%%%%%%%%%%%%%%
% A beamer poster style for the University of Oxford. Atilim Gunes Baydin <gunes@robots.ox.ac.uk>, November 2016.
% Based on the I6pd2 style created by Thomas Deselaers an Philippe Dreuw.
%
% Dreuw & Deselaer's Poster
% LaTeX Template
% Version 1.0 (11/04/13)
%
% Created by:
% Philippe Dreuw and Thomas Deselaers
% http://www-i6.informatik.rwth-aachen.de/~dreuw/latexbeamerposter.php
%
% This template has been downloaded from:
% http://www.LaTeXTemplates.com
%
% License:
% CC BY-NC-SA 3.0 (http://creativecommons.org/licenses/by-nc-sa/3.0/)
%
%%%%%%%%%%%%%%%%%%%%%%%%%%%%%%%%%%%%%%%%%

%----------------------------------------------------------------------------------------
%   PACKAGES AND OTHER DOCUMENT CONFIGURATIONS
%----------------------------------------------------------------------------------------

\documentclass[final,hyperref={pdfpagelabels=false}]{beamer}

\usepackage[orientation=portrait,size=a0,scale=1.3]{beamerposter} % Use the beamerposter package for laying out the poster with a portrait orientation and an a0 paper size

\usetheme{Oxford}

\usepackage[utf8]{inputenc} % allow utf-8 input
\usepackage{blindtext}
\usepackage{amsmath,amsthm,amssymb,latexsym} % For including math equations, theorems, symbols, etc
\usepackage[document]{ragged2e}
\usepackage{times}\usefonttheme{professionalfonts}  % Uncomment to use Times as the main font
\usefonttheme[onlymath]{serif} % Uncomment to use a Serif font within math environments
%\boldmath % Use bold for everything within the math environment
\usepackage{booktabs} % Top and bottom rules for tables
\usepackage{microtype}

\usecaptiontemplate{\small\structure{\insertcaptionname~\insertcaptionnumber: }\insertcaption} % A fix for figure numbering

\newcommand{\shrink}{-15pt}

\def\imagetop#1{\vtop{\null\hbox{#1}}}

\let\oldbibliography\thebibliography
\renewcommand{\thebibliography}[1]{\oldbibliography{#1}
\setlength{\itemsep}{-10pt}}

\usepackage{tikz}
\usetikzlibrary{arrows.meta,positioning,calc}
\usepackage{amsmath,amssymb,mathtools}

% ---------- Macros ----------
\newcommand{\R}{\mathbb{R}}
\newcommand{\C}{\mathbb{C}}
\newcommand{\Z}{\mathbb{Z}}
\newcommand{\Om}{\Omega}
\newcommand{\cO}{\mathcal{O}}
\newcommand{\cM}{\mathcal{M}}
\newcommand{\cK}{\mathcal{K}}
\newcommand{\cF}{\mathcal{F}}
\newcommand{\dd}{\,\mathrm{d}}
\newcommand{\delbar}{\bar{\partial}}
\newcommand{\Pic}{\mathrm{Pic}}
\newcommand{\divv}{\mathrm{div}}
\newcommand{\im}{\mathrm{im}}


%----------------------------------------------------------------------------------------
%   TITLE SECTION 
%----------------------------------------------------------------------------------------
\title{\Huge Riemann-Roch Theorem and its Application\\ \LARGE\; - From de Rham to sheaf cohomology, Serre duality -} % Poster title
\author{\bfseries Ji, Yonghyeon}
\institute{Department of Cyber Security, University of Kookmin\\\vspace{4mm}
\texttt{hacker3740@kookmin.ac.kr}}

%----------------------------------------------------------------------------------------
%   FOOTER TEXT
%----------------------------------------------------------------------------------------
\newcommand{\leftfoot}{} % Left footer text
\newcommand{\rightfoot}{} % Right footer text


%----------------------------------------------------------------------------------------

\begin{document}
\addtobeamertemplate{block end}{}{\vspace*{2ex}} % White space under blocks

\begin{frame}[t] % The whole poster is enclosed in one beamer frame

\begin{columns}[t] % The whole poster consists of three major columns, each of which can be subdivided further with another \begin{columns} block - the [t] argument aligns each column's content to the top

\begin{column}{.01\textwidth}\end{column} % Empty spacer column
\begin{column}{.49\textwidth} % 1st column
\begin{block}{Preliminaries I: Vector calculus as a cochain complex}
	On \(\R^3\), the familiar operators form a complex:
	\[
	C^\infty \xrightarrow{\nabla} \Gamma(T^*) \xrightarrow{\nabla\times} \Gamma(\Lambda^2 T^*) \xrightarrow{\nabla\cdot} C^\infty,
	\] with $(\nabla\times)\circ\nabla=0,\;(\nabla\cdot)\circ(\nabla\times)=0$.
	The de Rham complex is \((\Om^\bullet(M),d)\): \[\boxed{
	0\to \Om^0(M)\xrightarrow{d}\Om^1(M)\xrightarrow{d}\Om^2(M)\xrightarrow{d}\cdots}
	\]
	Cohomology measures \underline{\emph{\color{red}global obstructions}} to solving \underline{\color{red}\(d\eta=\omega\)}.
\end{block}  

\begin{block}{Preliminaries II: Local-to-global glueing: Mayer--Vietoris}
	For \(M=U\cup V\), we uses a partition of unity to build the short exact sequence \[
	0\to \Om^k(M)\to \Om^k(U)\oplus \Om^k(V)\to \Om^k(U\cap V)\to 0,
	\]
	and then the long exact Mayer--Vietoris sequence in cohomology.
	
	\medskip
	
	This means that ``curl-free locally'' does not imply ``gradient globally'' when topology obstructs global potentials.
	
	---
	
	Given \(M=U\cup V\), we uses the short exact sequence of de Rham complexes
	\[
	0\to \Omega^k(M)\xrightarrow{\alpha^k}\Omega^k(U)\oplus\Omega^k(V)\xrightarrow{\beta^k}\Omega^k(U\cap V)\to 0
	\]
	with \(\beta^k(\omega_U,\omega_V)=\omega_U|_{U\cap V}-\omega_V|_{U\cap V}\),
	and surjectivity is built using a partition of unity \(\{\rho_U,\rho_V\}\).
	
	\textbf{Why this belongs on a Riemann--Roch:}
	\begin{itemize}
		\item[$\bullet$] Fine sheaves (built from partitions of unity) let we compute sheaf cohomology via resolutions.
		\item[$\bullet$] This is the conceptual bridge from \texttt{grad/curl/div} (de Rham) to \(\bar\partial\)- and sheaf-cohomological formulas that culminate in Riemann--Roch.
	\end{itemize}
\end{block}

\begin{block}{Preliminaries III: From de Rham to Complex Geometry}
	
	On a compact Riemann surface \(M\), complex structure refines \(d=\partial+\bar{\partial}\).
	Holomorphic data live in sheaves:
	\[
	\cO \ (\text{holomorphic functions}), \qquad \Omega \ (\text{holomorphic 1-forms}).
	\]
	Divisors \(D=\sum_p n_p p\) encode zeros/poles, and determine a line bundle \(\cO(D)\): \[
	\cO(D)(V)=\{ TBA \}
	\]
\end{block}

\begin{block}{Core I: Riemann--Roch (Curves)}
	Let \(M\) be a compact Riemann surface of genus \(g\), and \(D\) a divisor. Define
	\[
	\ell(D)=\dim H^0(M,\cO(D)),\qquad i(D)=\dim H^0(M,\Omega(-D)).
	\]
	\textbf{Riemann--Roch:}
	\[
	\boxed{\ \ell(D)-i(D)=1-g+\deg D\ } \qquad
	\text{equivalently}\quad i(D)=\ell(K-D).
	\]
\end{block}

\begin{block}{Core II: Cohomological mechanism: Euler characteristic + Serre duality}
	We derives Riemann--Roch from the Euler characteristic of \(\cO(D)\):\[
	\chi(\cO(D)):=\dim H^0(M,\cO(D))-\dim H^1(M,\cO(D))=\ell(D)-i(D),
	\]
	using Serre duality to identify \(H^1(M,\cO(D))^\ast\simeq I(D)\).
	\vspace{0.4em}
	
	\textbf{Key principle:}
	\[
	\chi(\cO(D))=\chi(\cO_M)+\deg D,\qquad \chi(\cO_M)=1-g.
	\]
	So \(\chi\) is ``topology + degree,'' while \(\ell(K-D)\) is the global obstruction.
\end{block}

\begin{block}{Examples}
	\textbf{Sphere \(\mathbb{P}^1\) (\(g=0\), \(K=-2[\infty]\)).} For \(D=n[\infty]\),
	\(\ell(D)=n+1\) for \(n\ge -1\) (polynomials of degree \(\le n\)).
	
	\medskip
	\textbf{Elliptic curve (\(g=1\), \(\deg K=0\)).} Riemann--Roch becomes \(\ell(D)-i(D)=\deg D\), and for \(\deg D>0\) one has \(\ell(D)=\deg D\).
\end{block}

\begin{block}{Topology link: Gauss--Bonnet parallel}
	We emphasizes the analogy with Gauss--Bonnet:
	\[
	\int_M K_{\mathrm{Gauss}}\ \omega = 2\pi\chi(M)=2\pi(2-2g),
	\]
	and notes that \(\deg K=2g-2\) mirrors curvature/Euler characteristic.
\end{block}

\end{column}

\begin{column}{.02\textwidth}\end{column} % Empty spacer column
\begin{column}{.49\textwidth} % 2nd column
\begin{block}{Applications I: Algebraic--geometric (AG) codes from Riemann--Roch}
	Let \(X/\mathbb{F}_q\) be a smooth projective curve of genus \(g\). Choose:
	\begin{align*}
		D&=P_1+\cdots+P_n \ (\text{distinct rational points}),\\
		 G &\ (\text{divisor, }\mathrm{supp}(G)\cap\mathrm{supp}(D)=\emptyset).
	\end{align*}
	Define the evaluation code
	\begin{align*}
		C_L(D,G)&=\{(f(P_1),\ldots,f(P_n)):\ f\in L(G)\}\subseteq \mathbb{F}_q^n,\\ L(G)&=H^0(X,\cO(G)).
	\end{align*}
	\textbf{Dimension:} \(k=\dim C_L(D,G)=\ell(G)-\ell(G-D)\), and for \(\deg G\) sufficiently large,
	\[
	\ell(G)=\deg G+1-g,
	\]
	as large-degree/embedding discussion.
	
	\medskip
	
	\textbf{Designed distance:} \(d\ge n-\deg G\). (Standard AG-code bound.)
\end{block}
\begin{block}{Example for Applications I: Classic McElice and FALOMA}
\end{block}

	\begin{block}{Applications II: ``Surface direction'' via Riemann--Roch for surfaces}
	We states the surface analogue (for a smooth projective surface \(X\) and divisor \(D\)):
	\[
	\chi(\cO_X(D))=\chi(\cO_X)+\frac12\bigl(D\cdot D - D\cdot K_X\bigr),
	\]
	where \(\cdot\) is the intersection pairing and \(K_X\) is the canonical divisor.
	
	\vspace{0.4em}
	
%	\textbf{Coding-theory bridge (template):}
	\begin{itemize}
		\item \emph{AG surface codes:} evaluate \(H^0(X,\cO_X(D))\) on rational points/curves; use the above \(\chi\) plus vanishing theorems to estimate \(\dim H^0\).
		\item \emph{Topological/surface codes (quantum):} homological dimension depends on \(H_1\) (genus); the same genus parameter appears in curve Riemann--Roch and in the obstruction term.
	\end{itemize}
\end{block}

\begin{block}{The pipeline: Calculus (grad/curl/div) \(\to\) Riemann-Roch}
	\centering
	\begin{tikzpicture}[
		box/.style={draw,rounded corners,fill=black!5,inner sep=6pt,align=center},
		arr/.style={-Latex,thick}
		]
		\node[box] (vr) {\textbf{Vector calculus}\\[-0.1em]\(\nabla,\nabla\times,\nabla\cdot\)};
		\node[box, right=10mm of vr] (dr) {\textbf{de Rham}\\[-0.1em]\((\Om^\bullet,d)\)};
		\node[box, right=10mm of dr] (mv) {\textbf{Mayer--Vietoris}\\[-0.1em]PoU glueing};
		\node[box, below=8mm of mv] (sh) {\textbf{Sheaves}\\[-0.1em]\(\cO(D),\Omega\)};
		\node[box, left=10mm of sh] (sd) {\textbf{Serre duality}\\[-0.1em]\(H^1(\cO(D))^\ast\)};
		\node[box, left=10mm of sd] (rr) {\textbf{Riemann--Roch}\\[-0.1em]\(\ell(D)-i(D)=1-g+\deg D\)};
		
		\draw[arr] (vr) -- (dr);
		\draw[arr] (dr) -- (mv);
		\draw[arr] (mv) -- (sh);
		\draw[arr] (sh) -- (sd);
		\draw[arr] (sd) -- (rr);
	\end{tikzpicture}
\end{block}

\begin{block}{Vector calculus as a cochain complex}
On an oriented surface \(M\), smooth differential forms package familiar operators:

\medskip
\begin{center}
	\begin{tikzpicture}[scale=1.05, baseline=(current bounding box.center)]
		\node (o0) at (0,0) {$0$};
		\node (a0) at (3,0) {$\Omega^0(M)$};
		\node (a1) at (8,0) {$\Omega^1(M)$};
		\node (a2) at (13,0) {$\Omega^2(M)$};
		\node (o3) at (16,0) {$0$};
		
		\draw[->, line width=1.2pt] (o0) -- (a0) node[midway, above] {};
		\draw[->, line width=1.2pt] (a0) -- (a1) node[midway, above] {$d$};
		\draw[->, line width=1.2pt] (a1) -- (a2) node[midway, above] {$d$};
		\draw[->, line width=1.2pt] (a2) -- (o3) node[midway, above] {};
	\end{tikzpicture}
\end{center}

\begin{itemize}
	\item \(d:\Omega^0\to\Omega^1\) is the \emph{gradient} operator in disguise.
	\item \(d:\Omega^1\to\Omega^2\) is the \emph{curl} (scalar curl on a surface).
	\item The \emph{divergence} is the codifferential \(d^\ast\) (adjoint of \(d\) via a metric).
	\item The Laplacian \(\Delta = dd^\ast + d^\ast d\) yields \textbf{Hodge decomposition}:
	\[
	\Omega^1(M) \;=\; \underbrace{d\Omega^0(M)}_{\text{grad part}}
	\;\oplus\;
	\underbrace{d^\ast\Omega^2(M)}_{\text{div part}}
	\;\oplus\;
	\underbrace{\mathcal{H}^1(M)}_{\text{harmonic}}
	\]
	and \(\mathcal{H}^1(M) \cong H^1_{\mathrm{dR}}(M)\) (closed mod exact).
\end{itemize}

\vspace{0.5em}
\textbf{Index-counting slogan:} ``\#(global solutions) $-$ \#(global obstructions)'' is an Euler characteristic.

Riemann--Roch is precisely such an index formula for \(\cO(D)\).
\begin{center}
\end{center}

\end{block}

\end{column}



%%%%%%%%%%%%%%%%%%%%%%%%%%%%%%%%%%%%%%%%%%
%% Column 1
%%%%%%%%%%%%%%%%%%%%%%%%%%%%%%%%%%%%%%%%%%

%  \begin{column}{.3\textwidth} % 1st column
%    \vspace{\shrink}
%%    \begin{block}{Overview}
%%	    \begin{itemize}
%%	         \item Overview 1
%%	         \item Overview 2
%%	         \item Overview 3
%%	         \item Overview 4
%%	     \end{itemize}
%%	\end{block}
%%	\begin{block}{Another block}
%%	    Block text \cite{le-2016-inference-compilation}...
%%	\end{block}
%  \end{column} % End of the 1st column

%%%%%%%%%%%%%%%%%%%%%%%%%%%%%%%%%%%%%%%%%%%
%%% Column 2
%%%%%%%%%%%%%%%%%%%%%%%%%%%%%%%%%%%%%%%%%%%
%
%  \begin{column}{.02\textwidth}\end{column} % Empty spacer column
%
%  \begin{column}{.3\textwidth} % 2nd column
%    \vspace{\shrink}
%    \begin{block}{Method}
%      \textbf{Method details.} We formulate... $\mathcal S = \{s_1, s_2, s_3, s_4, s_5, s_6 \}$
%
%      Image example:
%%      \begin{center}
%%        \includegraphics[width=0.9\columnwidth]{ox_brand_cmyk_rev}
%%      \end{center}
%    \end{block}
%
%  \end{column} % End of the 2nd column
%
%%%%%%%%%%%%%%%%%%%%%%%%%%%%%%%%%%%%%%%%%%%
%%% Column 3
%%%%%%%%%%%%%%%%%%%%%%%%%%%%%%%%%%%%%%%%%%%
%
%  \begin{column}{.02\textwidth}\end{column} % Empty spacer column
%
%  \begin{column}{.3\textwidth} % 3rd column
%
%	\vspace{\shrink}
%    \begin{block}{Results}
%      Results...
%    \end{block}
%
%    \begin{block}{References}
%      \nocite{*} % Insert publications even if they are not cited in the poster
%      \linespread{0.928}\selectfont
%      \footnotesize{\bibliographystyle{unsrt}
%      \bibliography{oxford_poster}}
%    \end{block}
%
%  \end{column} % End of the 3rd column

  \begin{column}{.01\textwidth}\end{column} % Empty spacer column
\end{columns} % End of all the columns in the poster

\end{frame} % End of the enclosing frame

\end{document}