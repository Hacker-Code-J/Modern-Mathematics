% Linear Algebra and Group
\begin{note}[General Definition of Vector Space]
The operations $+$ and $\cdot$ must satisfy the following properties for all $\mathbf{u}, \mathbf{v} \in V$ and $\alpha, \beta \in \mathbb{F}$:
\begin{enumerate}
	\item \textbf{Associativity of Addition}:
	\begin{equation*}
		(\mathbf{u} + \mathbf{v}) + \mathbf{w} = \mathbf{u} + (\mathbf{v} + \mathbf{w})
	\end{equation*}
	\item \textbf{Commutativity of Addition}:
	\begin{equation*}
		\mathbf{u} + \mathbf{v} = \mathbf{v} + \mathbf{u}
	\end{equation*}
	\item \textbf{Existence of Additive Identity}:
	\begin{equation*}
		\vec{u}\in V\implies\exists \mathbf{0} \in V:\mathbf{u} + \mathbf{0} = \mathbf{u}
	\end{equation*}
	\item \textbf{Existence of Additive Inverse}:
	\begin{equation*}
		\vec{u}\in V\implies \exists -\mathbf{u} \in V : \mathbf{u} + (-\mathbf{u}) = \mathbf{0}
	\end{equation*}
	\item \textbf{Distributivity of Scalar Multiplication over Vector Addition}:
	\begin{equation*}
		\alpha \cdot (\mathbf{u} + \mathbf{v}) = (\alpha \cdot \mathbf{u}) + (\alpha \cdot \mathbf{v})
	\end{equation*}
	\item \textbf{Distributivity of Scalar Multiplication over Field Addition}:
	\begin{equation*}
		(\alpha + \beta) \cdot \mathbf{u} = (\alpha \cdot \mathbf{u}) + (\beta \cdot \mathbf{u})
	\end{equation*}
	\item \textbf{Compatibility of Scalar Multiplication with Field Multiplication}:
	\begin{equation*}
		(\alpha \beta) \cdot \mathbf{u} = \alpha \cdot (\beta \cdot \mathbf{u})
	\end{equation*}
	\item \textbf{Identity Element of Scalar Multiplication}:
	\begin{equation*}
		1 \cdot \mathbf{u} = \mathbf{u}
	\end{equation*}
\end{enumerate}
\end{note}

\defbox[Linear Operation]{\begin{definition}
Let $V$ be a set over a field $\mathbb{F}$. We define the following linear operations on $V$:
\begin{enumerate}
	\item An \textbf{addition operation} 
	\begin{align*}
		+ : V \times V &\rightarrow V \\
		(\mathbf{u}, \mathbf{v}) &\mapsto \mathbf{u} + \mathbf{v}
	\end{align*}
	on $V$ such that $(V, +)$ is an abelian group.
		
	\item A \textbf{scalar multiplication operation} 
	\begin{align*}
		\cdot : \mathbb{F} \times V &\rightarrow V \\
		(\alpha, \mathbf{u}) &\mapsto \alpha \cdot \mathbf{u}
	\end{align*}
	on $V$. Here $0\cdot\vec{u}:=\vec{0}$ and $1\cdot\vec{u}:=\vec{u}$, \ie, \begin{align*}
		\cdot : \F \times V &\rightarrow V & \cdot : \F \times V &\rightarrow V \\
		(0, \mathbf{u}) &\mapsto \mathbf{0} & (1, \mathbf{u}) &\mapsto \mathbf{u}
	\end{align*}
\end{enumerate}
\end{definition}}
\begin{remark}
Consider \[
\cdot:\F\to[V\to V].
\] Then 
\begin{center}\adjustbox{scale=.9}{
\begin{tikzpicture}
	% Draw the sets A and B
	\draw[thick] (-4,0) ellipse (2 and 3);
	\draw[thick] ( 4,0) ellipse (2 and 3);
	
	% Labels for sets
	\node at (-4, 3.25) {$\F$};
	\node at ( 4, 3.25) {$V^V$};
	
	% Draw the arrows representing the function
	\draw[-Stealth, thick] (-3.5, 3.25) -- (3.5,3.25) node[midway, above] {$\cdot$};
	\draw[fill] (-4,.75) circle (.05) node[left] {$0=e_{(\F,+)}$};
	\draw[fill] (-4,-.75) circle (.05) node[left] {$1=e_{(\F,\times)}$};
	
	\draw[fill] (4,.75) circle (.05) node[right] {$Z_V$ (Zero Transformation)};
	\draw[fill] (4,-.75) circle (.05) node[right] {$I_V$ (Identity Transformation)};
	
	\draw[-Stealth, thick] (-4, .75) -- (4, .75);
	\draw[-Stealth, thick] (-4, -.75) -- (4, -.75);
\end{tikzpicture}}
\end{center}
\end{remark}
\begin{remark}
	Let $V$ be a set over a field $\F$. Assume that, for $\vec{x},\vec{y}\in V$ and $\alpha,\beta\in F$, \[
	\alpha\cdot\vec{x} + \beta\cdot\vec{y}\in V.
	\] Then \[\begin{array}{rcll}
		\alpha=1=\beta &\implies &\vec{x} + \vec{y} \in V &\cdots\cdots\text{(Additivity)} \\
		\beta=0 &\implies &\alpha\cdot \vec{x} \in V &\cdots\cdots\text{(Homogeneity)}
	\end{array}
	\]
\end{remark}

\defbox[Vector Space]{\begin{definition}
A \textbf{vector space} \((V,+,\cdot)\), simply \( V \), over a field \( F \) is a set \( V \) together with two operations:
\begin{enumerate}
	\item \textbf{Vector Addition:} \[
	\fullfunction{+}{V\times V}{V}{(\vec{u},\vec{v})}{\vec{u}+\vec{v}},
	\] such that \( (V, +) \) forms an \underline{abelian group}.
	\item \textbf{Scalar Multiplication:} \[
	\fullfunction{\cdot}{F\times V}{V}{(a,\vec{u})}{a\cdot\vec{u}},
	\] such that \( (V, \cdot) \) satisfies the following properties:
	\begin{enumerate}
		\item \textbf{Distributivity of Scalar Multiplication with Respect to Vector Addition:} \[
		a \cdot (\mathbf{u} + \mathbf{v}) = (a \cdot \mathbf{u}) + (a \cdot \mathbf{v}).
		\]
		\item \textbf{Distributivity of Scalar Multiplication with Respect to Field Addition:} \[
		(a + b) \cdot \mathbf{v} = (a \cdot \mathbf{v}) + (b \cdot \mathbf{v}).
		\]
		\item \textbf{Associativity of Scalar Multiplication:} \[
		a \cdot (b \cdot \mathbf{v}) = (a \cdot b) \cdot \mathbf{v}.
		\]
		\item \textbf{Multiplicative Identity:} \[
		\vec{v}\in V\implies 1 \cdot \mathbf{v} = \mathbf{v},
		\] where 1 is the multiplicative identity in \( F \).
	\end{enumerate}
\end{enumerate}
\end{definition}}

\defbox[Linear Transformation]{\begin{definition}
Let \( V \) and \( W \) be vector spaces over the same field \( F \). A function \( T : V \to W \) is called a \textbf{linear transformation} (or linear map) if for all \( \mathbf{u}, \mathbf{v} \in V \) and all scalars \( a \in F \), the following two conditions are satisfied:

\begin{enumerate}
	\item \textbf{Additivity:} \[
	T(\mathbf{u} + \mathbf{v}) = T(\mathbf{u}) + T(\mathbf{v}).
	\]
	\item \textbf{Homogeneity of Scalar Multiplication:} \[
	T(a \cdot \mathbf{u}) = a \cdot T(\mathbf{u}).
	\]
\end{enumerate}
That is, \( T \) preserves the operations of vector addition and scalar multiplication.
\end{definition}}
\begin{remark}
Given that \( T : V \to W \) is a linear transformation, the following properties hold:
\begin{enumerate}
	\item \( T(\mathbf{0}_V) = \mathbf{0}_W \), where \( \mathbf{0}_V \) and \( \mathbf{0}_W \) are the zero vectors in \( V \) and \( W \), respectively.
	\item \( T\left(\sum_{i=1}^n a_i \mathbf{u}_i\right) = \sum_{i=1}^n a_i T(\mathbf{u}_i) \) for any finite set of vectors \( \mathbf{u}_1, \mathbf{u}_2, \ldots, \mathbf{u}_n \in V \) and scalars \( a_1, a_2, \ldots, a_n \in F \).
\end{enumerate}
\end{remark}
\begin{remark}
\ \begin{center}
\begin{tikzpicture}[auto, node distance=2cm, thick, >=Stealth]
	% Vector Addition
	\node (O1) at (0,0) {};
	\node (U) at (2,1) {};
	\node (V) at (1,2) {};
	\node (U+V) at ($(U) + (V)$) {};
	
	\node (TO1) at (0,-5) {};
	\node (TU) at (2,-4) {};
	\node (TV) at (1,-3) {};
	\node (TU+TV) at (3,-2) {};
	
	% Vectors for Addition
	\draw[->] (O1.center) -- (U.center) node[midway, below right] {$\mathbf{u}$};
	\draw[->] (O1.center) -- (V.center) node[midway, above left] {$\mathbf{v}$};
	\draw[->, dashed] (U.center) -- (U+V.center) node[midway, above right] {};
	\draw[->, dashed] (V.center) -- (U+V.center) node[midway, below left] {};
	\draw[->, very thick, magenta] (O1.center) -- (U+V.center) node[midway, above] {};
	
	\draw[->] (TO1.center) -- (TU.center) node[midway, below right] {$T(\mathbf{u})$};
	\draw[->] (TO1.center) -- (TV.center) node[midway, above left] {$T(\mathbf{v})$};
	\draw[->, dashed] (TU.center) -- (TU+TV.center) node[midway, above right] {};
	\draw[->, dashed] (TV.center) -- (TU+TV.center) node[midway, below left] {};
	\draw[->, very thick, magenta] (TO1.center) -- (TU+TV.center) node[midway, above] {};
	
	% Points for Addition
	\fill (O1) circle (2pt) node[below left] {$\mathbf{O}$};
	\fill (U) circle (2pt) node[below right] {};
	\fill (V) circle (2pt) node[above left] {};
	\fill (U+V) circle (2pt) node[above right] {\textcolor{magenta}{$\mathbf{u} + \mathbf{v}$}};
	
	\fill (TO1) circle (2pt) node[below left] {$\mathbf{O}$};
	\fill (TU) circle (2pt) node[below right] {};
	\fill (TV) circle (2pt) node[above left] {};
	\fill (TU+TV) circle (2pt) node[above right] {\textcolor{magenta}{$T(\mathbf{u}) + T(\mathbf{v})$}};
	
	% Scalar Multiplication
	\node (O2) at (7,0) {};
	\node (U2) at (9,1) {};
	\node (AU2) at ($(O2)!2!(U2)$) {};
	
	\node (TO2) at (7,-5) {};
	\node (TU2) at (9,-4) {};
	\node (TAU2) at (11,-3) {};
	
	% Vectors for Multiplication
	\draw[->] (O2.center) -- (U2.center) node[midway, below right] {$\mathbf{u}$};
	\draw[->, very thick, magenta] (O2.center) -- (AU2.center) node[midway, above] {};
	
	\draw[->] (TO2.center) -- (TU2.center) node[midway, below right] {$T(\mathbf{u})$};
	\draw[->, very thick, magenta] (TO2.center) -- (TAU2.center) node[midway, above] {};
	
	% Points for Multiplication
	\fill (O2) circle (2pt) node[below left] {$\mathbf{O}$};
	\fill (U2) circle (2pt) node[below right] {};
	\fill (AU2) circle (2pt) node[above] {\textcolor{magenta}{$a \cdot \mathbf{u}$}};
	
	\fill (TO2) circle (2pt) node[below left] {$\mathbf{O}$};
	\fill (TU2) circle (2pt) node[below right] {};
	\fill (TAU2) circle (2pt) node[above] {\textcolor{magenta}{$a \cdot T(\mathbf{u})$}};
	
	% Mapping
	\draw[draw=black] (-1, 3.5) rectangle (6,-.5);
	\draw[draw=black] (-1, -1.25) rectangle (6,-5.75);
	\draw[draw=black] (6.25, 3.5) rectangle (13,-.5);
	\draw[draw=black] (6.25, -1.25) rectangle (13,-5.75);
%	\draw[|-Stealth, thick, shorten <= 10pt, shorten >= 5pt] (3, 3) to node[midway, right] {$T(\vec{u}+\vec{v})$} (3,-2);
%	\draw[|-Stealth, thick, shorten <= 10pt, shorten >= 5pt] (11, 2) to node[midway, right] {$T(a\cdot\vec{u})$} (11,-3);
\end{tikzpicture}
%\vspace{36pt}
%\begin{tikzpicture}[auto, node distance=2cm, thick, >=Stealth]
%	% Vector Addition
%	\node (O1) at (0,0) {};
%	\node (U) at (2,1) {};
%	\node (V) at (1,2) {};
%	\node (U+V) at ($(U) + (V)$) {};
%	
%	% Vectors for Addition
%	\draw[->] (O1.center) -- (U.center) node[midway, below right] {$T(\mathbf{u})$};
%	\draw[->] (O1.center) -- (V.center) node[midway, above left] {$T(\mathbf{v})$};
%	\draw[->, dashed] (U.center) -- (U+V.center) node[midway, above right] {};
%	\draw[->, dashed] (V.center) -- (U+V.center) node[midway, below left] {};
%	\draw[->, very thick, magenta] (O1.center) -- (U+V.center) node[midway, above] {};
%	
%	% Points for Addition
%	\fill (O1) circle (2pt) node[below left] {$\mathbf{O}$};
%	\fill (U) circle (2pt) node[below right] {};
%	\fill (V) circle (2pt) node[above left] {};
%	\fill (U+V) circle (2pt) node[above right] {\textcolor{magenta}{$T(\mathbf{u}) + T(\mathbf{v})$}};
%	
%	% Scalar Multiplication
%	\node (O2) at (7,0) {};
%	\node (U2) at (9,1) {};
%	\node (AU2) at ($(O2)!2!(U2)$) {};
%	
%	% Vectors for Multiplication
%	\draw[->] (O2.center) -- (U2.center) node[midway, below right] {$T(\mathbf{u})$};
%	\draw[->, very thick, magenta] (O2.center) -- (AU2.center) node[midway, above] {};
%	
%	% Points for Multiplication
%	\fill (O2) circle (2pt) node[below left] {$\mathbf{O}$};
%	\fill (U2) circle (2pt) node[below right] {};
%	\fill (AU2) circle (2pt) node[above] {\textcolor{magenta}{$a \cdot T(\mathbf{u})$}};
%\end{tikzpicture}
\end{center}
\end{remark}

