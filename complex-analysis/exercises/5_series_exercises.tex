\subsection{Exercises}
\begin{enumerate}[\bf 1.]
	\item Show that the limit of a convergent complex sequence is unique by appealing to the corresponding result for a sequence of real numbers.
	\begin{proof}[\sol]
		We want to show that \begin{center}
		``If a complex sequence $\set{z_n}$ converges to both $L$ and $M$ in $\mathbb{C}$, then $L=M$.''
		\end{center}
		Write $z_n=x_n+iy_n$, $L=a+ib$, $M=c+id$ with $x_n,y_n,a,b,c,d\in\mathbb{R}$.
		Assume that \[
		z_n\to L\quad\text{and}\quad z_n\to M
		\] as $n\to\infty$. Taking real and imaginary parts, \[
		x_n=\Re z_n \to \Re L=a \quad\text{and}\quad x_n=\Re z_n \to \Re M=c,
		\] \[
		y_n=\Im z_n \to \Im L=b \quad\text{and}\quad y_n=\Im z_n \to \Im M=d.
		\]
		By the \emph{uniqueness of limits for real sequences}, these imply $a=c$ and $b=d$. Hence \[
		L=a+ib=c+id=M.
		\]
%		\begin{center}
%		\begin{tikzpicture}[>=Latex,scale=1]
%			\tikzset{
%				axis/.style={gray!40, line cap=round},
%				pt/.style={black, fill=red},
%				seq/.style={blue!70, fill=blue!70},
%				ballL/.style={draw=green!60!black, fill=green!20},
%				ballM/.style={draw=red!70!black,   fill=red!15},
%				note/.style={gray!60, font=\small},
%				title/.style={font=\small}
%			}
%			% ================= LEFT: ε-balls argument =================
%			\begin{scope}
%				% axes
%				\draw[axis] (-3.2,0) -- (3.2,0) node[below right] {$\Re z$};
%				\draw[axis] (0,-2.6) -- (0,2.6) node[left] {$\Im z$};
%				\node[title] at (0,2.9) {Uniqueness via disjoint $\varepsilon$-balls};
%				
%				% candidate limits
%				\coordinate (L) at (-1.0,0.6);
%				\coordinate (M) at ( 1.1,-0.2);
%				
%				% pick radii (schematic: half the distance)
%				\def\rL{0.9}
%				\def\rM{0.9}
%				
%				% balls and points
%				\draw[ballL] (L) circle (\rL);
%				\draw[ballM] (M) circle (\rM);
%				\fill[pt] (L) circle (2pt) node[above left] {$L$};
%				\fill[pt] (M) circle (2pt) node[below right] {$M$};
%				
%				% sample tail points near L
%				\fill[seq] ($(L)+(-0.25, 0.10)$) circle (1.8pt);
%				\fill[seq] ($(L)+(-0.10, 0.00)$) circle (1.8pt);
%				\fill[seq] ($(L)+(-0.05,-0.08)$) circle (1.8pt);
%				
%				% sample tail points near M
%				\fill[seq] ($(M)+( 0.20,-0.05)$) circle (1.8pt);
%				\fill[seq] ($(M)+( 0.05, 0.07)$) circle (1.8pt);
%				\fill[seq] ($(M)+(-0.10, 0.00)$) circle (1.8pt);
%				
%				\node[note,align=left] at (0.0,2.0)
%				{If $L\neq M$, take $\varepsilon=\tfrac12|L-M|$ so
%					$B(L,\varepsilon)\cap B(M,\varepsilon)=\varnothing$.\\
%					Since $z_n\to L$ and $z_n\to M$, the tail must lie in both balls — impossible.};
%			\end{scope}
%			% ================= RIGHT: component-wise argument =================
%			\begin{scope}[xshift=9.0cm]
%				\node[title] at (0,2.9) {Component-wise: real limits are unique};
%				
%				% real axis (top) and imaginary axis (bottom)
%				\draw[axis] (-3.2, 1.5) -- (3.2, 1.5) node[below right] {$\Re$};
%				\draw[axis] (-3.2,-1.5) -- (3.2,-1.5) node[below right] {$\Im$};
%				
%				% real parts: a = Re L, c = Re M
%				\coordinate (a) at (-0.6, 1.5);
%				\coordinate (c) at ( 0.6, 1.5);
%				\fill[pt] (a) circle (2pt) node[above] {$a=\Re L$};
%				\fill[pt] (c) circle (2pt) node[above] {$c=\Re M$};
%				
%				% imaginary parts: b = Im L, d = Im M
%				\coordinate (b) at (-0.4,-1.5);
%				\coordinate (d) at ( 0.9,-1.5);
%				\fill[pt] (b) circle (2pt) node[below] {$b=\Im L$};
%				\fill[pt] (d) circle (2pt) node[below] {$d=\Im M$};
%				
%				% schematic tails for x_n and y_n
%				\foreach \x in {-1.6,-1.2,-0.9,-0.75,-0.65} {
%					\fill[seq] (\x,1.5) circle (1.5pt);
%				}
%				\foreach \x in {1.6,1.2,0.9,0.75,0.65} {
%					\fill[seq] (\x,1.5) circle (1.5pt);
%				}
%				\foreach \y in {-1.7,-1.3,-1.0,-0.7,-0.5} {
%					\fill[seq] (\y,-1.5) circle (1.5pt);
%				}
%				\foreach \y in {1.7,1.3,1.0,0.7,0.5} {
%					\fill[seq] (\y,-1.5) circle (1.5pt);
%				}
%				
%				% notes
%				\node[note,align=left] at (0,0.5)
%				{$x_n=\Re z_n\to a\ \text{and}\ x_n\to c\ \Rightarrow\ a=c$ (uniqueness in $\mathbb R$).};
%				\node[note,align=left] at (0,-0.1)
%				{$y_n=\Im z_n\to b\ \text{and}\ y_n\to d\ \Rightarrow\ b=d$ (uniqueness in $\mathbb R$).};
%				
%				% conclusion
%				\node[note,draw,rounded corners=2pt,fill=gray!5] at (0,-2.3)
%				{$a=c,\ b=d \ \Rightarrow\ L=a+ib=c+id=M$.};
%			\end{scope}
%		\end{tikzpicture}
%		\end{center}
	\end{proof}
	\item Show that \[
	\sum_{n=1}^{\infty} z_n=S\implies \sum_{n=1}^{\infty} \overline{z_n}=\overline{S}.
	\]
	\begin{proof}[\sol]
		Let $s_N:=\sum_{n=1}^{N} z_n$ be the partial sums. By hypothesis $s_N\to S$ as $N\to\infty$.
		Consider the conjugated partial sums
		\[
		\overline{s_N}=\overline{\sum_{n=1}^{N} z_n}=\sum_{n=1}^{N}\overline{z_n},
		\]
		so $\{\overline{s_N}\}$ are the partial sums of $\sum_{n=1}^{\infty}\overline{z_n}$.
		Since complex conjugation is continuous (indeed, an isometry: $|\overline{w}-\overline{z}|=|w-z|$),
		we have $\overline{s_N}\to\overline{S}$. Therefore the series $\sum_{n=1}^{\infty}\overline{z_n}$ converges and
		\[
		\sum_{n=1}^{\infty}\overline{z_n}=\lim_{N\to\infty}\sum_{n=1}^{N}\overline{z_n}
		=\lim_{N\to\infty}\overline{s_N}=\overline{S}.
		\]
	\end{proof}
	\item Derive the Taylor series representation \[
	\frac{1}{1-z}=\sum_{n=0}^{\infty}\frac{(z-i)^n}{(1-i)^{\,n+1}},\qquad |z-i|<\sqrt{2}.
	\] 
	\begin{proof}[\sol]
	Note that \[
	\frac{1}{1-z}
	=\frac{1}{(1-i)-(z-i)}
	=\frac{1}{1-i}\cdot\frac{1}{1-\left(\frac{z-i}{1-i}\right)}.
	\] For \(\left|\dfrac{z-i}{1-i}\right|<1\) (i.e. \(|z-i|<|1-i|=\sqrt{2}\)), expand the geometric series:
	\[
	\frac{1}{1-w}=\sum_{n=0}^{\infty} w^{n}\quad (|w|<1),\qquad
	w=\frac{z-i}{1-i}.
	\] Hence \[
	\frac{1}{1-z}
	=\frac{1}{1-i}\sum_{n=0}^{\infty}\left(\frac{z-i}{1-i}\right)^n
	=\sum_{n=0}^{\infty}\frac{(z-i)^n}{(1-i)^{\,n+1}},
	\]
	which converges for \(|z-i|<\sqrt{2}\).
	\begin{center}
	\begin{tikzpicture}[>=Latex,scale=1]
		\tikzset{
			axis/.style={black, line cap=round},
			disk/.style={draw=blue!70, fill=blue!8, thick, opacity=.25, dashed},
			unit/.style={draw=green!60!black, fill=green!8, thick, opacity=.25, dashed},
			note/.style={gray!60, font=\small},
			title/.style={font=\small},
			point/.style={black, fill=black},
			maparrow/.style={-{Latex}, thick}
		}
		% ================= LEFT: z-plane =================
		\begin{scope}
			% axes
			\draw[axis] (-3.2,0) -- (3.2,0) node[right] {$\Re z$};
			\draw[axis] (0,-2.2) -- (0,3.0) node[above] {$\Im z$};
			\node[title] at (0,-2.5) {$z$-plane: expansion about $z=i$};
			% disk centered at i with radius sqrt(2)
			\def\R{1.4142}
			\draw[disk] (0,1) circle (\R);
			% key points: i and 1
			\fill[point] (0,1) circle (2.2pt) node[above left] {$i$};
			\fill[point] (1,0) circle (2.2pt) node[below right] {$1$};	
			% show that 1 is on the boundary: |1-i|=sqrt(2)
			\draw[gray!55,dashed] (0,1) -- (1,0);
			\node[note] at (0.7,0.7) {$|1-i|=\sqrt{2}$};	
			% a sample z inside the disk
			\coordinate (Z) at (0.6,1.6);
			\fill (Z) circle (2.0pt) node[above right] {$z$};
		\end{scope}
		% mapping arrow
		\draw[maparrow] (4.6,1.2) -- (6.2,1.2)
		node[midway,above] {$\displaystyle w=\frac{z-i}{\,1-i\,}$};
		
		% ================= RIGHT: w-plane =================
		\begin{scope}[xshift=9.2cm]
			% axes
			\draw[axis] (-2.7,0) -- (2.7,0) node[right] {$\Re w$};
			\draw[axis] (0,-2.2) -- (0,3) node[above] {$\Im w$};
			\node[title] at (0,-2.5) {$w$-plane: geometric series region};
			
			% unit disk |w|<1
			\draw[unit] (0,0) circle (1);
			
			% images of key points: i -> 0, 1 -> 1
			\fill[point] (0,0) circle (2.2pt) node[below] {$w( i)=0$};
			\fill[point] (1,0) circle (2.2pt) node[above right] {$w(1)=1$};
			
			% image of sample z
			% For Z=(0.6,1.6): z-i=(0.6,0.6), 1-i=(1,-1) so w=(0.6+0.6i)/(1-i)=0.6i
			\coordinate (W) at (0,0.6);
			\fill (W) circle (2.0pt) node[above right] {$w$};
			
%			% formula box
%			\node[note,draw,rounded corners=2pt,fill=gray!5,align=left] at (0,-2.0)
%			{$\displaystyle \frac{1}{1-z}
%				=\frac{1}{(1-i)-(z-i)}
%				=\frac{1}{1-i}\cdot\frac{1}{1-\left(\frac{z-i}{1-i}\right)}$\\[4pt]
%				If $|w|=\left|\dfrac{z-i}{1-i}\right|<1$:
%				$\displaystyle\ \frac{1}{1-w}=\sum_{n=0}^\infty w^n$\\[4pt]
%				$\displaystyle \Rightarrow\ \frac{1}{1-z}
%				=\sum_{n=0}^\infty \frac{(z-i)^n}{(1-i)^{\,n+1}},\quad |z-i|<\sqrt{2}.$};
		\end{scope}
	\end{tikzpicture}
	\end{center}
	\end{proof}
	\newpage
	\item Show that the two Laurent series in powers of $z$ that represent the function \[
	f(z)=\frac{1}{z(1+z^2)}
	\] are\[
	\sum_{n=0}^{\infty}(-1)^{n+1} z^{2n+1}+\frac{1}{z}\quad(0<|z|<1),
	\qquad
	\sum_{n=1}^{\infty}\frac{(-1)^{n+1}}{z^{2n+1}}\quad(1<|z|<\infty).
	\]
	\begin{proof}[\sol]
	\begin{enumerate}[(1)]
		\item (\(0<|z|<1\))\quad Since \[
		\frac{1}{1+z^2}=\frac{1}{1-(-z^2)}=\sum_{n=0}^\infty (-z^2)^n=\sum_{n=0}^\infty (-1)^n z^{2n}\qquad(|z|<1),
		\] we have \begin{align*}
			\frac{1}{z(1+z^2)}=\frac{1}{z}\sum_{n=0}^\infty (-1)^n z^{2n}
			&=\sum_{n=0}^\infty (-1)^n z^{2n-1}\\
			&=\frac{1}{z}+\left(-z\right)+z^3+(-z^5)+z^7+\cdots\\
			&=\frac{1}{z}+\sum_{n=0}^\infty (-1)^{n+1} z^{2n+1}.
		\end{align*}
		Therefore the Laurent series on \(0<|z|<1\) is $\displaystyle \sum_{n=0}^{\infty}(-1)^{n+1} z^{2n+1}+\frac{1}{z}$.
		\item (\(1<|z|<\infty\))\quad Since \[
		\frac{1}{1+z^2}=\frac{1}{z^2}\,\frac{1}{1+z^{-2}}=\frac{1}{z^2}\,\frac{1}{1-(-z^{-2})}
		=\frac{1}{z^2}\sum_{n=0}^\infty (-1)^n z^{-2n}\qquad(|z|>1),
		\] we obtain \begin{align*}		
			\frac{1}{z(1+z^2)}=\frac{1}{z}\cdot\frac{1}{z^2}\sum_{n=0}^\infty (-1)^n z^{-2n}
			&=\sum_{n=0}^\infty \frac{(-1)^n}{z^{2n+3}}\\
			&=\frac{1}{z^3}+\frac{-1}{z^5}+\frac{1}{z^7}+\frac{-1}{z^9}+\cdots \\
			&=\sum_{n=1}^\infty \frac{(-1)^{n+1}}{z^{2n+1}},
		\end{align*} Hence the Laurent series is $\displaystyle \sum_{n=1}^{\infty}\frac{(-1)^{n+1}}{z^{2n+1}}$ on \(1<|z|<\infty\).
	\end{enumerate} 
	\end{proof}
	\item Let $a\in\R$, where $-1<a<1$. Then the Laurent series representation $a/(z-a)$ is \[
	\frac{a}{z-a}=\sum_{n=1}^{\infty}\frac{a^{n}}{z^{n}},\qquad |a|<|z|<\infty.
	\]
	After writing $z=e^{i\theta}$ in the above equation, equate real parts and then imaginary parts on each side of the result to derive the summation formulas:
	\[
	\sum_{n=1}^{\infty} a^n \cos(n\theta)=\frac{a\cos\theta-a^2}{1-2a\cos\theta+a^2}\quad\text{and}\quad
	\sum_{n=1}^{\infty} a^n \sin(n\theta)=\frac{a\sin\theta}{1-2a\cos\theta+a^2}.
	\]
	\begin{proof}[\sol]
		For $|a|<|z|$, we know that
		\[
		\frac{a}{z-a}
		=\frac{a}{z}\,\frac{1}{1-a/z}
		=\frac{a}{z}\,\sum_{n=0}^\infty\left(\frac{a}{z}\right)^n
		=\sum_{n=0}^\infty\left(\frac{a}{z}\right)^{n+1}
		=\sum_{n=1}^{\infty}\frac{a^{n}}{z^{n}}.
		\] Set $z=e^{i\theta}$ (so $|a|<|z|=1$). Then
		\[
		\frac{a}{e^{i\theta}-a}=\sum_{n=1}^{\infty} a^n e^{-in\theta}
		=\sum_{n=1}^{\infty} a^n\bigl(\cos(n\theta)-i\sin(n\theta)\bigr).
		\]
		Note that \begin{align*}
			\frac{a}{e^{i\theta}-a}
			=\frac{e^{-i\theta}}{e^{-i\theta}}\cdot\frac{a}{e^{i\theta}-a}
			=\frac{a e^{-i\theta}}{1-a e^{-i\theta}}
			=\frac{a e^{-i\theta}(1-a e^{i\theta})}{(1-a e^{i\theta})(1-a e^{-i\theta})}
			&=\frac{a e^{-i\theta}(1-a e^{i\theta})}{1-a(e^{i\theta}+e^{-i\theta})+a^2e^{i\theta-i\theta}}\\
			&=\frac{a\,(e^{-i\theta}-a)}{1-2a\cos\theta+a^2}\\
			&=\frac{a\,(\cos\theta-i\sin\theta-a)}{1-2a\cos\theta+a^2}\\
			&=\frac{a(\cos\theta-a)-i\,a\sin\theta}{1-2a\cos\theta+a^2}.
		\end{align*} Thus, we obtain \[
		\sum_{n=1}^{\infty} a^n\bigl(\cos(n\theta)-i\sin(n\theta)\bigr)=\frac{a}{e^{i\theta}-a}=\frac{a(\cos\theta-a)-i\,a\sin\theta}{1-2a\cos\theta+a^2}.
		\] Therefore \[
		\boxed{\ \sum_{n=1}^{\infty} a^n \cos(n\theta)=\frac{a\cos\theta-a^2}{1-2a\cos\theta+a^2},\qquad
			\sum_{n=1}^{\infty} a^n \sin(n\theta)=\frac{a\sin\theta}{1-2a\cos\theta+a^2}\ },
		\]
		valid for $-1<a<1$ (indeed $1-2a\cos\theta+a^2=(1-a e^{i\theta})(\overline{1-a e^{i\theta}})=|1-ae^{i\theta}|^2>0$).
	\end{proof}
	\item With the aid of series, show that the function $f$ defined by means of the equations \[
	f(z)=\begin{cases}
		(\sin z)/z &: z\neq 0\\
		1 &: z = 0
	\end{cases}\] is entire. Use this result to establish the limit \[
	\lim_{z\to0}\frac{\sin z}{z}=1.
	\]
	\begin{proof}[\sol]
		The Maclaurin series of $\sin z$ (entire) is
		\[
		\sin z=\sum_{n=0}^{\infty}(-1)^n\frac{z^{2n+1}}{(2n+1)!}\,.
		\]
		For $z\neq 0$, divide by $z$:
		\[
		\frac{\sin z}{z}=\sum_{n=0}^{\infty}(-1)^n\frac{z^{2n}}{(2n+1)!}
		=1-\frac{z^{2}}{3!}+\frac{z^{4}}{5!}-\cdots .
		\]
		This is a power series with infinite radius of convergence, hence defines an entire function
		\[
		F(z):=\sum_{n=0}^{\infty}(-1)^n\frac{z^{2n}}{(2n+1)!}.
		\]
		Note that $F(0)=1$, and for $z\neq 0$ we have $F(z)=\sin z/z$. Therefore $f\equiv F$ on $\mathbb{C}$; in particular, $f$ is entire (the singularity at $0$ is removable). By continuity of $F$ at $0$, \[
		\lim_{z\to 0}\frac{\sin z}{z}=\lim_{z\to 0}F(z)=F(0)=1.
		\]
	\end{proof}
%	\begin{center}
%	\begin{tikzpicture}[>=Latex,scale=1]
%		\tikzset{
%			axis/.style={gray!40, line cap=round},
%			title/.style={font=\small},
%			note/.style={gray!60, font=\small},
%			entire/.style={draw=blue!60, line width=0.9pt},
%			hole/.style={draw=red!70, fill=white, line width=0.9pt},
%			filled/.style={draw=green!60!black, fill=green!60!black},
%			seriesbox/.style={draw, rounded corners=2pt, fill=gray!5, inner sep=4pt}
%		}
%		
%		% ================= LEFT: Complex plane & removable singularity =================
%		\begin{scope}
%			% axes
%			\draw[axis] (-3.4,0)--(3.4,0) node[below right] {$\Re z$};
%			\draw[axis] (0,-3.0)--(0,3.2) node[left] {$\Im z$};
%			\node[title] at (0,3.5) {$F(z)=\sum_{n=0}^{\infty}(-1)^n\dfrac{z^{2n}}{(2n+1)!}$ is entire; $F(0)=1$};
%			
%			% concentric disks to suggest infinite radius (entire)
%			\foreach \R in {0.6,1.2,1.8,2.4,3.0}{
%				\draw[entire] (0,0) circle (\R);
%			}
%			\node[note] at (2.5,2.4) {radius $=\infty$};
%			
%			% removable singularity at 0 for sin z / z (hole) and its filling by F(0)=1
%			\draw[hole] (0,0) circle (2.4pt); % the "hole" of sin z / z at 0
%			\node[note,anchor=west] at (0.15,0.15) {$\dfrac{\sin z}{z}$ undefined at $0$};
%			
%			% same point filled when defining F(0)=1
%			\fill[filled] (0,0) circle (1.7pt);
%			\node[note,anchor=west] at (0.15,-0.2) {$F(0)=1$ (removable)};
%			
%			% series box
%			\node[seriesbox,align=left] at (-2.1,-2.2)
%			{$\displaystyle \sin z=\sum_{n\ge0}(-1)^n\frac{z^{2n+1}}{(2n+1)!}$\\[4pt]
%				For $z\neq0$:\ $\displaystyle \frac{\sin z}{z}=\sum_{n\ge0}(-1)^n\frac{z^{2n}}{(2n+1)!}$\\[4pt]
%				Define $F(0)=1$ $\Rightarrow$ $F$ entire and $F(z)=\dfrac{\sin z}{z}$ for $z\neq0$.};
%		\end{scope}
%		
%		% =============== RIGHT: Real slice y = sin x / x with removable point ===============
%		\begin{scope}[xshift=9.5cm]
%			% axes for real plot
%			\draw[axis] (-5.4,0)--(5.4,0) node[below right] {$x=\Re z$};
%			\draw[axis] (0,-0.5)--(0,1.6) node[left] {$y$};
%			\node[title] at (0,1.9) {Real slice: $y=\dfrac{\sin x}{x}$ and the limit $x\to0$};
%			
%			% plot sin x / x on (-5.2,-0.2] and [0.2,5.2]
%			\draw[blue!70, line width=0.9pt, domain=-5.2:-0.2, samples=200, smooth]
%			plot(\x,{sin(deg(\x))/\x});
%			\draw[blue!70, line width=0.9pt, domain=0.2:5.2, samples=200, smooth]
%			plot(\x,{sin(deg(\x))/\x});
%			
%			% open circle at (0,1) for sin x / x
%			\draw[hole] (0,1) circle (2.4pt);
%			\node[note,anchor=west] at (0.15,1.1) {$\big(\sin x)/x$ undefined at $0$};
%			
%			% filled dot at (0,1) for F(0)=1
%			\fill[filled] (0,1) circle (1.7pt);
%			\node[note,anchor=west] at (0.15,0.8) {$F(0)=1$};
%			
%			% label of limit
%			\node[seriesbox] at (3.2,1.35) {$\displaystyle \lim_{x\to0}\frac{\sin x}{x}=1$};
%		\end{scope}
%	\end{tikzpicture}
%	\end{center}
\end{enumerate}

%\vspace{1em}
%\noindent\textbf{Source.} Adapted from the provided lecture note \emph{Complex Variables and Applications — Chapter 5: Series}. :contentReference[oaicite:0]{index=0}