\subsection{Exercises}
\begin{enumerate}[\bfseries 1.]
	\item Show that $f(z)=\exp(\overline{z})$ is not analytic anywhere.
	
	\noindent 
	\textit{(Hint: use the Cauchy--Riemann equations.)}
	\begin{proof}[\sol]
		(\textbf{Proof via Cauchy--Riemann equations}) Write \(z=x+iy\). Then \[
		f(z)=e^{\overline z}=e^{x-iy}=e^{x}\bigl(\cos y-i\sin y\bigr),
		\] so \[
		u(x,y)=e^{x}\cos y,\qquad v(x,y)=-e^{x}\sin y.
		\] Then \[
		u_x=e^{x}\cos y,\quad u_y=-e^{x}\sin y,\qquad
		v_x=-e^{x}\sin y,\quad v_y=-e^{x}\cos y.
		\] If \(f\) is complex differentiable at \((x,y)\), the Cauchy--Riemann equations would hold: \[
		u_x=v_y \quad\text{and}\quad u_y=-\,v_x.
		\] That is, \begin{align*}
			u_x=v_y&\implies e^x\cos y=-e^x\cos y&\implies \cos y=0, \\
			u_y=-v_x&\implies -e^x\sin y=e^x\sin y&\implies \sin y=0.
		\end{align*}
		There is no \(y\in\mathbb{R}\) with \(\cos y=0\) and \(\sin y=0\) simultaneously. Hence the Cauchy--Riemann equations fail at every point, so \(f\) is nowhere analytic.
		
	\medskip\noindent
	(\textbf{Proof via Wirtinger derivatives})
		Using \(\partial/\partial z=\tfrac12(\partial_x- i\,\partial_y)\) and
		\(\partial/\partial\overline z=\tfrac12(\partial_x+ i\,\partial_y)\),
		one checks directly that \[
		\frac{\partial f}{\partial z}=0,
		\qquad
		\frac{\partial f}{\partial \overline z}=e^{\overline z}\neq 0\ \text{ for all } z.
		\]
		A function is holomorphic iff \(\partial f/\partial \overline z\equiv 0\) on its domain. Since this is not the case, \(f\) is nowhere holomorphic.
	\end{proof}
	%\begin{remark}
	%		The map \(z\mapsto e^{\overline z}\) is \emph{anti-holomorphic}: the composition \(z\mapsto \overline z\) (conjugation) with the holomorphic \(w\mapsto e^{w}\). It is holomorphic only on the empty set.
	%	\end{remark}
	\newpage
	\item Let $f(z)=u(x,y)+iv(x,y)$ be analytic in a domain $D$. Show that \[
		U(x,y)=e^{u(x,y)}\cos v(x,y),\qquad V(x,y)=e^{u(x,y)}\sin v(x,y)
		\] are harmonic in $D$, and that $V$ is a harmonic conjugate of $U$.
	\begin{proof}[\sol]
		Since $f$ is analytic on $D$, the composition
		\[
		F(z):=e^{f(z)}=e^{u(x,y)}\big(\cos v(x,y)+i\sin v(x,y)\big)=U(x,y)+iV(x,y)
		\]
		is analytic on $D$ (composition of analytic maps). It follows that
		$U=\Re F$ and $V=\Im F$ are harmonic and satisfy the Cauchy--Riemann equations.
%		For completeness, we verify the CR equations directly and then deduce harmonicity.
		\begin{center}
		\begin{tikzpicture}[>=Latex,scale=1]
			% ================= z-plane =================
			\begin{scope}
				% axes
				\draw[->] (-2.6,0) -- (2.6,0) node[right] {$\Re z=x$};
				\draw[->] (0,-2.3) -- (0,2.6) node[above] {$\Im z=y$};
				\node at (0,-2.9) {$z$-plane (domain $D$)};
				
				% suggestive orthogonal families u=const (blue) and v=const (green) pulled back by f
				% (schematic curves—just to show orthogonality/conformality)
				\draw[blue!60,thick,domain=-2.3:2.3,samples=100,smooth] plot(\x,{0.5*sin(1.2*\x r)+0.4});
				\draw[blue!60,thick,domain=-2.3:2.3,samples=100,smooth] plot(\x,{-0.5*sin(1.2*\x r)-0.2});
				\draw[green!60,thick,domain=-2.3:2.3,samples=100,smooth] plot(\x,{0.6*cos(1.1*\x r)});
				\draw[green!60,thick,domain=-2.3:2.3,samples=100,smooth] plot(\x,{0.6*cos(1.1*\x r)+0.9});
				
				% sample point z0
				\fill[black] (0.8,0.7) circle (2pt) node[above right] {$z$};
			\end{scope}
			
			% arrow to (u,v)-plane
			\draw[->,thick] (5.0,1.2) -- (6.5,1.2) node[midway,above] {$f(z)=u+iv$};
			
			% ================= (u,v)-plane =================
			\begin{scope}[xshift=10cm]
				% axes
				\draw[->] (-2.6,0) -- (2.8,0) node[below right] {$u$};
				\draw[->] (0,-2.3) -- (0,2.6) node[left] {$v$};
				\node at (0,2.9) {$f$-plane};
				
				% horizontals u=const
				\foreach \uu/\lab in {-1.3/{$u=u_1$},0/{$u=0$},1.1/{$u=u_2$}}{
					\draw[blue!60,thick] (\uu,-2.2) -- (\uu,2.2);
					\node[blue!60,anchor=west] at (\uu+0.05,2.2) {\lab};
				}
				% verticals v=const
				\foreach \vv/\lab in {-1.2/{$v=v_1$},0/{$v=0$},1.0/{$v=v_2$}}{
					\draw[green!60,thick] (-2.4,\vv) -- (2.4,\vv);
					\node[green!60,anchor=west] at (2.45,\vv) {\lab};
				}
				
				% sample point (u0,v0) and its small orthonormal frame
				\fill (0.6,0.8) circle (2pt) node[above right] {$(u,v)$};
				% Jacobian of exp at (u,v) is E*[ [cos v, -sin v],[sin v, cos v] ]
				\node[gray!70,align=left,anchor=north west] at (-2.5,-2.1)
				{\small CR in $z\!\mapsto\!(u,v)$:\\[-1pt]
					\small $u_x=v_y,\ \ u_y=-\,v_x$ (conformal)};
			\end{scope}
			% arrow to w-plane
			\draw[->,thick] (5,1.2-7) -- (6.5,1.2-7) node[midway,above] {$F(z)=e^{f(z)}$};
			% ================= w-plane =================
			\begin{scope}[xshift=10cm, yshift=-7cm]
				% axes
				\draw[->] (-2.6,0) -- (2.8,0) node[below right] {$\Re w=U$};
				\draw[->] (0,-2.6) -- (0,2.8) node[left] {$\Im w=V$};
				\node at (0,3.1) {$w$-plane};
				
				% image of u-const: circles of radius e^u
				\draw[blue!60] (0,0) circle (0.27); % e^{-1.3} ~ 0.27
				\draw[blue!60] (0,0) circle (1.00); % e^{0}   = 1
				\draw[blue!60] (0,0) circle (3.00); % e^{~1.1}\approx 3.0 (schematic)
				\node[blue!60] at (2.4,0.25) {$|w|=e^{u}$};
				
				% image of v-const: rays with angle v
				\foreach \ang/\txt in {-1.2/{$v_1$},0/{$0$},1.0/{$v_2$}}{
					\draw[green!60,thick] (0,0) -- ({2.6*cos(\ang r)},{2.6*sin(\ang r)});
				}
				\node[green!60,anchor=west] at (1.9,1.5) {$\arg w=v$};
				
				% image point
				% choose u0=0.6, v0=0.8 -> radius e^{0.6}\approx 1.82, angle 0.8
				\fill[brown!80!black] ({1.82*cos(0.8 r)},{1.82*sin(0.8 r)}) circle (2.2pt)
				node[above right] {$w=U+iV=e^{u}(\cos v+i\sin v)$};
				
				% CR & harmonic box
				\node[draw,rounded corners=2pt,align=left,anchor=north west] at (-2.5,-2.2)
				{\small Since $F$ is analytic:\ $U=\Re F,\ V=\Im F$ satisfy CR\\[-2pt]
					\small $U_x=V_y,\ \ U_y=-V_x$ and are harmonic:\ $\Delta U=\Delta V=0$.};
			\end{scope}
		\end{tikzpicture}
		\end{center}
		Note that
		\begin{align*}
			U_x &= e^{u(x,y)}\,(u_x\cos v - v_x\sin v), &\qquad
			U_y &= e^{u(x,y)}\,(u_y\cos v - v_y\sin v),\\
			V_x &= e^{u(x,y)}\,(u_x\sin v + v_x\cos v), &\qquad
			V_y &= e^{u(x,y)}\,(u_y\sin v + v_y\cos v).
		\end{align*}
		Because $f$ is analytic, $u,v$ satisfy the CR equations $u_x=v_y$ and $u_y=-\,v_x$. Substituting,
		\begin{align*}
			U_x &= e^{u(x,y)}\,(v_y\cos v - v_x\sin v)=V_y,\\
			U_y &= e^{u(x,y)}\,(-v_x\cos v - v_y\sin v)= -\,V_x.
		\end{align*}
		Thus $U_x=V_y$ and $U_y=-V_x$, i.e.\ $V$ is a harmonic conjugate of $U$.
		
		To show harmonicity, differentiate the CR relations and use equality of mixed partials:
		\[
		U_{xx} = (V_y)_x = V_{yx},\qquad
		U_{yy} = (-V_x)_y = -V_{xy}.
		\]
		Hence $\Delta U:=U_{xx}+U_{yy}=V_{yx}-V_{xy}=0$. Similarly,
		\[
		V_{xx} = (-U_y)_x = -U_{yx},\qquad
		V_{yy} = (U_x)_y = U_{xy},
		\]
		so $\Delta V:=V_{xx}+V_{yy}=-U_{yx}+U_{xy}=0$. Therefore $U$ and $V$ are harmonic on $D$, and $V$ is a harmonic conjugate of $U$.
	\end{proof}
	\newpage
	\item Show that $f(z)=\Log(z-i)$ is analytic except on portion $x\leq 0$ of the line $y=1$ and that the function
		\[
		f(z)=\frac{\Log(z+4)}{z^2+i}
		\] is analytic everywhere except at the points $\pm{(1-i)}/{\sqrt2}$ and on the portion $x\leq -4$ of the real axis.
	\begin{proof}[\sol]
		Consider $\Log\; z=\ln|z|+i\Arg\; z$, the principal branch of the complex logarithm,
		with $\Arg\; z\in(-\pi,\pi)$, so that $\Log$ is analytic on \begin{align*}
			\mathbb{C}\setminus(-\infty,0] &= \C\setminus\set{z\in\C:\Re z\leq 0\land\Im z=0}\\
			&=\{\, z\in\mathbb{C} : \Re z>0\lor \Im z\neq 0\,\}.
		\end{align*} Then
		\begin{enumerate}[(1)]
			\item Since $\Log$ is analytic on $\mathbb{C}\setminus(-\infty,0]$ and the map $z\mapsto z-i$ is entire,
			the composition $z\mapsto \Log(z-i)$ is analytic precisely where $z-i\notin(-\infty,0]$.
			Equivalently, \begin{align*}
				z-i\in(-\infty,0] &\iff \Re(z-i)\le 0 \text{ and } \Im(z-i)=0 \\
				&\iff \Re(x+i(y-1))=x\le 0 \text{ and } \Im(x+i(y-1))=y-1=0.
			\end{align*}
			That is, $f(z):=\Log(z-i)$ is analytic on $\mathbb{C}\setminus\{\,x+iy:\; x\le 0,\ y=1\}$. 
	%i.e. it is analytic except on the portion $x\le0$ of the line $y=1$.
			\begin{center}
				\begin{tikzpicture}[>=Latex,scale=1.05]
					\draw[gray!20, dashed] (-4,-1) grid (4,3);
					% axes
					\draw[->] (-4.2,0) -- (4.2,0) node[right] {$\Re z=x$};
					\draw[->] (0,-1) -- (0,2.6) node[above] {$\Im z=y$};
					\node at (0,-2.2) {$f(z)=\Log(z-i)$};
					% a few reference gridlines for context
					\draw[gray!40] (-4.0, 1.0) -- (4.0, 1.0) node[right] {$y=1$};
					\draw[gray!40] ( 1.0,2.4) -- ( 1.0, -1) node[below] {$x=1$};
					% branch cut in z-plane: y = 1, x <= 0  (comes from w=z-i \in (-\infty,0])
					\draw[ultra thick,red!70] (-4.0,1.0) -- (0.0,1.0) node[midway,above] {$\;x\le 0,\ y=1$ (excluded)};
					\fill[red!70] (0.0,1.0) circle (2pt); % includes endpoint x=0
					% legend
					\node[align=left,anchor=west] at (-4.0,-2.9)
					{Analytic domain: $\ \C \setminus \{(x,y)\mid y=1,\ x\le 0\}$.};
				\end{tikzpicture}
			\end{center}
			\item The numerator $z\mapsto \Log(z+4)$ is analytic wherever $z+4\notin(-\infty,0]$, i.e., $z\notin\intoc{-\infty,-4}$. In other words, $\Log(z+4)$ is analytic on \[
			\C\setminus\intoc{-\infty,-4}=\C\setminus\set{z\in\C:\Re z\leq -4\land \Im z = 0}=\set{z\in\C:\Re z>-4\lor \Im z\neq 0 }
			\] $\C\setminus\intoc{-\infty}$ for
			$z\notin(-\infty,-4]$, which is the portion $x\le -4$ of the real axis.
			The denominator $z^2+i$ vanishes exactly at the zeros of $z^2=-i$, namely
			\[
			z=\pm(-i)^{1/2}=\pm e^{-i\pi/4}=\pm\frac{1-i}{\sqrt2}.
			\]
			Therefore $g$ is analytic on the domain where the numerator is analytic and the denominator is nonzero, i.e.
			\[
			\mathbb{C}\setminus\Big( (-\infty,-4]\ \cup\ \{\pm\tfrac{1-i}{\sqrt2}\}\Big),
			\]
			which is exactly the stated set.
			
			$g(z):=\dfrac{\Log(z+4)}{z^2+i}$ is analytic on
			\[
			\mathbb{C}\setminus\Big(\{\,x+iy:\; y=0,\ x\le -4\,\}\ \cup\ \{\pm\tfrac{1-i}{\sqrt2}\}\Big),
			\]
			i.e. everywhere except at the branch cut $x\le -4$ on the real axis and at the two points $\pm{(1-i)}/{\sqrt2}$.
			
			\begin{center}
				\begin{tikzpicture}[>=Latex,scale=1.05]
					\draw[gray!20, dashed] (-4,-4) grid (4,4);
					% axes
					\draw[->] (-6.2,0) -- (4.2,0) node[right] {$\Re z=x$};
					\draw[->] (0,-3.2) -- (0,3.2) node[above] {$\Im z=y$};
					\node at (0,-3.5) {$g(z)=\dfrac{\Log(z+4)}{z^2+i}$};
					% branch cut from Log(z+4): real axis x <= -4
					\draw[very thick,red!70] (-6.0,0.0) -- (-4.0,0.0) node[midway,above] {$\ (-\infty,-4]$ (excluded)};
					\fill[red!70] (-4.0,0.0) circle (2pt); % includes endpoint x=-4
					% poles from z^2+i=0: z = ±(1 - i)/√2  ≈ (±0.7071, ∓0.7071)
					\draw[red!80,very thick] ( 0.7071,-0.7071) +(0.12,0.12) -- +(-0.12,-0.12);
					\draw[red!80,very thick] ( 0.7071,-0.7071) +(-0.12,0.12) -- +(0.12,-0.12);
					\node[red!80,anchor=west] at (0.78,-0.70) {$\ \ \tfrac{1-i}{\sqrt2}$};
					\draw[red!80,very thick] (-0.7071, 0.7071) +(0.12,0.12) -- +(-0.12,-0.12);
					\draw[red!80,very thick] (-0.7071, 0.7071) +(-0.12,0.12) -- +(0.12,-0.12);
					\node[red!80,anchor=east] at (-0.78,0.70) {$\tfrac{-1+i}{\sqrt2}\ $};
					\draw[red!80,very thick, dotted] ( 0.7071,0) -- ( 0.7071,-0.7071);
					\draw[red!80,very thick, dotted] ( 0,-0.7071) -- ( 0.7071,-0.7071);
					\draw[red!80,very thick, dotted] ( -0.7071,0.7071) -- ( -0.7071,0);
					\draw[red!80,very thick, dotted] ( 0,0.7071) -- ( -0.7071,0.7071);
%					% legend
%					\node[align=left,anchor=west] at (-6.0,-2.7)
%					{Analytic domain: $\ \C\setminus\big((-\infty,-4]\cup\{\pm\tfrac{1-i}{\sqrt2}\}\big)$.};
				\end{tikzpicture}
			\end{center}
		\end{enumerate}
	\end{proof}
	\newpage
	\item Show that the function $\ln(x^2+y^2)$ is harmonic in every domain that does not contain the origin.
	\begin{proof}[\sol]
%		We need to show that $u(x,y) = \ln(x^2 + y^2)$
%		satisfies Laplace’s equation \[
%		u_{xx} + u_{yy} = 0
%		\] at every point $(x,y)\neq(0,0)$. 
		For $(x,y)\neq(0,0)$, we can differentiate: \begin{align*}
		u_x &= \frac{\partial}{\partial x}\ln(x^2 + y^2) = \frac{2x}{x^2 + y^2},\\
		u_y &= \frac{\partial}{\partial y}\ln(x^2 + y^2) = \frac{2y}{x^2 + y^2}.
		\end{align*} And then \begin{align*}
		u_{xx}
		&= \frac{2(x^2 + y^2) - 2x\cdot 2x}{(x^2 + y^2)^2}
		= \frac{2(x^2 + y^2) - 4x^2}{(x^2 + y^2)^2}
		= \frac{-2x^2 + 2y^2}{(x^2 + y^2)^2} \\
		u_{yy}
		&= \frac{2(x^2 + y^2) - 2y\cdot 2y}{(x^2 + y^2)^2}
		= \frac{2(x^2 + y^2) - 4y^2}{(x^2 + y^2)^2}
		= \frac{2x^2 - 2y^2}{(x^2 + y^2)^2}.
		\end{align*}	
		Now compute the Laplacian: \[
		u_{xx} + u_{yy}
		= \frac{-2x^2 + 2y^2}{(x^2 + y^2)^2}+
		\frac{2x^2 - 2y^2}{(x^2 + y^2)^2}
		= \frac{(-2x^2 + 2y^2) + (2x^2 - 2y^2)}{(x^2 + y^2)^2}
		= \frac{0}{(x^2 + y^2)^2}
		= 0
		\] for all $(x,y)\neq(0,0)$.
%		So $\ln(x^2 + y^2)$ is harmonic at every point where it is twice continuously differentiable --- that is, on any domain that does not contain the origin (since at ((0,0)) the function is not even defined, and our derivatives blow up).	
%		Therefore, **(\ln(x^2 + y^2)) is harmonic in every domain that does not contain the origin.**
		
		\medskip\noindent
		(\textbf{Proof via Wirtinger-operator})
		Let $z=x+iy$ and
		\[
		u(x,y)=\ln(x^2+y^2)=\ln(\abs{z}^2)=\ln(z\bar z).
		\]
		Recall the Wirtinger operators $
		\partial:=\tfrac12(\partial_x-i\,\partial_y)$ and
		$\bar\partial:=\tfrac12(\partial_x+i\,\partial_y),$
		so that the Laplacian satisfies
		\[
		\Delta=\partial_{xx}+\partial_{yy}=4\,\partial\bar\partial=4\,\bar\partial\partial.
		\]
		On $\mathbb{C}\setminus\{0\}$ the chain rule gives \begin{align*}
		\partial u
		&=\partial\big(\ln(z\bar z)\big)
		=\frac{1}{z\bar z}\,\partial(z\bar z)
		=\frac{1}{z\bar z}\,\bar z
		=\frac{1}{z},\\
		\bar\partial u
		&=\bar\partial\big(\ln(z\bar z)\big)
		=\frac{1}{z\bar z}\,\bar\partial(z\bar z)
		=\frac{1}{z\bar z}\,z
		=\frac{1}{\bar z}.
		\end{align*}
		Therefore,
		\[
		\Delta u
		=4\,\partial\bar\partial u
		=4\,\partial\!\left(\frac{1}{\bar z}\right)
		=0\qquad\text{on }\mathbb{C}\setminus\{0\}.
		\]
%		since $\partial(1/\bar z)=0$ away from the origin (equivalently, $\bar\partial(1/z)=0$).
%		Hence $u(x,y)=\ln(x^2+y^2)$ is harmonic on any domain not containing $0$.
%		\emph{Remark (distributional):} Globally, $\Delta\ln|z|^2=4\pi\,\delta_0$; the failure of harmonicity is a point mass at the origin.
	\end{proof}
	\newpage
	\item Show that neither $\sin\bar z$ nor $\cos\bar z$ is an analytic function of $z$ anywhere.
	
	\textit{(Hint: use the Cauchy--Riemann equation.)}
	\begin{proof}[\sol]
		(\textbf{Proof via CR-equations})
		Write \(z=x+iy\).
		\begin{enumerate}[(1)]
			\item Let \[
			f(z)=\sin(\overline z)=\sin(x-iy)=\sin(x)\cos(iy)-\cos x\sin(iy)=\sin x\cosh y - i\,\cos x\sinh y
			\] then \[
			u(x,y)=\sin x\cosh y,\qquad v(x,y)=-\cos x\sinh y.
			\]
			Compute the partials:
			\[
			u_x=\cos x\cosh y,\quad u_y=\sin x\sinh y,\qquad
			v_x=\sin x\sinh y,\quad v_y=-\cos x\cosh y.
			\]
			The Cauchy--Riemann (CR) equations \(u_x=v_y\) and \(u_y=-v_x\) become
			\[
			\cos x\cosh y=-\cos x\cosh y \quad\Longrightarrow\quad \cos x\cosh y=0,
			\]
			\[
			\sin x\sinh y=-\sin x\sinh y \quad\Longrightarrow\quad \sin x\sinh y=0.
			\]
			Since \(\cosh y\neq 0\) for all \(y\), the first forces \(\cos x=0\). Then the second gives either
			\(\sin x=0\) (impossible simultaneously with \(\cos x=0\)) or \(\sinh y=0\), i.e.\ \(y=0\).
			Hence the CR equations can hold only at isolated points with \(y=0\) and \(\cos x=0\) (i.e.\ \(x=\tfrac{\pi}{2}+k\pi\)).
			They \emph{cannot} hold on any open set. Therefore \(f\) is not analytic anywhere.
			\item \(g(z)=\cos(\overline z)=\cos(x-iy)=\cos x\cosh y + i\,\sin x\sinh y\).
			Thus
			\[
			u(x,y)=\cos x\cosh y,\qquad v(x,y)=\sin x\sinh y.
			\]
			Compute the partials:
			\[
			u_x=-\sin x\cosh y,\quad u_y=\cos x\sinh y,\qquad
			v_x=\cos x\sinh y,\quad v_y=\sin x\cosh y.
			\]
			The CR equations give
			\[
			u_x=v_y \ \Longrightarrow\ -\sin x\cosh y=\sin x\cosh y \ \Longrightarrow\ \sin x\cosh y=0,
			\]
			\[
			u_y=-v_x \ \Longrightarrow\ \cos x\sinh y=-\cos x\sinh y \ \Longrightarrow\ \cos x\sinh y=0.
			\]
			Again, since \(\cosh y\neq 0\), the first forces \(\sin x=0\); then the second forces either \(\cos x=0\) (incompatible)
			or \(\sinh y=0\), i.e.\ \(y=0\). Thus CR can hold only at isolated points with \(y=0\) and \(\sin x=0\) (i.e.\ \(x=k\pi\)),
			and not on any open set. Therefore \(g\) is not analytic anywhere.
		\end{enumerate}

		\medskip\noindent
		(\textbf{Proof via Wirtinger-operators})
		Recall the Wirtinger operators
		\[
		\partial=\tfrac12(\partial_x-i\,\partial_y),\qquad
		\bar\partial=\tfrac12(\partial_x+i\,\partial_y),
		\]
		and the criterion: a \(C^1\) function \(F\) is holomorphic on an open set iff \(\bar\partial F\equiv 0\) there.
		
		Let \(h(w)=\sin w\). Then \(f_1(z):=\sin(\overline z)=h(\overline z)\) satisfies
		\[
		\partial f_1(z)=0,\qquad \bar\partial f_1(z)=h'(\overline z)=\cos(\overline z).
		\]
		Similarly, with \(h(w)=\cos w\), \(f_2(z):=\cos(\overline z)\) satisfies
		\[
		\partial f_2(z)=0,\qquad \bar\partial f_2(z)=h'(\overline z)=-\sin(\overline z).
		\]
		In each case, \(\bar\partial f_j\) is \emph{not} identically zero on any open set (its zero set is discrete). Hence neither
		\(f_1\) nor \(f_2\) is holomorphic on any nonempty open set; i.e. they are nowhere analytic.
		
		\emph{Remark.} At isolated points where \(\cos(\overline z)=0\) (respectively \(\sin(\overline z)=0\)), the complex
		difference quotient may happen to have limit \(0\); however, analyticity requires \(\bar\partial f\equiv 0\) on a neighborhood, which fails here.
	\end{proof}
	\item Show that $\cosh^2 z-\sinh^2 z=1$ and $\sinh z+\cosh z=e^z$.
	\begin{proof}[\sol]
	Recall the exponential definitions (valid for all $z\in\mathbb{C}$):
	\[
	\cosh z=\frac{e^{z}+e^{-z}}{2},\qquad
	\sinh z=\frac{e^{z}-e^{-z}}{2}.
	\]
	
	\medskip
	\noindent\textbf{(1) $\cosh^2 z-\sinh^2 z=1$.}
	\[
	\cosh^2 z-\sinh^2 z
	=\left(\frac{e^{z}+e^{-z}}{2}\right)^{\!2}
	-\left(\frac{e^{z}-e^{-z}}{2}\right)^{\!2}
	=\frac{(e^{z}+e^{-z})^2-(e^{z}-e^{-z})^2}{4}.
	\]
	Expanding,
	\[
	(e^{z}+e^{-z})^2-(e^{z}-e^{-z})^2
	=e^{2z}+2+e^{-2z}-(e^{2z}-2+e^{-2z})=4,
	\]
	so $\cosh^2 z-\sinh^2 z=\frac{4}{4}=1$.
	
	\medskip
	\noindent\textbf{(2) $\sinh z+\cosh z=e^{z}$.}
	\[
	\sinh z+\cosh z
	=\frac{e^{z}-e^{-z}}{2}+\frac{e^{z}+e^{-z}}{2}
	=e^{z}.
	\]
	\end{proof}
\end{enumerate}
