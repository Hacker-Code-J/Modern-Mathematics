\begin{tcolorbox}
Use Cauchy's residue theorem to evaluate integral of each these functions around the circle $|z|=3$ in the positive sense: \[
\frac{e^{-z}}{z^2},\qquad
\frac{e^{-z}}{(z-1)^2},\qquad
z^2 \exp\left(\frac{1}{z}\right),\qquad
\frac{z+1}{z^2-2z}.
\] (Answers: $-2\pi i$, $-2\pi i/e$, $\pi i/3$, $2\pi i$.)
\end{tcolorbox}
\begin{proof}[\sol]
	All integrals are $\displaystyle \int_{|z|=3} (\cdot)\,d z$ with positive orientation.
	\begin{enumerate}[(1)]
		\item \(\displaystyle \int_{|z|=3} \frac{e^{-z}}{z^2}\,\d z\).\quad
		Let $f(z) = e^{-z}$. Then $f$ is entire (analytic everywhere), and the only
		singularity of the integrand inside $|z|=3$ is a pole of order $2$ at $z=0$.
		By Cauchy's integral formula for the first derivative, 
		%			if $f$ is analytic on and inside a simple closed contour $C$ and $z_0$ is interior to $C$, then
		\[
		f'(z_0) = \frac{1}{2\pi i} \int_{|z|=3} \frac{f(z)}{(z - z_0)^2}\,\d z.\implies \int_{|z|=3} \frac{f(z)}{(z - z_0)^2}\,\d z = 2\pi i\, f'(z_0).
		\] With $z_0 = 0$,
		\[
		\int_{|z|=3} \frac{e^{-z}}{z^2}\,\d z = 2\pi i\, f'(0)=2\pi i\cdot \frac{\d}{\d z} e^{-z}\Big|_{z=0}=2\pi i\cdot -e^{-z}\big|_{z=0}=2\pi i\cdot (-1)=-2\pi i.
		\]
		\item \(\displaystyle \int \frac{e^{-z}}{(z-1)^2}\,\d z\).\quad
		Let $f(z) = e^{-z}$, which is entire. The integrand has a pole of order
		$2$ at $z=1$. Since $|1|<3$, this singularity lies inside the circle $|z|=3$,
		and there are no other singularities inside the contour. By Cauchy's integral formula for the first derivative, with $z_0=1$,
		\[
		\int_{|z|=3} \frac{f(z)}{(z-1)^2}\,\d z = 2\pi i\cdot f'(1)=2\pi i\cdot (-e^{-z})\big|_{z=1}=2\pi i\cdot\left(-\frac{1}{e}\right)=\frac{-2\pi i}{e}.
		\]
		\item \(\displaystyle \int_{|z|=3} z^2 \exp\!\left(\frac{1}{z}\right)\,\d z\).\quad
		The only singularity of the integrand is at $z=0$, due to
		the factor $e^{1/z}$. This is an essential singularity at $z=0$, which lies
		inside the contour $|z|=3$. We use the residue theorem:
		%			If $f$ has isolated singularities $z_1,\dots,z_n$
		%			inside a simple closed contour $C$, and is analytic on $C$, then
		\[
		\int_C f(z)\,dz = 2\pi i \sum_{k=1}^n \operatorname{Res}(f, z_k).
		\]
		Here there is only one singularity at $z=0$, so
		\[
		\int_{|z|=3} z^2 \exp\!\left(\frac{1}{z}\right)\,\d z = 2\pi i \,\operatorname{Res}\!\left(z^2 e^{1/z}, 0\right).
		\]
		To find the residue, expand $\exp\left(\frac{1}{z}\right)$ in a Laurent series around $z=0$:
		\begin{align*}
			\exp\left(\frac{1}{z}\right) &= \sum_{n=0}^\infty \frac{1}{n!}\left(\frac{1}{z}\right)^n
			= 1 + \frac{1}{z} + \frac{1}{2!z^2} + \frac{1}{3!z^3} + \cdots,\\
			z^2 \exp\left(\frac{1}{z}\right)
			&= z^2 \left(1 + \frac{1}{z} + \frac{1}{2!z^2} + \frac{1}{3!z^3} + \cdots\right)\\
			&= z^2 + z + \frac{1}{2!} + \frac{1}{3!}\frac{1}{z} + \frac{1}{4!}\frac{1}{z^2} + \cdots.
		\end{align*} The residue at $z=0$ is $
		\operatorname{Res}\!\left(z^2 e^{1/z},0\right) = \frac{1}{3!} = \frac{1}{6}$. Therefore \[
		\int_{|z|=3} z^2 \exp\!\left(\frac{1}{z}\right)\,\d z = 2\pi i \cdot \frac{1}{6} = \frac{\pi i}{3}.
		\]
		\item \(\displaystyle \int_{|z|=3} \frac{z+1}{z^2-2z}\,\d z=\int \frac{z+1}{z(z-2)}\,\d z\).\quad
		The singularities are simple poles at $z=0$ and $z=2$. 
		%			Both of these points satisfy $|z|<3$, hence they lie inside the contour $|z|=3$.
		We can either compute residues directly or use partial fraction decomposition: \begin{align*}
			\frac{z+1}{z(z-2)} = \frac{A}{z} + \frac{B}{z-2} &\implies z+1 = A(z-2) + Bz\\
			&\implies z+1 = (A+B)\,z - 2A\\
			&\implies \begin{cases}
				A + B = 1,\\
				-2A = 1.
			\end{cases}\\
			&\implies A=-1/2\quad\text{and}\quad B=3/2.
		\end{align*} Thus \[
		\frac{z+1}{z^2 - 2z}
		= -\frac{1}{2z} + \frac{3}{2(z-2)}.
		\]
		Now the integral becomes
		\[
		\int \frac{z+1}{z(z-2)}\,\d z = \int_{|z|=3} \left(-\frac{1}{2z} + \frac{3}{2(z-2)}\right)\,dz
		= -\frac{1}{2} \int_{|z|=3} \frac{1}{z}\,dz
		+ \frac{3}{2} \int_{|z|=3} \frac{1}{z-2}\,dz.
		\] Note that \[
		\int_{|z|=3} \frac{1}{z}\,dz = 2\pi i
		\quad\text{and}\quad
		\int_{|z|=3} \frac{1}{z-2}\,dz = 2\pi i,
		\] since both $z=0$ and $z=2$ lie inside $|z|=3$. Therefore
		\[
		\int \frac{z+1}{z(z-2)}\,\d z = -\frac{1}{2} \cdot 2\pi i + \frac{3}{2} \cdot 2\pi i
		= -\pi i + 3\pi i = 2\pi i.
		\]
	\end{enumerate}
\end{proof}

\newpage
\begin{tcolorbox}
Show that the singular point of each of the following functions is a pole. \[
f(z)=\frac{1-\cosh z}{z^3},\quad g(z)=\frac{1-\exp(2z)}{z^4},\quad h(z)=\frac{\exp(2z)}{(z-1)^2}.
\] Determine the order $m$ of that pole and the corresponding residue $B$.

(Answers: $f(z)$: $m=1$, $B=-1/2$; $g(z)$: $m=3$, $B=-{4}/{3}$; $h(z)$: $m=2$, $B=2e^2$.)
\end{tcolorbox}
\begin{proof}[\sol]
	\ \begin{enumerate}[(1)]
		\item $\displaystyle f(z)=\frac{1-\cosh z}{z^3}$ at $z=0$.\quad
		Note that $\cosh z=1+\frac{1}{2!}\,z^2+\frac{1}{4!}\,z^4+\frac{1}{6!}\,z^6+\cdots$, and so
		\begin{align*}
			1-\cosh z&= -\frac{1}{2}\,z^2-\frac{1}{4!}\,z^4-\frac{1}{6!}\,z^6-\cdots,\\
			\frac{1-\cosh z}{z^3}&= -\frac{1}{2}\,\frac{1}{z}-\frac{1}{4!}\,z+\frac{1}{6!}\,z^3+\cdots.
		\end{align*}
		Hence the singularity is a \emph{simple pole} ($m=1$) with residue $
		B=\Res_{z=0} f = -\frac{1}{2}.$
		\item $\displaystyle g(z)=\frac{1-\exp(2z)}{z^4}$ at $z=0$.
		Note that \begin{align*}
			\exp(2z)&=1+2z+\frac{(2z)^2}{2!}+\frac{(2z)^3}{3!}+\frac{(2z)^4}{4!}+\cdots \\
			&=1+2z+2z^2+\frac{4}{3}z^3+\frac{2}{3}z^4+\cdots,\\
			1-\exp(2z)&= -\Big(2z+2z^2+\frac{4}{3}z^3+\frac{2}{3}z^4+\cdots\Big),\\
			\frac{1-\exp(2z)}{z^4}&= -\frac{2}{z^3}-\frac{2}{z^2}-\frac{4}{3}\,\frac{1}{z}-\frac{2}{3}+\cdots.
		\end{align*} Thus we have a pole of order $3$ ($m=3$) with residue $
		B=\Res_{z=0} g = -\frac{4}{3}.$
		\item $\displaystyle h(z)=\frac{\exp(2)}{(z-1)^2}$ at $z=1$.\quad
		Let $w:=z-1$, i.e., $z=1+w$. Then \begin{align*}
			\exp(2z)&=\exp(2(1+w))=\exp(2)\,\exp(2w)\\
			&=\exp(2)\left(1+2w+\frac{2^2}{2!}w^2+\frac{2^3}{3!}w^3+\cdots\right),\\
			\frac{\exp(2z)}{(z-1)^2}
			&=\frac{\exp(2)}{w^2}\left(1+2w+\frac{2^2}{2!}w^2+\cdots\right)
			=\exp(2)\left(\frac{1}{w^2}+\frac{2}{w}+\frac{2^2}{2!}+\cdots\right).
		\end{align*}
		Therefore the singularity is a pole of order $2$ ($m=2$) with residue $
		B=\Res_{z=1} h = 2\exp(2).$
	\end{enumerate}
\end{proof}

\begin{tcolorbox}
Show that \begin{align*}
	\Res_{z=-1}\frac{z^{1/4}}{z+1}&=\frac{1+i}{\sqrt2} &(\abs{z}>0,\; 0<\arg z<2\pi),\\
	\Res_{z=i}\frac{\Log\; z}{(z^2+1)^2}&=\frac{\pi+2i}{8},&\\
	\Res_{z=i}\frac{z^{1/2}}{(z^2+1)^2}&=\frac{1-i}{8\sqrt2} &(\abs{z}>0,\; 0<\arg z<2\pi).
\end{align*}
\end{tcolorbox}
	\begin{proof}[\sol]
	\ \begin{enumerate}[(1)]
		\item \textbf{(Residue of $f_1(z)=\dfrac{z^{1/4}}{z+1}$ at $z=-1$)}\quad We work with the branch \[
		|z|>0,\quad 0<\arg z<2\pi,
		\] so the branch cut is along the positive real axis, and \[
		z^{1/4} = \exp(\frac14 \Log z),\qquad
		\Log\; z = \ln|z| + i\arg z,\quad 0<\arg z<2\pi.
		\] 	At $z=-1$ we have $|-1|=1$ and $\arg(-1)=\pi$, hence \[
		\Log(-1) = \ln 1 + i\pi = i\pi,
		\] and therefore \[
		z^{1/4}\big|_{z=-1}=(-1)^{1/4} = \exp\left(i\frac{\pi}{4}\right)
		= \cos\frac{\pi}{4} + i\sin\frac{\pi}{4}
		= \frac{1+i}{\sqrt2}.
		\] 	The integrand $\displaystyle f_1(z) = \frac{z^{1/4}}{z+1}$ has a simple pole at $z=-1$, and $z^{1/4}$ is analytic at $z=-1$ on this branch. Consider \[
		g(z):=(z+1)\,f_1(z) = (z+1)\,\frac{z^{1/4}}{z+1} = z^{1/4}.
		\] Since $z^{1/4}$ is analytic at $z=-1$, the function $(z+1)f_1(z)$ is analytic
		at $z=-1$. Therefore it has a Taylor expansion around $z=-1$:
		\[
		(z+1)f_1(z) = z^{1/4}
		= a_0 + a_1(z+1) + a_2(z+1)^2 + \cdots,\quad\text{with}\quad a_n=\frac{g^{(n)}(-1)}{n!}\;(n=0,1,2,\dots),
		\]
		valid for $z$ near $-1$. Dividing both sides by $(z+1)$, we get a Laurent
		expansion for $f_1$ at $z=-1$:
		\[
		f_1(z)
		= \frac{a_0}{z+1} + a_1 + a_2(z+1) + \cdots.
		\]
		By definition, the residue of $f_1$ at $z=-1$ is $
		\Res_{z=-1} f_1(z) = a_0$. Thus \[
		\Res_{z=-1}\frac{z^{1/4}}{z+1}
		=a_0=\frac{g^{(0)}(-1)}{0!}
		= \lim_{z\to-1}(z+1)\frac{z^{1/4}}{z+1}
		= (-1)^{1/4}
		= \frac{1+i}{\sqrt2}.
		\]
		\item \textbf{(Residue of $f_2(z)=\dfrac{\Log z}{(z^2+1)^2}$ at $z=i$)}\quad 
		We factor $z^2+1 = (z-i)(z+i)$, so near $z=i$, \[
		(z^2+1)^2 = (z-i)^2(z+i)^2.
		\]
		Hence \[
		f_2(z) = \frac{\Log z}{(z-i)^2(z+i)^2}
		= \frac{g(z)}{(z-i)^2},\qquad
		g(z) := \frac{\Log z}{(z+i)^2}.
		\]
		The function $g$ is analytic at $z=i$. Thus $z=i$ is a double pole of $f$ of the form $f_2(z) = g(z)/(z-i)^2$ with $g$ analytic at $i$. Then \[
		\Res_{z=i} f_2(z) = \Res_{z=i} \frac{g(z)}{(z-i)^2} = \frac{g^{(1)}(i)}{1!}=g'(i).
		\] Compute $g'(i)$: \begin{align*}
			g'(i)=\frac{d}{dz}g(z)\Bigg|_{z=i}&=\frac{d}{dz}\left((\Log z)(z+i)^{-2}\right)\Bigg|_{z=i}\\
			&=\left[\frac{1}{z}(z+i)^{-2} - 2\Log z\,(z+i)^{-3}\right]_{z=i}\\
			&=\frac{1}{i}\cdot\frac{1}{-4}-2\Log(i)\cdot\frac{1}{-8i}\\
			&=\frac{-1}{4i}+\frac{1}{4i}\cdot(\ln|i|+i\arg(i))\\
			&=\frac{i}{4}+\frac{-i}{4}\cdot\left(0+\frac{\pi i}{2}\right)\\
			&=\frac{2i}{8}+\frac{\pi}{8}\\
			&=\frac{\pi+2i}{8}.
		\end{align*}
		\item \textbf{(Residue of $f_3(z)=\dfrac{z^{1/2}}{(z^2+1)^2}$ at $z=i$)}\quad
		As before, $
		(z^2+1)^2 = (z-i)^2(z+i)^2,$
		so \[
		f_3(z) = \frac{z^{1/2}}{(z-i)^2(z+i)^2}
		= \frac{h(z)}{(z-i)^2},\qquad
		h(z) := \frac{z^{1/2}}{(z+i)^2},
		\] and $h(z)$ is analytic at $z=i$. Thus $z=i$ is again a double pole of $f_3$, and $\Res_{z=i} f(z) = h'(i).$ Compute $h'(i)$: \begin{align*}
			h'(i)=\frac{d}{dz}h(z)\Bigg|_{z=i}&=\frac{d}{dz}\left(z^{1/2}(z+i)^{-2}\right)\Bigg|_{z=i}\\
			&=\left[ \frac{1}{2} z^{-1/2}(z+i)^{-2} - 2 z^{1/2}(z+i)^{-3}\right]_{z=i}\\
			&=\frac{1}{2}\cdot i^{-1/2}\cdot\frac{1}{-4}-2\cdot i^{1/2}\cdot\frac{1}{-8i}.
		\end{align*}
		We need the branch values of $z^{1/2}$ and $z^{-1/2}$ at $z=i$ for $0<\arg z<2\pi$: \begin{align*}
			i^{1/2} &= \exp(\frac12\Log i)
			= \exp(\frac12 (\ln|i|+i\arg(i)))
			= \exp(\frac12 \left(\frac{\pi i}{2}\right))
			= \exp(i\pi/4)
			= \cos\frac{\pi}{4} + i\sin\frac{\pi}{4}
			= \frac{1+i}{\sqrt2},\\
			i^{-1/2} &= \frac{1}{i^{1/2}}
			=\frac{1}{\frac{1+i}{\sqrt{2}}}
			= \frac{\sqrt2}{1+i}
			= \frac{\sqrt2(1-i)}{(1+i)(1-i)}
			= \frac{\sqrt2(1-i)}{2}
			= \frac{1-i}{\sqrt2}.
		\end{align*}
		Therefore \begin{align*}
			h'(i)&=\frac{1}{2}\cdot i^{-1/2}\cdot\frac{1}{-4}-2\cdot i^{1/2}\cdot\frac{1}{-8i}\\
			&=\frac{-1}{8}\left(\frac{1-i}{\sqrt{2}}\right)+\frac{-i}{4}\left(\frac{1+i}{\sqrt{2}}\right)\\
			&=\frac{-(1-i)+2(1-i)}{8\sqrt{2}} \\
			&=\frac{(1-i)}{8\sqrt{2}}
		\end{align*}
		Thus
		\[
		\Res_{z=i}\frac{z^{1/2}}{(z^2+1)^2}
		= h'(i)
		= \frac{1-i}{8\sqrt2}.
		\]
	\end{enumerate}
\end{proof}

\newpage
\begin{tcolorbox}
 Find the value of the integral \[
\int_{|z|=3}\frac{z^3\,e^{1/z}}{1+z^3}\,dz
\] taken CCW around the circle $\abs{z}=3$. 

(Answer: $2\pi i$.)
\end{tcolorbox}
\begin{proof}[\sol]
	Let \[
	f(z) = \frac{z^3 e^{1/z}}{1+z^3}.
	\] Set
	\[
	w = \frac{1}{z} \quad\Longrightarrow\quad z = \frac{1}{w},\quad dz = -\frac{1}{w^2}\,dw.
	\] The circle $|z| = 3$ corresponds to $|w| = 1/3$. As $z$ runs CCW, $w=1/z$ runs clockwise. Therefore,
	\begin{align*}
		\int_{|z|=3} f(z)\,dz
		&= \underbrace{\int_{\substack{|w|=1/3}} 	f\!\left(\frac{1}{w}\right)\left(-\frac{1}{w^2}\,dw\right)}_{\text{CW}}\\
		&= \underbrace{-\int_{\substack{|w|=1/3}} 	f\!\left(\frac{1}{w}\right)\left(-\frac{1}{w^2}\,dw\right)}_{\text{CCW}}\\
		&= \int_{|w|=1/3} \frac{f(1/w)}{w^2}\,dw.
	\end{align*}
	Since \[
	f\!\left(\frac{1}{w}\right)
	= \frac{(1/w)^3 e^{w}}{1 + (1/w)^3}
	= \frac{\frac{e^w}{w^3}}{\frac{w^3 + 1}{w^3}}
	= \frac{e^w}{w^3+1},
	\] we have \[
	\int_{|z|=3} f(z)\,dz=\int_{|w|=1/3} \left(\frac{1}{w^2}\cdot f\left(\frac{1}{w}\right)\right)\,dw
	=\int_{|w|=1/3} \frac{e^w}{w^2(w^3+1)}\,dw.
	\]
	The only singularity of
	\[
	h(w) = \frac{e^w}{w^2(w^3+1)}
	\]
	inside the circle $|w|=1/3$ is at $w=0$, since the roots of $w^3+1=0$ are
	\[
	w=-1,\quad w=e^{i\pi/3},\quad w=e^{-i\pi/3},
	\]
	all of which satisfy $|w|=1$. Hence, by the residue theorem,
	\[
	\int_{|z|=3} f(z)\,dz = 2\pi i \,\Res_{w=0} h(w).
	\]
	We compute the residue via series expansion. Write \[
	h(w) = \frac{1}{w^2}\cdot \frac{e^w}{1+w^3}.
	\] Since \begin{align*}
		\frac{1}{1+w^3} &= 1 - w^3 + w^6 - w^9 + \cdots \quad (|w|<1)\quad\text{and} \\
		e^w &= 1 + w + \frac{w^2}{2} + \frac{w^3}{6} + \cdots,
	\end{align*} we have
	\[
	\frac{e^w}{1+w^3}
	= (1 + w + \frac{w^2}{2} + \frac{w^3}{6} + \cdots)(1 - w^3 + w^6 - \cdots),
	\] and so \begin{align*}
		h(w) &= \frac{1}{w^2}\cdot \left(\left(1 + w + \frac{w^2}{2} + \frac{w^3}{6} + \cdots\right)(1 - w^3 + w^6 - \cdots)\right)\\
		&=\frac{1}{w^2}\left(\left(1 + w + \frac{w^2}{2} + \frac{w^3}{6} + \cdots\right)-
		\left(w^3 + w^4 + \frac{w^5}{2} + \frac{w^6}{6} + \cdots\right)+
		\left(w^6 + w^7 + \frac{w^8}{2} + \frac{w^9}{6} + \cdots\right)
		-\cdots\right)\\
		&=\left(\frac{1}{w^2} + {\color{red}\frac{1}{w}} + \frac{1}{2} + \frac{w}{6} + \cdots\right)-
		\left(w^1 + w^2 + \frac{w^3}{2} + \frac{w^4}{6} + \cdots\right)+
		\left(w^4 + w^5 + \frac{w^6}{2} + \frac{w^7}{6} + \cdots\right)
		-\cdots.
	\end{align*}
	Thus $\Res_{w=0} h(w) = 1$. By the residue theorem, \[
	\int_{|z|=3}\frac{z^3 e^{1/z}}{1+z^3}\,dz = 2\pi i \,\Res_{w=0} h(w) = 2\pi i \cdot 1 = 2\pi i.
	\]
\end{proof}