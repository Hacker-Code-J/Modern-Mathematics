\begin{tcolorbox}
Show that $f(z)=\exp(\overline{z})$ is not analytic anywhere. \textit{(Hint: use the Cauchy--Riemann equations.)}
\end{tcolorbox}
\begin{proof}[\sol]
	(\textbf{Proof via Cauchy--Riemann equations}) Write \(z=x+iy\). Then \[
	f(z)=e^{\overline z}=e^{x-iy}=e^{x}\bigl(\cos y-i\sin y\bigr),
	\] so \[
	u(x,y)=e^{x}\cos y,\qquad v(x,y)=-e^{x}\sin y.
	\] Then \[
	u_x=e^{x}\cos y,\quad u_y=-e^{x}\sin y,\qquad
	v_x=-e^{x}\sin y,\quad v_y=-e^{x}\cos y.
	\] If \(f\) is complex differentiable at \((x,y)\), the Cauchy--Riemann equations would hold: \[
	u_x=v_y \quad\text{and}\quad u_y=-\,v_x.
	\] That is, \begin{align*}
		u_x=v_y&\implies e^x\cos y=-e^x\cos y&\implies \cos y=0, \\
		u_y=-v_x&\implies -e^x\sin y=e^x\sin y&\implies \sin y=0.
	\end{align*}
	There is no \(y\in\mathbb{R}\) with \(\cos y=0\) and \(\sin y=0\) simultaneously. Hence the Cauchy--Riemann equations fail at every point, so \(f\) is nowhere analytic.
	
	\medskip\noindent
	(\textbf{Proof via Wirtinger derivatives})
	Using \(\partial/\partial z=\tfrac12(\partial_x- i\,\partial_y)\) and
	\(\partial/\partial\overline z=\tfrac12(\partial_x+ i\,\partial_y)\),
	one checks directly that \[
	\frac{\partial f}{\partial z}=0,
	\qquad
	\frac{\partial f}{\partial \overline z}=e^{\overline z}\neq 0\ \text{ for all } z.
	\]
	A function is holomorphic iff \(\partial f/\partial \overline z\equiv 0\) on its domain. Since this is not the case, \(f\) is nowhere holomorphic.
\end{proof}
\vfill
\begin{tcolorbox}
Show that $f(z)=\Log(z-i)$ is analytic except on portion $x\leq 0$ of the line $y=1$ and that the function
\[
f(z)=\frac{\Log(z+4)}{z^2+i}
\] is analytic everywhere except at the points $\pm{(1-i)}/{\sqrt2}$ and on the portion $x\leq -4$ of the real axis.
\end{tcolorbox}
\begin{proof}[\sol]
	Consider $\Log\; z=\ln|z|+i\Arg\; z$, the principal branch of the complex logarithm,
	with $\Arg\; z\in(-\pi,\pi)$, so that $\Log$ is analytic on \begin{align*}
		\mathbb{C}\setminus(-\infty,0] &= \C\setminus\set{z\in\C:\Re z\leq 0\land\Im z=0}\\
		&=\{\, z\in\mathbb{C} : \Re z>0\lor \Im z\neq 0\,\}.
	\end{align*} Then
	\begin{enumerate}[(1)]
		\item Since $\Log$ is analytic on $\mathbb{C}\setminus(-\infty,0]$ and the map $z\mapsto z-i$ is entire,
		the composition $z\mapsto \Log(z-i)$ is analytic precisely where $z-i\notin(-\infty,0]$.
		Equivalently, \begin{align*}
			z-i\in(-\infty,0] &\iff \Re(z-i)\le 0 \text{ and } \Im(z-i)=0 \\
			&\iff \Re(x+i(y-1))=x\le 0 \text{ and } \Im(x+i(y-1))=y-1=0.
		\end{align*}
		That is, $f(z):=\Log(z-i)$ is analytic on $\mathbb{C}\setminus\{\,x+iy:\; x\le 0,\ y=1\}$. 
		%i.e. it is analytic except on the portion $x\le0$ of the line $y=1$.
		\begin{center}
			\begin{tikzpicture}[>=Latex,scale=1.05]
				\draw[gray!20, dashed] (-4,-1) grid (4,3);
				% axes
				\draw[->] (-4.2,0) -- (4.2,0) node[right] {$\Re z=x$};
				\draw[->] (0,-1) -- (0,2.6) node[above] {$\Im z=y$};
				\node at (0,-2.2) {$f(z)=\Log(z-i)$};
				% a few reference gridlines for context
				\draw[gray!40] (-4.0, 1.0) -- (4.0, 1.0) node[right] {$y=1$};
				\draw[gray!40] ( 1.0,2.4) -- ( 1.0, -1) node[below] {$x=1$};
				% branch cut in z-plane: y = 1, x <= 0  (comes from w=z-i \in (-\infty,0])
				\draw[ultra thick,red!70] (-4.0,1.0) -- (0.0,1.0) node[midway,above] {$\;x\le 0,\ y=1$ (excluded)};
				\fill[red!70] (0.0,1.0) circle (2pt); % includes endpoint x=0
				% legend
				\node[align=left,anchor=west] at (-4.0,-2.9)
				{Analytic domain: $\ \C \setminus \{(x,y)\mid y=1,\ x\le 0\}$.};
			\end{tikzpicture}
		\end{center}
		\item The numerator $z\mapsto \Log(z+4)$ is analytic wherever $z+4\notin(-\infty,0]$, i.e., $z\notin\intoc{-\infty,-4}$. In other words, $\Log(z+4)$ is analytic on \[
		\C\setminus\intoc{-\infty,-4}=\C\setminus\set{z\in\C:\Re z\leq -4\land \Im z = 0}=\set{z\in\C:\Re z>-4\lor \Im z\neq 0 }
		\] $\C\setminus\intoc{-\infty}$ for
		$z\notin(-\infty,-4]$, which is the portion $x\le -4$ of the real axis.
		The denominator $z^2+i$ vanishes exactly at the zeros of $z^2=-i$, namely
		\[
		z=\pm(-i)^{1/2}=\pm e^{-i\pi/4}=\pm\frac{1-i}{\sqrt2}.
		\]
		Therefore $g$ is analytic on the domain where the numerator is analytic and the denominator is nonzero, i.e.
		\[
		\mathbb{C}\setminus\Big( (-\infty,-4]\ \cup\ \{\pm\tfrac{1-i}{\sqrt2}\}\Big),
		\]
		which is exactly the stated set.
		
		$g(z):=\dfrac{\Log(z+4)}{z^2+i}$ is analytic on
		\[
		\mathbb{C}\setminus\Big(\{\,x+iy:\; y=0,\ x\le -4\,\}\ \cup\ \{\pm\tfrac{1-i}{\sqrt2}\}\Big),
		\]
		i.e. everywhere except at the branch cut $x\le -4$ on the real axis and at the two points $\pm{(1-i)}/{\sqrt2}$.
		
		\begin{center}
			\begin{tikzpicture}[>=Latex,scale=1.05]
%				\draw[gray!20, dashed] (-4,-4) grid (4,4);
				% axes
				\draw[->] (-6.2,0) -- (4.2,0) node[right] {$\Re z=x$};
				\draw[->] (0,-1.2) -- (0,3.2) node[above] {$\Im z=y$};
%				\node at (0,-3.5) {$g(z)=\dfrac{\Log(z+4)}{z^2+i}$};
				% branch cut from Log(z+4): real axis x <= -4
				\draw[very thick,red!70] (-6.0,0.0) -- (-4.0,0.0) node[midway,above] {$\ (-\infty,-4]$ (excluded)};
				\fill[red!70] (-4.0,0.0) circle (2pt); % includes endpoint x=-4
				% poles from z^2+i=0: z = ±(1 - i)/√2  ≈ (±0.7071, ∓0.7071)
				\draw[red!80,very thick] ( 0.7071,-0.7071) +(0.12,0.12) -- +(-0.12,-0.12);
				\draw[red!80,very thick] ( 0.7071,-0.7071) +(-0.12,0.12) -- +(0.12,-0.12);
				\node[red!80,anchor=west] at (0.78,-0.70) {$\ \ \tfrac{1-i}{\sqrt2}$};
				\draw[red!80,very thick] (-0.7071, 0.7071) +(0.12,0.12) -- +(-0.12,-0.12);
				\draw[red!80,very thick] (-0.7071, 0.7071) +(-0.12,0.12) -- +(0.12,-0.12);
				\node[red!80,anchor=east] at (-0.78,0.70) {$\tfrac{-1+i}{\sqrt2}\ $};
				\draw[red!80,very thick, dotted] ( 0.7071,0) -- ( 0.7071,-0.7071);
				\draw[red!80,very thick, dotted] ( 0,-0.7071) -- ( 0.7071,-0.7071);
				\draw[red!80,very thick, dotted] ( -0.7071,0.7071) -- ( -0.7071,0);
				\draw[red!80,very thick, dotted] ( 0,0.7071) -- ( -0.7071,0.7071);
				%					% legend
				%					\node[align=left,anchor=west] at (-6.0,-2.7)
				%					{Analytic domain: $\ \C\setminus\big((-\infty,-4]\cup\{\pm\tfrac{1-i}{\sqrt2}\}\big)$.};
			\end{tikzpicture}
		\end{center}
	\end{enumerate}
\end{proof}

\begin{tcolorbox}
Show that the function $\ln(x^2+y^2)$ is harmonic in every domain that does not contain the origin.
\end{tcolorbox}
\begin{proof}[\sol]
	%		We need to show that $u(x,y) = \ln(x^2 + y^2)$
	%		satisfies Laplace’s equation \[
	%		u_{xx} + u_{yy} = 0
	%		\] at every point $(x,y)\neq(0,0)$. 
	For $(x,y)\neq(0,0)$, we can differentiate: \begin{align*}
		u_x &= \frac{\partial}{\partial x}\ln(x^2 + y^2) = \frac{2x}{x^2 + y^2},\\
		u_y &= \frac{\partial}{\partial y}\ln(x^2 + y^2) = \frac{2y}{x^2 + y^2}.
	\end{align*} And then \begin{align*}
		u_{xx}
		&= \frac{2(x^2 + y^2) - 2x\cdot 2x}{(x^2 + y^2)^2}
		= \frac{2(x^2 + y^2) - 4x^2}{(x^2 + y^2)^2}
		= \frac{-2x^2 + 2y^2}{(x^2 + y^2)^2} \\
		u_{yy}
		&= \frac{2(x^2 + y^2) - 2y\cdot 2y}{(x^2 + y^2)^2}
		= \frac{2(x^2 + y^2) - 4y^2}{(x^2 + y^2)^2}
		= \frac{2x^2 - 2y^2}{(x^2 + y^2)^2}.
	\end{align*}	
	Now compute the Laplacian: \[
	u_{xx} + u_{yy}
	= \frac{-2x^2 + 2y^2}{(x^2 + y^2)^2}+
	\frac{2x^2 - 2y^2}{(x^2 + y^2)^2}
	= \frac{(-2x^2 + 2y^2) + (2x^2 - 2y^2)}{(x^2 + y^2)^2}
	= \frac{0}{(x^2 + y^2)^2}
	= 0
	\] for all $(x,y)\neq(0,0)$.
	%		So $\ln(x^2 + y^2)$ is harmonic at every point where it is twice continuously differentiable --- that is, on any domain that does not contain the origin (since at ((0,0)) the function is not even defined, and our derivatives blow up).	
	%		Therefore, **(\ln(x^2 + y^2)) is harmonic in every domain that does not contain the origin.**
	
	\medskip\noindent
	(\textbf{Proof via Wirtinger-operator})
	Let $z=x+iy$ and
	\[
	u(x,y)=\ln(x^2+y^2)=\ln(\abs{z}^2)=\ln(z\bar z).
	\]
	Recall the Wirtinger operators $
	\partial:=\tfrac12(\partial_x-i\,\partial_y)$ and
	$\bar\partial:=\tfrac12(\partial_x+i\,\partial_y),$
	so that the Laplacian satisfies
	\[
	\Delta=\partial_{xx}+\partial_{yy}=4\,\partial\bar\partial=4\,\bar\partial\partial.
	\]
	On $\mathbb{C}\setminus\{0\}$ the chain rule gives \begin{align*}
		\partial u
		&=\partial\big(\ln(z\bar z)\big)
		=\frac{1}{z\bar z}\,\partial(z\bar z)
		=\frac{1}{z\bar z}\,\bar z
		=\frac{1}{z},\\
		\bar\partial u
		&=\bar\partial\big(\ln(z\bar z)\big)
		=\frac{1}{z\bar z}\,\bar\partial(z\bar z)
		=\frac{1}{z\bar z}\,z
		=\frac{1}{\bar z}.
	\end{align*}
	Therefore, $\displaystyle
	\Delta u
	=4\,\partial\bar\partial u
	=4\,\partial\!\left(\frac{1}{\bar z}\right)
	=0$ on $\mathbb{C}\setminus\{0\}$. 
\end{proof}

\begin{tcolorbox}
Show that $\cosh^2 z-\sinh^2 z=1$ and $\sinh z+\cosh z=e^z$.
\end{tcolorbox}
\begin{proof}[\sol]
	Recall the exponential definitions (valid for all $z\in\mathbb{C}$):
	\[
	\cosh z=\frac{e^{z}+e^{-z}}{2},\qquad
	\sinh z=\frac{e^{z}-e^{-z}}{2}.
	\]
	
	\medskip
	\noindent\textbf{(1) $\cosh^2 z-\sinh^2 z=1$.}
	\[
	\cosh^2 z-\sinh^2 z
	=\left(\frac{e^{z}+e^{-z}}{2}\right)^{\!2}
	-\left(\frac{e^{z}-e^{-z}}{2}\right)^{\!2}
	=\frac{(e^{z}+e^{-z})^2-(e^{z}-e^{-z})^2}{4}.
	\]
	Expanding,
	\[
	(e^{z}+e^{-z})^2-(e^{z}-e^{-z})^2
	=e^{2z}+2+e^{-2z}-(e^{2z}-2+e^{-2z})=4,
	\]
	so $\cosh^2 z-\sinh^2 z=\frac{4}{4}=1$.
	
	\medskip
	\noindent\textbf{(2) $\sinh z+\cosh z=e^{z}$.}
	\[
	\sinh z+\cosh z
	=\frac{e^{z}-e^{-z}}{2}+\frac{e^{z}+e^{-z}}{2}
	=e^{z}.
	\]
\end{proof}
